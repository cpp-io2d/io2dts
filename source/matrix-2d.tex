%!TEX root = io2d.tex
\rSec0 [\iotwod.\matrixtwod] {Class \tcode{basic_matrix_2d}}

\rSec1 [\iotwod.\matrixtwod.intro] {\tcode{basic_matrix_2d} description}

\indexlibrary{\idxcode{basic_matrix_2d}}%
\pnum
The class template \tcode{basic_matrix_2d} represents a three row by three column matrix. Its purpose is to perform affine transformations.

\pnum
The matrix is composed of nine \tcode{float} values: \tcode{m00}, \tcode{m01}, \tcode{m02}, \tcode{m10}, \tcode{m11}, \tcode{m12}, \tcode{m20}, \tcode{m21}, and \tcode{m22}. The ordering of these \tcode{float} values in the \tcode{basic_matrix_2d} class is unspecified.

\pnum
The specification of the \tcode{basic_matrix_2d} class, as described in this subclause, uses the following ordering: \\
$
~[~[~m00~m01~m02~]~] \\
~[~[~m10~m11~m12~]~] \\
~[~[~m20~m21~m22~]~]$

\pnum
\begin{note}
The naming convention and the layout shown above are consistent with a row-major layout. Though the naming convention is fixed, the unspecified layout allows for a column-major layout (or any other layout, though row-major and column-major are the only layouts typically used).
\end{note}

\pnum
The performance of any mathematical operation upon a \tcode{basic_matrix_2d} shall be carried out as-if the omitted third column data members were present with the values prescribed in the previous paragraph.

\pnum
The data are stored in an object of type \tcode{typename GraphicsMath::matrix_2d_data_type}. It is accessible using the \tcode{data} member functions.

\rSec1 [\iotwod.\matrixtwod.synopsis] {\tcode{basic_matrix_2d} synopsis}

\begin{codeblock}
namespace std::experimental::io2d::v1 {
  template <class GraphicsMath>
  class basic_matrix_2d {
  public:
    using data_type = typename GraphicsMath::matrix_2d_data_type;
    
    // \ref{\iotwod.\matrixtwod.cons}, constructors:
    basic_matrix_2d() noexcept;
    basic_matrix_2d(float v00, float v01, float v10, float v11, float v20, float v21) noexcept;
    basic_matrix_2d(const typename GraphicsMath::matrix_2d_data_type& v) noexcept;

    // \ref{\iotwod.\matrixtwod.accessors}, accessors:
    const data_type& data() const noexcept;
    data_type& data() noexcept;

    // \ref{\iotwod.\matrixtwod.staticfactories}, static factory functions:
    static basic_matrix_2d init_translate(const basic_point_2d<GraphicsMath>& v) noexcept;
    static basic_matrix_2d init_scale(const basic_point_2d<GraphicsMath>& v) noexcept;
    static basic_matrix_2d init_rotate(float radians) noexcept;
    static basic_matrix_2d init_rotate(float radians,
      const basic_point_2d<GraphicsMath>& origin) noexcept;
    static basic_matrix_2d init_reflect(float radians) noexcept;
    static basic_matrix_2d init_shear_x(float factor) noexcept;
    static basic_matrix_2d init_shear_y(float factor) noexcept;

    // \ref{\iotwod.\matrixtwod.modifiers}, modifiers:
    void m00(float v) noexcept;
    void m01(float v) noexcept;
    void m10(float v) noexcept;
    void m11(float v) noexcept;
    void m20(float v) noexcept;
    void m21(float v) noexcept;
    basic_matrix_2d& translate(const basic_point_2d<GraphicsMath>& v) noexcept;
    basic_matrix_2d& scale(const basic_point_2d<GraphicsMath>& v) noexcept;
    basic_matrix_2d& rotate(float radians) noexcept;
    basic_matrix_2d& rotate(float radians, const basic_point_2d<GraphicsMath>& origin) noexcept;
    basic_matrix_2d& reflect(float radians) noexcept;
    basic_matrix_2d& shear_x(float factor) noexcept;
    basic_matrix_2d& shear_y(float factor) noexcept;

    // \ref{\iotwod.\matrixtwod.observers}, observers:
    float m00() const noexcept;
    float m01() const noexcept;
    float m10() const noexcept;
    float m11() const noexcept;
    float m20() const noexcept;
    float m21() const noexcept;
    bool is_finite() const noexcept;
    bool is_invertible() const noexcept;
    float determinant() const noexcept;
    basic_matrix_2d inverse() const noexcept;
    basic_point_2d<GraphicsMath> transform_pt(const basic_point_2d<GraphicsMath>& pt)
      const noexcept;

    // \ref{\iotwod.\matrixtwod.member.ops}, member operators:
    basic_matrix_2d& operator*=(const basic_matrix_2d& other) noexcept;
  };

  // \ref{\iotwod.\matrixtwod.ops}, member operators:
  template <class GraphicsMath>
  basic_matrix_2d<GraphicsMath> operator*(const basic_matrix_2d<GraphicsMath>& lhs,
    const basic_matrix_2d<GraphicsMath>& rhs) noexcept;
  template <class GraphicsMath>
  basic_point_2d<GraphicsMath> operator*(const basic_point_2d<GraphicsMath>& lhs,
    const basic_matrix_2d<GraphicsMath>& rhs) noexcept;
  template <class GraphicsMath>
  bool operator==(const basic_matrix_2d<GraphicsMath>& lhs,
    const basic_matrix_2d<GraphicsMath>& rhs) noexcept;
  template <class GraphicsMath>
  bool operator!=(const basic_matrix_2d<GraphicsMath>& lhs,
    const basic_matrix_2d<GraphicsMath>& rhs) noexcept;
}
\end{codeblock}

\rSec1 [\iotwod.\matrixtwod.cons] {\tcode{basic_matrix_2d} constructors}

\indexlibrary{\idxcode{basic_matrix_2d}!constructor}%
\begin{itemdecl}
basic_matrix_2d() noexcept;
\end{itemdecl}
\begin{itemdescr}
\pnum
\effects
Constructs an object of type \tcode{basic_matrix_2d}.

\pnum
\begin{note}
The resulting matrix is the identity matrix.
\end{note}

\pnum
\postconditions
\tcode{data() == GraphicsMath::create_matrix_2d()}.
\end{itemdescr}

\indexlibrary{\idxcode{basic_matrix_2d}!constructor}%
\begin{itemdecl}
basic_matrix_2d(float m00, float m01, float m10, float m11,
  float m20, float m21) noexcept;
\end{itemdecl}
\begin{itemdescr}
\pnum
\effects
Constructs an object of type \tcode{basic_matrix_2d}.

\pnum
\postconditions{} \tcode{data() == GraphicsMath::create_matrix_2d(m00, m01, m10, m11, m20, m21)}.
\end{itemdescr}

\indexlibrary{\idxcode{basic_matrix_2d}!constructor}%
\begin{itemdecl}
basic_matrix_2d(const data_type& v) noexcept;
\end{itemdecl}
\begin{itemdescr}
\pnum
\effects
Constructs an object of type \tcode{basic_matrix_2d}.
	
\pnum
\postconditions{} \tcode{data() == v}.
\end{itemdescr}

\rSec1 [\iotwod.\matrixtwod.accessors] {\tcode{basic_matrix_2d} accessors}

\indexlibrarymember{data}{basic_matrix_2d}%
\begin{itemdecl}
const data_type& data() const noexcept;
data_type& data() noexcept;
\end{itemdecl}
\begin{itemdescr}
\pnum
\returns
A reference to the \tcode{basic_matrix_2d} object's data object (See: \ref{\iotwod.\matrixtwod.intro}).
\end{itemdescr}

\rSec1 [\iotwod.\matrixtwod.staticfactories] {\tcode{basic_matrix_2d} static factory 
functions}

\indexlibrarymember{init_translate}{basic_matrix_2d}%
\begin{itemdecl}
static basic_matrix_2d init_translate(basic_point_2d<GraphicsMath> v) noexcept;
\end{itemdecl}
\begin{itemdescr}
\pnum
\returns
\tcode{basic_matrix_2d(GraphicsMath::create_translate(v.data()))}.
\end{itemdescr}

\indexlibrarymember{init_scale}{basic_matrix_2d}%
\begin{itemdecl}
static basic_matrix_2d init_scale(basic_point_2d<GraphicsMath> v) noexcept;
\end{itemdecl}
\begin{itemdescr}
\pnum
\returns
\tcode{basic_matrix_2d(GraphicsMath::create_scale(v.data())}.
\end{itemdescr}

\indexlibrarymember{init_rotate}{basic_matrix_2d}%
\begin{itemdecl}
static basic_matrix_2d init_rotate(float radians) noexcept;
\end{itemdecl}
\begin{itemdescr}
\pnum
\returns
\tcode{basic_matrix_2d(GraphicsMath::create_rotate(radians))}.
\end{itemdescr}

\indexlibrarymember{init_rotate}{basic_matrix_2d}%
\begin{itemdecl}
static basic_matrix_2d init_rotate(float radians, basic_point_2d<GraphicsMath> origin) noexcept;
\end{itemdecl}
\begin{itemdescr}
\pnum
\returns
\tcode{basic_matrix_2d(GraphicsMath::create_rotate(radians, origin.data()))}.
\end{itemdescr}

\indexlibrarymember{init_reflect}{basic_matrix_2d}%
\begin{itemdecl}
static basic_matrix_2d init_reflect(float radians) noexcept;
\end{itemdecl}
\begin{itemdescr}
\pnum
\returns
\tcode{basic_matrix_2d(GraphicsMath::create_reflect(radians))}
\end{itemdescr}

\indexlibrarymember{init_shear_y}{basic_matrix_2d}%
\begin{itemdecl}
static basic_matrix_2d init_shear_x(float factor) noexcept;
\end{itemdecl}
\begin{itemdescr}
\pnum
\returns
\tcode{basic_matrix_2d(GraphicsMath::create_shear_x(factor))}.
\end{itemdescr}

\indexlibrarymember{init_shear_y}{basic_matrix_2d}%
\begin{itemdecl}
static basic_matrix_2d init_shear_y(float factor) noexcept;
\end{itemdecl}
\begin{itemdescr}
\pnum
\returns
\tcode{basic_matrix_2d(GraphicsMath::create_shear_y(factor))}.
\end{itemdescr}

\rSec1 [\iotwod.\matrixtwod.modifiers] {\tcode{basic_matrix_2d} modifiers}

\indexlibrarymember{m00}{basic_matrix_2d}%
\begin{itemdecl}
void m00(float v) noexcept;
\end{itemdecl}
\begin{itemdescr}
\pnum
\effects
Equivalent to \tcode{GraphicsMath::m00(data(), v);}
\end{itemdescr}

\indexlibrarymember{m01}{basic_matrix_2d}%
\begin{itemdecl}
void m01(float v) noexcept;
\end{itemdecl}
\begin{itemdescr}
\pnum
\effects
Equivalent to \tcode{GraphicsMath::m01(data(), v);}
\end{itemdescr}

\indexlibrarymember{m10}{basic_matrix_2d}%
\begin{itemdecl}
void m10(float v) noexcept;
\end{itemdecl}
\begin{itemdescr}
\pnum
\effects
Equivalent to \tcode{GraphicsMath::m10(data(), v);}
\end{itemdescr}

\indexlibrarymember{m11}{basic_matrix_2d}%
\begin{itemdecl}
void m11(float v) noexcept;
\end{itemdecl}
\begin{itemdescr}
\pnum
\effects
Equivalent to \tcode{GraphicsMath::m11(data(), v);}
\end{itemdescr}

\indexlibrarymember{m20}{basic_matrix_2d}%
\begin{itemdecl}
void m20(float v) noexcept;
\end{itemdecl}
\begin{itemdescr}
\pnum
\effects
Equivalent to \tcode{GraphicsMath::m20(data(), v);}
\end{itemdescr}

\indexlibrarymember{m21}{basic_matrix_2d}%
\begin{itemdecl}
void m21(float v) noexcept;
\end{itemdecl}
\begin{itemdescr}
\pnum
\effects
Equivalent to \tcode{GraphicsMath::m21(data(), v);}
\end{itemdescr}

\indexlibrarymember{translate}{basic_matrix_2d}%
\begin{itemdecl}
basic_matrix_2d& translate(basic_point_2d<GraphicsMath> v) noexcept;
\end{itemdecl}
\begin{itemdescr}
\pnum
\effects
Equivalent to \tcode{data() = GraphicsMath::translate(data(), v.data());}

\pnum
\returns
\tcode{*this}.
\end{itemdescr}

\indexlibrarymember{scale}{basic_matrix_2d}%
\begin{itemdecl}
basic_matrix_2d& scale(basic_point_2d<GraphicsMath> v) noexcept;
\end{itemdecl}
\begin{itemdescr}
\pnum
\effects
Equivalent to \tcode{data() = GraphicsMath::scale(data(), v.data());}

\pnum
\returns
\tcode{*this}.
\end{itemdescr}

\indexlibrarymember{rotate}{basic_matrix_2d}%
\begin{itemdecl}
basic_matrix_2d& rotate(float radians) noexcept;
\end{itemdecl}
\begin{itemdescr}
\pnum
\effects
Equivalent to \tcode{data() = GraphicsMath::rotate(data(), radians);}

\pnum
\returns
\tcode{*this}.
\end{itemdescr}

\indexlibrarymember{rotate}{basic_matrix_2d}%
\begin{itemdecl}
basic_matrix_2d& rotate(float radians, basic_point_2d<GraphicsMath> origin) noexcept;
\end{itemdecl}
\begin{itemdescr}
\pnum
\effects
Equivalent to \tcode{data() = GraphicsMath::rotate(data(), radians, origin.data());}

\pnum
\returns
\tcode{*this}.
\end{itemdescr}

\indexlibrarymember{reflect}{basic_matrix_2d}%
\begin{itemdecl}
basic_matrix_2d& reflect(float radians) noexcept;
\end{itemdecl}
\begin{itemdescr}
\pnum
\effects
Equivalent to \tcode{data() = GraphicsMath::reflect(data(), radians);}

\pnum
\returns
\tcode{*this}.
\end{itemdescr}

\indexlibrarymember{shear_x}{basic_matrix_2d}%
\begin{itemdecl}
basic_matrix_2d& shear_x(float factor) noexcept;
\end{itemdecl}
\begin{itemdescr}
\pnum
\effects
Equivalent to \tcode{data() = GraphicsMath::shear_x(data(), factor);}

\pnum
\returns
\tcode{*this}.
\end{itemdescr}

\indexlibrarymember{shear_y}{basic_matrix_2d}%
\begin{itemdecl}
basic_matrix_2d& shear_y(float factor) noexcept;
\end{itemdecl}
\begin{itemdescr}
\pnum
\effects
Equivalent to \tcode{data() = GraphicsMath::shear_y(factor);}

\pnum
\returns
\tcode{*this}.
\end{itemdescr}

\rSec1 [\iotwod.\matrixtwod.observers] {\tcode{basic_matrix_2d} observers}

\indexlibrarymember{m00}{basic_matrix_2d}%
\begin{itemdecl}
float m00() const noexcept;
\end{itemdecl}
\begin{itemdescr}
\pnum
\returns
\tcode{GraphicsMath::m00(data())}.
\end{itemdescr}

\indexlibrarymember{m01}{basic_matrix_2d}%
\begin{itemdecl}
float m01() const noexcept;
\end{itemdecl}
\begin{itemdescr}
\pnum
\returns
\tcode{GraphicsMath::m01(data())}.
\end{itemdescr}

\indexlibrarymember{m10}{basic_matrix_2d}%
\begin{itemdecl}
float m10() const noexcept;
\end{itemdecl}
\begin{itemdescr}
\pnum
\returns
\tcode{GraphicsMath::m10(data())}.
\end{itemdescr}

\indexlibrarymember{m11}{basic_matrix_2d}%
\begin{itemdecl}
float m11() const noexcept;
\end{itemdecl}
\begin{itemdescr}
\pnum
\returns
\tcode{GraphicsMath::m11(data())}.
\end{itemdescr}

\indexlibrarymember{m20}{basic_matrix_2d}%
\begin{itemdecl}
float m20() const noexcept;
\end{itemdecl}
\begin{itemdescr}
\pnum
\returns
\tcode{GraphicsMath::m20(data())}.
\end{itemdescr}

\indexlibrarymember{m21}{basic_matrix_2d}%
\begin{itemdecl}
float m21() const noexcept;
\end{itemdecl}
\begin{itemdescr}
\pnum
\returns
\tcode{GraphicsMath::m21(data())}.
\end{itemdescr}

\indexlibrarymember{is_finite}{basic_matrix_2d}%
\begin{itemdecl}
bool is_finite() const noexcept;
\end{itemdecl}
\begin{itemdescr}
\pnum
\returns
\tcode{GraphicsMath::is_finite(data())}.
\end{itemdescr}

\indexlibrarymember{is_invertible}{basic_matrix_2d}%
\begin{itemdecl}
bool is_invertible() const noexcept;
\end{itemdecl}
\begin{itemdescr}
\pnum
\requires
\tcode{is_finite() == true}.

\pnum
\returns
\tcode{GraphicsMath::is_invertible(data())}.
\end{itemdescr}

\indexlibrarymember{inverse}{basic_matrix_2d}%
\begin{itemdecl}
basic_matrix_2d inverse() const noexcept;
\end{itemdecl}
\begin{itemdescr}
\pnum
\requires
\tcode{is_invertible() == true}.

\pnum
\returns
\tcode{basic_matrix_2d(GraphicsMath::inverse(data()))}.
\end{itemdescr}

\indexlibrarymember{determinant}{basic_matrix_2d}%
\begin{itemdecl}
float determinant() const noexcept;
\end{itemdecl}
\begin{itemdescr}
\pnum
\requires
\tcode{is_finite() == true}.

\pnum
\returns
\tcode{GraphicsMath::determinant(data())}.
\end{itemdescr}

\indexlibrarymember{transform_pt}{basic_matrix_2d}%
\begin{itemdecl}
basic_point_2d<GraphicsMath> transform_pt(basic_point_2d<GraphicsMath> pt) const noexcept;
\end{itemdecl}
\begin{itemdescr}
\pnum
\returns
\tcode{basic_point_2d<GraphicsMath>(GraphicsMath::transform_pt(data(), pt.data()))}.
\end{itemdescr}

\rSec1 [\iotwod.\matrixtwod.member.ops] {\tcode{basic_matrix_2d} member operators}

\indexlibrarymember{operator*=}{basic_matrix_2d}%
\begin{itemdecl}
basic_matrix_2d& operator*=(const basic_matrix_2d& rhs) noexcept;
\end{itemdecl}
\begin{itemdescr}
\pnum
\effects
Equivalent to \tcode{data() = GraphicsMath::multiply(data(), rhs.data());}

\pnum
\returns
\tcode{*this}.
\end{itemdescr}

\rSec1 [\iotwod.\matrixtwod.ops] {\tcode{basic_matrix_2d} non-member operators}

\indexlibrarymember{operator*}{basic_matrix_2d}%
\begin{itemdecl}
basic_matrix_2d operator*(const basic_matrix_2d& lhs, const basic_matrix_2d& rhs)
  noexcept;
\end{itemdecl}
\begin{itemdescr}
\pnum
\returns
\tcode{basic_matrix_2d(GraphicsMath::multiply(lhs.data(), rhs.data())}.
\end{itemdescr}

\indexlibrarymember{operator*}{basic_matrix_2d}%
\begin{itemdecl}
basic_point_2d<GraphicsMath> operator*(basic_point_2d<GraphicsMath> v, const basic_matrix_2d& m) noexcept;
\end{itemdecl}
\begin{itemdescr}
\pnum
\returns
Equivalent to \tcode{m.transform_pt(v)}.
\end{itemdescr}

\indexlibrarymember{operator==}{basic_matrix_2d}%
\begin{itemdecl}
bool operator==(const basic_matrix_2d& lhs, const basic_matrix_2d& rhs) noexcept;
\end{itemdecl}
\begin{itemdescr}
\pnum
\returns
\tcode{GraphicsMath::equal(lhs.data(), rhs.data())}.
\end{itemdescr}

\indexlibrarymember{operator!=}{basic_matrix_2d}%
\begin{itemdecl}
bool operator!=(const basic_matrix_2d& lhs, const basic_matrix_2d& rhs) noexcept;
\end{itemdecl}
\begin{itemdescr}
\pnum
\returns
Equivalent to \tcode{GraphicsMath::not_equal(lhs.data(), rhs.data())}.
\end{itemdescr}
