%!TEX root = io2d.tex
\rSec0 [\iotwod.compositingop] {Enum class \tcode{compositing_op}}

\rSec1 [\iotwod.compositingop.summary] {\tcode{compositing_op} 
Summary}

\pnum
The \tcode{compositing_op} enum class specifies composition algorithms. See Table~\ref{tab:\iotwod.compositingop.meanings.basic}, 
Table~\ref{tab:\iotwod.compositingop.meanings.blend} and 
Table~\ref{tab:\iotwod.compositingop.meanings.hsl} for the meaning of 
each \tcode{compositing_op} enumerator.

\rSec1 [\iotwod.compositingop.synopsis] {\tcode{compositing_op} 
Synopsis}

\begin{codeblock}
namespace std::experimental::io2d::v1 {
  enum class compositing_op {
    // basic
    over,
    clear,
    source,
    in,
    out,
    atop,
    dest_over,
    dest_in,
    dest_out,
    dest_atop,
    xor_op,
    add,
    saturate,
    // blend
    multiply,
    screen,
    overlay,
    darken,
    lighten,
    color_dodge,
    color_burn,
    hard_light,
    soft_light,
    difference,
    exclusion,
    // hsl
    hsl_hue,
    hsl_saturation,
    hsl_color,
    hsl_luminosity
  };
}
\end{codeblock}

\rSec1 [\iotwod.compositingop.enumerators] {\tcode{compositing_op} 
Enumerators}
\pnum
The tables below specifies the mathematical formula for each enumerator's composition algorithm. The formulas differentiate between three color channels (red, green, and blue) and an alpha channel (transparency). For all channels, valid channel values are in the range $[0.0, 1.0]$.

\pnum
Where a visual data format for a visual data element has no alpha channel, the visual data format shall be treated as though it had an alpha channel with a value of $1.0$ for purposes of evaluating the formulas.

\pnum
Where a visual data format for a visual data element has no color channels, the visual data format shall be treated as though it had a value of $0.0$ for all color channels for purposes of evaluating the formulas.

\pnum
The following symbols and specifiers are used:\\
\hspace*{1em}The $R$ symbol means the result color value\\
\hspace*{1em}The $S$ symbol means the source color value\\
\hspace*{1em}The $D$ symbol means the destination color value\\
\hspace*{1em}The $c$ specifier means the color channels of the value it 
follows\\
\hspace*{1em}The $a$ specifier means the alpha channel of the value it follows

\pnum
The color symbols $R$, $S$, and $D$ may appear with or without any specifiers.

\pnum 
If a color symbol appears alone, it designates the entire color as a tuple in 
the unsigned normalized form (red, green, blue, alpha).

\pnum
The specifiers $c$ and $a$ may appear alone or together after any of the three 
color symbols.

\pnum
The presence of the $c$ specifier alone means the three color channels of the 
color as a tuple in the unsigned normalized form (red, green, blue).

\pnum
The presence of the $a$ specifier alone means the alpha channel of the color in 
unsigned normalized form.

\pnum
The presence of the specifiers together in the form $ca$ means the 
value of the color as a tuple in the unsigned normalized form (red, green, 
blue, alpha), where the value of each color channel is the product of each 
color channel and the alpha channel and the value of the alpha channel is the 
original value of the alpha channel.
\begin{example}
When it appears in a formula, $Sca$ means (($Sc \times Sa$), $Sa$), such that, 
given a source color $Sc = (1.0, 0.5, 0.0)$ and an source alpha $Sa = (0.5)$, 
the value of $Sca$ when specified in one of the formulas would be $Sca = (1.0 
\times 0.5, 0.5 \times 0.5, 0.0 \times 0.5, 0.5) = (0.5, 0.25, 0.0, 0.5)$. The 
same is true for $Dca$ and $Rca$.
\end{example}

\pnum
No space is left between a value and its channel specifiers. Channel 
specifiers will be preceded by exactly one value symbol.

\pnum
When performing an operation that involves evaluating the color 
channels, each color channel should be evaluated individually to produce its 
own value.

\pnum
The basic enumerators specify a value for \term{bound}. This value may be 'Yes', 
'No', or 'N/A'.

\pnum
If the bound value is 'Yes', then the source is treated as though it is 
also a mask. As such, only areas of the surface where the source would affect 
the surface are altered. The remaining areas of the surface have the same color 
value as before the compositing operation.

\pnum
If the bound value is 'No', then every area of the surface that is not 
affected by the source will become transparent black. In effect, it is as 
though the source was treated as being the same size as the destination surface 
with every part of the source that does not already have a color value assigned 
to it being treated as though it were transparent black. Application of the 
formula with this precondition results in those areas evaluating to transparent 
black such that evaluation can be bypassed due to the predetermined outcome.

\pnum
If the bound value is 'N/A', the operation would have the same effect 
regardless of whether it was treated as 'Yes' or 'No' such that those 
bound values are not applicable to the operation. A 'N/A' formula when 
applied to an area where the source does not provide a value will evaluate to 
the original value of the destination even if the source is treated as having a 
value there of transparent black. As such the result is the same as-if the 
source were treated as being a mask, i.e. 'Yes' and 'No' treatment each 
produce the same result in areas where the source does not have a value.

\pnum
If a clip is set and the bound value is 'Yes' or 'N/A', then only those
areas of the surface that the are within the clip will be affected by the
compositing operation.

\pnum
If a clip is set and the bound value is 'No', then only those areas of
the surface that the are within the clip will be affected by the compositing
operation. Even if no part of the source is within the clip, the operation will
still set every area within the clip to transparent black. Areas outside the
clip are not modified.

\begin{libiotwodreqtab4b}
 {\tcode{compositing_op} basic enumerator meanings}
 {tab:\iotwod.compositingop.meanings.basic}
 \\ \topline
 \lhdr{Enumerator}
 & \chdr{Bound}
 & \chdr{Color}
 & \rhdr{Alpha}
 \\ \capsep
 \endfirsthead
 \continuedcaption\\
 \hline
 \lhdr{Enumerator}
 & \chdr{Bound}
 & \chdr{Color}
 & \rhdr{Alpha}
 \\ \capsep
 \endhead
 \tcode{clear}
 & Yes
 & $Rc = 0$
 & $Ra = 0$
 \\
 \tcode{source}
 & Yes
 & $Rc = Sc$
 & $Ra = Sa$
 \\
 %
 &%
 &%
 &%
 \\
 \tcode{over}
 & N/A
 & $Rc = \dfrac{(Sca + Dca \times (1 - Sa))}{Ra}$
 & $Ra = Sa + Da \times (1 - Sa)$
 \\
 %
 &%
 &%
 &%
 \\
 \tcode{in}
 & No
 & $Rc = Sc$
 & $Ra = Sa \times Da$
 \\
 \tcode{out}
 & No
 & $Rc = Sc$
 & $Ra = Sa \times (1 - Da)$
 \\
 \tcode{atop}
 & N/A
 & $Rc = Sca + Dc \times (1 - Sa)$
 & $Ra = Da$
 \\
 %
 &%
 &%
 &%
 \\
 \tcode{dest_over}
 & N/A
 & $Rc = \dfrac{(Sca \times (1 - Da) + Dca)}{Ra}$
 & $Ra = (1 - Da) \times Sa + Da$
 \\
 %
 &%
 &%
 &%
 \\
 \tcode{dest_in}
 & No
 & $Rc = Dc$
 & $Ra = Sa \times Da$
 \\
 \tcode{dest_out}
 & N/A
 & $Rc = Dc$
 & $Ra = (1 - Sa) \times Da$
 \\
 \tcode{dest_atop}
 & No
 & $Rc = Sc \times (1 - Da) + Dca$
 & $Ra = Sa$
 \\
 %
 &%
 &%
 &%
 \\
 \tcode{xor_op}
 & N/A
 & $Rc = \dfrac{(Sca \times (1 - Da) + Dca \times (1 - Sa))}{Ra}$
 & $Ra = Sa + Da - 2 \times Sa \times Da$
 \\
 %
 &%
 &%
 &%
 \\
 \tcode{add}
 & N/A
 & $Rc = \dfrac{(Sca + Dca)}{Ra}$
 & $Ra = min(1, Sa + Da)$
 \\
 %
 &%
 &%
 &%
 \\
 \tcode{saturate}
 & N/A
 & $Rc = \dfrac{(min(Sa, 1 - Da) \times Sc + Dca)}{Ra}$
 & $Ra = min(1, Sa + Da)$
 \\
 %
 &%
 &%
 &%
 \\
\end{libiotwodreqtab4b}

\pnum
The blend enumerators and hsl enumerators share a common formula for the result 
color's color channel, with only one part of it changing depending on the 
enumerator. The result color's color channel value formula is as follows: $Rc = 
\dfrac{1}{Ra} \times ((1 - Da) \times Sca + (1 - Sa) \times Dca + Sa \times Da 
\times f(Sc, Dc))$. The function $f(Sc, Dc)$ is the component of the formula 
that is enumerator dependent.

\pnum
For the blend enumerators, the color channels shall be treated as separable, 
meaning that the color formula shall be evaluated separately for each color 
channel: red, green, and blue.

\pnum
The color formula divides 1 by the result color's alpha channel value. As a 
result, if the result color's alpha channel is zero then a division by zero 
would normally occur. Implementations shall not throw an exception nor  
otherwise produce any observable error condition if the result color's alpha 
channel is zero. Instead, implementations shall bypass the division by zero and 
produce the result color (0, 0, 0, 0), i.e. \defnx{transparent 
black}{color!transparent black}, if the result color alpha channel formula 
evaluates to zero.
\begin{note}
The simplest way to comply with this requirement is to bypass evaluation of the 
color channel formula in the event that the result alpha is zero. However, in 
order to allow implementations the greatest latitude possible, only the result 
is specified.
\end{note}

\pnum
For the enumerators in 
Table~\ref{tab:\iotwod.compositingop.meanings.blend} and 
Table~\ref{tab:\iotwod.compositingop.meanings.hsl} the result color's 
alpha channel value formula is as follows: $Ra = Sa + Da \times (1 - Sa)$.
\begin{note}
Since it is the same formula for all enumerators in those tables, the formula 
is not included in those tables.
\end{note}

\pnum
All of the blend enumerators and hsl enumerators have a bound value of 'N/A'.

\begin{libreqtab2}
 {\tcode{compositing_op} blend enumerator meanings}
 {tab:\iotwod.compositingop.meanings.blend}
 \\ \topline
 \lhdr{Enumerator}
 & \rhdr{Color}
 \\ \capsep
 \endfirsthead
 \continuedcaption\\
 \hline
 \lhdr{Enumerator}
 & \rhdr{Color}
 \\ \capsep
 \endhead
 \tcode{multiply}
 & $f(Sc, Dc) = Sc \times Dc$
 \\
 \tcode{screen}
 & $f(Sc, Dc) = Sc + Dc - Sc \times Dc$
 \\
 \tcode{overlay}
 & $if (Dc \le 0.5f)~\{\br
 \hspace*{1em}f(Sc, Dc) = 2 \times Sc \times Dc\br
 \}\br
 else~\{\br
 \hspace*{1em}f(Sc,Dc) =\br
 \hspace*{2em}1 - 2 \times (1 - Sc) \times\br
 \hspace*{2em}(1 - Dc)\br
 \}$\br
 \begin{note}
 The difference between this enumerator and \tcode{hard_light} is that this 
 tests the destination color ($Dc$) whereas \tcode{hard_light} tests the source 
 color ($Sc$).
 \end{note}
 \\
 \tcode{darken}
 & $f(Sc, Dc) = min(Sc, Dc)$
 \\
 \tcode{lighten}
 & $f(Sc, Dc) = max(Sc, Dc)$
 \\
 \tcode{color_dodge}
 & $if (Dc < 1)~\{\br
 \hspace*{1em}f(Sc, Dc) = min(1, \dfrac{Dc}{(1 - Sc)})\br
 \}\br
 else~\{\br
 \hspace*{1em}f(Sc,Dc) = 1
 \}$
 \\
 \tcode{color_burn}
 & $if~(Dc > 0)~\{\br
 \hspace*{1em}f(Sc, Dc) = 1 - min(1, \dfrac{1 - Dc}{Sc})\br
 \}\br
 else~\{\br
 \hspace*{1em}f(Sc,Dc) = 0\br
 \}$
 \\
 \tcode{hard_light}
 & $if~(Sc \le 0.5f)~\{\br
 \hspace*{1em}f(Sc, Dc) = 2 \times Sc \times Dc\br
 \}\br
 else~\{\br
 \hspace*{1em}f(Sc,Dc) =\br
 \hspace*{2em}1 - 2 \times (1 - Sc) \times\br
 \hspace*{2em}(1 - Dc)\br
 \}$\br
 \begin{note}
 The difference between this enumerator and \tcode{overlay} is that this 
 tests the source color ($Sc$) whereas \tcode{overlay} tests the destination 
 color ($Dc$).
 \end{note}
 \\
 \tcode{soft_light}
 & $if~(Sc \le 0.5)~\{\br
 \hspace*{1em}f(Sc, Dc) =\br
 \hspace*{2em}Dc - (1 - 2 \times Sc) \times Dc \times\br
 \hspace*{2em}(1 - Dc)\br
 \}\br
 else~\{\br
 \hspace*{1em}f(Sc,Dc) =\br
 \hspace*{2em}Dc + (2 \times Sc - 1) \times\br
 \hspace*{2em}(g(Dc) - Sc)\br
 \}$\br
 \br
 $g(Dc)$ is defined as follows:\br
 
 $if~(Dc \le 0.25)~\{\br
 \hspace*{1em}g(Dc) =\br
 \hspace*{2em}((16 \times Dc - 12) \times Dc +\br
 \hspace*{2em}4) \times Dc\br
 \}\br
 else~\{\br
 \hspace*{1em}g(Dc) = \sqrt{Dc}\br
 \}$
 \\
 \tcode{difference}
 & $f(Sc, Dc) = abs(Dc - Sc)$
 \\
 \tcode{exclusion}
 & $f(Sc, Dc) = Sc + Dc - 2 \times Sc \times Dc$
 \\
\end{libreqtab2}

\pnum
For the hsl enumerators, the color channels shall be treated as nonseparable, 
meaning that the color formula shall be evaluated once, with the colors being 
passed in as tuples in the form (red, green, blue).

\pnum
The following additional functions are used to define the hsl enumerator 
formulas:

\pnum
$min(x,~y,~z)~=~min(x,~min(y,~z))$

\pnum
$max(x,~y,~z)~=~max(x,~max(y,~z))$

\pnum
$sat(C) = max(Cr,~Cg,~Cb) - min(Cr,~Cg,~Cb)$

\pnum
$lum(C) = Cr \times 0.3 + Cg \times 0.59 + Cb \times 0.11$

\pnum
$clip\_color(C) =~\{\\
\hspace*{1em}L = lum(C)\\
\hspace*{1em}N = min(Cr, Cg, Cb)\\
\hspace*{1em}X = max(Cr, Cg, Cb)\\
\hspace*{1em}if~(N < 0.0)~\{\\
\hspace*{2em}Cr = L + \dfrac{((Cr - L) \times L)}{(L - N)}\\
\hspace*{2em}Cg = L + \dfrac{((Cg - L) \times L)}{(L - N)}\\
\hspace*{2em}Cb = L + \dfrac{((Cb - L) \times L)}{(L - N)}\\
\hspace*{1em}\}\\
\hspace*{1em}if~(X > 1.0)~\{\\
\hspace*{2em}Cr = L + \dfrac{((Cr - L) \times (1 - L))}{(X - L)}\\
\hspace*{2em}Cg = L + \dfrac{((Cg - L) \times (1 - L))}{(X - L)}\\
\hspace*{2em}Cb = L + \dfrac{((Cb - L) \times (1 - L))}{(X - L)}\\
\hspace*{1em}\}\\
\hspace*{1em}return~C\\
\} $

\pnum
$set\_lum(C, L) =~\{\\
\hspace*{1em}D = L - lum(C)\\
\hspace*{1em}Cr = Cr + D\\
\hspace*{1em}Cg = Cg + D\\
\hspace*{1em}Cb = Cb + D\\
\hspace*{1em}return~clip\_color(C)\\
\}$

\pnum
$set\_sat(C, S) =~\{\\
\hspace*{1em}R = C\\
\hspace*{1em}auto\&~max = (Rr > Rg)~?~((Rr > Rb)~?~Rr : Rb) : ((Rg > Rb)~?~Rg 
: Rb)\\
\hspace*{1em}auto\&~mid = (Rr > Rg)~?~((Rr > Rb)~?~((Rg > Rb)~?~Rg : Rb) : 
Rr) : ((Rg > Rb)~?~((Rr > Rb)~?~Rr : Rb) : Rg)\\
\hspace*{1em}auto\&~min = (Rr > Rg)~?~((Rg > Rb)~?~Rb : Rg) : ((Rr > Rb)~?~Rb : 
Rr)\\
\hspace*{1em}if~(max > min)~\{\\
\hspace*{2em}mid = \dfrac{((mid - min) \times S)}{max - min}\\
\hspace*{2em}max = S\\
\hspace*{1em}\}\\
\hspace*{1em}else~\{\\
\hspace*{2em}mid = 0.0\\
\hspace*{2em}max = 0.0\\
\hspace*{1em}\}\\
\hspace*{1em}min = 0.0\\
\hspace*{1em}return~R\\
\}$
\begin{note}
In the formula, $max$, $mid$, and $min$ are reference variables which are bound 
to the highest value, second highest value, and lowest value color channels of 
the (red, blue, green) tuple $R$ such that the subsequent operations 
modify the values of $R$ directly.
\end{note}

\begin{libreqtab2}
 {\tcode{compositing_op} hsl enumerator meanings}
 {tab:\iotwod.compositingop.meanings.hsl}
 \\ \topline
 \lhdr{Enumerator}
 & \rhdr{Color \& Alpha}
 \\ \capsep
 \endfirsthead
 \continuedcaption\\
 \hline
 \lhdr{Enumerator}
 & \chdr{Color \& Alpha}
 \\ \capsep
 \endhead
 
 \tcode{hsl_hue}
 & $f(Sc, Dc) = set\_lum(set\_sat(Sc,~sat(Dc)),~lum(Dc))$
 \\
 \tcode{hsl_saturation}
 & $(Sc, Dc) = set\_lum(set\_sat(Dc, sat(Sc)),~lum(Dc))$
 \\
 \tcode{hsl_color}
 & $f(Sc, Dc) = set\_lum(Sc,~lum(Dc))$
 \\
 \tcode{hsl_luminosity}
 & $f(Sc, Dc) = set\_lum(Dc,~lum(Sc))$
 \\
\end{libreqtab2}
