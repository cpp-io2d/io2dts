%!TEX root = io2d.tex

\rSec0 [surfaces] {Surfaces}

\pnum
Surfaces are composed of visual data, stored in a graphics data graphics resource.
\enternote
All well-defined \tcode{surface}-derived types are currently raster graphics data graphics resources with defined bounds. To allow for easier additions of future surface-derived types which are not composed of raster graphics data or do not have fixed bounds, such as a vector graphics-based surface, the less constrained term graphics data graphics resource is used.
\exitnote

\pnum
The surface's visual data is manipulated by rendering and composing operations (\ref{surface.rendering}).

\pnum
Surfaces are stateful objects.

\pnum
The various \tcode{surface}-derived classes each provide specific, unique functionality that enables a broad variety of 2D graphics operations to be accomplished efficiently.

\addtocounter{SectionDepthBase}{1}

%!TEX root = io2d.tex

\rSec0 [surfaceprops] {Class \tcode{surface_props}}

\rSec1 [surfaceprops.summary] {\tcode{surface_props} summary}

\pnum
The \tcode{surface_props} class provides general state information that is applicable to all rendering and composing operations (\ref{surface.rendering}).

\pnum
It has a Surface Matrix of type \tcode{matrix_2d}, and a Compositing Operator of type \tcode{compositing_op}.

\rSec1 [surfaceprops.synopsis] {\tcode{surface_props} synopsis}

\begin{codeblock}
namespace std { namespace experimental { namespace io2d { inline namespace v1 {
  class surface_props {
  public:
    // \ref{surfaceprops.cons}, constructors:
    constexpr surface_props() noexcept;
    constexpr explicit surface_props(const matrix_2d& m,
      compositing_op co = compositing_op::over) noexcept;
    
    // \ref{surfaceprops.modifiers}, modifiers:
    constexpr void compositing(compositing_op co) noexcept;
    constexpr void surface_matrix(const matrix_2d& m) noexcept;
    
    // \ref{surfaceprops.observers}, observers:
    constexpr compositing_op compositing() const noexcept;
    constexpr matrix_2d surface_matrix() const noexcept;
  };
}}}}
\end{codeblock}

\rSec1 [surfaceprops.cons] {\tcode{surface_props} constructors}

\indexlibrary{\idxcode{surface_props}!constructor}
\begin{itemdecl}
constexpr surface_props() noexcept;
\end{itemdecl}
\begin{itemdescr}
\pnum
\effects
The value of Surface Matrix is its default-constructed value.

\pnum
The value of Compositing Operator is \tcode{compositing_op::over}.
\end{itemdescr}

\indexlibrary{\idxcode{surface_props}!constructor}
\begin{itemdecl}
constexpr explicit surface_props(const matrix_2d& m,
  compositing_op co = compositing_op::over) noexcept;
\end{itemdecl}
\begin{itemdescr}
\pnum
\requires
\tcode{m.is_invertible() == true}.

\pnum
\effects
The value of Surface Matrix is \tcode{m}.

\pnum
The value of Compositing Operator is \tcode{co}.
\end{itemdescr}

\rSec1 [surfaceprops.modifiers] {\tcode{surface_props} modifiers}

\indexlibrary{\idxcode{surface_props}!\idxcode{compositing}}
\begin{itemdecl}
constexpr void compositing(compositing_op co) noexcept;
\end{itemdecl}
\begin{itemdescr}
\pnum
\effects
The value of Compositing Operator is \tcode{co}.
\end{itemdescr}

\indexlibrary{\idxcode{surface_props}!\idxcode{surface_matrix}}
\begin{itemdecl}
constexpr void surface_matrix(const matrix_2d& m) noexcept;
\end{itemdecl}
\begin{itemdescr}
\pnum
\requires
\tcode{m.is_invertible() == true}.

\pnum
\effects
The value of Surface Matrix is \tcode{m}.
\end{itemdescr}

\rSec1 [surfaceprops.observers] {\tcode{surface_props} observers}

\indexlibrary{\idxcode{surface_props}!\idxcode{compositing}}
\begin{itemdecl}
constexpr compositing_op compositing() const noexcept;
\end{itemdecl}
\begin{itemdescr}
\pnum
\returns
The value of Compositing Operator.
\end{itemdescr}

\indexlibrary{\idxcode{surface_props}!\idxcode{surface_matrix}}
\begin{itemdecl}
constexpr matrix_2d surface_matrix() const noexcept;
\end{itemdecl}
\begin{itemdescr}
\pnum
\returns
The value of Surface Matrix.
\end{itemdescr}


%!TEX root = io2d.tex

\rSec0 [\iotwod.strokeprops] {Class \tcode{stroke_props}}

\rSec1 [\iotwod.strokeprops.summary] {\tcode{stroke_props} summary}

\pnum
The \tcode{stroke_props} class provides state information that is applicable to the Stroke rendering and composing operation (\ref{\iotwod.surface.rendering} and \ref{\iotwod.surface.stroking}).

\pnum
It has a Line Width of type \tcode{double}, a Line Cap of type \tcode{line_cap}, a Line Join of type \tcode{line_join}, and a Miter Limit of type \tcode{double}.

%\pnum
%The Line Width determines the perpendicular width of a path segment at each point along the path segment. Half of the value of the Line Width extends to each side of a point along the path segment.
%\begin{example}
%Given a 20x20 \tcode{image_surface} *** FIXME ***
%\end{example}
%
\rSec1 [\iotwod.strokeprops.synopsis] {\tcode{stroke_props} synopsis}

\begin{codeblock}
namespace std { namespace experimental { namespace io2d { inline namespace v1 {
  class stroke_props {
  public:
    // \ref{\iotwod.strokeprops.cons}, constructors:
    constexpr stroke_props() noexcept;
    constexpr explicit stroke_props(double w,
      experimental::io2d::line_cap lc = experimental::io2d::line_cap::none,
      experimental::io2d::line_join lj = experimental::io2d::line_join::miter,
      double ml = 10.0) noexcept    

    // \ref{\iotwod.strokeprops.modifiers}, modifiers:
    constexpr void line_width(double w) noexcept;
    constexpr void line_cap(experimental::io2d::line_cap lc) noexcept;
    constexpr void line_join(experimental::io2d::line_join lj) noexcept;
    constexpr void miter_limit(double ml) noexcept;
    
    // \ref{\iotwod.strokeprops.observers}, observers:
    constexpr double line_width() const noexcept;
    constexpr experimental::io2d::line_cap line_cap() const noexcept;
    constexpr experimental::io2d::line_join line_join() const noexcept;
    constexpr double miter_limit() const noexcept;
    constexpr double max_miter_limit() const noexcept;
  };
}}}}
\end{codeblock}

\rSec1 [\iotwod.strokeprops.cons] {\tcode{stroke_props} constructors}

\indexlibrary{\idxcode{stroke_props}!constructor}
\begin{itemdecl}
constexpr stroke_props() noexcept;
\end{itemdecl}
\begin{itemdescr}
\pnum
\effects
The value of Line Width is \tcode{2.0}.

\pnum
The value of Line Cap is \tcode{experimental::io2d::line_cap::none}.

\pnum
The value of Line Join is \tcode{experimental::io2d::line_join::miter}.

\pnum
The value of Miter Limit is \tcode{10.0}.
\end{itemdescr}

\indexlibrary{\idxcode{stroke_props}!constructor}
\begin{itemdecl}
constexpr explicit stroke_props(double w,
  experimental::io2d::line_cap lc = experimental::io2d::line_cap::none,
  experimental::io2d::line_join lj = experimental::io2d::line_join::miter,
  double ml = 10.0) noexcept    
\end{itemdecl}
\begin{itemdescr}
\pnum
\requires
\tcode{w >= 0.0}.

\pnum
\tcode{ml >= 1.0 \&\& ml <= max_miter_limit()}

\pnum
\effects
The value of Line Width is \tcode{w}.

\pnum
The value of Line Cap is \tcode{lc}.

\pnum
The value of Line Join is \tcode{lj}.

\pnum
The value of Miter Limit is \tcode{ml}.
\end{itemdescr}

\rSec1 [\iotwod.strokeprops.modifiers] {\tcode{stroke_props} modifiers}

\indexlibrary{\idxcode{stroke_props}!\idxcode{line_width}}
\begin{itemdecl}
constexpr void line_width(double w) noexcept;
\end{itemdecl}
\begin{itemdescr}
\pnum
\requires
\tcode{w >= 0.0}.

\pnum
\effects
The value of Line Width is \tcode{w}.
\end{itemdescr}

\indexlibrary{\idxcode{stroke_props}!\idxcode{line_cap}}
\begin{itemdecl}
constexpr void line_cap(experimental::io2d::line_cap lc) noexcept;
\end{itemdecl}
\begin{itemdescr}
\pnum
\effects
The value of Line Cap is \tcode{lc}.
\end{itemdescr}

\indexlibrary{\idxcode{stroke_props}!\idxcode{line_join}}
\begin{itemdecl}
constexpr void line_join(experimental::io2d::line_join lj) noexcept;constexpr \end{itemdecl}
\begin{itemdescr}
\pnum
\effects
The value of Line Join is \tcode{lj}.
\end{itemdescr}

\indexlibrary{\idxcode{stroke_props}!\idxcode{miter_limit}}
\begin{itemdecl}
constexpr void miter_limit(double ml) noexcept;
\end{itemdecl}
\begin{itemdescr}
\pnum
\requires
\tcode{ml >= 1.0 \&\& ml <= max_miter_limit}.

\pnum
The value of Miter Limit if \tcode{ml}.
\end{itemdescr}

\rSec1 [\iotwod.strokeprops.observers] {\tcode{stroke_props} observers}

\indexlibrary{\idxcode{stroke_props}!\idxcode{line_width}}
\begin{itemdecl}
constexpr double line_width() const noexcept;
\end{itemdecl}
\begin{itemdescr}
\pnum
\returns
The value of Line Width.
\end{itemdescr}

\indexlibrary{\idxcode{stroke_props}!\idxcode{line_cap}}
\begin{itemdecl}
constexpr experimental::io2d::line_cap line_cap() const noexcept;
\end{itemdecl}
\begin{itemdescr}
\pnum
\returns
The value of Line Cap.
\end{itemdescr}

\indexlibrary{\idxcode{stroke_props}!\idxcode{line_join}}
\begin{itemdecl}
constexpr experimental::io2d::line_join line_join() const noexcept;
\end{itemdecl}
\begin{itemdescr}
\pnum
\returns
The value of Line Join.
\end{itemdescr}

\indexlibrary{\idxcode{stroke_props}!\idxcode{miter_limit}}
\begin{itemdecl}
constexpr double miter_limit() const noexcept;
\end{itemdecl}
\begin{itemdescr}
\pnum
\returns
The value of Miter Limit.
\end{itemdescr}

\indexlibrary{\idxcode{stroke_props}!\idxcode{max_miter_limit}}
\begin{itemdecl}
constexpr double max_miter_limit() const noexcept;
\end{itemdecl}
\begin{itemdescr}
\pnum
\returns
The \impldefplain{\idxcode{stroke_props}!\idxcode{max_miter_limit}} maximum allowable value of Miter Limit.

\pnum
\remarks
It is possible for this value to be \tcode{numeric_limits<double>::infinity()}.
\end{itemdescr}

%!TEX root = io2d.tex

\rSec0 [fillprops] {Class \tcode{fill_props}}

\rSec1 [fillprops.summary] {\tcode{fill_props} summary}

\pnum
FIXME

%!TEX root = io2d.tex

\rSec0 [maskprops] {Class \tcode{mask_props}}

\rSec1 [maskprops.summary] {\tcode{mask_props} summary}

\pnum
The \tcode{mask_props} class provides state information that is applicable to the Mask rendering and composing operation (\ref{surface.rendering}).

\pnum
It has a Wrap Mode of type \tcode{wrap_mode}, a Filter of type \tcode{filter}, and a Mask Matrix of type \tcode{matrix_2d}.

\rSec1 [maskprops.synopsis] {\tcode{mask_props} synopsis}

\begin{codeblock}
namespace std { namespace experimental { namespace io2d { inline namespace v1 {
  class mask_props {
  public:
    // \ref{maskprops.cons}, constructors:
    constexpr mask_props(
      experimental::io2d::wrap_mode w = experimental::io2d::wrap_mode::repeat,
      experimental::io2d::filter fi = experimental::io2d::filter::good,
      matrix_2d m = matrix_2d{}) noexcept;

    // \ref{maskprops.modifiers}, modifiers:
    constexpr void wrap_mode(experimental::io2d::wrap_mode w) noexcept;
    constexpr void filter(experimental::io2d::filter fi) noexcept;
    constexpr void mask_matrix(const matrix_2d& m) noexcept;

    // \ref{maskprops.observers}, observers:
    constexpr experimental::io2d::wrap_mode wrap_mode() const noexcept;
    constexpr experimental::io2d::filter filter() const noexcept;
    constexpr matrix_2d mask_matrix() const noexcept;
  };
}}}}
\end{codeblock}

\rSec1 [maskprops.cons] {\tcode{mask_props} constructors}

\indexlibrary{\idxcode{mask_props}!constructor}
\begin{itemdecl}
constexpr mask_props (
  experimental::io2d::wrap_mode w = experimental::io2d::wrap_mode::repeat,
  experimental::io2d::filter fi = experimental::io2d::filter::good,
  matrix_2d m = matrix_2d{}) noexcept    
\end{itemdecl}
\begin{itemdescr}
\onecolumn
\preconditions
\tcode{m.is_invertible() == true}.

\pnum
\effects
The value of Wrap Mode is \tcode{w}.

\pnum
The value of Filter is \tcode{fi}.

\pnum
The value of Mask Matrix is \tcode{m}.
\end{itemdescr}

\rSec1 [maskprops.modifiers] {\tcode{mask_props} modifiers}

\indexlibrary{\idxcode{mask_props}!\idxcode{wrap_mode}}
\begin{itemdecl}
constexpr void wrap_mode(experimental::io2d::wrap_mode w) noexcept;
\end{itemdecl}
\begin{itemdescr}
\pnum
\effects
The value of Wrap Mode is \tcode{w}.
\end{itemdescr}

\indexlibrary{\idxcode{mask_props}!\idxcode{filter}}
\begin{itemdecl}
constexpr void filter(experimental::io2d::filter fi) noexcept;
\end{itemdecl}
\begin{itemdescr}
\pnum
\effects
The value of Filter is \tcode{fi}.
\end{itemdescr}

\indexlibrary{\idxcode{mask_props}!\idxcode{mask_matrix}}
\begin{itemdecl}
constexpr void mask_matrix(const matrix_2d& m) noexcept;
\end{itemdecl}
\begin{itemdescr}
\pnum
\preconditions
\tcode{m.is_invertible() == true}.

\pnum
\effects
The value of Mask Matrix is \tcode{m}.
\end{itemdescr}

\rSec1 [maskprops.observers] {\tcode{mask_props} observers}

\indexlibrary{\idxcode{mask_props}!\idxcode{wrap_mode}}
\begin{itemdecl}
constexpr experimental::io2d::wrap_mode wrap_mode() const noexcept;
\end{itemdecl}
\begin{itemdescr}
\pnum
\returns
The value of Wrap Mode.
\end{itemdescr}

\indexlibrary{\idxcode{mask_props}!\idxcode{filter}}
\begin{itemdecl}
constexpr experimental::io2d::filter filter() const noexcept;
\end{itemdecl}
\begin{itemdescr}
\pnum
\returns
The value of Filter.
\end{itemdescr}

\indexlibrary{\idxcode{mask_props}!\idxcode{mask_matrix}}
\begin{itemdecl}
constexpr matrix_2d mask_matrix() const noexcept;
\end{itemdecl}
\begin{itemdescr}
\pnum
\returns
The value of Mask Matrix.
\end{itemdescr}

%!TEX root = io2d.tex

\rSec0 [\iotwod.clipprops] {Class \tcode{clip_props}}

\rSec1 [\iotwod.clipprops.summary] {\tcode{clip_props} summary}

\pnum
The \tcode{clip_props} class provides general state information that is applicable to all rendering and composing operations (\ref{\iotwod.surface.rendering}).

\pnum
It has a Clip Area of type \tcode{path_group} and a Fill Rule of type \tcode{fill_rule}.

\rSec1 [\iotwod.clipprops.synopsis] {\tcode{clip_props} synopsis}

\begin{codeblock}
namespace std { namespace experimental { namespace io2d { inline namespace v1 {
  class clip_props {
  public:
    // \ref{\iotwod.clipprops.cons}, constructors:
    clip_props() noexcept;
    template <class Allocator>
    explicit clip_props(const path_builder<Allocator>& pb,
      experimental::io2d::fill_rule = experimental::io2d::fill_rule::winding);
    explicit clip_props(const path_group& pg, experimental::io2d::fill_rule =
      experimental::io2d::fill_rule::winding) noexcept;

    // \ref{\iotwod.clipprops.modifiers}, modifiers:
    template <class Allocator>
    void clip(const path_builder<Allocator>& pb);
    void clip(const path_group& pg) noexcept;
    void fill_rule(experimental::io2d::fill_rule fr) noexcept;
    
    // \ref{\iotwod.clipprops.observers}, observers:
    path_group clip() const noexcept;
    experimental::io2d::fill_rule fill_rule() const noexcept;
  };
}}}}
\end{codeblock}

\rSec1 [\iotwod.clipprops.cons] {\tcode{clip_props} constructors}

\indexlibrary{\idxcode{clip_props}!constructor}
\begin{itemdecl}
clip_props() noexcept;
\end{itemdecl}
\begin{itemdescr}
\pnum
\effects
The value of Clip Area is its default-constructed value.

\pnum
The value of Fill Rule is \tcode{experimental::io2d::fill_rule::winding}.
\end{itemdescr}

\rSec1 [\iotwod.clipprops.modifiers] {\tcode{clip_props} modifiers}

\indexlibrary{\idxcode{clip_props}!\idxcode{clip}}
\begin{itemdecl}
template <class Allocator>
void clip(const path_builder<Allocator>& pb);
void clip(const path_group& pg) noexcept;
\end{itemdecl}
\begin{itemdescr}
\pnum
\effects
The value of Clip Area is:
\begin{itemize}
\item \tcode{path_group\{pb\}}; or
\item \tcode{pg}.
\end{itemize}
\end{itemdescr}

\indexlibrary{\idxcode{clip_props}!\idxcode{fill_rule}}
\begin{itemdecl}
void fill_rule(experimental::io2d::fill_rule fr) noexcept;
\end{itemdecl}
\begin{itemdescr}
\pnum
\effects
The value of Fill Rule is \tcode{fr}.
\end{itemdescr}

\rSec1 [\iotwod.clipprops.observers] {\tcode{clip_props} observers}

\indexlibrary{\idxcode{clip_props}!\idxcode{clip}}
\begin{itemdecl}
path_group clip() const noexcept;
\end{itemdecl}
\begin{itemdescr}
\pnum
\returns
The value of Clip Area.
\end{itemdescr}

\indexlibrary{\idxcode{clip_props}!\idxcode{fill_rule}}
\begin{itemdecl}
experimental::io2d::fill_rule fill_rule() const noexcept;
\end{itemdecl}
\begin{itemdescr}
\pnum
\returns
The value of Fill Rule.
\end{itemdescr}

%!TEX root = io2d.tex
\rSec0 [antialias] {Enum class \tcode{antialias}}

\rSec1 [antialias.summary] {\tcode{antialias} Summary}

\pnum
The antialias enum class specifies the type of anti-aliasing that the rendering
system uses for rendering and composing paths. See 
Table~\ref{tab:antialias.meanings} for the meaning of each
\tcode{antialias} enumerator.

\rSec1 [antialias.synopsis] {\tcode{antialias} Synopsis}

\indexlibrary{\idxcode{antialias}}
\begin{codeblock}
namespace std { namespace experimental { namespace io2d { inline namespace v1 {
  enum class antialias {
    none,
    fast,
    good,
    best
  };
} } } }
\end{codeblock}

\rSec1 [antialias.enumerators] {\tcode{antialias} Enumerators}

\begin{libreqtab2}
 {\tcode{antialias} enumerator meanings}
 {tab:antialias.meanings}
 \\ \topline
 \lhdr{Enumerator}
 & \rhdr{Meaning}
 \\ \capsep
 \endfirsthead
 \continuedcaption\\
 \hline
 \lhdr{Enumerator}
 & \rhdr{Meaning}
 \\ \capsep
 \endhead
 & No anti-aliasing is performed.
 \\
 \tcode{fast}
 & Some form of anti-aliasing shall be used when this option is selected, but the form used is \impldefplain{antialiasing!fast}.
 \begin{note}
 By specifying this value, the user is hinting that faster anti-aliasing is 
 preferable to better anti-aliasing.
 \end{note}
 \\
 \tcode{good}
 & Some form of anti-aliasing shall be used when this option is selected, but the form used is \impldefplain{antialiasing!good}.
 \begin{note}
 By specifying this value, the user is hinting that sacrificing some performance 
 to obtain better anti-aliasing is acceptable but that performance is still a 
 concern.
 \end{note}
 \\
 \tcode{best}
 & Some form of anti-aliasing shall be used when this option is selected, but the form used is \impldefplain{antialiasing!best}.
 \begin{note}
 By specifying this value, the user is hinting that anti-aliasing is more 
 important than performance.
 \end{note}
 \\
\end{libreqtab2}

%!TEX root = io2d.tex
\rSec0 [\iotwod.fillrule] {Enum class \tcode{fill_rule}}

\rSec1 [\iotwod.fillrule.summary] {\tcode{fill_rule} Summary}

\pnum
The \tcode{fill_rule} enum class determines how the Filling operation (\ref{\iotwod.surface.filling}) is performed on a path group.

\pnum
For each point, draw a ray from that point to infinity which does not pass through the start point or end point of any non-degenerate path segment in the path group, is not tangent to any non-degenerate path segment in the path group, and is not coincident with any non-degenerate path segment in the path group.

\pnum
See Table~\ref{tab:\iotwod.fillrule.meanings} for the meaning of each \tcode{fill_rule} enumerator.

\rSec1 [\iotwod.fillrule.synopsis] {\tcode{fill_rule} Synopsis}

\begin{codeblock}
namespace std { namespace experimental { namespace io2d { inline namespace v1 {
  enum class fill_rule {
    winding,
    even_odd
  };
} } } }
\end{codeblock}

\rSec1 [\iotwod.fillrule.enumerators] {\tcode{fill_rule} Enumerators}

\begin{libreqtab2}
 {\tcode{fill_rule} enumerator meanings}
 {tab:\iotwod.fillrule.meanings}
 \\ \topline
 \lhdr{Enumerator}
 & \rhdr{Meaning}
 \\ \capsep
 \endfirsthead
 \continuedcaption\\
 \hline
 \lhdr{Enumerator}
 & \rhdr{Meaning}
 \\ \capsep
 \endhead
 \tcode{winding}
 & If the Fill Rule (\ref{\iotwod.brushprops.summary}) is \tcode{fill_rule::winding}, then using the ray described above and beginning with a count of zero, add one to the count each time a non-degenerate path segment crosses the ray going left-to-right from its begin point to its end point, and subtract one each time a non-degenerate path segment crosses the ray going from right-to-left from its begin point to its end point. If the resulting count is zero after all non-degenerate path segments that cross the ray have been evaluated, the point shall not be filled; otherwise the point shall be filled.
 \\
 \tcode{even_odd}
 & If the Fill Rule is \tcode{fill_rule::even_odd}, then using the ray described above and beginning with a count of zero, add one to the count each time a non-degenerate path segment crosses the ray. If the resulting count is an odd number after all non-degenerate path segments that cross the ray have been evaluated, the point shall be filled; otherwise the point shall not be filled.
 \begin{note}
 Mathematically, zero is an even number, not an odd number.
 \end{note}
 \\ 
\end{libreqtab2}

%!TEX root = io2d.tex
\rSec0 [\iotwod.linecap] {Enum class \tcode{line_cap}}

\rSec1 [\iotwod.linecap.summary] {\tcode{line_cap} Summary}

\pnum
The \tcode{line_cap} enum class specifies how the ends of lines should be 
rendered when a \tcode{path} is stroked. See 
Table~\ref{tab:\iotwod.linecap.meanings} for the meaning of each 
\tcode{line_cap} enumerator.

\rSec1 [\iotwod.linecap.synopsis] {\tcode{line_cap} Synopsis}

\begin{codeblock}
namespace std { namespace experimental { namespace io2d { inline namespace v1 {
  enum class line_cap {
    butt,
    round,
    square
  };
} } } }
\end{codeblock}

\rSec1 [\iotwod.linecap.enumerators] {\tcode{line_cap} Enumerators}
\begin{libreqtab2}
 {\tcode{line_cap} enumerator meanings}
 {tab:\iotwod.linecap.meanings}
 \\ \topline
 \lhdr{Enumerator}
 & \rhdr{Meaning}
 \\ \capsep
 \endfirsthead
 \continuedcaption\\
 \hline
 \lhdr{Enumerator}
 & \rhdr{Meaning}
 \\ \capsep
 \endhead
 \tcode{butt}
 & The line has no cap. It terminates exactly at the end \tcode{point}.
 \\
 \tcode{round}
 & The line has a circular cap, with the end \tcode{point} serving as the 
 center of the circle and the line width serving as its diameter.
 \\
 \tcode{square}
 & The line has a square cap, with the end \tcode{point} serving as the center 
 of the square and the line width serving as the length of each side.
 \\
\end{libreqtab2}

%!TEX root = io2d.tex
\rSec0 [linejoin] {Enum class \tcode{line_join}}

\rSec1 [linejoin.summary] {\tcode{line_join} Summary}

\pnum
The \tcode{line_join} enum class specifies how the junction of two line 
segments should be rendered when a \tcode{path_group} is stroked.
See Table~\ref{tab:linejoin.meanings} for the meaning of each
\tcode{} enumerator.

\rSec1 [linejoin.synopsis] {\tcode{line_join} Synopsis}

\begin{codeblock}
namespace std { namespace experimental { namespace drawing { inline namespace 
v1 {
  enum class line_join {
    miter,
    round,
    bevel
  };
} } } }
\end{codeblock}

\rSec1 [linejoin.enumerators] {\tcode{line_join} Enumerators}
\begin{libreqtab2}
 {\tcode{line_join} enumerator meanings}
 {tab:linejoin.meanings}
 \\ \topline
 \lhdr{Enumerator}
 & \rhdr{Meaning}
 \\ \capsep
 \endfirsthead
 \continuedcaption\\
 \hline
 \lhdr{Enumerator}
 & \rhdr{Meaning}
 \\ \capsep
 \endhead
 \tcode{miter}
 & Joins will be mitered or beveled, depending on the Miter Limit (\ref{strokeprops.summary}).
 \\
 \tcode{round}
 & Joins will be rounded, with the center of the circle being the join point.
 \\
 \tcode{bevel}
 & Joins will be beveled, with the join cut off at half the line width from the 
 join point. Implementations may vary the cut off distance by an amount that is 
 less than one pixel at each join for aesthetic or technical reasons.
 \\
\end{libreqtab2}

%!TEX root = io2d.tex
\rSec0 [\iotwod.compositingop] {Enum class \tcode{compositing_op}}

\rSec1 [\iotwod.compositingop.summary] {\tcode{compositing_op} 
Summary}

\pnum
The \tcode{compositing_op} enum class specifies composition algorithms. See Table~\ref{tab:\iotwod.compositingop.meanings.basic}, 
Table~\ref{tab:\iotwod.compositingop.meanings.blend} and 
Table~\ref{tab:\iotwod.compositingop.meanings.hsl} for the meaning of 
each \tcode{compositing_op} enumerator.

\rSec1 [\iotwod.compositingop.synopsis] {\tcode{compositing_op} 
Synopsis}

\begin{codeblock}
namespace @\fullnamespace{}@ {
  enum class compositing_op {
    // basic
    over,
    clear,
    source,
    in,
    out,
    atop,
    dest_over,
    dest_in,
    dest_out,
    dest_atop,
    xor_op,
    add,
    saturate,
    // blend
    multiply,
    screen,
    overlay,
    darken,
    lighten,
    color_dodge,
    color_burn,
    hard_light,
    soft_light,
    difference,
    exclusion,
    // hsl
    hsl_hue,
    hsl_saturation,
    hsl_color,
    hsl_luminosity
  };
}
\end{codeblock}

\rSec1 [\iotwod.compositingop.enumerators] {\tcode{compositing_op} 
Enumerators}
\pnum
The tables below specifies the mathematical formula for each enumerator's composition algorithm. The formulas differentiate between three color channels (red, green, and blue) and an alpha channel (transparency). For all channels, valid channel values are in the range $[0.0, 1.0]$.

\pnum
Where a visual data format for a visual data element has no alpha channel, the visual data format shall be treated as though it had an alpha channel with a value of $1.0$ for purposes of evaluating the formulas.

\pnum
Where a visual data format for a visual data element has no color channels, the visual data format shall be treated as though it had a value of $0.0$ for all color channels for purposes of evaluating the formulas.

\pnum
The following symbols and specifiers are used:\\
\hspace*{1em}The $R$ symbol means the result color value\\
\hspace*{1em}The $S$ symbol means the source color value\\
\hspace*{1em}The $D$ symbol means the destination color value\\
\hspace*{1em}The $c$ specifier means the color channels of the value it 
follows\\
\hspace*{1em}The $a$ specifier means the alpha channel of the value it follows

\pnum
The color symbols $R$, $S$, and $D$ may appear with or without any specifiers.

\pnum 
If a color symbol appears alone, it designates the entire color as a tuple in 
the unsigned normalized form (red, green, blue, alpha).

\pnum
The specifiers $c$ and $a$ may appear alone or together after any of the three 
color symbols.

\pnum
The presence of the $c$ specifier alone means the three color channels of the 
color as a tuple in the unsigned normalized form (red, green, blue).

\pnum
The presence of the $a$ specifier alone means the alpha channel of the color in 
unsigned normalized form.

\pnum
The presence of the specifiers together in the form $ca$ means the 
value of the color as a tuple in the unsigned normalized form (red, green, 
blue, alpha), where the value of each color channel is the product of each 
color channel and the alpha channel and the value of the alpha channel is the 
original value of the alpha channel.
\begin{example}
When it appears in a formula, $Sca$ means (($Sc \times Sa$), $Sa$), such that, 
given a source color $Sc = (1.0, 0.5, 0.0)$ and an source alpha $Sa = (0.5)$, 
the value of $Sca$ when specified in one of the formulas would be $Sca = (1.0 
\times 0.5, 0.5 \times 0.5, 0.0 \times 0.5, 0.5) = (0.5, 0.25, 0.0, 0.5)$. The 
same is true for $Dca$ and $Rca$.
\end{example}

\pnum
No space is left between a value and its channel specifiers. Channel 
specifiers will be preceded by exactly one value symbol.

\pnum
When performing an operation that involves evaluating the color 
channels, each color channel should be evaluated individually to produce its 
own value.

\pnum
The basic enumerators specify a value for \term{bound}. This value may be 'Yes', 
'No', or 'N/A'.

\pnum
If the bound value is 'Yes', then the source is treated as though it is 
also a mask. As such, only areas of the surface where the source would affect 
the surface are altered. The remaining areas of the surface have the same color 
value as before the compositing operation.

\pnum
If the bound value is 'No', then every area of the surface that is not 
affected by the source will become transparent black. In effect, it is as 
though the source was treated as being the same size as the destination surface 
with every part of the source that does not already have a color value assigned 
to it being treated as though it were transparent black. Application of the 
formula with this precondition results in those areas evaluating to transparent 
black such that evaluation can be bypassed due to the predetermined outcome.

\pnum
If the bound value is 'N/A', the operation would have the same effect 
regardless of whether it was treated as 'Yes' or 'No' such that those 
bound values are not applicable to the operation. A 'N/A' formula when 
applied to an area where the source does not provide a value will evaluate to 
the original value of the destination even if the source is treated as having a 
value there of transparent black. As such the result is the same as-if the 
source were treated as being a mask, i.e. 'Yes' and 'No' treatment each 
produce the same result in areas where the source does not have a value.

\pnum
If a clip is set and the bound value is 'Yes' or 'N/A', then only those
areas of the surface that the are within the clip will be affected by the
compositing operation.

\pnum
If a clip is set and the bound value is 'No', then only those areas of
the surface that the are within the clip will be affected by the compositing
operation. Even if no part of the source is within the clip, the operation will
still set every area within the clip to transparent black. Areas outside the
clip are not modified.

\begin{libiotwodreqtab4b}
 {\tcode{compositing_op} basic enumerator meanings}
 {tab:\iotwod.compositingop.meanings.basic}
 \\ \topline
 \lhdr{Enumerator}
 & \chdr{Bound}
 & \chdr{Color}
 & \rhdr{Alpha}
 \\ \capsep
 \endfirsthead
 \continuedcaption\\
 \hline
 \lhdr{Enumerator}
 & \chdr{Bound}
 & \chdr{Color}
 & \rhdr{Alpha}
 \\ \capsep
 \endhead
 \tcode{clear}
 & Yes
 & $Rc = 0$
 & $Ra = 0$
 \\
 \tcode{source}
 & Yes
 & $Rc = Sc$
 & $Ra = Sa$
 \\
 %
 &%
 &%
 &%
 \\
 \tcode{over}
 & N/A
 & $Rc = \dfrac{(Sca + Dca \times (1 - Sa))}{Ra}$
 & $Ra = Sa + Da \times (1 - Sa)$
 \\
 %
 &%
 &%
 &%
 \\
 \tcode{in}
 & No
 & $Rc = Sc$
 & $Ra = Sa \times Da$
 \\
 \tcode{out}
 & No
 & $Rc = Sc$
 & $Ra = Sa \times (1 - Da)$
 \\
 \tcode{atop}
 & N/A
 & $Rc = Sca + Dc \times (1 - Sa)$
 & $Ra = Da$
 \\
 %
 &%
 &%
 &%
 \\
 \tcode{dest_over}
 & N/A
 & $Rc = \dfrac{(Sca \times (1 - Da) + Dca)}{Ra}$
 & $Ra = (1 - Da) \times Sa + Da$
 \\
 %
 &%
 &%
 &%
 \\
 \tcode{dest_in}
 & No
 & $Rc = Dc$
 & $Ra = Sa \times Da$
 \\
 \tcode{dest_out}
 & N/A
 & $Rc = Dc$
 & $Ra = (1 - Sa) \times Da$
 \\
 \tcode{dest_atop}
 & No
 & $Rc = Sc \times (1 - Da) + Dca$
 & $Ra = Sa$
 \\
 %
 &%
 &%
 &%
 \\
 \tcode{xor_op}
 & N/A
 & $Rc = \dfrac{(Sca \times (1 - Da) + Dca \times (1 - Sa))}{Ra}$
 & $Ra = Sa + Da - 2 \times Sa \times Da$
 \\
 %
 &%
 &%
 &%
 \\
 \tcode{add}
 & N/A
 & $Rc = \dfrac{(Sca + Dca)}{Ra}$
 & $Ra = min(1, Sa + Da)$
 \\
 %
 &%
 &%
 &%
 \\
 \tcode{saturate}
 & N/A
 & $Rc = \dfrac{(min(Sa, 1 - Da) \times Sc + Dca)}{Ra}$
 & $Ra = min(1, Sa + Da)$
 \\
 %
 &%
 &%
 &%
 \\
\end{libiotwodreqtab4b}

\pnum
The blend enumerators and hsl enumerators share a common formula for the result 
color's color channel, with only one part of it changing depending on the 
enumerator. The result color's color channel value formula is as follows: $Rc = 
\dfrac{1}{Ra} \times ((1 - Da) \times Sca + (1 - Sa) \times Dca + Sa \times Da 
\times f(Sc, Dc))$. The function $f(Sc, Dc)$ is the component of the formula 
that is enumerator dependent.

\pnum
For the blend enumerators, the color channels shall be treated as separable, 
meaning that the color formula shall be evaluated separately for each color 
channel: red, green, and blue.

\pnum
The color formula divides 1 by the result color's alpha channel value. As a 
result, if the result color's alpha channel is zero then a division by zero 
would normally occur. Implementations shall not throw an exception nor  
otherwise produce any observable error condition if the result color's alpha 
channel is zero. Instead, implementations shall bypass the division by zero and 
produce the result color (0, 0, 0, 0), i.e. \defnx{transparent 
black}{color!transparent black}, if the result color alpha channel formula 
evaluates to zero.
\begin{note}
The simplest way to comply with this requirement is to bypass evaluation of the 
color channel formula in the event that the result alpha is zero. However, in 
order to allow implementations the greatest latitude possible, only the result 
is specified.
\end{note}

\pnum
For the enumerators in 
Table~\ref{tab:\iotwod.compositingop.meanings.blend} and 
Table~\ref{tab:\iotwod.compositingop.meanings.hsl} the result color's 
alpha channel value formula is as follows: $Ra = Sa + Da \times (1 - Sa)$.
\begin{note}
Since it is the same formula for all enumerators in those tables, the formula 
is not included in those tables.
\end{note}

\pnum
All of the blend enumerators and hsl enumerators have a bound value of 'N/A'.

\begin{libreqtab2}
 {\tcode{compositing_op} blend enumerator meanings}
 {tab:\iotwod.compositingop.meanings.blend}
 \\ \topline
 \lhdr{Enumerator}
 & \rhdr{Color}
 \\ \capsep
 \endfirsthead
 \continuedcaption\\
 \hline
 \lhdr{Enumerator}
 & \rhdr{Color}
 \\ \capsep
 \endhead
 \tcode{multiply}
 & $f(Sc, Dc) = Sc \times Dc$
 \\
 \tcode{screen}
 & $f(Sc, Dc) = Sc + Dc - Sc \times Dc$
 \\
 \tcode{overlay}
 & $if (Dc \le 0.5f)~\{\br
 \hspace*{1em}f(Sc, Dc) = 2 \times Sc \times Dc\br
 \}\br
 else~\{\br
 \hspace*{1em}f(Sc,Dc) =\br
 \hspace*{2em}1 - 2 \times (1 - Sc) \times\br
 \hspace*{2em}(1 - Dc)\br
 \}$\br
 \begin{note}
 The difference between this enumerator and \tcode{hard_light} is that this 
 tests the destination color ($Dc$) whereas \tcode{hard_light} tests the source 
 color ($Sc$).
 \end{note}
 \\
 \tcode{darken}
 & $f(Sc, Dc) = min(Sc, Dc)$
 \\
 \tcode{lighten}
 & $f(Sc, Dc) = max(Sc, Dc)$
 \\
 \tcode{color_dodge}
 & $if (Dc < 1)~\{\br
 \hspace*{1em}f(Sc, Dc) = min(1, \dfrac{Dc}{(1 - Sc)})\br
 \}\br
 else~\{\br
 \hspace*{1em}f(Sc,Dc) = 1
 \}$
 \\
 \tcode{color_burn}
 & $if~(Dc > 0)~\{\br
 \hspace*{1em}f(Sc, Dc) = 1 - min(1, \dfrac{1 - Dc}{Sc})\br
 \}\br
 else~\{\br
 \hspace*{1em}f(Sc,Dc) = 0\br
 \}$
 \\
 \tcode{hard_light}
 & $if~(Sc \le 0.5f)~\{\br
 \hspace*{1em}f(Sc, Dc) = 2 \times Sc \times Dc\br
 \}\br
 else~\{\br
 \hspace*{1em}f(Sc,Dc) =\br
 \hspace*{2em}1 - 2 \times (1 - Sc) \times\br
 \hspace*{2em}(1 - Dc)\br
 \}$\br
 \begin{note}
 The difference between this enumerator and \tcode{overlay} is that this 
 tests the source color ($Sc$) whereas \tcode{overlay} tests the destination 
 color ($Dc$).
 \end{note}
 \\
 \tcode{soft_light}
 & $if~(Sc \le 0.5)~\{\br
 \hspace*{1em}f(Sc, Dc) =\br
 \hspace*{2em}Dc - (1 - 2 \times Sc) \times Dc \times\br
 \hspace*{2em}(1 - Dc)\br
 \}\br
 else~\{\br
 \hspace*{1em}f(Sc,Dc) =\br
 \hspace*{2em}Dc + (2 \times Sc - 1) \times\br
 \hspace*{2em}(g(Dc) - Sc)\br
 \}$\br
 \br
 $g(Dc)$ is defined as follows:\br
 
 $if~(Dc \le 0.25)~\{\br
 \hspace*{1em}g(Dc) =\br
 \hspace*{2em}((16 \times Dc - 12) \times Dc +\br
 \hspace*{2em}4) \times Dc\br
 \}\br
 else~\{\br
 \hspace*{1em}g(Dc) = \sqrt{Dc}\br
 \}$
 \\
 \tcode{difference}
 & $f(Sc, Dc) = abs(Dc - Sc)$
 \\
 \tcode{exclusion}
 & $f(Sc, Dc) = Sc + Dc - 2 \times Sc \times Dc$
 \\
\end{libreqtab2}

\pnum
For the hsl enumerators, the color channels shall be treated as nonseparable, 
meaning that the color formula shall be evaluated once, with the colors being 
passed in as tuples in the form (red, green, blue).

\pnum
The following additional functions are used to define the hsl enumerator 
formulas:

\pnum
$min(x,~y,~z)~=~min(x,~min(y,~z))$

\pnum
$max(x,~y,~z)~=~max(x,~max(y,~z))$

\pnum
$sat(C) = max(Cr,~Cg,~Cb) - min(Cr,~Cg,~Cb)$

\pnum
$lum(C) = Cr \times 0.3 + Cg \times 0.59 + Cb \times 0.11$

\pnum
$clip\_color(C) =~\{\\
\hspace*{1em}L = lum(C)\\
\hspace*{1em}N = min(Cr, Cg, Cb)\\
\hspace*{1em}X = max(Cr, Cg, Cb)\\
\hspace*{1em}if~(N < 0.0)~\{\\
\hspace*{2em}Cr = L + \dfrac{((Cr - L) \times L)}{(L - N)}\\
\hspace*{2em}Cg = L + \dfrac{((Cg - L) \times L)}{(L - N)}\\
\hspace*{2em}Cb = L + \dfrac{((Cb - L) \times L)}{(L - N)}\\
\hspace*{1em}\}\\
\hspace*{1em}if~(X > 1.0)~\{\\
\hspace*{2em}Cr = L + \dfrac{((Cr - L) \times (1 - L))}{(X - L)}\\
\hspace*{2em}Cg = L + \dfrac{((Cg - L) \times (1 - L))}{(X - L)}\\
\hspace*{2em}Cb = L + \dfrac{((Cb - L) \times (1 - L))}{(X - L)}\\
\hspace*{1em}\}\\
\hspace*{1em}return~C\\
\} $

\pnum
$set\_lum(C, L) =~\{\\
\hspace*{1em}D = L - lum(C)\\
\hspace*{1em}Cr = Cr + D\\
\hspace*{1em}Cg = Cg + D\\
\hspace*{1em}Cb = Cb + D\\
\hspace*{1em}return~clip\_color(C)\\
\}$

\pnum
$set\_sat(C, S) =~\{\\
\hspace*{1em}R = C\\
\hspace*{1em}auto\&~max = (Rr > Rg)~?~((Rr > Rb)~?~Rr : Rb) : ((Rg > Rb)~?~Rg 
: Rb)\\
\hspace*{1em}auto\&~mid = (Rr > Rg)~?~((Rr > Rb)~?~((Rg > Rb)~?~Rg : Rb) : 
Rr) : ((Rg > Rb)~?~((Rr > Rb)~?~Rr : Rb) : Rg)\\
\hspace*{1em}auto\&~min = (Rr > Rg)~?~((Rg > Rb)~?~Rb : Rg) : ((Rr > Rb)~?~Rb : 
Rr)\\
\hspace*{1em}if~(max > min)~\{\\
\hspace*{2em}mid = \dfrac{((mid - min) \times S)}{max - min}\\
\hspace*{2em}max = S\\
\hspace*{1em}\}\\
\hspace*{1em}else~\{\\
\hspace*{2em}mid = 0.0\\
\hspace*{2em}max = 0.0\\
\hspace*{1em}\}\\
\hspace*{1em}min = 0.0\\
\hspace*{1em}return~R\\
\}$
\begin{note}
In the formula, $max$, $mid$, and $min$ are reference variables which are bound 
to the highest value, second highest value, and lowest value color channels of 
the (red, blue, green) tuple $R$ such that the subsequent operations 
modify the values of $R$ directly.
\end{note}

\begin{libreqtab2}
 {\tcode{compositing_op} hsl enumerator meanings}
 {tab:\iotwod.compositingop.meanings.hsl}
 \\ \topline
 \lhdr{Enumerator}
 & \rhdr{Color \& Alpha}
 \\ \capsep
 \endfirsthead
 \continuedcaption\\
 \hline
 \lhdr{Enumerator}
 & \chdr{Color \& Alpha}
 \\ \capsep
 \endhead
 
 \tcode{hsl_hue}
 & $f(Sc, Dc) = set\_lum(set\_sat(Sc,~sat(Dc)),~lum(Dc))$
 \\
 \tcode{hsl_saturation}
 & $(Sc, Dc) = set\_lum(set\_sat(Dc, sat(Sc)),~lum(Dc))$
 \\
 \tcode{hsl_color}
 & $f(Sc, Dc) = set\_lum(Sc,~lum(Dc))$
 \\
 \tcode{hsl_luminosity}
 & $f(Sc, Dc) = set\_lum(Dc,~lum(Sc))$
 \\
\end{libreqtab2}

%!TEX root = io2d.tex
\rSec0 [\iotwod.format] {Enum class \tcode{format}}

\rSec1 [\iotwod.format.summary] {\tcode{format} summary}

\pnum
The \tcode{format} enum class indicates a visual data format. See Table~\ref{tab:\iotwod.format.meanings} for 
the meaning of each \tcode{format} enumerator.

%\pnum
%A pixel is a value composed from one or more channels.
%
%\pnum
%A \term{channel} is a visual data element that represents color data (red, green,
%or blue) or transparency data (alpha). A pixel can be comprised of color data,
%transparency data, or both color and transparency data.
%
\pnum
Unless otherwise specified, a visual data format shall be an unsigned integral
value of the specified bit size in native-endian format.

%\pnum
%Unless otherwise specified, each channel of a pixel shall be treated as an 
%unsigned integral value of the specified bit size at the specified bit location
%within the pixel.
%
\pnum
A channel value of 0x0 means that there is no contribution from that channel. 
As the channel value increases towards the maximum unsigned integral value 
representable by the number of bits of the channel, the contribution from that 
channel also increases, with the maximum value representing the maximum
contribution from that channel.
\begin{example}
Given a 5-bit channel representing the color , a value of 0x0 means that the red channel does not 
contribute any value towards the final color of the pixel. A value of 0x1F 
means that the red channel makes its maximum contribution to the final color of 
the pixel.

A
\end{example}

\rSec1 [\iotwod.format.synopsis] {\tcode{format} synopsis}

\begin{codeblock}
namespace std::experimental::io2d::v1 {
  enum class format {
    invalid,
    argb32,
    rgb24,
    a8,
    rgb16_565,
    rgb30
  };
}
\end{codeblock}

\rSec1 [\iotwod.format.enumerators] {\tcode{format} enumerators}
\begin{libreqtab2}
 {\tcode{format} enumerator meanings}
 {tab:\iotwod.format.meanings}
 \\ \topline
 \lhdr{Enumerator}
 & \rhdr{Meaning}
 \\ \capsep
 \endfirsthead
 \continuedcaption\\
 \hline
 \lhdr{Enumerator}
 & \rhdr{Meaning}
 \\ \capsep
 \endhead
 \tcode{invalid}
 & A previously specified \tcode{format} is unsupported by the implementation.
 \\
 \tcode{argb32}
 & A 32-bit RGB color model pixel format. The upper 8 bits are an alpha channel, 
 followed by an 8-bit red color channel, then an 8-bit green color channel, and 
 finally an 8-bit blue color channel. The value in each channel is an unsigned 
 normalized integer. This is a premultiplied format.
 \\
 \tcode{rgb24}
 & A 32-bit RGB color model pixel format. The upper 8 bits are unused, followed by an 8-bit red 
 color channel, then an 8-bit green color channel, and finally an 8-bit blue color channel. 
 \\
 \tcode{a8}
 & An 8-bit transparency data pixel format. All 8 bits are an alpha channel.
 \\
 \tcode{rgb16_565}
 & A 16-bit RGB color model pixel format. The upper 5 bits are a red color channel,
 followed by a 6-bit green color channel, and finally a 5-bit blue color channel.
 \\
 \tcode{rgb30}
 & A 32-bit RGB color model pixel format. The upper 2 bits are unused, followed by a 10-bit red 
 color channel, a 10-bit green color channel, and finally a 10-bit blue color channel. The value 
 in each channel is an unsigned normalized integer.
 \\
\end{libreqtab2}

%!TEX root = io2d.tex
\rSec0 [\iotwod.scaling] {Enum class \tcode{scaling}}

\rSec1 [\iotwod.scaling.summary] {\tcode{scaling} Summary}

\pnum
The scaling enum class specifies the type of scaling a \tcode{display_surface} 
will use when the size of its Display Buffer (\ref{\iotwod.displaysurface.intro}) differs from the size of its Back Buffer (\ref{\iotwod.displaysurface.intro}).

\pnum
See Table~\ref{tab:\iotwod.scaling.meanings} for the meaning of each \tcode{scaling} enumerator.

\rSec1 [\iotwod.scaling.synopsis] {\tcode{scaling} Synopsis}

\begin{codeblock}
namespace std { namespace experimental { namespace io2d { inline namespace v1 {
  enum class scaling {
    letterbox,
    uniform,
    fill_uniform,
    fill_exact,
    none
  };
} } } }
\end{codeblock}

\rSec1 [\iotwod.scaling.enumerators] {\tcode{scaling} Enumerators}

\pnum
\begin{note}
In the following table, examples will be given to help explain the meaning of each enumerator. The examples will all use a \tcode{display_surface} called \tcode{ds}.

The Back Buffer (\ref{\iotwod.displaysurface.intro}) of \tcode{ds} is 640x480 (i.e. it has a width of 640 pixels and a height of 480 pixels), giving it an aspect ratio of $1.\bar{3}$.

The Display Buffer (\ref{\iotwod.displaysurface.intro}) of \tcode{ds} is 1280x720, giving it an aspect ratio of $1.\bar{7}$.

When a rectangle is defined in an example, the coordinate $(x1,y1)$ denotes the top left corner of the rectangle, inclusive, and the coordinate $(x2,y2)$ denotes the bottom right corner of the rectangle, exclusive. As such, a rectangle with $(x1,y1) = (10,10)$, $(x2,y2) = (20, 20)$ is 10 pixels wide and 10 pixels tall and includes the pixel $(x,y) = (19,19)$ but does not include the pixels $(x,y) = (20,19)$ or $(x,y) = (19,20)$.
\end{note}

\begin{libreqtab2}
 {\tcode{scaling} enumerator meanings}
 {tab:\iotwod.scaling.meanings}
 \\ \topline
 \lhdr{Enumerator}
 & \rhdr{Meaning}
 \\ \capsep
 \endfirsthead
 \continuedcaption\\
 \hline
 \lhdr{Enumerator}
 & \rhdr{Meaning}
 \\ \capsep
 \endhead
 \tcode{letterbox}
 & Fill the Display Buffer with the Letterbox Brush (\ref{\iotwod.displaysurface.state}) of the \tcode{display_surface}. Uniformly scale the Back Buffer so that one dimension of it is the same length as the same dimension of the Display Buffer and the second dimension of it is not longer than the second dimension of the Display Buffer and transfer the scaled Back Buffer to the Display Buffer using sampling such that it is centered in the Display Buffer.

 \begin{example}
 The Display Buffer of \tcode{ds} will be filled with the \tcode{brush} object returned by \tcode{ds.letterbox_brush();}. The Back Buffer of \tcode{ds} will be scaled so that it is 960x720, thereby retaining its original aspect ratio. The scaled Back Buffer will be transfered to the Display Buffer using sampling such that it is in the rectangle $(x1,y1) = (\dfrac{1280}{2} - \dfrac{960}{2},0) = (160,0)$, $(x2,y2) = (960 + (\dfrac{1280}{2} - \dfrac{960}{2}),720) = (1120,720)$. This fulfills all of the conditions. At least one dimension of the scaled Back Buffer is the same length as the same dimension of the Display Buffer (both have a height of 720 pixels). The second dimension of the scaled Back Buffer is not longer than the second dimension of the Display Buffer (the Back Buffer's scaled width is 960 pixels, which is not longer than the Display Buffer's width of 1280 pixels. Lastly, the scaled Back Buffer is centered in the Display Buffer (on the $x$ axis there are 160 pixels between each vertical side of the scaled Back Buffer and the nearest vertical edge of the Display Buffer and on the $y$ axis there are 0 pixels between each horizontal side of the scaled Back Buffer and the nearest horizontal edge of the Display Buffer).
 \end{example}
 \\
 \tcode{uniform}
 & Uniformly scale the Back Buffer so that one dimension of it is the same length as the same dimension of the Display Buffer and the second dimension of it is not longer than the second dimension of the Display Buffer and transfer the scaled Back Buffer to the Display Buffer using sampling such that it is centered in the Display Buffer.
 
 \begin{example}
 The Back Buffer of \tcode{ds} will be scaled so that it is 960x720, thereby retaining its original aspect ratio. The scaled Back Buffer will be transfered to the Display Buffer using sampling such that it is in the rectangle $(x1,y1) = (\dfrac{1280}{2} - \dfrac{960}{2},0) = (160,0)$, $(x2,y2) = (960 + (\dfrac{1280}{2} - \dfrac{960}{2}),720) = (1120,720)$. This fulfills all of the conditions. At least one dimension of the scaled Back Buffer is the same length as the same dimension of the Display Buffer (both have a height of 720 pixels). The second dimension of the scaled Back Buffer is not longer than the second dimension of the Display Buffer (the Back Buffer's scaled width is 960 pixels, which is not longer than the Display Buffer's width of 1280 pixels. Lastly, the scaled Back Buffer is centered in the Display Buffer (on the $x$ axis there are 160 pixels between each vertical side of the scaled Back Buffer and the nearest vertical edge of the Display Buffer and on the $y$ axis there are 0 pixels between each horizontal side of the scaled Back Buffer and the nearest horizontal edge of the Display Buffer).
 \end{example}
 \begin{note}
 The difference between \tcode{uniform} and \tcode{letterbox} is that \tcode{uniform} does not modify the contents of the Display Buffer that fall outside of the rectangle into which the scaled Back Buffer is drawn while \tcode{letterbox} fills those areas with the \tcode{display_surface} object's Letterbox Brush.
 \end{note}
 \\
 \tcode{fill_uniform}
 & Uniformly scale the Back Buffer so that one dimension of it is the same length as the same dimension of the Display Buffer and the second dimension of it is not shorter than the second dimension of the Display Buffer and transfer the scaled Back Buffer to the Display Buffer using sampling such that it is centered in the Display Buffer.
 
 \begin{example}
 The Back Buffer of \tcode{ds} will be drawn in the rectangle $(x1,y1) = (0,-120)$, $(x2,y2) = (1280,840)$. This fulfills all of the conditions. At least one dimension of the scaled Back Buffer is the same length as the same dimension of the Display Buffer (both have a width of 1280 pixels). The second dimension of the scaled Back Buffer is not shorter than the second dimension of the Display Buffer (the Back Buffer's scaled height is 840 pixels, which is not shorter than the Display Buffer's height of 720 pixels). Lastly, the scaled Back Buffer is centered in the Display Buffer (on the $x$ axis there are 0 pixels between each vertical side of the rectangle and the nearest vertical edge of the Display Buffer and on the $y$ axis there are 120 pixels between each horizontal side of the rectangle and the nearest horizontal edge of the Display Buffer).
 \end{example} 
 \\
 \tcode{fill_exact}
 & Scale the Back Buffer so that each dimension of it is the same length as the same dimension of the Display Buffer and transfer the scaled Back Buffer to the Display Buffer using sampling such that its origin is at the origin of the Display Buffer.
 
 \begin{example}
 The Back Buffer will be drawn in the rectangle $(x1,y1) = (0,0)$, $(x2,y2) = (1280,720)$. This fulfills all of the conditions. Each dimension of the scaled Back Buffer is the same length as the same dimension of the Display Buffer (both have a width of 1280 pixels and a height of 720 pixels) and the origin of the scaled Back Buffer is at the origin of the Display Buffer.
 \end{example}
 \\
 \tcode{none}
 & Do not perform any scaling. Transfer the Back Buffer to the Display Buffer using sampling such that its origin is at the origin of the Display Buffer.
 
 \begin{example}
 The Back Buffer of \tcode{ds} will be drawn in the rectangle $(x1,y1) = (0,0)$, $(x2,y2) = (640,480)$ such that no scaling occurs and the origin of the Back Buffer is at the origin of the Display Buffer.
 \end{example}
 \\
\end{libreqtab2}

%!TEX root = io2d.tex
\rSec0 [\iotwod.refreshrate] {Enum class \tcode{refresh_rate}}

\rSec1 [\iotwod.refreshrate.summary] {\tcode{refresh_rate} Summary}

\pnum
The \tcode{refresh_rate} enum class describes when the Draw Callback (Table~\ref{tab:\iotwod.displaysurface.state.listing}) of a \tcode{display_surface} object shall be called. See Table~\ref{tab:\iotwod.refreshrate.meanings} for the meaning of each \tcode{} enumerator.

\rSec1 [\iotwod.refreshrate.synopsis] {\tcode{refresh_rate} Synopsis}

\begin{codeblock}
namespace std::experimental::io2d::v1 {
  enum class refresh_rate {
    as_needed,
    as_fast_as_possible,
    fixed
  };
}
\end{codeblock}

\rSec1 [\iotwod.refreshrate.enumerators] {\tcode{refresh_rate} Enumerators}

\begin{libreqtab2}
 {\tcode{refresh_rate} value meanings}
 {tab:\iotwod.refreshrate.meanings}
 \\ \topline
 \lhdr{Enumerator}
 & \rhdr{Meaning}
 \\ \capsep
 \endfirsthead
 \continuedcaption\\
 \hline
 \lhdr{Enumerator}
 & \rhdr{Meaning}
 \\ \capsep
 \endhead
 \tcode{as_needed}
 & The Draw Callback shall be called when the implementation needs to do so.
 \begin{note}
 The intention of this enumerator is that implementations will call the Draw Callback as little as possible in order to minimize power usage. Users can call \tcode{display_surface::redraw_required} to make the implementation run the Draw Callback whenever the user requires.
 \end{note}
 \\
 \tcode{as_fast_as_possible}
 & The Draw Callback shall be called as frequently as possible, subject to any limits of the execution environment and the \underlyingrendandpresenttechs.
 \\
 \tcode{fixed}
 & The Draw Callback shall be called as frequently as needed to maintain the Desired Frame Rate (Table~\ref{tab:\iotwod.displaysurface.state.listing}) as closely as possible. If more time has passed between two successive calls to the Draw Callback than is required, it shall be called \term{excess time} and it shall count towards the \term{required time}, which is the time that is required to pass after a call to the Draw Callback before the next successive call to the Draw Callback shall be made. If the excess time is greater than the required time, implementations shall call the Draw Callback and then repeatedly subtract the required time from the excess time until the excess time is less than the required time. If the implementation needs to call the Draw Callback for some other reason, it shall use that call as the new starting point for maintaining the Desired Frame Rate.
 \begin{example}
 Given a Desired Frame Rate of \tcode{20.0f}, then as per the above, the implementation would call the Draw Callback at 50 millisecond intervals or as close thereto as possible.
 
 If for some reason the excess time is 51 milliseconds, the implementation would call the Draw Callback, subtract 50 milliseconds from the excess time, and then would wait 49 milliseconds before calling the Draw Callback again.
 
 If only 15 milliseconds have passed since the Draw Callback was last called and the implementation needs to call the Draw Callback again, then the implementation shall call the Draw Callback immediately and proceed to wait 50 milliseconds before calling the Draw Callback again.
 \end{example}
 \\
\end{libreqtab2}

%!TEX root = io2d.tex

\rSec0 [\iotwod.imagefileformat] {Enum class \tcode{image_file_format}}

\rSec1 [\iotwod.imagefileformat.summary] {\tcode{image_file_format} summary}

\pnum
The \tcode{image_file_format} enum class specifies the data format that an \tcode{image_surface} object is constructed from or saved to. This allows data in a format that is required to be supported to be read or written regardless of its extension.

\pnum
It also has a value that allows implementations to support additional file formats if it recognizes them.

\rSec1 [\iotwod.imagefileformat.synopsis] {\tcode{image_file_format} synopsis}

\indexlibrary{\idxcode{image_file_format}}
\begin{codeblock}
namespace @\fullnamespace{}@ {
  enum class image_file_format {
    unknown,
    png,
    jpeg,
    tiff
  };
}
\end{codeblock}

\rSec1 [\iotwod.imagefileformat.enumerators] {\tcode{image_file_format} enumerators}

\begin{libreqtab2}
 {\tcode{imagefileformat} enumerator meanings}
 {tab:\iotwod.imagefileformat.meanings}
 \\ \topline
 \lhdr{Enumerator}
 & \rhdr{Meaning}
 \\ \capsep
 \endfirsthead
 \continuedcaption\\
 \hline
 \lhdr{Enumerator}
 & \rhdr{Meaning}
 \\ \capsep
 \endhead
 \tcode{unknown}
 & The format is unknown because it is not an image file format that is required to be supported. It may be known and supported by the implementation.
 \\
 \tcode{png}
 & The PNG format.
 \\
 \tcode{jpeg}
 & The JPEG format.
 \\
 \tcode{tiff}
 & The TIFF format.
 \\
\end{libreqtab2}

%!TEX root = io2d.tex
\rSec0 [device] {Class \tcode{device}}

\rSec1 [device.synopsis] {\tcode{device} synopsis}

\begin{codeblock}
namespace std { namespace experimental { namespace io2d { inline namespace v1 {
  class device {
  public:
    // See~\ref{\iotwod.req.native}
    typedef @\impdef@ native_handle_type; //              \expos
    native_handle_type native_handle() const noexcept; // \expos

    device() = delete;
    device(const device&) = delete;
    device& operator=(const device&) = delete;
    device(device&& other);
    device& operator=(device&& other);

    // \ref{device.modifiers}, modifiers:
    void flush() noexcept;
    void lock();
    void lock(error_code& ec) noexcept;
    void unlock();
    void unlock(error_code& ec) noexcept;
  };
} } } }
\end{codeblock}

\rSec1 [device.intro] {\tcode{device} Description}

\pnum
\indexlibrary{\idxcode{device}}
The \tcode{device} class provides access to the underlying rendering and 
presentation technologies, such graphics devices, graphics device contexts, and swap chains.

\pnum
A \tcode{device} object is obtained from a \tcode{surface} or \tcode{surface}-derived object.

\rSec1 [device.modifiers] {\tcode{device} modifiers}

\indexlibrary{\idxcode{device}!\idxcode{flush}}
\indexlibrary{\idxcode{flush}!\idxcode{device}}
\begin{itemdecl}
void flush() noexcept;
\end{itemdecl}
\begin{itemdescr}
	\pnum
	\effects
	The user shall be able to manipulate the underlying rendering and 
	presentation technologies used by the implementation without introducing a 
	race condition.
	
	\pnum
	\postconditions
	Any pending device operations shall be executed, batched, or otherwise committed to the underlying rendering and presentation technologies.
	
	\pnum 
	Saved device state, if any, shall be restored.

	\pnum
	\remarks
	This function exists primarily to allow the user to take control of the 
	underlying rendering and presentation technologies using an 
	implementation-provided native handle.
	
	\pnum
	The implementation's responsibility is to ensure that the user can safely make changes to the underlying rendering and presentation technologies using a native handle after calling this function.
	
	\pnum
	The implementation is not required to ensure that every last operation has fully completed so long as those operations which are not complete do not prevent safe use of the underlying rendering and presentation technologies.

	\pnum
	If the underlying technologies internally batch operations in a way that allows them to receive and batch further commands without introducing race conditions, the implementation should return as soon as all pending operations have been submitted to the batch queue.
	
	\pnum
	This function should not flush the surface to which the device is bound.
	
	\pnum
	If the implementation does not provide a native handle to the underlying rendering and presentation technologies, this function shall have no observable behavior.
	
	\pnum
	\begin{note}
	Users call this function because they wish to use a native handle to the \underlyingrendandpresenttechs in order to do something not provided by this \documenttypename{} (e.g. render native UI controls). As such, the user needs to know that using the underlying rendering system outside of this library will not introduce any race conditions. This function, in combination with locking the device, exists to provide that surety.
	\end{note}
\end{itemdescr}

\indexlibrary{\idxcode{device}!\idxcode{lock}}
\indexlibrary{\idxcode{lock}!\idxcode{device}}
\begin{itemdecl}
void lock();
void lock(error_code& ec) noexcept;
\end{itemdecl}
\begin{itemdescr}
	\pnum
	\effects
	Produces all effects of \tcode{m.lock()} from \term{BasicLockable}, 
	30.2.5.2 in \CppXIV. Implementations shall make this function capable of 
	being recursively reentered from the same thread.
	
	\pnum
	\throws
	As described in Error reporting (\ref{\iotwod.err.report}).
	
	\pnum
	\errors
	\tcode{errc::resource_unavailable_try_again} if a lock cannot be obtained.
	\begin{note}
	One reason this error may occur is if a system limit on the maximum number of times a lock could be recursively acquired would be exceeded.
	\end{note}
\end{itemdescr}

\indexlibrary{\idxcode{device}!\idxcode{unlock}}
\indexlibrary{\idxcode{unlock}!\idxcode{device}}
\begin{itemdecl}
void unlock();
void unlock(error_code& ec) noexcept;
\end{itemdecl}
\begin{itemdescr}
	\pnum
	\requires
	Meets all requirements of \tcode{m.unlock()} from \term{BasicLockable}, 
	30.2.5.2 in \CppXIV.
	
	\pnum
	\effects
	Produces all effects of \tcode{m.unlock()} from \term{BasicLockable}, 
	30.2.5.2 in \CppXIV. The lock on \tcode{m} shall not be fully released 
	until \tcode{m.unlock} has been called a number of times equal to the 
	number of times \tcode{m.lock} was successfully called.
	
	\pnum
	\throws
	As described in Error reporting (\ref{\iotwod.err.report}).
	
	\pnum
	\remarks
	This function shall not be called more times than \tcode{lock} has been called; no diagnostic is required.

\end{itemdescr}

%!TEX root = io2d.tex
\rSec0 [surface] {Class \tcode{surface}}

\rSec1 [surface.intro] {\tcode{surface} description}

\pnum
\indexlibrary{\idxcode{surface}}
The \tcode{surface} class provides an interface for managing a graphics
data graphics resource.

\pnum
A \tcode{surface} object is a move-only object.

\pnum
The \tcode{surface} class provides two ways to modify its graphics resource:
\begin{itemize}
	\item Rendering and composing operations.
	\item Mapping.
\end{itemize}

\pnum
\begin{note}
While a \tcode{surface} object manages a graphics data graphics resource, the \tcode{surface} class does not provide well-defined semantics for the graphics resource. The \tcode{surface} class is intended to serve only as a base class and as such is not directly instantiable.
\end{note}

\pnum
Directly instantiable types which derive, directly or indirectly, from the \tcode{surface} class shall either provide well-defined semantics for the graphics data graphics resource or inherit well-defined semantics for the graphics data graphics resource from a base class.

\pnum
\begin{example}
The \tcode{image_surface} class and the \tcode{display_surface} class each specify that they manage a raster graphics data graphics resource and that the members they inherit from the \tcode{surface} class shall use that raster graphics data graphics resource as their graphics data graphics resource. Since, unlike graphics data, raster graphics data provides well-defined semantics, these classes meet the requirements for being directly instantiable.
\end{example}

\pnum
The definitions of the rendering and composing operations in \ref{surface.rendering} shall only be applicable when the graphics data graphics resource on which the \tcode{surface} members operate is a raster graphics data graphics resource. In all other cases, any attempt to invoke the rendering and composing operations shall result in undefined behavior.

\rSec1 [surface.synopsis] {\tcode{surface} synopsis}

\begin{codeblock}
namespace std { namespace experimental { namespace io2d { inline namespace v1 {
  class surface {
  public:
    surface() = delete;
    
    // \ref{surface.modifiers.state}, state modifiers:
    void flush();
    void flush(error_code& ec) noexcept;
    void mark_dirty();
    void mark_dirty(error_code& ec) noexcept;
    void mark_dirty(const rectangle& rect);
    void mark_dirty(const rectangle& rect, error_code& ec) noexcept;
    void map(const function<void(mapped_surface&)>& action);
    void map(const function<void(mapped_surface&, error_code&)>& action,
      error_code& ec);
    void map(const function<void(mapped_surface&)>& action,
      const rectangle& extents);
    void map(const function<void(mapped_surface&, error_code&)>& action,
      const rectangle& extents, error_code& ec);

    // \ref{surface.modifiers.render}, render modifiers:
    void paint(const brush& b, const optional<brush_props>& bp = nullopt,
      const optional<render_props>& rp = nullopt,
      const optional<clip_props>& cl = nullopt);
    template <class Allocator>
    void stroke(const brush& b, const path_builder<Allocator>& pb,
      const optional<brush_props>& bp = nullopt,
      const optional<stroke_props>& sp = nullopt,
      const optional<dashes>& d = nullopt,
      const optional<render_props>& rp = nullopt,
      const optional<clip_props>& cl = nullopt);
    void stroke(const brush& b, const path_group& pg,
      const optional<brush_props>& bp = nullopt,
      const optional<stroke_props>& sp = nullopt,
      const optional<dashes>& d = nullopt,
      const optional<render_props>& rp = nullopt,
      const optional<clip_props>& cl = nullopt);
    template <class Allocator>
    void fill(const brush& b, const path_builder<Allocator>& pb,
      const optional<brush_props>& bp = nullopt,
      const optional<render_props>& rp = nullopt,
      const optional<clip_props>& cl = nullopt);
    void fill(const brush& b, const path_group& pg,
      const optional<brush_props>& bp = nullopt,
      const optional<render_props>& rp = nullopt,
      const optional<clip_props>& cl = nullopt);
    template <class Allocator>
    void mask(const brush& b, const brush& mb,
      const optional<brush_props>& bp = nullopt,
      const optional<mask_props>& mp = nullopt,
      const optional<render_props>&rp = nullopt,
      const optional<clip_props>& cl = nullopt);
    void mask(const brush& b, const brush& mb,
      const optional<brush_props>& bp = nullopt,
      const optional<mask_props>& mp = nullopt,
      const optional<render_props>&rp = nullopt,
      const optional<clip_props>& cl = nullopt);
  };
} } } }
\end{codeblock}

\rSec1 [surface.rendering] {Rendering and composing}

\rSec2 [surface.rendering.ops] {Operations}

\pnum
The \tcode{surface} class provides four fundamental rendering and composing operations:
\begin{libreqtab2}
 {\tcode{surface} rendering and composing operations}
 {tab:surface.rendering.operations}
 \\ \topline
 \lhdr{Operation}
 & \rhdr{Function(s)}
 \\ \capsep
 \endfirsthead
 \continuedcaption\\
 \hline
 \lhdr{Operation}
 & \rhdr{Function(s)}
 \\ \capsep
 \endhead
 Painting
 & \tcode{surface::paint}
 \\
 Filling
 & \tcode{surface::fill}
 \\
 Stroking
 & \tcode{surface::stroke}
 \\
 Masking
 & \tcode{surface::mask}
 \\
\end{libreqtab2}

\rSec2 [surface.rendering.brushes] {Rendering and composing brushes}

\pnum
All rendering and composing operations use a Source Brush of type \tcode{brush}.

\pnum
The Masking rendering and composing operation uses a Mask Brush of type \tcode{brush}.

\rSec2 [surface.rendering.sourcepath] {Rendering and composing source path}

\pnum
In addition to brushes (\ref{surface.rendering.brushes}), all rendering and composing operation except for Painting use a Source Path of type \tcode{path_group}.

\rSec2 [surface.rendering.commonstate] {Common state data}

\pnum
All rendering and composing operations use the following state data:

\begin{libreqtab2}
 {\tcode{surface} rendering and composing common state data}
 {tab:surface.rendering.commonstate.listing}
 \\ \topline
 \lhdr{Name}
 & \rhdr{Type}
 \\ \capsep
 \endfirsthead
 \hline
 \lhdr{Name}
 & \rhdr{Type}
 \\ \capsep
 \endhead
 Brush Properties
 & \tcode{brush_props}
 \\
 Surface Properties
 & \tcode{render_props}
 \\
 Clip Properties
 & \tcode{clip_props}
 \\
\end{libreqtab2}

\rSec2 [surface.rendering.specificstate] {Specific state data}

\pnum
In addition to the common state data (\ref{surface.rendering.commonstate}), certain rendering and composing operations use state data that is specific to each of them:

\begin{libiotwodtab3e}
 {\tcode{surface} rendering and composing specific state data}
 {tab:surface.rendering.specificstate.listing}
 \\ \topline
 \lhdr{Operation}
 & \chdr{Name}
 & \rhdr{Type}
 \\ \capsep
 \endfirsthead
 \hline
 \lhdr{Operation}
 & \chdr{Name}
 & \rhdr{Type}
 \\ \capsep
 \endhead
 Stroking
 & Stroke Properties
 & \tcode{stroke_props}
 \\
 Masking
 & Mask Properties
 & \tcode{mask_props}
 \\
\end{libiotwodtab3e}

\rSec2 [surface.rendering.statedefaults] {State data default values}
\pnum
For all rendering and composing operations, the state data objects named above are provided using \tcode{optional<T>} class template arguments.

\pnum
If there is no contained value for a state data object, it is interpreted as-if the \tcode{optional<T>} argument contained a default constructed object of the relevant state data object.

\rSec1 [surface.coordinatespaces] {Standard coordinate spaces}

\pnum
There are four standard coordinate spaces relevant to the rendering and composing operations (\ref{surface.rendering}):
\begin{itemize}
\item the Brush Coordinate Space;
\item the Mask Coordinate Space;
\item the User Coordinate Space; and
\item the Surface Coordinate Space.
\end{itemize}

\pnum
The \term{Brush Coordinate Space} is the standard coordinate space of the Source Brush (\ref{surface.rendering.brushes}). Its transformation matrix is the Brush Properties' Brush Matrix (\ref{brushprops.summary}).

\pnum
The \term{Mask Coordinate Space} is the standard coordinate space of the Mask Brush (\ref{surface.rendering.brushes}). Its transformation matrix is the Mask Properties' Mask Matrix (\ref{maskprops.summary}).

\pnum
The \term{User Coordinate Space} is the standard coordinate space of \tcode{path_group} objects. Its transformation matrix is a default-constructed \tcode{matrix_2d}.

\pnum
The \term{Surface Coordinate Space} is the standard coordinate space of the \tcode{surface} object's \underlyingsurface. Its transformation matrix is the Surface Properties' Surface Matrix (\ref{renderprops.summary}).

\pnum
Given a point \tcode{pt}, a Brush Coordinate Space transformation matrix \tcode{bcsm}, a Mask Coordinate Space transformation matrix \tcode{mcsm}, a User Coordinate Space transformation matrix \tcode{ucsm}, and a Surface Coordinate Space transformation matrix \tcode{scsm}, the following table describes how to transform it from each of these standard coordinate spaces to the other standard coordinate spaces:

\begin{libiotwodreqtab3}
 {Point transformations}
 {tab:surface.pointtransforms.listing}
 \\ \topline
 \lhdr{From}
 & \chdr{To}
 & \rhdr{Transform}
 \\ \capsep
 \endfirsthead
 \continuedcaption\\
 \hline
 \lhdr{From}
 & \chdr{To}
 & \rhdr{Transform}
 \\ \capsep
 \endhead
 Brush Coordinate Space
 & Mask Coordinate Space
 & \tcode{mcsm.transform_point(bcsm.invert().transform_point(pt))}.
 \\
 Brush Coordinate Space
 & User Coordinate Space
 & \tcode{bcsm.invert().transform_point(pt)}.
 \\
 Brush Coordinate Space
 & Surface Coordinate Space
 & \tcode{scsm.transform_point(bcsm.invert().transform_point(pt))}.
 \\
 User Coordinate Space
 & Brush Coordinate Space
 & \tcode{bcsm.transform_point(pt)}.
 \\
 User Coordinate Space
 & Mask Coordinate Space
 & \tcode{mcsm.transform_point(pt)}.
 \\
 User Coordinate Space
 & Surface Coordinate Space
 & \tcode{scsm.transform_point(pt)}.
 \\
 Surface Coordinate Space
 & Brush Coordinate Space
 & \tcode{bcsm.transform_point(scsm.invert().transform_point(pt))}.
 \\
 Surface Coordinate Space
 & Mask Coordinate Space
 & \tcode{mcsm.transform_point(scsm.invert().transform_point(pt))}.
 \\
 Surface Coordinate Space
 & User Coordinate Space
 & \tcode{scsm.invert().transform_point(pt)}.
 \\
\end{libiotwodreqtab3}

\rSec1 [surface.painting] {\tcode{surface} painting}

\pnum
When a Painting operation is initiated on a surface, the implementation shall produce results as-if the following steps were performed:

\begin{enumerate}
\item For each integral point $sp$ of the \underlyingsurface, determine if $sp$ is within the Clip Area (\tcode{clipprops.summary}); if so, proceed with the remaining steps.
\item Transform $sp$ from the Surface Coordinate Space (\ref{surface.coordinatespaces}) to the Brush Coordinate Space (Table~\ref{tab:surface.pointtransforms.listing}), resulting in point $bp$.
\item Sample from point $bp$ of the Source Brush (\ref{surface.rendering.brushes}), combine the resulting visual data with the visual data at point $sp$ in the \underlyingsurface in the manner specified by the surface's current Composition Operator (\ref{renderprops.summary}), and modify the visual data of the \underlyingsurface at point $sp$ to reflect the result produced by application of the Composition Operator.
\end{enumerate}

\rSec1 [surface.filling] {\tcode{surface} filling}

\pnum
When a Filling operation is initiated on a surface, the implementation shall produce results as-if the following steps were performed:

\begin{enumerate}
\item For each integral point $sp$ of the \underlyingsurface, determine if $sp$ is within the Clip Area (\ref{clipprops.summary}); if so, proceed with the remaining steps.
\item Transform $sp$ from the Surface Coordinate Space (\ref{surface.coordinatespaces}) to the User Coordinate Space (Table~\ref{tab:surface.pointtransforms.listing}), resulting in point $up$.
\item Using the Source Path (\ref{surface.rendering.sourcepath}) and the Fill Rule (\ref{brushprops.summary}), determine whether $up$ shall be filled; if so, proceed with the remaining steps.
\item Transform $up$ from the User Coordinate Space to the Brush Coordinate Space (\ref{surface.coordinatespaces} and Table~\ref{tab:surface.pointtransforms.listing}), resulting in point $bp$.
\item Sample from point $bp$ of the Source Brush (\ref{surface.rendering.brushes}), combine the resulting visual data with the visual data at point $sp$ in the \underlyingsurface in the manner specified by the surface's current Composition Operator (\ref{renderprops.summary}), and modify the visual data of the \underlyingsurface at point $sp$ to reflect the result produced by application of the Composition Operator.
\end{enumerate}

\rSec1 [surface.stroking] {\tcode{surface} stroking}

\pnum
When a Stroking operation is initiated on a surface, the implementation shall carry out the Stroking operation for each path in the Source Path (\ref{surface.rendering}).

\pnum
The following rules shall apply when a Stroking operation is carried out on a pathy:
\begin{enumerate}
\item No part of the \underlyingsurface that is outside of the Clip Area shall be modified.

\item If the path only contains a degenerate path segment, then if the Line Cap value is either \tcode{line_cap::round} or \tcode{line_cap::square}, the line caps shall be rendered, resulting in a circle or a square, respectively. The remaining rules shall not apply.

\item If the path is a closed path, then the point where the end point of its final path segment meets the start point of the initial path segment shall be rendered as specified by the Line Join value; otherwise the start point of the initial path segment and end point of the final path segment shall each by rendered as specified by the Line Cap value. The remaining meetings between successive end points and start points shall be rendered as specified by the Line Join value.

\item If the Dash Pattern has its default value or if its \tcode{vector<double>} member is empty, the path segments shall be rendered as a continuous path.

\item If the Dash Pattern's \tcode{vector<double>} member contains only one value, that value shall be used to define a repeating pattern in which the path is shown then hidden. The ends of each shown portion of the path shall be rendered as specified by the Line Cap value.

\item If the Dash Pattern's \tcode{vector<double>} member contains two or more values, the values shall be used to define a pattern in which the path is alternatively rendered then not rendered for the length specified by the value. The ends of each rendered portion of the path shall be rendered as specified by the Line Cap value. If the Dash Pattern's \tcode{double} member, which specifies an offset value, is not \tcode{0.0}, the meaning of its value is \impldefplain{Dash Pattern!offset value}. If a rendered portion of the path overlaps a not rendered portion of the path, the rendered portion shall be rendered.
\end{enumerate}

\pnum
When a Stroking operation is carried out on a path, the width of each rendered portion shall be the Line Width. Ideally this means that the diameter of the stroke at each rendered point should be equal to the Line Width. However, because there is an infinite number of points along each rendered portion, implementations may choose an \unspecnorm method of determining minimum distances between points along each rendered portion and the diameter of the stroke between those points shall be the same.
\begin{note}
This concept is sometimes referred to as a tolerance. It allows for a balance between precision and performance, especially in situations where the end result is in a non-exact format such as raster graphics data.
\end{note}

\pnum
After all paths in the path group have been rendered but before the rendered result is composed to the \underlyingsurface, the rendered result shall be transformed from the User Coordinate Space (\ref{surface.coordinatespaces}) to the Surface Coordinate Space (\ref{surface.coordinatespaces}).
\begin{example}
If an open path consisting solely of a vertical line from \tcode{vector_2d(20.0, 20.0)} to \tcode{vector_2d(20.0, 60.0)} is to be composed to the \underlyingsurface, the Line Cap is \tcode{line_cap::none}, the Line Width is \tcode{12.0}, and the Transformation Matrix is \tcode{matrix_2d::init_scale(0.5, 1.0)}, then the line will end up being composed within the area \tcode{rectangle( \{ 7.0, 20.0 \}, \{ 13.0, 60.0 \} )} on the \underlyingsurface. The Transformation Matrix causes the center of the \xaxis of the line to move from \tcode{20.0} to \tcode{10.0} and then causes the horizontal width of the line to be reduced from \tcode{12.0} to \tcode{6.0}.
\end{example}

\rSec1 [surface.masking] {\tcode{surface} masking}

\pnum
A \term{Mask Brush} is composed of a graphics data graphics resource, a \tcode{wrap_mode} value, a \tcode{filter} value, and a \tcode{matrix_2d} object.

\pnum
When a Masking operation is initiated on a surface, the implementation shall produce results as-if the following steps were performed:

\begin{enumerate}
\item For each integral point $sp$ of the \underlyingsurface, determine if $sp$ is within the Clip Area (\ref{clipprops.summary}); if so, proceed with the remaining steps.
\item Transform $sp$ from the Surface Coordinate Space (\ref{surface.coordinatespaces}) to the Mask Coordinate Space (Table~\ref{tab:surface.pointtransforms.listing}), resulting in point $mp$.
\item Sample the alpha channel from point $mp$ of the Mask Brush and store the result in $mac$; if the visual data format of the Mask Brush does not have an alpha channel, the value of $mac$ shall always be $1.0$.
\item Transform $sp$ from the Surface Coordinate Space to the Brush Coordinate Space, resulting in point $bp$.
\item Sample from point $bp$ of the Source Brush (\ref{surface.rendering.brushes}), combine the resulting visual data with the visual data at point $sp$ in the \underlyingsurface in the manner specified by the surface's current Composition Operator (\ref{renderprops.summary}), multiply each channel of the result produced by application of the Composition Operator by $map$ if the visual data format of the \underlyingsurface is a premultiplied format and if not then just multiply the alpha channel of the result by $map$, and modify the visual data of the \underlyingsurface at point $sp$ to reflect the multiplied result.
\end{enumerate}

\rSec1 [surface.modifiers.state] {\tcode{surface} state modifiers}

\indexlibrary{\idxcode{surface}!\idxcode{flush}}
\indexlibrary{\idxcode{flush}!\idxcode{surface}}
\begin{itemdecl}
void flush();
void flush(error_code& ec) noexcept;
\end{itemdecl}
\begin{itemdescr}
\pnum
\effects
If the implementation does not provide a native handle to the surface's \underlyingsurface, this function does nothing.

\pnum
If the implementation does provide a native handle to the surface's \underlyingsurface, then the implementation performs every action necessary to ensure that all operations on the surface that produce observable effects occur.

\pnum
The implementation performs any other actions necessary to ensure that the surface will be usable again after a call to \tcode{surface::mark_dirty}.

\pnum
Once a call to \tcode{surface::flush} is made, \tcode{surface::mark_dirty} shall be called before any other member function of the surface is called or the surface is used as an argument to any other function.

\pnum
\throws
As specified in Error reporting (\ref{\iotwod.err.report}).

\pnum
\remarks
This function exists to allow the user to take control of the underlying surface using an implementation-provided native handle without introducing a race condition. The implementation's responsibility is to ensure that the user can safely use the underlying surface.

\pnum
\errors
The potential errors are \impldefplain{surface::flush errors}.

\pnum
Implementations should avoid producing errors here.

\pnum
If the implementation does not provide a native handle to the \tcode{surface} object's \underlyingsurface, this function shall not produce any errors.

\pnum
\begin{note}
There are several purposes for \tcode{surface::flush} and \tcode{surface::mark_dirty}.

\pnum
One is to allow implementation wide latitude in how they implement the rendering and composing operations (\ref{surface.rendering}), such as batching calls and then sending them to the \underlyingrendandpresenttechs at appropriate times.

\pnum
Another is to give implementations the chance during the call to \tcode{surface::flush} to save any internal state that might be modified by the user and then restore it during the call to \tcode{surface::mark_dirty}.

\pnum
Other uses of this pair of calls are also possible.
\end{note}
\end{itemdescr}

\indexlibrary{\idxcode{surface}!\idxcode{mark_dirty}}
\indexlibrary{\idxcode{mark_dirty}!\idxcode{surface}}
\begin{itemdecl}
void mark_dirty();
void mark_dirty(error_code& ec) noexcept;
void mark_dirty(const rectangle& extents);
void mark_dirty(const rectangle& extents, error_code& ec) noexcept;
\end{itemdecl}
\begin{itemdescr}
\pnum
\effects
If the implementation does not provide a native handle to the \tcode{surface} object's \underlyingsurface, this function shall do nothing.

\pnum
If the implementation does provide a native handle to the \tcode{surface} object's \underlyingsurface, then:
\begin{itemize}
	\item If called without a \tcode{rect} argument, informs the implementation that external changes using a native handle were potentially made to the entire \underlyingsurface.
	\item If called with a \tcode{rect} argument, informs the implementation that external changes using a native handle were potentially made to the \underlyingsurface within the bounds specified by the \term{bounding rectangle} \tcode{rectangle\{ round(extents.x()), round (extents.y()), round(extents.width()), round(extents.height())\}}. No part of the bounding rectangle shall be outside of the bounds of the \underlyingsurface; no diagnostic is required.
\end{itemize}

\pnum
\throws
As specified in Error reporting (\ref{\iotwod.err.report}).

\pnum
\remarks
After external changes are made to this \tcode{surface} object's \underlyingsurface using a native pointer, this function shall be called before using this \tcode{surface} object; no diagnostic is required.

\pnum
No call to this function shall be required solely as a result of changes made to a surface using the functionality provided by \tcode{surface::map}.
\begin{note}
The \tcode{mapped_surface} type, which is used by \tcode{surface::map}, provides its own functionality for managing any such changes.
\end{note}

\pnum
\errors
The errors, if any, produced by this function are \impldefplain{surface!mark_dirty}.

\pnum
If the implementation does not provide a native handle to the \tcode{surface} object's \underlyingsurface, this function shall not produce any errors.
\end{itemdescr}

\indexlibrary{\idxcode{surface}!\idxcode{map}}
\indexlibrary{\idxcode{map}!\idxcode{surface}}
\begin{itemdecl}
void map(const function<void(mapped_surface&)>& action);
void map(const function<void(mapped_surface&, error_code&)>& action, error_code& ec);
void map(const function<void(mapped_surface&)>& action, const rectangle& extents);
void map(const function<void(mapped_surface&, error_code&)>& action,
  const rectangle& extents, error_code& ec);
\end{itemdecl}
\begin{itemdescr}
\pnum
\effects
Creates a \tcode{mapped_surface} object and calls \tcode{action} using it.

\pnum
The \tcode{mapped_surface} object is created using \tcode{*this}, which allows direct manipulation of the \underlyingsurface.

\pnum
If called with a \tcode{const rectangle\& extents} argument, the \tcode{mapped_surface} object shall only allow manipulation of the portion of \tcode{*this} specified by the \term{bounding rectangle} \\
\tcode{rectangle\{ round(extents.x()), round(extents.y()), round(extents.width()), \\
round(extents.height())\}}. If any part of the bounding rectangle is outside of the bounds of \tcode{*this}, the call shall result in undefined behavior; no diagnostic is required.

\pnum
\throws
As specified in Error reporting (\ref{\iotwod.err.report}).

\pnum
\remarks
Whether changes are committed to the \underlyingsurface immediately or only when the \tcode{mapped_surface} object is destroyed is \unspecnorm.

\pnum
Calling this function on a \tcode{surface} object and then calling any function on the \tcode{surface} object or using the \tcode{surface} object before the call to this function has returned shall result in undefined behavior; no diagnostic is required.

\pnum
\errors
The errors, if any, produced by this function are \impldefplain{surface!map} or are produced by the user-provided function passed via the \tcode{action} argument.
\end{itemdescr}

\rSec1 [surface.modifiers.render] {\tcode{surface} render modifiers}

\indexlibrary{\idxcode{surface}!\idxcode{paint}}
\indexlibrary{\idxcode{paint}!\idxcode{surface}}
\begin{itemdecl}
void paint(const brush& b, const optional<brush_props>& bp = nullopt,
  const optional<render_props>& rp = nullopt,
  const optional<clip_props>& cl = nullopt);
\end{itemdecl}
\begin{itemdescr}
\pnum
\effects
Performs the Painting rendering and composing operation as specified by \ref{surface.painting}.

\pnum
The meanings of the parameters are specified by \ref{surface.rendering}.

\pnum
\throws
As specified in Error reporting (\ref{\iotwod.err.report}).

\pnum
\errors
The errors, if any, produced by this function are \impldefplain{surface!paint}.
\end{itemdescr}

\indexlibrary{\idxcode{surface}!\idxcode{stroke}}
\indexlibrary{\idxcode{stroke}!\idxcode{surface}}
\begin{itemdecl}
template <class Allocator>
void stroke(const brush& b, const path_builder<Allocator>& pb,
  const optional<brush_props>& bp = nullopt,
  const optional<stroke_props>& sp = nullopt,
  const optional<dashes>& d = nullopt,
  const optional<render_props>& rp = nullopt,
  const optional<clip_props>& cl = nullopt);
void stroke(const brush& b, const path_group& pg,
  const optional<brush_props>& bp = nullopt,
  const optional<stroke_props>& sp = nullopt,
  const optional<dashes>& d = nullopt,
  const optional<render_props>& rp = nullopt,
  const optional<clip_props>& cl = nullopt);
\end{itemdecl}
\begin{itemdescr}
\pnum
\effects
Performs the Stroking rendering and composing operation as specified by \ref{surface.stroking}.

\pnum
The meanings of the parameters are specified by \ref{surface.rendering}.

\pnum
\throws
As specified in Error reporting (\ref{\iotwod.err.report}).

\pnum
\errors
The errors, if any, produced by this function are \impldefplain{surface!stroke}.
\end{itemdescr}

\indexlibrary{\idxcode{surface}!\idxcode{fill}}
\indexlibrary{\idxcode{fill}!\idxcode{surface}}
\begin{itemdecl}
template <class Allocator>
void fill(const brush& b, const path_builder<Allocator>& pb,
  const optional<brush_props>& bp = nullopt,
  const optional<render_props>& rp = nullopt,
  const optional<clip_props>& cl = nullopt);
void fill(const brush& b, const path_group& pg,
  const optional<brush_props>& bp = nullopt,
  const optional<render_props>& rp = nullopt,
  const optional<clip_props>& cl = nullopt);
\end{itemdecl}
\begin{itemdescr}
\pnum
\effects
Performs the Filling rendering and composing operation as specified by \ref{surface.filling}.

\pnum
The meanings of the parameters are specified by \ref{surface.rendering}.

\pnum
\throws
As specified in Error reporting (\ref{\iotwod.err.report}).

\pnum
\errors
The errors, if any, produced by this function are \impldefplain{surface!fill}.
\end{itemdescr}

\indexlibrary{\idxcode{surface}!\idxcode{mask}}
\indexlibrary{\idxcode{mask}!\idxcode{surface}}
\begin{itemdecl}
template <class Allocator>
void mask(const brush& b, const brush& mb,
  const path_builder<Allocator>& pb,
  const optional<brush_props>& bp = nullopt,
  const optional<mask_props>& mp = nullopt,
  const optional<render_props>&rp = nullopt,
  const optional<clip_props>& cl = nullopt);
void mask(const brush& b, const brush& mb, const path_group& pg,
  const optional<brush_props>& bp = nullopt,
  const optional<mask_props>& mp = nullopt,
  const optional<render_props>&rp = nullopt,
  const optional<clip_props>& cl = nullopt);
\end{itemdecl}
\begin{itemdescr}
\pnum
\effects
Performs the Masking rendering and composing operation as specified by \ref{surface.masking}.

\pnum
The meanings of the parameters are specified by \ref{surface.rendering}.

\pnum
\throws
As specified in Error reporting (\ref{\iotwod.err.report}).

\pnum
\errors

The errors, if any, produced by this function are \impldefplain{surface!mask}.
\end{itemdescr}

%!TEX root = io2d.tex
\rSec0 [\iotwod.imagesurface] {Class \tcode{image_surface}}

\rSec1 [\iotwod.imagesurface.summary] {\tcode{image_surface} summary}

\pnum
\indexlibrary{\idxcode{image_surface}}
The class \tcode{image_surface} derives from the \tcode{surface} class and provides an interface to a raster graphics data graphics resource.

\pnum
\begin{note}
Because of the functionality it provides and what it can be used for, it is expected that developers familiar with other graphics technologies will think of the \tcode{image_surface} class as being a form of \term{render target}. This is intentional, though this \documenttypename{} does not formally define or use that term to avoid any minor ambiguities and differences in its meaning between the various graphics technologies that do use the term render target.
\end{note}

\rSec1 [\iotwod.imagesurface.synopsis] {\tcode{image_surface} synopsis}

\begin{codeblock}
namespace std::experimental::io2d::v1 {
  class image_surface : public surface {
  public:
    // \ref{\iotwod.imagesurface.cons}, construct/copy/move/destroy:
    image_surface() = delete;
    image_surface(experimental::io2d::format fmt, int width, int height);
    image_surface(filesystem::path f, image_file_format i, 
      experimental::io2d::format fmt);
    
    // \ref{\iotwod.imagesurface.members}, members:
    void save(filesystem::path p, image_file_format i);
    
    // \ref{\iotwod.imagesurface.observers}, observers:
    experimental::io2d::format format() const noexcept;
    int width() const noexcept;
    int height() const noexcept;
  };
}
\end{codeblock}

\rSec1 [\iotwod.imagesurface.cons] {\tcode{image_surface} constructors and assignment operators}

\indexlibrary{\idxcode{image_surface}!constructor}
\begin{itemdecl}
image_surface(experimental::io2d::format fmt, int width, int height);
\end{itemdecl}
\begin{itemdescr}
\pnum
\requires
\tcode{w >= 1}.

\pnum
\tcode{h >= 1}.

\pnum
\effects
Constructs an object of type \tcode{image_surface}.

\pnum
\postconditions
\tcode{this->format() == fmt}.

\pnum
\tcode{this->width() == width}.

\pnum
\tcode{this->height() == height}.
\end{itemdescr}

\indexlibrary{\idxcode{image_surface}!constructor}
\begin{itemdecl}
image_surface(filesystem::path f, image_file_format i,
  experimental::io2d::format fmt);
\end{itemdecl}
\begin{itemdescr}
\pnum
\requires
\tcode{f} is a file and its contents are data in either JPEG format or PNG format.

\pnum
\effects
Constructs an object of type \tcode{image_surface}.

\pnum
The data of the \underlyingimagesurface is the raster graphics data that results from processing \tcode{f} into uncompressed raster graphics in the manner specified by the standard that specifies how to transform the contents of data contained in \tcode{f} into raster graphics data and then transforming that raster graphics data into the format specified by \tcode{fmt}.

\pnum
The data of \tcode{f} is processed into uncompressed raster graphics data as specified by the value of \tcode{i}.

\pnum
If \tcode{i} is \tcode{image_file_format::unknown}, it is \impldefplain{image_surface!constructor} whether the surface is created in the image file format, if any, that the implementation associates with \tcode{p.extension()} provided that \tcode{p.has_extension() == true}. If \tcode{p.has_extension() == false}, the implementation does not associate an image file format with \tcode{p.extension()}, or the implementation does not support reading in graphics data in that image file format, the error specified below occurs.

\pnum
The resulting uncompressed raster graphics data is then transformed into the data format specified by \tcode{fmt}. If the format specified by \tcode{fmt} only contains an alpha channel, the values of the color channels, if any, of the \underlyingimagesurface are \unspecnorm. If the format specified by \tcode{fmt} only contains color channels and the resulting uncompressed raster graphics data is in a premultiplied format, then the value of each color channel for each pixel shall be divided by the value of the alpha channel for that pixel. The visual data shall then be set as the visual data of the \underlyingimagesurface.

\pnum
\throws
As specified in Error reporting [\iotwod.err.report].

\pnum
\errors
Any error that could result from trying to access \tcode{f}, open \tcode{f} for reading, or reading data from \tcode{f}.

\pnum
\tcode{errc::not_supported} if \tcode{image_file_format::unknown} is passed as an argument and the implementation is unable to determine the file format or does not support saving in the image file format it determined.

\pnum
Other errors, if any, produced by this function are \impldefplain{image_surface!data}.
\end{itemdescr}

\rSec1 [\iotwod.imagesurface.members] {\tcode{image_surface} members}

\indexlibrary{\idxcode{image_surface}!\idxcode{save}}
\begin{itemdecl}
void save(filesystem::path p, image_file_format i);
\end{itemdecl}
\begin{itemdescr}
\pnum
\requires
\tcode{p} shall be a valid path to a file. The file need not exist provided that the other components of the path are valid.

\pnum
If the file exists, it shall be writable. If the file does not exist, it shall be possible to create the file at the specified path and then the created file shall be writable.

\pnum
\effects
Any pending rendering and composing operations (\ref{\iotwod.surface.rendering}) are performed.

\pnum
The visual data of the \underlyingimagesurface is written to \tcode{p} in the data format specified by \tcode{i}.

\pnum
If \tcode{i} is \tcode{image_file_format::unknown}, it is \impldefplain{image_surface!save} whether the surface is saved in the image file format, if any, that the implementation associates with \tcode{p.extension()} provided that \tcode{p.has_extension() == true}. If \tcode{p.has_extension() == false}, the implementation does not associate an image file format with \tcode{p.extension()}, or the implementation does not support saving in that image file format, the error specified below occurs.

\pnum
\throws
As specified in Error reporting [\iotwod.err.report].

\pnum
\errors
Any error that could result from trying to create \tcode{f}, access \tcode{f}, or write data to \tcode{f}.

\pnum
\tcode{errc::not_supported} if \tcode{image_file_format::unknown} is passed as an argument and the implementation is unable to determine the file format or does not support saving in the image file format it determined.

\pnum
Other errors, if any, produced by this function are \impldefplain{image_surface!data}.
\end{itemdescr}

\rSec1 [\iotwod.imagesurface.observers] {\tcode{image_surface} observers}

\indexlibrary{\idxcode{image_surface}!\idxcode{format}}
\indexlibrary{\idxcode{format}!\idxcode{image_surface}}
\begin{itemdecl}
experimental::io2d::format format() const noexcept;
\end{itemdecl}
\begin{itemdescr}
\pnum
\returns
The pixel format of the \tcode{image_surface} object.

\pnum
\remarks
If the \tcode{image_surface} object is invalid, this function shall return \\ \tcode{experimental::io2d::format::invalid}.
\end{itemdescr}

\indexlibrary{\idxcode{image_surface}!\idxcode{width}}
\indexlibrary{\idxcode{width}!\idxcode{image_surface}}
\begin{itemdecl}
int width() const noexcept;
\end{itemdecl}
\begin{itemdescr}
\pnum
\returns
The number of pixels per horizontal line of the \tcode{image_surface} object.

\pnum
\remarks
This function shall return the value \tcode{0} if \\
\tcode{this->format() == experimental::io2d::format::invalid}.
\end{itemdescr}

\indexlibrary{\idxcode{image_surface}!\idxcode{height}}
\indexlibrary{\idxcode{height}!\idxcode{image_surface}}
\begin{itemdecl}
int height() const noexcept;
\end{itemdecl}
\begin{itemdescr}
\pnum
\returns
The number of horizontal lines of pixels in the \tcode{image_surface} object.

\pnum
\remarks
This function shall return the value \tcode{0} if \\
\tcode{this->format() == experimental::io2d::format::invalid}.
\end{itemdescr}

%!TEX root = io2d.tex
\rSec0 [\iotwod.displaysurface] {Class \tcode{display_surface}}

\rSec1 [\iotwod.displaysurface.intro] {\tcode{display_surface} Description}

\pnum
\indexlibrary{\idxcode{display_surface}}%
The class \tcode{display_surface} derives from the \tcode{surface} class and provides an interface to a raster graphics data graphics resource called the \tcode{Back Buffer} and to a second raster graphics data graphics resource called the \tcode{Display Buffer}.

\pnum
The pixel data of the Display Buffer can never be accessed by the user except through a native handle, if one is provided. As such, its pixel format need not equate to any of the pixel formats described by the \tcode{experimental::io2d::format} enumerators. This is meant to give implementors more flexibility in trying to display the pixels of the Back Buffer in a way that is visually as close as possible to the colors of those pixels.

\pnum
The Draw Callback (Table~\ref{tab:\iotwod.displaysurface.state.listing}) is called by \tcode{display_surface::show} as required by the Refresh Rate and when otherwise needed by the implementation in order to update the pixel content of the Back Buffer.

\pnum
After each execution of the Draw Callback, the contents of the Back Buffer are transferred using sampling with an \unspecnorm filter to the Display Buffer. The Display Buffer is then shown to the user via the \term{output device}.
\begin{note}
The filter is \unspecnorm to allow implementations to achieve the best possible result, including by changing filters at runtime depending on factors such as whether scaling is required and by using specialty hardware if available, while maintaining a balance between quality and performance that the implementer deems acceptable.

In the absence of specialty hardware, implementers are encouraged to use a filter that is the equivalent of a nearest neighbor interpolation filter if no scaling is required and otherwise to use a filter that produces results that are at least as good as those that would be obtained by using a bilinear interpolation filter.
\end{note}

\rSec1 [\iotwod.displaysurface.synopsis] {\tcode{display_surface} synopsis}

\begin{codeblock}
namespace std::experimental::io2d::v1 {
  class display_surface : public surface {
  public:
    // \ref{\iotwod.displaysurface.cons}, construct/copy/move/destroy:
    display_surface(display_surface&& other) noexcept;
    display_surface& operator=(display_surface&& other) noexcept;
    
    display_surface(int preferredWidth, int preferredHeight, 
      experimental::io2d::format preferredFormat,
      experimental::io2d::scaling scl = experimental::io2d::scaling::letterbox,
      experimental::io2d::refresh_rate rr =
      experimental::io2d::refresh_rate::as_fast_as_possible, double fps = 30.0);
    display_surface(int preferredWidth, int preferredHeight, 
      experimental::io2d::format preferredFormat, error_code& ec,
      experimental::io2d::scaling scl = experimental::io2d::scaling::letterbox,
      experimental::io2d::refresh_rate rr =
      experimental::io2d::refresh_rate::as_fast_as_possible, double fps = 30.0) 
      noexcept;
    
    display_surface(int preferredWidth, int preferredHeight, 
      experimental::io2d::format preferredFormat,
      int preferredDisplayWidth, int preferredDisplayHeight,
      experimental::io2d::scaling scl = experimental::io2d::scaling::letterbox,
      experimental::io2d::refresh_rate rr =
      experimental::io2d::refresh_rate::as_fast_as_possible, double fps = 30.0);
    display_surface(int preferredWidth, int preferredHeight, 
      experimental::io2d::format preferredFormat,
      int preferredDisplayWidth, int preferredDisplayHeight, error_code& ec,
      experimental::io2d::scaling scl = experimental::io2d::scaling::letterbox,
      experimental::io2d::refresh_rate rr =
      experimental::io2d::refresh_rate::as_fast_as_possible, double fps = 30.0) 
      noexcept;
    
    ~display_surface();
    
    // \ref{\iotwod.displaysurface.modifiers}, modifiers:
    void draw_callback(const function<void(display_surface& sfc)>& fn) noexcept;
    void size_change_callback(const function<void(display_surface& sfc)>& fn)
      noexcept;
    void width(int w);
    void width(int w, error_code& ec) noexcept;
    void height(int h);
    void height(int h, error_code& ec) noexcept;
    void display_width(int w);
    void display_width(int w, error_code& ec) noexcept;
    void display_height(int h);
    void display_height(int h, error_code& ec) noexcept;
    void dimensions(int w, int h);
    void dimensions(int w, int h, error_code& ec) noexcept;
    void display_dimensions(int dw, int dh);
    void display_dimensions(int dw, int dh, error_code& ec) noexcept;
    void scaling(experimental::io2d::scaling scl) noexcept;
    void user_scaling_callback(const function<experimental::io2d::rectangle(
      const display_surface&, bool&)>& fn) noexcept;
    void letterbox_brush(const optional<brush>& b,
      const optional<brush_props> = nullopt) noexcept;
    void auto_clear(bool val) noexcept;
    void refresh_rate(experimental::io2d::refresh_rate rr) noexcept;
    bool desired_frame_rate(double fps) noexcept;
    void redraw_required() noexcept;
    int begin_show();
    void end_show();
    
    // \ref{\iotwod.displaysurface.observers}, observers:
    experimental::io2d::format format() const noexcept;
    int width() const noexcept;
    int height() const noexcept;
    int display_width() const noexcept;
    int display_height() const noexcept;
    vector_2d dimensions() const noexcept;
    vector_2d display_dimensions() const noexcept;
    experimental::io2d::scaling scaling() const noexcept;
    function<experimental::io2d::rectangle(const display_surface&,
      bool&)> user_scaling_callback() const;
    function<experimental::io2d::rectangle(const display_surface&,
      bool&)> user_scaling_callback(error_code& ec) const noexcept;
    optional<brush> letterbox_brush() const noexcept;
    bool auto_clear() const noexcept;
    experimental::io2d::refresh_rate refresh_rate() const noexcept;
    double desired_frame_rate() const noexcept;
    double elapsed_draw_time() const noexcept;
  };
}
\end{codeblock}

\rSec1 [\iotwod.displaysurface.misc] {\tcode{display_surface} miscellaneous behavior}%

\pnum
What constitutes an output device is \impldefplain{output device}, with the sole constraint being that an output device must allow the user to see the dynamically-updated contents of the Display Buffer.
\begin{example}
An output device might be a window in a windowing system environment or the usable screen area of a smart phone or tablet.
\end{example}

\pnum
Implementations do not need to support the simultaneous existence of multiple \tcode{display_surface} objects.

\pnum
All functions inherited from \tcode{surface} that affect its \underlyingsurface shall operate on the Back Buffer.

\rSec1 [\iotwod.displaysurface.state] {\tcode{display_surface} state}

\pnum
Table~\ref{tab:\iotwod.displaysurface.state.listing} specifies the name, type, function, and default value for each item of a display surface's observable state.

\begin{libreqtab4b}
	{Display surface observable state}
	{tab:\iotwod.displaysurface.state.listing}
	\\ \topline
	\lhdr{Name}   &   \chdr{Type}  &   \chdr{Function}  &   \rhdr{Default value}       \\ \capsep
	\endfirsthead
	\continuedcaption\\
	\hline
	\lhdr{Name}   &   \chdr{Type}  &   \chdr{Function}  &   \rhdr{Default value}       \\ \capsep
	\endhead
	
	\term{Letterbox Brush} &
	\tcode{brush} &
	This is the brush that shall be used as specified by \tcode{scaling::letterbox} (Table~\ref{tab:\iotwod.scaling.meanings}) &
	\tcode{brush\{ \{ rgba_color::black() \} \}} \\ \rowsep
	
	\term{Letterbox Brush Props} &
	\tcode{brush_props} &
	This is the brush properties for the Letterbox Brush &
	\tcode{brush_props\{ \}} \\ \rowsep
	
	\term{Scaling Type} &
	\tcode{scaling} &
	When the User Scaling Callback is equal to its default value, this is the type of scaling that shall be used when transferring the Back Buffer to the Display Buffer &
	\tcode{scaling::letterbox}\\ \rowsep
	
	\term{Draw Width} &
	\tcode{int} &
	The width in pixels of the Back Buffer. The minimum value is \tcode{1}. The maximum value is \unspecnorm. Because users can only request a preferred value for the Draw Width when setting and altering it, the maximum value may be a run-time determined value. If the preferred Draw Width exceeds the maximum value, then if a preferred Draw Height has also been supplied then implementations should provide a Back Buffer with the largest dimensions possible that maintain as nearly as possible the aspect ratio between the preferred Draw Width and the preferred Draw Height otherwise implementations should provide a Back Buffer with the largest dimensions possible that maintain as nearly as possible the aspect ratio between the preferred Draw Width and the current Draw Height &
	\textit{N/A}
	\begin{note}
	It is impossible to create a \tcode{display_surface} object without providing a preferred Draw Width value; as such a default value cannot exist.
	\end{note} \\ \rowsep
	
	\term{Draw Height} &
	\tcode{int} &
	The height in pixels of the Back Buffer. The minimum value is \tcode{1}. The maximum value is \unspecnorm. Because users can only request a preferred value for the Draw Height when setting and altering it, the maximum value may be a run-time determined value. If the preferred Draw Height exceeds the maximum value, then if a preferred Draw Width has also been supplied then implementations should provide a Back Buffer with the largest dimensions possible that maintain as nearly as possible the aspect ratio between the preferred Draw Width and the preferred Draw Height otherwise implementations should provide a Back Buffer with the largest dimensions possible that maintain as nearly as possible the aspect ratio between the current Draw Width and the preferred Draw Height &
	\textit{N/A}
	\begin{note}
	It is impossible to create a \tcode{display_surface} object without providing a preferred Draw Height value; as such a default value cannot exist.
	\end{note} \\ \rowsep
	
	\term{Draw Format} &
	\tcode{format} &
	The pixel format of the Back Buffer. When a \tcode{display_surface} object is created, a preferred pixel format value is provided. If the implementation does not support the preferred pixel format value as the value of Draw Format, the resulting value of Draw Format is \impldefplain{display_surface!unsupported draw format} &
	\textit{N/A}
	\begin{note}
	It is impossible to create a \tcode{display_surface} object without providing a preferred Draw Format value; as such a default value cannot exist.
	\end{note} \\ \rowsep
	
	\term{Display Width} &
	\tcode{int} &
	The width in pixels of the Display Buffer. The minimum value is \unspecnorm. The maximum value is \unspecnorm. Because users can only request a preferred value for the Display Width when setting and altering it, both the minimum value and the maximum value may be run-time determined values. If the preferred Display Width is not within the range between the minimum value and the maximum value, inclusive, then if a preferred Display Height has also been supplied then implementations should provide a Display Buffer with the largest dimensions possible that maintain as nearly as possible the aspect ratio between the preferred Display Width and the preferred Display Height otherwise implementations should provide a Display Buffer with the largest dimensions possible that maintain as nearly as possible the aspect ratio between the preferred Display Width and the current Display Height &
	\textit{N/A}
	\begin{note}
	It is impossible to create a \tcode{display_surface} object without providing a preferred Display Width value since in the absence of an explicit Display Width argument the mandatory preferred Draw Width argument is used as the preferred Display Width; as such a default value cannot exist.
	\end{note} \\ \rowsep
	
	\term{Display Height} &
	\tcode{int} &
	The height in pixels of the Display Buffer. The minimum value is \unspecnorm. The maximum value is \unspecnorm. Because users can only request a preferred value for the Display Height when setting and altering it, both the minimum value and the maximum value may be run-time determined values. If the preferred Display Height is not within the range between the minimum value and the maximum value, inclusive, then if a preferred Display Width has also been supplied then implementations should provide a Display Buffer with the largest dimensions possible that maintain as nearly as possible the aspect ratio between the preferred Display Width and the preferred Display Height otherwise implementations should provide a Display Buffer with the largest dimensions possible that maintain as nearly as possible the aspect ratio between the current Display Width and the preferred Display Height &
	\textit{N/A}
	\begin{note}
	It is impossible to create a \tcode{display_surface} object without providing a preferred Display Height value since in the absence of an explicit Display Height argument the mandatory preferred Draw Height argument is used as the preferred Display Height; as such a default value cannot exist.
	\end{note} \\ \rowsep
	
	\term{Draw Callback} &
	\tcode{function<\br
	\tcode{void(}\br
	\tcode{display_surface\&)>}} &
	This function shall be called in a continuous loop when \tcode{display_surface::show} is executing. It is used to draw to the Back Buffer, which in turn results in the display of the drawn content to the user &
	\tcode{nullptr} \\ \rowsep
	
	\term{Size Change Callback} &
	\tcode{function<\br
	\tcode{void(}\br
	\tcode{display_surface\&)>}} &
	If it exists, this function shall be called whenever the Display Buffer has been resized. Neither the Display Width nor the Display Height shall be changed by the Size Change Callback; no diagnostic is required
	\begin{note}
	This means that there has been a change to the Display Width, Display Height, or both. Its intent is to allow the user the opportunity to change other observable state, such as the Draw Width, Draw Height, or Scaling Type, in reaction to the change.
	\end{note} &
	\tcode{nullptr} \\ \rowsep
	
	\term{User Scaling Callback} &
	\tcode{function<\br
	\tcode{rectangle(}\br
	\tcode{const display_surface\&,}\br
	\tcode{bool\&)>}} &
	If it exists, this function shall be called whenever the contents of the Back Buffer need to be copied to the Display Buffer. The function is called with the const reference to \tcode{display_surface} object and a reference to a \tcode{bool} variable which has the value \tcode{false}. If the value of the \tcode{bool} is \tcode{true} when the function returns, the Letterbox Brush shall be used as specified by \tcode{scaling::letterbox} (Table~\ref{tab:\iotwod.scaling.meanings}). The function shall return a \tcode{rectangle} object that defines the area within the Display Buffer to which the Back Buffer shall be transferred. The \tcode{rectangle} may include areas outside of the bounds of the Display Buffer, in which case only the area of the Back Buffer that lies within the bounds of the Display Buffer will ultimately be visible to the user &
	\tcode{nullptr} \\ \rowsep
	
	\term{Auto Clear} &
	\tcode{bool} &
	If \tcode{true} the implementation shall call \tcode{surface::clear}, which shall clear the Back Buffer, immediately before it executes the Draw Callback &
	\tcode{false} \\ \rowsep
	
	\term{Refresh Rate} &
	\tcode{refresh_rate} &
	The \tcode{refresh_rate} value that determines when the Draw Callback shall be called while \tcode{display_surface::show} is being executed &
	\tcode{refresh_rate::as_fast_as_possible} \\ \rowsep
	
	\term{Desired Frame Rate} &
	\tcode{double} &
	This value is the number of times the Draw Callback shall be called per second while \tcode{display_surface::show} is being executed when the value of Refresh Rate is \tcode{refresh_rate::fixed}, subject to the additional requirements documented in the meaning of \tcode{refresh_rate::fixed} (Table~\ref{tab:\iotwod.refreshrate.meanings}) \\ \rowsep
	
\end{libreqtab4b}

\rSec1 [\iotwod.displaysurface.cons] {\tcode{display_surface} constructors and assignment operators}

\indexlibrary{\idxcode{display_surface}!constructor}
\begin{itemdecl}
display_surface(int preferredWidth, int preferredHeight, 
  experimental::io2d::format preferredFormat,
  experimental::io2d::scaling scl = experimental::io2d::scaling::letterbox,
  experimental::io2d::refresh_rate rr =
  experimental::io2d::refresh_rate::as_fast_as_possible, double fps = 30.0);
display_surface(int preferredWidth, int preferredHeight, 
  experimental::io2d::format preferredFormat, error_code& ec,
  experimental::io2d::scaling scl = experimental::io2d::scaling::letterbox,
  experimental::io2d::refresh_rate rr =
  experimental::io2d::refresh_rate::as_fast_as_possible, double fps = 30.0)
  noexcept;
\end{itemdecl}
\begin{itemdescr}
\pnum
\requires
\tcode{preferredWidth > 0}.

\pnum
\tcode{preferredHeight > 0}.

\pnum
\tcode{preferredFormat != experimental::io2d::format::invalid}.


\pnum
\effects
Constructs an object of type \tcode{display_surface}.

\pnum
The preferredWidth parameter specifies the preferred width value for Draw Width and Display Width. The preferredHeight parameter specifies the preferred height value for Draw Height and Display Height. Draw Width and Display Width need not have the same value. Draw Height and Display Height need not have the same value.

\pnum
The preferredFormat parameter specifies the preferred pixel format value for Draw Format.

\pnum
The value of Scaling Type shall be the value of \tcode{scl}.

\pnum
The value of Refresh Rate shall be the value of \tcode{rr}.

\pnum
The value of Desired Frame Rate shall be as-if \tcode{display_surface::desired_frame_rate} was called with \tcode{fps} as its argument. If \tcode{!is_finite(fps)}, then the value of Desired Frame Rate shall be its default value.

\pnum
All other observable state data shall have their default values.

\pnum
\throws
As specified in Error reporting (\ref{\iotwod.err.report}).

\pnum
\errors
\tcode{errc::not_supported} if creating the \tcode{display_surface} object would exceed the maximum number of simultaneous \tcode{display_surface} objects the implementation supports.
\end{itemdescr}

\indexlibrary{\idxcode{display_surface}!constructor}
\begin{itemdecl}
display_surface(int preferredWidth, int preferredHeight, 
  experimental::io2d::format preferredFormat,
  int preferredDisplayWidth, int preferredDisplayHeight,
  experimental::io2d::scaling scl = experimental::io2d::scaling::letterbox,
    experimental::io2d::refresh_rate rr =
    experimental::io2d::refresh_rate::as_fast_as_possible, double fps = 30.0);
display_surface(int preferredWidth, int preferredHeight, 
  experimental::io2d::format preferredFormat,
  int preferredDisplayWidth, int preferredDisplayHeight, error_code& ec,
  experimental::io2d::scaling scl = experimental::io2d::scaling::letterbox,
    experimental::io2d::refresh_rate rr =
    experimental::io2d::refresh_rate::as_fast_as_possible, double fps = 30.0) 
  noexcept;
\end{itemdecl}
\begin{itemdescr}
\pnum
\requires
\tcode{preferredWidth > 0}.

\pnum
\tcode{preferredHeight > 0}.

\pnum
\tcode{preferredDisplayWidth > 0}.

\pnum
\tcode{preferredDisplayHeight > 0}.

\pnum
\tcode{preferredFormat != experimental::io2d::format::invalid}.

\pnum
\effects
Constructs an object of type \tcode{display_surface}.

\pnum
The preferredWidth parameter specifies the preferred width value for Draw Width. The preferredDisplayWidth parameter specifies the preferred display width value for Display Width. The preferredHeight parameter specifies the preferred height value for Draw Height. The preferredDisplayHeight parameter specifies the preferred display height value for Display Height.

\pnum
The preferredFormat parameter specifies the preferred pixel format value for Draw Format.

\pnum
The value of Scaling Type shall be the value of \tcode{scl}.

\pnum
The value of Refresh Rate shall be the value of \tcode{rr}.

\pnum
The value of Desired Frame Rate shall be as-if \tcode{display_surface::desired_frame_rate} was called with \tcode{fps} as its argument. If \tcode{!is_finite(fps)}, then the value of Desired Frame Rate shall be its default value.

\pnum
All other observable state data shall have their default values.

\pnum
\throws
As specified in Error reporting (\ref{\iotwod.err.report}).

\pnum
\errors
\tcode{errc::not_supported} if creating the \tcode{display_surface} object would exceed the maximum number of simultaneous \tcode{display_surface} objects the implementation supports.
\end{itemdescr}

\rSec1 [\iotwod.displaysurface.modifiers]{\tcode{display_surface} modifiers}

\indexlibrary{\idxcode{display_surface}!\idxcode{draw_callback}}
\indexlibrary{\idxcode{draw_callback}!\idxcode{display_surface}}
\begin{itemdecl}
void draw_callback(const function<void(display_surface& sfc)>& fn) noexcept;
\end{itemdecl}
\begin{itemdescr}
\pnum
\effects
Sets the Draw Callback to \tcode{fn}.
\end{itemdescr}

\indexlibrary{\idxcode{display_surface}!\idxcode{size_change_callback}}
\indexlibrary{\idxcode{size_change_callback}!\idxcode{display_surface}}
\begin{itemdecl}
void size_change_callback(const function<void(display_surface& sfc)>& fn)
  noexcept;
\end{itemdecl}
\begin{itemdescr}
\pnum
\effects
Sets the Size Change Callback to \tcode{fn}.
\end{itemdescr}

\indexlibrary{\idxcode{display_surface}!\idxcode{width}}
\indexlibrary{\idxcode{width}!\idxcode{display_surface}}
\begin{itemdecl}
void width(int w);
void width(int w, error_code& ec) noexcept;
\end{itemdecl}
\begin{itemdescr}
\pnum
\effects
If the value of Draw Width is the same as \tcode{w}, this function does nothing.

\pnum
Otherwise, Draw Width is set as specified by Table~\ref{tab:\iotwod.displaysurface.state.listing} with \tcode{w} treated as being the preferred Draw Width.

\pnum
If the value of Draw Width changes as a result, the implementation shall attempt to create a new Back Buffer with the updated dimensions while retaining the existing Back Buffer. The implementation may destroy the existing Back Buffer prior to creating a new Back Buffer with the updated dimensions only if it can guarantee that in doing so it will either succeed in creating the new Back Buffer or will be able to create a Back Buffer with the previous dimensions in the event of failure.

\pnum
\begin{note}
The intent of the previous paragraph is to ensure that, no matter the result, a valid Back Buffer continues to exist. Sometimes implementations will be able to determine that the new dimensions are valid but that to create the new Back Buffer successfully the previous one must be destroyed. The previous paragraph gives implementors that leeway. It goes even further in that it allows implementations to destroy the existing Back Buffer even if they cannot determine in advance that creating the new Back Buffer will succeed, provided that they can guarantee that if the attempt fails they can always successfully recreate a Back Buffer with the previous dimensions. Regardless, there must be a valid Back Buffer when this call completes.
\end{note}

\pnum
The value of the Back Buffer's pixel data shall be \unspecnorm upon completion of this function regardless of whether it succeeded.

\pnum
If an error occurs, the implementation shall ensure that the Back Buffer is valid and has the same dimensions it had prior to this call and that Draw Width shall retain its value prior to this call.

\pnum
\throws
As specified in Error reporting (\ref{\iotwod.err.report}).

\pnum
\errors
\tcode{errc::invalid_argument} if \tcode{w <= 0} or if the value of \tcode{w} is greater than the maximum value for Draw Width.

\tcode{errc::not_enough_memory} if there is insufficient memory to create a Back Buffer with the updated dimensions.

Other errors, if any, produced by this function are \impldefplain{display_surface!width}.
\end{itemdescr}

\indexlibrary{\idxcode{display_surface}!\idxcode{height}}
\indexlibrary{\idxcode{height}!\idxcode{display_surface}}
\begin{itemdecl}
void height(int h);
void height(int h, error_code& ec) noexcept;
\end{itemdecl}
\begin{itemdescr}
\pnum
\effects
If the value of Draw Height is the same as \tcode{h}, this function does nothing.

\pnum
Otherwise, Draw Height is set as specified by Table~\ref{tab:\iotwod.displaysurface.state.listing} with \tcode{h} treated as being the preferred Draw Height.

\pnum
If the value of Draw Height changes as a result, the implementation shall attempt to create a new Back Buffer with the updated dimensions while retaining the existing Back Buffer. The implementation may destroy the existing Back Buffer prior to creating a new Back Buffer with the updated dimensions only if it can guarantee that in doing so it will either succeed in creating the new Back Buffer or will be able to create a Back Buffer with the previous dimensions in the event of failure.

\pnum
\begin{note}
The intent of the previous paragraph is to ensure that, no matter the result, a valid Back Buffer continues to exist. Sometimes implementations will be able to determine that the new dimensions are valid but that to create the new Back Buffer successfully the previous one must be destroyed. The previous paragraph gives implementors that leeway. It goes even further in that it allows implementations to destroy the existing Back Buffer even if they cannot determine in advance that creating the new Back Buffer will succeed, provided that they can guarantee that if the attempt fails they can always successfully recreate a Back Buffer with the previous dimensions. Regardless, there must be a valid Back Buffer when this call completes.
\end{note}

\pnum
The value of the Back Buffer's pixel data shall be \unspecnorm upon completion of this function regardless of whether it succeeded.

\pnum
If an error occurs, the implementation shall ensure that the Back Buffer is valid and has the same dimensions it had prior to this call and that Draw Height shall retain its value prior to this call.

\pnum
\throws
As specified in Error reporting (\ref{\iotwod.err.report}).

\pnum
\errors
\tcode{errc::invalid_argument} if \tcode{h <= 0} or if the value of \tcode{h} is greater than the maximum value for Draw Height.

\tcode{errc::not_enough_memory} if there is insufficient memory to create a Back Buffer with the updated dimensions.

Other errors, if any, produced by this function are \impldefplain{display_surface!height}.
\end{itemdescr}

\indexlibrary{\idxcode{display_surface}!\idxcode{display_width}}
\indexlibrary{\idxcode{display_width}!\idxcode{display_surface}}
\begin{itemdecl}
void display_width(int w);
void display_width(int w, error_code& ec) noexcept;
\end{itemdecl}
\begin{itemdescr}
\pnum
\effects
If the value of Display Width is the same as \tcode{w}, this function does nothing.

\pnum
Otherwise, Display Width is set as specified by Table~\ref{tab:\iotwod.displaysurface.state.listing} with \tcode{w} treated as being the preferred Display Width.

\pnum
If the value of Display Width changes as a result, the implementation shall attempt to create a new Display Buffer with the updated dimensions while retaining the existing Display Buffer. The implementation may destroy the existing Display Buffer prior to creating a new Display Buffer with the updated dimensions only if it can guarantee that in doing so it will either succeed in creating the new Display Buffer or will be able to create a Display Buffer with the previous dimensions in the event of failure.

\pnum
\begin{note}
The intent of the previous paragraph is to ensure that, no matter the result, a valid Display Buffer continues to exist. Sometimes implementations will be able to determine that the new dimensions are valid but that to create the new Display Buffer successfully the previous one must be destroyed. The previous paragraph gives implementors that leeway. It goes even further in that it allows implementations to destroy the existing Display Buffer even if they cannot determine in advance that creating the new Display Buffer will succeed, provided that they can guarantee that if the attempt fails they can always successfully recreate a Display Buffer with the previous dimensions. Regardless, there must be a valid Display Buffer when this call completes.
\end{note}

\pnum
The value of the Display Buffer's pixel data shall be \unspecnorm upon completion of this function regardless of whether it succeeded.

\pnum
If an error occurs, the implementation shall ensure that the Display Buffer is valid and has the same dimensions it had prior to this call and that Display Width shall retain its value prior to this call.

\pnum
\throws
As specified in Error reporting (\ref{\iotwod.err.report}).

\pnum
\errors
\tcode{errc::invalid_argument} if the value of \tcode{w} is less than the minimum value for Display Width or if the value of \tcode{w} is greater than the maximum value for Display Width.

\tcode{errc::not_enough_memory} if there is insufficient memory to create a Display Buffer with the updated dimensions.

Other errors, if any, produced by this function are \impldefplain{display_surface!display_width}.
\end{itemdescr}

\indexlibrary{\idxcode{display_surface}!\idxcode{display_height}}
\indexlibrary{\idxcode{display_height}!\idxcode{display_surface}}
\begin{itemdecl}
void display_height(int h);
void display_height(int h, error_code& ec) noexcept;
\end{itemdecl}
\begin{itemdescr}
\pnum
\effects
If the value of Display Height is the same as \tcode{h}, this function does nothing.

\pnum
Otherwise, Display Height is set as specified by Table~\ref{tab:\iotwod.displaysurface.state.listing} with \tcode{h} treated as being the preferred Display Height.

\pnum
If the value of Display Height changes as a result, the implementation shall attempt to create a new Display Buffer with the updated dimensions while retaining the existing Display Buffer. The implementation may destroy the existing Display Buffer prior to creating a new Display Buffer with the updated dimensions only if it can guarantee that in doing so it will either succeed in creating the new Display Buffer or will be able to create a Display Buffer with the previous dimensions in the event of failure.

\pnum
\begin{note}
The intent of the previous paragraph is to ensure that, no matter the result, a valid Display Buffer continues to exist. Sometimes implementations will be able to determine that the new dimensions are valid but that to create the new Display Buffer successfully the previous one must be destroyed. The previous paragraph gives implementors that leeway. It goes even further in that it allows implementations to destroy the existing Display Buffer even if they cannot determine in advance that creating the new Display Buffer will succeed, provided that they can guarantee that if the attempt fails they can always successfully recreate a Display Buffer with the previous dimensions. Regardless, there must be a valid Display Buffer when this call completes.
\end{note}

\pnum
The value of the Display Buffer's pixel data shall be \unspecnorm upon completion of this function regardless of whether it succeeded.

\pnum
If an error occurs, the implementation shall ensure that the Display Buffer is valid and has the same dimensions it had prior to this call and that Display Height shall retain its value prior to this call.

\pnum
\throws
As specified in Error reporting (\ref{\iotwod.err.report}).

\pnum
\errors
\tcode{errc::invalid_argument} if the value of \tcode{h} is less than the minimum value for Display Height or if the value of \tcode{h} is greater than the maximum value for Display Height.

\tcode{errc::not_enough_memory} if there is insufficient memory to create a Display Buffer with the updated dimensions.

Other errors, if any, produced by this function are \impldefplain{display_surface!display_height}.
\end{itemdescr}

\indexlibrary{\idxcode{display_surface}!\idxcode{dimensions}}
\indexlibrary{\idxcode{dimensions}!\idxcode{display_surface}}
\begin{itemdecl}
void dimensions(int w, int h);
void dimensions(int w, int h, error_code& ec) noexcept;
\end{itemdecl}
\begin{itemdescr}
\pnum
\effects
If the value of Draw Width is the same as \tcode{w} and the value of Draw Height is the same as \tcode{h}, this function does nothing.

\pnum
Otherwise, Draw Width is set as specified by Table~\ref{tab:\iotwod.displaysurface.state.listing} with \tcode{w} treated as being the preferred Draw Width and Draw Height is set as specified by Table~\ref{tab:\iotwod.displaysurface.state.listing} with \tcode{h} treated as being the preferred Draw Height.

\pnum
If the value of Draw Width changes as a result or the value of Draw Height changes as a result, the implementation shall attempt to create a new Back Buffer with the updated dimensions while retaining the existing Back Buffer. The implementation may destroy the existing Back Buffer prior to creating a new Back Buffer with the updated dimensions only if it can guarantee that in doing so it will either succeed in creating the new Back Buffer or will be able to create a Back Buffer with the previous dimensions in the event of failure.

\pnum
\begin{note}
The intent of the previous paragraph is to ensure that, no matter the result, a valid Back Buffer continues to exist. Sometimes implementations will be able to determine that the new dimensions are valid but that to create the new Back Buffer successfully the previous one must be destroyed. The previous paragraph gives implementors that leeway. It goes even further in that it allows implementations to destroy the existing Back Buffer even if they cannot determine in advance that creating the new Back Buffer will succeed, provided that they can guarantee that if the attempt fails they can always successfully recreate a Back Buffer with the previous dimensions. Regardless, there must be a valid Back Buffer when this call completes.
\end{note}

\pnum
The value of the Back Buffer's pixel data shall be \unspecnorm upon completion of this function regardless of whether it succeeded.

\pnum
If an error occurs, the implementation shall ensure that the Back Buffer is valid and has the same dimensions it had prior to this call and that Draw Width and Draw Height shall retain the values they had prior to this call.

\pnum
\throws
As specified in Error reporting (\ref{\iotwod.err.report}).

\pnum
\errors
\tcode{errc::invalid_argument} if \tcode{w <= 0}, if the value of \tcode{w} is greater than the maximum value for Draw Width, if \tcode{h <= 0} or if the value of \tcode{h} is greater than the maximum value for Draw Height.

\tcode{errc::not_enough_memory} if there is insufficient memory to create a Back Buffer with the updated dimensions.

Other errors, if any, produced by this function are \impldefplain{display_surface!dimensions}.
\end{itemdescr}

\indexlibrary{\idxcode{display_surface}!\idxcode{display_dimensions}}
\indexlibrary{\idxcode{display_dimensions}!\idxcode{display_surface}}
\begin{itemdecl}
void display_dimensions(int dw, int dh);
void display_dimensions(int dw, int dh, error_code& ec) noexcept;
\end{itemdecl}
\begin{itemdescr}
\pnum
\effects
If the value of Display Width is the same as \tcode{w} and the value of Display Height is the same as \tcode{h}, this function does nothing.

\pnum
Otherwise, Display Width is set as specified by Table~\ref{tab:\iotwod.displaysurface.state.listing} with \tcode{w} treated as being the preferred Display Height and Display Height is set as specified by Table~\ref{tab:\iotwod.displaysurface.state.listing} with \tcode{h} treated as being the preferred Display Height.

\pnum
If the value of Display Width or the value of Display Height changes as a result, the implementation shall attempt to create a new Display Buffer with the updated dimensions while retaining the existing Display Buffer. The implementation may destroy the existing Display Buffer prior to creating a new Display Buffer with the updated dimensions only if it can guarantee that in doing so it will either succeed in creating the new Display Buffer or will be able to create a Display Buffer with the previous dimensions in the event of failure.

\pnum
\begin{note}
The intent of the previous paragraph is to ensure that, no matter the result, a valid Display Buffer continues to exist. Sometimes implementations will be able to determine that the new dimensions are valid but that to create the new Display Buffer successfully the previous one must be destroyed. The previous paragraph gives implementors that leeway. It goes even further in that it allows implementations to destroy the existing Display Buffer even if they cannot determine in advance that creating the new Display Buffer will succeed, provided that they can guarantee that if the attempt fails they can always successfully recreate a Display Buffer with the previous dimensions. Regardless, there must be a valid Display Buffer when this call completes.
\end{note}

\pnum
If an error occurs, the implementation shall ensure that the Display Buffer is valid and has the same dimensions it had prior to this call and that Display Width and Display Height shall retain the values they had prior to this call.

\pnum
If the Display Buffer has changed, even if its width and height have not changed, the Draw Callback shall be called.

\pnum
If the width or height of the Display Buffer has changed, the Size Change Callback shall be called if it's value is not its default value.

\pnum
\throws
As specified in Error reporting (\ref{\iotwod.err.report}).

\pnum
\errors
\tcode{errc::invalid_argument} if the value of \tcode{w} is less than the minimum value for Display Width, if the value of \tcode{w} is greater than the maximum value for Display Width, if the value of \tcode{h} is less than the minimum value for Display Height, or if the value of \tcode{h} is greater than the maximum value for Display Height.

\tcode{errc::not_enough_memory} if there is insufficient memory to create a Display Buffer with the updated dimensions.

Other errors, if any, produced by this function are \impldefplain{display_surface!display_dimensions}.
\end{itemdescr}

\indexlibrary{\idxcode{display_surface}!\idxcode{scaling}}
\indexlibrary{\idxcode{scaling}!\idxcode{display_surface}}
\begin{itemdecl}
void scaling(experimental::io2d::scaling scl) noexcept;
\end{itemdecl}
\begin{itemdescr}
\pnum
\effects
Sets Scaling Type to the value of \tcode{scl}.
\end{itemdescr}

\indexlibrary{\idxcode{display_surface}!\idxcode{user_scaling_callback}}
\indexlibrary{\idxcode{user_scaling_callback}!\idxcode{display_surface}}
\begin{itemdecl}
void user_scaling_callback(const function<experimental::io2d::rectangle(
  const display_surface&, bool&)>& fn) noexcept;
\end{itemdecl}
\begin{itemdescr}
\pnum
\effects
Sets the User Scaling Callback to \tcode{fn}.
\end{itemdescr}

\indexlibrary{\idxcode{display_surface}!\idxcode{letterbox_brush}}
\indexlibrary{\idxcode{letterbox_brush}!\idxcode{display_surface}}
\begin{itemdecl}
void letterbox_brush(const optional<brush&>b,
  const optional<brush_props>& bp = nullopt);
void letterbox_brush(const optional<brush&>b, error_code& ec,
  const optional<brush_props>& bp = nullopt) noexcept;
\end{itemdecl}
\begin{itemdescr}
\pnum
\effects
Sets the Letterbox Brush to the value contained in \tcode{b} if it contains a value, otherwise set Letterbox Brush to its default value.

\pnum
Sets the Letterbox Brush Props to the value contained in \tcode{bp} if it contains a value, otherwise sets it Letterbox Brush Props to its default value.

\pnum
\throws
As specified in Error reporting (\ref{\iotwod.err.report}).

\pnum
\errors
The errors, if any, produced by this function are \impldefplain{display_surface!letterbox_brush}.
\end{itemdescr}

\indexlibrary{\idxcode{display_surface}!\idxcode{auto_clear}}
\indexlibrary{\idxcode{auto_clear}!\idxcode{display_surface}}
\begin{itemdecl}
void auto_clear(bool val) noexcept;
\end{itemdecl}
\begin{itemdescr}
\pnum
\effects
Sets Auto Clear to the value of \tcode{val}.
\end{itemdescr}

\indexlibrary{\idxcode{display_surface}!\idxcode{refresh_rate}}
\indexlibrary{\idxcode{refresh_rate}!\idxcode{display_surface}}
\begin{itemdecl}
void refresh_rate(experimental::io2d::refresh_rate rr) noexcept;
\end{itemdecl}
\begin{itemdescr}
\pnum
\effects
Sets the Refresh Rate to the value of \tcode{rr}.
\end{itemdescr}

\indexlibrary{\idxcode{display_surface}!\idxcode{desired_frame_rate}}
\indexlibrary{\idxcode{desired_frame_rate}!\idxcode{display_surface}}
\begin{itemdecl}
bool desired_frame_rate(double fps) noexcept;
\end{itemdecl}
\begin{itemdescr}
\pnum
\effects
If \tcode{!is_finite(fps)}, this function has no effects.

\pnum
Sets the Desired Frame Rate to an \impldefplain{display_surface!minimum frame rate} minimum frame rate if \tcode{fps} is less than the minimum frame rate, an \impldefplain{display_surface!maximum frame rate} maximum frame rate if \tcode{fps} is greater than the maximum frame rate, otherwise to the value of \tcode{fps}.

\pnum
\returns
\tcode{false} if the Desired Frame Rate was set to the value of \tcode{fps}; otherwise \tcode{true}.
\end{itemdescr}

\indexlibrary{\idxcode{display_surface}!\idxcode{redraw_required}}
\indexlibrary{\idxcode{redraw_required}!\idxcode{display_surface}}
\begin{itemdecl}
void redraw_required() noexcept;
\end{itemdecl}
\begin{itemdescr}
\pnum
\effects
When \tcode{display_surface::begin_show} is executing, informs the implementation that the Draw Callback must be called as soon as possible.
\end{itemdescr}

\indexlibrary{\idxcode{display_surface}!\idxcode{begin_show}}
\begin{itemdecl}
int begin_show();
\end{itemdecl}
\begin{itemdescr}
\pnum
\effects
Performs the following actions in a continuous loop:
\begin{enumerate}
	\item Handle any implementation and host environment matters. If there are no pending implementation or host environment matters to handle, proceed immediately to the next action.
	\item Run the Size Change Callback if doing so is required by its specification and it does not have a value equivalent to its default value.
	\item If the Refresh Rate requires that the Draw Callback be called then:
	\begin{enumeratea}
		\item Evaluate Auto Clear and perform the actions required by its specification, if any.
		\item Run the Draw Callback.
		\item Ensure that all operations from the Draw Callback that can effect the Back Buffer have completed.
		\item Transfer the contents of the Back Buffer to the Display Buffer using sampling with an \unspecnorm filter. If the User Scaling Callback does not have a value equivalent to its default value, use it to determine the position where the contents of the Back Buffer shall be transferred to and whether or not the Letterbox Brush should be used. Otherwise use the value of Scaling Type to determine the position and whether the Letterbox Brush should be used.
	\end{enumeratea}
\end{enumerate}

\pnum
If \tcode{display_surface::end_show} is called from the Draw Callback, the implementation shall finish executing the Draw Callback and shall immediately cease to perform any actions in the continuous loop other than handling any implementation and host environment matters needed to exit the loop properly.

\pnum
No later than when this function returns, the output device shall cease to display the contents of the Display Buffer.

\pnum
What the output device shall display when it is not displaying the contents of the Display Buffer is \unspecnorm.

\pnum
\returns
The possible values and meanings of the possible values returned are \impldefplain{display_surface!show return value}.

\pnum
\throws
As specified in Error reporting (\ref{\iotwod.err.report}).

\pnum
\remarks
Since this function calls the Draw Callback and can call the Size Change Callback and the User Scaling Callback, in addition to the errors documented below, any errors that the callback functions produce can also occur.

\pnum
\errors
\pnum
\tcode{errc::operation_would_block} if the value of Draw Callback is equivalent to its default value or if it becomes equivalent to its default value before this function returns.

\pnum
Other errors, if any, produced by this function are \impldefplain{display_surface!show}.
\end{itemdescr}

\indexlibrary{\idxcode{display_surface}!\idxcode{end_show}}
\begin{itemdecl}
void end_show();
\end{itemdecl}
\begin{itemdescr}
\pnum
\effects
If this function is called outside of the Draw Callback while it is being executed in the \tcode{display_surface::begin_show} function's continuous loop, it does nothing.

\pnum
Otherwise, the implementation initiates the process of exiting the \tcode{display_surface::begin_show} function's continuous loop.

\pnum
If possible, any procedures that the host environment requires in order to cause the \tcode{display_surface::show} function's continuous loop to stop executing without error should be followed.

\pnum
The \tcode{display_surface::begin_show} function's loop continues execution until it returns.
\end{itemdescr}

\rSec1 [\iotwod.displaysurface.observers]{\tcode{display_surface} observers}

\indexlibrary{\idxcode{display_surface}!\idxcode{format}}
\indexlibrary{\idxcode{format}!\idxcode{display_surface}}
\begin{itemdecl}
experimental::io2d::format format() const noexcept;
\end{itemdecl}
\begin{itemdescr}
\pnum
\returns
The value of Draw Format.
\end{itemdescr}

\indexlibrary{\idxcode{display_surface}!\idxcode{width}}
\indexlibrary{\idxcode{width}!\idxcode{display_surface}}
\begin{itemdecl}
int width() const noexcept;
\end{itemdecl}
\begin{itemdescr}
\pnum
\returns
The Draw Width.
\end{itemdescr}

\indexlibrary{\idxcode{display_surface}!\idxcode{height}}
\indexlibrary{\idxcode{height}!\idxcode{display_surface}}
\begin{itemdecl}
int height() const noexcept;
\end{itemdecl}
\begin{itemdescr}
\pnum
\returns
The Draw Height.
\end{itemdescr}

\indexlibrary{\idxcode{display_surface}!\idxcode{display_width}}
\indexlibrary{\idxcode{display_width}!\idxcode{display_surface}}
\begin{itemdecl}
int display_width() const noexcept;
\end{itemdecl}
\begin{itemdescr}
\pnum
\returns
The Display Width.
\end{itemdescr}

\indexlibrary{\idxcode{display_surface}!\idxcode{display_height}}
\indexlibrary{\idxcode{display_height}!\idxcode{display_surface}}
\begin{itemdecl}
int display_height() const noexcept;
\end{itemdecl}
\begin{itemdescr}
\pnum
\returns
The Display Height.
\end{itemdescr}

\indexlibrary{\idxcode{display_surface}!\idxcode{dimensions}}
\indexlibrary{\idxcode{dimensions}!\idxcode{display_surface}}
\begin{itemdecl}
vector_2d dimensions() const noexcept;
\end{itemdecl}
\begin{itemdescr}
\pnum
\returns
A \tcode{vector_2d} constructed using the Draw Width as the first argument and the Draw Height as the second argument.
\end{itemdescr}

\indexlibrary{\idxcode{display_surface}!\idxcode{display_dimensions}}
\indexlibrary{\idxcode{display_dimensions}!\idxcode{display_surface}}
\begin{itemdecl}
vector_2d display_dimensions() const noexcept;
\end{itemdecl}
\begin{itemdescr}
\pnum
\returns
A \tcode{vector_2d} constructed using the Display Width as the first argument and the Display Height as the second argument.
\end{itemdescr}

\indexlibrary{\idxcode{display_surface}!\idxcode{scaling}}
\indexlibrary{\idxcode{scaling}!\idxcode{display_surface}}
\begin{itemdecl}
experimental::io2d::scaling scaling() const noexcept;
\end{itemdecl}
\begin{itemdescr}
\pnum
\returns
The Scaling Type.
\end{itemdescr}

\indexlibrary{\idxcode{display_surface}!\idxcode{user_scaling_callback}}
\indexlibrary{\idxcode{user_scaling_callback}!\idxcode{display_surface}}
\begin{itemdecl}
function<experimental::io2d::rectangle(const display_surface&, bool&)>
  user_scaling_callback() const;
function<experimental::io2d::rectangle(const display_surface&, bool&)>
  user_scaling_callback(error_code& ec) const noexcept;
\end{itemdecl}
\begin{itemdescr}
\pnum
\returns
A copy of User Scaling Callback.

\pnum
\throws
As specified in Error reporting (\ref{\iotwod.err.report}).

\pnum
\errors
\tcode{errc::not_enough_memory} if a failure to allocate memory occurs.
\end{itemdescr}

\indexlibrary{\idxcode{display_surface}!\idxcode{letterbox_brush}}
\indexlibrary{\idxcode{letterbox_brush}!\idxcode{display_surface}}
\begin{itemdecl}
optional<brush> letterbox_brush() const noexcept;
\end{itemdecl}
\begin{itemdescr}
\pnum
\returns
A \tcode{optional<brush>} object constructed using the user-provided Letterbox Brush or, if no user-provided Letterbox Brush is set, an empty \tcode{optional<brush>} object.
\end{itemdescr}

\indexlibrary{\idxcode{display_surface}!\idxcode{auto_clear}}
\indexlibrary{\idxcode{auto_clear}!\idxcode{display_surface}}
\begin{itemdecl}
bool auto_clear() const noexcept;
\end{itemdecl}
\begin{itemdescr}
\pnum
\returns
The value of Auto Clear.
\end{itemdescr}

\indexlibrary{\idxcode{display_surface}!\idxcode{display_width}}
\indexlibrary{\idxcode{display_width}!\idxcode{display_surface}}
\begin{itemdecl}
double desired_framerate() const noexcept;
\end{itemdecl}
\begin{itemdescr}
\pnum
\returns
The value of Desired Framerate.
\end{itemdescr}

\indexlibrary{\idxcode{display_surface}!\idxcode{elapsed_draw_time}}
\indexlibrary{\idxcode{elapsed_draw_time}!\idxcode{display_surface}}
\begin{itemdecl}
double elapsed_draw_time() const noexcept;
\end{itemdecl}
\begin{itemdescr}
\pnum
\returns
If called from the Draw Callback during the execution of \tcode{display_surface::show}, the amount of time in milliseconds that has passed since the previous call to the Draw Callback by the current execution of \tcode{display_surface::show}; otherwise \tcode{0.0}.
\end{itemdescr}

%!TEX root = io2d.tex
\rSec0 [mappedsurface] {Class \tcode{mapped_surface}}

\rSec1 [mappedsurface.synopsis] {\tcode{mapped_surface} synopsis}

\begin{codeblock}
namespace std { namespace experimental { namespace io2d { inline namespace v1 {
  class mapped_surface {
  public:
    // \ref{mappedsurface.cons}, construct/copy/move/destroy:
    mapped_surface() = delete;
    mapped_surface(const mapped_surface&) = delete;
    mapped_surface& operator=(const mapped_surface&) = delete;
    mapped_surface(mapped_surface&& other) = delete;
    mapped_surface& operator=(mapped_surface&& other) = delete;
    ~mapped_surface();
    
    // \ref{mappedsurface.modifiers}, modifiers:
    void commit_changes();
    void commit_changes(error_code& ec) noexcept;
    void commit_changes(const rectangle& area);
    void commit_changes(const rectangle& area, error_code& ec) noexcept;
    unsigned char* data();
    unsigned char* data(error_code& ec) noexcept;
    
    // \ref{mappedsurface.observers}, observers:
    const unsigned char* data() const;
    const unsigned char* data(error_code& ec) const noexcept;
    experimental::io2d::format format() const noexcept;
    int width() const noexcept;
    int height() const noexcept;
    int stride() const noexcept;
  };
} } } }
\end{codeblock}

\rSec1 [mappedsurface.intro] {\tcode{mapped_surface} Description}

\pnum
\indexlibrary{\idxcode{mapped_surface}}
The \tcode{mapped_surface} class provides access to inspect and modify the pixel data of a \tcode{surface} object's \underlyingsurface or a subsection thereof.

\pnum
A \tcode{mapped_surface} can only be created by the \tcode{surface::map} function. It cannot be copied or moved.

\pnum
The pixel data is presented as an array in the form of a pointer to (possibly \tcode{const}) \tcode{unsigned char}.

\pnum
The actual format of the pixel data depends on the \tcode{format} enumerator returned by calling \tcode{mapped_surface::format} and is native-endian. For more information, see the description of the \tcode{format} enum class (\ref{format}).

\pnum
The pixel data array is presented as a series of horizontal rows of pixels with row \tcode{0} being the top row of pixels of the \underlyingsurface and the bottom row being the row at \tcode{mapped_surface::height() - 1}.

\pnum
Each horizontal row of pixels begins with the leftmost pixel and proceeds right to \tcode{mapped_surface::width() - 1}.

\pnum
The width in bytes of each horizontal row is provided by \tcode{mapped_surface::stride}. This value may be larger than the result of multiplying the width in pixels of each horizontal row by the size in bytes of the pixel's format (most commonly as a result of implementation-dependent memory alignment requirements).

\pnum
Whether the pixel data array provides direct access to the \underlyingsurface's memory or provides indirect access as if through a proxy or a copy is \unspecnorm.

\pnum
Changes made to the pixel data array are considered to be \term{uncommitted} so long as those changes are not reflected in the \underlyingsurface.

\pnum
Changes made to the pixel data array are considered to be \term{committed} once they are reflected in the \underlyingsurface.

\rSec1 [mappedsurface.cons] {\tcode{mapped_surface} constructors and assignment operators}

\indexlibrary{\idxcode{mapped_surface}!destructor}
\begin{itemdecl}
~mapped_surface();
\end{itemdecl}
\begin{itemdescr}
\pnum
\effects
Destroys an object of type \tcode{mapped_surface}. 

\pnum
\remarks
Whether any uncommitted changes are committed during destruction of the \tcode{mapped_surface} object is \unspecnorm.

\pnum
Uncommitted changes shall not be committed during destruction of the \tcode{mapped_surface} object if doing so would result in an exception.

\pnum
\realnotes
It is recommended that users use the \tcode{mapped_surface::commit_changes} function to commit changes prior to the destruction of the \tcode{mapped_surface} object to ensure consistent behavior.
\end{itemdescr}

\rSec1 [mappedsurface.modifiers]{\tcode{mapped_surface} modifiers}

\indexlibrary{\idxcode{mapped_surface}!\idxcode{commit_changes}}
\indexlibrary{\idxcode{commit_changes}!\idxcode{mapped_surface}}
\begin{itemdecl}
void commit_changes();
void commit_changes(error_code& ec) noexcept;
\end{itemdecl}
\begin{itemdescr}
\pnum
\effects
Any uncommitted changes shall be committed.

\pnum
\throws
As specified in Error reporting (\ref{\iotwod.err.report}).

\pnum
\errors
The errors, if any, produced by this function are \impldef{mapped_surface!commit_changes}.
\end{itemdescr}

\indexlibrary{\idxcode{mapped_surface}!\idxcode{data}}
\indexlibrary{\idxcode{data}!\idxcode{mapped_surface}}
\begin{itemdecl}
unsigned char* data();
unsigned char* data(error_code& ec) noexcept;
\end{itemdecl}
\begin{itemdescr}
\pnum
\returns
A native-endian pointer to the pixel data array.
\enterexample
Given the following code:

\begin{codeblock}
image_surface imgsfc{ format::argb32, 100, 100 };
imgsfc.paint(rgba_color::red());
imgsfc.flush();
imgsfc.map([](mapped_surface& mapsfc) -> void {
    auto pixelData = mapsfc.data();
    auto p0 = static_cast<uint32_t>(pixelData[0]);
    auto p1 = static_cast<uint32_t>(pixelData[1]);
    auto p2 = static_cast<uint32_t>(pixelData[2]);
    auto p3 = static_cast<uint32_t>(pixelData[3]);
    printf("%X %X %X %X\n", p0, p1, p2, p3);
});
\end{codeblock}

In a little-endian environment, \tcode{p0 == 0x0}, \tcode{p1 == 0x0}, \tcode{p2 == 0xFF}, and \tcode{p3 == 0xFF}.

In a big-endian environment, \tcode{p0 == 0xFF}, \tcode{p1 == 0xFF}, \tcode{p2 == 0x0}, \tcode{p3 == 0x0}.
\exitexample

\pnum
\throws
As specified in Error reporting (\ref{\iotwod.err.report}).

\pnum
\remarks
The bounds of the pixel data array range from \tcode{a}, where \tcode{a} is the address returned by this function, to \tcode{a + this->stride() * this->height()}. Given a height \tcode{h} where \tcode{h} is any value from \tcode{0} to \tcode{this->height() - 1}, any attempt to read or write a byte with an address that is not within the range of addresses defined by \tcode{a + this->stride() * h} shall result in undefined behavior; no diagnostic is required.

\pnum
\errors
\tcode{io2d_error::null_pointer} if \tcode{this->format() == experimental::io2d::format::unknown || this->format() == experimental::io2d::format::invalid}.
\end{itemdescr}

\rSec1 [mappedsurface.observers]{\tcode{mapped_surface} observers}

\indexlibrary{\idxcode{mapped_surface}!\idxcode{data}}
\indexlibrary{\idxcode{data}!\idxcode{mapped_surface}}
\begin{itemdecl}
const unsigned char* data() const;
const unsigned char* data(error_code& ec) const noexcept;
\end{itemdecl}
\begin{itemdescr}
\pnum
\returns
A const native-endian pointer to the pixel data array.
\enterexample
Given the following code:

\begin{codeblock}
image_surface imgsfc{ format::argb32, 100, 100 };
imgsfc.paint(rgba_color::red());
imgsfc.flush();
imgsfc.map([](mapped_surface& mapsfc) -> void {
    auto pixelData = mapsfc.data();
    auto p0 = static_cast<uint32_t>(pixelData[0]);
    auto p1 = static_cast<uint32_t>(pixelData[1]);
    auto p2 = static_cast<uint32_t>(pixelData[2]);
    auto p3 = static_cast<uint32_t>(pixelData[3]);
    printf("%X %X %X %X\n", p0, p1, p2, p3);
});
\end{codeblock}

In a little-endian environment, \tcode{p0 == 0x0}, \tcode{p1 == 0x0}, \tcode{p2 == 0xFF}, and \tcode{p3 == 0xFF}.

In a big-endian environment, \tcode{p0 == 0xFF}, \tcode{p1 == 0xFF}, \tcode{p2 == 0x0}, \tcode{p3 == 0x0}.
\exitexample

\pnum
\throws
As specified in Error reporting (\ref{\iotwod.err.report}).

\pnum
\remarks
The bounds of the pixel data array range from \tcode{a}, where \tcode{a} is the address returned by this function, to \tcode{a + this->stride() * this->height()}. Given a height \tcode{h} where \tcode{h} is any value from \tcode{0} to \tcode{this->height() - 1}, any attempt to read a byte with an address that is not within the range of addresses defined by \tcode{a + this->stride() * h} shall result in undefined behavior; no diagnostic is required.

\pnum
\errors
\tcode{io2d_error::null_pointer} if \tcode{this->format() == experimental::io2d::format::unknown || this->format() == experimental::io2d::format::invalid}.
\end{itemdescr}

\indexlibrary{\idxcode{mapped_surface}!\idxcode{format}}
\indexlibrary{\idxcode{format}!\idxcode{mapped_surface}}
\begin{itemdecl}
experimental::io2d::format format() const noexcept;
\end{itemdecl}
\begin{itemdescr}
\pnum
\returns
The pixel format of the mapped surface.

\pnum
\remarks
If the mapped surface is invalid, this function shall return \tcode{experimental::io2d::format::invalid}.
\end{itemdescr}

\indexlibrary{\idxcode{mapped_surface}!\idxcode{width}}
\indexlibrary{\idxcode{width}!\idxcode{mapped_surface}}
\begin{itemdecl}
int width() const noexcept;
\end{itemdecl}
\begin{itemdescr}
\pnum
\returns
The number of pixels per horizontal line of the mapped surface.

\pnum
\remarks
This function shall return the value \tcode{0} if \tcode{this->format() == experimental::io2d::format::unknown || this->format() == experimental::io2d::format::invalid}.
\end{itemdescr}

\indexlibrary{\idxcode{mapped_surface}!\idxcode{height}}
\indexlibrary{\idxcode{height}!\idxcode{mapped_surface}}
\begin{itemdecl}
int height() const noexcept;
\end{itemdecl}
\begin{itemdescr}
\pnum
\returns
The number of horizontal lines of pixels in the mapped surface.

\pnum
\remarks
This function shall return the value \tcode{0} if \tcode{this->format() == experimental::io2d::format::unknown || this->format() == experimental::io2d::format::invalid}.
\end{itemdescr}

\indexlibrary{\idxcode{mapped_surface}!\idxcode{stride}}
\indexlibrary{\idxcode{stride}!\idxcode{mapped_surface}}
\begin{itemdecl}
int stride() const noexcept;
\end{itemdecl}
\begin{itemdescr}
\pnum
\returns
The length, in bytes, of a horizontal line of the mapped surface.
\enternote
This value is at least as large as the width in pixels of a horizontal line multiplied by the number of bytes per pixel but may be larger as a result of padding.
\exitnote

\pnum
\remarks
This function shall return the value \tcode{0} if \tcode{this->format() == experimental::io2d::format::unknown || this->format() == experimental::io2d::format::invalid}.
\end{itemdescr}

\addtocounter{SectionDepthBase}{-1}
