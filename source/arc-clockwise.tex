%!TEX root = io2d.tex
\rSec0 [arcclockwise] {Class \tcode{arc_clockwise}}

\pnum
\indexlibrary{\idxcode{arc_clockwise}}
The class \tcode{arc_clockwise} describes a path segment that is a circular arc with clockwise rotation.

\pnum
It has a Circle of type \tcode{circle}, a First Angle of type \tcode{double}, and a Second Angle of type \tcode{double}.

\pnum
The values for the First Angle and Second Angle are in radians.

\pnum
\begin{note}
Although the value of the First Angle may be greater than the value of the Second Angle, when processed as described in \ref{paths.processing}, \tcode{two_pi<double>} is added to the Second Angle until the value of the Second Angle is greater than or equal to the value of the First Angle.
\end{note}

\rSec1 [arcclockwise.synopsis] {\tcode{arc_clockwise} synopsis}

\begin{codeblock}
namespace std { namespace experimental { namespace io2d { inline namespace v1 {
  namespace path_data {
    class arc_clockwise {
    public:
      // \ref{arcclockwise.cons}, construct/copy/move/destroy:
      constexpr arc_clockwise() noexcept;
      constexpr arc_clockwise(const experimental::io2d::circle& c,
        double angle1, double angle2) noexcept;
      constexpr arc_clockwise(const vector_2d& ctr, double rad,
        double angle1, double angle2) noexcept;

      // \ref{arcclockwise.modifiers}, modifiers:
      constexpr void circle(const experimental::io2d::circle& c) noexcept;
      constexpr void center(const vector_2d& ctr) noexcept;
      constexpr void radius(double r) noexcept;
      constexpr void angle_1(double radians) noexcept;
      constexpr void angle_2(double radians) noexcept;

      // \ref{arcclockwise.observers}, observers:
      constexpr experimental::io2d::circle circle() const noexcept;
      constexpr vector_2d center() const noexcept;
      constexpr double radius() const noexcept;
      constexpr double angle_1() const noexcept;
      constexpr double angle_2() const noexcept;
    };
  };
} } } }
\end{codeblock}

\rSec1 [arcclockwise.cons] {\tcode{arc_clockwise} constructors and assignment operators}

\indexlibrary{\idxcode{arc_clockwise}!constructor}
\begin{itemdecl}
constexpr arc_clockwise() noexcept;
\end{itemdecl}
\begin{itemdescr}
\pnum
\effects
Constructs an object of type \tcode{arc_clockwise}.

\pnum
The Circle shall be set to the value of \tcode{experimental::io2d::circle\{ \}}.

\pnum
The First Angle shall be set to the value of \tcode{0.0}.

\pnum
The Second Angle shall be set to the value of \tcode{0.0}.
\end{itemdescr}

\indexlibrary{\idxcode{arc_clockwise}!constructor}
\begin{itemdecl}
constexpr arc_clockwise(const experimental::io2d::circle& c, double angle1,
  double angle2) noexcept;
\end{itemdecl}
\begin{itemdescr}
\pnum
\effects
Constructs an object of type \tcode{arc_clockwise}.

\pnum
The Circle shall be set to the value of \tcode{c}.

\pnum
The First Angle shall be set to the value of \tcode{angle1}.

\pnum
The Second Angle shall be set to the value of \tcode{angle2}.
\end{itemdescr}

\indexlibrary{\idxcode{arc_clockwise}!constructor}
\begin{itemdecl}
constexpr arc_clockwise(const vector_2d& ctr, double rad, double angle1,
  double angle2) noexcept;
\end{itemdecl}
\begin{itemdescr}
\pnum
\effects
Constructs an object of type \tcode{arc_clockwise}.

\pnum
The Circle's Center (\ref{circle.intro}) shall be set to the value of \tcode{ctr}.

\pnum
The Circle's Radius (\ref{circle.intro}) shall be set to the value of \tcode{rad}.

\pnum
The First Angle shall be set to the value of \tcode{angle1}.

\pnum
The Second Angle shall be set to the value of \tcode{angle2}.
\end{itemdescr}

\rSec1 [arcclockwise.modifiers]{\tcode{arc_clockwise} modifiers}

\indexlibrary{\idxcode{arc_clockwise}!\idxcode{circle}}
\begin{itemdecl}
constexpr void circle(const experimental::io2d::circle& c) noexcept;
\end{itemdecl}
\begin{itemdescr}
\pnum
\effects
The Circle shall be set to the value of \tcode{c}.
\end{itemdescr}

\indexlibrary{\idxcode{arc_clockwise}!\idxcode{center}}
\begin{itemdecl}
constexpr void center(const vector_2d& ctr) noexcept;
\end{itemdecl}
\begin{itemdescr}
\pnum
\effects
The Circle's Center (\ref{circle.intro}) shall be set to the value of \tcode{ctr}.
\end{itemdescr}

\indexlibrary{\idxcode{arc_clockwise}!\idxcode{radius}}
\begin{itemdecl}
constexpr void radius(double r) noexcept;
\end{itemdecl}
\begin{itemdescr}
\pnum
\effects
The Circle's Radius (\ref{circle.intro}) shall be set to the value of \tcode{r}.
\end{itemdescr}

\indexlibrary{\idxcode{arc_clockwise}!\idxcode{angle_1}}
\begin{itemdecl}
constexpr void angle_1(double radians) noexcept;
\end{itemdecl}
\begin{itemdescr}
\pnum
\effects
The First Angle shall be set to the value of \tcode{radians}.
\end{itemdescr}

\indexlibrary{\idxcode{arc_clockwise}!\idxcode{angle_2}}
\begin{itemdecl}
constexpr void angle_2(double radians) noexcept;
\end{itemdecl}
\begin{itemdescr}
\pnum
\effects
The Second Angle shall be set to the value of \tcode{radians}.
\end{itemdescr}

\rSec1 [arcclockwise.observers]{\tcode{arc_clockwise} observers}

\indexlibrary{\idxcode{arc_clockwise}!\idxcode{circle}}
\begin{itemdecl}
constexpr experimental::io2d::circle circle() const noexcept;
\end{itemdecl}
\begin{itemdescr}
\pnum
\returns
The value of the Circle.
\end{itemdescr}

\indexlibrary{\idxcode{arc_clockwise}!\idxcode{center}}
\begin{itemdecl}
constexpr vector_2d center() const noexcept;
\end{itemdecl}
\begin{itemdescr}
\pnum
\returns
The value of the Circle's Center (\ref{circle.intro}).
\end{itemdescr}

\indexlibrary{\idxcode{arc_clockwise}!\idxcode{radius}}
\begin{itemdecl}
constexpr double radius() const noexcept;
\end{itemdecl}
\begin{itemdescr}
\pnum
\returns
The value of the Circle's Radius (\ref{circle.intro}).
\end{itemdescr}

\indexlibrary{\idxcode{arc_clockwise}!\idxcode{angle_1}}
\begin{itemdecl}
constexpr double angle_1() const noexcept;
\end{itemdecl}
\begin{itemdescr}
\pnum
\returns
The value of the First Angle.
\end{itemdescr}

\indexlibrary{\idxcode{arc_clockwise}!\idxcode{angle_2}}
\begin{itemdecl}
constexpr double angle_2() const noexcept;
\end{itemdecl}
\begin{itemdescr}
\pnum
\returns
The value of the Second Angle.
\end{itemdescr}
