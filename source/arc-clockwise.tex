%!TEX root = io2d.tex
\rSec0 [arcclockwise] {Class \tcode{arc_clockwise}}

\pnum
\indexlibrary{\idxcode{arc_clockwise}}
The class \tcode{arc_clockwise} describes a path segment that is a circular arc with clockwise rotation.

\pnum
It has a center of type \tcode{vector_2d}, a radius of type \tcode{double}, a first angle of type \tcode{double}, and a second angle of type \tcode{double}.

\pnum
The values for the first angle and second angle are in radians.

\pnum
\enternote
Although the value of the first angle may be greater than the value of the second angle, when processed as described in \ref{paths.processing}, \tcode{two_pi<double>} is added to the second angle until the value of the second angle is greater than or equal to the value of the first angle.
\exitnote

\rSec1 [arcclockwise.synopsis] {\tcode{arc_clockwise} synopsis}

\begin{codeblock}
namespace std { namespace experimental { namespace io2d { inline namespace v1 {
  namespace path_data {
    class arc_clockwise {
    public:
      // \ref{arcclockwise.cons}, construct/copy/move/destroy:
      arc_clockwise(const vector_2d& ctr, double rad, double angle1,
        double angle2) noexcept;

      // \ref{arcclockwise.modifiers}, modifiers:
      void center(const vector_2d& ctr) noexcept;
      void radius(double r) noexcept;
      void angle_1(double radians) noexcept;
      void angle_2(double radians) noexcept;

      // \ref{arcclockwise.observers}, observers:
      vector_2d center() const noexcept;
      double radius() const noexcept;
      double angle_1() const noexcept;
      double angle_2() const noexcept;
    };
  };
} } } }
\end{codeblock}

\rSec1 [arcclockwise.cons] {\tcode{arc_clockwise} constructors and assignment operators}

\indexlibrary{\idxcode{arc_clockwise}!constructor}
\begin{itemdecl}
    arc_clockwise(const vector_2d& ctr, double rad, double angle1,
      double angle2) noexcept;
\end{itemdecl}
\begin{itemdescr}
	\pnum
	\effects
	Constructs an object of type \tcode{arc_clockwise}.
	
	\pnum
	The center shall be set to the value of \tcode{ctr}.
	
	\pnum
	The radius shall be set to the value of \tcode{rad}.
	
	\pnum
	The first angle shall be set to the value of \tcode{angle1}.
	
	\pnum
	The second angle shall be set to the value of \tcode{angle2}.
\end{itemdescr}

\rSec1 [arcclockwise.modifiers]{\tcode{arc_clockwise} modifiers}

\indexlibrary{\idxcode{arc_clockwise}!\idxcode{center}}
\indexlibrary{\idxcode{center}!\idxcode{arc_clockwise}}
\begin{itemdecl}
    void center(const vector_2d& ctr) noexcept;
\end{itemdecl}
\begin{itemdescr}
	\pnum
	\effects
	The center shall be set to the value of \tcode{ctr}.
\end{itemdescr}

\indexlibrary{\idxcode{arc_clockwise}!\idxcode{radius}}
\indexlibrary{\idxcode{radius}!\idxcode{arc_clockwise}}
\begin{itemdecl}
    void radius(double r) noexcept;
\end{itemdecl}
\begin{itemdescr}
	\pnum
	\effects
	The radius shall be set to the value of \tcode{r}.
\end{itemdescr}

\indexlibrary{\idxcode{arc_clockwise}!\idxcode{angle_1}}
\indexlibrary{\idxcode{angle_1}!\idxcode{arc_clockwise}}
\begin{itemdecl}
    void angle_1(double radians) noexcept;
\end{itemdecl}
\begin{itemdescr}
	\pnum
	\effects
	The first angle shall be set to the value of \tcode{radians}.
\end{itemdescr}

\indexlibrary{\idxcode{arc_clockwise}!\idxcode{angle_2}}
\indexlibrary{\idxcode{angle_2}!\idxcode{arc_clockwise}}
\begin{itemdecl}
    void angle_2(double radians) noexcept;
\end{itemdecl}
\begin{itemdescr}
	\pnum
	\effects
	The second angle shall be set to the value of \tcode{radians}.
\end{itemdescr}

\rSec1 [arcclockwise.observers]{\tcode{arc_clockwise} observers}

\indexlibrary{\idxcode{arc_clockwise}!\idxcode{center}}
\indexlibrary{\idxcode{center}!\idxcode{arc_clockwise}}
\begin{itemdecl}
    vector_2d center() const noexcept;
\end{itemdecl}
\begin{itemdescr}
	\pnum
	\returns
	The value of the center.
\end{itemdescr}

\indexlibrary{\idxcode{arc_clockwise}!\idxcode{radius}}
\indexlibrary{\idxcode{radius}!\idxcode{arc_clockwise}}
\begin{itemdecl}
    double radius() const noexcept;
\end{itemdecl}
\begin{itemdescr}
	\pnum
	\returns
	The value of the radius.
\end{itemdescr}

\indexlibrary{\idxcode{arc_clockwise}!\idxcode{angle_1}}
\indexlibrary{\idxcode{angle_1}!\idxcode{arc_clockwise}}
\begin{itemdecl}
    double angle_1() const noexcept;
\end{itemdecl}
\begin{itemdescr}
	\pnum
	\returns
	The value of the first angle.
\end{itemdescr}

\indexlibrary{\idxcode{arc_clockwise}!\idxcode{angle_2}}
\indexlibrary{\idxcode{angle_2}!\idxcode{arc_clockwise}}
\begin{itemdecl}
    double angle_2() const noexcept;
\end{itemdecl}
\begin{itemdescr}
	\pnum
	\returns
	The value of the second angle.
\end{itemdescr}
