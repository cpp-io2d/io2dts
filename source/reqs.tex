\rSec0 [\iotwod.req] {Requirements}

\rSec1 [\iotwod.req.namespace] {Namespaces and headers}

\pnum
The components described in this technical specification are experimental and 
not part of the C++ standard library. All components described in this 
technical specification are declared in namespace\\ \tcode{std::experimental::io2d::v1} or a sub-namespace thereof unless 
otherwise specified. The header described in this technical specification shall 
import the contents of \tcode{std::experimental::io2d::v1} into 
\tcode{std::experimental::io2d} as if by

\pnum
\begin{codeblock}
namespace std {
  namespace experimental {
    namespace io2d {
      inline namespace v1 { }
    }
  }
}
\end{codeblock}

\pnum
Unless otherwise specified, references to other entities described in this \documenttypename{} are assumed to be qualified with \tcode{std::experimental::io2d::v1::}, and references to entities
described in the C++ standard are assumed to be qualified with \tcode{std::}.

\rSec1 [\iotwod.req.macros] {Feature test macros}

\pnum
This macro allows users to determine which version of this \documenttypename{} is supported by header \tcode{<experimental/io2d>}.

\pnum
Header \tcode{<experimental/io2d>} shall supply the following macro definition:

\pnum
\tcode{\#define __cpp_lib_experimental_io2d 201707}

\pnum
\enternote
The value of macro __cpp_lib_experimental_io2d is yyyymm where yyyy is the year
and mm the month when the version of the Technical Specification was completed. \exitnote

\rSec1 [\iotwod.req.native] {Native handles}

\pnum
Several classes described in this \documenttypename{} have members 
\tcode{native_handle_type} and \tcode{native_handle}. The presence of these 
members and their semantics is \impldef{presence and meaning of 
\tcode{native_handle_type} and \tcode{native_handle}}.
\enternote
These members allow implementations to provide access to implementation 
details. Their names are specified to facilitate portable compile-time 
detection. Actual use of these members is inherently non-portable.
\exitnote

\rSec1 [\iotwod.req.\iecfivefivenine] {IEC 559 floating point support}

\pnum
Certain requirements of this \documenttypename{} assume that 
\tcode{numeric_limits<double>::is_iec559} evaluates to \tcode{true}.

\pnum
The 
behavior of these requirements is \impldef{\tcode{numeric_limits<double>::is_iec559} evaluates to \tcode{false}} 
if \tcode{numeric_limits<double>::is_iec559} evaluates to \tcode{false}.

%\rSec1 [\iotwod.req.cstdintopttypes] {Exact width integer types}
%
%\pnum
%In order to implement this \documenttypename{}, the implementation shall provide the following optional integer types from the \tcode{<cstdint>} header file:
%\begin{itemize}
%	\item \tcode{uint8_t}
%	\item \tcode{uint16_t}
%	\item \tcode{uint32_t}
%	\item \tcode{uint64_t}
%\end{itemize}
