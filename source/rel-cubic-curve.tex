%!TEX root = io2d.tex
\rSec0 [\iotwod.relcubiccurve] {Class \tcode{rel_cubic_curve}}

\pnum
\indexlibrary{\idxcode{rel_cubic_curve}}%
The class \tcode{rel_cubic_curve} describes a figure item that is a segment.

\pnum
It has a \term{first control point} of type \tcode{basic_point_2d}, a \term{second control point} of type \tcode{basic_point_2d}, and an \tcode{end point} of type \tcode{basic_point_2d}.

\rSec1 [\iotwod.relcubiccurve.cons] {\tcode{rel_cubic_curve} constructors}

\indexlibrary{\idxcode{rel_cubic_curve}!constructor}%
\begin{itemdecl}
rel_cubic_curve() noexcept;
\end{itemdecl}
\begin{itemdescr}
\pnum
\effects
Equivalent to \tcode{rel_cubic_curve\{ basic_point_2d(), basic_point_2d(), basic_point_2d() \}}.
\end{itemdescr}

\indexlibrary{\idxcode{rel_cubic_curve}!constructor}%
\begin{itemdecl}
rel_cubic_curve(const basic_point_2d<typename GraphicsSurfaces::graphics_math_type>& cpt1,
  const basic_point_2d<typename GraphicsSurfaces::graphics_math_type>& cpt2,
  const basic_point_2d<typename GraphicsSurfaces::graphics_math_type>& ept) noexcept;
\end{itemdecl}
\begin{itemdescr}
\pnum
\effects
Constructs an object of type \tcode{rel_cubic_curve}.

\pnum
The first control point is \tcode{cpt1}.

\pnum
The second control point is \tcode{cpt2}.

\pnum
The end point is \tcode{ept}.
\end{itemdescr}

\indexlibrary{\idxcode{rel_cubic_curve}!constructor}%
\begin{itemdecl}
rel_cubic_curve(const rel_cubic_curve& other);
rel_cubic_curve(rel_cubic_curve&& other) noexcept;
\end{itemdecl}
\begin{itemdescr}
\pnum
\effects
Constructs an object of type \tcode{rel_cubic_curve}. In the second form, other is left in a valid state with an unspecified value.

\pnum
The first control point is \tcode{other.control_pt1()}.

\pnum
The second control point is \tcode{other.control_pt2()}.

\pnum
The end point is \tcode{other.end_pt()}.
\end{itemdescr}

\rSec1 [\iotwod.relcubiccurve.assign] {\tcode{rel_cubic_curve} assignment operators}

\indexlibrary{\idxcode{rel_cubic_curve}!assignment}%
\begin{itemdecl}
rel_cubic_curve& operator=(const rel_cubic_curve& other);
\end{itemdecl}
\begin{itemdescr}
\pnum
\effects
If \tcode{*this} and \tcode{other} are not the same object, modifies \tcode{*this} such that \tcode{*this.control_pt1()} is \tcode{other.control_pt1()}, \tcode{*this.control_pt2()} is \tcode{other.control_pt2()} and \tcode{*this.end_pt()} is \tcode{other.end_pt()}

\pnum
If \tcode{*this} and \tcode{other} are the same object, the member has no effect.

\pnum
\returns
\tcode{*this}
\end{itemdescr}

\indexlibrary{\idxcode{rel_cubic_curve}!assignment}%
\begin{itemdecl}
rel_cubic_curve& operator=(rel_cubic_curve&& other) noexcept;
\end{itemdecl}
\begin{itemdescr}
\pnum
\effects
<TODO>

\pnum
\returns
\tcode{*this}
\end{itemdescr}

\rSec1 [\iotwod.relcubiccurve.modifiers]{\tcode{rel_cubic_curve} modifiers}

\indexlibrarymember{control_pt1}{rel_cubic_curve}%
\begin{itemdecl}
void control_pt1(const basic_point_2d<typename
  GraphicsSurfaces::graphics_math_type>& cpt) noexcept;
\end{itemdecl}
\begin{itemdescr}
\pnum
\effects
The first control point is \tcode{cpt}.
\end{itemdescr}

\indexlibrarymember{control_pt2}{rel_cubic_curve}%
\begin{itemdecl}
void control_pt2(const basic_point_2d<typename
  GraphicsSurfaces::graphics_math_type>& cpt) noexcept;
\end{itemdecl}
\begin{itemdescr}
\pnum
\effects
The second control point is \tcode{cpt}.
\end{itemdescr}

\indexlibrarymember{end_pt}{rel_cubic_curve}%
\begin{itemdecl}
void end_pt(const basic_point_2d<typename GraphicsSurfaces::graphics_math_type>& ept) noexcept;
\end{itemdecl}
\begin{itemdescr}
\pnum
\effects
The end point is \tcode{ept}.
\end{itemdescr}

\rSec1 [\iotwod.relcubiccurve.observers]{\tcode{rel_cubic_curve} observers}

\indexlibrarymember{control_pt1}{rel_cubic_curve}%
\begin{itemdecl}
basic_point_2d<typename GraphicsSurfaces::graphics_math_type> control_pt1() const noexcept;
\end{itemdecl}
\begin{itemdescr}
\pnum
\returns
The first control point.
\end{itemdescr}

\indexlibrarymember{control_pt2}{rel_cubic_curve}%
\begin{itemdecl}
basic_point_2d<typename GraphicsSurfaces::graphics_math_type> control_pt2() const noexcept;
\end{itemdecl}
\begin{itemdescr}
\pnum
\returns
The second control point.
\end{itemdescr}

\indexlibrarymember{end_pt}{rel_cubic_curve}%
\begin{itemdecl}
basic_point_2d<typename GraphicsSurfaces::graphics_math_type> end_pt() const noexcept;
\end{itemdecl}
\begin{itemdescr}
\pnum
\returns
The end point.
\end{itemdescr}

\rSec1 [\iotwod.relcubiccurve.ops]{\tcode{rel_cubic_curve} operators}

\indexlibrarymember{operator==}{rel_cubic_curve}%
\begin{itemdecl}
template <class GraphicsSurfaces>
bool operator==(const typename basic_figure_items<GraphicsSurfaces>::rel_cubic_curve& lhs,
  const typename basic_figure_items<GraphicsSurfaces>::rel_cubic_curve& rhs) noexcept;
\end{itemdecl}
\begin{itemdescr}
\pnum
\returns
\begin{codeblock}
lhs.control_pt1() == rhs.control_pt1() &&
lhs.control_pt2() == rhs.control_pt2() &&
lhs.end_pt() && rhs.end_pt()
\end{codeblock}
\end{itemdescr}
