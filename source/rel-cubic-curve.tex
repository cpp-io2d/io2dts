%!TEX root = io2d.tex
\rSec0 [\iotwod.relcubiccurve] {Class \tcode{rel_cubic_curve}}

\pnum
\indexlibrary{\idxcode{rel_cubic_curve}}%
The class \tcode{rel_cubic_curve} describes a path item that is a path segment.

\pnum
It has a \term{first control point} of type \tcode{vector_2d}, a \term{second control point} of type \tcode{vector_2d}, and an \tcode{end point} of type \tcode{vector_2d}.

\rSec1 [\iotwod.relcubiccurve.synopsis] {\tcode{rel_cubic_curve} synopsis}

\begin{codeblock}
namespace std::experimental::io2d::v1 {
  namespace path_data {
    class rel_cubic_curve {
    public:
      // \ref{\iotwod.relcubiccurve.cons}, construct
      constexpr rel_cubic_curve() noexcept;
      constexpr rel_cubic_curve(vector_2d cpt1, vector_2d cpt2,
        vector_2d ept) noexcept;

      // \ref{\iotwod.relcubiccurve.modifiers}, modifiers:
      constexpr void control_pt1(vector_2d cpt) noexcept;
      constexpr void control_pt2(vector_2d cpt) noexcept;
      constexpr void end_pt(vector_2d ept) noexcept;

      // \ref{\iotwod.relcubiccurve.observers}, observers:
      constexpr vector_2d control_pt1() const noexcept;
      constexpr vector_2d control_pt2() const noexcept;
      constexpr vector_2d end_pt() const noexcept;
    };
    
    // \ref{\iotwod.relcubiccurve.ops}, operators:
    constexpr bool operator==(const rel_cubic_curve& lhs,
      const rel_cubic_curve& rhs) noexcept;
    constexpr bool operator!=(const rel_cubic_curve& lhs,
      const rel_cubic_curve& rhs) noexcept;
  }
}
\end{codeblock}

\rSec1 [\iotwod.relcubiccurve.cons] {\tcode{rel_cubic_curve} constructors}

\indexlibrary{\idxcode{rel_cubic_curve}!constructor}%
\begin{itemdecl}
constexpr rel_cubic_curve() noexcept;
\end{itemdecl}
\begin{itemdescr}
\pnum
\effects
Equivalent to \tcode{rel_cubic_curve\{ vector_2d(), vector_2d(), vector_2d() \}}
\end{itemdescr}

\indexlibrary{\idxcode{rel_cubic_curve}!constructor}%
\begin{itemdecl}
constexpr rel_cubic_curve(vector_2d cpt1, vector_2d cpt2,
  vector_2d ept) noexcept;
\end{itemdecl}
\begin{itemdescr}
\pnum
\effects
Constructs an object of type \tcode{rel_cubic_curve}.

\pnum
The first control point is \tcode{cpt1}. The second control point is \tcode{cpt2}. The end point is \tcode{ept}.
\end{itemdescr}

\rSec1 [\iotwod.relcubiccurve.modifiers]{\tcode{rel_cubic_curve} modifiers}

\indexlibrarymember{control_pt2}{rel_cubic_curve}%
\begin{itemdecl}
constexpr void control_pt1(vector_2d cpt) noexcept;
\end{itemdecl}
\begin{itemdescr}
\pnum
\effects
The first control point is \tcode{cpt}.
\end{itemdescr}

\indexlibrarymember{control_pt2}{rel_cubic_curve}%
\begin{itemdecl}
constexpr void control_pt2(vector_2d cpt) noexcept;
\end{itemdecl}
\begin{itemdescr}
\pnum
\effects
The second control point is \tcode{cpt}.
\end{itemdescr}

\indexlibrarymember{end_pt}{rel_cubic_curve}%
\begin{itemdecl}
constexpr void end_pt(vector_2d ept) noexcept;
\end{itemdecl}
\begin{itemdescr}
\pnum
\effects
The end point is \tcode{ept}.
\end{itemdescr}

\rSec1 [\iotwod.relcubiccurve.observers]{\tcode{rel_cubic_curve} observers}

\indexlibrarymember{control_pt1}{rel_cubic_curve}%
\begin{itemdecl}
constexpr vector_2d control_pt1() const noexcept;
\end{itemdecl}
\begin{itemdescr}
\pnum
\returns
The first control point.
\end{itemdescr}

\indexlibrarymember{control_pt2}{rel_cubic_curve}%
\begin{itemdecl}
constexpr vector_2d control_pt2() const noexcept;
\end{itemdecl}
\begin{itemdescr}
\pnum
\returns
The second control point.
\end{itemdescr}

\indexlibrarymember{end_pt}{rel_cubic_curve}%
\begin{itemdecl}
constexpr vector_2d end_pt() const noexcept;
\end{itemdecl}
\begin{itemdescr}
\pnum
\returns
The end point.
\end{itemdescr}

\rSec1 [\iotwod.relcubiccurve.ops]{\tcode{rel_cubic_curve} operators}

\indexlibrarymember{operator==}{rel_cubic_curve}%
\begin{itemdecl}
constexpr bool operator==(const rel_cubic_curve& lhs,
  const rel_cubic_curve& rhs) noexcept;
\end{itemdecl}
\begin{itemdescr}
\pnum
\returns
\begin{codeblock}
lhs.control_pt1() == rhs.control_pt1() &&
lhs.control_pt2() == rhs.control_2pt() &&
lhs.end_pt() && rhs.end_pt()
\end{codeblock}
\end{itemdescr}
