%!TEX root = io2d.tex

\rSec0 [paths] {Paths}

\rSec1 [paths.overview]{Overview of paths}

\pnum
Paths define geometric objects which can be stroked (Table~\ref{tab:surface.rendering.operations}), filled, masked, and used to define a Clip Area (\ref{clipprops.summary}.

\pnum
A path group contains zero or more paths.

\pnum
A path is composed of at least one path segment.

\pnum
A path may contain degenerate path segments. When a path is rendered in certain rendering and composing operations, degenerate path segments can produce observable behavior.
\enterexample
When a degenerate path segment is rendered in a stroke rendering and composing operation (see \ref{surface.stroking}), the \tcode{line_cap} value contained in its \tcode{stroke_props} argument can result in a degenerate path segment producing observable behavior in the form of a circle or square, or some variation thereof.
\exitexample

\pnum
Paths provide vector graphics functionality. As such they are particularly useful in situations where an application is intended to run on a variety of platforms whose output devices (\ref{displaysurface.intro}) span a large gamut of sizes, both in terms of measurement units and in terms of a horizontal and vertical pixel count, in that order.
%
%\pnum
%For output devices, the measurement size of a pixel is determined by the physical size of the output device. Many output devices represent pixels as having the same horizontal and vertical measurement sizes. As such, when they display a rendered image which does not have the same horizontal to vertical pixel ratio as the output device, it 

\pnum
A \tcode{path_group} object is an immutable resource wrapper containing a path group (\ref{pathgroup}). A \tcode{path_group} object is created from the paths contained in a \tcode{path_builder} object. It can also be default constructed, in which case the \tcode{path_group} object contains no paths.
\enternote
\tcode{path_group} objects provide significant optimization opportunities for implementations due to being immutable and opaque.
\exitnote

\rSec1 [paths.example]{Path group examples (Informative)}

\pnum
Path groups are composed of zero or more paths. The following examples show the basics of how path groups work in practice.

\pnum
Every example is placed within the following code at the indicated spot. This code is shown here once to avoid repetition:

\begin{codeblock}
#include <experimental/io2d>

using namespace std;
using namespace std::experimental::io2d;

int main() {
  auto imgSfc = make_image_surface(format::argb32, 300, 200);
  brush backBrush{ bgra_color::black() };
  brush foreBrush{ bgra_color::white() };
  path_builder<> pb{};
  imgSfc.paint(backBrush);
  
  // Example code goes here.

  // Example code ends.
  
  imgSfc.save(filesystem::path("example.png"), image_file_format::png);
  return 0;
}
\end{codeblock}

\pnum
Example 1 consists of a single path, forming a trapezoid:

\begin{codeblock}
  pb.move({ 80.0, 20.0 }); // Begins the path.
  pb.line_to({ 220.0, 20.0 }); // Creates a line from the [80, 20] to [220, 20].
  pb.rel_line_to({ 60.0, 160.0 }); // Line from [220, 20] to
    // [220 + 60, 160 + 20]. The "to" point is relative to the starting point.
  pb.rel_line_to({ -260.0, 0.0 }); // Line from [280, 180] to 
    // [280 - 260, 180 + 0].
  pb.close_path(); // Creates a line from [20, 180] to [80, 20] 
    // (the last-move-to point), which makes this a closed path.
  imgSfc.stroke(foreBrush, pb);
\end{codeblock}

\begin{importgraphiciotwod}
{Example 1 result}
{fig:pathsexample1}
{pathexample01.png}
\end{importgraphiciotwod}

\FloatBarrier

\pnum
Example 2 consists of two paths. The first is a rectangular open path (on the left) and the second is a rectangular closed path (on the right):

\begin{codeblock}
  pb.move({ 20.0, 20.0 }); // Begin the first path.
  pb.rel_line_to({ 100.0, 0.0 });
  pb.rel_line_to({ 0.0, 160.0 });
  pb.rel_line_to({ -100.0, 0.0 });
  pb.rel_line_to({ 0.0, -160.0 });
  
  pb.move_to({ 180.0, 20.0 }); // End the first path and begin the second path.
  pb.rel_line_to({ 100.0, 0.0 });
  pb.rel_line_to({ 0.0, 160.0 });
  pb.rel_line_to({ -100.0, 0.0 });
  pb.close_path(); // End the second path.
  imgSfc.stroke(foreBrush, pb, nullopt, stroke_props{ 10.0 }); // Make the
    // stroke width 10.0 instead of the default 2.0.
\end{codeblock}

\begin{importgraphiciotwod}
{Example 2 result}
{fig:pathsexample2}
{pathexample02.png}
\end{importgraphiciotwod}

\FloatBarrier

\pnum
The resulting image from example 2 shows the difference between an open path and a closed path. Each path begins and ends at the same point. The difference is that with the closed path, that the rendering of the point where the initial path segment and final path segment meet is controlled by the \tcode{line_join} value in the \tcode{stroke_props} class, which in this case is the default value of \tcode{line_join::miter}. In the open path, the rendering of that point receives no special treatment such that each path segment at that point is rendered using the \tcode{line_cap} value in the \tcode{stroke_props} class, which in this case is the default value of \tcode{line_cap::none}.

\pnum
That difference between rendering as a \tcode{line_join} versus rendering as two \tcode{line_cap}s is what causes the notch to appear in the open path segment. Path segments are rendered such that half of the stroke width is rendered on each side of the point being evaluated. With no line cap, each segment begins and ends exactly at the point specified.

\pnum
So for the open path, the first line begins at \tcode{vector_2d\{ 20.0, 20.0 \}} and the last line ends there. Given the stroke width of \tcode{10.0}, the visible result for the first line is a rectangle with an upper left corner of \tcode{vector_2d\{ 20.0, 15.0 \}} and a lower right corner of \tcode{vector_2d\{ 120.0, 25.0 \}}. The last line appears as a rectangle with an upper left corner of \tcode{vector_2d\{ 15.0, 20.0 \}} and a lower right corner of \tcode{vector_2d\{ 25.0, 180.0 \}}. This produces the appearance of a square gap between \tcode{vector_2d\{ 15.0, 15.0 \}} and \tcode{vector_2d\{20.0, 20.0 \}}.

\pnum
For the closed path, adjusting for the coordinate differences, the rendering facts are the same as for the open path except for one key difference: the point where the first line and last line meet is rendered as a line join rather than two line caps, which, given the default value of \tcode{line_join::miter}, produces a miter, adding that square area to the rendering result.

%\pnum
%Example 3 demonstrates one circumstance in which degenerate path segments are rendered and also several operations that collapse into the establishment of a single path.
%
%\begin{codeblock}
%  pb.move({ 40.0, 40.0 }); // Begin the first path.
%  pb.move({ 40.0, 160.0 }); // Replace the first path since it is empty.
%  pb.line_to({ 40.0, 160.0 }); // Create a degenerate path segment.
%  pb.new_path(); // Begin the second path.
%  pb.line_to({ 100.0, 100.0 }); // Establish a current point and then create a 
%    // line that goes to it, i.e. a degenerate path segment.
%  pb.move_to({ 100.0, 160.0 }); // Begin the third path.
%  pb.rel_quadratic_curve_to({}, {}); // Create a degenerate path segment.
%  pb.move_to({ 100.0, 160.0 });
%  
%  imgSfc.stroke(foreBrush, pb, nullopt, stroke_props{ 10.0 }); // Make the
%    // stroke width 10.0 instead of the default 2.0.
%\end{codeblock}
%
%\begin{importgraphiciotwod}
%{Path example 3}
%{paths:example3}
%{pathexample03.png}
%\end{importgraphiciotwod}
%
%\FloatBarrier

\rSec1 [paths.processing] {Processing paths}

\pnum
This section is normative.

\pnum
It describes how to convert a properly formed path group into a \term{processed path group}. The steps required to create a processed path group require the existence of certain state data:

\begin{libiotwodreqtab3}
{path group processing state data}
{tab:paths.processing.statedata}
 \\ \topline
 \lhdr{Data}
 & \chdr{Type}
 & \rhdr{Initial Value}
 \\ \capsep
 \endfirsthead
 \continuedcaption\\
 \hline
 \lhdr{Data}
 & \chdr{Type}
 & \rhdr{Initial Value}
 \\ \capsep
 \endhead
 Transformation Matrix
 & \tcode{matrix_2d}
 & \tcode{matrix_2d::init_identity()}.
 \\
 Origin
 & \tcode{vector_2d}
 & \tcode{vector_2d\{ \}}.
 \\
 Current Point
 & \tcode{optional<vector_2d>}
 & \tcode{optional<vector_2d>\{ \}}.
 \\
\end{libiotwodreqtab3}

\pnum
Certain path instructions and path segments will modify this state data. The state data is used to ensure that the coordinates of all points in the path group's paths are properly transformed to their intended coordinates based on the effects of those path instructions and path segments.

\pnum
\enternote
The coordinates of a processed path group are in whatever units the user desires. This coordinate space is known as the User Coordinate Space (\ref{surface.coordinatespaces}). Path instructions such as \tcode{path_data::change_matrix} and \tcode{path_data::change_origin} affect the interpretation of path items that follow them within the path group. When a rendering and composing operation takes place, the coordinates of points within the processed path group are transformed into coordinates in the Surface Coordinate Space (\ref{surface.coordinatespaces}) using the Surface Properties' (\ref{surface.rendering.commonstate}) Surface Matrix (\ref{surfaceprops.summary}).
\exitnote

\pnum
The source code below demonstrates how to properly convert a path group into a processed path group.

\pnum
The \tcode{process_path_data} function template transforms the path group contained in a \tcode{path_builder} class template object into a processed path group which is returned as a \tcode{vector<path_data::path_data_types>} object. The processed path group only contains \tcode{path_data::abs_move}, \tcode{path_data::abs_line}, \tcode{path_data::abs_cubic_curve}, and \tcode{path_data::close_path} path items.

\pnum
\enternote
If the \underlyingrendandpresenttechs do not support lines or cubic \bezierlocal curves, Bresenham's algorithms and variations and improvements upon them allow computation of appropriate pixel values for these primitives.
\exitnote
\\

%!TEX root = io2d.tex

\begin{codeblock}
  #include <cmath>
  #include <vector>
  #include <variant>
  #include "io2d.h"
  
  
  namespace process_path {
    using namespace ::std;
    using namespace ::std::experimental::io2d;
  
    enum class abs_cubic_curve_sfinae {};
    constexpr abs_cubic_curve_sfinae abs_cubic_curve_sfinae_val = {};
    enum class abs_ellipse_sfinae {};
    constexpr static abs_ellipse_sfinae abs_ellipse_sfinae_val = {};
    enum class abs_line_sfinae {};
    constexpr abs_line_sfinae abs_line_sfinae_val = {};
    enum class abs_move_sfinae {};
    constexpr abs_move_sfinae abs_move_sfinae_val = {};
    enum class abs_quadratic_curve_sfinae {};
    constexpr abs_quadratic_curve_sfinae abs_quadratic_curve_sfinae_val = {};
    enum class abs_rectangle_sfinae {};
    constexpr static abs_rectangle_sfinae abs_rectangle_sfinae_val = {};
    enum class arc_clockwise_sfinae {};
    constexpr arc_clockwise_sfinae arc_clockwise_sfinae_val = {};
    enum class arc_counterclockwise_sfinae {};
    constexpr arc_counterclockwise_sfinae arc_counterclockwise_sfinae_val = {};
    enum class change_matrix_sfinae {};
    constexpr change_matrix_sfinae change_matrix_sfinae_val = {};
    enum class change_origin_sfinae {};
    constexpr change_origin_sfinae change_origin_sfinae_val = {};
    enum class close_path_sfinae {};
    constexpr close_path_sfinae close_path_sfinae_val = {};
    enum class new_path_sfinae {};
    constexpr new_path_sfinae new_path_sfinae_val = {};
    enum class rel_cubic_curve_sfinae {};
    constexpr rel_cubic_curve_sfinae rel_cubic_curve_sfinae_val = {};
    enum class rel_ellipse_sfinae {};
    constexpr static rel_ellipse_sfinae rel_ellipse_sfinae_val = {};
    enum class rel_line_sfinae {};
    constexpr rel_line_sfinae rel_line_sfinae_val = {};
    enum class rel_move_sfinae {};
    constexpr rel_move_sfinae rel_move_sfinae_val = {};
    enum class rel_quadratic_curve_sfinae {};
    constexpr rel_quadratic_curve_sfinae rel_quadratic_curve_sfinae_val = {};
    enum class rel_rectangle_sfinae {};
    constexpr static rel_rectangle_sfinae rel_rectangle_sfinae_val = {};
  
    template <class Allocator>
    vector<path_data::path_data_types> process_path_data(const path_factory<Allocator>& pf);
  
    template <class _TItem>
    struct path_factory_process_visit {
      constexpr static double twoThirds = 2.0 / 3.0;
  
      template <class T, enable_if_t<is_same_v<T, path_data::abs_cubic_curve>, abs_cubic_curve_sfinae> = abs_cubic_curve_sfinae_val>
      static void perform(const T& item, vector<path_data::path_data_types>& v, matrix_2d& m, vector_2d& origin, optional<vector_2d>& currentPoint, vector_2d& closePoint) {
        auto pt1 = m.transform_point(item.control_point_1() - origin) + origin;
        auto pt2 = m.transform_point(item.control_point_2() - origin) + origin;
        auto pt3 = m.transform_point(item.end_point() - origin) + origin;
        if (!currentPoint.has_value()) {
          currentPoint = item.control_point_1();
          v.emplace_back(in_place_type<path_data::abs_move>, pt1);
          closePoint = pt1;
        }
        v.emplace_back(in_place_type<path_data::abs_cubic_curve>, pt1,
          pt2, pt3);
        currentPoint = item.end_point();
      }
      template <class T, ::std::enable_if_t<::std::is_same_v<T, path_data::abs_ellipse>, abs_ellipse_sfinae> = abs_ellipse_sfinae_val>
      static void perform(const T& item, ::std::vector<path_data::path_data_types>&v, matrix_2d& m, vector_2d& origin, optional<vector_2d>& currentPoint, vector_2d& closePoint) {
        const auto m2 = m;
        const auto o2 = origin;
        currentPoint.reset();
        path_factory_process_visit<path_data::change_origin>::template perform(path_data::change_origin{ item.center() }, v, m, origin, currentPoint, closePoint);
        path_factory_process_visit<path_data::change_matrix>::template perform(path_data::change_matrix{ matrix_2d::init_scale({ item.x_axis() / item.y_axis(), 1.0 }) * m }, v, m, origin, currentPoint, closePoint);
        path_factory_process_visit<path_data::arc_clockwise>::template perform(path_data::arc_clockwise{ item.center(), item.y_axis(), 0.0, two_pi<double> }, v, m, origin, currentPoint, closePoint);
        path_factory_process_visit<path_data::change_matrix>::template perform(path_data::change_matrix{ m2 }, v, m, origin, currentPoint, closePoint);
        path_factory_process_visit<path_data::change_origin>::template perform(path_data::change_origin{ o2 }, v, m, origin, currentPoint, closePoint);
      }
      template <class T, enable_if_t<is_same_v<T, path_data::abs_line>, abs_line_sfinae> = abs_line_sfinae_val>
      static void perform(const T& item, vector<path_data::path_data_types>& v, matrix_2d& m, vector_2d& origin, optional<vector_2d>& currentPoint, vector_2d& closePoint) {
        if (currentPoint.has_value()) {
          currentPoint = item.to();
          auto pt = m.transform_point(currentPoint.value() - origin) + origin;
          v.emplace_back(in_place_type<path_data::abs_line>, pt);
        }
        else {
          currentPoint = item.to();
          auto pt = m.transform_point(currentPoint.value() - origin) + origin;
          v.emplace_back(in_place_type<path_data::abs_move>, pt);
          v.emplace_back(in_place_type<path_data::abs_line>, pt);
          closePoint = pt;
        }
      }
      template <class T, enable_if_t<is_same_v<T, path_data::abs_move>, abs_move_sfinae> = abs_move_sfinae_val>
      static void perform(const T& item, vector<path_data::path_data_types>& v, matrix_2d& m, vector_2d& origin, optional<vector_2d>& currentPoint, vector_2d& closePoint) {
        currentPoint = item.to();
        auto pt = m.transform_point(currentPoint.value() - origin) + origin;
        v.emplace_back(in_place_type<path_data::abs_move>, pt);
        closePoint = pt;
      }
      template <class T, enable_if_t<is_same_v<T, path_data::abs_quadratic_curve>, abs_quadratic_curve_sfinae> = abs_quadratic_curve_sfinae_val>
      static void perform(const T& item, vector<path_data::path_data_types>& v, matrix_2d& m, vector_2d& origin, optional<vector_2d>& currentPoint, vector_2d& closePoint) {
        // Turn it into a cubic curve since cairo doesn't have quadratic curves.
        vector_2d beginPt;
        auto controlPt = m.transform_point(item.control_point() - origin) + origin;
        auto endPt = m.transform_point(item.end_point() - origin) + origin;
        if (!currentPoint.has_value()) {
          currentPoint = item.control_point();
          v.emplace_back(in_place_type<path_data::abs_move>, controlPt);
          closePoint = controlPt;
          beginPt = controlPt;
        }
        else {
          beginPt = m.transform_point(currentPoint.value() - origin) + origin;
        }
        vector_2d cpt1 = { ((controlPt.x() - beginPt.x()) * twoThirds) + beginPt.x(), ((controlPt.y() - beginPt.y()) * twoThirds) + beginPt.y() };
        vector_2d cpt2 = { ((controlPt.x() - endPt.x()) * twoThirds) + endPt.x(), ((controlPt.y() - endPt.y()) * twoThirds) + endPt.y() };
        v.emplace_back(in_place_type<path_data::abs_cubic_curve>, cpt1, cpt2, endPt);
        currentPoint = item.end_point();
      }
      template <class T, ::std::enable_if_t<::std::is_same_v<T, path_data::abs_rectangle>, abs_rectangle_sfinae> = abs_rectangle_sfinae_val>
      static void perform(const T& item, ::std::vector<path_data::path_data_types>&v, matrix_2d& m, vector_2d& origin, optional<vector_2d>& currentPoint, vector_2d& closePoint) {
        path_factory_process_visit::template perform(path_data::abs_move{ { item.x(), item.y() } }, v, m, origin, currentPoint, closePoint);
        path_factory_process_visit::template perform(path_data::rel_line{ { item.width(), 0.0 } }, v, m, origin, currentPoint, closePoint);
        path_factory_process_visit::template perform(path_data::rel_line{ { 0.0, item.height() } }, v, m, origin, currentPoint, closePoint);
        path_factory_process_visit::template perform(path_data::rel_line{ { -item.width(), 0.0 } }, v, m, origin, currentPoint, closePoint);
        path_factory_process_visit::template perform(path_data::close_path{ { item.x(), item.y() } }, v, m, origin, currentPoint, closePoint);
      }
      template <class T, enable_if_t<is_same_v<T, path_data::arc_clockwise>, arc_clockwise_sfinae> = arc_clockwise_sfinae_val>
      static void perform(const T& item, vector<path_data::path_data_types>& v, matrix_2d& m, vector_2d& origin, optional<vector_2d>& currentPoint, vector_2d& closePoint) {
        {
          auto ctr = item.center();
          auto rad = item.radius();
          auto ang1 = item.angle_1();
          auto ang2 = item.angle_2();
          while (ang2 < ang1) {
            ang2 += two_pi<double>;
          }
          vector_2d pt0, pt1, pt2, pt3;
          int bezCount = 1;
          double theta = ang2 - ang1;
          double phi{};
          while (theta >= half_pi<double>) {
            theta /= 2.0;
            bezCount += bezCount;
          }
          phi = theta / 2.0;
          auto cosPhi = cos(phi);
          auto sinPhi = sin(phi);
          pt0.x(cosPhi);
          pt0.y(-sinPhi);
          pt3.x(pt0.x());
          pt3.y(-pt0.y());
          pt1.x((4.0 - cosPhi) / 3.0);
          pt1.y(-(((1.0 - cosPhi) * (3.0 - cosPhi)) / (3.0 * sinPhi)));
          pt2.x(pt1.x());
          pt2.y(-pt1.y());
          phi = -phi;
          auto rotCwFn = [](const vector_2d& pt, double a) -> vector_2d {
            return { pt.x() * cos(a) + pt.y() * sin(a),
              -(pt.x() * -(sin(a)) + pt.y() * cos(a)) };
          };
          pt0 = rotCwFn(pt0, phi);
          pt1 = rotCwFn(pt1, phi);
          pt2 = rotCwFn(pt2, phi);
          pt3 = rotCwFn(pt3, phi);
  
          auto currTheta = ang1;
          const auto startPt =
            ctr + rotCwFn({ pt0.x() * rad, pt0.y() * rad }, currTheta);
          if (currentPoint.has_value()) {
            currentPoint = startPt;
            auto pt = m.transform_point(currentPoint.value() - origin) + origin;
            v.emplace_back(in_place_type<path_data::abs_line>, pt);
          }
          else {
            currentPoint = startPt;
            auto pt = m.transform_point(currentPoint.value() - origin) + origin;
            v.emplace_back(in_place_type<path_data::abs_move>, pt);
            closePoint = pt;
          }
          for (; bezCount > 0; bezCount--) {
            auto cpt1 = ctr + rotCwFn({ pt1.x() * rad, pt1.y() * rad }, currTheta);
            auto cpt2 = ctr + rotCwFn({ pt2.x() * rad, pt2.y() * rad },
              currTheta);
            auto cpt3 = ctr + rotCwFn({ pt3.x() * rad, pt3.y() * rad },
              currTheta);
            currentPoint = cpt3;
            cpt1 = m.transform_point(cpt1 - origin) + origin;
            cpt2 = m.transform_point(cpt2 - origin) + origin;
            cpt3 = m.transform_point(cpt3 - origin) + origin;
            v.emplace_back(in_place_type<path_data::abs_cubic_curve>, cpt1,
              cpt2, cpt3);
            currTheta += theta;
          }
        }
      }
      template <class T, enable_if_t<is_same_v<T, path_data::arc_counterclockwise>, arc_counterclockwise_sfinae> = arc_counterclockwise_sfinae_val>
      static void perform(const T& item, vector<path_data::path_data_types>& v, matrix_2d& m, vector_2d& origin, optional<vector_2d>& currentPoint, vector_2d& closePoint) {
        {
          auto ctr = item.center();
          auto rad = item.radius();
          auto ang1 = item.angle_1();
          auto ang2 = item.angle_2();
          while (ang2 > ang1) {
            ang2 -= two_pi<double>;
          }
          vector_2d pt0, pt1, pt2, pt3;
          int bezCount = 1;
          double theta = ang1 - ang2;
          double phi{};
          while (theta >= half_pi<double>) {
            theta /= 2.0;
            bezCount += bezCount;
          }
          phi = theta / 2.0;
          auto cosPhi = cos(phi);
          auto sinPhi = sin(phi);
          pt0.x(cosPhi);
          pt0.y(-sinPhi);
          pt3.x(pt0.x());
          pt3.y(-pt0.y());
          pt1.x((4.0 - cosPhi) / 3.0);
          pt1.y(-(((1.0 - cosPhi) * (3.0 - cosPhi)) / (3.0 * sinPhi)));
          pt2.x(pt1.x());
          pt2.y(-pt1.y());
          auto rotCwFn = [](const vector_2d& pt, double a) -> vector_2d {
            return { pt.x() * cos(a) + pt.y() * sin(a),
              -(pt.x() * -(sin(a)) + pt.y() * cos(a)) };
          };
          pt0 = rotCwFn(pt0, phi);
          pt1 = rotCwFn(pt1, phi);
          pt2 = rotCwFn(pt2, phi);
          pt3 = rotCwFn(pt3, phi);
          auto shflPt = pt3;
          pt3 = pt0;
          pt0 = shflPt;
          shflPt = pt2;
          pt2 = pt1;
          pt1 = shflPt;
          auto currTheta = ang1;
          const auto startPt =
            ctr + rotCwFn({ pt0.x() * rad, pt0.y() * rad }, currTheta);
          if (currentPoint.has_value()) {
            currentPoint = startPt;
            auto pt = m.transform_point(currentPoint.value() - origin) + origin;
            v.emplace_back(in_place_type<path_data::abs_line>, pt);
          }
          else {
            currentPoint = startPt;
            auto pt = m.transform_point(currentPoint.value() - origin) + origin;
            v.emplace_back(in_place_type<path_data::abs_move>, pt);
            closePoint = pt;
          }
          for (; bezCount > 0; bezCount--) {
            auto cpt1 = ctr + rotCwFn({ pt1.x() * rad, pt1.y() * rad },
              currTheta);
            auto cpt2 = ctr + rotCwFn({ pt2.x() * rad, pt2.y() * rad },
              currTheta);
            auto cpt3 = ctr + rotCwFn({ pt3.x() * rad, pt3.y() * rad },
              currTheta);
            currentPoint = cpt3;
            cpt1 = m.transform_point(cpt1 - origin) + origin;
            cpt2 = m.transform_point(cpt2 - origin) + origin;
            cpt3 = m.transform_point(cpt3 - origin) + origin;
            v.emplace_back(in_place_type<path_data::abs_cubic_curve>, cpt1,
              cpt2, cpt3);
            currTheta -= theta;
          }
        }
      }
      template <class T, enable_if_t<is_same_v<T, path_data::change_matrix>, change_matrix_sfinae> = change_matrix_sfinae_val>
      static void perform(const T& item, vector<path_data::path_data_types>&, matrix_2d& m, vector_2d&, optional<vector_2d>&, vector_2d&) {
        if (!m.is_finite()) {
          throw system_error(make_error_code(io2d_error::invalid_matrix));
        }
        if (!m.is_invertible()) {
          throw system_error(make_error_code(io2d_error::invalid_matrix));
          return;
        }
        m = item.matrix();
      }
      template <class T, enable_if_t<is_same_v<T, path_data::change_origin>, change_origin_sfinae> = change_origin_sfinae_val>
      static void perform(const T& item, vector<path_data::path_data_types>&, matrix_2d&, vector_2d& origin, optional<vector_2d>&, vector_2d&) {
        origin = item.origin();
      }
      template <class T, ::std::enable_if_t<::std::is_same_v<T, path_data::close_path>, close_path_sfinae> = close_path_sfinae_val>
      static void perform(const T&, ::std::vector<path_data::path_data_types>& v, matrix_2d& m, vector_2d& origin, optional<vector_2d>& currentPoint, vector_2d& closePoint) {
        if (currentPoint.has_value()) {
          v.emplace_back(::std::in_place_type<path_data::close_path>, closePoint);
          v.emplace_back(::std::in_place_type<path_data::abs_move>,
            closePoint);
          if (!m.is_finite() || !m.is_invertible()) {
            throw ::std::system_error(::std::make_error_code(io2d_error::invalid_matrix));
          }
          auto invM = matrix_2d{ m }.invert();
          // Need to assign the untransformed closePoint value to currentPoint.
          currentPoint = invM.transform_point(closePoint - origin) + origin;
        }
      }
      template <class T, enable_if_t<is_same_v<T, path_data::new_path>, new_path_sfinae> = new_path_sfinae_val>
      static void perform(const T&, vector<path_data::path_data_types>&, matrix_2d&, vector_2d&, optional<vector_2d>& currentPoint, vector_2d&) {
        currentPoint.reset();
      }
      template <class T, enable_if_t<is_same_v<T, path_data::rel_cubic_curve>, rel_cubic_curve_sfinae> = rel_cubic_curve_sfinae_val>
      static void perform(const T& item, vector<path_data::path_data_types>& v, matrix_2d& m, vector_2d& origin, optional<vector_2d>& currentPoint, vector_2d&) {
        if (!currentPoint.has_value()) {
          throw system_error(make_error_code(io2d_error::invalid_path_data));
        }
        auto pt1 = m.transform_point(item.control_point_1() + currentPoint.value() -
          origin) + origin;
        auto pt2 = m.transform_point(item.control_point_2() + currentPoint.value() -
          origin) + origin;
        auto pt3 = m.transform_point(item.end_point() + currentPoint.value() - origin) +
          origin;
        v.emplace_back(in_place_type<path_data::abs_cubic_curve>,
          pt1, pt2, pt3);
        currentPoint = item.end_point() + currentPoint.value();
      }
      template <class T, ::std::enable_if_t<::std::is_same_v<T, path_data::rel_ellipse>, rel_ellipse_sfinae> = rel_ellipse_sfinae_val>
      static void perform(const T& item, ::std::vector<path_data::path_data_types>&v, matrix_2d& m, vector_2d& origin, optional<vector_2d>& currentPoint, vector_2d& closePoint) {
        if (!currentPoint.has_value()) {
          throw ::std::system_error(::std::make_error_code(io2d_error::invalid_path_data));
        }
        const auto m2 = m;
        const auto o2 = origin;
        const auto cpt2 = currentPoint;
        currentPoint.reset();
        path_factory_process_visit::template perform(path_data::change_origin{ item.center() + cpt2.value() }, v, m, origin, currentPoint, closePoint);
        path_factory_process_visit::template perform(path_data::change_matrix{ matrix_2d::init_scale({ item.x_axis() / item.y_axis(), 1.0 }) * m }, v, m, origin, currentPoint, closePoint);
        path_factory_process_visit::template perform(path_data::arc_clockwise{ item.center() + cpt2.value(), item.y_axis(), 0.0, two_pi<double> }, v, m, origin, currentPoint, closePoint);
        path_factory_process_visit::template perform(path_data::change_matrix{ m2 }, v, m, origin, currentPoint, closePoint);
        path_factory_process_visit::template perform(path_data::change_origin{ o2 }, v, m, origin, currentPoint, closePoint);
      }
      template <class T, enable_if_t<is_same_v<T, path_data::rel_line>, rel_line_sfinae> = rel_line_sfinae_val>
      static void perform(const T& item, vector<path_data::path_data_types>& v, matrix_2d& m, vector_2d& origin, optional<vector_2d>& currentPoint, vector_2d&) {
        if (!currentPoint.has_value()) {
          throw system_error(make_error_code(io2d_error::invalid_path_data));
        }
        currentPoint = item.to() + currentPoint.value();
        auto pt = m.transform_point(currentPoint.value() - origin) + origin;
        v.emplace_back(in_place_type<path_data::abs_line>, pt);
      }
      template <class T, enable_if_t<is_same_v<T, path_data::rel_move>, rel_move_sfinae> = rel_move_sfinae_val>
      static void perform(const T& item, vector<path_data::path_data_types>& v, matrix_2d& m, vector_2d& origin, optional<vector_2d>& currentPoint, vector_2d& closePoint) {
        if (!currentPoint.has_value()) {
          throw system_error(make_error_code(io2d_error::invalid_path_data));
        }
        currentPoint = item.to() + currentPoint.value();
        auto pt = m.transform_point(currentPoint.value() - origin) + origin;
        v.emplace_back(in_place_type<path_data::abs_move>, pt);
        closePoint = pt;
      }
      template <class T, enable_if_t<is_same_v<T, path_data::rel_quadratic_curve>, rel_quadratic_curve_sfinae> = rel_quadratic_curve_sfinae_val>
      static void perform(const T& item, vector<path_data::path_data_types>& v, matrix_2d& m, vector_2d& origin, optional<vector_2d>& currentPoint, vector_2d&) {
        if (!currentPoint.has_value()) {
          throw system_error(make_error_code(io2d_error::invalid_path_data));
        }
        // Turn it into a cubic curve since cairo doesn't have quadratic curves.
        vector_2d beginPt;
        auto controlPt = m.transform_point(item.control_point() + currentPoint.value() -
          origin) + origin;
        auto endPt = m.transform_point(item.end_point() + currentPoint.value() -
          origin) + origin;
        beginPt = m.transform_point(currentPoint.value() - origin) + origin;
        vector_2d cpt1 = { ((controlPt.x() - beginPt.x()) * twoThirds) + beginPt.x(), ((controlPt.y() - beginPt.y()) * twoThirds) + beginPt.y() };
        vector_2d cpt2 = { ((controlPt.x() - endPt.x()) * twoThirds) + endPt.x(), ((controlPt.y() - endPt.y()) * twoThirds) + endPt.y() };
        v.emplace_back(in_place_type<path_data::abs_cubic_curve>, cpt1, cpt2, endPt);
        currentPoint = item.end_point() + currentPoint.value();
      }
      template <class T, ::std::enable_if_t<::std::is_same_v<T, path_data::rel_rectangle>, rel_rectangle_sfinae> = rel_rectangle_sfinae_val>
      static void perform(const T& item, ::std::vector<path_data::path_data_types>&v, matrix_2d& m, vector_2d& origin, optional<vector_2d>& currentPoint, vector_2d& closePoint) {
        path_factory_process_visit::template perform(path_data::rel_move{ { item.x(), item.y() } }, v, m, origin, currentPoint, closePoint);
        path_factory_process_visit::template perform(path_data::rel_line{ { item.width(), 0.0 } }, v, m, origin, currentPoint, closePoint);
        path_factory_process_visit::template perform(path_data::rel_line{ { 0.0, item.height() } }, v, m, origin, currentPoint, closePoint);
        path_factory_process_visit::template perform(path_data::rel_line{ { -item.width(), 0.0 } }, v, m, origin, currentPoint, closePoint);
        path_factory_process_visit::template perform(path_data::close_path{ { item.x(), item.y() } }, v, m, origin, currentPoint, closePoint);
      }
    };
  
    template <class Allocator>
    inline vector<path_data::path_data_types> process_path_data(const path_factory<Allocator>& pf) {
      matrix_2d m;
      vector_2d origin;
      optional<vector_2d> currentPoint = optional<vector_2d>{}; // Tracks the untransformed current point.
      vector_2d closePoint;   // Tracks the transformed close point.
      vector<path_data::path_data_types> v;
  
      for (const path_data::path_data_types& val : pf) {
        visit([&m, &origin, &currentPoint, &closePoint, &v](auto&& item) {
          using T = remove_cv_t<remove_reference_t<decltype(item)>>;
          path_factory_process_visit<T>::template perform<T>(item, v, m, origin, currentPoint, closePoint);
        }, val);
      }
      return v;
    }
  }
\end{codeblock}


\addtocounter{SectionDepthBase}{1}
%!TEX root = io2d.tex
\rSec0 [\iotwod.abscubiccurve] {Class template \tcode{basic_figure_items<GraphicsSurfaces>::abs_cubic_curve}}

\rSec1 [\iotwod.abscubiccurve.intro] {Overview}

\pnum
\indexlibrary{\idxcode{abs_cubic_curve}}%
The class \tcode{basic_figure_items<GraphicsSurfaces>::abs_cubic_curve} describes a figure item that is a segment.

\pnum
It has a \term{first control point} of type \tcode{basic_point_2d<GraphicsSurfaces::graphics_math_type>}, a \term{second control point} of type \tcode{basic_point_2d<GraphicsSurfaces::graphics_math_type>}, and an \tcode{end point} of type \tcode{basic_point_2d<GraphicsSurfaces::graphics_math_type>}.

\pnum
The data are stored in an object of type \tcode{typename GraphicsSurfaces::paths::abs_cubic_curve_data_type}. It is accessible using the \tcode{data} member functions.

\rSec1 [\iotwod.abscubiccurve.synopsis] {Synopsis}
\begin{codeblock}
namespace @\fullnamespace{}@ {
  template <class GraphicsSurfaces>
  class basic_figure_items<GraphicsSurfaces>::abs_cubic_curve {
  public:
    using graphics_math_type = typename GraphicsSurfaces::graphics_math_type;
    using data_type =
      typename GraphicsSurfaces::paths::abs_cubic_curve_data_type;

    // \ref{\iotwod.abscubiccurve.ctor}, construct:
    abs_cubic_curve();
    abs_cubic_curve(const basic_point_2d<graphics_math_type>& cpt1,
       const basic_point_2d<graphics_math_type>& cpt2,
       const basic_point_2d<graphics_math_type>& ept) noexcept;
    abs_cubic_curve(const abs_cubic_curve& other) = default;
    abs_cubic_curve(abs_cubic_curve&& other) noexcept = default;

    // assign:
    abs_cubic_curve& operator=(const abs_cubic_curve& other) = default;
    abs_cubic_curve& operator=(abs_cubic_curve&& other) noexcept = default;

    // \ref{\iotwod.abscubiccurve.acc}, accessors:
    const data_type& data() const noexcept;
    data_type& data() noexcept;

    // \ref{\iotwod.abscubiccurve.mod}, modifiers:
    void control_pt1(const basic_point_2d<graphics_math_type>& cpt) noexcept;
    void control_pt2(const basic_point_2d<graphics_math_type>& cpt) noexcept;
    void end_pt(const basic_point_2d<graphics_math_type>& ept) noexcept;

    // \ref{\iotwod.abscubiccurve.obs}, observers:
    basic_point_2d<graphics_math_type> control_pt1() const noexcept;
    basic_point_2d<graphics_math_type> control_pt2() const noexcept;
    basic_point_2d<graphics_math_type> end_pt() const noexcept;
  };

  // \ref{\iotwod.abscubiccurve.eq}, equality operators:
  template <class GraphicsSurfaces>
  bool operator==(
    const typename basic_figure_items<GraphicsSurfaces>::abs_cubic_curve& lhs,
    const typename basic_figure_items<GraphicsSurfaces>::abs_cubic_curve& rhs) 
    noexcept;  
  template <class GraphicsSurfaces>
  bool operator!=(
    const typename basic_figure_items<GraphicsSurfaces>::abs_cubic_curve& lhs,
    const typename basic_figure_items<GraphicsSurfaces>::abs_cubic_curve& rhs) 
    noexcept;  
}
\end{codeblock}

\rSec1 [\iotwod.abscubiccurve.ctor] {Constructors}%

\indexlibrary{\idxcode{abs_cubic_curve}!constructor}%
\begin{itemdecl}
abs_cubic_curve() noexcept;
\end{itemdecl}
\begin{itemdescr}
\pnum
\effects
Equivalent to \tcode{abs_cubic_curve\{ basic_point_2d(), basic_point_2d(), basic_point_2d() \}}.
\end{itemdescr}

\indexlibrary{\idxcode{abs_cubic_curve}!constructor}%
\begin{itemdecl}
abs_cubic_curve(const basic_point_2d<typename GraphicsSurfaces::graphics_math_type>& cpt1,
  const basic_point_2d<typename GraphicsSurfaces::graphics_math_type>& cpt2,
  const basic_point_2d<typename GraphicsSurfaces::graphics_math_type>& ept) noexcept;
\end{itemdecl}
\begin{itemdescr}
\pnum
\effects Constructs an object of type \tcode{abs_cubic_curve}.

\pnum
\remarks The first control point is \tcode{cpt1}.

\pnum
\remarks The second control point is \tcode{cpt2}.

\pnum
\remarks The end point is \tcode{ept}.
\end{itemdescr}

\rSec1 [\iotwod.abscubiccurve.acc] {Accessors}%

\indexlibrarymember{data}{abs_cubic_curve}%
\begin{itemdecl}
const data_type& data() const noexcept;
data_type& data() noexcept;
\end{itemdecl}
\begin{itemdescr}
\pnum
\returns A reference to the \tcode{rel_matrix} object's data object (See: \ref{\iotwod.abscubiccurve.intro}).
\end{itemdescr}

\rSec1 [\iotwod.abscubiccurve.mod] {Modifiers}

\indexlibrarymember{control_pt1}{abs_cubic_curve}%
\begin{itemdecl}
void control_pt1(const basic_point_2d<typename
  GraphicsSurfaces::graphics_math_type>& cpt) noexcept;
\end{itemdecl}
\begin{itemdescr}
\pnum
\effects
The first control point is \tcode{cpt}.
\end{itemdescr}

\indexlibrarymember{control_pt2}{abs_cubic_curve}%
\begin{itemdecl}
void control_pt2(const basic_point_2d<typename
  GraphicsSurfaces::graphics_math_type>& cpt) noexcept;
\end{itemdecl}
\begin{itemdescr}
\pnum
\effects
The second control point is \tcode{cpt}.
\end{itemdescr}

\indexlibrarymember{end_pt}{abs_cubic_curve}%
\begin{itemdecl}
void end_pt(const basic_point_2d<typename GraphicsSurfaces::graphics_math_type>& ept) noexcept;
\end{itemdecl}
\begin{itemdescr}
\pnum
\effects
The end point is \tcode{ept}.
\end{itemdescr}

\rSec1 [\iotwod.abscubiccurve.obs] {Observers}

\indexlibrarymember{control_pt1}{abs_cubic_curve}%
\begin{itemdecl}
basic_point_2d<graphics_math_type> control_pt1() const noexcept;
\end{itemdecl}
\begin{itemdescr}
\pnum
\returns The first control point.
\end{itemdescr}

\indexlibrarymember{control_pt2}{abs_cubic_curve}%
\begin{itemdecl}
basic_point_2d<graphics_math_type> control_pt2() const noexcept;
\end{itemdecl}
\begin{itemdescr}
\pnum
\returns The second control point.
\end{itemdescr}

\indexlibrarymember{end_pt}{abs_cubic_curve}%
\begin{itemdecl}
basic_point_2d<graphics_math_type> end_pt() const noexcept;
\end{itemdecl}
\begin{itemdescr}
\pnum
\returns The end point.
\end{itemdescr}

\rSec1 [\iotwod.abscubiccurve.eq] {Equality operators}%

\indexlibrarymember{operator==}{abs_cubic_curve}%
\begin{itemdecl}
template <class GraphicsSurfaces>
bool operator==(
  const typename basic_figure_items<GraphicsSurfaces>::abs_cubic_curve& lhs,
  const typename basic_figure_items<GraphicsSurfaces>::abs_cubic_curve& rhs) 
  noexcept;
\end{itemdecl}
\begin{itemdescr}
\pnum
\returns
\tcode{lhs.control_pt1() == rhs.control_pt1() \&\& lhs.control_pt2() == rhs.control_pt2() \&\& lhs.end_pt() == rhs.end_pt()}.
\end{itemdescr}

\indexlibrarymember{operator!=}{abs_cubic_curve}%
\begin{itemdecl}
template <class GraphicsSurfaces>
bool operator!=(
  const typename basic_figure_items<GraphicsSurfaces>::abs_cubic_curve& lhs,
  const typename basic_figure_items<GraphicsSurfaces>::abs_cubic_curve& rhs) 
  noexcept;
\end{itemdecl}
\begin{itemdescr}
\pnum
\returns
\tcode{lhs.control_pt1() != rhs.control_pt1() || lhs.control_pt2() != rhs.control_pt2() || lhs.end_pt() != rhs.end_pt()}.
\end{itemdescr}

%!TEX root = io2d.tex
\rSec0 [absellipse] {Class \tcode{abs_ellipse}}

\pnum
\indexlibrary{\idxcode{abs_ellipse}}
The class \tcode{abs_ellipse} describes a path instruction that add a new path consisting of an ellipse and closes it.

\pnum
It has a Center of type \tcode{vector_2d}, an X Axis Radius of type \tcode{double}, and a Y Axis Radius of type \tcode{double}.

\rSec1 [absellipse.synopsis] {\tcode{abs_ellipse} synopsis}

\begin{codeblock}
namespace std { namespace experimental { namespace io2d { inline namespace v1 {
  namespace path_data {
    class abs_ellipse {
    public:
      // \ref{absellipse.cons}, construct/copy/move/destroy:
      constexpr abs_ellipse() noexcept;
      constexpr abs_ellipse(const vector_2d& ctr, double x, double y) noexcept;
      constexpr explicit abs_ellipse(const circle& c) noexcept;

      // \ref{absellipse.modifiers}, modifiers:
      constexpr void center(const vector_2d& ctr) noexcept;
      constexpr void x_axis(double rad) noexcept;
      constexpr void y_axis(double rad) noexcept;
    
      // \ref{absellipse.observers}, observers:
      constexpr vector_2d center() const noexcept;
      constexpr double x_axis() const noexcept;
      constexpr double y_axis() const noexcept;
    };
  }
} } } }
\end{codeblock}

\rSec1 [absellipse.cons] {\tcode{abs_ellipse} constructors}

\indexlibrary{\idxcode{abs_ellipse}!constructor}
\begin{itemdecl}
constexpr abs_ellipse() noexcept;
\end{itemdecl}
\begin{itemdescr}
\pnum
\effects
Constructs an object of type \tcode{abs_ellipse}.

\pnum
The value of Center is \tcode{vector_2d\{0,0, 0.0\}}.

\pnum
The value of X Axis Radius is \tcode{0.0}.

\pnum
The value of Y Axis Radius is \tcode{0.0}.
\end{itemdescr}

\indexlibrary{\idxcode{abs_ellipse}!constructor}
\begin{itemdecl}
constexpr abs_ellipse(const vector_2d& ctr, double x, double y) noexcept;
\end{itemdecl}
\begin{itemdescr}
\pnum
\preconditions
\tcode{x >= 0.0}.

\pnum
\tcode{y >= 0.0}.

\pnum
\effects
Constructs an object of type \tcode{abs_ellipse}.

\pnum
The value of Center is \tcode{ctr}.

\pnum
The value of X Axis Radius is \tcode{x}.

\pnum
The value of Y Axis Radius is \tcode{y}.
\end{itemdescr}

\indexlibrary{\idxcode{abs_ellipse}!constructor}
\begin{itemdecl}
constexpr explicit abs_ellipse(const circle& c) noexcept;
\end{itemdecl}
\begin{itemdescr}
\pnum
\preconditions
\tcode{c.radius() >= 0.0}.

\pnum
\effects
Constructs an object of type \tcode{abs_ellipse}.

\pnum
The value of Center is \tcode{c.center()}.

\pnum
The value of X Axis Radius is \tcode{c.radius()}.

\pnum
The value of Y Axis Radius is \tcode{c.radius()}.
\end{itemdescr}

\rSec1 [absellipse.modifiers]{\tcode{abs_ellipse} modifiers}

\indexlibrary{\idxcode{abs_ellipse}!\idxcode{center}}
\begin{itemdecl}
constexpr void center(const vector_2d& ctr) noexcept;
\end{itemdecl}

\begin{itemdescr}
\pnum
\effects
The value of Center is \tcode{ctr}.
\end{itemdescr}

\indexlibrary{\idxcode{abs_ellipse}!\idxcode{x_axis}}
\begin{itemdecl}
constexpr void x_axis(double rad) noexcept;
\end{itemdecl}
\begin{itemdescr}
\preconditions
\tcode{rad >= 0.0}.

\pnum
\effects
The value of X Axis Radius is \tcode{rad}.
\end{itemdescr}

\indexlibrary{\idxcode{abs_ellipse}!\idxcode{y_axis}}
\begin{itemdecl}
constexpr void y_axis(double rad) noexcept;
\end{itemdecl}
\begin{itemdescr}
\preconditions
\tcode{rad >= 0.0}.

\pnum
\effects
The value of Y Axis Radius is \tcode{rad}.
\end{itemdescr}

\rSec1 [absellipse.observers]{\tcode{abs_ellipse} observers}

\indexlibrary{\idxcode{abs_ellipse}!\idxcode{center}}
\begin{itemdecl}
constexpr double center() const noexcept;
\end{itemdecl}
\begin{itemdescr}
\pnum
\returns
The value of Center.
\end{itemdescr}

\indexlibrary{\idxcode{abs_ellipse}!\idxcode{x_axis}}
\begin{itemdecl}
constexpr double x_axis() const noexcept;
\end{itemdecl}
\begin{itemdescr}
\pnum
\returns
The value of X Axis Radius.
\end{itemdescr}

\indexlibrary{\idxcode{abs_ellipse}!\idxcode{y_axis}}
\begin{itemdecl}
constexpr double y_axis() const noexcept;
\end{itemdecl}
\begin{itemdescr}
\pnum
\returns
The value of Y Axis Radius.
\end{itemdescr}

%!TEX root = io2d.tex
\rSec0 [absline] {Class \tcode{abs_line}}

\pnum
\indexlibrary{\idxcode{abs_line}}
The class \tcode{abs_line} describes a path segment that is a line.

\pnum
It has an end point of type \tcode{vector_2d}.

\rSec1 [absline.synopsis] {\tcode{abs_line} synopsis}

\begin{codeblock}
namespace std { namespace experimental { namespace io2d { inline namespace v1 {
  namespace path_data {
    class abs_line {
    public:
      // \ref{absline.cons}, construct:
      constexpr abs_line() noexcept;
      constexpr explicit abs_line(const vector_2d& pt) noexcept;

      // \ref{absline.modifiers}, modifiers:
      constexpr void to(const vector_2d& pt) noexcept;

      // \ref{absline.observers}, observers:
      constexpr vector_2d to() const noexcept;
    };
  };
} } } }
\end{codeblock}

\rSec1 [absline.cons] {\tcode{abs_line} constructors and assignment operators}

\indexlibrary{\idxcode{abs_line}!constructor}
\begin{itemdecl}
constexpr abs_line() noexcept;
\end{itemdecl}
\begin{itemdescr}
\pnum
\effects
Constructs an object of type \tcode{abs_line}.

\pnum
The end point shall be set to the value of \tcode{vector_2d\{0.0, 0.0\}}.
\end{itemdescr}

\indexlibrary{\idxcode{abs_line}!constructor}
\begin{itemdecl}
constexpr explicit abs_line(const vector_2d& pt) noexcept;
\end{itemdecl}
\begin{itemdescr}
\pnum
\effects
Constructs an object of type \tcode{abs_line}.

\pnum
The end point shall be set to the value of \tcode{pt}.
\end{itemdescr}

\rSec1 [absline.modifiers]{\tcode{abs_line} modifiers}

\indexlibrary{\idxcode{abs_line}!\idxcode{to}}
\begin{itemdecl}
constexpr void to(const vector_2d& pt) noexcept;
\end{itemdecl}
\begin{itemdescr}
\pnum
\effects
The end point shall be set to the value of \tcode{pt}.
\end{itemdescr}

\rSec1 [absline.observers]{\tcode{abs_line} observers}

\indexlibrary{\idxcode{abs_line}!\idxcode{to}}
\begin{itemdecl}
constexpr vector_2d to() const noexcept;
\end{itemdecl}
\begin{itemdescr}
\pnum
\returns
The value of the end point.
\end{itemdescr}

%!TEX root = io2d.tex
\rSec0 [absmove] {Class \tcode{abs_move}}

\pnum
\indexlibrary{\idxcode{abs_move}}
The class \tcode{abs_move} describes a path operation that creates a new path and makes the previous path, if any, an open path unless it was closed by \tcode{close_path}.

\pnum
It has an end point of type \tcode{vector_2d}.

\pnum
The end point is also the start point of the new path and its last-move-to point.

\rSec1 [absmove.synopsis] {\tcode{abs_move} synopsis}

\begin{codeblock}
namespace std { namespace experimental { namespace io2d { inline namespace v1 {
  namespace path_data {
    class abs_move {
    public:
      // \ref{absmove.cons}, construct:
      explicit abs_move(const vector_2d& pt) noexcept;

      // \ref{absmove.modifiers}, modifiers:
      void to(const vector_2d& pt) noexcept;

      // \ref{absmove.observers}, observers:
      vector_2d to() const noexcept;
    };
  };
} } } }
\end{codeblock}

\rSec1 [absmove.cons] {\tcode{abs_move} constructors}

\indexlibrary{\idxcode{abs_move}!constructor}
\begin{itemdecl}
    explicit abs_move(const vector_2d& pt) noexcept;
\end{itemdecl}
\begin{itemdescr}
	\pnum
	\effects
	Constructs an object of type \tcode{abs_move}.
	
	\pnum
	The end point shall be set to the value of \tcode{pt}.
\end{itemdescr}

\rSec1 [absmove.modifiers]{\tcode{abs_move} modifiers}

\indexlibrary{\idxcode{abs_move}!\idxcode{to}}
\indexlibrary{\idxcode{to}!\idxcode{abs_move}}
\begin{itemdecl}
    void to(const vector_2d& pt) noexcept;
\end{itemdecl}
\begin{itemdescr}
	\pnum
	\effects
	The end point shall be set to the value of \tcode{pt}.
\end{itemdescr}

\rSec1 [absmove.observers]{\tcode{abs_move} observers}

\indexlibrary{\idxcode{abs_move}!\idxcode{to}}
\indexlibrary{\idxcode{to}!\idxcode{abs_move}}
\begin{itemdecl}
    vector_2d to() const noexcept;
\end{itemdecl}
\begin{itemdescr}
	\pnum
	\returns
	The value of the end point.
\end{itemdescr}

%!TEX root = io2d.tex
\rSec0 [\iotwod.absquadraticcurve] {Class \tcode{abs_quadratic_curve}}

\pnum
\indexlibrary{\idxcode{abs_quadratic_curve}}%
The class \tcode{abs_quadratic_curve} describes a figure item that is a segment.

\pnum
It has a \term{control point} of type \tcode{basic_point_2d} and an \term{end point} of type \tcode{basic_point_2d}.

\rSec1 [\iotwod.absquadraticcurve.cons] {\tcode{abs_quadratic_curve} constructors}

\indexlibrary{\idxcode{abs_quadratic_curve}!constructor}%
\begin{itemdecl}
abs_quadratic_curve() noexcept;
\end{itemdecl}
\begin{itemdescr}
\pnum
\effects
Equivalent to: \tcode{abs_quadratic_curve\{ basic_point_2d(), basic_point_2d() \};}
\end{itemdescr}

\indexlibrary{\idxcode{abs_quadratic_curve}!constructor}%
\begin{itemdecl}
abs_quadratic_curve(const basic_point_2d<typename GraphicsSurfaces::graphics_math_type>& cpt,
  const basic_point_2d<typename GraphicsSurfaces::graphics_math_type>& ept) noexcept;
\end{itemdecl}
\begin{itemdescr}
\pnum
\effects
Constructs an object of type \tcode{abs_quadratic_curve}.

\pnum
The control point is \tcode{cpt}.

\pnum
The end point is \tcode{ept}.
\end{itemdescr}

\indexlibrary{\idxcode{abs_quadratic_curve}!constructor}%
\begin{itemdecl}
abs_quadratic_curve(const abs_quadratic_curve& other);
abs_quadratic_curve(abs_quadratic_curve&& other) noexcept;
\end{itemdecl}
\begin{itemdescr}
\pnum
\effects
Constructs an object of type \tcode{abs_quadratic_curve}. In the second form, other is left in a valid state with an unspecified value.

\pnum
The control point is \tcode{other.control_pt()}.

\pnum
The end point is \tcode{other.end_pt()}.
\end{itemdescr}

\rSec1 [\iotwod.absquadraticcurve.assign] {\tcode{abs_quadratic_curve} assignment operators}

\indexlibrary{\idxcode{abs_quadratic_curve}!assignment}%
\begin{itemdecl}
abs_quadratic_curve& operator=(const abs_quadratic_curve& other);
\end{itemdecl}
\begin{itemdescr}
\pnum
\effects
If \tcode{*this} and \tcode{other} are not the same object, modifies \tcode{*this} such that \tcode{*this.control_pt()} is \tcode{other.control_pt()} and \tcode{*this.end_pt()} is \tcode{other.end_pt()}

\pnum
If \tcode{*this} and \tcode{other} are the same object, the member has no effect.

\pnum
\returns
\tcode{*this}
\end{itemdescr}

\indexlibrary{\idxcode{abs_quadratic_curve}!assignment}%
\begin{itemdecl}
abs_quadratic_curve& operator=(abs_quadratic_curve&& other) noexcept;
\end{itemdecl}
\begin{itemdescr}
\pnum
\effects
<TODO>

\pnum
\returns
\tcode{*this}
\end{itemdescr}

\rSec1 [\iotwod.absquadraticcurve.modifiers]{\tcode{abs_quadratic_curve} modifiers}

\indexlibrarymember{control_pt}{abs_quadratic_curve}%
\begin{itemdecl}
void control_pt(const basic_point_2d<typename GraphicsSurfaces::graphics_math_type>& cpt) noexcept;
\end{itemdecl}
\begin{itemdescr}
\pnum
\effects
The control point is \tcode{cpt}.
\end{itemdescr}

\indexlibrarymember{end_pt}{abs_quadratic_curve}%
\begin{itemdecl}
void end_pt(const basic_point_2d<typename GraphicsSurfaces::graphics_math_type>& ept) noexcept;
\end{itemdecl}
\begin{itemdescr}
\pnum
\effects
The end point is \tcode{ept}.
\end{itemdescr}

\rSec1 [\iotwod.absquadraticcurve.observers]{\tcode{abs_quadratic_curve} observers}

\indexlibrarymember{control_pt}{abs_quadratic_curve}%
\begin{itemdecl}
basic_point_2d<typename GraphicsSurfaces::graphics_math_type> control_pt() const noexcept;
\end{itemdecl}
\begin{itemdescr}
\pnum
\returns
The control point.
\end{itemdescr}

\indexlibrarymember{end_pt}{abs_quadratic_curve}%
\begin{itemdecl}
basic_point_2d<typename GraphicsSurfaces::graphics_math_type> end_pt() const noexcept;
\end{itemdecl}
\begin{itemdescr}
\pnum
\returns
The end point.
\end{itemdescr}

\rSec1 [\iotwod.absquadraticcurve.ops]{\tcode{abs_quadratic_curve} operators}

\indexlibrarymember{operator==}{abs_quadratic_curve}%
\begin{itemdecl}
template <class GraphicsSurfaces>
bool operator==(const typename basic_figure_items<GraphicsSurfaces>::abs_quadratic_curve& lhs,
  const typename basic_figure_items<GraphicsSurfaces>::abs_quadratic_curve& rhs) noexcept;
\end{itemdecl}
\begin{itemdescr}
\pnum
\returns
\tcode{lhs.control_pt() == rhs.control_pt() \&\& lhs.end_pt() == rhs.end_pt()}.
\end{itemdescr}

%!TEX root = io2d.tex
\rSec0 [\iotwod.absrectangle] {Class \tcode{abs_rectangle}}

\pnum
\indexlibrary{\idxcode{abs_rectangle}}
The class \tcode{abs_rectangle} describes a path instruction that adds a rectangle to the current path.

\pnum
It has an X coordinate of type \tcode{double}, a Y coordinate of type \tcode{double}, a Width of type \tcode{double}, and a Height of type \tcode{double}.

\rSec1 [\iotwod.absrectangle.synopsis] {\tcode{abs_rectangle} synopsis}

\begin{codeblock}
namespace std { namespace experimental { namespace io2d { inline namespace v1 {
  namespace path_data {
    class abs_rectangle {
    public:
      // \ref{\iotwod.absrectangle.cons}, constructors:
      constexpr abs_rectangle() noexcept;
      constexpr abs_rectangle(double x, double y, double w, double h) noexcept;
      constexpr abs_rectangle(const vector_2d& tl, const vector_2d& br) 
        noexcept;
      constexpr abs_rectangle(const rectangle& r);

      // \ref{\iotwod.absrectangle.modifiers}, modifiers:
      constexpr void x(double value) noexcept;
      constexpr void y(double value) noexcept;
      constexpr void width(double value) noexcept;
      constexpr void height(double value) noexcept;
      constexpr void top_left(const vector_2d& value) noexcept;
      constexpr void bottom_right(const vector_2d& value) noexcept;
      constexpr void top_left_bottom_right(const vector_2d& tl,
        const vector_2d& br) noexcept;

      // \ref{\iotwod.absrectangle.observers}, observers:
      constexpr double x() const noexcept;
      constexpr double y() const noexcept;
      constexpr double width() const noexcept;
      constexpr double height() const noexcept;
      constexpr double left() const noexcept;
      constexpr double right() const noexcept;
      constexpr double top() const noexcept;
      constexpr double bottom() const noexcept;
      constexpr vector_2d top_left() const noexcept;
      constexpr vector_2d bottom_right() const noexcept;
    };
  }
} } } }
\end{codeblock}

\rSec1 [\iotwod.absrectangle.cons] {\tcode{abs_rectangle} constructors}

\indexlibrary{\idxcode{abs_rectangle}!constructor}
\begin{itemdecl}
constexpr abs_rectangle() noexcept;
\end{itemdecl}
\begin{itemdescr}
\pnum
\effects
Constructs an object of type \tcode{abs_rectangle}.

\pnum
The X coordinate, Y coordinate, Width, and Height shall each be set to the value \tcode{0.0}.
\end{itemdescr}

\indexlibrary{\idxcode{abs_rectangle}!constructor}
\begin{itemdecl}
constexpr abs_rectangle(double x, double y, double w, double h) noexcept;
\end{itemdecl}
\begin{itemdescr}
\pnum
\effects
Constructs an object of type \tcode{abs_rectangle}.

\pnum
The X coordinate shall be set to the value of \tcode{x}.

\pnum
The Y coordinate shall be set to the value of \tcode{y}.

\pnum
The Width shall be set to the value of \tcode{w}.

\pnum
The Height shall be set to the value of \tcode{h}.
\end{itemdescr}

\indexlibrary{\idxcode{abs_rectangle}!constructor}
\begin{itemdecl}
constexpr abs_rectangle(const vector_2d& tl, const vector_2d& br) noexcept;
\end{itemdecl}
\begin{itemdescr}
\pnum
\effects
Constructs an object of type \tcode{abs_rectangle}.

\pnum
The X coordinate shall be set to the value of \tcode{tl.x()}.

\pnum
The Y coordinate shall be set to the value of \tcode{tl.y()}.

\pnum
The Width shall be set to the value of \tcode{max(0.0, br.x() - tl.x())}.

\pnum
The Height shall be set to the value of \tcode{max(0.0, br.y() - tl.y())}.
\end{itemdescr}

\rSec1 [\iotwod.absrectangle.modifiers]{\tcode{abs_rectangle} modifiers}

\indexlibrary{\idxcode{abs_rectangle}!\idxcode{x}}
\begin{itemdecl}
constexpr void x(double val) noexcept;
\end{itemdecl}

\begin{itemdescr}
\pnum
\effects
The X coordinate shall be set to the value of \tcode{val}.
\end{itemdescr}

\indexlibrary{\idxcode{abs_rectangle}!\idxcode{y}}
\begin{itemdecl}
constexpr void y(double value) noexcept;
\end{itemdecl}
\begin{itemdescr}
\pnum
\effects
The Y coordinate shall be set to the value of \tcode{val}.
\end{itemdescr}

\indexlibrary{\idxcode{abs_rectangle}!\idxcode{width}}
\begin{itemdecl}
constexpr void width(double value) noexcept;
\end{itemdecl}
\begin{itemdescr}
\pnum
\effects
The Width shall be set to the value of \tcode{val}.
\end{itemdescr}

\indexlibrary{\idxcode{abs_rectangle}!\idxcode{height}}
\begin{itemdecl}
constexpr void height(double value) noexcept;
\end{itemdecl}
\begin{itemdescr}
\pnum
\effects
The Height shall be set to the value of \tcode{val}.
\end{itemdescr}

\indexlibrary{\idxcode{abs_rectangle}!\idxcode{top_left}}
\begin{itemdecl}
constexpr void top_left(const vector_2d& val) noexcept;
\end{itemdecl}
\begin{itemdescr}
\pnum
\effects
The X coordinate shall be set to the value of \tcode{val.x()}.

\effects
The Y coordinate shall be set to the value of \tcode{val.y()}.
\end{itemdescr}

\indexlibrary{\idxcode{abs_rectangle}!\idxcode{bottom_right}}
\begin{itemdecl}
constexpr void bottom_right(const vector_2d& val) noexcept;
\end{itemdecl}
\begin{itemdescr}
\pnum
\effects
The Width shall be set to the value of \tcode{max(0.0, val.x() - *this.x())}.

\pnum
The Height shall be set to the value of \tcode{max(0.0, value.y() - *this.y())}.
\end{itemdescr}

\rSec1 [\iotwod.absrectangle.observers]{\tcode{abs_rectangle} observers}

\indexlibrary{\idxcode{abs_rectangle}!\idxcode{x}}
\begin{itemdecl}
constexpr double x() const noexcept;
\end{itemdecl}
\begin{itemdescr}
\pnum
\returns
The value of the X coordinate.
\end{itemdescr}

\indexlibrary{\idxcode{abs_rectangle}!\idxcode{y}}
\begin{itemdecl}
constexpr double y() const noexcept;
\end{itemdecl}
\begin{itemdescr}
\pnum
\returns
The value of the Y coordinate.
\end{itemdescr}

\indexlibrary{\idxcode{abs_rectangle}!\idxcode{width}}
\begin{itemdecl}
constexpr double width() const noexcept;
\end{itemdecl}
\begin{itemdescr}
\pnum
\returns
The value of the Width.
\end{itemdescr}

\indexlibrary{\idxcode{abs_rectangle}!\idxcode{height}}
\begin{itemdecl}
constexpr double height() const noexcept;
\end{itemdecl}
\begin{itemdescr}
\pnum
\returns
The value of the Height.
\end{itemdescr}

\indexlibrary{\idxcode{abs_rectangle}!\idxcode{top_left}}
\begin{itemdecl}
constexpr vector_2d top_left() const noexcept;
\end{itemdecl}
\begin{itemdescr}
\pnum
\returns
A \tcode{vector_2d} object constructed from the value of the X coordinate as its first argument and the value of the Y coordinate as its second argument.
\end{itemdescr}

\indexlibrary{\idxcode{abs_rectangle}!\idxcode{bottom_right}}
\begin{itemdecl}
constexpr vector_2d bottom_right() const noexcept;
\end{itemdecl}
\begin{itemdescr}
\pnum
\returns
A \tcode{vector_2d} object constructed from the value of the Width added to the value of the X coordinate as its first argument and the value of the Height added to the value of the Y coordinate as its second argument.
\end{itemdescr}

%!TEX root = io2d.tex
\rSec0 [arcclockwise] {Class \tcode{arc_clockwise}}

\pnum
\indexlibrary{\idxcode{arc_clockwise}}
The class \tcode{arc_clockwise} describes a path segment that is a circular arc with clockwise rotation.

\pnum
It has a Circle of type \tcode{circle}, a First Angle of type \tcode{double}, and a Second Angle of type \tcode{double}.

\pnum
The values for the First Angle and Second Angle are in radians.

\pnum
\begin{note}
Although the value of the First Angle may be greater than the value of the Second Angle, when processed as described in \ref{paths.processing}, \tcode{two_pi<double>} is added to the Second Angle until the value of the Second Angle is greater than or equal to the value of the First Angle.
\end{note}

\rSec1 [arcclockwise.synopsis] {\tcode{arc_clockwise} synopsis}

\begin{codeblock}
namespace std { namespace experimental { namespace io2d { inline namespace v1 {
  namespace path_data {
    class arc_clockwise {
    public:
      // \ref{arcclockwise.cons}, construct/copy/move/destroy:
      constexpr arc_clockwise() noexcept;
      constexpr arc_clockwise(const experimental::io2d::circle& c,
        double angle1, double angle2) noexcept;
      constexpr arc_clockwise(const vector_2d& ctr, double rad,
        double angle1, double angle2) noexcept;

      // \ref{arcclockwise.modifiers}, modifiers:
      constexpr void circle(const experimental::io2d::circle& c) noexcept;
      constexpr void center(const vector_2d& ctr) noexcept;
      constexpr void radius(double r) noexcept;
      constexpr void angle_1(double radians) noexcept;
      constexpr void angle_2(double radians) noexcept;

      // \ref{arcclockwise.observers}, observers:
      constexpr experimental::io2d::circle circle() const noexcept;
      constexpr vector_2d center() const noexcept;
      constexpr double radius() const noexcept;
      constexpr double angle_1() const noexcept;
      constexpr double angle_2() const noexcept;
    };
  };
} } } }
\end{codeblock}

\rSec1 [arcclockwise.cons] {\tcode{arc_clockwise} constructors and assignment operators}

\indexlibrary{\idxcode{arc_clockwise}!constructor}
\begin{itemdecl}
constexpr arc_clockwise() noexcept;
\end{itemdecl}
\begin{itemdescr}
\pnum
\effects
Constructs an object of type \tcode{arc_clockwise}.

\pnum
The Circle shall be set to the value of \tcode{experimental::io2d::circle\{ \}}.

\pnum
The First Angle shall be set to the value of \tcode{0.0}.

\pnum
The Second Angle shall be set to the value of \tcode{0.0}.
\end{itemdescr}

\indexlibrary{\idxcode{arc_clockwise}!constructor}
\begin{itemdecl}
constexpr arc_clockwise(const experimental::io2d::circle& c, double angle1,
  double angle2) noexcept;
\end{itemdecl}
\begin{itemdescr}
\pnum
\effects
Constructs an object of type \tcode{arc_clockwise}.

\pnum
The Circle shall be set to the value of \tcode{c}.

\pnum
The First Angle shall be set to the value of \tcode{angle1}.

\pnum
The Second Angle shall be set to the value of \tcode{angle2}.
\end{itemdescr}

\indexlibrary{\idxcode{arc_clockwise}!constructor}
\begin{itemdecl}
constexpr arc_clockwise(const vector_2d& ctr, double rad, double angle1,
  double angle2) noexcept;
\end{itemdecl}
\begin{itemdescr}
\pnum
\effects
Constructs an object of type \tcode{arc_clockwise}.

\pnum
The Circle's Center (\ref{circle.intro}) shall be set to the value of \tcode{ctr}.

\pnum
The Circle's Radius (\ref{circle.intro}) shall be set to the value of \tcode{rad}.

\pnum
The First Angle shall be set to the value of \tcode{angle1}.

\pnum
The Second Angle shall be set to the value of \tcode{angle2}.
\end{itemdescr}

\rSec1 [arcclockwise.modifiers]{\tcode{arc_clockwise} modifiers}

\indexlibrary{\idxcode{arc_clockwise}!\idxcode{circle}}
\begin{itemdecl}
constexpr void circle(const experimental::io2d::circle& c) noexcept;
\end{itemdecl}
\begin{itemdescr}
\pnum
\effects
The Circle shall be set to the value of \tcode{c}.
\end{itemdescr}

\indexlibrary{\idxcode{arc_clockwise}!\idxcode{center}}
\begin{itemdecl}
constexpr void center(const vector_2d& ctr) noexcept;
\end{itemdecl}
\begin{itemdescr}
\pnum
\effects
The Circle's Center (\ref{circle.intro}) shall be set to the value of \tcode{ctr}.
\end{itemdescr}

\indexlibrary{\idxcode{arc_clockwise}!\idxcode{radius}}
\begin{itemdecl}
constexpr void radius(double r) noexcept;
\end{itemdecl}
\begin{itemdescr}
\pnum
\effects
The Circle's Radius (\ref{circle.intro}) shall be set to the value of \tcode{r}.
\end{itemdescr}

\indexlibrary{\idxcode{arc_clockwise}!\idxcode{angle_1}}
\begin{itemdecl}
constexpr void angle_1(double radians) noexcept;
\end{itemdecl}
\begin{itemdescr}
\pnum
\effects
The First Angle shall be set to the value of \tcode{radians}.
\end{itemdescr}

\indexlibrary{\idxcode{arc_clockwise}!\idxcode{angle_2}}
\begin{itemdecl}
constexpr void angle_2(double radians) noexcept;
\end{itemdecl}
\begin{itemdescr}
\pnum
\effects
The Second Angle shall be set to the value of \tcode{radians}.
\end{itemdescr}

\rSec1 [arcclockwise.observers]{\tcode{arc_clockwise} observers}

\indexlibrary{\idxcode{arc_clockwise}!\idxcode{circle}}
\begin{itemdecl}
constexpr experimental::io2d::circle circle() const noexcept;
\end{itemdecl}
\begin{itemdescr}
\pnum
\returns
The value of the Circle.
\end{itemdescr}

\indexlibrary{\idxcode{arc_clockwise}!\idxcode{center}}
\begin{itemdecl}
constexpr vector_2d center() const noexcept;
\end{itemdecl}
\begin{itemdescr}
\pnum
\returns
The value of the Circle's Center (\ref{circle.intro}).
\end{itemdescr}

\indexlibrary{\idxcode{arc_clockwise}!\idxcode{radius}}
\begin{itemdecl}
constexpr double radius() const noexcept;
\end{itemdecl}
\begin{itemdescr}
\pnum
\returns
The value of the Circle's Radius (\ref{circle.intro}).
\end{itemdescr}

\indexlibrary{\idxcode{arc_clockwise}!\idxcode{angle_1}}
\begin{itemdecl}
constexpr double angle_1() const noexcept;
\end{itemdecl}
\begin{itemdescr}
\pnum
\returns
The value of the First Angle.
\end{itemdescr}

\indexlibrary{\idxcode{arc_clockwise}!\idxcode{angle_2}}
\begin{itemdecl}
constexpr double angle_2() const noexcept;
\end{itemdecl}
\begin{itemdescr}
\pnum
\returns
The value of the Second Angle.
\end{itemdescr}

%!TEX root = io2d.tex
\rSec0 [arccounterclockwise] {Class \tcode{arc_counterclockwise}}

\pnum
\indexlibrary{\idxcode{arc_counterclockwise}}
The class \tcode{arc_counterclockwise} describes a path segment that is a circular arc with counterclockwise rotation.

\pnum
It has a Circle of type \tcode{circle}, a First Angle of type \tcode{double}, and a Second Angle of type \tcode{double}.

\pnum
The values for the First Angle and Second Angle are in radians.

\pnum
\enternote
Although the value of the Second Angle may be greater than the value of the First Angle, when processed as described in \ref{paths.processing}, \tcode{two_pi<double>} is subtracted from the Second Angle until the value of the First Angle is greater than or equal to the value of the Second Angle.
\exitnote

\rSec1 [arccounterclockwise.synopsis] {\tcode{arc_counterclockwise} synopsis}

\begin{codeblock}
namespace std { namespace experimental { namespace io2d { inline namespace v1 {
  namespace path_data {
    class arc_counterclockwise {
    public:
      // \ref{arccounterclockwise.cons}, construct:
      constexpr arc_counterclockwise() noexcept;
      constexpr arc_counterclockwise(const experimental::io2d::circle& c,
        double angle1, double angle2) noexcept;
      constexpr arc_counterclockwise(const vector_2d& ctr, double rad,
        double angle1, double angle2) noexcept;
      constexpr arc_counterclockwise(const arc_clockwise&) noexcept = default;
      constexpr arc_counterclockwise& operator=(const arc_clockwise&&)
        noexcept = default;
      arc_counterclockwise(arc_counterclockwise&&) noexcept = default;
      arc_counterclockwise& operator=(arc_counterclockwise&&)
        noexcept = default;

      // \ref{arccounterclockwise.modifiers}, modifiers:
      void circle(const experimental::io2d::circle& c) noexcept;
      void center(const vector_2d& ctr) noexcept;
      void radius(double r) noexcept;
      void angle_1(double radians) noexcept;
      void angle_2(double radians) noexcept;

      // \ref{arccounterclockwise.observers}, observers:
      constexpr experimental::io2d::circle circle() const noexcept;
      constexpr vector_2d center() const noexcept;
      constexpr double radius() const noexcept;
      constexpr double angle_1() const noexcept;
      constexpr double angle_2() const noexcept;
    };
  };
} } } }
\end{codeblock}

\rSec1 [arccounterclockwise.cons] {\tcode{arc_counterclockwise} constructors and assignment operators}

\indexlibrary{\idxcode{arc_counterclockwise}!constructor}
\begin{itemdecl}
constexpr arc_counterclockwise() noexcept;
\end{itemdecl}
\begin{itemdescr}
\pnum
\effects
Constructs an object of type \tcode{arc_counterclockwise}.

\pnum
The Circle shall be set to the value of \tcode{experimental::io2d::circle\{ \}}.

\pnum
The First Angle shall be set to the value of \tcode{0.0}.

\pnum
The Second Angle shall be set to the value of \tcode{0.0}.
\end{itemdescr}

\indexlibrary{\idxcode{arc_counterclockwise}!constructor}
\begin{itemdecl}
constexpr arc_counterclockwise(const experimental::io2d::circle& c,
  double angle1, double angle2) noexcept;
\end{itemdecl}
\begin{itemdescr}
\pnum
\effects
Constructs an object of type \tcode{arc_counterclockwise}.
arc_counterclockwise
\pnum
The Circle shall be set to the value of \tcode{c}.

\pnum
The First Angle shall be set to the value of \tcode{angle1}.

\pnum
The Second Angle shall be set to the value of \tcode{angle2}.
\end{itemdescr}

\indexlibrary{\idxcode{arc_counterclockwise}!constructor}
\begin{itemdecl}
constexpr arc_counterclockwise(const vector_2d& ctr, double rad, double angle1,
  double angle2) noexcept;
\end{itemdecl}
\begin{itemdescr}
\pnum
\effects
Constructs an object of type \tcode{arc_counterclockwise}.

\pnum
The Circle's Center (\ref{circle.intro}) shall be set to the value of \tcode{ctr}.

\pnum
The Circle's Radius (\ref{circle.intro}) shall be set to the value of \tcode{rad}.

\pnum
The First Angle shall be set to the value of \tcode{angle1}.

\pnum
The Second Angle shall be set to the value of \tcode{angle2}.
\end{itemdescr}

\rSec1 [arccounterclockwise.modifiers]{\tcode{arc_counterclockwise} modifiers}

\indexlibrary{\idxcode{arc_counterclockwise}!\idxcode{circle}}
\indexlibrary{\idxcode{circle}!\idxcode{arc_counterclockwise}}
\begin{itemdecl}
void circle(const experimental::io2d::circle& c) noexcept;
\end{itemdecl}
\begin{itemdescr}
\pnum
\effects
The Circle shall be set to the value of \tcode{c}.
\end{itemdescr}

\indexlibrary{\idxcode{arc_counterclockwise}!\idxcode{center}}
\indexlibrary{\idxcode{center}!\idxcode{arc_counterclockwise}}
\begin{itemdecl}
void center(const vector_2d& ctr) noexcept;
\end{itemdecl}
\begin{itemdescr}
\pnum
\effects
The Circle's Center (\ref{circle.intro}) shall be set to the value of \tcode{ctr}.
\end{itemdescr}

\indexlibrary{\idxcode{arc_counterclockwise}!\idxcode{radius}}
\indexlibrary{\idxcode{radius}!\idxcode{arc_counterclockwise}}
\begin{itemdecl}
void radius(double r) noexcept;
\end{itemdecl}
\begin{itemdescr}
\pnum
\effects
The Circle's Radius (\ref{circle.intro}) shall be set to the value of \tcode{r}.
\end{itemdescr}

\indexlibrary{\idxcode{arc_counterclockwise}!\idxcode{angle_1}}
\indexlibrary{\idxcode{angle_1}!\idxcode{arc_counterclockwise}}
\begin{itemdecl}
void angle_1(double radians) noexcept;
\end{itemdecl}
\begin{itemdescr}
\pnum
\effects
The First Angle shall be set to the value of \tcode{radians}.
\end{itemdescr}

\indexlibrary{\idxcode{arc_counterclockwise}!\idxcode{angle_2}}
\indexlibrary{\idxcode{angle_2}!\idxcode{arc_counterclockwise}}
\begin{itemdecl}
void angle_2(double radians) noexcept;
\end{itemdecl}
\begin{itemdescr}
\pnum
\effects
The Second Angle shall be set to the value of \tcode{radians}.
\end{itemdescr}

\rSec1 [arccounterclockwise.observers]{\tcode{arc_counterclockwise} observers}

\indexlibrary{\idxcode{arc_counterclockwise}!\idxcode{circle}}
\indexlibrary{\idxcode{circle}!\idxcode{arc_counterclockwise}}
\begin{itemdecl}
constexpr experimental::io2d::circle circl() const noexcept;
\end{itemdecl}
\begin{itemdescr}
\pnum
\returns
The value of the Circle.
\end{itemdescr}

\indexlibrary{\idxcode{arc_counterclockwise}!\idxcode{center}}
\indexlibrary{\idxcode{center}!\idxcode{arc_counterclockwise}}
\begin{itemdecl}
constexpr vector_2d center() const noexcept;
\end{itemdecl}
\begin{itemdescr}
\pnum
\returns
The value of the Circle's Center (\ref{circle.intro}).
\end{itemdescr}

\indexlibrary{\idxcode{arc_counterclockwise}!\idxcode{radius}}
\indexlibrary{\idxcode{radius}!\idxcode{arc_counterclockwise}}
\begin{itemdecl}
constexpr double radius() const noexcept;
\end{itemdecl}
\begin{itemdescr}
\pnum
\returns
The value of the Circle's Radius (\ref{circle.intro}).
\end{itemdescr}

\indexlibrary{\idxcode{arc_counterclockwise}!\idxcode{angle_1}}
\indexlibrary{\idxcode{angle_1}!\idxcode{arc_counterclockwise}}
\begin{itemdecl}
constexpr double angle_1() const noexcept;
\end{itemdecl}
\begin{itemdescr}
\pnum
\returns
The value of the First Angle.
\end{itemdescr}

\indexlibrary{\idxcode{arc_counterclockwise}!\idxcode{angle_2}}
\indexlibrary{\idxcode{angle_2}!\idxcode{arc_counterclockwise}}
\begin{itemdecl}
constexpr double angle_2() const noexcept;
\end{itemdecl}
\begin{itemdescr}
\pnum
\returns
The value of the Second Angle.
\end{itemdescr}

%!TEX root = io2d.tex
\rSec0 [pathfactory.pathchangematrix] {Class \tcode{path_factory::path_change_matrix}}

\rSec1 [pathfactory.pathchangematrix.synopsis] {\tcode{path_factory::path_change_matrix} synopsis}

\pnum
\indexlibrary{\idxcode{path_factory::path_change_matrix}}
The class \tcode{path_factory::path_change_matrix} describes an operation on a path group.

\pnum
This operation changes the transformation matrix for a path group to be the value returned by \tcode{*this.matrix()}. As shown in \ref{paths.processing}, the new transformation matrix does not affect any operations that came before this operation. It is only used in processing operations that come after it. It continues to be used until another \tcode{path_factory::path_change_matrix} object is encountered or the end of the path group is reached.

\begin{codeblock}
namespace std { namespace experimental { namespace io2d { inline namespace v1 {
  class path_factory::path_change_matrix {
  public:
    // \ref{pathfactory.pathchangematrix.cons}, construct/copy/move/destroy:
    change_matrix() noexcept;
    change_matrix(const change_matrix&) noexcept;
    path_factory::path_change_matrix& operator=(const change_matrix&) noexcept;
    change_matrix(change_matrix&&) noexcept;
    path_factory::path_change_matrix& operator=(change_matrix&&) noexcept;
    explicit change_matrix(const matrix_2d& m) noexcept;

    // \ref{pathfactory.pathchangematrix.modifiers}, modifiers:
    void matrix(const matrix_2d& value) noexcept;

    // \ref{pathfactory.pathchangematrix.observers}, observers:
    matrix_2d matrix() const noexcept;
    virtual path_data_type type() const noexcept override;
    
  private:
    matrix_2d _Matrix; // \expos
  };
} } } }
\end{codeblock}

\rSec1 [pathfactory.pathchangematrix.cons] {\tcode{path_factory::path_change_matrix} constructors and assignment operators}

\indexlibrary{\idxcode{path_factory::path_change_matrix}!constructor}
\begin{itemdecl}
    change_matrix() noexcept;
\end{itemdecl}
\begin{itemdescr}
	\pnum
	\effects
	Constructs an object of type \tcode{path_factory::path_change_matrix}.
	
	\pnum
	\postconditions
	\tcode{_Matrix == matrix_2d\{\}}.
\end{itemdescr}

\indexlibrary{\idxcode{path_factory::path_change_matrix}!constructor}
\begin{itemdecl}
    explicit change_matrix(const matrix_2d& m) noexcept;
\end{itemdecl}
\begin{itemdescr}
	\pnum
	\effects
	Constructs an object of type \tcode{path_factory::path_change_matrix}.
	
	\pnum
	\postconditions
	\tcode{_Matrix == m}.
\end{itemdescr}

\rSec1 [pathfactory.pathchangematrix.modifiers]{\tcode{path_factory::path_change_matrix} modifiers}

\indexlibrary{\idxcode{path_factory::path_change_matrix}!\idxcode{matrix}}
\indexlibrary{\idxcode{matrix}!\idxcode{path_factory::path_change_matrix}}
\begin{itemdecl}
    void matrix(const matrix_2d& value) noexcept;
\end{itemdecl}
\begin{itemdescr}
	\pnum
	\postconditions
	\tcode{_Matrix == value}.
\end{itemdescr}

\rSec1 [pathfactory.pathchangematrix.observers]{\tcode{path_factory::path_change_matrix} observers}

\indexlibrary{\idxcode{path_factory::path_change_matrix}!\idxcode{matrix}}
\indexlibrary{\idxcode{matrix}!\idxcode{path_factory::path_change_matrix}}
\begin{itemdecl}
    matrix_2d matrix() const noexcept;
\end{itemdecl}
\begin{itemdescr}
	\pnum
	\returns
	\tcode{_Matrix}.
\end{itemdescr}

\indexlibrary{\idxcode{path_factory::path_change_matrix}!\idxcode{type}}
\indexlibrary{\idxcode{type}!\idxcode{path_factory::path_change_matrix}}
\begin{itemdecl}
    virtual path_data_type type() const noexcept override;
\end{itemdecl}
\begin{itemdescr}
	\pnum
	\returns
	\tcode{path_data_type::change_matrix}.
\end{itemdescr}

%!TEX root = io2d.tex
\rSec0 [pathdataitem.changeorigin] {Class \tcode{path_factory::path_change_origin}}

\pnum
\indexlibrary{\idxcode{path_factory::path_change_origin}}
The class \tcode{path_factory::path_change_origin} describes an operation on a path geometry collection.

\pnum
This operation changes the origin point for a path geometry collection to be the value returned by \tcode{*this.origin()}. As shown in \ref{pathgeometries.processing}, the new origin point does not affect any operations that came before this operation. It is only used in processing operations that come after it. It continues to be used until another \tcode{path_factory::path_change_origin} object is encountered or the end of the path geometry collection is reached.

\rSec1 [pathdataitem.changeorigin.synopsis] {\tcode{path_factory::path_change_origin} synopsis}

\begin{codeblock}
namespace std { namespace experimental { namespace io2d { inline namespace v1 {
  class path_factory::path_change_origin {
  public:
    // \ref{pathdataitem.changeorigin.cons}, construct/copy/move/destroy:
    change_origin() noexcept;
    change_origin(const change_origin&) noexcept;
    path_factory::path_change_origin& operator=(const change_origin&) noexcept;
    change_origin(change_origin&&) noexcept;
    path_factory::path_change_origin& operator=(change_origin&&) noexcept;
    explicit change_origin(const vector_2d& pt) noexcept;

    // \ref{pathdataitem.changeorigin.modifiers}, modifiers:
    void origin(const vector_2d& value) noexcept;

    // \ref{pathdataitem.changeorigin.observers}, observers:
    vector_2d origin() const noexcept;
    virtual path_data_type type() const noexcept override;
    
  private:
    vector_2d _Data; // \expos
  };
} } } }
\end{codeblock}

\rSec1 [pathdataitem.changeorigin.cons] {\tcode{path_factory::path_change_origin} constructors and assignment operators}

\indexlibrary{\idxcode{path_factory::path_change_origin}!constructor}
\begin{itemdecl}
    change_origin() noexcept;
\end{itemdecl}
\begin{itemdescr}
	\pnum
	\effects
	Constructs an object of type \tcode{path_factory::path_change_origin}.
	
	\pnum
	\postconditions
	\tcode{_Data == vector_2d(0.0, 0.0)}.
\end{itemdescr}

\indexlibrary{\idxcode{path_factory::path_change_origin}!constructor}
\begin{itemdecl}
    explicit change_origin(const vector_2d& pt) noexcept;
\end{itemdecl}
\begin{itemdescr}
	\pnum
	\effects
	Constructs an object of type \tcode{path_factory::path_change_origin}.
	
	\pnum
	\postconditions
	\tcode{_Data == pt}.
\end{itemdescr}

\rSec1 [pathdataitem.changeorigin.modifiers]{\tcode{path_factory::path_change_origin} modifiers}

\indexlibrary{\idxcode{path_factory::path_change_origin}!\idxcode{origin}}
\indexlibrary{\idxcode{origin}!\idxcode{path_factory::path_change_origin}}
\begin{itemdecl}
    void origin(const vector_2d& value) noexcept;
\end{itemdecl}
\begin{itemdescr}
	\pnum
	\postconditions
	\tcode{_Data == value}.
\end{itemdescr}

\rSec1 [pathdataitem.changeorigin.observers]{\tcode{change_origin} observers}

\indexlibrary{\idxcode{path_factory::path_change_origin}!\idxcode{origin}}
\indexlibrary{\idxcode{origin}!\idxcode{path_factory::path_change_origin}}
\begin{itemdecl}
    vector_2d origin() const noexcept;
\end{itemdecl}
\begin{itemdescr}
	\pnum
	\returns
	\tcode{_Data}.
\end{itemdescr}

\indexlibrary{\idxcode{path_factory::path_move_to}!\idxcode{type}}
\indexlibrary{\idxcode{type}!\idxcode{path_factory::path_move_to}}
\begin{itemdecl}
    virtual path_data_type type() const noexcept override;
\end{itemdecl}
\begin{itemdescr}
	\pnum
	\returns
	\tcode{path_data_type::change_origin}.
\end{itemdescr}

%!TEX root = io2d.tex
\rSec0 [pathfactory.pathclosepath] {Class \tcode{path_factory::path_close_path}}

%\pnum
%\indexlibrary{\idxcode{path_factory::path_close_path}}
%The class \tcode{path_factory::path_close_path} describes a path instruction that affects the interpretation of a path factory's path group. It is described in terms of its effect on the evaluation of the path group. 
%
%\pnum
%If the current point in the path group contains a value. If it does, this instruction creates a line from the current point to the path group's last-move-to point. It then sets the path group's current point and last-move-to point to the value of the previous path geometry's last-move-to point.
%
%\pnum
%If there is no current point, then this operation does nothing.
%\enternote
%Because this operation does nothing if there is no current point, there is no need to track whether or not a path geometry has a valid last-move-to point. This operation is the only operation that uses the last-move-to point and all operations that establish a current point for a path geometry also establish a valid last-move-to point for that path geometry.
%\exitnote
%
\rSec1 [pathfactory.pathclosepath.synopsis] {\tcode{path_factory::path_close_path} synopsis}

\begin{codeblock}
namespace std { namespace experimental { namespace io2d { inline namespace v1 {
  class path_factory::path_close_path {
  };
} } } }
\end{codeblock}

\enternote
This class is a path instruction that contains no data. It exists to enable certain operations within a path group.
\exitnote

%!TEX root = io2d.tex
\rSec0 [pathfactory.pathnewpath] {Class \tcode{path_factory::path_new_path}}

\pnum
\indexlibrary{\idxcode{path_factory::path_new_path}}
The class \tcode{path_factory::path_new_path} describes a path operation that creates a new path and makes the previous path, if any, an open path unless it was closed by \tcode{path_factory::path_close_path}.

\pnum
The new path has no current point.

\rSec1 [pathfactory.pathnewpath.synopsis] {\tcode{path_factory::path_new_path} synopsis}

\begin{codeblock}
namespace std { namespace experimental { namespace io2d { inline namespace v1 {
  class path_factory::path_new_path {
  };
} } } }
\end{codeblock}

%!TEX root = io2d.tex
\rSec0 [\iotwod.relcubiccurve] {Class template \tcode{basic_figure_items<GraphicsSurfaces>::rel_cubic_curve}}

\rSec1 [\iotwod.relcubiccurve.intro] {Overview}

\pnum
\indexlibrary{\idxcode{rel_cubic_curve}}%
The class \tcode{basic_figure_items<GraphicsSurfaces>::rel_cubic_curve} describes a figure item that is a segment.

\pnum
It has a \term{first control point} of type \tcode{basic_point_2d<GraphicsSurfaces::graphics_math_type>}, a \term{second control point} of type \tcode{basic_point_2d<GraphicsSurfaces::graphics_math_type>}, and an \tcode{end point} of type \tcode{basic_point_2d<GraphicsSurfaces::graphics_math_type>}.

\pnum
The data are stored in an object of type \tcode{typename GraphicsSurfaces::paths::rel_cubic_curve_data_type}. It is accessible using the \tcode{data} member functions.

\rSec1 [\iotwod.relcubiccurve.synopsis] {Synopsis}
\begin{codeblock}
namespace std::experimemtal::io2d::v1 {
  template <class GraphicsSurfaces>
  class basic_figure_items<GraphicsSurfaces>::rel_cubic_curve {
  public:
    using graphics_math_type = typename GraphicsSurfaces::graphics_math_type;
    using data_type =
      typename GraphicsSurfaces::paths::rel_cubic_curve_data_type;

    // \ref{\iotwod.relcubiccurve.ctor}, construct:
    rel_cubic_curve();
    rel_cubic_curve(const basic_point_2d<graphics_math_type>& cpt1,
       const basic_point_2d<graphics_math_type>& cpt2,
       const basic_point_2d<graphics_math_type>& ept) noexcept;
    rel_cubic_curve(const rel_cubic_curve& other) = default;
    rel_cubic_curve(rel_cubic_curve&& other) noexcept = default;

    // assign:
    rel_cubic_curve& operator=(const rel_cubic_curve& other) = default;
    rel_cubic_curve& operator=(rel_cubic_curve&& other) noexcept = default;

    // \ref{\iotwod.relcubiccurve.acc}, accessors:
    const data_type& data() const noexcept;
    data_type& data() noexcept;

    // \ref{\iotwod.relcubiccurve.mod}, modifiers:
    void control_pt1(const basic_point_2d<graphics_math_type>& cpt) noexcept;
    void control_pt2(const basic_point_2d<graphics_math_type>& cpt) noexcept;
    void end_pt(const basic_point_2d<graphics_math_type>& ept) noexcept;

    // \ref{\iotwod.relcubiccurve.obs}, observers:
    basic_point_2d<graphics_math_type> control_pt1() const noexcept;
    basic_point_2d<graphics_math_type> control_pt2() const noexcept;
    basic_point_2d<graphics_math_type> end_pt() const noexcept;
  };

  // \ref{\iotwod.relcubiccurve.eq}, equality operators:
  template <class GraphicsSurfaces>
  bool operator==(
    const typename basic_figure_items<GraphicsSurfaces>::rel_cubic_curve& lhs,
    const typename basic_figure_items<GraphicsSurfaces>::rel_cubic_curve& rhs) 
    noexcept;  
  template <class GraphicsSurfaces>
  bool operator!=(
    const typename basic_figure_items<GraphicsSurfaces>::rel_cubic_curve& lhs,
    const typename basic_figure_items<GraphicsSurfaces>::rel_cubic_curve& rhs) 
    noexcept;  
}
\end{codeblock}

\rSec1 [\iotwod.relcubiccurve.ctor] {Constructors}%

\indexlibrary{\idxcode{rel_cubic_curve}!constructor}%
\begin{itemdecl}
rel_cubic_curve() noexcept;
\end{itemdecl}
\begin{itemdescr}
\pnum
\effects
Equivalent to \tcode{rel_cubic_curve\{ basic_point_2d(), basic_point_2d(), basic_point_2d() \}}.
\end{itemdescr}

\indexlibrary{\idxcode{rel_cubic_curve}!constructor}%
\begin{itemdecl}
rel_cubic_curve(const basic_point_2d<typename GraphicsSurfaces::graphics_math_type>& cpt1,
  const basic_point_2d<typename GraphicsSurfaces::graphics_math_type>& cpt2,
  const basic_point_2d<typename GraphicsSurfaces::graphics_math_type>& ept) noexcept;
\end{itemdecl}
\begin{itemdescr}
\pnum
\effects Constructs an object of type \tcode{rel_cubic_curve}.

\pnum
\remarks The first control point is \tcode{cpt1}.

\pnum
\remarks The second control point is \tcode{cpt2}.

\pnum
\remarks The end point is \tcode{ept}.
\end{itemdescr}

\rSec1 [\iotwod.relcubiccurve.acc] {Accessors}%

\indexlibrarymember{data}{rel_cubic_curve}%
\begin{itemdecl}
const data_type& data() const noexcept;
data_type& data() noexcept;
\end{itemdecl}
\begin{itemdescr}
\pnum
\returns A reference to the \tcode{rel_matrix} object's data object (See: \ref{\iotwod.relcubiccurve.intro}).
\end{itemdescr}

\rSec1 [\iotwod.relcubiccurve.mod] {Modifiers}

\indexlibrarymember{control_pt1}{rel_cubic_curve}%
\begin{itemdecl}
void control_pt1(const basic_point_2d<typename
  GraphicsSurfaces::graphics_math_type>& cpt) noexcept;
\end{itemdecl}
\begin{itemdescr}
\pnum
\effects
The first control point is \tcode{cpt}.
\end{itemdescr}

\indexlibrarymember{control_pt2}{rel_cubic_curve}%
\begin{itemdecl}
void control_pt2(const basic_point_2d<typename
  GraphicsSurfaces::graphics_math_type>& cpt) noexcept;
\end{itemdecl}
\begin{itemdescr}
\pnum
\effects
The second control point is \tcode{cpt}.
\end{itemdescr}

\indexlibrarymember{end_pt}{rel_cubic_curve}%
\begin{itemdecl}
void end_pt(const basic_point_2d<typename GraphicsSurfaces::graphics_math_type>& ept) noexcept;
\end{itemdecl}
\begin{itemdescr}
\pnum
\effects
The end point is \tcode{ept}.
\end{itemdescr}

\rSec1 [\iotwod.relcubiccurve.obs] {Observers}

\indexlibrarymember{control_pt1}{rel_cubic_curve}%
\begin{itemdecl}
basic_point_2d<graphics_math_type> control_pt1() const noexcept;
\end{itemdecl}
\begin{itemdescr}
\pnum
\returns The first control point.
\end{itemdescr}

\indexlibrarymember{control_pt2}{rel_cubic_curve}%
\begin{itemdecl}
basic_point_2d<graphics_math_type> control_pt2() const noexcept;
\end{itemdecl}
\begin{itemdescr}
\pnum
\returns The second control point.
\end{itemdescr}

\indexlibrarymember{end_pt}{rel_cubic_curve}%
\begin{itemdecl}
basic_point_2d<graphics_math_type> end_pt() const noexcept;
\end{itemdecl}
\begin{itemdescr}
\pnum
\returns The end point.
\end{itemdescr}

\rSec1 [\iotwod.relcubiccurve.eq] {Equality operators}%

\indexlibrarymember{operator==}{rel_cubic_curve}%
\begin{itemdecl}
template <class GraphicsSurfaces>
bool operator==(
  const typename basic_figure_items<GraphicsSurfaces>::rel_cubic_curve& lhs,
  const typename basic_figure_items<GraphicsSurfaces>::rel_cubic_curve& rhs) 
  noexcept;
\end{itemdecl}
\begin{itemdescr}
\pnum
\returns
\tcode{lhs.control_pt1() == rhs.control_pt1() \&\& lhs.control_pt2() == rhs.control_pt2() \&\& lhs.end_pt() == rhs.end_pt()}.
\end{itemdescr}

\indexlibrarymember{operator!=}{rel_cubic_curve}%
\begin{itemdecl}
template <class GraphicsSurfaces>
bool operator!=(
  const typename basic_figure_items<GraphicsSurfaces>::rel_cubic_curve& lhs,
  const typename basic_figure_items<GraphicsSurfaces>::rel_cubic_curve& rhs) 
  noexcept;
\end{itemdecl}
\begin{itemdescr}
\pnum
\returns
\tcode{lhs.control_pt1() != rhs.control_pt1() || lhs.control_pt2() != rhs.control_pt2() || lhs.end_pt() != rhs.end_pt()}.
\end{itemdescr}

%!TEX root = io2d.tex
\rSec0 [relellipse] {Class \tcode{rel_ellipse}}

\pnum
\indexlibrary{\idxcode{abs_ellipse}}
The class \tcode{rel_ellipse} describes a path instruction that adds an ellipse to the current path.

\pnum
It has a Center of type \tcode{vector_2d}, an X Axis Radius of type \tcode{double}, and a Y Axis Radius of type \tcode{double}.

\rSec1 [relellipse.synopsis] {\tcode{abs_ellipse} synopsis}

\begin{codeblock}
namespace std { namespace experimental { namespace io2d { inline namespace v1 {
  namespace path_data {
    class rel_ellipse {
    public:
      // \ref{relellipse.cons}, construct/copy/move/destroy:
      constexpr rel_ellipse() noexcept;
      constexpr rel_ellipse(const vector_2d& ctr, double x, double y) noexcept;
      constexpr explicit rel_ellipse(const circle& c) noexcept;

      // \ref{relellipse.modifiers}, modifiers:
      constexpr void center(const vector_2d& ctr) noexcept;
      constexpr void x_axis(double rad) noexcept;
      constexpr void y_axis(double rad) noexcept;
    
      // \ref{relellipse.observers}, observers:
      constexpr vector_2d center() const noexcept;
      constexpr double x_axis() const noexcept;
      constexpr double y_axis() const noexcept;
    };
  }
} } } }
\end{codeblock}

\rSec1 [relellipse.cons] {\tcode{rel_ellipse} constructors}

\indexlibrary{\idxcode{rel_ellipse}!constructor}
\begin{itemdecl}
constexpr rel_ellipse() noexcept;
\end{itemdecl}
\begin{itemdescr}
\pnum
\effects
Constructs an object of type \tcode{rel_ellipse}.

\pnum
The value of Center is \tcode{vector_2d\{0,0, 0.0\}}.

\pnum
The value of X Axis Radius is \tcode{0.0}.

\pnum
The value of Y Axis Radius is \tcode{0.0}.
\end{itemdescr}

\indexlibrary{\idxcode{rel_ellipse}!constructor}
\begin{itemdecl}
constexpr rel_ellipse(const vector_2d& ctr, double x, double y) noexcept;
\end{itemdecl}
\begin{itemdescr}
\pnum
\preconditions
\tcode{x >= 0.0}.

\pnum
\tcode{y >= 0.0}.

\pnum
\effects
Constructs an object of type \tcode{rel_ellipse}.

\pnum
The value of Center is \tcode{ctr}.

\pnum
The value of X Axis Radius is \tcode{x}.

\pnum
The value of Y Axis Radius is \tcode{y}.
\end{itemdescr}

\indexlibrary{\idxcode{rel_ellipse}!constructor}
\begin{itemdecl}
constexpr explicit rel_ellipse(const circle& c) noexcept;
\end{itemdecl}
\begin{itemdescr}
\pnum
\preconditions
\tcode{c.radius() >= 0.0}.

\pnum
\effects
Constructs an object of type \tcode{rel_ellipse}.

\pnum
The value of Center is \tcode{c.center()}.

\pnum
The value of X Axis Radius is \tcode{c.radius()}.

\pnum
The value of Y Axis Radius is \tcode{c.radius()}.
\end{itemdescr}

\rSec1 [relellipse.modifiers]{\tcode{rel_ellipse} modifiers}

\indexlibrary{\idxcode{rel_ellipse}!\idxcode{center}}
\begin{itemdecl}
constexpr void center(const vector_2d& ctr) noexcept;
\end{itemdecl}

\begin{itemdescr}
\pnum
\effects
The value of Center is \tcode{ctr}.
\end{itemdescr}

\indexlibrary{\idxcode{rel_ellipse}!\idxcode{x_axis}}
\begin{itemdecl}
constexpr void x_axis(double rad) noexcept;
\end{itemdecl}
\begin{itemdescr}
\preconditions
\tcode{rad >= 0.0}.

\pnum
\effects
The value of X Axis Radius is \tcode{rad}.
\end{itemdescr}

\indexlibrary{\idxcode{rel_ellipse}!\idxcode{y_axis}}
\begin{itemdecl}
constexpr void y_axis(double rad) noexcept;
\end{itemdecl}
\begin{itemdescr}
\preconditions
\tcode{rad >= 0.0}.

\pnum
\effects
The value of Y Axis Radius is \tcode{rad}.
\end{itemdescr}

\rSec1 [relellipse.observers]{\tcode{rel_ellipse} observers}

\indexlibrary{\idxcode{rel_ellipse}!\idxcode{center}}
\begin{itemdecl}
constexpr double center() const noexcept;
\end{itemdecl}
\begin{itemdescr}
\pnum
\returns
The value of Center.
\end{itemdescr}

\indexlibrary{\idxcode{rel_ellipse}!\idxcode{x_axis}}
\begin{itemdecl}
constexpr double x_axis() const noexcept;
\end{itemdecl}
\begin{itemdescr}
\pnum
\returns
The value of X Axis Radius.
\end{itemdescr}

\indexlibrary{\idxcode{rel_ellipse}!\idxcode{y_axis}}
\begin{itemdecl}
constexpr double y_axis() const noexcept;
\end{itemdecl}
\begin{itemdescr}
\pnum
\returns
The value of Y Axis Radius.
\end{itemdescr}

%!TEX root = io2d.tex
\rSec0 [\iotwod.relline] {Class \tcode{rel_line}}

\pnum
\indexlibrary{\idxcode{rel_line}}%
The class \tcode{rel_line} describes a path item that is a path segment.

\pnum
It has an \term{end point} of type \tcode{vector_2d}.

\rSec1 [\iotwod.relline.synopsis] {\tcode{rel_line} synopsis}

\begin{codeblock}
namespace std::experimental::io2d::v1 {
  namespace path_data {
    class rel_line {
    public:
      // \ref{\iotwod.relline.cons}, construct:
      constexpr rel_line() noexcept;
      constexpr explicit rel_line(const vector_2d& pt) noexcept;

      // \ref{\iotwod.relline.modifiers}, modifiers:
      constexpr void to(const vector_2d& pt) noexcept;

      // \ref{\iotwod.relline.observers}, observers:
      constexpr vector_2d to() const noexcept;
    };
    
    // \ref{\iotwod.relline.ops}, operators:
    constexpr bool operator==(const rel_line& lhs, const rel_line& rhs) 
      noexcept;
    constexpr bool operator!=(const rel_line& lhs, const rel_line& rhs) 
      noexcept;
  }
}
\end{codeblock}

\rSec1 [\iotwod.relline.cons] {\tcode{rel_line} constructors}

\indexlibrary{\idxcode{rel_line}!constructor}%
\begin{itemdecl}
constexpr rel_line() noexcept;
\end{itemdecl}
\begin{itemdescr}
\pnum
\effects
Equivalent to: \tcode{rel_line\{ vector_2d() \};}
\end{itemdescr}

\indexlibrary{\idxcode{rel_line}!constructor}%
\begin{itemdecl}
constexpr explicit rel_line(const vector_2d& pt) noexcept;
\end{itemdecl}
\begin{itemdescr}
\pnum
\effects
Constructs an object of type \tcode{rel_line}.

\pnum
The end point is \tcode{pt}.
\end{itemdescr}

\rSec1 [\iotwod.relline.modifiers]{\tcode{rel_line} modifiers}

\indexlibrarymember{rel_line}{to}
\begin{itemdecl}
constexpr void to(const vector_2d& pt) noexcept;
\end{itemdecl}
\begin{itemdescr}
\pnum
\effects
The end point is \tcode{pt}.
\end{itemdescr}

\rSec1 [\iotwod.relline.observers]{\tcode{rel_line} observers}

\indexlibrary{\idxcode{rel_line}!\idxcode{to}}%
\begin{itemdecl}
constexpr vector_2d to() const noexcept;
\end{itemdecl}
\begin{itemdescr}
\pnum
\returns
The end point.
\end{itemdescr}

\rSec1 [\iotwod.relline.ops]{\tcode{rel_line} operators}

\indexlibrarymember{operator==}{rel_line}%
\begin{itemdecl}
constexpr bool operator==(const rel_line& lhs, const rel_line& rhs) noexcept;
\end{itemdecl}
\begin{itemdescr}
\pnum
\returns
\tcode{lhs.to() == rhs.to()}.
\end{itemdescr}

%!TEX root = io2d.tex
\rSec0 [relmove] {Class \tcode{rel_move}}

\pnum
\indexlibrary{\idxcode{rel_move}}
The class \tcode{rel_move} describes a path operation that creates a new path and makes the previous path, if any, an open path unless it was closed by \tcode{close_path}.

\pnum
It has an end point of type \tcode{vector_2d}.

\pnum
Its end point is relative to the most recently established current point.

\pnum
The relative end point is also the start point of the new path and its last-move-to point.

\rSec1 [relmove.synopsis] {\tcode{rel_move} synopsis}

\begin{codeblock}
namespace std { namespace experimental { namespace io2d { inline namespace v1 {
  namespace path_data {
    class rel_move {
    public:
      // \ref{relmove.cons}, construct:
      constexpr rel_move() noexcept;
      constexpr explicit rel_move(const vector_2d& pt) noexcept;
      constexpr rel_move(const rel_move&) noexcept = default;
      constexpr rel_move& operator=(const rel_move&) noexcept = default;
      rel_move(rel_move&&) noexcept = default;
      rel_move& operator=(rel_move&&) noexcept = default;

      // \ref{relmove.modifiers}, modifiers:
      void to(const vector_2d& pt) noexcept;

      // \ref{relmove.observers}, observers:
      vector_2d to() const noexcept;
    };
  };
} } } }
\end{codeblock}

\rSec1 [relmove.cons] {\tcode{rel_move} constructors}

\indexlibrary{\idxcode{rel_move}!constructor}
\begin{itemdecl}
constexpr rel_move() noexcept;
\end{itemdecl}
\begin{itemdescr}
\pnum
\effects
Constructs an object of type \tcode{rel_move}.

\pnum
The end point shall be set to the value \tcode{vector_2d\{0.0, 0.0\}}.
\end{itemdescr}

\indexlibrary{\idxcode{rel_move}!constructor}
\begin{itemdecl}
constexpr explicit rel_move(const vector_2d& pt) noexcept;
\end{itemdecl}
\begin{itemdescr}
\pnum
\effects
Constructs an object of type \tcode{rel_move}.

\pnum
The end point shall be set to the value of \tcode{pt}.
\end{itemdescr}

\rSec1 [relmove.modifiers]{\tcode{rel_move} modifiers}

\indexlibrary{\idxcode{rel_move}!\idxcode{to}}
\indexlibrary{\idxcode{to}!\idxcode{rel_move}}
\begin{itemdecl}
void to(const vector_2d& pt) noexcept;
\end{itemdecl}
\begin{itemdescr}
\pnum
\effects
The end point shall be set to the value of \tcode{pt}.
\end{itemdescr}

\rSec1 [relmove.observers]{\tcode{rel_move} observers}

\indexlibrary{\idxcode{rel_move}!\idxcode{to}}
\indexlibrary{\idxcode{to}!\idxcode{rel_move}}
\begin{itemdecl}
constexpr vector_2d to() const noexcept;
\end{itemdecl}
\begin{itemdescr}
\pnum
\returns
The value of the end point.
\end{itemdescr}

%!TEX root = io2d.tex
\rSec0 [\iotwod.relquadraticcurve] {Class \tcode{rel_quadratic_curve}}

\pnum
\indexlibrary{\idxcode{rel_quadratic_curve}}%
The class \tcode{rel_quadratic_curve} describes a figure item that is a segment.

\pnum
It has a \term{control point} of type \tcode{basic_point_2d} and an \term{end point} of type \tcode{basic_point_2d}.

\rSec1 [\iotwod.relquadraticcurve.cons] {\tcode{rel_quadratic_curve} constructors}

\indexlibrary{\idxcode{rel_quadratic_curve}!constructor}%
\begin{itemdecl}
rel_quadratic_curve() noexcept;
\end{itemdecl}
\begin{itemdescr}
\pnum
\effects
Equivalent to: \tcode{rel_quadratic_curve\{ basic_point_2d(), basic_point_2d() \};}
\end{itemdescr}

\indexlibrary{\idxcode{rel_quadratic_curve}!constructor}%
\begin{itemdecl}
rel_quadratic_curve(const basic_point_2d<typename GraphicsSurfaces::graphics_math_type>& cpt,
  const basic_point_2d<typename GraphicsSurfaces::graphics_math_type>& ept) noexcept;
\end{itemdecl}
\begin{itemdescr}
\pnum
\effects
Constructs an object of type \tcode{rel_quadratic_curve}.

\pnum
The control point is \tcode{cpt}.

\pnum
The end point is \tcode{ept}.
\end{itemdescr}

\indexlibrary{\idxcode{rel_quadratic_curve}!constructor}%
\begin{itemdecl}
rel_quadratic_curve(const rel_quadratic_curve& other);
rel_quadratic_curve(rel_quadratic_curve&& other) noexcept;
\end{itemdecl}
\begin{itemdescr}
\pnum
\effects
Constructs an object of type \tcode{rel_quadratic_curve}. In the second form, other is left in a valid state with an unspecified value.

\pnum
The control point is \tcode{other.control_pt()}.

\pnum
The end point is \tcode{other.end_pt()}.
\end{itemdescr}

\rSec1 [\iotwod.relquadraticcurve.assign] {\tcode{rel_quadratic_curve} assignment operators}

\indexlibrary{\idxcode{rel_quadratic_curve}!assignment}%
\begin{itemdecl}
rel_quadratic_curve& operator=(const rel_quadratic_curve& other);
\end{itemdecl}
\begin{itemdescr}
\pnum
\effects
If \tcode{*this} and \tcode{other} are not the same object, modifies \tcode{*this} such that \tcode{*this.control_pt()} is \tcode{other.control_pt()} and \tcode{*this.end_pt()} is \tcode{other.end_pt()}

\pnum
If \tcode{*this} and \tcode{other} are the same object, the member has no effect.

\pnum
\returns
\tcode{*this}
\end{itemdescr}

\indexlibrary{\idxcode{rel_quadratic_curve}!assignment}%
\begin{itemdecl}
rel_quadratic_curve& operator=(rel_quadratic_curve&& other) noexcept;
\end{itemdecl}
\begin{itemdescr}
\pnum
\effects
<TODO>

\pnum
\returns
\tcode{*this}
\end{itemdescr}

\rSec1 [\iotwod.relquadraticcurve.modifiers]{\tcode{rel_quadratic_curve} modifiers}

\indexlibrarymember{control_pt}{rel_quadratic_curve}%
\begin{itemdecl}
void control_pt(const basic_point_2d<typename GraphicsSurfaces::graphics_math_type>& cpt) noexcept;
\end{itemdecl}
\begin{itemdescr}
\pnum
\effects
The control point is \tcode{cpt}.
\end{itemdescr}

\indexlibrarymember{end_pt}{rel_quadratic_curve}%
\begin{itemdecl}
void end_pt(const basic_point_2d<typename GraphicsSurfaces::graphics_math_type>& ept) noexcept;
\end{itemdecl}
\begin{itemdescr}
\pnum
\effects
The end point is \tcode{ept}.
\end{itemdescr}

\rSec1 [\iotwod.relquadraticcurve.observers]{\tcode{rel_quadratic_curve} observers}

\indexlibrarymember{control_pt}{rel_quadratic_curve}%
\begin{itemdecl}
basic_point_2d<typename GraphicsSurfaces::graphics_math_type> control_pt() const noexcept;
\end{itemdecl}
\begin{itemdescr}
\pnum
\returns
The control point.
\end{itemdescr}

\indexlibrarymember{end_pt}{rel_quadratic_curve}%
\begin{itemdecl}
basic_point_2d<typename GraphicsSurfaces::graphics_math_type> end_pt() const noexcept;
\end{itemdecl}
\begin{itemdescr}
\pnum
\returns
The end point.
\end{itemdescr}

\rSec1 [\iotwod.relquadraticcurve.ops]{\tcode{rel_quadratic_curve} operators}

\indexlibrarymember{operator==}{rel_quadratic_curve}%
\begin{itemdecl}
template <class GraphicsSurfaces>
bool operator==(const typename basic_figure_items<GraphicsSurfaces>::rel_quadratic_curve& lhs,
  const typename basic_figure_items<GraphicsSurfaces>::rel_quadratic_curve& rhs) noexcept;
\end{itemdecl}
\begin{itemdescr}
\pnum
\returns
\tcode{lhs.control_pt() == rhs.control_pt() \&\& lhs.end_pt() == rhs.end_pt()}.
\end{itemdescr}

%!TEX root = io2d.tex
\rSec0 [relrectangle] {Class \tcode{rel_rectangle}}

\pnum
\indexlibrary{\idxcode{rel_rectangle}}
The class \tcode{rel_rectangle} describes a path instruction that adds a rectangle to the current path.

\pnum
This is a relative path item.

\pnum
It has an X coordinate of type \tcode{double}, a Y coordinate of type \tcode{double}, a Width of type \tcode{double}, and a Height of type \tcode{double}.

\rSec1 [relrectangle.synopsis] {\tcode{rel_rectangle} synopsis}

\begin{codeblock}
namespace std { namespace experimental { namespace io2d { inline namespace v1 {
  namespace path_data {
    class rel_rectangle {
    public:
      // \ref{relrectangle.cons}, constructors:
      constexpr rel_rectangle() noexcept;
      constexpr rel_rectangle(double x, double y, double w, double h) noexcept;
      constexpr rel_rectangle(const vector_2d& tl, const vector_2d& br) 
        noexcept;
      constexpr rel_rectangle(const rectangle& r);

      // \ref{relrectangle.modifiers}, modifiers:
      constexpr void x(double value) noexcept;
      constexpr void y(double value) noexcept;
      constexpr void width(double value) noexcept;
      constexpr void height(double value) noexcept;
      constexpr void top_left(const vector_2d& value) noexcept;
      constexpr void bottom_right(const vector_2d& value) noexcept;
      constexpr void top_left_bottom_right(const vector_2d& tl,
        const vector_2d& br) noexcept;

      // \ref{relrectangle.observers}, observers:
      constexpr double x() const noexcept;
      constexpr double y() const noexcept;
      constexpr double width() const noexcept;
      constexpr double height() const noexcept;
      constexpr double left() const noexcept;
      constexpr double right() const noexcept;
      constexpr double top() const noexcept;
      constexpr double bottom() const noexcept;
      constexpr vector_2d top_left() const noexcept;
      constexpr vector_2d bottom_right() const noexcept;
    };
  }
} } } }
\end{codeblock}

\rSec1 [relrectangle.cons] {\tcode{rel_rectangle} constructors}

\indexlibrary{\idxcode{rel_rectangle}!constructor}
\begin{itemdecl}
constexpr rel_rectangle() noexcept;
\end{itemdecl}
\begin{itemdescr}
\pnum
\effects
Constructs an object of type \tcode{rel_rectangle}.

\pnum
The X coordinate, Y coordinate, Width, and Height shall each be set to the value \tcode{0.0}.
\end{itemdescr}

\indexlibrary{\idxcode{rel_rectangle}!constructor}
\begin{itemdecl}
constexpr rel_rectangle(double x, double y, double w, double h) noexcept;
\end{itemdecl}
\begin{itemdescr}
\pnum
\effects
Constructs an object of type \tcode{rel_rectangle}.

\pnum
The X coordinate shall be set to the value of \tcode{x}.

\pnum
The Y coordinate shall be set to the value of \tcode{y}.

\pnum
The Width shall be set to the value of \tcode{w}.

\pnum
The Height shall be set to the value of \tcode{h}.
\end{itemdescr}

\indexlibrary{\idxcode{rel_rectangle}!constructor}
\begin{itemdecl}
constexpr rel_rectangle(const vector_2d& tl, const vector_2d& br) noexcept;
\end{itemdecl}
\begin{itemdescr}
\pnum
\effects
Constructs an object of type \tcode{rel_rectangle}.

\pnum
The X coordinate shall be set to the value of \tcode{tl.x()}.

\pnum
The Y coordinate shall be set to the value of \tcode{tl.y()}.

\pnum
The Width shall be set to the value of \tcode{max(0.0, br.x() - tl.x())}.

\pnum
The Height shall be set to the value of \tcode{max(0.0, br.y() - tl.y())}.
\end{itemdescr}

\rSec1 [relrectangle.modifiers]{\tcode{rel_rectangle} modifiers}

\indexlibrary{\idxcode{rel_rectangle}!\idxcode{x}}
\begin{itemdecl}
constexpr void x(double val) noexcept;
\end{itemdecl}

\begin{itemdescr}
\pnum
\effects
The X coordinate shall be set to the value of \tcode{val}.
\end{itemdescr}

\indexlibrary{\idxcode{rel_rectangle}!\idxcode{y}}
\begin{itemdecl}
constexpr void y(double value) noexcept;
\end{itemdecl}
\begin{itemdescr}
\pnum
\effects
The Y coordinate shall be set to the value of \tcode{val}.
\end{itemdescr}

\indexlibrary{\idxcode{rel_rectangle}!\idxcode{width}}
\begin{itemdecl}
constexpr void width(double value) noexcept;
\end{itemdecl}
\begin{itemdescr}
\pnum
\effects
The Width shall be set to the value of \tcode{val}.
\end{itemdescr}

\indexlibrary{\idxcode{rel_rectangle}!\idxcode{height}}
\begin{itemdecl}
constexpr void height(double value) noexcept;
\end{itemdecl}
\begin{itemdescr}
\pnum
\effects
The Height shall be set to the value of \tcode{val}.
\end{itemdescr}

\indexlibrary{\idxcode{rel_rectangle}!\idxcode{top_left}}
\begin{itemdecl}
constexpr void top_left(const vector_2d& val) noexcept;
\end{itemdecl}
\begin{itemdescr}
\pnum
\effects
The X coordinate shall be set to the value of \tcode{val.x()}.

\effects
The Y coordinate shall be set to the value of \tcode{val.y()}.
\end{itemdescr}

\indexlibrary{\idxcode{rel_rectangle}!\idxcode{bottom_right}}
\begin{itemdecl}
constexpr void bottom_right(const vector_2d& val) noexcept;
\end{itemdecl}
\begin{itemdescr}
\pnum
\effects
The Width shall be set to the value of \tcode{max(0.0, val.x() - *this.x())}.

\pnum
The Height shall be set to the value of \tcode{max(0.0, value.y() - *this.y())}.
\end{itemdescr}

\rSec1 [relrectangle.observers]{\tcode{rel_rectangle} observers}

\indexlibrary{\idxcode{rel_rectangle}!\idxcode{x}}
\begin{itemdecl}
constexpr double x() const noexcept;
\end{itemdecl}
\begin{itemdescr}
\pnum
\returns
The value of the X coordinate.
\end{itemdescr}

\indexlibrary{\idxcode{rel_rectangle}!\idxcode{y}}
\begin{itemdecl}
constexpr double y() const noexcept;
\end{itemdecl}
\begin{itemdescr}
\pnum
\returns
The value of the Y coordinate.
\end{itemdescr}

\indexlibrary{\idxcode{rel_rectangle}!\idxcode{width}}
\begin{itemdecl}
constexpr double width() const noexcept;
\end{itemdecl}
\begin{itemdescr}
\pnum
\returns
The value of the Width.
\end{itemdescr}

\indexlibrary{\idxcode{rel_rectangle}!\idxcode{height}}
\begin{itemdecl}
constexpr double height() const noexcept;
\end{itemdecl}
\begin{itemdescr}
\pnum
\returns
The value of the Height.
\end{itemdescr}

\indexlibrary{\idxcode{rel_rectangle}!\idxcode{top_left}}
\begin{itemdecl}
constexpr vector_2d top_left() const noexcept;
\end{itemdecl}
\begin{itemdescr}
\pnum
\returns
A \tcode{vector_2d} object constructed from the value of the X coordinate as its first argument and the value of the Y coordinate as its second argument.
\end{itemdescr}

\indexlibrary{\idxcode{rel_rectangle}!\idxcode{bottom_right}}
\begin{itemdecl}
constexpr vector_2d bottom_right() const noexcept;
\end{itemdecl}
\begin{itemdescr}
\pnum
\returns
A \tcode{vector_2d} object constructed from the value of the Width added to the value of the X coordinate as its first argument and the value of the Height added to the value of the Y coordinate as its second argument.
\end{itemdescr}

%!TEX root = io2d.tex
\rSec0 [path] {Class \tcode{path}}
%%%%% Rename path to path_group so that a path group contains paths rather than path geometries. Rework all working accordingly and eliminate "sub path" since it is now just "path".
\pnum
\indexlibrary{\idxcode{path}}
The class \tcode{path} contains a path geometry graphics resource that is usable with a \tcode{surface}-derived object.

\pnum
A \tcode{path} object is constructed from the path geometry collection data of a \tcode{path_factory} object. The path geometries of its path geometry graphics resource are immutable, however its path geometry graphics resource can be changed using copy assignment or move assignment.

\pnum
An \tcode{path} object can be default constructed. Default construction of a \tcode{path} object results in a \tcode{path} object which has a path geometry graphics resource that contains no path geometries.

\pnum
When a \tcode{path} object is set on a \tcode{surface} object using 
\tcode{surface::path}, the geometric paths represented by it can be 
stroked or filled.

%\pnum
%A \tcode{path} object shall be usable with any \tcode{surface} or \tcode{surface}-derived object.
%
\rSec1 [path.synopsis] {\tcode{path} synopsis}

\begin{codeblock}
namespace std { namespace experimental { namespace io2d { inline namespace v1 {
  class path {
    public:
    // \ref{path.cons}, construct/copy/destroy:
    explicit path(const path_factory& pb);
    path(const path_factory& pb, error_code& ec) noexcept;
  };
} } } }
\end{codeblock}

\rSec1 [path.cons] {\tcode{path} constructors and assignment operators}

\indexlibrary{\idxcode{path}!constructor}
\begin{itemdecl}
    explicit path(const path_factory& pb);
    path(const path_factory& pb, error_code& ec) noexcept;
\end{itemdecl}
\begin{itemdescr}
	\pnum
	\effects
	Constructs an object of class \tcode{path}. Implementations shall create a path geometry graphics resource from the path geometries contained in \tcode{pb.data_ref()} as if they followed the procedure set forth in \ref{pathgeometries.processing}.

	\pnum
	\throws
	As specified in Error reporting (\ref{\iotwod.err.report}).

	\pnum
	\remarks
	It is unspecified whether a \tcode{path} object shall require further processing when it is passed as an argument to a \tcode{surface} or \tcode{surface}-derived object.
	
	\pnum
	Implementations should avoid or minimize the need for further processing of a \tcode{path} object after it has been constructed.

	\pnum
	\errors
	\tcode{errc::not_enough_memory} if there was a failure to allocate memory.
	
%	\pnum
%	\tcode{io2d_error::no_current_point} if, when processing the path geometries, an operation was encountered which required a current point and the current path geometry had no current point.
%	
%	\pnum
%	\tcode{io2d_error::invalid_matrix} if, when processing the path geometries, an operation was encountered which required the current transformation matrix to be invertible and the matrix was not invertible.
	
\end{itemdescr}

%!TEX root = io2d.tex
\rSec0 [\iotwod.pathbuilder] {Class \tcode{path_builder}}

\pnum
\indexlibrary{\idxcode{path_builder}}%
The class \tcode{path_builder} is a container that stores and manipulates objects of type \tcode{figure_items::figure_item} from which \tcode{interpreted_path} objects are created.

\pnum
A \tcode{path_builder} is a contiguous container. (See [container.requirements.general] in \cppseventeen.)

\pnum
The collection of \tcode{figure_items::figure_item} objects in a path builder is referred to as its path.

\rSec1 [\iotwod.pathbuilder.synopsis] {\tcode{path_builder} synopsis}%

\begin{codeblock}
namespace std::experimental::io2d::v1 {
  template <class Allocator = allocator<figure_items::figure_item>>
  class path_builder {
  public:
    using value_type = figure_items::figure_item;
    using allocator_type = Allocator;
    using reference = value_type&;
    using const_reference = const value_type&;
    using size_type       = @\impdefx{type of \tcode{path_builder::size_type}}@. // See [container.requirements] in \cppseventeen.
    using difference_type = @\impdefx{type of \tcode{path_builder::size_type}}@. // See [container.requirements] in \cppseventeen.
    using iterator       = @\impdefx{type of \tcode{path_builder::iterator}}@. // See [container.requirements] in \cppseventeen.
    using const_iterator = @\impdefx{type of \tcode{path_builder::const_iterator}}@. // See [container.requirements] in \cppseventeen.
    using reverse_iterator       = std::reverse_iterator<iterator>;
    using const_reverse_iterator = std::reverse_iterator<const_iterator>;
    
    // \ref{\iotwod.pathbuilder.cons}, construct, copy, move, destroy:
    path_builder() noexcept(noexcept(Allocator())) :
      path_builder(Allocator()) { }
    explicit path_builder(const Allocator&) noexcept;
    explicit path_builder(size_type n, const Allocator& = Allocator());
    path_builder(size_type n, const value_type& value,
      const Allocator& = Allocator());
    template <class InputIterator>
    path_builder(InputIterator first, InputIterator last,
      const Allocator& = Allocator());
    path_builder(const path_builder& x);
    path_builder(path_builder&&) noexcept;
    path_builder(const path_builder&, const Allocator&);
    path_builder(path_builder&&, const Allocator&);
    path_builder(initializer_list<value_type>, const Allocator& = Allocator());
    ~path_builder();
    path_builder& operator=(const path_builder& x);
    path_builder& operator=(path_builder&& x)
      noexcept(
      allocator_traits<Allocator>::propagate_on_container_move_assignment::value
      ||
      allocator_traits<Allocator>::is_always_equal::value);
    path_builder& operator=(initializer_list<value_type>);
    template <class InputIterator>
    void assign(InputIterator first, InputIterator last);
    void assign(size_type n, const value_type& u);
    void assign(initializer_list<value_type>);
    allocator_type get_allocator() const noexcept;
    
    // \ref{\iotwod.pathbuilder.iterators}, iterators:
    iterator begin() noexcept;
    const_iterator begin() const noexcept;
    const_iterator cbegin() const noexcept;

    iterator end() noexcept;
    const_iterator end() const noexcept;
    const_iterator cend() const noexcept;
    
    reverse_iterator rbegin() noexcept;
    const_reverse_iterator rbegin() const noexcept;
    const_reverse_iterator crbegin() const noexcept;

    reverse_iterator rend() noexcept;
    const_reverse_iterator rend() const noexcept;
    const_reverse_iterator crend() const noexcept;
    
    // \ref{\iotwod.pathbuilder.capacity}, capacity
    bool empty() const noexcept;
    size_type size() const noexcept;
    size_type max_size() const noexcept;
    size_type capacity() const noexcept;
    void resize(size_type sz);
    void resize(size_type sz, const value_type& c);
    void reserve(size_type n);
    void shrink_to_fit();

    // element access:
    reference operator[](size_type n);
    const_reference operator[](size_type n) const;
    const_reference at(size_type n) const;
    reference at(size_type n);
    reference front();
    const_reference front() const;
    reference back();
    const_reference back() const;

    // \ref{\iotwod.pathbuilder.modifiers}, modifiers:
    void new_figure(point_2d pt) noexcept;
    void rel_new_figure(point_2d pt) noexcept;
    void close_figure() noexcept;
    void matrix(const matrix_2d& m) noexcept;
    void rel_matrix(const matrix_2d& m) noexcept;
    void revert_matrix() noexcept;
    void line(point_2d pt) noexcept;
    void rel_line(point_2d dpt) noexcept;
    void quadratic_curve(point_2d pt0, point_2d pt2)
      noexcept;
    void rel_quadratic_curve(point_2d pt0, point_2d pt2)
      noexcept;
    void cubic_curve(point_2d pt0, point_2d pt1,
      point_2d pt2) noexcept;
    void rel_cubic_curve(point_2d dpt0, point_2d dpt1,
      point_2d dpt2) noexcept;
    void arc(point_2d rad, float rot, float sang = pi<float>)
      noexcept;
    
    template <class... Args>
    reference emplace_back(Args&&... args);
    void push_back(const value_type& x);
    void push_back(value_type&& x);
    void pop_back();
    template <class... Args>
    iterator emplace(const_iterator position, Args&&... args);
    iterator insert(const_iterator position, const value_type& x);
    iterator insert(const_iterator position, value_type&& x);
    iterator insert(const_iterator position, size_type n, const value_type& x);
    template <class InputIterator>
    iterator insert(const_iterator position, InputIterator first,
      InputIterator last);
    iterator insert(const_iterator position,
      initializer_list<value_type> il);
    iterator erase(const_iterator position);
    iterator erase(const_iterator first, const_iterator last);
    void swap(path_builder&)
      noexcept(allocator_traits<Allocator>::propagate_on_container_swap::value 
        || allocator_traits<Allocator>::is_always_equal::value);
    void clear() noexcept;
  };
  
  template <class Allocator>
  bool operator==(const path_builder<Allocator>& lhs, 
    const path_builder<Allocator>& rhs);
  template <class Allocator>
  bool operator!=(const path_builder<Allocator>& lhs, 
    const path_builder<Allocator>& rhs);
  
  // \ref{\iotwod.pathbuilder.special}, specialized algorithms:
  template <class Allocator>
  void swap(path_builder<Allocator>& lhs, path_builder<Allocator>& rhs)
    noexcept(noexcept(lhs.swap(rhs)));
}
\end{codeblock}

\rSec1 [\iotwod.pathbuilder.containerrequirements] {\tcode{path_builder} container requirements}

\pnum
This class is a sequence container, as defined in [containers] in \cppseventeen, and all sequence container requirements that apply specifically to \tcode{vector} shall also apply to this class.

\rSec1 [\iotwod.pathbuilder.cons] {\tcode{path_builder} constructors, copy, and assignment}

\indexlibrary{\idxcode{path_builder}!constructor}%
\begin{itemdecl}
explicit path_builder(const Allocator&);
\end{itemdecl}
\begin{itemdescr}
\pnum
\effects
Constructs an empty \tcode{path_builder}, using the specified allocator.

\pnum
\complexity
Constant.
\end{itemdescr}

\indexlibrary{\idxcode{path_builder}!constructor}%
\begin{itemdecl}
explicit path_builder(size_type n, const Allocator& = Allocator());
\end{itemdecl}
\begin{itemdescr}
\pnum
\effects
Constructs a \tcode{path_builder} with \tcode{n} default-inserted elements using the specified allocator.

\pnum
\complexity
Linear in \tcode{n}.
\end{itemdescr}

\indexlibrary{\idxcode{path_builder}!constructor}%
\begin{itemdecl}
path_builder(size_type n, const value_type& value,
  const Allocator& = Allocator());
\end{itemdecl}
\begin{itemdescr}
\pnum
\requires
\tcode{value_type} shall be \tcode{CopyInsertable} into \tcode{*this}.

\pnum
\effects
Constructs a \tcode{path_builder} with n copies of \tcode{value}, using the specified allocator.

\pnum
\complexity
Linear in \tcode{n}.
\end{itemdescr}

\indexlibrary{\idxcode{path_builder}!constructor}%
\begin{itemdecl}
template <class InputIterator>
path_builder(InputIterator first, InputIterator last,
  const Allocator& = Allocator());
\end{itemdecl}
\begin{itemdescr}
\pnum
\effects
Constructs a \tcode{path_builder} equal to the range \range{first}{last}, using the specified allocator.

\pnum
\complexity
Makes only $N$ calls to the copy constructor of \tcode{value_type} (where $N$
is the distance between
\tcode{first}
and
\tcode{last})
and no reallocations if iterators \tcode{first} and \tcode{last} are of forward, bidirectional, or random access categories.
It makes order
\tcode{N}
calls to the copy constructor of
\tcode{value_type}
and order
$\log(N)$
reallocations if they are just input iterators.

\end{itemdescr}

\rSec1 [\iotwod.pathbuilder.capacity] {\tcode{path_builder} capacity}%

\indexlibrarymember{capacity}{path_builder}%
\begin{itemdecl}
size_type capacity() const noexcept;
\end{itemdecl}
\begin{itemdescr}
\pnum
\returns
The total number of elements that the path builder can hold without requiring reallocation.
\end{itemdescr}

\indexlibrarymember{path_builder}{reserve}%
\begin{itemdecl}
void reserve(size_type n);
\end{itemdecl}
\begin{itemdescr}
\pnum
\requires
\tcode{value_type} shall be \tcode{MoveInsertable} into \tcode{*this}.

\pnum
\effects
A directive that informs a path builder of a planned change in size, so that it can manage the storage
allocation accordingly. After \tcode{reserve()}, \tcode{capacity()} is greater or equal to the argument of \tcode{reserve} if
reallocation happens; and equal to the previous value of \tcode{capacity()} otherwise. Reallocation happens
at this point if and only if the current capacity is less than the argument of \tcode{reserve()}. If an exception
is thrown other than by the move constructor of a non-\tcode{CopyInsertable} type, there are no effects.

\pnum
\complexity
It does not change the size of the sequence and takes at most linear time in the size of the
sequence.

\pnum
\throws
\tcode{length_error} if \tcode{n >
max_size()}.\footnote{\tcode{reserve()} uses \tcode{Allocator::allocate()} which
may throw an appropriate exception.}

\pnum
\remarks
Reallocation invalidates all the references, pointers, and iterators
referring to the elements in the sequence.
No reallocation shall take place during insertions that happen
after a call to
\tcode{reserve()}
until the time when an insertion would make the size of the vector
greater than the value of
\tcode{capacity()}.
\end{itemdescr}

\indexlibrarymember{path_builder}{shrink_to_fit}%
\begin{itemdecl}
void shrink_to_fit();
\end{itemdecl}
\begin{itemdescr}
\pnum
\requires
\tcode{value_type} shall be \tcode{MoveInsertable} into \tcode{*this}.

\pnum
\effects
\tcode{shrink_to_fit} is a non-binding request to reduce
\tcode{capacity()} to \tcode{size()}.
\begin{note}
The request is non-binding to allow latitude for
implementation-specific optimizations.
\end{note}
It does not increase \tcode{capacity()}, but may reduce \tcode{capacity()}
by causing reallocation. 
If an exception is thrown other than by the move constructor
of a non-\tcode{CopyInsertable} \tcode{value_type} there are no effects.

\pnum
\complexity Linear in the size of the sequence.

\pnum
\remarks Reallocation invalidates all the references, pointers, and 
iterators referring to the elements in the sequence. If no reallocation 
happens, they remain valid.
\end{itemdescr}

\indexlibrarymember{path_builder}{swap}%
\begin{itemdecl}
void swap(path_builder&)
  noexcept(allocator_traits<Allocator>::propagate_on_container_swap::value ||
  allocator_traits<Allocator>::is_always_equal::value);
\end{itemdecl}
\begin{itemdescr}
\pnum
\effects
Exchanges the contents and
\tcode{capacity()}
of
\tcode{*this}
with that of \tcode{x}.

\pnum
\complexity
Constant time.
\end{itemdescr}

\indexlibrary{path_builder}{resize}%
\begin{itemdecl}
void resize(size_type sz);
\end{itemdecl}
\begin{itemdescr}
\pnum
\effects
If \tcode{sz < size()}, erases the last \tcode{size() - sz} elements
from the sequence. Otherwise, appends \tcode{sz - size()} default-inserted 
elements to the sequence.

\pnum
\requires
\tcode{value_type} shall be
\tcode{MoveInsertable} and \tcode{DefaultInsertable} into \tcode{*this}.

\pnum
\remarks
If an exception is thrown other than by the move constructor of a 
non-\tcode{CopyInsertable}
\tcode{value_type} there are no effects.
\end{itemdescr}

\indexlibrary{path_builder}{resize}%
\begin{itemdecl}
void resize(size_type sz, const value_type& c);
\end{itemdecl}
\begin{itemdescr}
\pnum
\effects
If \tcode{sz < size()}, erases the last \tcode{size() - sz} elements
from the sequence. Otherwise,
appends \tcode{sz - size()} copies of \tcode{c} to the sequence.

\pnum
\requires
\tcode{value_type} shall be \tcode{CopyInsertable} into \tcode{*this}.

\pnum
\remarks
If an exception is thrown there are no effects.
\end{itemdescr}

\rSec1 [\iotwod.pathbuilder.modifiers] {\tcode{path_builder} modifiers}

\indexlibrarymember{path_builder}{new_figure}%
\begin{itemdecl}
void new_figure(point_2d pt) noexcept;
\end{itemdecl}
\begin{itemdescr}
\pnum
\effects
Adds a \tcode{figure_items::figure_item} object constructed from \tcode{figure_items::abs_new_figure(pt)} to the end of the path.
\end{itemdescr}

\indexlibrarymember{path_builder}{rel_new_figure}%
\begin{itemdecl}
void rel_new_figure(point_2d pt) noexcept;
\end{itemdecl}
\begin{itemdescr}
\pnum
\effects
Adds a \tcode{figure_items::figure_item} object constructed from \tcode{figure_items::rel_new_figure(pt)} to the end of the path.
\end{itemdescr}

\indexlibrarymember{path_builder}{close_figure}%
\begin{itemdecl}
void close_figure() noexcept;
\end{itemdecl}
\begin{itemdescr}
\pnum
\requires
The current point contains a value.

\pnum
\effects
Adds a \tcode{figure_items::figure_item} object constructed from \tcode{figure_items::close_figure()} to the end of the path.
\end{itemdescr}

\indexlibrarymember{path_builder}{set_matrix}%
\begin{itemdecl}
void matrix(const matrix_2d& m) noexcept;
\end{itemdecl}
\begin{itemdescr}
\pnum
\requires
The matrix \tcode{m} shall be invertible.

\pnum
\effects
Adds a \tcode{figure_items::figure_item} object constructed from \tcode{(figure_items::abs_matrix(m)} to the end of the path.
\end{itemdescr}

\indexlibrarymember{path_builder}{modify_matrix}%
\begin{itemdecl}
void rel_matrix(const matrix_2d& m) noexcept;
\end{itemdecl}
\begin{itemdescr}
\pnum
\requires
The matrix \tcode{m} shall be invertible.

\pnum
\effects
Adds a \tcode{figure_items::figure_item} object constructed from \tcode{(figure_items::rel_matrix(m)} to the end of the path.
\end{itemdescr}

\indexlibrarymember{path_builder}{revert_matrix}%
\begin{itemdecl}
void revert_matrix() noexcept;
\end{itemdecl}
\begin{itemdescr}
\pnum
\effects
Adds a \tcode{figure_items::figure_item} object constructed from \tcode{(figure_items::revert_matrix()} to the end of the path.
\end{itemdescr}

\indexlibrarymember{path_builder}{line}%
\begin{itemdecl}
void line(point_2d pt) noexcept;
\end{itemdecl}
\begin{itemdescr}
\pnum
Adds a \tcode{figure_items::figure_item} object constructed from \tcode{figure_items::abs_line(pt)} to the end of the path.
\end{itemdescr}

\indexlibrarymember{path_builder}{rel_line}%
\begin{itemdecl}
void rel_line(point_2d dpt) noexcept;
\end{itemdecl}
\begin{itemdescr}
\pnum
\effects
Adds a \tcode{figure_items::figure_item} object constructed from \tcode{figure_items::rel_line(pt)} to the end of the path.
\end{itemdescr}

\indexlibrarymember{path_builder}{quadratic_curve}%
\begin{itemdecl}
void quadratic_curve(point_2d pt0, point_2d pt1) noexcept;
\end{itemdecl}
\begin{itemdescr}
\pnum
\effects
Adds a \tcode{figure_items::figure_item} object constructed from\\ \tcode{figure_items::abs_quadratic_curve(pt0, pt1)} to the end of the path.
\end{itemdescr}

\indexlibrarymember{path_builder}{rel_quadratic_curve}%
\begin{itemdecl}
void rel_quadratic_curve(point_2d dpt0, point_2d dpt1)
  noexcept;
\end{itemdecl}
\begin{itemdescr}
\pnum
\effects
Adds a \tcode{figure_items::figure_item} object constructed from\\ \tcode{figure_items::rel_quadratic_curve(dpt0, dpt1)} to the end of the path.
\end{itemdescr}

\indexlibrarymember{path_builder}{cubic_curve}%
\begin{itemdecl}
void cubic_curve(point_2d pt0, point_2d pt1,
  point_2d pt2) noexcept;
\end{itemdecl}
\begin{itemdescr}
\pnum
\effects
\pnum
Adds a \tcode{figure_items::figure_item} object constructed from \tcode{figure_items::abs_cubic_curve(pt0, pt1, pt2)} to the end of the path.
\end{itemdescr}

\indexlibrarymember{path_builder}{rel_cubic_curve}%
\begin{itemdecl}
void rel_cubic_curve(point_2d dpt0, point_2d dpt1,
  point_2d dpt2) noexcept;
\end{itemdecl}
\begin{itemdescr}
\pnum
\effects
Adds a \tcode{figure_items::figure_item} object constructed from \tcode{figure_items::rel_cubic_curve(dpt0, dpt1, dpt2)} to the end of the path.
\end{itemdescr}

\indexlibrarymember{path_builder}{arc}%
\begin{itemdecl}
void arc(point_2d rad, float rot, float sang) noexcept;
\end{itemdecl}
\begin{itemdescr}
\pnum
\effects
Adds a \tcode{figure_items::figure_item} object constructed from \\ \tcode{figure_items::arc(rad, rot, sang)} to the end of the path.
\end{itemdescr}

\indexlibrarymember{path_builder}{insert}%
\indexlibrarymember{path_builder}{emplace_back}%
\indexlibrarymember{path_builder}{push_back}%
\begin{itemdecl}
iterator insert(const_iterator position, const value_type& x);
iterator insert(const_iterator position, value_type&& x);
iterator insert(const_iterator position, size_type n, const value_type& x);
template <class InputIterator>
iterator insert(const_iterator position, InputIterator first,
  InputIterator last);
iterator insert(const_iterator position, initializer_list<value_type>);
template <class... Args>
reference emplace_back(Args&&... args);
template <class... Args>
iterator emplace(const_iterator position, Args&&... args);
void push_back(const value_type& x);
void push_back(value_type&& x);
\end{itemdecl}

\begin{itemdescr}
\pnum
\remarks
Causes reallocation if the new size is greater than the old capacity.
Reallocation invalidates all the references, pointers, and iterators
referring to the elements in the sequence.
If no reallocation happens, all the iterators and references before the insertion point remain valid.
If an exception is thrown other than by
the copy constructor, move constructor,
assignment operator, or move assignment operator of
\tcode{value_type} or by any \tcode{InputIterator} operation
there are no effects.
If an exception is thrown while inserting a single element at the end and
\tcode{value_type} is \tcode{CopyInsertable} or \tcode{is_nothrow_move_constructible_v<value_type>}
is \tcode{true}, there are no effects.
Otherwise, if an exception is thrown by the move constructor of a non-\tcode{CopyInsertable}
\tcode{value_type}, the effects are unspecified.

\pnum
\complexity
The complexity is linear in the number of elements inserted plus the 
distance to the end of the path builder.
\end{itemdescr}

\indexlibrarymember{erase}{path_builder}%
\indexlibrarymember{pop_back}{path_builder}%
\begin{itemdecl}
iterator erase(const_iterator position);
iterator erase(const_iterator first, const_iterator last);
void pop_back();
\end{itemdecl}

\begin{itemdescr}
\pnum
\effects
Invalidates iterators and references at or after the point of the erase.

\pnum
\complexity
The destructor of \tcode{value_type} is called the number of times equal to 
the number of the elements erased, but the assignment operator
of \tcode{value_type} is called the number of times equal to the number of
elements in the path builder after the erased elements.

\pnum
\throws
Nothing unless an exception is thrown by the copy constructor, move 
constructor, assignment operator, or move assignment operator of
\tcode{value_type}.
\end{itemdescr}

\rSec1 [\iotwod.pathbuilder.iterators] {\tcode{path_builder} iterators}

\indexlibrarymember{begin}{path_builder}%
\indexlibrarymember{cbegin}{path_builder}%
\begin{itemdecl}
iterator begin() noexcept;
const_iterator begin() const noexcept;
const_iterator cbegin() const noexcept;
\end{itemdecl}
\begin{itemdescr}
\pnum
\returns
An iterator referring to the first \tcode{figure_items::figure_item} item in the path.

\pnum
\remarks
Changing a \tcode{figure_items::figure_item} object or otherwise modifying the path in a way that violates the preconditions of that \tcode{figure_items::figure_item} object or of any subsequent \tcode{figure_items::figure_item} object in the path produces undefined behavior when the path is interpreted as described in \ref{\iotwod.paths.interpretation} unless all of the violations are fixed prior to such interpretation.
\end{itemdescr}

\indexlibrarymember{end}{path_builder}%
\indexlibrarymember{cend}{path_builder}%
\begin{itemdecl}
iterator end() noexcept;
const_iterator end() const noexcept;
const_iterator cend() const noexcept;
\end{itemdecl}
\begin{itemdescr}
\pnum
\returns
An iterator which is the past-the-end value.

\pnum
\remarks
Changing a \tcode{figure_items::figure_item} object or otherwise modifying the path in a way that violates the preconditions of that \tcode{figure_items::figure_item} object or of any subsequent \tcode{figure_items::figure_item} object in the path produces undefined behavior when the path is interpreted as described in \ref{\iotwod.paths.interpretation} unless all of the violations are fixed prior to such interpretation.
\end{itemdescr}

\indexlibrarymember{rbegin}{path_builder}%
\indexlibrarymember{crbegin}{path_builder}%
\begin{itemdecl}
reverse_iterator rbegin() noexcept;
const_reverse_iterator rbegin() const noexcept;
const_reverse_iterator crbegin() const noexcept;
\end{itemdecl}
\begin{itemdescr}
\pnum
\returns
An iterator which is semantically equivalent to \tcode{reverse_iterator(end)}.

\pnum
\remarks
Changing a \tcode{figure_items::figure_item} object or otherwise modifying the path in a way that violates the preconditions of that \tcode{figure_items::figure_item} object or of any subsequent \tcode{figure_items::figure_item} object in the path produces undefined behavior when the path is interpreted as described in \ref{\iotwod.paths.interpretation} all of the violations are fixed prior to such interpretation.
\end{itemdescr}

\indexlibrarymember{rend}{path_builder}%
\indexlibrarymember{crend}{path_builder}%
\begin{itemdecl}
reverse_iterator rend() noexcept;
const_reverse_iterator rend() const noexcept;
const_reverse_iterator crend() const noexcept;
\end{itemdecl}
\begin{itemdescr}
\pnum
\returns
An iterator which is semantically equivalent to \tcode{reverse_iterator(begin)}.

\pnum
\remarks
Changing a \tcode{figure_items::figure_item} object or otherwise modifying the path in a way that violates the preconditions of that \tcode{figure_items::figure_item} object or of any subsequent \tcode{figure_items::figure_item} object in the path produces undefined behavior when the path is interpreted as described in \ref{\iotwod.paths.interpretation} unless all of the violations are fixed prior to such interpretation.
\end{itemdescr}

\rSec1[\iotwod.pathbuilder.special] {\tcode{path_builder} specialized algorithms}

\indexlibrary{\idxcode{swap}!\idxcode{path_builder}}%
\begin{itemdecl}
template <class Allocator>
void swap(path_builder<Allocator>& lhs, path_builder<Allocator>& rhs)
  noexcept(noexcept(lhs.swap(rhs)));
\end{itemdecl}
\begin{itemdescr}
\pnum
\effects
As if by \tcode{lhs.swap(rhs)}.
\end{itemdescr}

\addtocounter{SectionDepthBase}{-1}
