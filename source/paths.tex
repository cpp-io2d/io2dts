%!TEX root = io2d.tex

\rSec0 [\iotwod.paths] {Paths}

\rSec1 [\iotwod.paths.overview]{Overview of paths}

\pnum
Paths define geometric objects which can be stroked (Table~\ref{tab:\iotwod.surface.rendering.operations}), filled, masked, and used to define a clip area (See: \ref{\iotwod.clipprops.summary}.

\pnum
A path contains zero or more figures.

\pnum
A figure is composed of at least one segment.

\pnum
A figure may contain degenerate segments. When a path is interpreted (\ref{\iotwod.paths.interpretation}), degenerate segments are removed from figures.
\begin{note}
If a path command exists or is inserted between segments, it's possible that points which might have compared equal will no longer compare equal as a result of interpretation (\ref{\iotwod.paths.interpretation}).
\end{note}

\pnum
Paths provide vector graphics functionality. As such they are particularly useful in situations where an application is intended to run on a variety of platforms whose output devices (\ref{\iotwod.displaysurface.intro}) span a large gamut of sizes, both in terms of measurement units and in terms of a horizontal and vertical pixel count, in that order.

\pnum
An \tcode{interpreted_path} object is an immutable resource wrapper containing a path (\ref{\iotwod.pathgroup}). An \tcode{interpreted_path} object is created by interpreting the path contained in a \tcode{path_builder} object. It can also be default constructed, in which case the \tcode{interpreted_path} object contains no figures.
\begin{note}
\tcode{interpreted_path} objects provide significant optimization opportunities for implementations. Because they are immutable and opaque, they are intended to be used to store a path in the most efficient representation available.
\end{note}

%!TEX root = io2d.tex

% Note: The rSec is included in paths.tex. If added here instead, it would be:

\rSec1 [paths.example]{Path group examples (Informative)}

\rSec2 [\iotwod.paths.example.intro] {Overview}

\pnum
Path groups are composed of zero or more paths. The following examples show the basics of how path groups work in practice.

\pnum
Every example is placed within the following code at the indicated spot. This code is shown here once to avoid repetition:

\begin{codeblock}
#include <experimental/io2d>

using namespace std;
using namespace std::experimental::io2d;

int main() {
  auto imgSfc = make_image_surface(format::argb32, 300, 200);
  brush backBrush{ rgba_color::black() };
  brush foreBrush{ rgba_color::white() };
  render_props aliased{ antialias::none };
  path_builder<> pb{};
  imgSfc.paint(backBrush);
  
  // Example code goes here.

  // Example code ends.
  
  imgSfc.save(filesystem::path("example.png"), image_file_format::png);
  return 0;
}
\end{codeblock}

\rSec2 [\iotwod.paths.examples.one] {Example 1}

\pnum
Example 1 consists of a single path, forming a trapezoid:

\begin{codeblock}
  pb.new_path({ 80.0, 20.0 }); // Begins the path.
  pb.line_to({ 220.0, 20.0 }); // Creates a line from the [80, 20] to [220, 20].
  pb.rel_line_to({ 60.0, 160.0 }); // Line from [220, 20] to
    // [220 + 60, 160 + 20]. The "to" point is relative to the starting point.
  pb.rel_line_to({ -260.0, 0.0 }); // Line from [280, 180] to 
    // [280 - 260, 180 + 0].
  pb.close_path(); // Creates a line from [20, 180] to [80, 20] 
    // (the last-move-to point), which makes this a closed path.
  imgSfc.stroke(foreBrush, pb, nullopt, nullopt, nullopt, aliased);
\end{codeblock}

\begin{importgraphiciotwod}
{Example 1 result}
{fig:pathsexample1}
{pathexample01.png}
\end{importgraphiciotwod}

\FloatBarrier

\rSec2 [\iotwod.paths.examples.two] {Example 2}

\pnum
Example 2 consists of two paths. The first is a rectangular open path (on the left) and the second is a rectangular closed path (on the right):

\begin{codeblock}
  pb.new_path({ 20.0, 20.0 }); // Begin the first path.
  pb.rel_line_to({ 100.0, 0.0 });
  pb.rel_line_to({ 0.0, 160.0 });
  pb.rel_line_to({ -100.0, 0.0 });
  pb.rel_line_to({ 0.0, -160.0 });
  
  pb.new_path({ 180.0, 20.0 }); // End the first path and begin the second path.
  pb.rel_line_to({ 100.0, 0.0 });
  pb.rel_line_to({ 0.0, 160.0 });
  pb.rel_line_to({ -100.0, 0.0 });
  pb.close_path(); // End the second path.
  imgSfc.stroke(foreBrush, pb, nullopt, stroke_props{ 10.0 }, nullopt, alised);
\end{codeblock}

\begin{importgraphiciotwod}
{Example 2 result}
{fig:pathsexample2}
{pathexample02.png}
\end{importgraphiciotwod}

\FloatBarrier

\pnum
The resulting image from example 2 shows the difference between an open path and a closed path. Each path begins and ends at the same point. The difference is that with the closed path, that the rendering of the point where the initial path segment and final path segment meet is controlled by the \tcode{line_join} value in the \tcode{stroke_props} class, which in this case is the default value of \tcode{line_join::miter}. In the open path, the rendering of that point receives no special treatment such that each path segment at that point is rendered using the \tcode{line_cap} value in the \tcode{stroke_props} class, which in this case is the default value of \tcode{line_cap::none}.

\pnum
That difference between rendering as a \tcode{line_join} versus rendering as two \tcode{line_cap}s is what causes the notch to appear in the open path segment. Path segments are rendered such that half of the stroke width is rendered on each side of the point being evaluated. With no line cap, each segment begins and ends exactly at the point specified.

\pnum
So for the open path, the first line begins at \tcode{vector_2d\{ 20.0, 20.0 \}} and the last line ends there. Given the stroke width of \tcode{10.0}, the visible result for the first line is a rectangle with an upper left corner of \tcode{vector_2d\{ 20.0, 15.0 \}} and a lower right corner of \tcode{vector_2d\{ 120.0, 25.0 \}}. The last line appears as a rectangle with an upper left corner of \tcode{vector_2d\{ 15.0, 20.0 \}} and a lower right corner of \tcode{vector_2d\{ 25.0, 180.0 \}}. This produces the appearance of a square gap between \tcode{vector_2d\{ 15.0, 15.0 \}} and \tcode{vector_2d\{20.0, 20.0 \}}.

\pnum
For the closed path, adjusting for the coordinate differences, the rendering facts are the same as for the open path except for one key difference: the point where the first line and last line meet is rendered as a line join rather than two line caps, which, given the default value of \tcode{line_join::miter}, produces a miter, adding that square area to the rendering result.

\rSec2 [\iotwod.paths.examples.three] {Example 3}

\pnum
Example 3 demonstrates open and closed paths each containing either a quadratic curve or a cubic curve.

\begin{codeblock}
pb.new_path({ 20.0, 20.0 });
pb.rel_quadratic_curve_to({ 60.0, 120.0 }, { 60.0, -120.0 });
pb.rel_new_path({ 20.0, 0.0 });
pb.rel_quadratic_curve_to({ 60.0, 120.0 }, { 60.0, -120.0 });
pb.close_path();
pb.new_path({ 20.0, 150.0 });
pb.rel_cubic_curve_to({ 40.0, -120.0 }, { 40.0, 120.0 * 2.0 },
  { 40.0, -120.0 });
pb.rel_new_path({ 20.0, 0.0 });
pb.rel_cubic_curve_to({ 40.0, -120.0 }, { 40.0, 120.0 * 2.0 },
  { 40.0, -120.0 });
pb.close_path();
imgSfc.stroke(foreBrush, pb, nullopt, nullopt, nullopt, aliased);
\end{codeblock}

\begin{importgraphiciotwod}
{Path example 3}
{paths:example3}
{pathexample03.png}
\end{importgraphiciotwod}

\FloatBarrier

\pnum
\begin{note}
\tcode{pb.quadratic_curve_to(\{ 80.0, 140.0 \}, \{ 140.0, 20.0 \});} would be the absolute equivalent of the first curve in example 3.
\end{note}

\rSec2 [\iotwod.paths.examples.four] {Example 4}

\pnum
Example 4 shows how to draw "C++" using paths.

\pnum
For the "C", it is created using an arc. A scaling matrix is used to make it  slightly elliptical. It is also desirable that the arc has a fixed center point, \tcode{vector_2d\{ 85.0, 100.0 \}}. The inverse of the scaling matrix is used in combination with the \tcode{point_for_angle} function to determine the point at which the arc should begin in order to get achieve this fixed center point. The "C" is then stroked.

\pnum
Unlike the "C", which is created using an open path that is stroked, each "+" is created using a closed path that is filled. To avoid filling the "C", \tcode{pb.clear();} is called to empty the container. The first "+" is created using a series of lines and is then filled.

\pnum
Taking advantage of the fact that \tcode{path_builder} is a container, rather than create a brand new path for the second "+", a translation matrix is applied by inserting a \tcode{path_data::change_matrix} path item before the \tcode{path_data::new_path} object in the existing plus, reverting back to the old matrix immediately after the  and then filling it again.

\begin{codeblock}
// Create the "C".
const matrix_2d scl = matrix_2d::init_scale({ 0.9, 1.1 });
auto pt = scl.inverse().transform_point({ 85.0, 100.0 }) +
  point_for_angle(half_pi<double> / 2.0, 50.0);
pb.matrix(scl);
pb.new_path(pt);
pb.arc({ 50.0, 50.0 }, three_pi_over_two<double>, half_pi<double> / 2.0);
imgSfc.stroke(foreBrush, pb, nullopt, stroke_props{ 10.0 });
// Create the first "+".
pb.clear();
pb.new_path({ 130.0, 105.0 });
pb.rel_line_to({ 0.0, -10.0 });
pb.rel_line_to({ 25.0, 0.0 });
pb.rel_line_to({ 0.0, -25.0 });
pb.rel_line_to({ 10.0, 0.0 });
pb.rel_line_to({ 0.0, 25.0 });
pb.rel_line_to({ 25.0, 0.0 });
pb.rel_line_to({ 0.0, 10.0 });
pb.rel_line_to({ -25.0, 0.0 });
pb.rel_line_to({ 0.0, 25.0 });
pb.rel_line_to({ -10.0, 0.0 });
pb.rel_line_to({ 0.0, -25.0 });
pb.close_path();
imgSfc.fill(foreBrush, pb);
// Create the second "+".
pb.insert(pb.begin(), path_data::change_matrix(
  matrix_2d::init_translate({ 80.0, 0.0 })));
// Revert to the old matrix after the path_data::new_path object.
pb.insert(pb.begin() + 2, path_data::revert_matrix());
imgSfc.fill(foreBrush, pb);
\end{codeblock}

\begin{importgraphiciotwod}
{Path example 4}
{paths:example4}
{pathexample04.png}
\end{importgraphiciotwod}

\FloatBarrier
%
%\rSec2 [\iotwod.paths.examples.five] {Example 5}
%
%\pnum
%Example 5 shows the difference between filling a path and then stroking it versus stroking that path then filling it.
%
%\begin{codeblock}
%brush blueBrush{ rgba_color::blue() };
%stroke_props ten{ 10.0 };
%pb.new_path({ 30.0, 30.0 });
%pb.rel_line_to({ 105.0, 0.0 });
%pb.rel_line_to({ 0.0, 140.0 });
%pb.rel_line_to({ -105.0, 0.0 });
%pb.close_path();
%imgSfc.stroke(foreBrush, pb, nullopt, ten);
%imgSfc.fill(blueBrush, pb);
%pb.insert(pb.begin(),
%  path_data::change_matrix(matrix_2d::init_translate({ 135.0, 0.0 })));
%imgSfc.fill(blueBrush, pb);
%imgSfc.stroke(foreBrush, pb, nullopt, ten);
%\end{codeblock}
%
%\begin{importgraphiciotwod}
%{Path example 5}
%{paths:example5}
%{pathexample05.png}
%\end{importgraphiciotwod}
%
%\FloatBarrier
%
%\pnum
%As can be seen, a path is filled exactly to the lines of the path.


\rSec1 [\iotwod.paths.items] {Figure items}

\addtocounter{SectionDepthBase}{2}
%!TEX root = io2d.tex

\rSec0 [\iotwod.paths.figureitems.intro] {Introduction}

\pnum
The nested classes within the class template \tcode{basic_figure_items} describe figure items.

\pnum
A figure begins with an \tcode{abs_new_figure} or \tcode{rel_new_figure} object. A figure ends when:

\begin{itemize}
\item a \tcode{close_figure} object is encountered;
\item a \tcode{abs_new_figure} or \tcode{rel_new_figure} object is encountered; or
\item there are no more figure items in the path.
\end{itemize}

\pnum
The \tcode{basic_path_builder} class is a sequential container that contains a path. It provides a simple interface for building a path but a path can be created using any container that stores \tcode{basic_figure_items::figure_item} objects.

\rSec0 [\iotwod.paths.figureitems.synopsis] {Synopsis}

\begin{codeblock}
namespace std::experimental::io2d::v1 {
  template <class GraphicsSurfaces>
  class basic_figure_items {
  public:
    class abs_new_figure;
    class rel_new_figure;
    class close_figure;
    class abs_matrix;
    class rel_matrix;
    class revert_matrix;
    class abs_cubic_curve;
    class abs_line;
    class abs_quadratic_curve;
    class arc;
    class rel_cubic_curve;
    class rel_line;
    class rel_quadratic_curve;

    using figure_item = variant<abs_cubic_curve, abs_line, abs_matrix,
      abs_new_figure, abs_quadratic_curve, arc, close_figure, rel_cubic_curve, 
      rel_line, rel_matrix, rel_new_figure, rel_quadratic_curve, revert_matrix>;

    class abs_matrix {
	public:
      // \ref{\iotwod.absmatrix.cons}, construct:
      abs_matrix() noexcept;
      explicit abs_matrix(const matrix_2d& m);
      abs_matrix(abs_matrix&& other) noexcept;

      // \ref{\iotwod.absmatrix.assign}, assign:
      abs_matrix& operator=(const abs_matrix& other);
      abs_matrix& operator=(abs_matrix&& other) noexcept;
	
      // \ref{\iotwod.absmatrix.modifiers}, modifiers:
      void matrix(const basic_matrix_2d<typename
        GraphicsSurfaces::graphics_math_type>& m) noexcept;
	
      // \ref{\iotwod.absmatrix.observers}, observers:
      basic_matrix_2d<typename GraphicsSurfaces::graphics_math_type> matrix() const noexcept;
    };

    class rel_matrix {
    public:
      // \ref{\iotwod.relmatrix.cons}, construct:
      rel_matrix() noexcept;
      explicit rel_matrix(const basic_matrix_2d<typename
        GraphicsSurfaces::graphics_math_type>& m) noexcept;
      rel_matrix(const rel_matrix& other);
      rel_matrix(rel_matrix&& other) noexcept;

      // \ref{\iotwod.relmatrix.assign}, assign:
      rel_matrix& operator=(const rel_matrix& other);
      rel_matrix& operator=(rel_matrix&& other) noexcept;

      // \ref{\iotwod.relmatrix.modifiers}, modifiers:
      void matrix(const basic_matrix_2d<typename
        GraphicsSurfaces::graphics_math_type>& m) noexcept;
	
      // \ref{\iotwod.relmatrix.observers}, observers:
      basic_matrix_2d<typename GraphicsSurfaces::graphics_math_type> matrix() const noexcept;
    };

    class revert_matrix {
    public:
      // \ref{\iotwod.revertmatrix.cons}, construct:
      revert_matrix() noexcept;
    };

    class abs_line {
    public:
      // \ref{\iotwod.absline.cons}, construct:
      abs_line() noexcept;
      explicit abs_line(const basic_point_2d<typename
        GraphicsSurfaces::graphics_math_type>& pt) noexcept;
      abs_line(const abs_line& other);
      abs_line(abs_line&& other) noexcept;

      // \ref{\iotwod.absline.assign}, assign:
      abs_line& operator=(const abs_line& other);
      abs_line& operator=(abs_line&& other) noexcept;

      // \ref{\iotwod.absline.modifiers}, modifiers:
      void to(const basic_point_2d<typename GraphicsSurfaces::graphics_math_type>& pt) noexcept;

      // \ref{\iotwod.absline.observers}, observers:
      basic_point_2d<typename GraphicsSurfaces::graphics_math_type> to() const noexcept;
    };

    class rel_line {
    public:
      // \ref{\iotwod.relline.cons}, construct:
      rel_line() noexcept;
      explicit rel_line(const basic_point_2d<typename
        GraphicsSurfaces::graphics_math_type>& pt) noexcept;
      rel_line(const rel_line& other);
      rel_line(rel_line&& other) noexcept;

      // \ref{\iotwod.relline.assign}, assign:
      rel_line& operator=(rel_line&& other) noexcept;
      rel_line& operator=(const rel_line& other);

      // \ref{\iotwod.relline.modifiers}, modifiers:
      void to(const basic_point_2d<typename GraphicsSurfaces::graphics_math_type>& pt) noexcept;

      // \ref{\iotwod.relline.observers}, observers:
      basic_point_2d<typename GraphicsSurfaces::graphics_math_type> to() const noexcept;
    };

    class abs_quadratic_curve {
    public:
      // \ref{\iotwod.absquadraticcurve.cons}, construct:
      abs_quadratic_curve() noexcept;
      abs_quadratic_curve(const basic_point_2d<typename
        GraphicsSurfaces::graphics_math_type>& cpt, const basic_point_2d<typename
        GraphicsSurfaces::graphics_math_type>& ept) noexcept;
      abs_quadratic_curve(const abs_quadratic_curve& other);
      abs_quadratic_curve(abs_quadratic_curve&& other) noexcept;

      // \ref{\iotwod.absquadraticcurve.assign}, assign:
      abs_quadratic_curve& operator=(abs_quadratic_curve&& other) noexcept;
      abs_quadratic_curve& operator=(const abs_quadratic_curve& other);

      // \ref{\iotwod.absquadraticcurve.modifiers}, modifiers:
      void control_pt(const basic_point_2d<typename
        GraphicsSurfaces::graphics_math_type>& cpt) noexcept;
      void end_pt(const basic_point_2d<typename
        GraphicsSurfaces::graphics_math_type>& ept) noexcept;

      // \ref{\iotwod.absquadraticcurve.observers}, observers:
      basic_point_2d<typename GraphicsSurfaces::graphics_math_type> control_pt() const noexcept;
      basic_point_2d<typename GraphicsSurfaces::graphics_math_type> end_pt() const noexcept;
    };

    class rel_quadratic_curve {
    public:
      // \ref{\iotwod.relquadraticcurve.cons}, construct:
      rel_quadratic_curve() noexcept;
      rel_quadratic_curve(const basic_point_2d<typename
        GraphicsSurfaces::graphics_math_type>& cpt, const basic_point_2d<typename
        GraphicsSurfaces::graphics_math_type>& ept) noexcept;
      rel_quadratic_curve(const rel_quadratic_curve& other);
      rel_quadratic_curve(rel_quadratic_curve&& other) noexcept;

      // \ref{\iotwod.relquadraticcurve.assign}, assign:
      rel_quadratic_curve& operator=(rel_quadratic_curve&& other) noexcept;
      rel_quadratic_curve& operator=(const rel_quadratic_curve& other);

      // \ref{\iotwod.relquadraticcurve.modifiers}, modifiers:
      void control_pt(const basic_point_2d<typename
        GraphicsSurfaces::graphics_math_type>& cpt) noexcept;
      void end_pt(const basic_point_2d<typename
        GraphicsSurfaces::graphics_math_type>& ept) noexcept;

      // \ref{\iotwod.relquadraticcurve.observers}, observers:
      basic_point_2d<typename GraphicsSurfaces::graphics_math_type> control_pt() const noexcept;
      basic_point_2d<typename GraphicsSurfaces::graphics_math_type> end_pt() const noexcept;
    };

    class abs_cubic_curve {
    public:
      // \ref{\iotwod.abscubiccurve.cons}, construct:
      abs_cubic_curve() noexcept;
      abs_cubic_curve(const basic_point_2d<typename GraphicsSurfaces::graphics_math_type>& cpt1,
        const basic_point_2d<typename GraphicsSurfaces::graphics_math_type>& cpt2,
        const basic_point_2d<typename GraphicsSurfaces::graphics_math_type>& ept) noexcept;
      abs_cubic_curve(const abs_cubic_curve& other);
      abs_cubic_curve(abs_cubic_curve&& other) noexcept;

      // \ref{\iotwod.abscubiccurve.assign}, assign:
      abs_cubic_curve& operator=(const abs_cubic_curve& other);
      abs_cubic_curve& operator=(abs_cubic_curve&& other) noexcept;

      // \ref{\iotwod.abscubiccurve.modifiers}, modifiers:
      void control_pt1(const basic_point_2d<typename
        GraphicsSurfaces::graphics_math_type>& cpt) noexcept;
      void control_pt2(const basic_point_2d<typename
        GraphicsSurfaces::graphics_math_type>& cpt) noexcept;
      void end_pt(const basic_point_2d<typename
        GraphicsSurfaces::graphics_math_type>& ept) noexcept;

      // \ref{\iotwod.abscubiccurve.observers}, observers:
      basic_point_2d<typename GraphicsSurfaces::graphics_math_type> control_pt1() const noexcept;
      basic_point_2d<typename GraphicsSurfaces::graphics_math_type> control_pt2() const noexcept;
      basic_point_2d<typename GraphicsSurfaces::graphics_math_type> end_pt() const noexcept;
    };

    class rel_cubic_curve {
    public:
      // \ref{\iotwod.relcubiccurve.cons}, construct:
      rel_cubic_curve() noexcept;
      rel_cubic_curve(const basic_point_2d<typename GraphicsSurfaces::graphics_math_type>& cpt1,
      const basic_point_2d<typename GraphicsSurfaces::graphics_math_type>& cpt2,
      const basic_point_2d<typename GraphicsSurfaces::graphics_math_type>& ept) noexcept;
      rel_cubic_curve(const rel_cubic_curve& other);
      rel_cubic_curve(rel_cubic_curve&& other) noexcept;

      // \ref{\iotwod.relcubiccurve.assign}, assign:
      rel_cubic_curve& operator=(const rel_cubic_curve& other);
      rel_cubic_curve& operator=(rel_cubic_curve&& other) noexcept;

      // \ref{\iotwod.relcubiccurve.modifiers}, modifiers:
      void control_pt1(const basic_point_2d<typename
        GraphicsSurfaces::graphics_math_type>& cpt) noexcept;
      void control_pt2(const basic_point_2d<typename
        GraphicsSurfaces::graphics_math_type>& cpt) noexcept;
      void end_pt(const basic_point_2d<typename
        GraphicsSurfaces::graphics_math_type>& ept) noexcept;

      // \ref{\iotwod.relcubiccurve.observers}, observers:
      basic_point_2d<typename GraphicsSurfaces::graphics_math_type> control_pt1() const noexcept;
      basic_point_2d<typename GraphicsSurfaces::graphics_math_type> control_pt2() const noexcept;
      basic_point_2d<typename GraphicsSurfaces::graphics_math_type> end_pt() const noexcept;
    };

    class arc {
    public:
      // \ref{\iotwod.arc.cons}, construct:
      arc() noexcept;
      arc(const basic_point_2d<typename GraphicsSurfaces::graphics_math_type>& rad, float rot, float sang) noexcept;
      arc(const arc& other);
      arc(arc&& other) noexcept;

      // \ref{\iotwod.arc.assign}, assign:
      arc& operator=(const arc& other);
      arc& operator=(arc&& other) noexcept;

      // \ref{\iotwod.arc.modifiers}, modifiers:
      void radius(const basic_point_2d<typename GraphicsSurfaces::graphics_math_type>& rad) noexcept;
      void rotation(float rot) noexcept;
      void start_angle(float sang) noexcept;

      // \ref{\iotwod.arc.observers}, observers:
      basic_point_2d<typename GraphicsSurfaces::graphics_math_type> radius() const noexcept;
      float rotation() const noexcept;
      float start_angle() const noexcept;
      basic_point_2d<typename GraphicsSurfaces::graphics_math_type> center(const basic_point_2d<typename
        GraphicsSurfaces::graphics_math_type>& cpt, const basic_matrix_2d<typename
        GraphicsSurfaces::graphics_math_type>& m = basic_matrix_2d<typename
        GraphicsSurfaces::graphics_math_type>{}) const noexcept;
      basic_point_2d<typename GraphicsSurfaces::graphics_math_type> end_pt(const basic_point_2d<typename
        GraphicsSurfaces::graphics_math_type>& cpt, const basic_matrix_2d<typename
        GraphicsSurfaces::graphics_math_type>& m = basic_matrix_2d<typename
        GraphicsSurfaces::graphics_math_type>{}) const noexcept;
    };

    using figure_item = variant<abs_cubic_curve, abs_line, abs_matrix, abs_new_figure,
      abs_quadratic_curve, arc, close_figure, rel_cubic_curve, rel_line, rel_matrix,
      rel_new_figure, rel_quadratic_curve, revert_matrix>;
  };

  // \ref{\iotwod.absmatrix.ops}, abs_matrix operators:
  template <class GraphicsSurfaces>
  bool operator==(
    const typename basic_figure_items<GraphicsSurfaces>::abs_matrix& lhs,
    const typename basic_figure_items<GraphicsSurfaces>::abs_matrix& rhs) 
    noexcept;
  template <class GraphicsSurfaces>
  bool operator!=(
    const typename basic_figure_items<GraphicsSurfaces>::abs_matrix& lhs,
    const typename basic_figure_items<GraphicsSurfaces>::abs_matrix& rhs) 
    noexcept;

  // \ref{\iotwod.relmatrix.ops}, rel_matrix operators:
  template <class GraphicsSurfaces>
  bool operator==(
    const typename basic_figure_items<GraphicsSurfaces>::rel_matrix& lhs,
    const typename basic_figure_items<GraphicsSurfaces>::rel_matrix& rhs) 
    noexcept;
  template <class GraphicsSurfaces>
  bool operator!=(
    const typename basic_figure_items<GraphicsSurfaces>::rel_matrix& lhs,
    const typename basic_figure_items<GraphicsSurfaces>::rel_matrix& rhs) 
    noexcept;

  // \ref{\iotwod.revertmatrix.ops}, revert_matrix operators:
  template <class GraphicsSurfaces>
  bool operator==(
    const typename basic_figure_items<GraphicsSurfaces>::revert_matrix& lhs,
    const typename basic_figure_items<GraphicsSurfaces>::revert_matrix& rhs) 
    noexcept;
  template <class GraphicsSurfaces>
  bool operator!=(
    const typename basic_figure_items<GraphicsSurfaces>::revert_matrix& lhs,
    const typename basic_figure_items<GraphicsSurfaces>::revert_matrix& rhs) 
    noexcept;

  // \ref{\iotwod.absline.ops}, abs_line operators:
  template <class GraphicsSurfaces>
  bool operator==(
    const typename basic_figure_items<GraphicsSurfaces>::abs_line& lhs,
    const typename basic_figure_items<GraphicsSurfaces>::abs_line& rhs) 
    noexcept;
  template <class GraphicsSurfaces>
  bool operator!=(
    const typename basic_figure_items<GraphicsSurfaces>::abs_line& lhs,
    const typename basic_figure_items<GraphicsSurfaces>::abs_line& rhs) 
    noexcept;

  // \ref{\iotwod.relline.ops}, rel_line operators:
  template <class GraphicsSurfaces>
  bool operator==(
    const typename basic_figure_items<GraphicsSurfaces>::rel_line& lhs,
    const typename basic_figure_items<GraphicsSurfaces>::rel_line& rhs) 
    noexcept;
  template <class GraphicsSurfaces>
  bool operator!=(
    const typename basic_figure_items<GraphicsSurfaces>::rel_line& lhs,
    const typename basic_figure_items<GraphicsSurfaces>::rel_line& rhs) 
    noexcept;

  // \ref{\iotwod.absquadraticcurve.ops}, abs_quadratic_curve operators:
  template <class GraphicsSurfaces>
  bool operator==(const typename
    basic_figure_items<GraphicsSurfaces>::abs_quadratic_curve& lhs,
    const typename basic_figure_items<GraphicsSurfaces>::abs_quadratic_curve& 
    rhs) noexcept;
  template <class GraphicsSurfaces>
  bool operator!=(const typename 
    basic_figure_items<GraphicsSurfaces>::abs_quadratic_curve& lhs,
    const typename basic_figure_items<GraphicsSurfaces>::abs_quadratic_curve& 
    rhs) noexcept;

  // \ref{\iotwod.relquadraticcurve.ops}, rel_quadratic_curve operators:
  template <class GraphicsSurfaces>
  bool operator==(const typename 
    basic_figure_items<GraphicsSurfaces>::rel_quadratic_curve& lhs,
    const typename basic_figure_items<GraphicsSurfaces>::rel_quadratic_curve& 
    rhs) noexcept;
  template <class GraphicsSurfaces>
  bool operator!=(const typename 
    basic_figure_items<GraphicsSurfaces>::rel_quadratic_curve& lhs,
    const typename basic_figure_items<GraphicsSurfaces>::rel_quadratic_curve& 
    rhs) noexcept;

  // \ref{\iotwod.abscubiccurve.ops}, abs_cubic_curve operators:
  template <class GraphicsSurfaces>
  bool operator==(const typename 
    basic_figure_items<GraphicsSurfaces>::abs_cubic_curve& lhs,
    const typename basic_figure_items<GraphicsSurfaces>::abs_cubic_curve& rhs) 
    noexcept;
  template <class GraphicsSurfaces>
  bool operator!=(const typename 
    basic_figure_items<GraphicsSurfaces>::abs_cubic_curve& lhs,
    const typename basic_figure_items<GraphicsSurfaces>::abs_cubic_curve& rhs) 
    noexcept;

  // \ref{\iotwod.relcubiccurve.ops}, rel_cubic_curve operators:
  template <class GraphicsSurfaces>
  bool operator==(const typename 
    basic_figure_items<GraphicsSurfaces>::rel_cubic_curve& lhs,
    const typename basic_figure_items<GraphicsSurfaces>::rel_cubic_curve& rhs) 
    noexcept;
  template <class GraphicsSurfaces>
  bool operator!=(const typename 
    basic_figure_items<GraphicsSurfaces>::rel_cubic_curve& lhs,
    const typename basic_figure_items<GraphicsSurfaces>::rel_cubic_curve& rhs) 
    noexcept;

  // \ref{\iotwod.arc.ops}, arc operators:
  template <class GraphicsSurfaces>
  bool operator==(const typename basic_figure_items<GraphicsSurfaces>::arc& lhs,
    const typename basic_figure_items<GraphicsSurfaces>::arc& rhs) noexcept;
  template <class GraphicsSurfaces>
  bool operator!=(const typename basic_figure_items<GraphicsSurfaces>::arc& lhs,
    const typename basic_figure_items<GraphicsSurfaces>::arc& rhs) noexcept;
}
\end{codeblock}
%!TEX root = io2d.tex
\rSec0 [\iotwod.absnewfigure] {Class \tcode{abs_new_figure}}

\pnum
\indexlibrary{\idxcode{abs_new_figure}}%
The class \tcode{abs_new_figure} describes a figure item that is a new figure command.

\pnum
It has an \term{at point} of type \tcode{point_2d}.

\rSec1 [\iotwod.absnewfigure.synopsis] {\tcode{abs_new_figure} synopsis}%

\begin{codeblock}
namespace std::experimental::io2d::v1 {
  namespace figure_items {
    class abs_new_figure {
    public:
      // \ref{\iotwod.absnewfigure.cons}, construct:
      constexpr abs_new_figure() noexcept;
      constexpr explicit abs_new_figure(point_2d pt) noexcept;

      // \ref{\iotwod.absnewfigure.modifiers}, modifiers:
      constexpr void at(point_2d pt) noexcept;

      // \ref{\iotwod.absnewfigure.observers}, observers:
      constexpr point_2d at() const noexcept;
    };
    
    // \ref{\iotwod.absnewfigure.ops}, operators:
    constexpr bool operator==(const abs_new_figure& lhs, const abs_new_figure& rhs) 
      noexcept;
    constexpr bool operator!=(const abs_new_figure& lhs, const abs_new_figure& rhs) 
      noexcept;
  }
}
\end{codeblock}

\rSec1 [\iotwod.absnewfigure.cons] {\tcode{abs_new_figure} constructors}%

\indexlibrary{\idxcode{abs_new_figure}!constructor}%
\begin{itemdecl}
constexpr abs_new_figure() noexcept;
\end{itemdecl}
\begin{itemdescr}
\pnum
\effects
Equivalent to: \tcode{abs_new_figure\{ point_2d() \};}
\end{itemdescr}

\indexlibrary{\idxcode{abs_new_figure}!constructor}%
\begin{itemdecl}
constexpr explicit abs_new_figure(point_2d pt) noexcept;
\end{itemdecl}
\begin{itemdescr}
\pnum
\effects
Constructs an object of type \tcode{abs_new_figure}.

\pnum
The at point is \tcode{pt}.
\end{itemdescr}

\rSec1 [\iotwod.absnewfigure.modifiers]{\tcode{abs_new_figure} modifiers}%

\indexlibrarymember{at}{abs_new_figure}%
\begin{itemdecl}
constexpr void at(point_2d pt) noexcept;
\end{itemdecl}
\begin{itemdescr}
\pnum
\effects
The at point is \tcode{pt}.
\end{itemdescr}

\rSec1 [\iotwod.absnewfigure.observers]{\tcode{abs_new_figure} observers}%

\indexlibrarymember{at}{abs_new_figure}%
\begin{itemdecl}
constexpr point_2d at() const noexcept;
\end{itemdecl}
\begin{itemdescr}
\pnum
\returns
The at point.
\end{itemdescr}

\rSec1 [\iotwod.absnewfigure.ops]{\tcode{abs_new_figure} operators}%

\indexlibrarymember{operator==}{abs_new_figure}%
\begin{itemdecl}
constexpr bool operator==(const abs_new_figure& lhs, const abs_new_figure& rhs) 
  noexcept;
\end{itemdecl}
\begin{itemdescr}
\pnum
\returns
\tcode{lhs.at() == rhs.at()}.
\end{itemdescr}

%!TEX root = io2d.tex
\rSec0 [\iotwod.relnewpath] {Class \tcode{rel_new_path}}%

\pnum
\indexlibrary{\idxcode{rel_new_path}}%
The class \tcode{rel_new_path} describes a path item that is a new path instruction.

\pnum
It has an \term{at point} of type \tcode{vector_2d}.

\rSec1 [\iotwod.relnewpath.synopsis] {\tcode{rel_new_path} synopsis}%

\begin{codeblock}
namespace std::experimental::io2d::v1 {
  namespace path_data {
    class rel_new_path {
    public:
      // \ref{\iotwod.relnewpath.cons}, construct:
      constexpr rel_new_path() noexcept;
      constexpr explicit rel_new_path(const vector_2d& pt) noexcept;

      // \ref{\iotwod.relnewpath.modifiers}, modifiers:
      constexpr void at(const vector_2d& pt) noexcept;

      // \ref{\iotwod.relnewpath.observers}, observers:
      constexpr vector_2d at() const noexcept;
    };
    
  // \ref{\iotwod.relnewpath.ops}, operators:
  constexpr bool operator==(const rel_new_path& lhs, const rel_new_path& rhs) 
    noexcept;
  constexpr bool operator!=(const rel_new_path& lhs, const rel_new_path& rhs) 
    noexcept;
  }
}
\end{codeblock}

\rSec1 [\iotwod.relnewpath.cons] {\tcode{rel_new_path} constructors}%

\indexlibrary{\idxcode{rel_new_path}!constructor}%
\begin{itemdecl}
constexpr rel_new_path() noexcept;
\end{itemdecl}
\begin{itemdescr}
\pnum
\effects
Equivalent to: \tcode{rel_new_path\{ vector_2d() \};}
\end{itemdescr}

\indexlibrary{\idxcode{rel_new_path}!constructor}%
\begin{itemdecl}
constexpr explicit rel_new_path(const vector_2d& pt) noexcept;
\end{itemdecl}
\begin{itemdescr}
\pnum
\effects
Constructs an object of type \tcode{rel_new_path}.

\pnum
The at point is \tcode{pt}.
\end{itemdescr}

\rSec1 [\iotwod.relnewpath.modifiers]{\tcode{rel_new_path} modifiers}%

\indexlibrarymember{at}{rel_new_path}%
\begin{itemdecl}
constexpr void at(const vector_2d& pt) noexcept;
\end{itemdecl}
\begin{itemdescr}
\pnum
\effects
The at point is \tcode{pt}.
\end{itemdescr}

\rSec1 [\iotwod.relnewpath.observers]{\tcode{rel_new_path} observers}%

\indexlibrarymember{at}{rel_new_path}%
\begin{itemdecl}
constexpr vector_2d at() const noexcept;
\end{itemdecl}
\begin{itemdescr}
\pnum
\returns
The at point.
\end{itemdescr}

\rSec1 [\iotwod.relnewpath.ops]{\tcode{rel_new_path} operators}%

\indexlibrarymember{operator==}{rel_new_path}%
\begin{itemdecl}
constexpr bool operator==(const rel_new_path& lhs, const rel_new_path& rhs) 
  noexcept;
\end{itemdecl}
\begin{itemdescr}
\pnum
\returns
\tcode{lhs.at() == rhs.at()}.
\end{itemdescr}

%!TEX root = io2d.tex
\rSec0 [pathfactory.pathclosepath] {Class \tcode{path_factory::path_close_path}}

%\pnum
%\indexlibrary{\idxcode{path_factory::path_close_path}}
%The class \tcode{path_factory::path_close_path} describes a path instruction that affects the interpretation of a path factory's path group. It is described in terms of its effect on the evaluation of the path group. 
%
%\pnum
%If the current point in the path group contains a value. If it does, this instruction creates a line from the current point to the path group's last-move-to point. It then sets the path group's current point and last-move-to point to the value of the previous path geometry's last-move-to point.
%
%\pnum
%If there is no current point, then this operation does nothing.
%\enternote
%Because this operation does nothing if there is no current point, there is no need to track whether or not a path geometry has a valid last-move-to point. This operation is the only operation that uses the last-move-to point and all operations that establish a current point for a path geometry also establish a valid last-move-to point for that path geometry.
%\exitnote
%
\rSec1 [pathfactory.pathclosepath.synopsis] {\tcode{path_factory::path_close_path} synopsis}

\begin{codeblock}
namespace std { namespace experimental { namespace io2d { inline namespace v1 {
  class path_factory::path_close_path {
  };
} } } }
\end{codeblock}

\enternote
This class is a path instruction that contains no data. It exists to enable certain operations within a path group.
\exitnote

%!TEX root = io2d.tex
\rSec0 [\iotwod.absmatrix] {Class \tcode{abs_matrix}}%

\rSec1 [\iotwod.absmatrix.synopsis] {\tcode{abs_matrix} synopsis}%

\pnum
\indexlibrary{\idxcode{abs_matrix}}%
The class \tcode{abs_matrix} describes a path item that assigns the value of the \tcode{abs_matrix} object's \term{transform matrix} to a path group transformation matrix.

\pnum
The transform matrix is a \tcode{matrix_2d} object.

\begin{codeblock}
namespace std::experimental::io2d::v1 {
  namespace path_data {
    class abs_matrix {
    public:
      // \ref{\iotwod.absmatrix.cons}, construct:
      constexpr abs_matrix() noexcept;
      constexpr explicit abs_matrix(const matrix_2d& m) noexcept;

      // \ref{\iotwod.absmatrix.modifiers}, modifiers:
      constexpr void matrix(const matrix_2d& m) noexcept;

      // \ref{\iotwod.absmatrix.observers}, observers:
      constexpr matrix_2d matrix() const noexcept;
    };
    
    \ref{\iotwod.absmatrix.nonmember}, non-members
    constexpr bool operator==(const abs_matrix& lhs, const abs_matrix& rhs) 
      noexcept;
    constexpr bool operator!=(const abs_matrix& lhs, const abs_matrix& rhs) 
      noexcept;
  }
}
\end{codeblock}

\rSec1 [\iotwod.absmatrix.cons] {\tcode{abs_matrix} constructors}

\indexlibrary{\idxcode{change_matrix}!constructor}%
\begin{itemdecl}
constexpr abs_matrix() noexcept;
\end{itemdecl}
\begin{itemdescr}
\pnum
\effects
Equivalent to: \tcode{abs_matrix\{ matrix_2d() \};}
\end{itemdescr}

\indexlibrary{\idxcode{abs_matrix}!constructor}%
\begin{itemdecl}
constexpr explicit abs_matrix(const matrix_2d& m) noexcept;
\end{itemdecl}
\begin{itemdescr}
\pnum
\requires
\tcode{m.is_invertible()} is \tcode{true}.

\pnum
\effects
Constructs an object of type \tcode{abs_matrix}.

\pnum
The transform matrix is \tcode{m}.
\end{itemdescr}

\rSec1 [\iotwod.absmatrix.modifiers]{\tcode{abs_matrix} modifiers}

\indexlibrarymember{matrix}{abs_matrix}%
\begin{itemdecl}
constexpr void matrix(const matrix_2d& m) noexcept;
\end{itemdecl}
\begin{itemdescr}
\pnum
\requires
\tcode{m.is_invertible()} is \tcode{true}.

\pnum
\effects
The transform matrix is \tcode{m}.
\end{itemdescr}

\rSec1 [\iotwod.absmatrix.observers]{\tcode{abs_matrix} observers}

\indexlibrarymember{matrix}{abs_matrix}%
\begin{itemdecl}
constexpr matrix_2d matrix() const noexcept;
\end{itemdecl}
\begin{itemdescr}
\pnum
\returns
The transform matrix.
\end{itemdescr}

\rSec1 [\iotwod.absmatrix.nonmember]{Non-member functions}

\indexlibrarymember{operator==}{abs_matrix}%
\begin{itemdecl}
constexpr bool operator==(const abs_matrix& lhs, const abs_matrix& rhs) 
  noexcept;
\end{itemdecl}
\begin{itemdescr}
\pnum
\returns
\tcode{lhs.matrix() == rhs.matrix()}.
\end{itemdescr}

\indexlibrarymember{operator!=}{abs_matrix}%
\begin{itemdecl}
constexpr bool operator!=(const abs_matrix& lhs, const abs_matrix& rhs) 
  noexcept;
\end{itemdecl}
\begin{itemdescr}
\pnum
\returns
\tcode{!(lhs == rhs)}.
\end{itemdescr}

%!TEX root = io2d.tex
\rSec0 [\iotwod.relmatrix] {Class \tcode{rel_matrix}}%

\rSec1 [\iotwod.relmatrix.synopsis] {\tcode{rel_matrix} synopsis}%

\pnum
\indexlibrary{\idxcode{rel_matrix}}%
The class \tcode{rel_matrix} describes a path item that assigns the value of a to a path group transformation matrix multiplied by the \tcode{rel_matrix} object's \term{transform matrix} to a path group transformation matrix.

\pnum
The transform matrix is a \tcode{matrix_2d} object.

\begin{codeblock}
namespace std::experimental::io2d::v1 {
  namespace path_data {
    class rel_matrix {
    public:
      // \ref{\iotwod.relmatrix.cons}, construct:
      constexpr rel_matrix() noexcept;
      constexpr explicit rel_matrix(const matrix_2d& m) noexcept;

      // \ref{\iotwod.relmatrix.modifiers}, modifiers:
      constexpr void matrix(const matrix_2d& m) noexcept;

      // \ref{\iotwod.relmatrix.observers}, observers:
      constexpr matrix_2d matrix() const noexcept;
    };
    
    \ref{\iotwod.relmatrix.nonmember}, non-members
    constexpr bool operator==(const rel_matrix& lhs, const rel_matrix& rhs) 
      noexcept;
    constexpr bool operator!=(const rel_matrix& lhs, const rel_matrix& rhs) 
      noexcept;
  }
}
\end{codeblock}

\rSec1 [\iotwod.relmatrix.cons] {\tcode{rel_matrix} constructors}

\indexlibrary{\idxcode{change_matrix}!constructor}%
\begin{itemdecl}
constexpr rel_matrix() noexcept;
\end{itemdecl}
\begin{itemdescr}
\pnum
\effects
Equivalent to: \tcode{rel_matrix\{ matrix_2d() \};}
\end{itemdescr}

\indexlibrary{\idxcode{rel_matrix}!constructor}%
\begin{itemdecl}
constexpr explicit rel_matrix(const matrix_2d& m) noexcept;
\end{itemdecl}
\begin{itemdescr}
\pnum
\requires
\tcode{m.is_invertible()} is \tcode{true}.

\pnum
\effects
Constructs an object of type \tcode{rel_matrix}.

\pnum
The transform matrix is \tcode{m}.
\end{itemdescr}

\rSec1 [\iotwod.relmatrix.modifiers]{\tcode{rel_matrix} modifiers}

\indexlibrarymember{matrix}{rel_matrix}%
\begin{itemdecl}
constexpr void matrix(const matrix_2d& m) noexcept;
\end{itemdecl}
\begin{itemdescr}
\pnum
\requires
\tcode{m.is_invertible()} is \tcode{true}.

\pnum
\effects
The transform matrix is \tcode{m}.
\end{itemdescr}

\rSec1 [\iotwod.relmatrix.observers]{\tcode{rel_matrix} observers}

\indexlibrarymember{matrix}{rel_matrix}%
\begin{itemdecl}
constexpr matrix_2d matrix() const noexcept;
\end{itemdecl}
\begin{itemdescr}
\pnum
\returns
The transform matrix.
\end{itemdescr}

\rSec1 [\iotwod.relmatrix.nonmember]{Non-member functions}

\indexlibrarymember{operator==}{rel_matrix}%
\begin{itemdecl}
constexpr bool operator==(const rel_matrix& lhs, const rel_matrix& rhs) 
  noexcept;
\end{itemdecl}
\begin{itemdescr}
\pnum
\returns
\tcode{lhs.matrix() == rhs.matrix()}.
\end{itemdescr}

\indexlibrarymember{operator!=}{rel_matrix}%
\begin{itemdecl}
constexpr bool operator!=(const rel_matrix& lhs, const rel_matrix& rhs) 
  noexcept;
\end{itemdecl}
\begin{itemdescr}
\pnum
\returns
\tcode{!(lhs == rhs)}.
\end{itemdescr}

%!TEX root = io2d.tex
\rSec0 [\iotwod.revertmatrix] {Class \tcode{revert_matrix}}

\rSec1 [\iotwod.revertmatrix.synopsis] {\tcode{revert_matrix} synopsis}

\pnum
\indexlibrary{\idxcode{revert_matrix}}%
The class \tcode{revert_matrix} describes a figure item that is a path command.

\begin{codeblock}
namespace std::experimental::io2d::v1 {
  namespace figure_items {
    class revert_matrix {
    public:
      // \ref{\iotwod.revertmatrix.cons}, construct:
      constexpr revert_matrix() noexcept;
    };
    
    // \ref{\iotwod.revertmatrix.ops}, operators:
    constexpr bool operator==(const revert_matrix& lhs,
      const revert_matrix& rhs) noexcept;
    constexpr bool operator!=(const revert_matrix& lhs,
      const revert_matrix& rhs) noexcept;
  }
}
\end{codeblock}

\rSec1 [\iotwod.revertmatrix.cons] {\tcode{revert_matrix} constructors}

\indexlibrary{\idxcode{revert_matrix}!constructor}%
\begin{itemdecl}
constexpr revert_matrix() noexcept;
\end{itemdecl}
\begin{itemdescr}
\pnum
\effects
Constructs an object of type \tcode{revert_matrix}.
\end{itemdescr}

\rSec1 [\iotwod.revertmatrix.ops]{\tcode{revert_matrix} operators}

\indexlibrarymember{operator==}{revert_matrix}%
\begin{itemdecl}
constexpr bool operator==(const revert_matrix& lhs, const revert_matrix& rhs) 
  noexcept;
\end{itemdecl}
\begin{itemdescr}
\pnum
\returns
\tcode{true}.
\end{itemdescr}

%!TEX root = io2d.tex
\rSec0 [absline] {Class \tcode{abs_line}}

\pnum
\indexlibrary{\idxcode{abs_line}}
The class \tcode{abs_line} describes a path segment that is a line.

\pnum
It has an end point of type \tcode{vector_2d}.

\rSec1 [absline.synopsis] {\tcode{abs_line} synopsis}

\begin{codeblock}
namespace std { namespace experimental { namespace io2d { inline namespace v1 {
  namespace path_data {
    class abs_line {
    public:
      // \ref{absline.cons}, construct:
      constexpr abs_line() noexcept;
      constexpr explicit abs_line(const vector_2d& pt) noexcept;

      // \ref{absline.modifiers}, modifiers:
      constexpr void to(const vector_2d& pt) noexcept;

      // \ref{absline.observers}, observers:
      constexpr vector_2d to() const noexcept;
    };
  };
} } } }
\end{codeblock}

\rSec1 [absline.cons] {\tcode{abs_line} constructors and assignment operators}

\indexlibrary{\idxcode{abs_line}!constructor}
\begin{itemdecl}
constexpr abs_line() noexcept;
\end{itemdecl}
\begin{itemdescr}
\pnum
\effects
Constructs an object of type \tcode{abs_line}.

\pnum
The end point shall be set to the value of \tcode{vector_2d\{0.0, 0.0\}}.
\end{itemdescr}

\indexlibrary{\idxcode{abs_line}!constructor}
\begin{itemdecl}
constexpr explicit abs_line(const vector_2d& pt) noexcept;
\end{itemdecl}
\begin{itemdescr}
\pnum
\effects
Constructs an object of type \tcode{abs_line}.

\pnum
The end point shall be set to the value of \tcode{pt}.
\end{itemdescr}

\rSec1 [absline.modifiers]{\tcode{abs_line} modifiers}

\indexlibrary{\idxcode{abs_line}!\idxcode{to}}
\begin{itemdecl}
constexpr void to(const vector_2d& pt) noexcept;
\end{itemdecl}
\begin{itemdescr}
\pnum
\effects
The end point shall be set to the value of \tcode{pt}.
\end{itemdescr}

\rSec1 [absline.observers]{\tcode{abs_line} observers}

\indexlibrary{\idxcode{abs_line}!\idxcode{to}}
\begin{itemdecl}
constexpr vector_2d to() const noexcept;
\end{itemdecl}
\begin{itemdescr}
\pnum
\returns
The value of the end point.
\end{itemdescr}

%!TEX root = io2d.tex
\rSec0 [\iotwod.relline] {Class \tcode{rel_line}}

\pnum
\indexlibrary{\idxcode{rel_line}}%
The class \tcode{rel_line} describes a path item that is a path segment.

\pnum
It has an \term{end point} of type \tcode{vector_2d}.

\rSec1 [\iotwod.relline.synopsis] {\tcode{rel_line} synopsis}

\begin{codeblock}
namespace std::experimental::io2d::v1 {
  namespace path_data {
    class rel_line {
    public:
      // \ref{\iotwod.relline.cons}, construct:
      constexpr rel_line() noexcept;
      constexpr explicit rel_line(const vector_2d& pt) noexcept;

      // \ref{\iotwod.relline.modifiers}, modifiers:
      constexpr void to(const vector_2d& pt) noexcept;

      // \ref{\iotwod.relline.observers}, observers:
      constexpr vector_2d to() const noexcept;
    };
    
    // \ref{\iotwod.relline.ops}, operators:
    constexpr bool operator==(const rel_line& lhs, const rel_line& rhs) 
      noexcept;
    constexpr bool operator!=(const rel_line& lhs, const rel_line& rhs) 
      noexcept;
  }
}
\end{codeblock}

\rSec1 [\iotwod.relline.cons] {\tcode{rel_line} constructors}

\indexlibrary{\idxcode{rel_line}!constructor}%
\begin{itemdecl}
constexpr rel_line() noexcept;
\end{itemdecl}
\begin{itemdescr}
\pnum
\effects
Equivalent to: \tcode{rel_line\{ vector_2d() \};}
\end{itemdescr}

\indexlibrary{\idxcode{rel_line}!constructor}%
\begin{itemdecl}
constexpr explicit rel_line(const vector_2d& pt) noexcept;
\end{itemdecl}
\begin{itemdescr}
\pnum
\effects
Constructs an object of type \tcode{rel_line}.

\pnum
The end point is \tcode{pt}.
\end{itemdescr}

\rSec1 [\iotwod.relline.modifiers]{\tcode{rel_line} modifiers}

\indexlibrarymember{rel_line}{to}
\begin{itemdecl}
constexpr void to(const vector_2d& pt) noexcept;
\end{itemdecl}
\begin{itemdescr}
\pnum
\effects
The end point is \tcode{pt}.
\end{itemdescr}

\rSec1 [\iotwod.relline.observers]{\tcode{rel_line} observers}

\indexlibrary{\idxcode{rel_line}!\idxcode{to}}%
\begin{itemdecl}
constexpr vector_2d to() const noexcept;
\end{itemdecl}
\begin{itemdescr}
\pnum
\returns
The end point.
\end{itemdescr}

\rSec1 [\iotwod.relline.ops]{\tcode{rel_line} operators}

\indexlibrarymember{operator==}{rel_line}%
\begin{itemdecl}
constexpr bool operator==(const rel_line& lhs, const rel_line& rhs) noexcept;
\end{itemdecl}
\begin{itemdescr}
\pnum
\returns
\tcode{lhs.to() == rhs.to()}.
\end{itemdescr}

%!TEX root = io2d.tex
\rSec0 [\iotwod.absquadraticcurve] {Class \tcode{abs_quadratic_curve}}

\pnum
\indexlibrary{\idxcode{abs_quadratic_curve}}%
The class \tcode{abs_quadratic_curve} describes a figure item that is a segment.

\pnum
It has a \term{control point} of type \tcode{basic_point_2d} and an \term{end point} of type \tcode{basic_point_2d}.

\rSec1 [\iotwod.absquadraticcurve.cons] {\tcode{abs_quadratic_curve} constructors}

\indexlibrary{\idxcode{abs_quadratic_curve}!constructor}%
\begin{itemdecl}
abs_quadratic_curve() noexcept;
\end{itemdecl}
\begin{itemdescr}
\pnum
\effects
Equivalent to: \tcode{abs_quadratic_curve\{ basic_point_2d(), basic_point_2d() \};}
\end{itemdescr}

\indexlibrary{\idxcode{abs_quadratic_curve}!constructor}%
\begin{itemdecl}
abs_quadratic_curve(const basic_point_2d<typename GraphicsSurfaces::graphics_math_type>& cpt,
  const basic_point_2d<typename GraphicsSurfaces::graphics_math_type>& ept) noexcept;
\end{itemdecl}
\begin{itemdescr}
\pnum
\effects
Constructs an object of type \tcode{abs_quadratic_curve}.

\pnum
The control point is \tcode{cpt}.

\pnum
The end point is \tcode{ept}.
\end{itemdescr}

\indexlibrary{\idxcode{abs_quadratic_curve}!constructor}%
\begin{itemdecl}
abs_quadratic_curve(const abs_quadratic_curve& other);
abs_quadratic_curve(abs_quadratic_curve&& other) noexcept;
\end{itemdecl}
\begin{itemdescr}
\pnum
\effects
Constructs an object of type \tcode{abs_quadratic_curve}. In the second form, other is left in a valid state with an unspecified value.

\pnum
The control point is \tcode{other.control_pt()}.

\pnum
The end point is \tcode{other.end_pt()}.
\end{itemdescr}

\rSec1 [\iotwod.absquadraticcurve.assign] {\tcode{abs_quadratic_curve} assignment operators}

\indexlibrary{\idxcode{abs_quadratic_curve}!assignment}%
\begin{itemdecl}
abs_quadratic_curve& operator=(const abs_quadratic_curve& other);
\end{itemdecl}
\begin{itemdescr}
\pnum
\effects
If \tcode{*this} and \tcode{other} are not the same object, modifies \tcode{*this} such that \tcode{*this.control_pt()} is \tcode{other.control_pt()} and \tcode{*this.end_pt()} is \tcode{other.end_pt()}

\pnum
If \tcode{*this} and \tcode{other} are the same object, the member has no effect.

\pnum
\returns
\tcode{*this}
\end{itemdescr}

\indexlibrary{\idxcode{abs_quadratic_curve}!assignment}%
\begin{itemdecl}
abs_quadratic_curve& operator=(abs_quadratic_curve&& other) noexcept;
\end{itemdecl}
\begin{itemdescr}
\pnum
\effects
<TODO>

\pnum
\returns
\tcode{*this}
\end{itemdescr}

\rSec1 [\iotwod.absquadraticcurve.modifiers]{\tcode{abs_quadratic_curve} modifiers}

\indexlibrarymember{control_pt}{abs_quadratic_curve}%
\begin{itemdecl}
void control_pt(const basic_point_2d<typename GraphicsSurfaces::graphics_math_type>& cpt) noexcept;
\end{itemdecl}
\begin{itemdescr}
\pnum
\effects
The control point is \tcode{cpt}.
\end{itemdescr}

\indexlibrarymember{end_pt}{abs_quadratic_curve}%
\begin{itemdecl}
void end_pt(const basic_point_2d<typename GraphicsSurfaces::graphics_math_type>& ept) noexcept;
\end{itemdecl}
\begin{itemdescr}
\pnum
\effects
The end point is \tcode{ept}.
\end{itemdescr}

\rSec1 [\iotwod.absquadraticcurve.observers]{\tcode{abs_quadratic_curve} observers}

\indexlibrarymember{control_pt}{abs_quadratic_curve}%
\begin{itemdecl}
basic_point_2d<typename GraphicsSurfaces::graphics_math_type> control_pt() const noexcept;
\end{itemdecl}
\begin{itemdescr}
\pnum
\returns
The control point.
\end{itemdescr}

\indexlibrarymember{end_pt}{abs_quadratic_curve}%
\begin{itemdecl}
basic_point_2d<typename GraphicsSurfaces::graphics_math_type> end_pt() const noexcept;
\end{itemdecl}
\begin{itemdescr}
\pnum
\returns
The end point.
\end{itemdescr}

\rSec1 [\iotwod.absquadraticcurve.ops]{\tcode{abs_quadratic_curve} operators}

\indexlibrarymember{operator==}{abs_quadratic_curve}%
\begin{itemdecl}
template <class GraphicsSurfaces>
bool operator==(const typename basic_figure_items<GraphicsSurfaces>::abs_quadratic_curve& lhs,
  const typename basic_figure_items<GraphicsSurfaces>::abs_quadratic_curve& rhs) noexcept;
\end{itemdecl}
\begin{itemdescr}
\pnum
\returns
\tcode{lhs.control_pt() == rhs.control_pt() \&\& lhs.end_pt() == rhs.end_pt()}.
\end{itemdescr}

%!TEX root = io2d.tex
\rSec0 [\iotwod.relquadraticcurve] {Class \tcode{rel_quadratic_curve}}

\pnum
\indexlibrary{\idxcode{rel_quadratic_curve}}%
The class \tcode{rel_quadratic_curve} describes a figure item that is a segment.

\pnum
It has a \term{control point} of type \tcode{basic_point_2d} and an \term{end point} of type \tcode{basic_point_2d}.

\rSec1 [\iotwod.relquadraticcurve.cons] {\tcode{rel_quadratic_curve} constructors}

\indexlibrary{\idxcode{rel_quadratic_curve}!constructor}%
\begin{itemdecl}
rel_quadratic_curve() noexcept;
\end{itemdecl}
\begin{itemdescr}
\pnum
\effects
Equivalent to: \tcode{rel_quadratic_curve\{ basic_point_2d(), basic_point_2d() \};}
\end{itemdescr}

\indexlibrary{\idxcode{rel_quadratic_curve}!constructor}%
\begin{itemdecl}
rel_quadratic_curve(const basic_point_2d<typename GraphicsSurfaces::graphics_math_type>& cpt,
  const basic_point_2d<typename GraphicsSurfaces::graphics_math_type>& ept) noexcept;
\end{itemdecl}
\begin{itemdescr}
\pnum
\effects
Constructs an object of type \tcode{rel_quadratic_curve}.

\pnum
The control point is \tcode{cpt}.

\pnum
The end point is \tcode{ept}.
\end{itemdescr}

\indexlibrary{\idxcode{rel_quadratic_curve}!constructor}%
\begin{itemdecl}
rel_quadratic_curve(const rel_quadratic_curve& other);
rel_quadratic_curve(rel_quadratic_curve&& other) noexcept;
\end{itemdecl}
\begin{itemdescr}
\pnum
\effects
Constructs an object of type \tcode{rel_quadratic_curve}. In the second form, other is left in a valid state with an unspecified value.

\pnum
The control point is \tcode{other.control_pt()}.

\pnum
The end point is \tcode{other.end_pt()}.
\end{itemdescr}

\rSec1 [\iotwod.relquadraticcurve.assign] {\tcode{rel_quadratic_curve} assignment operators}

\indexlibrary{\idxcode{rel_quadratic_curve}!assignment}%
\begin{itemdecl}
rel_quadratic_curve& operator=(const rel_quadratic_curve& other);
\end{itemdecl}
\begin{itemdescr}
\pnum
\effects
If \tcode{*this} and \tcode{other} are not the same object, modifies \tcode{*this} such that \tcode{*this.control_pt()} is \tcode{other.control_pt()} and \tcode{*this.end_pt()} is \tcode{other.end_pt()}

\pnum
If \tcode{*this} and \tcode{other} are the same object, the member has no effect.

\pnum
\returns
\tcode{*this}
\end{itemdescr}

\indexlibrary{\idxcode{rel_quadratic_curve}!assignment}%
\begin{itemdecl}
rel_quadratic_curve& operator=(rel_quadratic_curve&& other) noexcept;
\end{itemdecl}
\begin{itemdescr}
\pnum
\effects
<TODO>

\pnum
\returns
\tcode{*this}
\end{itemdescr}

\rSec1 [\iotwod.relquadraticcurve.modifiers]{\tcode{rel_quadratic_curve} modifiers}

\indexlibrarymember{control_pt}{rel_quadratic_curve}%
\begin{itemdecl}
void control_pt(const basic_point_2d<typename GraphicsSurfaces::graphics_math_type>& cpt) noexcept;
\end{itemdecl}
\begin{itemdescr}
\pnum
\effects
The control point is \tcode{cpt}.
\end{itemdescr}

\indexlibrarymember{end_pt}{rel_quadratic_curve}%
\begin{itemdecl}
void end_pt(const basic_point_2d<typename GraphicsSurfaces::graphics_math_type>& ept) noexcept;
\end{itemdecl}
\begin{itemdescr}
\pnum
\effects
The end point is \tcode{ept}.
\end{itemdescr}

\rSec1 [\iotwod.relquadraticcurve.observers]{\tcode{rel_quadratic_curve} observers}

\indexlibrarymember{control_pt}{rel_quadratic_curve}%
\begin{itemdecl}
basic_point_2d<typename GraphicsSurfaces::graphics_math_type> control_pt() const noexcept;
\end{itemdecl}
\begin{itemdescr}
\pnum
\returns
The control point.
\end{itemdescr}

\indexlibrarymember{end_pt}{rel_quadratic_curve}%
\begin{itemdecl}
basic_point_2d<typename GraphicsSurfaces::graphics_math_type> end_pt() const noexcept;
\end{itemdecl}
\begin{itemdescr}
\pnum
\returns
The end point.
\end{itemdescr}

\rSec1 [\iotwod.relquadraticcurve.ops]{\tcode{rel_quadratic_curve} operators}

\indexlibrarymember{operator==}{rel_quadratic_curve}%
\begin{itemdecl}
template <class GraphicsSurfaces>
bool operator==(const typename basic_figure_items<GraphicsSurfaces>::rel_quadratic_curve& lhs,
  const typename basic_figure_items<GraphicsSurfaces>::rel_quadratic_curve& rhs) noexcept;
\end{itemdecl}
\begin{itemdescr}
\pnum
\returns
\tcode{lhs.control_pt() == rhs.control_pt() \&\& lhs.end_pt() == rhs.end_pt()}.
\end{itemdescr}

%!TEX root = io2d.tex
\rSec0 [\iotwod.abscubiccurve] {Class template \tcode{basic_figure_items<GraphicsSurfaces>::abs_cubic_curve}}

\rSec1 [\iotwod.abscubiccurve.intro] {Overview}

\pnum
\indexlibrary{\idxcode{abs_cubic_curve}}%
The class \tcode{basic_figure_items<GraphicsSurfaces>::abs_cubic_curve} describes a figure item that is a segment.

\pnum
It has a \term{first control point} of type \tcode{basic_point_2d<GraphicsSurfaces::graphics_math_type>}, a \term{second control point} of type \tcode{basic_point_2d<GraphicsSurfaces::graphics_math_type>}, and an \tcode{end point} of type \tcode{basic_point_2d<GraphicsSurfaces::graphics_math_type>}.

\pnum
The data are stored in an object of type \tcode{typename GraphicsSurfaces::paths::abs_cubic_curve_data_type}. It is accessible using the \tcode{data} member functions.

\rSec1 [\iotwod.abscubiccurve.synopsis] {Synopsis}
\begin{codeblock}
namespace @\fullnamespace{}@ {
  template <class GraphicsSurfaces>
  class basic_figure_items<GraphicsSurfaces>::abs_cubic_curve {
  public:
    using graphics_math_type = typename GraphicsSurfaces::graphics_math_type;
    using data_type =
      typename GraphicsSurfaces::paths::abs_cubic_curve_data_type;

    // \ref{\iotwod.abscubiccurve.ctor}, construct:
    abs_cubic_curve();
    abs_cubic_curve(const basic_point_2d<graphics_math_type>& cpt1,
       const basic_point_2d<graphics_math_type>& cpt2,
       const basic_point_2d<graphics_math_type>& ept) noexcept;
    abs_cubic_curve(const abs_cubic_curve& other) = default;
    abs_cubic_curve(abs_cubic_curve&& other) noexcept = default;

    // assign:
    abs_cubic_curve& operator=(const abs_cubic_curve& other) = default;
    abs_cubic_curve& operator=(abs_cubic_curve&& other) noexcept = default;

    // \ref{\iotwod.abscubiccurve.acc}, accessors:
    const data_type& data() const noexcept;
    data_type& data() noexcept;

    // \ref{\iotwod.abscubiccurve.mod}, modifiers:
    void control_pt1(const basic_point_2d<graphics_math_type>& cpt) noexcept;
    void control_pt2(const basic_point_2d<graphics_math_type>& cpt) noexcept;
    void end_pt(const basic_point_2d<graphics_math_type>& ept) noexcept;

    // \ref{\iotwod.abscubiccurve.obs}, observers:
    basic_point_2d<graphics_math_type> control_pt1() const noexcept;
    basic_point_2d<graphics_math_type> control_pt2() const noexcept;
    basic_point_2d<graphics_math_type> end_pt() const noexcept;
  };

  // \ref{\iotwod.abscubiccurve.eq}, equality operators:
  template <class GraphicsSurfaces>
  bool operator==(
    const typename basic_figure_items<GraphicsSurfaces>::abs_cubic_curve& lhs,
    const typename basic_figure_items<GraphicsSurfaces>::abs_cubic_curve& rhs) 
    noexcept;  
  template <class GraphicsSurfaces>
  bool operator!=(
    const typename basic_figure_items<GraphicsSurfaces>::abs_cubic_curve& lhs,
    const typename basic_figure_items<GraphicsSurfaces>::abs_cubic_curve& rhs) 
    noexcept;  
}
\end{codeblock}

\rSec1 [\iotwod.abscubiccurve.ctor] {Constructors}%

\indexlibrary{\idxcode{abs_cubic_curve}!constructor}%
\begin{itemdecl}
abs_cubic_curve() noexcept;
\end{itemdecl}
\begin{itemdescr}
\pnum
\effects
Equivalent to \tcode{abs_cubic_curve\{ basic_point_2d(), basic_point_2d(), basic_point_2d() \}}.
\end{itemdescr}

\indexlibrary{\idxcode{abs_cubic_curve}!constructor}%
\begin{itemdecl}
abs_cubic_curve(const basic_point_2d<typename GraphicsSurfaces::graphics_math_type>& cpt1,
  const basic_point_2d<typename GraphicsSurfaces::graphics_math_type>& cpt2,
  const basic_point_2d<typename GraphicsSurfaces::graphics_math_type>& ept) noexcept;
\end{itemdecl}
\begin{itemdescr}
\pnum
\effects Constructs an object of type \tcode{abs_cubic_curve}.

\pnum
\remarks The first control point is \tcode{cpt1}.

\pnum
\remarks The second control point is \tcode{cpt2}.

\pnum
\remarks The end point is \tcode{ept}.
\end{itemdescr}

\rSec1 [\iotwod.abscubiccurve.acc] {Accessors}%

\indexlibrarymember{data}{abs_cubic_curve}%
\begin{itemdecl}
const data_type& data() const noexcept;
data_type& data() noexcept;
\end{itemdecl}
\begin{itemdescr}
\pnum
\returns A reference to the \tcode{rel_matrix} object's data object (See: \ref{\iotwod.abscubiccurve.intro}).
\end{itemdescr}

\rSec1 [\iotwod.abscubiccurve.mod] {Modifiers}

\indexlibrarymember{control_pt1}{abs_cubic_curve}%
\begin{itemdecl}
void control_pt1(const basic_point_2d<typename
  GraphicsSurfaces::graphics_math_type>& cpt) noexcept;
\end{itemdecl}
\begin{itemdescr}
\pnum
\effects
The first control point is \tcode{cpt}.
\end{itemdescr}

\indexlibrarymember{control_pt2}{abs_cubic_curve}%
\begin{itemdecl}
void control_pt2(const basic_point_2d<typename
  GraphicsSurfaces::graphics_math_type>& cpt) noexcept;
\end{itemdecl}
\begin{itemdescr}
\pnum
\effects
The second control point is \tcode{cpt}.
\end{itemdescr}

\indexlibrarymember{end_pt}{abs_cubic_curve}%
\begin{itemdecl}
void end_pt(const basic_point_2d<typename GraphicsSurfaces::graphics_math_type>& ept) noexcept;
\end{itemdecl}
\begin{itemdescr}
\pnum
\effects
The end point is \tcode{ept}.
\end{itemdescr}

\rSec1 [\iotwod.abscubiccurve.obs] {Observers}

\indexlibrarymember{control_pt1}{abs_cubic_curve}%
\begin{itemdecl}
basic_point_2d<graphics_math_type> control_pt1() const noexcept;
\end{itemdecl}
\begin{itemdescr}
\pnum
\returns The first control point.
\end{itemdescr}

\indexlibrarymember{control_pt2}{abs_cubic_curve}%
\begin{itemdecl}
basic_point_2d<graphics_math_type> control_pt2() const noexcept;
\end{itemdecl}
\begin{itemdescr}
\pnum
\returns The second control point.
\end{itemdescr}

\indexlibrarymember{end_pt}{abs_cubic_curve}%
\begin{itemdecl}
basic_point_2d<graphics_math_type> end_pt() const noexcept;
\end{itemdecl}
\begin{itemdescr}
\pnum
\returns The end point.
\end{itemdescr}

\rSec1 [\iotwod.abscubiccurve.eq] {Equality operators}%

\indexlibrarymember{operator==}{abs_cubic_curve}%
\begin{itemdecl}
template <class GraphicsSurfaces>
bool operator==(
  const typename basic_figure_items<GraphicsSurfaces>::abs_cubic_curve& lhs,
  const typename basic_figure_items<GraphicsSurfaces>::abs_cubic_curve& rhs) 
  noexcept;
\end{itemdecl}
\begin{itemdescr}
\pnum
\returns
\tcode{lhs.control_pt1() == rhs.control_pt1() \&\& lhs.control_pt2() == rhs.control_pt2() \&\& lhs.end_pt() == rhs.end_pt()}.
\end{itemdescr}

\indexlibrarymember{operator!=}{abs_cubic_curve}%
\begin{itemdecl}
template <class GraphicsSurfaces>
bool operator!=(
  const typename basic_figure_items<GraphicsSurfaces>::abs_cubic_curve& lhs,
  const typename basic_figure_items<GraphicsSurfaces>::abs_cubic_curve& rhs) 
  noexcept;
\end{itemdecl}
\begin{itemdescr}
\pnum
\returns
\tcode{lhs.control_pt1() != rhs.control_pt1() || lhs.control_pt2() != rhs.control_pt2() || lhs.end_pt() != rhs.end_pt()}.
\end{itemdescr}

%!TEX root = io2d.tex
\rSec0 [\iotwod.relcubiccurve] {Class template \tcode{basic_figure_items<GraphicsSurfaces>::rel_cubic_curve}}

\rSec1 [\iotwod.relcubiccurve.intro] {Overview}

\pnum
\indexlibrary{\idxcode{rel_cubic_curve}}%
The class \tcode{basic_figure_items<GraphicsSurfaces>::rel_cubic_curve} describes a figure item that is a segment.

\pnum
It has a \term{first control point} of type \tcode{basic_point_2d<GraphicsSurfaces::graphics_math_type>}, a \term{second control point} of type \tcode{basic_point_2d<GraphicsSurfaces::graphics_math_type>}, and an \tcode{end point} of type \tcode{basic_point_2d<GraphicsSurfaces::graphics_math_type>}.

\pnum
The data are stored in an object of type \tcode{typename GraphicsSurfaces::paths::rel_cubic_curve_data_type}. It is accessible using the \tcode{data} member functions.

\rSec1 [\iotwod.relcubiccurve.synopsis] {Synopsis}
\begin{codeblock}
namespace std::experimemtal::io2d::v1 {
  template <class GraphicsSurfaces>
  class basic_figure_items<GraphicsSurfaces>::rel_cubic_curve {
  public:
    using graphics_math_type = typename GraphicsSurfaces::graphics_math_type;
    using data_type =
      typename GraphicsSurfaces::paths::rel_cubic_curve_data_type;

    // \ref{\iotwod.relcubiccurve.ctor}, construct:
    rel_cubic_curve();
    rel_cubic_curve(const basic_point_2d<graphics_math_type>& cpt1,
       const basic_point_2d<graphics_math_type>& cpt2,
       const basic_point_2d<graphics_math_type>& ept) noexcept;
    rel_cubic_curve(const rel_cubic_curve& other) = default;
    rel_cubic_curve(rel_cubic_curve&& other) noexcept = default;

    // assign:
    rel_cubic_curve& operator=(const rel_cubic_curve& other) = default;
    rel_cubic_curve& operator=(rel_cubic_curve&& other) noexcept = default;

    // \ref{\iotwod.relcubiccurve.acc}, accessors:
    const data_type& data() const noexcept;
    data_type& data() noexcept;

    // \ref{\iotwod.relcubiccurve.mod}, modifiers:
    void control_pt1(const basic_point_2d<graphics_math_type>& cpt) noexcept;
    void control_pt2(const basic_point_2d<graphics_math_type>& cpt) noexcept;
    void end_pt(const basic_point_2d<graphics_math_type>& ept) noexcept;

    // \ref{\iotwod.relcubiccurve.obs}, observers:
    basic_point_2d<graphics_math_type> control_pt1() const noexcept;
    basic_point_2d<graphics_math_type> control_pt2() const noexcept;
    basic_point_2d<graphics_math_type> end_pt() const noexcept;
  };

  // \ref{\iotwod.relcubiccurve.eq}, equality operators:
  template <class GraphicsSurfaces>
  bool operator==(
    const typename basic_figure_items<GraphicsSurfaces>::rel_cubic_curve& lhs,
    const typename basic_figure_items<GraphicsSurfaces>::rel_cubic_curve& rhs) 
    noexcept;  
  template <class GraphicsSurfaces>
  bool operator!=(
    const typename basic_figure_items<GraphicsSurfaces>::rel_cubic_curve& lhs,
    const typename basic_figure_items<GraphicsSurfaces>::rel_cubic_curve& rhs) 
    noexcept;  
}
\end{codeblock}

\rSec1 [\iotwod.relcubiccurve.ctor] {Constructors}%

\indexlibrary{\idxcode{rel_cubic_curve}!constructor}%
\begin{itemdecl}
rel_cubic_curve() noexcept;
\end{itemdecl}
\begin{itemdescr}
\pnum
\effects
Equivalent to \tcode{rel_cubic_curve\{ basic_point_2d(), basic_point_2d(), basic_point_2d() \}}.
\end{itemdescr}

\indexlibrary{\idxcode{rel_cubic_curve}!constructor}%
\begin{itemdecl}
rel_cubic_curve(const basic_point_2d<typename GraphicsSurfaces::graphics_math_type>& cpt1,
  const basic_point_2d<typename GraphicsSurfaces::graphics_math_type>& cpt2,
  const basic_point_2d<typename GraphicsSurfaces::graphics_math_type>& ept) noexcept;
\end{itemdecl}
\begin{itemdescr}
\pnum
\effects Constructs an object of type \tcode{rel_cubic_curve}.

\pnum
\remarks The first control point is \tcode{cpt1}.

\pnum
\remarks The second control point is \tcode{cpt2}.

\pnum
\remarks The end point is \tcode{ept}.
\end{itemdescr}

\rSec1 [\iotwod.relcubiccurve.acc] {Accessors}%

\indexlibrarymember{data}{rel_cubic_curve}%
\begin{itemdecl}
const data_type& data() const noexcept;
data_type& data() noexcept;
\end{itemdecl}
\begin{itemdescr}
\pnum
\returns A reference to the \tcode{rel_matrix} object's data object (See: \ref{\iotwod.relcubiccurve.intro}).
\end{itemdescr}

\rSec1 [\iotwod.relcubiccurve.mod] {Modifiers}

\indexlibrarymember{control_pt1}{rel_cubic_curve}%
\begin{itemdecl}
void control_pt1(const basic_point_2d<typename
  GraphicsSurfaces::graphics_math_type>& cpt) noexcept;
\end{itemdecl}
\begin{itemdescr}
\pnum
\effects
The first control point is \tcode{cpt}.
\end{itemdescr}

\indexlibrarymember{control_pt2}{rel_cubic_curve}%
\begin{itemdecl}
void control_pt2(const basic_point_2d<typename
  GraphicsSurfaces::graphics_math_type>& cpt) noexcept;
\end{itemdecl}
\begin{itemdescr}
\pnum
\effects
The second control point is \tcode{cpt}.
\end{itemdescr}

\indexlibrarymember{end_pt}{rel_cubic_curve}%
\begin{itemdecl}
void end_pt(const basic_point_2d<typename GraphicsSurfaces::graphics_math_type>& ept) noexcept;
\end{itemdecl}
\begin{itemdescr}
\pnum
\effects
The end point is \tcode{ept}.
\end{itemdescr}

\rSec1 [\iotwod.relcubiccurve.obs] {Observers}

\indexlibrarymember{control_pt1}{rel_cubic_curve}%
\begin{itemdecl}
basic_point_2d<graphics_math_type> control_pt1() const noexcept;
\end{itemdecl}
\begin{itemdescr}
\pnum
\returns The first control point.
\end{itemdescr}

\indexlibrarymember{control_pt2}{rel_cubic_curve}%
\begin{itemdecl}
basic_point_2d<graphics_math_type> control_pt2() const noexcept;
\end{itemdecl}
\begin{itemdescr}
\pnum
\returns The second control point.
\end{itemdescr}

\indexlibrarymember{end_pt}{rel_cubic_curve}%
\begin{itemdecl}
basic_point_2d<graphics_math_type> end_pt() const noexcept;
\end{itemdecl}
\begin{itemdescr}
\pnum
\returns The end point.
\end{itemdescr}

\rSec1 [\iotwod.relcubiccurve.eq] {Equality operators}%

\indexlibrarymember{operator==}{rel_cubic_curve}%
\begin{itemdecl}
template <class GraphicsSurfaces>
bool operator==(
  const typename basic_figure_items<GraphicsSurfaces>::rel_cubic_curve& lhs,
  const typename basic_figure_items<GraphicsSurfaces>::rel_cubic_curve& rhs) 
  noexcept;
\end{itemdecl}
\begin{itemdescr}
\pnum
\returns
\tcode{lhs.control_pt1() == rhs.control_pt1() \&\& lhs.control_pt2() == rhs.control_pt2() \&\& lhs.end_pt() == rhs.end_pt()}.
\end{itemdescr}

\indexlibrarymember{operator!=}{rel_cubic_curve}%
\begin{itemdecl}
template <class GraphicsSurfaces>
bool operator!=(
  const typename basic_figure_items<GraphicsSurfaces>::rel_cubic_curve& lhs,
  const typename basic_figure_items<GraphicsSurfaces>::rel_cubic_curve& rhs) 
  noexcept;
\end{itemdecl}
\begin{itemdescr}
\pnum
\returns
\tcode{lhs.control_pt1() != rhs.control_pt1() || lhs.control_pt2() != rhs.control_pt2() || lhs.end_pt() != rhs.end_pt()}.
\end{itemdescr}

%!TEX root = io2d.tex
\rSec0 [\iotwod.arc] {Class \tcode{arc}}

\rSec1 [\iotwod.arc.general] {In general}

\pnum
\indexlibrary{\idxcode{arc}}%
The class \tcode{arc} describes a figure item that is a segment.

\pnum
It has a \term{radius} of type \tcode{basic_point_2d}, a \term{rotation} of type \tcode{float}, and a \term{start angle} of type \tcode{float}.

\pnum
It forms a portion of the circumference of a circle. The centre of the circle is implied by the start point, the radius and the start angle of the arc.

\rSec1 [\iotwod.arc.cons] {\tcode{arc} constructors}

\indexlibrary{\idxcode{arc}!constructor}%
\begin{itemdecl}
arc() noexcept;
\end{itemdecl}
\begin{itemdescr}
\pnum
\effects
Equivalent to: \tcode{arc\{ basic_point_2d(10.0f, 10.0f), pi<float>, pi<float> \};}.
\end{itemdescr}

\indexlibrary{\idxcode{arc}!constructor}%
\begin{itemdecl}
arc(const basic_point_2d<typename GraphicsSurfaces::graphics_math_type>& rad,
  float rot, float sang) noexcept;
\end{itemdecl}
\begin{itemdescr}
\pnum
\effects
Constructs an object of type \tcode{arc}.

\pnum
The radius is \tcode{rad}.

\pnum
The rotation is \tcode{rot}.

\pnum
The start angle is \tcode{sang}.
\end{itemdescr}

\indexlibrary{\idxcode{arc}!constructor}%
\begin{itemdecl}
arc(const arc& other);
arc(arc&& other) noexcept;
\end{itemdecl}
\begin{itemdescr}
\pnum
\effects
Constructs an object of type \tcode{arc}. In the second form, other is left in a valid state with an unspecified value.

\pnum
The radius is \tcode{other.radius()}.

\pnum
The rotation is \tcode{other.rotation()}.

\pnum
The start angle is \tcode{other.start_angle()}.
\end{itemdescr}

\rSec1 [\iotwod.arc.assign] {\tcode{arc} assignment operators}

\indexlibrary{\idxcode{arc}!assignment}%
\begin{itemdecl}
arc& operator=(const arc& other);
\end{itemdecl}
\begin{itemdescr}
\pnum
\effects
If \tcode{*this} and \tcode{other} are not the same object, modifies \tcode{*this} such that \tcode{*this.radius()} is \tcode{other.radius()}, \tcode{*this.rotation()} is \tcode{other.rotation()} and \tcode{*this.start_angle()} is \tcode{other.start_angle()}

\pnum
If \tcode{*this} and \tcode{other} are the same object, the member has no effect.

\pnum
\returns
\tcode{*this}
\end{itemdescr}

\indexlibrary{\idxcode{arc}!assignment}%
\begin{itemdecl}
arc& operator=(arc&& other) noexcept;
\end{itemdecl}
\begin{itemdescr}
\pnum
\effects
<TODO>

\pnum
\returns
\tcode{*this}
\end{itemdescr}

\rSec1 [\iotwod.arc.modifiers]{\tcode{arc} modifiers}

\indexlibrarymember{radius}{arc}%
\begin{itemdecl}
void radius(const basic_point_2d<typename GraphicsSurfaces::graphics_math_type>& rad) noexcept;
\end{itemdecl}
\begin{itemdescr}
\pnum
\effects
The radius is \tcode{rad}.
\end{itemdescr}

\indexlibrarymember{rotation}{arc}%
\begin{itemdecl}
constexpr void rotation(float rot) noexcept;
\end{itemdecl}
\begin{itemdescr}
\pnum
\effects
The rotation is \tcode{rot}.
\end{itemdescr}

\indexlibrarymember{start_angle}{arc}%
\begin{itemdecl}
void start_angle(float sang) noexcept;
\end{itemdecl}
\begin{itemdescr}
\pnum
\effects
The start angle is \tcode{sang}.
\end{itemdescr}

\rSec1 [\iotwod.arc.observers]{\tcode{arc} observers}

\indexlibrarymember{radius}{arc}%
\begin{itemdecl}
basic_point_2d<typename GraphicsSurfaces::graphics_math_type> radius() const noexcept;
\end{itemdecl}
\begin{itemdescr}
\pnum
\returns
The radius.
\end{itemdescr}

\indexlibrarymember{rotation}{arc}%
\begin{itemdecl}
float rotation() const noexcept;
\end{itemdecl}
\begin{itemdescr}
\pnum
\returns
The rotation.
\end{itemdescr}

\indexlibrarymember{start_angle}{arc}%
\begin{itemdecl}
float start_angle() const noexcept;
\end{itemdecl}
\begin{itemdescr}
\pnum
\returns
The start angle.
\end{itemdescr}

\indexlibrarymember{center}{arc}%
\begin{itemdecl}
basic_point_2d<typename GraphicsSurfaces::graphics_math_type> center(const basic_point_2d<typename
  GraphicsSurfaces::graphics_math_type>& cpt, const basic_matrix_2d<typename
  GraphicsSurfaces::graphics_math_type>& m = basic_matrix_2d<typename
  GraphicsSurfaces::graphics_math_type>{}) const noexcept;
\end{itemdecl}
\begin{itemdescr}
\pnum
\returns
As-if:
\begin{codeblock}
auto lmtx = m;
lmtx.m20 = 0.0f;
lmtx.m21 = 0.0f;
auto centerOffset = point_for_angle(two_pi<float> - start_angle(), radius());
centerOffset.y = -centerOffset.y;
return cpt - centerOffset * lmtx;
\end{codeblock}
\end{itemdescr}

\indexlibrarymember{start_angle}{arc}%
\begin{itemdecl}
basic_point_2d<typename GraphicsSurfaces::graphics_math_type> end_pt(const basic_point_2d<typename
  GraphicsSurfaces::graphics_math_type>& cpt, const basic_matrix_2d<typename
  GraphicsSurfaces::graphics_math_type>& m = basic_matrix_2d<typename
  GraphicsSurfaces::graphics_math_type>{}) const noexcept;
\end{itemdecl}
\begin{itemdescr}
\pnum
\returns
As-if:
\begin{codeblock}
auto lmtx = m;
auto tfrm = matrix_2d::init_rotate(start_angle() + rotation());
lmtx.m20 = 0.0f;
lmtx.m21 = 0.0f;
auto pt = (radius() * tfrm);
pt.y = -pt.y;
return cpt + pt * lmtx;
\end{codeblock}
\end{itemdescr}

\rSec1 [\iotwod.arc.ops]{\tcode{arc} operators}

\indexlibrarymember{operator==}{arc}%
\begin{itemdecl}
template <class GraphicsSurfaces>
bool operator==(const typename basic_figure_items<GraphicsSurfaces>::arc& lhs,
  const typename basic_figure_items<GraphicsSurfaces>::arc& rhs) noexcept;
\end{itemdecl}
\begin{itemdescr}
\pnum
\returns
\begin{codeblock}
lhs.radius() == rhs.radius() && lhs.rotation() == rhs.rotation() &&
lhs.start_angle() && rhs.start_angle()
\end{codeblock}
\end{itemdescr}

%!TEX root = io2d.tex

\rSec0 [\iotwod.paths.interpretation]{Path group interpretation}

\pnum
This subclause describes how to interpret a path group for use in a rendering and composing operation.

\pnum
Interpreting a path group consists of sequentially evaluating the \tcode{path_data::path_item} objects in a path group and transforming them into zero or more paths as-if in the manner specified in this subclause.

\pnum
The interpretation of a path group requires the state data specified in Table~\ref{tab:\iotwod.paths.interpretation.state}.

\begin{floattable}
{Path group interpretation state data}{tab:\iotwod.paths.interpretation.state}{llll}
\hline
\hdstyle{Name} &
\hdstyle{Description} &
\hdstyle{Type} &
\hdstyle{Initial value} \\ \hline
\tcode{mtx} &
Path group transformation matrix &
\tcode{matrix_2d} &
\tcode{matrix_2d\{ \}} \\
\tcode{currPt} &
Current point &
\tcode{vector_2d} &
\unspec \\
\tcode{lnPt} &
Last new point &
\tcode{vector_2d} &
\unspec \\
\tcode{mtxStk} &
Matrix stack &
\tcode{stack<matrix_2d>} &
\tcode{stack<matrix_2d>\{ \}} \\\hline
\end{floattable}

\FloatBarrier

\pnum
When interpreting a path group, until a \tcode{path_data::abs_new_path} path item is reached, a path shall only contain path group instruction path items; no diagnostic is required. If a path is a degenerate path, none of its path items have any effects, with two exceptions:
\begin{itemize}
\item the path's \tcode{path_data::abs_new_path} or \tcode{path_data::rel_new_path} path item sets the value of \tcode{currPt} as-if the path item was interpreted; and,
\item any path group instruction path items are evaluated with full effect.
\end{itemize}.

%\pnum
%\begin{note}
%The requirement above stating "until a \tcode{path_data::abs_new_path} path item is reached..." uses the word "reached" to make it clear that whether the first path is a degenerate path is irrelevant to that requirement.
%\end{note}
\pnum
The effects of a path item contained in a \tcode{path_data::path_item} object when that object is being evaluated during path group interpretation are described in Table~\ref{tab:\iotwod.paths.interpretation.effects}.

\begin{libreqtab2a} {Path item interpretation effects} {tab:\iotwod.paths.interpretation.effects}
\\ \topline
\lhdr{Path item} & \rhdr{Effects} \\ \capsep
\endfirsthead
\continuedcaption\\
\topline
\lhdr{Path item} & \rhdr{Effects} \\ \capsep
\endhead

\tcode{path_data::abs_new_path p} &
Creates a new path. Sets \tcode{currPt} to \tcode{mtx.transform_pt(\{ 0.0f, 0.0f \}) + p.at()}. Sets \tcode{lnPt} to \tcode{currPt}. \\ \rowsep

\tcode{path_data::rel_new_path p} &
Let \tcode{mm} equal \tcode{mtx}.  Let \tcode{mm.m20} equal {0.0f}. Let \tcode{mm.m21} equal \tcode{0.0f}. Creates a new path. Sets \tcode{currPt} to \tcode{currPt + p.at() * mm}. Sets \tcode{lnPt} to \tcode{currPt}. \\ \rowsep

\tcode{path_data::close_path p} &
Creates a line from \tcode{currPt} to \tcode{lnPt}. Makes the current path a closed path. Creates a new path. Sets \tcode{currPt} to \tcode{lnPt}. \\ \rowsep

\tcode{path_data::abs_matrix p} &
Calls \tcode{mtxStk.push(mtx)}. Sets \tcode{mtx} to \tcode{p.matrix()}. \\ \rowsep

\tcode{path_data::rel_matrix p} &
Calls \tcode{mtxStk.push(mtx)}. Sets \tcode{mtx} to \tcode{mtx * p.matrix()}. \\ \rowsep

\tcode{path_data::revert_matrix p} &
If \tcode{mtxStk.empty()} is \tcode{false}, sets \tcode{mtx} to \tcode{mtxStk.top()} then calls \tcode{mtxStk.pop()}. Otherwise sets \tcode{mtx} to its initial value as specified in Table~\ref{tab:\iotwod.paths.interpretation.state}. \\ \rowsep

\tcode{path_data::abs_line p} &
Let \tcode{pt} equal \tcode{mtx.transform_pt(p.to() - currPt) + currPt}. Creates a line from \tcode{currPt} to \tcode{pt}. Sets \tcode{currPt} to \tcode{pt}. \\ \rowsep

\tcode{path_data::rel_line p} &
Let \tcode{mm} equal \tcode{mtx}. Let \tcode{mm.m20} equal {0.0f}. Let \tcode{mm.m21} equal \tcode{0.0f}. Let \tcode{pt} equal \tcode{currPt + p.to() * mm}. Creates a line from \tcode{currPt} to \tcode{pt}. Sets \tcode{currPt} to \tcode{pt}. \\ \rowsep

\tcode{path_data::abs_quadratic_curve p} &
Let \tcode{cpt} equal \tcode{mtx.transform_pt(p.control_pt() - currPt) + currPt}. Let \tcode{ept} equal \tcode{mtx.transform_pt(p.end_pt() - currPt) + currPt}. Creates a quadratic \bezierlocal curve from \tcode{currPt} to \tcode{ept} using \tcode{cpt} as the curve's control point. Sets \tcode{currPt} to \tcode{ept}. \\ \rowsep

\tcode{path_data::rel_quadratic_curve p} &
Let \tcode{mm} equal \tcode{mtx}. Let \tcode{mm.m20} equal {0.0f}. Let \tcode{mm.m21} equal \tcode{0.0f}. Let \tcode{cpt} equal \tcode{currPt + p.control_pt() * mm}. Let \tcode{ept} equal \tcode{currPt + p.control_pt() * mm + p.end_pt() * mm}. Creates a quadratic \bezierlocal curve from \tcode{currPt} to \tcode{ept} using \tcode{cpt} as the curve's control point. Sets \tcode{currPt} to \tcode{ept}. \\ \rowsep

\tcode{path_data::abs_cubic_curve p} &
Let \tcode{cpt1} equal \tcode{mtx.transform_pt(p.control_pt1() - currPt) + currPt}. Let \tcode{cpt2} equal \tcode{mtx.transform_pt(p.control_pt2() - currPt) + currPt}. Let \tcode{ept} equal \tcode{mtx.transform_pt(p.end_pt() - currPt) + currPt}. Creates a cubic \bezierlocal curve from \tcode{currPt} to \tcode{ept} using \tcode{cpt1} as the curve's first control point and \tcode{cpt2} as the curve's second control point. Sets \tcode{currPt} to \tcode{ept}. \\ \rowsep

\tcode{path_data::rel_cubic_curve p} &
Let \tcode{mm} equal \tcode{mtx}. Let \tcode{mm.m20} equal {0.0f}. Let \tcode{mm.m21} equal \tcode{0.0f}. Let \tcode{cpt1} equal \tcode{currPt + p.control_pt1() * mm}. Let \tcode{cpt2} equal \tcode{currPt + p.control_pt1() * mm + p.control_pt2() * mm}. Let \tcode{ept} equal \tcode{currPt + p.control_pt1() * mm + p.control_pt2() * mm + p.end_pt() * mm}. Creates a cubic \bezierlocal curve from \tcode{currPt} to \tcode{ept} using \tcode{cpt1} as the curve's first control point and \tcode{cpt2} as the curve's second control point. Sets \tcode{currPt} to \tcode{ept}. \\ \rowsep

\tcode{path_data::arc p} &
Let \tcode{mm} equal \tcode{mtx}. Let \tcode{mm.m20} equal {0.0f}. Let \tcode{mm.m21} equal \tcode{0.0f}. Creates an arc. It begins at \tcode{currPt}, which is at \tcode{p.start_angle()} radians on the arc and rotates \tcode{p.rotation()} radians. If \tcode{p.rotation()} is positive, rotation is counterclockwise, otherwise it is clockwise. The center of the arc is located at \tcode{p.center(currPt, mm)}. The arc ends at \tcode{p.end_pt(currPt, mm)}. Sets \tcode{currPt} to \tcode{p.end_pt(currPt, mm)}.

\begin{note} \tcode{p.radius()}, which specifies the radius of the arc, is implicitly included in the above statement of effects by the specifications of the center of the arc and the end of the arcs. The use of the current point as the origin for the application of the path group transformation matrix is also implicitly included by the same specifications. \end{note} \\
\end{libreqtab2a}

\addtocounter{SectionDepthBase}{-2}

\addtocounter{SectionDepthBase}{1}
%!TEX root = io2d.tex
\rSec0 [path] {Class \tcode{path}}
%%%%% Rename path to path_group so that a path group contains paths rather than path geometries. Rework all working accordingly and eliminate "sub path" since it is now just "path".
\pnum
\indexlibrary{\idxcode{path}}
The class \tcode{path} contains a path geometry graphics resource that is usable with a \tcode{surface}-derived object.

\pnum
A \tcode{path} object is constructed from the path geometry collection data of a \tcode{path_factory} object. The path geometries of its path geometry graphics resource are immutable, however its path geometry graphics resource can be changed using copy assignment or move assignment.

\pnum
An \tcode{path} object can be default constructed. Default construction of a \tcode{path} object results in a \tcode{path} object which has a path geometry graphics resource that contains no path geometries.

\pnum
When a \tcode{path} object is set on a \tcode{surface} object using 
\tcode{surface::path}, the geometric paths represented by it can be 
stroked or filled.

%\pnum
%A \tcode{path} object shall be usable with any \tcode{surface} or \tcode{surface}-derived object.
%
\rSec1 [path.synopsis] {\tcode{path} synopsis}

\begin{codeblock}
namespace std { namespace experimental { namespace io2d { inline namespace v1 {
  class path {
    public:
    // \ref{path.cons}, construct/copy/destroy:
    explicit path(const path_factory& pb);
    path(const path_factory& pb, error_code& ec) noexcept;
  };
} } } }
\end{codeblock}

\rSec1 [path.cons] {\tcode{path} constructors and assignment operators}

\indexlibrary{\idxcode{path}!constructor}
\begin{itemdecl}
    explicit path(const path_factory& pb);
    path(const path_factory& pb, error_code& ec) noexcept;
\end{itemdecl}
\begin{itemdescr}
	\pnum
	\effects
	Constructs an object of class \tcode{path}. Implementations shall create a path geometry graphics resource from the path geometries contained in \tcode{pb.data_ref()} as if they followed the procedure set forth in \ref{pathgeometries.processing}.

	\pnum
	\throws
	As specified in Error reporting (\ref{\iotwod.err.report}).

	\pnum
	\remarks
	It is unspecified whether a \tcode{path} object shall require further processing when it is passed as an argument to a \tcode{surface} or \tcode{surface}-derived object.
	
	\pnum
	Implementations should avoid or minimize the need for further processing of a \tcode{path} object after it has been constructed.

	\pnum
	\errors
	\tcode{errc::not_enough_memory} if there was a failure to allocate memory.
	
%	\pnum
%	\tcode{io2d_error::no_current_point} if, when processing the path geometries, an operation was encountered which required a current point and the current path geometry had no current point.
%	
%	\pnum
%	\tcode{io2d_error::invalid_matrix} if, when processing the path geometries, an operation was encountered which required the current transformation matrix to be invertible and the matrix was not invertible.
	
\end{itemdescr}

%!TEX root = io2d.tex
\rSec0 [\iotwod.pathbuilder] {Class \tcode{path_builder}}

\pnum
\indexlibrary{\idxcode{path_builder}}%
The class \tcode{path_builder} is a container that stores and manipulates objects of type \tcode{figure_items::figure_item} from which \tcode{interpreted_path} objects are created.

\pnum
A \tcode{path_builder} is a contiguous container. (See [container.requirements.general] in \cppseventeen.)

\pnum
The collection of \tcode{figure_items::figure_item} objects in a path builder is referred to as its path.

\rSec1 [\iotwod.pathbuilder.synopsis] {\tcode{path_builder} synopsis}%

\begin{codeblock}
namespace std::experimental::io2d::v1 {
  template <class Allocator = allocator<figure_items::figure_item>>
  class path_builder {
  public:
    using value_type = figure_items::figure_item;
    using allocator_type = Allocator;
    using reference = value_type&;
    using const_reference = const value_type&;
    using size_type       = @\impdefx{type of \tcode{path_builder::size_type}}@. // See [container.requirements] in \cppseventeen.
    using difference_type = @\impdefx{type of \tcode{path_builder::size_type}}@. // See [container.requirements] in \cppseventeen.
    using iterator       = @\impdefx{type of \tcode{path_builder::iterator}}@. // See [container.requirements] in \cppseventeen.
    using const_iterator = @\impdefx{type of \tcode{path_builder::const_iterator}}@. // See [container.requirements] in \cppseventeen.
    using reverse_iterator       = std::reverse_iterator<iterator>;
    using const_reverse_iterator = std::reverse_iterator<const_iterator>;
    
    // \ref{\iotwod.pathbuilder.cons}, construct, copy, move, destroy:
    path_builder() noexcept(noexcept(Allocator())) :
      path_builder(Allocator()) { }
    explicit path_builder(const Allocator&) noexcept;
    explicit path_builder(size_type n, const Allocator& = Allocator());
    path_builder(size_type n, const value_type& value,
      const Allocator& = Allocator());
    template <class InputIterator>
    path_builder(InputIterator first, InputIterator last,
      const Allocator& = Allocator());
    path_builder(const path_builder& x);
    path_builder(path_builder&&) noexcept;
    path_builder(const path_builder&, const Allocator&);
    path_builder(path_builder&&, const Allocator&);
    path_builder(initializer_list<value_type>, const Allocator& = Allocator());
    ~path_builder();
    path_builder& operator=(const path_builder& x);
    path_builder& operator=(path_builder&& x)
      noexcept(
      allocator_traits<Allocator>::propagate_on_container_move_assignment::value
      ||
      allocator_traits<Allocator>::is_always_equal::value);
    path_builder& operator=(initializer_list<value_type>);
    template <class InputIterator>
    void assign(InputIterator first, InputIterator last);
    void assign(size_type n, const value_type& u);
    void assign(initializer_list<value_type>);
    allocator_type get_allocator() const noexcept;
    
    // \ref{\iotwod.pathbuilder.iterators}, iterators:
    iterator begin() noexcept;
    const_iterator begin() const noexcept;
    const_iterator cbegin() const noexcept;

    iterator end() noexcept;
    const_iterator end() const noexcept;
    const_iterator cend() const noexcept;
    
    reverse_iterator rbegin() noexcept;
    const_reverse_iterator rbegin() const noexcept;
    const_reverse_iterator crbegin() const noexcept;

    reverse_iterator rend() noexcept;
    const_reverse_iterator rend() const noexcept;
    const_reverse_iterator crend() const noexcept;
    
    // \ref{\iotwod.pathbuilder.capacity}, capacity
    bool empty() const noexcept;
    size_type size() const noexcept;
    size_type max_size() const noexcept;
    size_type capacity() const noexcept;
    void resize(size_type sz);
    void resize(size_type sz, const value_type& c);
    void reserve(size_type n);
    void shrink_to_fit();

    // element access:
    reference operator[](size_type n);
    const_reference operator[](size_type n) const;
    const_reference at(size_type n) const;
    reference at(size_type n);
    reference front();
    const_reference front() const;
    reference back();
    const_reference back() const;

    // \ref{\iotwod.pathbuilder.modifiers}, modifiers:
    void new_figure(point_2d pt) noexcept;
    void rel_new_figure(point_2d pt) noexcept;
    void close_figure() noexcept;
    void matrix(const matrix_2d& m) noexcept;
    void rel_matrix(const matrix_2d& m) noexcept;
    void revert_matrix() noexcept;
    void line(point_2d pt) noexcept;
    void rel_line(point_2d dpt) noexcept;
    void quadratic_curve(point_2d pt0, point_2d pt2)
      noexcept;
    void rel_quadratic_curve(point_2d pt0, point_2d pt2)
      noexcept;
    void cubic_curve(point_2d pt0, point_2d pt1,
      point_2d pt2) noexcept;
    void rel_cubic_curve(point_2d dpt0, point_2d dpt1,
      point_2d dpt2) noexcept;
    void arc(point_2d rad, float rot, float sang = pi<float>)
      noexcept;
    
    template <class... Args>
    reference emplace_back(Args&&... args);
    void push_back(const value_type& x);
    void push_back(value_type&& x);
    void pop_back();
    template <class... Args>
    iterator emplace(const_iterator position, Args&&... args);
    iterator insert(const_iterator position, const value_type& x);
    iterator insert(const_iterator position, value_type&& x);
    iterator insert(const_iterator position, size_type n, const value_type& x);
    template <class InputIterator>
    iterator insert(const_iterator position, InputIterator first,
      InputIterator last);
    iterator insert(const_iterator position,
      initializer_list<value_type> il);
    iterator erase(const_iterator position);
    iterator erase(const_iterator first, const_iterator last);
    void swap(path_builder&)
      noexcept(allocator_traits<Allocator>::propagate_on_container_swap::value 
        || allocator_traits<Allocator>::is_always_equal::value);
    void clear() noexcept;
  };
  
  template <class Allocator>
  bool operator==(const path_builder<Allocator>& lhs, 
    const path_builder<Allocator>& rhs);
  template <class Allocator>
  bool operator!=(const path_builder<Allocator>& lhs, 
    const path_builder<Allocator>& rhs);
  
  // \ref{\iotwod.pathbuilder.special}, specialized algorithms:
  template <class Allocator>
  void swap(path_builder<Allocator>& lhs, path_builder<Allocator>& rhs)
    noexcept(noexcept(lhs.swap(rhs)));
}
\end{codeblock}

\rSec1 [\iotwod.pathbuilder.containerrequirements] {\tcode{path_builder} container requirements}

\pnum
This class is a sequence container, as defined in [containers] in \cppseventeen, and all sequence container requirements that apply specifically to \tcode{vector} shall also apply to this class.

\rSec1 [\iotwod.pathbuilder.cons] {\tcode{path_builder} constructors, copy, and assignment}

\indexlibrary{\idxcode{path_builder}!constructor}%
\begin{itemdecl}
explicit path_builder(const Allocator&);
\end{itemdecl}
\begin{itemdescr}
\pnum
\effects
Constructs an empty \tcode{path_builder}, using the specified allocator.

\pnum
\complexity
Constant.
\end{itemdescr}

\indexlibrary{\idxcode{path_builder}!constructor}%
\begin{itemdecl}
explicit path_builder(size_type n, const Allocator& = Allocator());
\end{itemdecl}
\begin{itemdescr}
\pnum
\effects
Constructs a \tcode{path_builder} with \tcode{n} default-inserted elements using the specified allocator.

\pnum
\complexity
Linear in \tcode{n}.
\end{itemdescr}

\indexlibrary{\idxcode{path_builder}!constructor}%
\begin{itemdecl}
path_builder(size_type n, const value_type& value,
  const Allocator& = Allocator());
\end{itemdecl}
\begin{itemdescr}
\pnum
\requires
\tcode{value_type} shall be \tcode{CopyInsertable} into \tcode{*this}.

\pnum
\effects
Constructs a \tcode{path_builder} with n copies of \tcode{value}, using the specified allocator.

\pnum
\complexity
Linear in \tcode{n}.
\end{itemdescr}

\indexlibrary{\idxcode{path_builder}!constructor}%
\begin{itemdecl}
template <class InputIterator>
path_builder(InputIterator first, InputIterator last,
  const Allocator& = Allocator());
\end{itemdecl}
\begin{itemdescr}
\pnum
\effects
Constructs a \tcode{path_builder} equal to the range \range{first}{last}, using the specified allocator.

\pnum
\complexity
Makes only $N$ calls to the copy constructor of \tcode{value_type} (where $N$
is the distance between
\tcode{first}
and
\tcode{last})
and no reallocations if iterators \tcode{first} and \tcode{last} are of forward, bidirectional, or random access categories.
It makes order
\tcode{N}
calls to the copy constructor of
\tcode{value_type}
and order
$\log(N)$
reallocations if they are just input iterators.

\end{itemdescr}

\rSec1 [\iotwod.pathbuilder.capacity] {\tcode{path_builder} capacity}%

\indexlibrarymember{capacity}{path_builder}%
\begin{itemdecl}
size_type capacity() const noexcept;
\end{itemdecl}
\begin{itemdescr}
\pnum
\returns
The total number of elements that the path builder can hold without requiring reallocation.
\end{itemdescr}

\indexlibrarymember{path_builder}{reserve}%
\begin{itemdecl}
void reserve(size_type n);
\end{itemdecl}
\begin{itemdescr}
\pnum
\requires
\tcode{value_type} shall be \tcode{MoveInsertable} into \tcode{*this}.

\pnum
\effects
A directive that informs a path builder of a planned change in size, so that it can manage the storage
allocation accordingly. After \tcode{reserve()}, \tcode{capacity()} is greater or equal to the argument of \tcode{reserve} if
reallocation happens; and equal to the previous value of \tcode{capacity()} otherwise. Reallocation happens
at this point if and only if the current capacity is less than the argument of \tcode{reserve()}. If an exception
is thrown other than by the move constructor of a non-\tcode{CopyInsertable} type, there are no effects.

\pnum
\complexity
It does not change the size of the sequence and takes at most linear time in the size of the
sequence.

\pnum
\throws
\tcode{length_error} if \tcode{n >
max_size()}.\footnote{\tcode{reserve()} uses \tcode{Allocator::allocate()} which
may throw an appropriate exception.}

\pnum
\remarks
Reallocation invalidates all the references, pointers, and iterators
referring to the elements in the sequence.
No reallocation shall take place during insertions that happen
after a call to
\tcode{reserve()}
until the time when an insertion would make the size of the vector
greater than the value of
\tcode{capacity()}.
\end{itemdescr}

\indexlibrarymember{path_builder}{shrink_to_fit}%
\begin{itemdecl}
void shrink_to_fit();
\end{itemdecl}
\begin{itemdescr}
\pnum
\requires
\tcode{value_type} shall be \tcode{MoveInsertable} into \tcode{*this}.

\pnum
\effects
\tcode{shrink_to_fit} is a non-binding request to reduce
\tcode{capacity()} to \tcode{size()}.
\begin{note}
The request is non-binding to allow latitude for
implementation-specific optimizations.
\end{note}
It does not increase \tcode{capacity()}, but may reduce \tcode{capacity()}
by causing reallocation. 
If an exception is thrown other than by the move constructor
of a non-\tcode{CopyInsertable} \tcode{value_type} there are no effects.

\pnum
\complexity Linear in the size of the sequence.

\pnum
\remarks Reallocation invalidates all the references, pointers, and 
iterators referring to the elements in the sequence. If no reallocation 
happens, they remain valid.
\end{itemdescr}

\indexlibrarymember{path_builder}{swap}%
\begin{itemdecl}
void swap(path_builder&)
  noexcept(allocator_traits<Allocator>::propagate_on_container_swap::value ||
  allocator_traits<Allocator>::is_always_equal::value);
\end{itemdecl}
\begin{itemdescr}
\pnum
\effects
Exchanges the contents and
\tcode{capacity()}
of
\tcode{*this}
with that of \tcode{x}.

\pnum
\complexity
Constant time.
\end{itemdescr}

\indexlibrary{path_builder}{resize}%
\begin{itemdecl}
void resize(size_type sz);
\end{itemdecl}
\begin{itemdescr}
\pnum
\effects
If \tcode{sz < size()}, erases the last \tcode{size() - sz} elements
from the sequence. Otherwise, appends \tcode{sz - size()} default-inserted 
elements to the sequence.

\pnum
\requires
\tcode{value_type} shall be
\tcode{MoveInsertable} and \tcode{DefaultInsertable} into \tcode{*this}.

\pnum
\remarks
If an exception is thrown other than by the move constructor of a 
non-\tcode{CopyInsertable}
\tcode{value_type} there are no effects.
\end{itemdescr}

\indexlibrary{path_builder}{resize}%
\begin{itemdecl}
void resize(size_type sz, const value_type& c);
\end{itemdecl}
\begin{itemdescr}
\pnum
\effects
If \tcode{sz < size()}, erases the last \tcode{size() - sz} elements
from the sequence. Otherwise,
appends \tcode{sz - size()} copies of \tcode{c} to the sequence.

\pnum
\requires
\tcode{value_type} shall be \tcode{CopyInsertable} into \tcode{*this}.

\pnum
\remarks
If an exception is thrown there are no effects.
\end{itemdescr}

\rSec1 [\iotwod.pathbuilder.modifiers] {\tcode{path_builder} modifiers}

\indexlibrarymember{path_builder}{new_figure}%
\begin{itemdecl}
void new_figure(point_2d pt) noexcept;
\end{itemdecl}
\begin{itemdescr}
\pnum
\effects
Adds a \tcode{figure_items::figure_item} object constructed from \tcode{figure_items::abs_new_figure(pt)} to the end of the path.
\end{itemdescr}

\indexlibrarymember{path_builder}{rel_new_figure}%
\begin{itemdecl}
void rel_new_figure(point_2d pt) noexcept;
\end{itemdecl}
\begin{itemdescr}
\pnum
\effects
Adds a \tcode{figure_items::figure_item} object constructed from \tcode{figure_items::rel_new_figure(pt)} to the end of the path.
\end{itemdescr}

\indexlibrarymember{path_builder}{close_figure}%
\begin{itemdecl}
void close_figure() noexcept;
\end{itemdecl}
\begin{itemdescr}
\pnum
\requires
The current point contains a value.

\pnum
\effects
Adds a \tcode{figure_items::figure_item} object constructed from \tcode{figure_items::close_figure()} to the end of the path.
\end{itemdescr}

\indexlibrarymember{path_builder}{set_matrix}%
\begin{itemdecl}
void matrix(const matrix_2d& m) noexcept;
\end{itemdecl}
\begin{itemdescr}
\pnum
\requires
The matrix \tcode{m} shall be invertible.

\pnum
\effects
Adds a \tcode{figure_items::figure_item} object constructed from \tcode{(figure_items::abs_matrix(m)} to the end of the path.
\end{itemdescr}

\indexlibrarymember{path_builder}{modify_matrix}%
\begin{itemdecl}
void rel_matrix(const matrix_2d& m) noexcept;
\end{itemdecl}
\begin{itemdescr}
\pnum
\requires
The matrix \tcode{m} shall be invertible.

\pnum
\effects
Adds a \tcode{figure_items::figure_item} object constructed from \tcode{(figure_items::rel_matrix(m)} to the end of the path.
\end{itemdescr}

\indexlibrarymember{path_builder}{revert_matrix}%
\begin{itemdecl}
void revert_matrix() noexcept;
\end{itemdecl}
\begin{itemdescr}
\pnum
\effects
Adds a \tcode{figure_items::figure_item} object constructed from \tcode{(figure_items::revert_matrix()} to the end of the path.
\end{itemdescr}

\indexlibrarymember{path_builder}{line}%
\begin{itemdecl}
void line(point_2d pt) noexcept;
\end{itemdecl}
\begin{itemdescr}
\pnum
Adds a \tcode{figure_items::figure_item} object constructed from \tcode{figure_items::abs_line(pt)} to the end of the path.
\end{itemdescr}

\indexlibrarymember{path_builder}{rel_line}%
\begin{itemdecl}
void rel_line(point_2d dpt) noexcept;
\end{itemdecl}
\begin{itemdescr}
\pnum
\effects
Adds a \tcode{figure_items::figure_item} object constructed from \tcode{figure_items::rel_line(pt)} to the end of the path.
\end{itemdescr}

\indexlibrarymember{path_builder}{quadratic_curve}%
\begin{itemdecl}
void quadratic_curve(point_2d pt0, point_2d pt1) noexcept;
\end{itemdecl}
\begin{itemdescr}
\pnum
\effects
Adds a \tcode{figure_items::figure_item} object constructed from\\ \tcode{figure_items::abs_quadratic_curve(pt0, pt1)} to the end of the path.
\end{itemdescr}

\indexlibrarymember{path_builder}{rel_quadratic_curve}%
\begin{itemdecl}
void rel_quadratic_curve(point_2d dpt0, point_2d dpt1)
  noexcept;
\end{itemdecl}
\begin{itemdescr}
\pnum
\effects
Adds a \tcode{figure_items::figure_item} object constructed from\\ \tcode{figure_items::rel_quadratic_curve(dpt0, dpt1)} to the end of the path.
\end{itemdescr}

\indexlibrarymember{path_builder}{cubic_curve}%
\begin{itemdecl}
void cubic_curve(point_2d pt0, point_2d pt1,
  point_2d pt2) noexcept;
\end{itemdecl}
\begin{itemdescr}
\pnum
\effects
\pnum
Adds a \tcode{figure_items::figure_item} object constructed from \tcode{figure_items::abs_cubic_curve(pt0, pt1, pt2)} to the end of the path.
\end{itemdescr}

\indexlibrarymember{path_builder}{rel_cubic_curve}%
\begin{itemdecl}
void rel_cubic_curve(point_2d dpt0, point_2d dpt1,
  point_2d dpt2) noexcept;
\end{itemdecl}
\begin{itemdescr}
\pnum
\effects
Adds a \tcode{figure_items::figure_item} object constructed from \tcode{figure_items::rel_cubic_curve(dpt0, dpt1, dpt2)} to the end of the path.
\end{itemdescr}

\indexlibrarymember{path_builder}{arc}%
\begin{itemdecl}
void arc(point_2d rad, float rot, float sang) noexcept;
\end{itemdecl}
\begin{itemdescr}
\pnum
\effects
Adds a \tcode{figure_items::figure_item} object constructed from \\ \tcode{figure_items::arc(rad, rot, sang)} to the end of the path.
\end{itemdescr}

\indexlibrarymember{path_builder}{insert}%
\indexlibrarymember{path_builder}{emplace_back}%
\indexlibrarymember{path_builder}{push_back}%
\begin{itemdecl}
iterator insert(const_iterator position, const value_type& x);
iterator insert(const_iterator position, value_type&& x);
iterator insert(const_iterator position, size_type n, const value_type& x);
template <class InputIterator>
iterator insert(const_iterator position, InputIterator first,
  InputIterator last);
iterator insert(const_iterator position, initializer_list<value_type>);
template <class... Args>
reference emplace_back(Args&&... args);
template <class... Args>
iterator emplace(const_iterator position, Args&&... args);
void push_back(const value_type& x);
void push_back(value_type&& x);
\end{itemdecl}

\begin{itemdescr}
\pnum
\remarks
Causes reallocation if the new size is greater than the old capacity.
Reallocation invalidates all the references, pointers, and iterators
referring to the elements in the sequence.
If no reallocation happens, all the iterators and references before the insertion point remain valid.
If an exception is thrown other than by
the copy constructor, move constructor,
assignment operator, or move assignment operator of
\tcode{value_type} or by any \tcode{InputIterator} operation
there are no effects.
If an exception is thrown while inserting a single element at the end and
\tcode{value_type} is \tcode{CopyInsertable} or \tcode{is_nothrow_move_constructible_v<value_type>}
is \tcode{true}, there are no effects.
Otherwise, if an exception is thrown by the move constructor of a non-\tcode{CopyInsertable}
\tcode{value_type}, the effects are unspecified.

\pnum
\complexity
The complexity is linear in the number of elements inserted plus the 
distance to the end of the path builder.
\end{itemdescr}

\indexlibrarymember{erase}{path_builder}%
\indexlibrarymember{pop_back}{path_builder}%
\begin{itemdecl}
iterator erase(const_iterator position);
iterator erase(const_iterator first, const_iterator last);
void pop_back();
\end{itemdecl}

\begin{itemdescr}
\pnum
\effects
Invalidates iterators and references at or after the point of the erase.

\pnum
\complexity
The destructor of \tcode{value_type} is called the number of times equal to 
the number of the elements erased, but the assignment operator
of \tcode{value_type} is called the number of times equal to the number of
elements in the path builder after the erased elements.

\pnum
\throws
Nothing unless an exception is thrown by the copy constructor, move 
constructor, assignment operator, or move assignment operator of
\tcode{value_type}.
\end{itemdescr}

\rSec1 [\iotwod.pathbuilder.iterators] {\tcode{path_builder} iterators}

\indexlibrarymember{begin}{path_builder}%
\indexlibrarymember{cbegin}{path_builder}%
\begin{itemdecl}
iterator begin() noexcept;
const_iterator begin() const noexcept;
const_iterator cbegin() const noexcept;
\end{itemdecl}
\begin{itemdescr}
\pnum
\returns
An iterator referring to the first \tcode{figure_items::figure_item} item in the path.

\pnum
\remarks
Changing a \tcode{figure_items::figure_item} object or otherwise modifying the path in a way that violates the preconditions of that \tcode{figure_items::figure_item} object or of any subsequent \tcode{figure_items::figure_item} object in the path produces undefined behavior when the path is interpreted as described in \ref{\iotwod.paths.interpretation} unless all of the violations are fixed prior to such interpretation.
\end{itemdescr}

\indexlibrarymember{end}{path_builder}%
\indexlibrarymember{cend}{path_builder}%
\begin{itemdecl}
iterator end() noexcept;
const_iterator end() const noexcept;
const_iterator cend() const noexcept;
\end{itemdecl}
\begin{itemdescr}
\pnum
\returns
An iterator which is the past-the-end value.

\pnum
\remarks
Changing a \tcode{figure_items::figure_item} object or otherwise modifying the path in a way that violates the preconditions of that \tcode{figure_items::figure_item} object or of any subsequent \tcode{figure_items::figure_item} object in the path produces undefined behavior when the path is interpreted as described in \ref{\iotwod.paths.interpretation} unless all of the violations are fixed prior to such interpretation.
\end{itemdescr}

\indexlibrarymember{rbegin}{path_builder}%
\indexlibrarymember{crbegin}{path_builder}%
\begin{itemdecl}
reverse_iterator rbegin() noexcept;
const_reverse_iterator rbegin() const noexcept;
const_reverse_iterator crbegin() const noexcept;
\end{itemdecl}
\begin{itemdescr}
\pnum
\returns
An iterator which is semantically equivalent to \tcode{reverse_iterator(end)}.

\pnum
\remarks
Changing a \tcode{figure_items::figure_item} object or otherwise modifying the path in a way that violates the preconditions of that \tcode{figure_items::figure_item} object or of any subsequent \tcode{figure_items::figure_item} object in the path produces undefined behavior when the path is interpreted as described in \ref{\iotwod.paths.interpretation} all of the violations are fixed prior to such interpretation.
\end{itemdescr}

\indexlibrarymember{rend}{path_builder}%
\indexlibrarymember{crend}{path_builder}%
\begin{itemdecl}
reverse_iterator rend() noexcept;
const_reverse_iterator rend() const noexcept;
const_reverse_iterator crend() const noexcept;
\end{itemdecl}
\begin{itemdescr}
\pnum
\returns
An iterator which is semantically equivalent to \tcode{reverse_iterator(begin)}.

\pnum
\remarks
Changing a \tcode{figure_items::figure_item} object or otherwise modifying the path in a way that violates the preconditions of that \tcode{figure_items::figure_item} object or of any subsequent \tcode{figure_items::figure_item} object in the path produces undefined behavior when the path is interpreted as described in \ref{\iotwod.paths.interpretation} unless all of the violations are fixed prior to such interpretation.
\end{itemdescr}

\rSec1[\iotwod.pathbuilder.special] {\tcode{path_builder} specialized algorithms}

\indexlibrary{\idxcode{swap}!\idxcode{path_builder}}%
\begin{itemdecl}
template <class Allocator>
void swap(path_builder<Allocator>& lhs, path_builder<Allocator>& rhs)
  noexcept(noexcept(lhs.swap(rhs)));
\end{itemdecl}
\begin{itemdescr}
\pnum
\effects
As if by \tcode{lhs.swap(rhs)}.
\end{itemdescr}

\addtocounter{SectionDepthBase}{-1}
