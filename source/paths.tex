%!TEX root = io2d.tex

\rSec0 [paths] {Paths}

\pnum
Paths define geometric objects which can be stroked (Table~\ref{tab:surface.rendering.operations}), filled, masked, and used to define or modify a Clip Area (Table~\ref{tab:surface.state.listing}).

\pnum
Paths are created using a \tcode{path_factory} object, which stores a path group. 

\pnum
Paths provide vector graphics functionality. As such they are particularly useful in situations where an application is intended to run on a variety of platforms whose output devices (\ref{displaysurface.intro}) span a large gamut of sizes, both in terms of measurement units and in terms of a horizontal and vertical pixel count, in that order. For example, a pixel count expressed as 1280x720 means that there are 1280 horizontal pixels per row of pixels and 720 vertical pixels per column of pixels for a total of 921600 pixels.
%
%\pnum
%For output devices, the measurement size of a pixel is determined by the physical size of the output device. Many output devices represent pixels as having the same horizontal and vertical measurement sizes. As such, when they display a rendered image which does not have the same horizontal to vertical pixel ratio as the output device, it 

\pnum
A path may contain degenerate path segments because of special rules, which are set forth below.

\pnum
A \tcode{path_group} object is an immutable resource wrapper containing a path group (\ref{pathgroup}). A \tcode{path_group} object is created from a \tcode{path_factory} object. It can also be default constructed, in which case the \tcode{path_group} object contains no paths.

\rSec1 [paths.processing] {Processing paths}

\pnum
This section describes how to convert the path group of a a properly formed \tcode{path_factory} object from a collection of path instruction and path segments to a path group that consists entirely of path segments.

\pnum
The \tcode{native_path_group} class, described below, is informative. It is used to demonstrate how to perform this process. 

\pnum
The \tcode{native_path_group} class has the following state data, the types of which are unspecified:

\begin{libreqtab2}
	{\tcode{native_path_group} state data}
	{tab:paths.processing.natpathgroup}
	\\ \topline
	\lhdr{Name}
	& \rhdr{Use}
	\\ \capsep
	\endfirsthead
	\continuedcaption\\
	\hline
	\lhdr{Name}
	& \rhdr{Use}
	\\ \capsep
	\endhead
	Current Point
	& The start point for a path segment that is added to the Current Path Geometry.
	\\
	Close Point
	& The start point for the initial path segment in the Current Path Geometry.
	\\
	Current Path Geometry
	& The path geometry to which path segments are added.
	\\
	Collection
	& The collection of all path geometries added to the \tcode{native_path_group} object. A new path geometry that is added to the collection is added to the end of the collection.
	\\
\end{libreqtab2}

\begin{codeblock}
	// \textit{This class is }\expos
	class native_path_group {
		public:
		void current_point(const vector_2d& pt) noexcept;
		void close_point(const vector_2d& pt) noexcept;
		void line_to(const vector_2d& pt) noexcept;
		void curve_to(const vector_2d& control1, const vector_2d& control2,
		const vector_2d& endPt) noexcept;
		void close_path() noexcept;
	};
\end{codeblock}

\begin{itemdecl}
	void current_point(const vector_2d& pt) noexcept;
\end{itemdecl}
\begin{itemdescr}
	\pnum
	\effects
	If the last member function called was not \tcode{current_point} or if the Collection contains no path geometries, a new path geometry is created, added to the Collection, and set as the Current Path Geometry.
	
	\pnum
	If a new path geometry is created and the Collection contained at least one path geometry prior to this member function being called, then unless the last member function called was \tcode{close_path}, the previous Current Path Geometry shall be an open path geometry.
	
	\pnum
	Sets \tcode{pt} as the Current Point.
\end{itemdescr}

\begin{itemdecl}
	void close_point(const vector_2d& pt) noexcept;
\end{itemdecl}
\begin{itemdescr}
	\pnum
	\preconditions
	There is a Current Path Geometry.
	
	\pnum
	\effects
	Sets \tcode{pt} as the Close Point.
	
\end{itemdescr}

\begin{itemdecl}
	void line_to(const vector_2d& pt) noexcept;
\end{itemdecl}
\begin{itemdescr}
	\pnum
	\preconditions
	There is a Current Path Geometry.
	
	\pnum
	\effects
	Creates a line segment from the Current Point to \tcode{pt} and adds it to the Current Path Geometry.
	
\end{itemdescr}

\begin{itemdecl}
	void curve_to(const vector_2d& cpt1, const vector_2d& cpt2,
	const vector_2d& endPt) noexcept;
\end{itemdecl}
\begin{itemdescr}
	\pnum
	\preconditions
	There is a Current Path Geometry.
	
	\pnum
	\effects
	Creates a cubic B\'ezier curve from the Current Point to \tcode{endPt} using \tcode{cpt1} as the first control point and \tcode{cpt2} as the second control point and adds it to the Current Path Geometry.
	
\end{itemdescr}

\begin{itemdecl}
	void close_path() noexcept;
\end{itemdecl}
\begin{itemdescr}
	\pnum
	\preconditions
	There is a Current Path Geometry.
	
	\pnum
	\effects
	Creates a line segment from the Current Point to the Close Point.
	
	\pnum
	The Current Path Geometry becomes a closed path geometry.
	
\end{itemdescr}

\pnum
\enternote
A path geometry graphics resource that only supports rendering triangles is possible. The triangles would be used to form lines and to approximate curves. This description assumes the existence of a path geometry graphics resource that performs those actions where needed.
\exitnote

\pnum
The following code shows how to properly process a \tcode{path_factory} object \tcode{p} and store the results into \tcode{native_path_group n}:

\begin{codeblock}
  const double pi =     3.1415926535897932384626433832795;
  const double halfpi = 1.57079632679489661923132169163985;
  const double twopi =  6.283185307179586476925286766559;
  matrix_2d m;
  vector_2d origin;
  vector_2d currentPoint;
  bool hasCurrentPoint = false;
  vector_2d closePoint;
  native_geometry_group n;
  
  for (auto v : p) {
    std::visit([&](auto&& item) {
      using T = std::remove_cv_t<std::remove_reference_t<decltype item>>;

      if constexpr(std::is_same_v<T, path_factory::path_move_to>) {
        currentPoint = item.to();
        auto pt = m.transform_point(currentPoint - origin) + origin;
        n.current_point(pt);
        hasCurrentPoint = true;
        closePoint = pt
        n.close_point(pt);
      }
      else if constexpr(std::is_same_v<T, path_factory::path_line_to>) {
        currentPoint = item.to();
        auto pt = m.transform_point(currentPoint - origin) + origin;
        if (hasCurrentPoint) {
          n.line_to(pt);
        }
        else {
          n.current_point(pt);
          hasCurrentPoint = true;
          closePoint = pt;
          n.close_point(pt);
        }
      }
      else if constexpr(std::is_same_v<T, path_factory::path_curve_to>) {
        auto pt1 = m.transform_point(item.control_point_1() - origin) + origin;
        auto pt2 = m.transform_point(item.control_point_2() - origin) + origin;
        auto pt3 = m.transform_point(item.end_point() - origin) + origin;
        if (!hasCurrentPoint) {
          currentPoint = item.control_point_1();
          n.current_point(pt1);
          hasCurrentPoint = true;
          closePoint = pt1;
          n.close_point(pt1);
        }
        n.curve_to(pt1, pt2, pt3);
        currentPoint = item.end_point();
      }
      else if constexpr(std::is_same_v<T, path_factory::path_new_path>) {
        hasCurrentPoint = false;
      }
      else if constexpr(std::is_same_v<T, path_factory::path_close_path>) {
        if (hasCurrentPoint) {
          n.close_path();
          n.current_point(closePoint);
          auto invM = matrix_2d{m}.invert();
          // Need the untransformed value for currentPoint.
          currentPoint = invM.transform_point(closePoint - origin) + origin;
        }
      }
      else if constexpr(std::is_same_v<T, path_factory::path_rel_move_to>) {
        currentPoint = item.to() + currentPoint;
        auto pt = m.transform_point(currentPoint - origin) + origin;
        n.current_point(pt);
        hasCurrentPoint = true;
        closePoint = pt    
        n.close_point(pt);
      }
      else if constexpr(std::is_same_v<T, path_factory::path_rel_line_to>) {
        currentPoint = item.to() + currentPoint;
        auto pt = m.transform_point(currentPoint - origin) + origin;
        n.line_to(pt);
      }
      else if constexpr(std::is_same_v<T, path_factory::path_rel_curve_to>) {
        auto pt1 = m.transform_point(item.control_point_1() + currentPoint -
        origin) + origin;
        auto pt2 = m.transform_point(item.control_point_2() + currentPoint -
        origin) + origin;
        auto pt3 = m.transform_point(item.end_point() + currentPoint - origin) +
        origin;
        n.curve_to(pt1, pt2, pt3);
        currentPoint = item.end_point() + currentPoint;
      }
      else if constexpr(std::is_same_v<T, path_factory::path_arc_clockwise>) {
        auto ctr = item.center();
        auto rad = item.radius();
        auto ang1 = item.angle_1();
        auto ang2 = item.angle_2();
        while(ang2 < ang1) {
          ang2 += twopi;
        }
        vector_2d pt0, pt1, pt2, pt3;
        int bezCount = 1;
        double theta = ang2 - ang1;
        double phi;
        while (theta >= halfpi) {
          theta /= 2.0;
          bezCount += bezCount;
        }
        phi = theta / 2.0;
        auto cosPhi = cos(phi);
        auto sinPhi = sin(phi);
        pt0.x(cosPhi);
        pt0.y(-sinPhi);
        pt3.x(pt0.x());
        pt3.y(-pt0.y());
        pt1.x((4.0 - cosPhi) / 3.0);
        pt1.y(-(((1.0 - cosPhi) * (3.0 - cosPhi)) / (3.0 * sinPhi)));
        pt2.x(pt1.x());
        pt2.y(-pt1.y());
        phi = -phi;
        auto rotCwFn = [](const vector_2d& pt, double a) -> vector_2d {
          return { pt.x() * cos(a) + pt.y() * sin(a),
            -(pt.x() * -(sin(a)) + pt.y() * cos(a)) };
        };
        pt0 = rotCwFn(pt0, phi);
        pt1 = rotCwFn(pt1, phi);
        pt2 = rotCwFn(pt2, phi);
        pt3 = rotCwFn(pt3, phi);
        
        auto currTheta = ang1;
        const auto startPt =
        ctr + rotCwFn({ pt0.x() * rad, pt0.y() * rad }, currTheta);
        if (hasCurrentPoint) {
          currentPoint = startPt;
          auto pt = m.transform_point(currentPoint - origin) + origin;
          n.line_to(pt);
        }
        else {
          currentPoint = startPt;
          auto pt = m.transform_point(currentPoint - origin) + origin;
          n.current_point(pt);
          hasCurrentPoint = true;
          closePt = pt;
          n.close_point(pt);
        }
        for (; bezCount > 0; bezCount--) {
          auto cpt1 = ctr + rotCwFn({ pt1.x() * rad, pt1.y() * rad }, currTheta);
          auto cpt2 = ctr + rotCwFn({ pt2.x() * rad, pt2.y() * rad }, currTheta);
          auto cpt3 = ctr + rotCwFn({ pt3.x() * rad, pt3.y() * rad }, currTheta);
          currentPoint = cpt3;
          cpt1 = m.transform_point(cpt1 - origin) + origin;
          cpt2 = m.transform_point(cpt2 - origin) + origin;
          cpt3 = m.transform_point(cpt3 - origin) + origin;
          n.curve_to(cpt1, cpt2, cpt3);
          currTheta += theta;
        }
      }
      else if constexpr(std::is_same_v<T, path_factory::path_arc_counterclockwise>) {
      {
        auto ctr = item.center();
        auto rad = item.radius();
        auto ang1 = item.angle_1();
        auto ang2 = item.angle_2();
        while(ang2 > ang1) {
          ang2 -= twopi;
        }
        vector_2d pt0, pt1, pt2, pt3;
        int bezCount = 1;
        double theta = ang1 - ang2;
        double phi;
        while (theta >= halfpi) {
          theta /= 2.0;
          bezCount += bezCount;
        }
        phi = theta / 2.0;
        auto cosPhi = cos(phi);
        auto sinPhi = sin(phi);
        pt0.x(cosPhi);
        pt0.y(-sinPhi);
        pt3.x(pt0.x());
        pt3.y(-pt0.y());
        pt1.x((4.0 - cosPhi) / 3.0);
        pt1.y(-(((1.0 - cosPhi) * (3.0 - cosPhi)) / (3.0 * sinPhi)));
        pt2.x(pt1.x());
        pt2.y(-pt1.y());
        auto rotCwFn = [](const vector_2d& pt, double a) -> vector_2d {
          return { pt.x() * cos(a) + pt.y() * sin(a),
            -(pt.x() * -(sin(a)) + pt.y() * cos(a)) };
        };
        pt0 = rotCwFn(pt0, phi);
        pt1 = rotCwFn(pt1, phi);
        pt2 = rotCwFn(pt2, phi);
        pt3 = rotCwFn(pt3, phi);
        auto shflPt = pt3;
        pt3 = pt0;
        pt0 = shflPt;
        shflPt = pt2;
        pt2 = pt1;
        pt1 = shflPt;
        auto currTheta = ang1;
        const auto startPt =
        ctr + rotCwFn({ pt0.x() * rad, pt0.y() * rad }, currTheta);
        if (hasCurrentPoint) {
          currentPoint = startPt;
          auto pt = m.transform_point(currentPoint - origin) + origin;
          n.line_to(pt);
        }
        else {
          currentPoint = startPt;
          auto pt = m.transform_point(currentPoint - origin) + origin;
          n.current_point(pt);
          hasCurrentPoint = true;
          closePt = pt;
          n.close_point(pt);
        }
        for (; bezCount > 0; bezCount--) {
          auto cpt1 = ctr + rotCwFn({ pt1.x() * rad, pt1.y() * rad }, currTheta);
          auto cpt2 = ctr + rotCwFn({ pt2.x() * rad, pt2.y() * rad }, currTheta);
          auto cpt3 = ctr + rotCwFn({ pt3.x() * rad, pt3.y() * rad }, currTheta);
          currentPoint = cpt3;
          cpt1 = m.transform_point(cpt1 - origin) + origin;
          cpt2 = m.transform_point(cpt2 - origin) + origin;
          cpt3 = m.transform_point(cpt3 - origin) + origin;
          n.curve_to(cpt1, cpt2, cpt3);
          currTheta -= theta;
        }
      }
      else if constexpr(std::is_same_v<T, path_factory::path_change_matrix>) {
        m = item.matrix();
      }
      else if constexpr(std::is_same_v<T, path_factory::path_change_origin>) {
        origin = item.origin();
      }
    }, v);
  }
\end{codeblock}

%\rSec1 [pathgeometries.strokerules] {Stroking path geometries}
%
%\pnum
%The following rules shall apply when a Stroking operation (\ref{surface.stroking}) is carried out on a path geometry.
%
%\begin{enumerate}
%\item If the path geometry only contains a degenerate path segment, then if the \tcode{line_cap} value is
%\end{enumerate}
%
%\begin{enumerate}
%  \item Except as otherwise specified in these rules, the start point and end point of a path segment shall be rendered as specified by the meaning of the surface's current \tcode{line_cap} value (\ref{linecap}).
%  
%  \item If the end point of a path segment \textit{a} is set as the current point and is then used as the start point of another path segment, \textit{b}, the point where \tcode{a}'s end point meets \tcode{b}'s start point shall be rendered as specified by the meaning of the surface's current \tcode{line_join} value (\ref{linejoin}).
%  
%  \item ***FIXME***
%\end{enumerate}
%
%\rSec1 [pathgeometries.fillrules] {Filling path geometries}
%
%\pnum
%***FIXME***

\addtocounter{SectionDepthBase}{1}
%%!TEX root = io2d.tex
%\rSec0 [pathgeometries] {Path geometries}


%%!TEX root = io2d.tex
\rSec0 [\iotwod.pathdatatype] {Enum class \tcode{path_data_type}}

\rSec1 [\iotwod.pathdatatype.summary] {\tcode{path_data_type} Summary}

\pnum
The \tcode{path_data_type} enum class specifies the polymorphic type of a 
\tcode{path_data} object.
See Table~\ref{tab:\iotwod.pathdatatype.meanings} for the meaning of each
\tcode{path_data_type} enumerator.

\rSec1 [\iotwod.pathdatatype.synopsis] {\tcode{path_data_type} Synopsis}

\begin{codeblock}
namespace std { namespace experimental { namespace io2d { inline namespace v1 {
  enum class path_data_type {
    move_to,
    line_to,
    curve_to,
    new_sub_path,
    close_path,
    rel_move_to,
    rel_line_to,
    rel_curve_to,
    arc,
    arc_negative,
    change_matrix,
    change_origin
  };
} } } }
\end{codeblock}

\rSec1 [\iotwod.pathdatatype.enumerators] {\tcode{path_data_type} Enumerators}

\begin{libreqtab2}
 {\tcode{path_data_type} enumerator meanings}
 {tab:\iotwod.pathdatatype.meanings}
 \\ \topline
 \lhdr{Enumerator}
 & \rhdr{Meaning}
 \\ \capsep
 \endfirsthead
 \continuedcaption\\
 \hline
 \lhdr{Enumerator}
 & \rhdr{Meaning}
 \\ \capsep
 \endhead
 \tcode{move_to}
 & The object is of type \tcode{move_to}.
 \\
 \tcode{line_to}
 & The object is of type \tcode{line_to}.
 \\
 \tcode{curve_to}
 & The object is of type \tcode{curve_to}.
 \\
 \tcode{new_sub_path}
 & The object is of type \tcode{new_sub_path}.
 \\
 \tcode{close_path}
 & The object is of type \tcode{close_path}.
 \\
 \tcode{rel_move_to}
 & The object is of type \tcode{rel_move_to}.
 \\
 \tcode{rel_line_to}
 & The object is of type \tcode{rel_line_to}.
 \\
 \tcode{rel_curve_to}
 & The object is of type \tcode{rel_curve_to}.
 \\
 \tcode{arc}
 & The object is of type \tcode{arc}.
 \\
 \tcode{arc_negative}
 & The object is of type \tcode{arc_negative}.
 \\
 \tcode{change_matrix}
 & The object is of type \tcode{change_matrix}.
 \\
 \tcode{change_origin}
 & The object is of type \tcode{change_origin}.
 \\
\end{libreqtab2}

%%!TEX root = io2d.tex
\rSec0 [pathdataitem] {Class \tcode{path_data_item}}

\rSec1 [pathdataitem.synopsis] {\tcode{path_data_item} synopsis}

\begin{codeblock}
namespace std { namespace experimental { namespace io2d { inline namespace v1 {
  class path_data_item {
  public:
    // \ref{pathdataitem.cons}, construct/copy/move/destroy:
    path_data_item() noexcept;
    path_data_item(const path_data_item& other) noexcept;
    path_data_item& operator=(const path_data_item& other) noexcept;
    path_data_item(path_data_item&& other) noexcept;
    path_data_item& operator=(path_data_item&& other) noexcept;
    path_data_item(const arc& value) noexcept;
    path_data_item(const arc_negative& value) noexcept;
    path_data_item(const change_matrix& value) noexcept;
    path_data_item(const change_origin& value) noexcept;
    path_data_item(const close_path& value) noexcept;
    path_data_item(const curve_to& value) noexcept;
    path_data_item(const rel_curve_to& value) noexcept;
    path_data_item(const new_sub_path& value) noexcept;
    path_data_item(const line_to& value) noexcept;
    path_data_item(const move_to& value) noexcept;
    path_data_item(const rel_line_to& value) noexcept;
    path_data_item(const rel_move_to& value) noexcept;

    // \ref{pathdataitem.modifiers}, modifiers:
    void assign(const arc& value) noexcept;
    void assign(const arc_negative& value) noexcept;
    void assign(const change_matrix& value) noexcept;
    void assign(const change_origin& value) noexcept;
    void assign(const close_path& value) noexcept;
    void assign(const curve_to& value) noexcept;
    void assign(const rel_curve_to& value) noexcept;
    void assign(const new_sub_path& value) noexcept;
    void assign(const line_to& value) noexcept;
    void assign(const move_to& value) noexcept;
    void assign(const rel_line_to& value) noexcept;
    void assign(const rel_move_to& value) noexcept;

    // \ref{pathdataitem.observers}, observers:
    bool has_data() const noexcept;
    path_data_type type() const;
    path_data_type type(::std::error_code& ec) const noexcept;

    template <class T>
    T get() const;
    template <class T>
    T get(::std::error_code& ec) const noexcept;

    template <>
    arc get() const;
    template <>
    arc get(::std::error_code& ec) const noexcept;
    template <>
    arc_negative get() const;
    template <>
    arc_negative get(::std::error_code& ec) const noexcept;
    template <>
    change_matrix get() const;
    template <>
    change_matrix get(::std::error_code& ec) const noexcept;
    template <>
    change_origin get() const;
    template <>
    change_origin get(::std::error_code& ec) const noexcept;
    template <>
    close_path get() const;
    template <>
    close_path get(::std::error_code& ec) const noexcept;
    template <>
    curve_to get() const;
    template <>
    curve_to get(::std::error_code& ec) const noexcept;
    template <>
    rel_curve_to get() const;
    template <>
    rel_curve_to get(::std::error_code& ec) const noexcept;
    template <>
    new_sub_path get() const;
    template <>
    new_sub_path get(::std::error_code& ec) const noexcept;
    template <>
    line_to get() const;
    template <>
    line_to get(::std::error_code& ec) const noexcept;
    template <>
    move_to get() const;
    template <>
    move_to get(::std::error_code& ec) const noexcept;
    template <>
    rel_line_to get() const;
    template <>
    rel_line_to get(::std::error_code& ec) const noexcept;
    template <>
    rel_move_to get() const;
    template <>
    rel_move_to get(::std::error_code& ec) const noexcept;

  private:
    bool _Has_data;         // \expos
    union {
      struct {
        double centerX;
        double centerY;
        double radius;
        double angle1;
        double angle2;
      } arc;
      struct {
        double m00;
        double m01;
        double m10;
        double m11;
        double m20;
        double m21;
      } matrix;
      struct {
        double cpt1x;
        double cpt1y;
        double cpt2x;
        double cpt2y;
        double eptx;
        double epty;
      } curve;
      struct {
        double x;
        double y;
      } point;
    } _Data;               // \expos

    path_data_type _Type;  // \expos
  };
} } } }
\end{codeblock}

\rSec1 [pathdataitem.intro] {\tcode{path_data_item} Description}

\pnum
\indexlibrary{\idxcode{path_data_item}}
The class \tcode{path_data_item} describes an opaque container capable of storing and retrieving an object of a type derived from \tcode{path_data}.

\rSec1 [pathdataitem.cons] {\tcode{path_data_item} constructors and assignment operators}

\indexlibrary{\idxcode{path_data_item}!constructor}
\begin{itemdecl}
    path_data_item() noexcept;
\end{itemdecl}
\begin{itemdescr}
	\pnum
	\effects
	Constructs an object of type \tcode{path_data_item}.
	
	\pnum
	\postconditions
	\tcode{_Has_data == false}.

\end{itemdescr}

\indexlibrary{\idxcode{path_data_item}!constructor}
\begin{itemdecl}
    path_data_item(const arc& value) noexcept;
\end{itemdecl}
\begin{itemdescr}
	\pnum
	\effects
	Constructs an object of type \tcode{path_data_item}.
	
	\pnum
	\postconditions
	\tcode{_Has_data == true}.
	
	\pnum
	\tcode{_Type == path_data_type::arc}.
	
	\pnum
	\tcode{_Data.arc.centerX == value.center().x()}.
	
	\pnum
	\tcode{_Data.arc.centerY == value.center().y()}.
	
	\pnum
	\tcode{_Data.arc.radius == value.radius()}.
	
	\pnum
	\tcode{_Data.arc.angle1 == value.angle_1()}.
	
	\pnum
	\tcode{_Data.arc.angle2 == value.angle_2()}.
	
\end{itemdescr}

\indexlibrary{\idxcode{path_data_item}!constructor}
\begin{itemdecl}
    path_data_item(const arc_negative& value) noexcept;
\end{itemdecl}
\begin{itemdescr}
	\pnum
	\effects
	Constructs an object of type \tcode{path_data_item}.
	
	\pnum
	\postconditions
	\tcode{_Has_data == true}.
	
	\pnum
	\tcode{_Type == path_data_type::arc_negative}.
	
	\pnum
	\tcode{_Data.arc.centerX == value.center().x()}.
	
	\pnum
	\tcode{_Data.arc.centerY == value.center().y()}.
	
	\pnum
	\tcode{_Data.arc.radius == value.radius()}.
	
	\pnum
	\tcode{_Data.arc.angle1 == value.angle_1()}.
	
	\pnum
	\tcode{_Data.arc.angle2 == value.angle_2()}.

\end{itemdescr}

\indexlibrary{\idxcode{path_data_item}!constructor}
\begin{itemdecl}
    path_data_item(const change_matrix& value) noexcept;
\end{itemdecl}
\begin{itemdescr}
	\pnum
	\effects
	Constructs an object of type \tcode{path_data_item}.
	
	\pnum
	\postconditions
	\tcode{_Has_data == true}.
	
	\pnum
	\tcode{_Type == path_data_type::change_matrix}.
	
	\pnum
	\tcode{_Data.matrix.m00 == value.matrix().m00()}.
	
	\pnum
	\tcode{_Data.matrix.m01 == value.matrix().m01()}.
	
	\pnum
	\tcode{_Data.matrix.m10 == value.matrix().m10()}.
	
	\pnum
	\tcode{_Data.matrix.m11 == value.matrix().m11()}.
	
	\pnum
	\tcode{_Data.matrix.m20 == value.matrix().m20()}.
	
	\pnum
	\tcode{_Data.matrix.m21 == value.matrix().m21()}.
	
\end{itemdescr}

\indexlibrary{\idxcode{path_data_item}!constructor}
\begin{itemdecl}
    path_data_item(const change_origin& value) noexcept;
\end{itemdecl}
\begin{itemdescr}
	\pnum
	\effects
	Constructs an object of type \tcode{path_data_item}.
	
	\pnum
	\postconditions
	\tcode{_Has_data == true}.
	
	\pnum
	\tcode{_Type == path_data_type::change_origin}.

	\pnum
	\tcode{_Data.point.x == value.origin().x()}.	

	\pnum
	\tcode{_Data.point.y == value.origin().y()}.	
	
\end{itemdescr}

\indexlibrary{\idxcode{path_data_item}!constructor}
\begin{itemdecl}
    path_data_item(const close_path& value) noexcept;
\end{itemdecl}
\begin{itemdescr}
	\pnum
	\effects
	Constructs an object of type \tcode{path_data_item}.
	
	\pnum
	\postconditions
	\tcode{_Has_data == true}.
	
	\pnum
	\tcode{_Type == path_data_type::close_path}.
	
\end{itemdescr}

\indexlibrary{\idxcode{path_data_item}!constructor}
\begin{itemdecl}
    path_data_item(const curve_to& value) noexcept;
\end{itemdecl}
\begin{itemdescr}
	\pnum
	\effects
	Constructs an object of type \tcode{path_data_item}.
	
	\pnum
	\postconditions
	\tcode{_Has_data == true}.
	
	\pnum
	\tcode{_Type == path_data_type::curve_to}.
	
	\pnum
	\tcode{_Data.curve.cpt1x == value.control_point_1().x()}.
	
	\pnum
	\tcode{_Data.curve.cpt1y == value.control_point_1().y()}.
	
	\pnum
	\tcode{_Data.curve.cpt2x == value.control_point_2().x()}.
	
	\pnum
	\tcode{_Data.curve.cpt2y == value.control_point_2().y()}.
	
	\pnum
	\tcode{_Data.curve.eptx == value.end_point().x()}.
	
	\pnum
	\tcode{_Data.curve.epty == value.end_point().y()}.
	
\end{itemdescr}

\indexlibrary{\idxcode{path_data_item}!constructor}
\begin{itemdecl}
    path_data_item(const rel_curve_to& value) noexcept;
\end{itemdecl}
\begin{itemdescr}
	\pnum
	\effects
	Constructs an object of type \tcode{path_data_item}.
	
	\pnum
	\postconditions
	\tcode{_Has_data == true}.
	
	\pnum
	\tcode{_Type == path_data_type::rel_curve_to}.
	
	\pnum
	\tcode{_Data.curve.cpt1x == value.control_point_1().x()}.
	
	\pnum
	\tcode{_Data.curve.cpt1y == value.control_point_1().y()}.
	
	\pnum
	\tcode{_Data.curve.cpt2x == value.control_point_2().x()}.
	
	\pnum
	\tcode{_Data.curve.cpt2y == value.control_point_2().y()}.
	
	\pnum
	\tcode{_Data.curve.eptx == value.end_point().x()}.
	
	\pnum
	\tcode{_Data.curve.epty == value.end_point().y()}.

\end{itemdescr}

\indexlibrary{\idxcode{path_data_item}!constructor}
\begin{itemdecl}
    path_data_item(const new_sub_path& value) noexcept;
\end{itemdecl}
\begin{itemdescr}
	\pnum
	\effects
	Constructs an object of type \tcode{path_data_item}.
	
	\pnum
	\postconditions
	\tcode{_Has_data == true}.
	
	\pnum
	\tcode{_Type == path_data_type::new_sub_path}.
	
\end{itemdescr}

\indexlibrary{\idxcode{path_data_item}!constructor}
\begin{itemdecl}
    path_data_item(const line_to& value) noexcept;
\end{itemdecl}
\begin{itemdescr}
	\pnum
	\effects
	Constructs an object of type \tcode{path_data_item}.
	
	\pnum
	\postconditions
	\tcode{_Has_data == true}.
	
	\pnum
	\tcode{_Type == path_data_type::line_to}.

	\pnum
	\tcode{_Data.point.x == value.to().x()}.	

	\pnum
	\tcode{_Data.point.y == value.to().y()}.	
	
\end{itemdescr}

\indexlibrary{\idxcode{path_data_item}!constructor}
\begin{itemdecl}
    path_data_item(const move_to& value) noexcept;
\end{itemdecl}
\begin{itemdescr}
	\pnum
	\effects
	Constructs an object of type \tcode{path_data_item}.
	
	\pnum
	\postconditions
	\tcode{_Has_data == true}.
	
	\pnum
	\tcode{_Type == path_data_type::move_to}.

	\pnum
	\tcode{_Data.point.x == value.to().x()}.	

	\pnum
	\tcode{_Data.point.y == value.to().y()}.	
	
\end{itemdescr}

\indexlibrary{\idxcode{path_data_item}!constructor}
\begin{itemdecl}
    path_data_item(const rel_line_to& value) noexcept;
\end{itemdecl}
\begin{itemdescr}
	\pnum
	\effects
	Constructs an object of type \tcode{path_data_item}.
	
	\pnum
	\postconditions
	\tcode{_Has_data == true}.
	
	\pnum
	\tcode{_Type == path_data_type::rel_line_to}.

	\pnum
	\tcode{_Data.point.x == value.to().x()}.	

	\pnum
	\tcode{_Data.point.y == value.to().y()}.	
	
\end{itemdescr}

\indexlibrary{\idxcode{path_data_item}!constructor}
\begin{itemdecl}
    path_data_item(const rel_move_to& value) noexcept;
\end{itemdecl}
\begin{itemdescr}
	\pnum
	\effects
	Constructs an object of type \tcode{path_data_item}.
	
	\pnum
	\postconditions
	\tcode{_Has_data == true}.
	
	\pnum
	\tcode{_Type == path_data_type::rel_move_to}.

	\pnum
	\tcode{_Data.point.x == value.to().x()}.	

	\pnum
	\tcode{_Data.point.y == value.to().y()}.	
	
\end{itemdescr}

\rSec1 [pathdataitem.modifiers] {\tcode{path_data_item} modifiers}

\indexlibrary{\idxcode{path_data_item}!\idxcode{assign}}
\indexlibrary{\idxcode{assign}!\idxcode{path_data_item}}
\begin{itemdecl}
    void assign(const arc& value) noexcept;
\end{itemdecl}
\begin{itemdescr}
	\pnum
	\postconditions
	\tcode{_Has_data == true}.
	
	\pnum
	\tcode{_Type == path_data_type::arc}.
	
	\pnum
	\tcode{_Data.arc.centerX == value.center().x()}.
	
	\pnum
	\tcode{_Data.arc.centerY == value.center().y()}.
	
	\pnum
	\tcode{_Data.arc.radius == value.radius()}.
	
	\pnum
	\tcode{_Data.arc.angle1 == value.angle_1()}.
	
	\pnum
	\tcode{_Data.arc.angle2 == value.angle_2()}.
	
\end{itemdescr}

\indexlibrary{\idxcode{path_data_item}!\idxcode{assign}}
\indexlibrary{\idxcode{assign}!\idxcode{path_data_item}}
\begin{itemdecl}
    void assign(const arc_negative& value) noexcept;
\end{itemdecl}
\begin{itemdescr}
	\pnum
	\postconditions
	\tcode{_Has_data == true}.
	
	\pnum
	\tcode{_Type == path_data_type::arc_negative}.
	
	\pnum
	\tcode{_Data.arc.centerX == value.center().x()}.
	
	\pnum
	\tcode{_Data.arc.centerY == value.center().y()}.
	
	\pnum
	\tcode{_Data.arc.radius == value.radius()}.
	
	\pnum
	\tcode{_Data.arc.angle1 == value.angle_1()}.
	
	\pnum
	\tcode{_Data.arc.angle2 == value.angle_2()}.
	
\end{itemdescr}

\indexlibrary{\idxcode{path_data_item}!\idxcode{assign}}
\indexlibrary{\idxcode{assign}!\idxcode{path_data_item}}
\begin{itemdecl}
    void assign(const change_matrix& value) noexcept;
\end{itemdecl}
\begin{itemdescr}
	\pnum
	\postconditions
	\tcode{_Has_data == true}.
	
	\pnum
	\tcode{_Type == path_data_type::change_matrix}.
	
	\pnum
	\tcode{_Data.matrix.m00 == value.matrix().m00()}.
	
	\pnum
	\tcode{_Data.matrix.m01 == value.matrix().m01()}.
	
	\pnum
	\tcode{_Data.matrix.m10 == value.matrix().m10()}.
	
	\pnum
	\tcode{_Data.matrix.m11 == value.matrix().m11()}.
	
	\pnum
	\tcode{_Data.matrix.m20 == value.matrix().m20()}.
	
	\pnum
	\tcode{_Data.matrix.m21 == value.matrix().m21()}.
	
\end{itemdescr}

\indexlibrary{\idxcode{path_data_item}!\idxcode{assign}}
\indexlibrary{\idxcode{assign}!\idxcode{path_data_item}}
\begin{itemdecl}
    void assign(const change_origin& value) noexcept;
\end{itemdecl}
\begin{itemdescr}
	\pnum
	\postconditions
	\tcode{_Has_data == true}.
	
	\pnum
	\tcode{_Type == path_data_type::change_origin}.

	\pnum
	\tcode{_Data.point.x == value.origin().x()}.	

	\pnum
	\tcode{_Data.point.y == value.origin().y()}.	
	
\end{itemdescr}

\indexlibrary{\idxcode{path_data_item}!\idxcode{assign}}
\indexlibrary{\idxcode{assign}!\idxcode{path_data_item}}
\begin{itemdecl}
    void assign(const close_path& value) noexcept;
\end{itemdecl}
\begin{itemdescr}
	\pnum
	\postconditions
	\tcode{_Has_data == true}.
	
	\pnum
	\tcode{_Type == path_data_type::close_path}.

\end{itemdescr}

\indexlibrary{\idxcode{path_data_item}!\idxcode{assign}}
\indexlibrary{\idxcode{assign}!\idxcode{path_data_item}}
\begin{itemdecl}
    void assign(const curve_to& value) noexcept;
\end{itemdecl}
\begin{itemdescr}
	\pnum
	\postconditions
	\tcode{_Has_data == true}.
	
	\pnum
	\tcode{_Type == path_data_type::curve_to}.
	
	\pnum
	\tcode{_Data.curve.cpt1x == value.control_point_1().x()}.
	
	\pnum
	\tcode{_Data.curve.cpt1y == value.control_point_1().y()}.
	
	\pnum
	\tcode{_Data.curve.cpt2x == value.control_point_2().x()}.
	
	\pnum
	\tcode{_Data.curve.cpt2y == value.control_point_2().y()}.
	
	\pnum
	\tcode{_Data.curve.eptx == value.end_point().x()}.
	
	\pnum
	\tcode{_Data.curve.epty == value.end_point().y()}.
	
\end{itemdescr}

\indexlibrary{\idxcode{path_data_item}!\idxcode{assign}}
\indexlibrary{\idxcode{assign}!\idxcode{path_data_item}}
\begin{itemdecl}
    void assign(const rel_curve_to& value) noexcept;
\end{itemdecl}
\begin{itemdescr}
	\pnum
	\postconditions
	\tcode{_Has_data == true}.
	
	\pnum
	\tcode{_Type == path_data_type::rel_curve_to}.
	
	\pnum
	\tcode{_Data.curve.cpt1x == value.control_point_1().x()}.
	
	\pnum
	\tcode{_Data.curve.cpt1y == value.control_point_1().y()}.
	
	\pnum
	\tcode{_Data.curve.cpt2x == value.control_point_2().x()}.
	
	\pnum
	\tcode{_Data.curve.cpt2y == value.control_point_2().y()}.
	
	\pnum
	\tcode{_Data.curve.eptx == value.end_point().x()}.
	
	\pnum
	\tcode{_Data.curve.epty == value.end_point().y()}.
	
\end{itemdescr}

\indexlibrary{\idxcode{path_data_item}!\idxcode{assign}}
\indexlibrary{\idxcode{assign}!\idxcode{path_data_item}}
\begin{itemdecl}
    void assign(const new_sub_path& value) noexcept;
\end{itemdecl}
\begin{itemdescr}
	\pnum
	\postconditions
	\tcode{_Has_data == true}.
	
	\pnum
	\tcode{_Type == path_data_type::new_sub_path}.
	
\end{itemdescr}

\indexlibrary{\idxcode{path_data_item}!\idxcode{assign}}
\indexlibrary{\idxcode{assign}!\idxcode{path_data_item}}
\begin{itemdecl}
    void assign(const line_to& value) noexcept;
\end{itemdecl}
\begin{itemdescr}
	\pnum
	\postconditions
	\tcode{_Has_data == true}.
	
	\pnum
	\tcode{_Type == path_data_type::line_to}.

	\pnum
	\tcode{_Data.point.x == value.to().x()}.	

	\pnum
	\tcode{_Data.point.y == value.to().y()}.	
	
\end{itemdescr}

\indexlibrary{\idxcode{path_data_item}!\idxcode{assign}}
\indexlibrary{\idxcode{assign}!\idxcode{path_data_item}}
\begin{itemdecl}
    void assign(const move_to& value) noexcept;
\end{itemdecl}
\begin{itemdescr}
	\pnum
	\postconditions
	\tcode{_Has_data == true}.
	
	\pnum
	\tcode{_Type == path_data_type::move_to}.

	\pnum
	\tcode{_Data.point.x == value.to().x()}.	

	\pnum
	\tcode{_Data.point.y == value.to().y()}.	
	
\end{itemdescr}

\indexlibrary{\idxcode{path_data_item}!\idxcode{assign}}
\indexlibrary{\idxcode{assign}!\idxcode{path_data_item}}
\begin{itemdecl}
    void assign(const rel_line_to& value) noexcept;
\end{itemdecl}
\begin{itemdescr}
	\pnum
	\postconditions
	\tcode{_Has_data == true}.
	
	\pnum
	\tcode{_Type == path_data_type::rel_line_to}.

	\pnum
	\tcode{_Data.point.x == value.to().x()}.	

	\pnum
	\tcode{_Data.point.y == value.to().y()}.	
	
\end{itemdescr}

\indexlibrary{\idxcode{path_data_item}!\idxcode{assign}}
\indexlibrary{\idxcode{assign}!\idxcode{path_data_item}}
\begin{itemdecl}
    void assign(const rel_move_to& value) noexcept;
\end{itemdecl}
\begin{itemdescr}
	\pnum
	\postconditions
	\tcode{_Has_data == true}.
	
	\pnum
	\tcode{_Type == path_data_type::rel_move_to}.

	\pnum
	\tcode{_Data.point.x == value.to().x()}.	

	\pnum
	\tcode{_Data.point.y == value.to().y()}.	
	
\end{itemdescr}

\rSec1 [pathdataitem.observers] {\tcode{path_data_item} observers}

\indexlibrary{\idxcode{path_data_item}!\idxcode{has_data}}
\indexlibrary{\idxcode{has_data}!\idxcode{path_data_item}}
\begin{itemdecl}
    bool has_data() const noexcept;
\end{itemdecl}
\begin{itemdescr}
	\pnum
	\returns
	\tcode{_Has_data}.
	
\end{itemdescr}

\indexlibrary{\idxcode{path_data_item}!\idxcode{type}}
\indexlibrary{\idxcode{type}!\idxcode{path_data_item}}
\begin{itemdecl}
    path_data_type type() const;
    path_data_type type(::std::error_code& ec) const noexcept;
\end{itemdecl}
\begin{itemdescr}
	\pnum
	\preconditions
	\tcode{_Has_data == true}.
	
	\pnum
	\returns
	\tcode{_Type}.
	
	\pnum
	\throws
	As specified in Error reporting (\ref{\iotwod.err.report}).
	
	\pnum
	\errors
	\tcode{errc::operation_not_permitted} if the preconditions are violated.

\end{itemdescr}

\indexlibrary{\idxcode{path_data_item}!\idxcode{get}}
\indexlibrary{\idxcode{get}!\idxcode{path_data_item}}
\begin{itemdecl}
    template <>
    arc get() const;
    template <>
    arc get(::std::error_code& ec) const noexcept;
\end{itemdecl}
\begin{itemdescr}
	\pnum
	\preconditions
	\tcode{_Has_data == true}.
	
	\pnum
	\tcode{_Type == path_data_type::arc}.
	
	\pnum
	\returns
	\tcode{arc\{vector_2d\{ _Data.arc.centerX, _Data.arc.centerY \},
	  _Data.arc.radius, _Data.arc.angle1, _Data.arc.angle2 \}}.
	
	\pnum
	\throws
	As specified in Error reporting (\ref{\iotwod.err.report}).
	
	\pnum
	\remarks
	If a non-throwing error occurs, this function returns a default constructed object of its specified return type.
	
	\pnum
	\errors
	\tcode{errc::operation_not_permitted} if \tcode{!_Has_data}.
	
	\pnum
	\tcode{errc::invalid_argument} if \tcode{_Type != path_data_type::arc}.

\end{itemdescr}

\indexlibrary{\idxcode{path_data_item}!\idxcode{get}}
\indexlibrary{\idxcode{get}!\idxcode{path_data_item}}
\begin{itemdecl}
    template <>
    arc_negative get() const;
    template <>
    arc_negative get(::std::error_code& ec) const noexcept;
\end{itemdecl}
\begin{itemdescr}
	\pnum
	\preconditions
	\tcode{_Has_data == true}.
	
	\pnum
	\tcode{_Type == path_data_type::arc_negative}.
	
	\pnum
	\returns
	\tcode{arc_negative\{vector_2d\{ _Data.arc.centerX, _Data.arc.centerY \},
	  _Data.arc.radius, _Data.arc.angle1, _Data.arc.angle2 \}}.
	
	\pnum
	\throws
	As specified in Error reporting (\ref{\iotwod.err.report}).
	
	\pnum
	\remarks
	If a non-throwing error occurs, this function returns a default constructed object of its specified return type.
	
	\pnum
	\errors
	\tcode{errc::operation_not_permitted} if \tcode{!_Has_data}.
	
	\pnum
	\tcode{errc::invalid_argument} if \tcode{_Type != path_data_type::arc_negative}.

\end{itemdescr}

\indexlibrary{\idxcode{path_data_item}!\idxcode{get}}
\indexlibrary{\idxcode{get}!\idxcode{path_data_item}}
\begin{itemdecl}
    template <>
    change_matrix get() const;
    template <>
    change_matrix get(::std::error_code& ec) const noexcept;
\end{itemdecl}
\begin{itemdescr}
	\pnum
	\preconditions
	\tcode{_Has_data == true}.
	
	\pnum
	\tcode{_Type == path_data_type::change_matrix}.
	
	\pnum
	\returns
	\tcode{change_matrix\{ matrix_2d\{ _Data.matrix.m00, _Data.matrix.m01, _Data.matrix.m10, _Data.matrix.m11, _Data.matrix.m20, _Data.matrix.m21 \} \}}.
	
	\pnum
	\throws
	As specified in Error reporting (\ref{\iotwod.err.report}).
	
	\pnum
	\remarks
	If a non-throwing error occurs, this function returns a default constructed object of its specified return type.
	
	\pnum
	\errors
	\tcode{errc::operation_not_permitted} if \tcode{!_Has_data}.
	
	\pnum
	\tcode{errc::invalid_argument} if \tcode{_Type != path_data_type::change_matrix}.

\end{itemdescr}

\indexlibrary{\idxcode{path_data_item}!\idxcode{get}}
\indexlibrary{\idxcode{get}!\idxcode{path_data_item}}
\begin{itemdecl}
    template <>
    change_origin get() const;
    template <>
    change_origin get(::std::error_code& ec) const noexcept;
\end{itemdecl}
\begin{itemdescr}
	\pnum
	\preconditions
	\tcode{_Has_data == true}.
	
	\pnum
	\tcode{_Type == path_data_type::change_origin}.
	
	\pnum
	\returns
	\tcode{change_origin\{ vector_2d\{ _Data.point.x, _Data.point.y \} \}}.
	
	\pnum
	\throws
	As specified in Error reporting (\ref{\iotwod.err.report}).
	
	\pnum
	\remarks
	If a non-throwing error occurs, this function returns a default constructed object of its specified return type.
	
	\pnum
	\errors
	\tcode{errc::operation_not_permitted} if \tcode{!_Has_data}.
	
	\pnum
	\tcode{errc::invalid_argument} if \tcode{_Type != path_data_type::change_origin}.

\end{itemdescr}

\indexlibrary{\idxcode{path_data_item}!\idxcode{get}}
\indexlibrary{\idxcode{get}!\idxcode{path_data_item}}
\begin{itemdecl}
    template <>
    close_path get() const;
    template <>
    close_path get(::std::error_code& ec) const noexcept;
\end{itemdecl}
\begin{itemdescr}
	\pnum
	\preconditions
	\tcode{_Has_data == true}.
	
	\pnum
	\tcode{_Type == path_data_type::close_path}.
	
	\pnum
	\returns
	\tcode{close_path\{ \}}.
	
	\pnum
	\throws
	As specified in Error reporting (\ref{\iotwod.err.report}).
	
	\pnum
	\remarks
	If a non-throwing error occurs, this function returns a default constructed object of its specified return type.
	
	\pnum
	\errors
	\tcode{errc::operation_not_permitted} if \tcode{!_Has_data}.
	
	\pnum
	\tcode{errc::invalid_argument} if \tcode{_Type != path_data_type::close_path}.

\end{itemdescr}

\indexlibrary{\idxcode{path_data_item}!\idxcode{get}}
\indexlibrary{\idxcode{get}!\idxcode{path_data_item}}
\begin{itemdecl}
    template <>
    curve_to get() const;
    template <>
    curve_to get(::std::error_code& ec) const noexcept;
\end{itemdecl}
\begin{itemdescr}
	\pnum
	\preconditions
	\tcode{_Has_data == true}.
	
	\pnum
	\tcode{_Type == path_data_type::curve_to}.
	
	\pnum
	\returns
	\tcode{curve_to\{ vector_2d\{ _Data.curve.cpt1x, _Data.curve.cpt1y \}, vector_2d\{ _Data.curve.cpt2x, _Data.curve.cpt2y \}, vector_2d\{ _Data.curve.eptx, _Data.curve.epty \} \}}.
	
	\pnum
	\throws
	As specified in Error reporting (\ref{\iotwod.err.report}).
	
	\pnum
	\remarks
	If a non-throwing error occurs, this function returns a default constructed object of its specified return type.
	
	\pnum
	\errors
	\tcode{errc::operation_not_permitted} if \tcode{!_Has_data}.
	
	\pnum
	\tcode{errc::invalid_argument} if \tcode{_Type != path_data_type::curve_to}.

\end{itemdescr}

\indexlibrary{\idxcode{path_data_item}!\idxcode{get}}
\indexlibrary{\idxcode{get}!\idxcode{path_data_item}}
\begin{itemdecl}
    template <>
    rel_curve_to get() const;
    template <>
    rel_curve_to get(::std::error_code& ec) const noexcept;
\end{itemdecl}
\begin{itemdescr}
	\pnum
	\preconditions
	\tcode{_Has_data == true}.
	
	\pnum
	\tcode{_Type == path_data_type::rel_curve_to}.
	
	\pnum
	\returns
	\tcode{rel_curve_to\{ vector_2d\{ _Data.curve.cpt1x, _Data.curve.cpt1y \}, vector_2d\{ _Data.curve.cpt2x, _Data.curve.cpt2y \}, vector_2d\{ _Data.curve.eptx, _Data.curve.epty \} \}}.
	
	\pnum
	\throws
	As specified in Error reporting (\ref{\iotwod.err.report}).
	
	\pnum
	\remarks
	If a non-throwing error occurs, this function returns a default constructed object of its specified return type.
	
	\pnum
	\errors
	\tcode{errc::operation_not_permitted} if \tcode{!_Has_data}.
	
	\pnum
	\tcode{errc::invalid_argument} if \tcode{_Type != path_data_type::rel_curve_to}.

\end{itemdescr}

\indexlibrary{\idxcode{path_data_item}!\idxcode{get}}
\indexlibrary{\idxcode{get}!\idxcode{path_data_item}}
\begin{itemdecl}
    template <>
    new_sub_path get() const;
    template <>
    new_sub_path get(::std::error_code& ec) const noexcept;
\end{itemdecl}
\begin{itemdescr}
	\pnum
	\preconditions
	\tcode{_Has_data == true}.
	
	\pnum
	\tcode{_Type == path_data_type::new_sub_path}.
	
	\pnum
	\returns
	\tcode{new_sub_path\{ \}}.
	
	\pnum
	\throws
	As specified in Error reporting (\ref{\iotwod.err.report}).
	
	\pnum
	\remarks
	If a non-throwing error occurs, this function returns a default constructed object of its specified return type.
	
	\pnum
	\errors
	\tcode{errc::operation_not_permitted} if \tcode{!_Has_data}.
	
	\pnum
	\tcode{errc::invalid_argument} if \tcode{_Type != path_data_type::new_sub_path}.

\end{itemdescr}

\indexlibrary{\idxcode{path_data_item}!\idxcode{get}}
\indexlibrary{\idxcode{get}!\idxcode{path_data_item}}
\begin{itemdecl}
    template <>
    line_to get() const;
    template <>
    line_to get(::std::error_code& ec) const noexcept;
\end{itemdecl}
\begin{itemdescr}
	\pnum
	\preconditions
	\tcode{_Has_data == true}.
	
	\pnum
	\tcode{_Type == path_data_type::line_to}.
	
	\pnum
	\returns
	\tcode{line_to\{ vector_2d\{ _Data.point.x, _Data.point.y \} \}}.
	
	\pnum
	\throws
	As specified in Error reporting (\ref{\iotwod.err.report}).
	
	\pnum
	\remarks
	If a non-throwing error occurs, this function returns a default constructed object of its specified return type.
	
	\pnum
	\errors
	\tcode{errc::operation_not_permitted} if \tcode{!_Has_data}.
	
	\pnum
	\tcode{errc::invalid_argument} if \tcode{_Type != path_data_type::line_to}.

\end{itemdescr}

\indexlibrary{\idxcode{path_data_item}!\idxcode{get}}
\indexlibrary{\idxcode{get}!\idxcode{path_data_item}}
\begin{itemdecl}
    template <>
    move_to get() const;
    template <>
    move_to get(::std::error_code& ec) const noexcept;
\end{itemdecl}
\begin{itemdescr}
	\pnum
	\preconditions
	\tcode{_Has_data == true}.
	
	\pnum
	\tcode{_Type == path_data_type::move_to}.
	
	\pnum
	\returns
	\tcode{move_to\{ vector_2d\{ _Data.point.x, _Data.point.y \} \}}.
	
	\pnum
	\throws
	As specified in Error reporting (\ref{\iotwod.err.report}).
	
	\pnum
	\remarks
	If a non-throwing error occurs, this function returns a default constructed object of its specified return type.
	
	\pnum
	\errors
	\tcode{errc::operation_not_permitted} if \tcode{!_Has_data}.
	
	\pnum
	\tcode{errc::invalid_argument} if \tcode{_Type != path_data_type::move_to}.

\end{itemdescr}

\indexlibrary{\idxcode{path_data_item}!\idxcode{get}}
\indexlibrary{\idxcode{get}!\idxcode{path_data_item}}
\begin{itemdecl}
    template <>
    rel_line_to get() const;
    template <>
    rel_line_to get(::std::error_code& ec) const noexcept;
\end{itemdecl}
\begin{itemdescr}
	\pnum
	\preconditions
	\tcode{_Has_data == true}.
	
	\pnum
	\tcode{_Type == path_data_type::rel_line_to}.
	
	\pnum
	\returns
	\tcode{rel_line_to\{ vector_2d\{ _Data.point.x, _Data.point.y \} \}}.
	
	\pnum
	\throws
	As specified in Error reporting (\ref{\iotwod.err.report}).
	
	\pnum
	\remarks
	If a non-throwing error occurs, this function returns a default constructed object of its specified return type.
	
	\pnum
	\errors
	\tcode{errc::operation_not_permitted} if \tcode{!_Has_data}.
	
	\pnum
	\tcode{errc::invalid_argument} if \tcode{_Type != path_data_type::rel_line_to}.

\end{itemdescr}

\indexlibrary{\idxcode{path_data_item}!\idxcode{get}}
\indexlibrary{\idxcode{get}!\idxcode{path_data_item}}
\begin{itemdecl}
    template <>
    rel_move_to get() const;
    template <>
    rel_move_to get(::std::error_code& ec) const noexcept;
\end{itemdecl}
\begin{itemdescr}
	\pnum
	\preconditions
	\tcode{_Has_data == true}.
	
	\pnum
	\tcode{_Type == path_data_type::rel_move_to}.
	
	\pnum
	\returns
	\tcode{rel_move_to\{ vector_2d\{ _Data.point.x, _Data.point.y \} \}}.
	
	\pnum
	\throws
	As specified in Error reporting (\ref{\iotwod.err.report}).
	
	\pnum
	\remarks
	If a non-throwing error occurs, this function returns a default constructed object of its specified return type.
	
	\pnum
	\errors
	\tcode{errc::operation_not_permitted} if \tcode{!_Has_data}.
	
	\pnum
	\tcode{errc::invalid_argument} if \tcode{_Type != path_data_type::rel_move_to}.

\end{itemdescr}

%!TEX root = io2d.tex
\rSec0 [path] {Class \tcode{path}}

\pnum
\indexlibrary{\idxcode{path}}
The \tcode{path} class represents an immutable resource wrapper containing a path geometry graphics resource.

\pnum
When a \tcode{path} object is set on a \tcode{surface} object using 
\tcode{surface::path}, the geometric paths represented by it can be 
stroked or filled.

\pnum
A \tcode{path} object shall be usable with any \tcode{surface} or \tcode{surface}-derived object.

\rSec1 [path.synopsis] {\tcode{path} synopsis}

\begin{codeblock}
namespace std { namespace experimental { namespace io2d { inline namespace v1 {
  class path {
    public:
    // \ref{path.cons}, construct/copy/destroy:
    path() = delete;
    explicit path(const path_factory& pb);
    path(const path_factory& pb, error_code& ec) noexcept;
    explicit path(const vector<path_data_item>& p);
    path(const vector<path_data_item>& p, error_code& ec) noexcept;
    path(const path&) noexcept;
    path& operator=(const path&) noexcept;
    path(path&&) noexcept;
    path& operator=(path&&) noexcept;
  };
} } } }
\end{codeblock}

\rSec1 [path.cons] {\tcode{path} constructors and assignment operators}

\indexlibrary{\idxcode{path}!constructor}
\begin{itemdecl}
    explicit path(const path_factory& pb);
    path(const path_factory& pb, error_code& ec) noexcept;
    explicit path(const vector<path_data_item>& p);
    path(const vector<path_data_item>& p, error_code& ec) noexcept;
\end{itemdecl}
\begin{itemdescr}
	\pnum
	\effects
	Constructs an object of class \tcode{path}. Implementations shall create a path geometry graphics resource from the path geometries contained in \tcode{p} or \tcode{pb.data_ref()} as if they followed the procedure set forth in \ref{pathgeometries.processing}.

	\pnum
	\throws
	As specified in Error reporting (\ref{\iotwod.err.report}).

	\pnum
	\remarks
	It is unspecified whether a \tcode{path} object shall require further processing when it is passed as an argument to a \tcode{surface} or \tcode{surface}-derived object.
	
	\pnum
	Implementations should avoid or minimize the need for further processing of a \tcode{path} object after it has been constructed.

	\pnum
	\errors
	\tcode{errc::not_enough_memory} if there was a failure to allocate memory.
	
	\pnum
	\tcode{io2d_error::no_current_point} if, when processing the path geometries, an operation was encountered which required a current point and the current path geometry had no current point.
	
	\pnum
	\tcode{io2d_error::invalid_matrix} if, when processing the path geometries, an operation was encountered which required the current transformation matrix to be invertible and the matrix was not invertible.
	
\end{itemdescr}

%!TEX root = io2d.tex
\rSec0 [\iotwod.pathfactory] {Class \tcode{path_factory}}

\rSec1 [\iotwod.pathfactory.intro] {\tcode{path_factory} Description}

\pnum
\indexlibrary{\idxcode{path_factory}}%
The \tcode{path_factory} class is a factory class that produces \tcode{path} 
objects. The \tcode{path} objects produced by a \tcode{path_factory} instance 
are immutable such that subsequent changes to the \tcode{path_factory} do not 
modify a previous \tcode{path} object obtained from it.

\rSec1 [\iotwod.pathfactory.synopsis] {\tcode{path_factory} synopsis}

\begin{codeblock}
namespace std { namespace experimental { namespace io2d { inline namespace v1 {
  class path_factory {
    // \ref{\iotwod.pathfactory.cons}, construct/copy/destroy:
    path_factory();
    path_factory(const path_factory& x);
    path_factory& operator=(const path_factory& x);
    path_factory(path_factory&& x);
    path_factory& operator=(path_factory&& x);
    
    // \ref{\iotwod.pathfactory.modifiers}, modifiers:
    void append(const path& p);
    void append(const path_factory& p);
    void append(const ::std::vector<path_data>& p);
    void reset();
    void new_sub_path();
    void close_path();
    void arc(const point& center, double radius, double angle1,
    double angle2);
    void arc_negative(const point& center, double radius,
    double angle1, double angle2);
    void curve_to(const point& pt0, const point& pt1,
    const point& pt2);
    void line_to(const point& pt);
    void move_to(const point& pt);
    void rect(const rectangle& r);
    void rel_curve_to(const point& dpt0, const point& dpt1,
    const point& dpt2);
    void rel_line_to(const point& dpt);
    void rel_move_to(const point& dpt);
    void set_transform_matrix(const matrix_2d& m);
    void set_origin(const point& pt);
    
    // \ref{\iotwod.pathfactory.observers}, observers:
    matrix_2d get_transform_matrix() const;
    point get_origin() const;
    bool has_current_point() const;
    point get_current_point() const;
    path get_path() const;
    rectangle get_path_extents() const;
    ::std::vector<path_data> get_data() const;
    const ::std::vector<path_data>& get_data_ref() const;
  };
} } } } // namespaces std::experimental::io2d::v1
\end{codeblock}

\rSec1 [\iotwod.pathfactory.cons] {\tcode{path_factory} constructors and 
assignment operators}

\indexlibrary{\idxcode{path_factory}!constructor}%
\begin{itemdecl}
path_factory();
\end{itemdecl}
\begin{itemdescr}
	\pnum
	\effects
	Constructs an object of type \tcode{path_factory}.
	
	\pnum
	\postconditions
	\tcode{get_data_ref()} returns a const reference to an empty vector.
	\tcode{get_data()} returns an empty vector.
	\tcode{has_current_point()} returns false.
	\tcode{get_origin()} returns a value equivalent to \tcode{point\{0.0, 
	0.0\}}.
	\tcode{get_transform_matrix()} returns a value equivalent to the return 
	value of \tcode{matrix_2d::init_identity()}.
	\tcode{get_path_extents()} returns a value equivalent to 
	\tcode{rectangle\{0.0, 0.0, 0.0, 0.0\}}.
	
	\pnum
	\complexity
	Constant.
\end{itemdescr}

\rSec1 [\iotwod.pathfactory.modifiers] {\tcode{path_factory} modifiers}

\rSec2 [\iotwod.pathfactory::append] {\tcode{path_factory::append}}

\indexlibrary{\idxcode{path_factory}!\idxcode{append}}%
\indexlibrary{\idxcode{append}!\idxcode{path_factory}}%
\begin{itemdecl}
void append(const path& p);
void append(const path_factory& p);
\end{itemdecl}
\begin{itemdescr}
	\pnum
	\effects
	Appends the results of \tcode{p.get_data()} to the \tcode{path_data} stored 
	by \tcode{*this}.
	
	\pnum
	\postconditions
	If \tcode{p.get_data().empty() == false}, the
	\tcode{vector<path_data>} returned by \tcode{get_data_ref} will change in 
	the ways documented for \tcode{vector<T>::push_back(const T\&)}.
	The values returned by \tcode{get_transform_matrix()}, 
	\tcode{get_origin()}, \tcode{has_current_point()}, and 
	\tcode{get_current_point()} are the same as they would be if the 
	\tcode{path_factory} member function calls necessary to generate the 
	\tcode{path_data} in \tcode{p.get_data()} had been called.
	
	\pnum
	\complexity
	Linear in the number of elements in \tcode{p.get_data()}.
\end{itemdescr}

\indexlibrary{\idxcode{path_factory}!\idxcode{append}}%
\indexlibrary{\idxcode{append}!\idxcode{path_factory}}%
\begin{itemdecl}
void append(const ::std::vector<path_data>& p);
\end{itemdecl}
\begin{itemdescr}
	\pnum
	\effects
	The same as \tcode{path_factory::append_path(const path\& p)}.
	
	\pnum
	\throws
	\tcode{system_error} with an error code equivalent to 
	\tcode{io2d_error::invalid_path_data} and an error category of type 
	\tcode{io2d_error_category} if a \tcode{path_data} with a type that is not 
	a member of \tcode{path_data_type} is encountered or if a \tcode{path_data} 
	with a type of \tcode{path_data_type::rel_move_to}, 
	\tcode{path_data_type::rel_line_to}, or 
	\tcode{path_data_type::rel_curve_to} is encountered when no current point 
	is established.

	\pnum
	\postconditions
	If \tcode{p.empty() == false}, the \tcode{vector<path_data>} returned by 
	\tcode{get_data_ref} will change in the ways documented for 
	\tcode{vector<T>::push_back(const T\&)}.
	The values returned by \tcode{get_transform_matrix()}, 
	\tcode{get_origin()}, \tcode{has_current_point()}, and 
	\tcode{get_current_point()} are the same as they would be if the 
	\tcode{path_factory} member function calls necessary to generate the 
	\tcode{path_data} in \tcode{p} had been called.

	\pnum
	\complexity
	Linear in the number of elements in \tcode{p}.
\end{itemdescr}

\rSec2 [\iotwod.pathfactory::reset] {\tcode{path_factory::reset}}

\indexlibrary{\idxcode{path_factory}!\idxcode{reset}}%
\indexlibrary{\idxcode{reset}!\idxcode{path_factory}}%
\begin{itemdecl}
void reset();
\end{itemdecl}
\begin{itemdescr}
	\pnum
	\effects
	Modify \tcode{*this} so that its observable state matches that of a 
	\tcode{path_factory} object that was just default constructed.

	\pnum
	\postcondition
	The \tcode{vector<path_data>} returned by \tcode{get_data_ref} will change 
	in the ways documented for \tcode{vector<T>::push_back(const T\&)}.
	
	\pnum
	\complexity
	Constant.
\end{itemdescr}

\rSec2 [\iotwod.pathfactory::new_sub_path] {\tcode{path_factory::new_sub_path}}

\indexlibrary{\idxcode{path_factory}!\idxcode{new_sub_path}}%
\indexlibrary{\idxcode{new_sub_path}!\idxcode{path_factory}}%
\begin{itemdecl}
void new_sub_path();
\end{itemdecl}
\begin{itemdescr}
	\pnum
	\effects
	Append a new \tcode{path_data} object to the \tcode{path_data} stored by 
	\tcode{*this}. The appended object's \tcode{path_data::type} is 
	\tcode{path_data_type::new_sub_path} and its \tcode{path_data::data} has 
	\tcode{path_data::data::unused} set to \tcode{0}.
	
	\pnum
	\postconditions
	The \tcode{vector<path_data>} returned by \tcode{get_data_ref} will change 
	in the ways documented for \tcode{vector<T>::push_back(const T\&)}.
	\tcode{has_current_point()} will return \tcode{false}.
	
	\pnum
	\complexity
	Constant.
\end{itemdescr}

\rSec2 [\iotwod.pathfactory::close_path] {\tcode{path_factory::close_path}}

\indexlibrary{\idxcode{path_factory}!\idxcode{close_path}}%
\indexlibrary{\idxcode{close_path}!\idxcode{path_factory}}%
\begin{itemdecl}
void close_path();
\end{itemdecl}
\begin{itemdescr}
	\pnum
	\effects
	If \tcode{has_current_point() == true}, append a new \tcode{path_data} 
	object to the \tcode{path_data} stored by \tcode{*this}. The appended 
	object's \tcode{path_data::type} is \tcode{path_data_type::close_path} and 
	its \tcode{path_data::data} has \tcode{path_data::data::unused} set to 
	\tcode{0}. If \tcode{has_current_point() == false}, do nothing.
	
	\pnum
	\postconditions
	If \tcode{has_current_point() == true}, the \tcode{vector<path_data>} 
	returned by \tcode{get_data_ref} will change in the ways documented for 
	\tcode{vector<T>::push_back(const T\&)} and the value returned by 
	\tcode{get_current_point()} will be the value of 
	\tcode{path_data::data::move} from the most recent \tcode{path_data} 
	appended to the \tcode{path_data} stored by \tcode{*this} with a 
	{path_data::type} of \tcode{path_data_type::move_to}.
	
	\pnum
	\complexity
	Constant.
\end{itemdescr}

\rSec2 [\iotwod.pathfactory::arc] {\tcode{path_factory::arc}}

\indexlibrary{\idxcode{path_factory}!\idxcode{arc}}%
\indexlibrary{\idxcode{arc}!\idxcode{path_factory}}%
\begin{itemdecl}
void arc(const point& center, double radius, double angle1, double angle2);
\end{itemdecl}
\begin{itemdescr}
	\pnum
	\effects
	Append a new \tcode{path_data} object to the \tcode{path_data} stored by 
	\tcode{*this}. The appended object's \tcode{path_data::type} is 
	\tcode{path_data_type::arc} and its \tcode{path_data::data} has 
	\tcode{path_data::data::arc::center} set to \tcode{center}, 
	\tcode{path_data::data::arc::radius} set to \tcode{radius}, 
	\tcode{path_data::data::arc::angle1} set to \tcode{angle1}, and
	\tcode{path_data::data::arc::angle2} set to \tcode{angle2}.
	
	\pnum
	\postconditions
	The \tcode{vector<path_data>} returned by \tcode{get_data_ref} will change 
	in the ways documented for \tcode{vector<T>::push_back(const T\&)}.
	\tcode{has_current_point()} will return \tcode{true}.
	\tcode{get_current_point()} will return a \tcode{point} placed at 
	\tcode{point(center.x + radius, center.y} and rotated around \tcode{center} 
	such that it is at \tcode{angle2} radians.
	
	\pnum
	\complexity
	Constant.
\end{itemdescr}

\rSec2 [\iotwod.pathfactory::arc_negative] {\tcode{path_factory::arc_negative}}

\indexlibrary{\idxcode{path_factory}!\idxcode{arc_negative}}%
\indexlibrary{\idxcode{arc_negative}!\idxcode{path_factory}}%
\begin{itemdecl}
void arc_negative(const point& center, double radius, double angle1, double 
angle2);
\end{itemdecl}
\begin{itemdescr}
	\pnum
	\effects
	Append a new \tcode{path_data} object to the \tcode{path_data} stored by 
	\tcode{*this}. The appended object's \tcode{path_data::type} is 
	\tcode{path_data_type::arc_negative} and its \tcode{path_data::data} has 
	\tcode{path_data::data::arc::center} set to \tcode{center}, 
	\tcode{path_data::data::arc::radius} set to \tcode{radius}, 
	\tcode{path_data::data::arc::angle1} set to \tcode{angle1}, and
	\tcode{path_data::data::arc::angle2} set to \tcode{angle2}.
	
	\pnum
	\postconditions
	The \tcode{vector<path_data>} returned by \tcode{get_data_ref} will change 
	in the ways documented for \tcode{vector<T>::push_back(const T\&)}.
	\tcode{has_current_point()} will return \tcode{true}. 
	\tcode{get_current_point()} will return a \tcode{point} placed at 
	\tcode{point(center.x + radius, center.y} and rotated around \tcode{center} 
	such that it is at \tcode{angle2} radians.
	
	\pnum
	\complexity
	Constant.
\end{itemdescr}

\rSec2 [\iotwod.pathfactory::curve_to] {\tcode{path_factory::curve_to}}

\indexlibrary{\idxcode{path_factory}!\idxcode{curve_to}}%
\indexlibrary{\idxcode{curve_to}!\idxcode{path_factory}}%
\begin{itemdecl}
void curve_to(const point& pt0, const point& pt1, const point& pt2);
\end{itemdecl}
\begin{itemdescr}
	\pnum
	\effects
	Append a new \tcode{path_data} object to the \tcode{path_data} stored by 
	\tcode{*this}. The appended object's \tcode{path_data::type} is 
	\tcode{path_data_type::curve_to} and its \tcode{path_data::data} has 
	\tcode{path_data::data::curve::pt1} set to \tcode{pt0}, 
	\tcode{path_data::data::curve::pt2} set to \tcode{pt1}, and 
	\tcode{path_data::data::curve::pt3} set to \tcode{pt2}.
	
	\pnum
	\postcondition
	The \tcode{vector<path_data>} returned by \tcode{get_data_ref} will change 
	in the ways documented for \tcode{vector<T>::push_back(const T\&)}.
	\tcode{has_current_point()} will return \tcode{true}. 
	\tcode{get_current_point()} will return \tcode{point(pt3)}.
	
	\pnum
	\complexity
	Constant.
\end{itemdescr}

\rSec2 [\iotwod.pathfactory::line_to] {\tcode{path_factory::line_to}}

\indexlibrary{\idxcode{path_factory}!\idxcode{line_to}}%
\indexlibrary{\idxcode{line_to}!\idxcode{path_factory}}%
\begin{itemdecl}
void line_to(const point& pt);
\end{itemdecl}
\begin{itemdescr}
	\pnum
	\effects
	If \tcode{has_current_point() == true}, append a new \tcode{path_data} 
	object to the \tcode{path_data} stored by \tcode{*this}. The appended 
	object's \tcode{path_data::type} is \tcode{path_data_type::line_to} and its 
	\tcode{path_data::data} has \tcode{path_data::data::line} set to 
	\tcode{pt}. Otherwise has the same effect as calling \tcode{move_to(pt)}.
	
	\pnum
	\postconditions
	The \tcode{vector<path_data>} returned by \tcode{get_data_ref} will change 
	in the ways documented for \tcode{vector<T>::push_back(const T\&)}.
	\tcode{has_current_point()} will return \tcode{true}. 
	\tcode{get_current_point()} will return \tcode{point(pt)}.
	
	\pnum
	\complexity
	Constant.
\end{itemdescr}

\rSec2 [\iotwod.pathfactory::move_to] {\tcode{path_factory::move_to}}

\indexlibrary{\idxcode{path_factory}!\idxcode{move_to}}%
\indexlibrary{\idxcode{move_to}!\idxcode{path_factory}}%
\begin{itemdecl}
void move_to(const point& pt);
\end{itemdecl}
\begin{itemdescr}
	\pnum
	\effects
	Append a new \tcode{path_data} object to the \tcode{path_data} stored by 
	\tcode{*this}. The appended object's \tcode{path_data::type} is 
	\tcode{path_data_type::move_to} and its \tcode{path_data::data} has 
	\tcode{path_data::data::move} set to \tcode{pt}.
	
	\pnum
	\postconditions
	The \tcode{vector<path_data>} returned by \tcode{get_data_ref} will change 
	in the ways documented for \tcode{vector<T>::push_back(const T\&)}.
	\tcode{has_current_point()} will return \tcode{true}. 
	\tcode{get_current_point()} will return \tcode{point(pt)}.
	
	\pnum
	\complexity
	Constant.
\end{itemdescr}

\rSec2 [\iotwod.pathfactory::rect] {\tcode{path_factory::rect}}

\indexlibrary{\idxcode{path_factory}!\idxcode{rect}}%
\indexlibrary{\idxcode{rect}!\idxcode{path_factory}}%
\begin{itemdecl}
void rect(const rectangle& r);
\end{itemdecl}
\begin{itemdescr}
	\pnum
	\effects
	Append five new \tcode{path_data} objects to the \tcode{path_data} stored 
	by \tcode{*this}.
	The first appended object's \tcode{path_data::type} is 
	\tcode{path_data_type::move_to} and its \tcode{path_data::data} has 
	\tcode{path_data::data::move.x} set to \tcode{r.x} and 
	\tcode{path_data::data::move.y} set to \tcode{r.y}.
	The second appended object's \tcode{path_data::type} is 
	\tcode{path_data_type::rel_line_to} and its \tcode{path_data::data} has 
	\tcode{path_data::data::line.x} set to \tcode{r.width} and 
	\tcode{path_data::data::line.y} set to \tcode{0.0}.
	The third appended object's \tcode{path_data::type} is 
	\tcode{path_data_type::rel_line_to} and its \tcode{path_data::data} has 
	\tcode{path_data::data::line.x} set to \tcode{0.0} and 
	\tcode{path_data::data::line.y} set to \tcode{r.height}.
	The fourth appended object's \tcode{path_data::type} is 
	\tcode{path_data_type::rel_line_to} and its \tcode{path_data::data} has 
	\tcode{path_data::data::line.x} set to \tcode{-r.width} and 
	\tcode{path_data::data::line.y} set to \tcode{0.0}.
	The fifth appended object's \tcode{path_data::type} is 
	\tcode{path_data_type::close_path} and its \tcode{path_data::data} has 
	\tcode{path_data::data::unused} set to \tcode{0}.
	
	\pnum
	\postconditions
	The \tcode{vector<path_data>} returned by \tcode{get_data_ref} will change 
	in the ways documented for \tcode{vector<T>::push_back(const T\&)}. 
	\tcode{has_current_point} will return \tcode{false}.
	\tcode{has_current_point()} will return \tcode{true}. 
	\tcode{get_current_point()} will return \tcode{point\{r.x, r.y\}}.	
	
	\pnum
	\complexity
	Constant.
\end{itemdescr}

\rSec2 [\iotwod.pathfactory::rel_curve_to] {\tcode{path_factory::rel_curve_to}}

\indexlibrary{\idxcode{path_factory}!\idxcode{rel_curve_to}}%
\indexlibrary{\idxcode{rel_curve_to}!\idxcode{path_factory}}%
\begin{itemdecl}
void rel_curve_to(const point& dpt0, const point& dpt1, const point& dpt2);
\end{itemdecl}
\begin{itemdescr}
	\pnum
	\effects
	Append a new \tcode{path_data} object to the \tcode{path_data} stored by 
	\tcode{*this}. The appended object's \tcode{path_data::type} is 
	\tcode{path_data_type::rel_curve_to} and its \tcode{path_data::data} has 
	\tcode{path_data::data::curve::pt1} set to \tcode{pt0}, 
	\tcode{path_data::data::curve::pt2} set to \tcode{pt1}, and 
	\tcode{path_data::data::curve::pt3} set to \tcode{pt2}.
	
	\pnum
	\postcondition
	The \tcode{vector<path_data>} returned by \tcode{get_data_ref} will change 
	in the ways documented for \tcode{vector<T>::push_back(const T\&)}.
	\tcode{has_current_point()} will return \tcode{true}.
	\tcode{get_current_point()} will return the result of 
	\tcode{get_current_point()} before the function was called with 
	\tcode{point(pt3)} added to it.
	
	\pnum
	\complexity
	Constant.
\end{itemdescr}

\rSec2 [\iotwod.pathfactory::rel_line_to] {\tcode{path_factory::rel_line_to}}

\indexlibrary{\idxcode{path_factory}!\idxcode{rel_line_to}}%
\indexlibrary{\idxcode{rel_line_to}!\idxcode{path_factory}}%
\begin{itemdecl}
void rel_line_to(const point& dpt);
\end{itemdecl}
\begin{itemdescr}
	\pnum
	\effects
	If \tcode{has_current_point() == true}, append a new \tcode{path_data} 
	object to the \tcode{path_data} stored by \tcode{*this}. The appended 
	object's \tcode{path_data::type} is \tcode{path_data_type::rel_line_to} and 
	its \tcode{path_data::data} has \tcode{path_data::data::line} set to 
	\tcode{pt}. Otherwise has the same effect as calling \tcode{move_to(pt)}.
	
	\pnum
	\postconditions
	The \tcode{vector<path_data>} returned by \tcode{get_data_ref} will change 
	in the ways documented for \tcode{vector<T>::push_back(const T\&)}.
	\tcode{has_current_point()} will return \tcode{true}. 
	\tcode{get_current_point()} will return the result of 
	\tcode{get_current_point()} before the function was called with 
	\tcode{point(pt)} added to it.
	
	\pnum
	\complexity
	Constant.
\end{itemdescr}

\rSec2 [\iotwod.pathfactory::rel_move_to] {\tcode{path_factory::rel_move_to}}

\indexlibrary{\idxcode{path_factory}!\idxcode{rel_move_to}}%
\indexlibrary{\idxcode{rel_move_to}!\idxcode{path_factory}}%
\begin{itemdecl}
void rel_move_to(const point& dpt);
\end{itemdecl}
\begin{itemdescr}
	\pnum
	\effects
	Append a new \tcode{path_data} object to the \tcode{path_data} stored by 
	\tcode{*this}. The appended object's \tcode{path_data::type} is 
	\tcode{path_data_type::rel_move_to} and its \tcode{path_data::data} has 
	\tcode{path_data::data::move} set to \tcode{pt}.
	
	\pnum
	\postconditions
	The \tcode{vector<path_data>} returned by \tcode{get_data_ref} will change 
	in the ways documented for \tcode{vector<T>::push_back(const T\&)}.
	\tcode{has_current_point()} will return \tcode{true}. 
	\tcode{get_current_point()} will return the result of 
	\tcode{get_current_point()} before the function was called with 
	\tcode{point(pt)} added to it.
	
	\pnum
	\complexity
	Constant.
\end{itemdescr}

\rSec2 [\iotwod.pathfactory::set_transform_matrix] 
{\tcode{path_factory::set_transform_matrix}}

\indexlibrary{\idxcode{path_factory}!\idxcode{set_transform_matrix}}%
\indexlibrary{\idxcode{set_transform_matrix}!\idxcode{path_factory}}%
\begin{itemdecl}
void set_transform_matrix(const matrix_2d& m);
\end{itemdecl}
\begin{itemdescr}
	\pnum
	\effects
	Append a new \tcode{path_data} object to the \tcode{path_data} stored by 
	\tcode{*this}. The appended object's \tcode{path_data::type} is 
	\tcode{path_data_type::change_matrix} and its \tcode{path_data::data} has 
	\tcode{path_data::data::matrix} set to \tcode{m}.
	
	\pnum
	\postconditions
	The \tcode{vector<path_data>} returned by \tcode{get_data_ref} will change 
	in the ways documented for \tcode{vector<T>::push_back(const T\&)}.
	
	\pnum
	\complexity
	Constant.
\end{itemdescr}

\rSec2 [\iotwod.pathfactory::set_origin] {\tcode{path_factory::set_origin}}

\indexlibrary{\idxcode{path_factory}!\idxcode{set_origin}}%
\indexlibrary{\idxcode{set_origin}!\idxcode{path_factory}}%
\begin{itemdecl}
void set_origin(const point& pt);
\end{itemdecl}
\begin{itemdescr}
	\pnum
	\effects
	Append a new \tcode{path_data} object to the \tcode{path_data} stored by 
	\tcode{*this}. The appended object's \tcode{path_data::type} is 
	\tcode{path_data_type::change_origin} and its \tcode{path_data::data} has 
	\tcode{path_data::data::origin} set to \tcode{pt}.
	
	\pnum
	\postconditions
	The \tcode{vector<path_data>} returned by \tcode{get_data_ref} will change 
	in the ways documented for \tcode{vector<T>::push_back(const T\&)}.
	
	\pnum
	\complexity
	Constant.
\end{itemdescr}

\rSec1 [\iotwod.pathfactory.observers] {\tcode{path_factory} observers}


\rSec2 [\iotwod.pathfactory::get_transform_matrix] 
{\tcode{path_factory::get_transform_matrix}}

\indexlibrary{\idxcode{path_factory}!\idxcode{get_transform_matrix}}%
\indexlibrary{\idxcode{get_transform_matrix}!\idxcode{path_factory}}%
\begin{itemdecl}
matrix_2d get_transform_matrix() const;
\end{itemdecl}
\begin{itemdescr}
	\pnum
	\returns
	The value of \tcode{path_data::data::matrix} in the element nearest to the 
	end of \tcode{get_data_ref()} with a \tcode{path_data::type} of 
	\tcode{path_data_type::change_matrix} or, if no such element exists, a 
	value equal to the return value of \tcode{matrix_2d::init_identity()}.
	
	\pnum
	\complexity
	Constant.
\end{itemdescr}

\rSec2 [\iotwod.pathfactory::get_origin] {\tcode{path_factory::get_origin}}

\indexlibrary{\idxcode{path_factory}!\idxcode{get_origin}}%
\indexlibrary{\idxcode{get_origin}!\idxcode{path_factory}}%
\begin{itemdecl}
point get_origin() const;
\end{itemdecl}
\begin{itemdescr}
	\pnum
	\returns
	The value of \tcode{path_data::data::origin} in the element nearest to the 
	end of \tcode{get_data_ref()} with a \tcode{path_data::type} of 
	\tcode{path_data_type::change_origin} or, if no such element exists, a 
	value equal to the return value of \tcode{point\{0.0, 0.0\}}.
	
	\pnum
	\complexity
	Constant.
\end{itemdescr}

\rSec2 [\iotwod.pathfactory::has_current_point] 
{\tcode{path_factory::has_current_point}}

\indexlibrary{\idxcode{path_factory}!\idxcode{has_current_point}}%
\indexlibrary{\idxcode{has_current_point}!\idxcode{path_factory}}%
\begin{itemdecl}
bool has_current_point() const;
\end{itemdecl}
\begin{itemdescr}
	\pnum
	\returns
	True if there is a current point, false if not.
	
	\pnum
	\complexity
	Constant.
\end{itemdescr}

\rSec2 [\iotwod.pathfactory::get_current_point] 
{\tcode{path_factory::get_current_point}}

\indexlibrary{\idxcode{path_factory}!\idxcode{get_current_point}}%
\indexlibrary{\idxcode{get_current_point}!\idxcode{path_factory}}%
\begin{itemdecl}
point get_current_point() const;
\end{itemdecl}
\begin{itemdescr}
	\pnum
	\returns
	The value of the current point if \tcode{has_current_point() == true}, 
	otherwise an undefined but valid value.
	
	\pnum
	\complexity
	Constant.
\end{itemdescr}

\rSec2 [\iotwod.pathfactory::get_path] {\tcode{path_factory::get_path}}

\indexlibrary{\idxcode{path_factory}!\idxcode{get_path}}%
\indexlibrary{\idxcode{get_path}!\idxcode{path_factory}}%
\begin{itemdecl}
path get_path() const;
\end{itemdecl}
\begin{itemdescr}
	\pnum
	\returns
	\tcode{path(*this)}.
\end{itemdescr}

\rSec2 [\iotwod.pathfactory::get_path_extents] 
{\tcode{path_factory::get_path_extents}}

\indexlibrary{\idxcode{path_factory}!\idxcode{get_path_extents}}%
\indexlibrary{\idxcode{get_path_extents}!\idxcode{path_factory}}%
\begin{itemdecl}
rectangle get_path_extents() const;
\end{itemdecl}
\begin{itemdescr}
	\pnum
	\returns
	The smallest \tcode{rectangle} that can contain the points contained in the 
	\tcode{path_data} of \tcode{*this}.
	
	\pnum
	\remarks
	When determining the return value of this function, each point should be 
	transformed in the way described in \ref{\iotwod.surface.pathtransform}.
\end{itemdescr}

\rSec2 [\iotwod.pathfactory::get_data] {\tcode{path_factory::get_data}}

\indexlibrary{\idxcode{path_factory}!\idxcode{get_data}}%
\indexlibrary{\idxcode{get_data}!\idxcode{path_factory}}%
\begin{itemdecl}
::std::vector<path_data> get_data() const;
\end{itemdecl}
\begin{itemdescr}
	\pnum
	\returns
	A copy of the \tcode{vector<path_data>} stored by \tcode{*this}.
\end{itemdescr}

\rSec2 [\iotwod.pathfactory::get_data_ref] {\tcode{path_factory::get_data_ref}}

\indexlibrary{\idxcode{path_factory}!\idxcode{get_data_ref}}%
\indexlibrary{\idxcode{get_data_ref}!\idxcode{path_factory}}%
\begin{itemdecl}
const ::std::vector<path_data>& get_data_ref() const;
\end{itemdecl}
\begin{itemdescr}
	\pnum
	\returns
	A const reference to the \tcode{vector<path_data>} stored by \tcode{*this}.
\end{itemdescr}

\addtocounter{SectionDepthBase}{1}
%%!TEX root = io2d.tex
\rSec0 [pathdata] {Class \tcode{path_data}}

\rSec1 [pathdata.synopsis] {\tcode{path_data} synopsis}

\begin{codeblock}
namespace std { namespace experimental { namespace io2d { inline namespace v1 {
  class path_data {
  public:
    // \ref{pathdata.cons}, construct/copy/move/destroy:
    path_data() noexcept;
    path_data(const path_data& other) noexcept;
    path_data& operator=(const path_data& other) noexcept;
    path_data(path_data&& other) noexcept;
    path_data& operator=(path_data&& other) noexcept;
    virtual ~path_data() noexcept;

    // \ref{pathdata.observers}, observers:
    virtual path_data_type type() const noexcept = 0;
  };
} } } }
\end{codeblock}

\rSec1 [pathdata.intro] {\tcode{path_data} Description}

\pnum
\indexlibrary{\idxcode{path_data}}
The class \tcode{path_data} serves as an abstract base class for classes that describe operations performed on path geometries.

\rSec1 [pathdata.cons] {\tcode{path_data} constructors and assignment operators}

\indexlibrary{\idxcode{path_data}!destructor}
\begin{itemdecl}
    virtual ~path_data() noexcept;
\end{itemdecl}
\begin{itemdescr}
	\pnum
	\effects
	Destroys an object of type \tcode{path_data}.
	
\end{itemdescr}

\rSec1 [pathdata.observers]{\tcode{path_data} observers}

\indexlibrary{\idxcode{path_data}!\idxcode{type}}
\indexlibrary{\idxcode{type}!\idxcode{path_data}}
\begin{itemdecl}
    virtual path_data_type type() const noexcept = 0;
\end{itemdecl}
\begin{itemdescr}
	\pnum
	\returns
	The \tcode{path_data_type} of the \tcode{path_data}-derived object.
	
	\pnum
	\realnote
	This is used for casting to the correct type when iterating through a \tcode{vector<path_data>} object.
\end{itemdescr}

%!TEX root = io2d.tex
\rSec0 [\iotwod.arc] {Class \tcode{arc}}

\rSec1 [\iotwod.arc.general] {In general}

\pnum
\indexlibrary{\idxcode{arc}}%
The class \tcode{arc} describes a path item that is a path segment.

\pnum
It has a \term{radius} of type \tcode{vector_2d}, a \term{rotation} of type \tcode{float}, and a \term{start angle} of type \tcode{float}.

\rSec1 [\iotwod.arc.synopsis] {\tcode{arc} synopsis}

\begin{codeblock}
namespace std::experimental::io2d::v1 {
  namespace path_data {
    class arc {
    public:
      // \ref{\iotwod.arc.cons}, construct/copy/move/destroy:
      constexpr arc() noexcept;
      constexpr arc(const vector_2d& rad,
        float rot, float sang) noexcept;

      // \ref{\iotwod.arc.modifiers}, modifiers:
      constexpr void radius(const vector_2d& rad) noexcept;
      constexpr void rotation(float rot) noexcept;
      constexpr void start_angle(float radians) noexcept;

      // \ref{\iotwod.arc.observers}, observers:
      constexpr vector_2d radius() const noexcept;
      constexpr float rotation() const noexcept;
      constexpr float start_angle() const noexcept;
      vector_2d center(const vector_2d& cpt, const matrix_2d& m = matrix_2d{}) 
        const noexcept;
      vector_2d end_pt(const vector_2d& cpt, const matrix_2d& m = matrix_2d{}) 
        const noexcept;
    };
    
    // \ref{\iotwod.arc.nonmember}, non-members
    constexpr bool operator==(const arc& lhs, const arc& rhs) noexcept;
    constexpr bool operator!=(const arc& lhs, const arc& rhs) noexcept;
  }
}
\end{codeblock}

\rSec1 [\iotwod.arc.cons] {\tcode{arc} constructors}

\indexlibrary{\idxcode{arc}!constructor}%
\begin{itemdecl}
constexpr arc() noexcept;
\end{itemdecl}
\begin{itemdescr}
\pnum
\effects
Equivalent to: \tcode{arc\{ vector_2d(10.0f, 10.0f), pi<float>, pi<float> \};}.
\end{itemdescr}

\indexlibrary{\idxcode{arc}!constructor}%
\begin{itemdecl}
constexpr arc(const vector_2d& rad, float rot,
  float start_angle = pi<float>) noexcept;
\end{itemdecl}
\begin{itemdescr}
\pnum
\effects
Constructs an object of type \tcode{arc}.

\pnum
The radius is \tcode{rad}.

\pnum
The rotation is \tcode{rot}.

\pnum
The start angle is \tcode{sang}.
\end{itemdescr}

\rSec1 [\iotwod.arc.modifiers]{\tcode{arc} modifiers}

\indexlibrarymember{radius}{arc}%
\begin{itemdecl}
constexpr void radius(const vector_2d& rad) noexcept;
\end{itemdecl}
\begin{itemdescr}
\pnum
\effects
The radius is \tcode{rad}.
\end{itemdescr}

\indexlibrarymember{rotation}{arc}%
\begin{itemdecl}
constexpr void rotation(float rot) noexcept;
\end{itemdecl}
\begin{itemdescr}
\pnum
\effects
The rotation is \tcode{rot}.
\end{itemdescr}

\indexlibrarymember{start_angle}{arc}%
\begin{itemdecl}
constexpr void start_angle(float sang) noexcept;
\end{itemdecl}
\begin{itemdescr}
\pnum
\effects
The start angle is \tcode{sang}.
\end{itemdescr}

\rSec1 [\iotwod.arc.observers]{\tcode{arc} observers}

\indexlibrarymember{radius}{arc}%
\begin{itemdecl}
constexpr vector_2d radius() const noexcept;
\end{itemdecl}
\begin{itemdescr}
\pnum
\returns
The radius.
\end{itemdescr}

\indexlibrarymember{rotation}{arc}%
\begin{itemdecl}
constexpr float rotation() const noexcept;
\end{itemdecl}
\begin{itemdescr}
\pnum
\returns
The rotation.
\end{itemdescr}

\indexlibrarymember{start_angle}{arc}%
\begin{itemdecl}
constexpr float start_angle() const noexcept;
\end{itemdecl}
\begin{itemdescr}
\pnum
\returns
The start angle.
\end{itemdescr}

\indexlibrarymember{center}{arc}%
\begin{itemdecl}
vector_2d center(const vector_2d& cpt, const matrix_2d& m = matrix_2d{})
  const noexcept;
\end{itemdecl}
\begin{itemdescr}
\pnum
\returns
As-if:
\begin{codeblock}
auto lmtx = m;
lmtx.m20(0.0f); lmtx.m21(0.0f); // Eliminate translation.
auto centerOffset = point_for_angle(two_pi<float> - _Start_angle, _Radius);
centerOffset.y(-centerOffset.y());
return cpt - centerOffset * lmtx;
\end{codeblock}
\end{itemdescr}

\indexlibrarymember{start_angle}{arc}%
\begin{itemdecl}
vector_2d end_pt(const vector_2d& cpt, const matrix_2d& m = matrix_2d{})
  const noexcept;
\end{itemdecl}
\begin{itemdescr}
\pnum
\returns
As-if:
\begin{codeblock}
auto lmtx = m;
auto tfrm = matrix_2d::init_rotate(_Start_angle + _Rotation);
lmtx.m20(0.0f); lmtx.m21(0.0f); // Eliminate translation.
auto pt = (_Radius * tfrm);
pt.y(-pt.y());
return cpt + pt * lmtx;
\end{codeblock}
\end{itemdescr}

\rSec1 [\iotwod.arc.nonmember]{Non-member functions}

\indexlibrarymember{operator==}{arc}%
\begin{itemdecl}
constexpr bool operator==(const arc& lhs, const arc& rhs) noexcept;
\end{itemdecl}
\begin{itemdescr}
\pnum
\returns
\begin{codeblock}
lhs.radius() == rhs.radius() && lhs.rotation() == rhs.rotation() &&
lhs.start_angle() && rhs.start_angle()
\end{codeblock}
\end{itemdescr}

\indexlibrarymember{operator!=}{arc}%
\begin{itemdecl}
constexpr bool operator!=(const arc& lhs, const arc& rhs) noexcept;
\end{itemdecl}
\begin{itemdescr}
\pnum
\returns
\tcode{!(lhs == rhs)}.
\end{itemdescr}

%!TEX root = io2d.tex
\rSec0 [pathfactory.patharccounterclockwise] {Class \tcode{path_factory::path_arc_counterclockwise}}

\pnum
\indexlibrary{\idxcode{path_factory::path_arc_counterclockwise}}
The class \tcode{path_factory::path_arc_counterclockwise} describes a path segment that is a circular arc with counterclockwise rotation.

\pnum
It has a center of type \tcode{vector_2d}, a radius of type \tcode{double}, a first angle of type \tcode{double}, and a second angle of type \tcode{double}.

\pnum
The values for the first angle and second angle are in radians.

\pnum
\enternote
Although the value of the second angle may be greater than the value of the first angle, when processed as described in \ref{paths.processing}, \tcode{two_pi<double>} is subtracted from the second angle until the value of the first angle is greater than or equal to the value of the second angle.
\exitnote

\rSec1 [pathfactory.patharccounterclockwise.synopsis] {\tcode{path_factory::path_arc_counterclockwise} synopsis}

\begin{codeblock}
namespace std { namespace experimental { namespace io2d { inline namespace v1 {
  class path_factory::path_arc_counterclockwise {
  public:
    // \ref{pathfactory.patharccounterclockwise.cons}, construct:
    path_arc_counterclockwise(const vector_2d& ctr, double rad, double angle1,
      double angle2) noexcept;

    // \ref{pathfactory.patharccounterclockwise.modifiers}, modifiers:
    void center(const vector_2d& ctr) noexcept;
    void radius(double r) noexcept;
    void angle_1(double radians) noexcept;
    void angle_2(double radians) noexcept;

    // \ref{pathfactory.patharccounterclockwise.observers}, observers:
    vector_2d center() const noexcept;
    double radius() const noexcept;
    double angle_1() const noexcept;
    double angle_2() const noexcept;
  };
} } } }
\end{codeblock}

\rSec1 [pathfactory.patharccounterclockwise.cons] {\tcode{path_factory::path_arc_counterclockwise} constructors and assignment operators}

\indexlibrary{\idxcode{path_factory::path_arc_counterclockwise}!constructor}
\indexlibrary{\idxcode{path_factory::path_arc_counterclockwise}!constructor}
\begin{itemdecl}
    path_arc_counterclockwise(const vector_2d& ctr, double rad, double angle1,
      double angle2) noexcept;
\end{itemdecl}
\begin{itemdescr}
	\pnum
	\effects
	Constructs an object of type \tcode{path_factory::path_arc_counterclockwise}.
	
	\pnum
	The center shall be set to the value of \tcode{ctr}.
	
	\pnum
	The radius shall be set to the value of \tcode{rad}.
	
	\pnum
	The first angle shall be set to the value of \tcode{angle1}.

	\pnum
	The second angle shall be set to the value of \tcode{angle2}.
\end{itemdescr}

\rSec1 [pathfactory.patharccounterclockwise.modifiers]{\tcode{path_factory::path_arc_counterclockwise} modifiers}

\indexlibrary{\idxcode{path_factory::path_arc_counterclockwise}!\idxcode{center}}
\indexlibrary{\idxcode{center}!\idxcode{path_factory::path_arc_counterclockwise}}
\begin{itemdecl}
    void center(const vector_2d& ctr) noexcept;
\end{itemdecl}
\begin{itemdescr}
	\pnum
	\effects
	The center shall be set to the value of \tcode{ctr}.
\end{itemdescr}

\indexlibrary{\idxcode{path_factory::path_arc_counterclockwise}!\idxcode{radius}}
\indexlibrary{\idxcode{radius}!\idxcode{path_factory::path_arc_counterclockwise}}
\begin{itemdecl}
    void radius(double r) noexcept;
\end{itemdecl}
\begin{itemdescr}
	\pnum
	\effects
	The radius shall be set to the value of \tcode{r}.
\end{itemdescr}

\indexlibrary{\idxcode{path_factory::path_arc_counterclockwise}!\idxcode{angle_1}}
\indexlibrary{\idxcode{angle_1}!\idxcode{path_factory::path_arc_counterclockwise}}
\begin{itemdecl}
    void angle_1(double radians) noexcept;
\end{itemdecl}
\begin{itemdescr}
	\pnum
	\effects
	The first angle shall be set to the value of \tcode{radians}.
\end{itemdescr}

\indexlibrary{\idxcode{path_factory::path_arc_counterclockwise}!\idxcode{angle_2}}
\indexlibrary{\idxcode{angle_2}!\idxcode{path_factory::path_arc_counterclockwise}}
\begin{itemdecl}
    void angle_2(double radians) noexcept;
\end{itemdecl}
\begin{itemdescr}
	\pnum
	\effects
	The second angle shall be set to the value of \tcode{radians}.
\end{itemdescr}

\rSec1 [pathfactory.patharccounterclockwise.observers]{\tcode{path_factory::path_arc_counterclockwise} observers}

\indexlibrary{\idxcode{path_factory::path_arc_counterclockwise}!\idxcode{center}}
\indexlibrary{\idxcode{center}!\idxcode{path_factory::path_arc_counterclockwise}}
\begin{itemdecl}
    vector_2d center() const noexcept;
\end{itemdecl}
\begin{itemdescr}
	\pnum
	\returns
	The value of the center.
\end{itemdescr}

\indexlibrary{\idxcode{path_factory::path_arc_counterclockwise}!\idxcode{radius}}
\indexlibrary{\idxcode{radius}!\idxcode{path_factory::path_arc_counterclockwise}}
\begin{itemdecl}
    double radius() const noexcept;
\end{itemdecl}
\begin{itemdescr}
	\pnum
	\returns
	The value of the radius.
\end{itemdescr}

\indexlibrary{\idxcode{path_factory::path_arc_counterclockwise}!\idxcode{angle_1}}
\indexlibrary{\idxcode{angle_1}!\idxcode{path_factory::path_arc_counterclockwise}}
\begin{itemdecl}
    double angle_1() const noexcept;
\end{itemdecl}
\begin{itemdescr}
	\pnum
	\returns
	The value of the first angle.
\end{itemdescr}

\indexlibrary{\idxcode{path_factory::path_arc_counterclockwise}!\idxcode{angle_2}}
\indexlibrary{\idxcode{angle_2}!\idxcode{path_factory::path_arc_counterclockwise}}
\begin{itemdecl}
    double angle_2() const noexcept;
\end{itemdecl}
\begin{itemdescr}
	\pnum
	\returns
	The value of the second angle.
\end{itemdescr}

%!TEX root = io2d.tex
\rSec0 [\iotwod.changematrix] {Class \tcode{change_matrix}}

\rSec1 [\iotwod.changematrix.synopsis] {\tcode{change_matrix} synopsis}

\begin{codeblock}
namespace std { namespace experimental { namespace io2d { inline namespace v1 {
  class change_matrix : public path_data {
  public:
    // \ref{\iotwod.changematrix.cons}, construct/copy/move/destroy:
    change_matrix() noexcept;
    change_matrix(const change_matrix& other) noexcept;
    change_matrix& operator=(const change_matrix& other) noexcept;
    change_matrix(change_matrix&& other) noexcept;
    change_matrix& operator=(change_matrix&& other) noexcept;
    change_matrix(const matrix_2d& m) noexcept;

    // \ref{\iotwod.changematrix.modifiers}, modifiers:
    void matrix(const matrix_2d& value) noexcept;

    // \ref{\iotwod.changematrix.observers}, observers:
    matrix_2d matrix() const noexcept;
    virtual path_data_type type() const noexcept override;
    
  private:
    matrix_2d _Matrix; // \expos
  };
} } } }
\end{codeblock}

\rSec1 [\iotwod.changematrix.intro] {\tcode{change_matrix} Description}

\pnum
\indexlibrary{\idxcode{change_matrix}}
The class \tcode{change_matrix} describes a \tcode{path} operation. For a description of its meaning within a \tcode{path}, see the meaning of \tcode{path_data_type::change_matrix} in Table~\ref{tab:\iotwod.pathdatatype.meanings}.

\rSec1 [\iotwod.changematrix.cons] {\tcode{change_matrix} constructors and assignment operators}

\indexlibrary{\idxcode{change_matrix}!constructor}
\begin{itemdecl}
    change_matrix() noexcept;
\end{itemdecl}
\begin{itemdescr}
	\pnum
	\effects
	Constructs an object of type \tcode{change_matrix}.
	
	\pnum
	\postconditions
	\tcode{_Matrix == matrix_2d\{\}}.
\end{itemdescr}

\indexlibrary{\idxcode{change_matrix}!constructor}
\begin{itemdecl}
    change_matrix(const matrix_2d& m) noexcept;
\end{itemdecl}
\begin{itemdescr}
	\pnum
	\effects
	Constructs an object of type \tcode{change_matrix}.
	
	\pnum
	\postconditions
	\tcode{_Matrix == m}.
\end{itemdescr}

\rSec1 [\iotwod.changematrix.modifiers]{\tcode{change_matrix} modifiers}

\indexlibrary{\idxcode{change_matrix}!\idxcode{matrix}}
\indexlibrary{\idxcode{matrix}!\idxcode{change_matrix}}
\begin{itemdecl}
    void matrix(const matrix_2d& value) noexcept;
\end{itemdecl}
\begin{itemdescr}
	\pnum
	\postconditions
	\tcode{_Matrix == value}.
	
\end{itemdescr}

\rSec1 [\iotwod.changematrix.observers]{\tcode{change_matrix} observers}

\indexlibrary{\idxcode{change_matrix}!\idxcode{matrix}}
\indexlibrary{\idxcode{matrix}!\idxcode{change_matrix}}
\begin{itemdecl}
    matrix_2d matrix() const noexcept;
\end{itemdecl}
\begin{itemdescr}
	\pnum
	\returns
	\tcode{_Matrix}.

\end{itemdescr}

\indexlibrary{\idxcode{change_matrix}!\idxcode{type}}
\indexlibrary{\idxcode{type}!\idxcode{change_matrix}}
\begin{itemdecl}
    virtual path_data_type type() const noexcept override;
\end{itemdecl}
\begin{itemdescr}
	\pnum
	\returns
	\tcode{path_data_type::change_matrix}.

\end{itemdescr}

%!TEX root = io2d.tex
\rSec0 [pathdataitem.changeorigin] {Class \tcode{path_factory::path_change_origin}}

\pnum
\indexlibrary{\idxcode{path_factory::path_change_origin}}
The class \tcode{path_factory::path_change_origin} describes an operation on a path geometry collection.

\pnum
This operation changes the origin point for a path geometry collection to be the value returned by \tcode{*this.origin()}. As shown in \ref{pathgeometries.processing}, the new origin point does not affect any operations that came before this operation. It is only used in processing operations that come after it. It continues to be used until another \tcode{path_factory::path_change_origin} object is encountered or the end of the path geometry collection is reached.

\rSec1 [pathdataitem.changeorigin.synopsis] {\tcode{path_factory::path_change_origin} synopsis}

\begin{codeblock}
namespace std { namespace experimental { namespace io2d { inline namespace v1 {
  class path_factory::path_change_origin {
  public:
    // \ref{pathdataitem.changeorigin.cons}, construct/copy/move/destroy:
    change_origin() noexcept;
    change_origin(const change_origin&) noexcept;
    path_factory::path_change_origin& operator=(const change_origin&) noexcept;
    change_origin(change_origin&&) noexcept;
    path_factory::path_change_origin& operator=(change_origin&&) noexcept;
    explicit change_origin(const vector_2d& pt) noexcept;

    // \ref{pathdataitem.changeorigin.modifiers}, modifiers:
    void origin(const vector_2d& value) noexcept;

    // \ref{pathdataitem.changeorigin.observers}, observers:
    vector_2d origin() const noexcept;
    virtual path_data_type type() const noexcept override;
    
  private:
    vector_2d _Data; // \expos
  };
} } } }
\end{codeblock}

\rSec1 [pathdataitem.changeorigin.cons] {\tcode{path_factory::path_change_origin} constructors and assignment operators}

\indexlibrary{\idxcode{path_factory::path_change_origin}!constructor}
\begin{itemdecl}
    change_origin() noexcept;
\end{itemdecl}
\begin{itemdescr}
	\pnum
	\effects
	Constructs an object of type \tcode{path_factory::path_change_origin}.
	
	\pnum
	\postconditions
	\tcode{_Data == vector_2d(0.0, 0.0)}.
\end{itemdescr}

\indexlibrary{\idxcode{path_factory::path_change_origin}!constructor}
\begin{itemdecl}
    explicit change_origin(const vector_2d& pt) noexcept;
\end{itemdecl}
\begin{itemdescr}
	\pnum
	\effects
	Constructs an object of type \tcode{path_factory::path_change_origin}.
	
	\pnum
	\postconditions
	\tcode{_Data == pt}.
\end{itemdescr}

\rSec1 [pathdataitem.changeorigin.modifiers]{\tcode{path_factory::path_change_origin} modifiers}

\indexlibrary{\idxcode{path_factory::path_change_origin}!\idxcode{origin}}
\indexlibrary{\idxcode{origin}!\idxcode{path_factory::path_change_origin}}
\begin{itemdecl}
    void origin(const vector_2d& value) noexcept;
\end{itemdecl}
\begin{itemdescr}
	\pnum
	\postconditions
	\tcode{_Data == value}.
\end{itemdescr}

\rSec1 [pathdataitem.changeorigin.observers]{\tcode{change_origin} observers}

\indexlibrary{\idxcode{path_factory::path_change_origin}!\idxcode{origin}}
\indexlibrary{\idxcode{origin}!\idxcode{path_factory::path_change_origin}}
\begin{itemdecl}
    vector_2d origin() const noexcept;
\end{itemdecl}
\begin{itemdescr}
	\pnum
	\returns
	\tcode{_Data}.
\end{itemdescr}

\indexlibrary{\idxcode{path_factory::path_move_to}!\idxcode{type}}
\indexlibrary{\idxcode{type}!\idxcode{path_factory::path_move_to}}
\begin{itemdecl}
    virtual path_data_type type() const noexcept override;
\end{itemdecl}
\begin{itemdescr}
	\pnum
	\returns
	\tcode{path_data_type::change_origin}.
\end{itemdescr}

%!TEX root = io2d.tex
\rSec0 [closepath] {Class \tcode{close_path}}

\pnum
\indexlibrary{\idxcode{close_path}}
This class is a path instruction that creates a closed path within a path group.

\pnum
It has an end point of type \tcode{vector_2d}.

\rSec1 [closepath.synopsis] {\tcode{close_path} synopsis}

\begin{codeblock}
namespace std { namespace experimental { namespace io2d { inline namespace v1 {
  namespace path_data {
    class close_path {
      // \ref{closepath.cons}, construct:
      constexpr close_path() noexcept;
      constexpr explicit close_path(const vector_2d& to) noexcept;

      // \ref{closepath.modifiers}, modifiers:
      constexpr void to(const vector_2d& value) noexcept;

      // \ref{closepath.observers}, observers:
      constexpr vector_2d to() const noexcept;
    };
  };
} } } }
\end{codeblock}

\rSec1 [closepath.cons] {\tcode{close_path} constructors}

\indexlibrary{\idxcode{close_path}!constructor}
\begin{itemdecl}
constexpr close_path() noexcept;
\end{itemdecl}
\begin{itemdescr}
\pnum
\effects
Constructs an object of type \tcode{close_path}.

\pnum
The end point shall be set to the value of \tcode{vector_2d\{\}}.
\end{itemdescr}

\indexlibrary{\idxcode{close_path}!constructor}
\begin{itemdecl}
constexpr explicit close_path(const vector_2d& pt) noexcept;
\end{itemdecl}
\begin{itemdescr}
\pnum
\effects
Constructs an object of type \tcode{close_path}.

\pnum
The end point shall be set to the value of \tcode{pt}.
\end{itemdescr}

\rSec1 [closepath.modifiers]{\tcode{abs_move} modifiers}

\indexlibrary{\idxcode{close_path}!\idxcode{to}}
\begin{itemdecl}
constexpr void to(const vector_2d& pt) noexcept;
\end{itemdecl}
\begin{itemdescr}
\pnum
\effects
The end point shall be set to the value of \tcode{pt}.
\end{itemdescr}

\rSec1 [closepath.observers]{\tcode{abs_move} observers}

\indexlibrary{\idxcode{close_path}!\idxcode{to}}
\begin{itemdecl}
constexpr vector_2d to() const noexcept;
\end{itemdecl}
\begin{itemdescr}
\pnum
\returns
The value of the end point.
\end{itemdescr}

%!TEX root = io2d.tex
\rSec0 [\iotwod.curveto] {Class \tcode{curve_to}}

\rSec1 [\iotwod.curveto.synopsis] {\tcode{curve_to} synopsis}

\begin{codeblock}
namespace std { namespace experimental { namespace io2d { inline namespace v1 {
  class curve_to : public path_data {
  public:
    // \ref{\iotwod.curveto.cons}, construct/copy/move/destroy:
    curve_to() noexcept;
    curve_to(const curve_to& other) noexcept;
    curve_to& operator=(const curve_to& other) noexcept;
    curve_to(curve_to&& other) noexcept;
    curve_to& operator=(curve_to&& other) noexcept;
    curve_to(const point& controlPoint1, const point& controlPoint2,
      const point& endPoint) noexcept;

    // \ref{\iotwod.curveto.modifiers}, modifiers:
    void control_point_1(const point& value) noexcept;
    void control_point_2(const point& value) noexcept;
    void end_point(const point& value) noexcept;


    // \ref{\iotwod.curveto.observers}, observers:
    point control_point_1() const noexcept;
    point control_point_2() const noexcept;
    point end_point() const noexcept;
    virtual path_data_type type() const noexcept override;
    
  private:
    point _Control_pt1; // \expos
    point _Control_pt2; // \expos
    point _End_pt;      // \expos
  };
} } } }
\end{codeblock}

\rSec1 [\iotwod.curveto.intro] {\tcode{curve_to} Description}

\pnum
\indexlibrary{\idxcode{curve_to}}
The class \tcode{curve_to} describes a \tcode{path} operation. For a description of its meaning within a \tcode{path}, see the meaning of \tcode{path_data_type::curve_to} in Table~\ref{tab:\iotwod.pathdatatype.meanings}.

\rSec1 [\iotwod.curveto.cons] {\tcode{curve_to} constructors and assignment operators}

\indexlibrary{\idxcode{curve_to}!constructor}
\begin{itemdecl}
    curve_to() noexcept;
\end{itemdecl}
\begin{itemdescr}
	\pnum
	\effects
	Constructs an object of type \tcode{curve_to}.
	
	\pnum
	\postconditions
	\tcode{_Control_pt1 == point(0.0, 0.0)}.

	\tcode{_Control_pt2 == point(0.0, 0.0)}.

	\tcode{_End_pt == point(0.0, 0.0)}.

\end{itemdescr}

\indexlibrary{\idxcode{curve_to}!constructor}
\begin{itemdecl}
    curve_to(const point& controlPoint1, const point& controlPoint2,
      const point& endPoint) noexcept;
\end{itemdecl}
\begin{itemdescr}
	\pnum
	\effects
	Constructs an object of type \tcode{curve_to}.
	
	\pnum
	\postconditions
	\tcode{_Control_pt1 == controlPoint1}.

	\tcode{_Control_pt2 == controlPoint2}.

	\tcode{_End_pt == endPoint}.

\end{itemdescr}

\rSec1 [\iotwod.curveto.modifiers]{\tcode{curve_to} modifiers}

\indexlibrary{\idxcode{curve_to}!\idxcode{control_point_1}}
\indexlibrary{\idxcode{control_point_1}!\idxcode{curve_to}}
\begin{itemdecl}
    void control_point_1(const point& value) noexcept;
\end{itemdecl}
\begin{itemdescr}
	\pnum
	\postconditions
	\tcode{_Control_pt_1 == value}.
	
\end{itemdescr}

\indexlibrary{\idxcode{curve_to}!\idxcode{control_point_2}}
\indexlibrary{\idxcode{control_point_2}!\idxcode{curve_to}}
\begin{itemdecl}
    void control_point_2(const point& value) noexcept;
\end{itemdecl}
\begin{itemdescr}
	\pnum
	\postconditions
	\tcode{_Control_pt_2 == value}.
	
\end{itemdescr}

\indexlibrary{\idxcode{curve_to}!\idxcode{end_point}}
\indexlibrary{\idxcode{end_point}!\idxcode{curve_to}}
\begin{itemdecl}
    void end_point(const point& value) noexcept;
\end{itemdecl}
\begin{itemdescr}
	\pnum
	\postconditions
	\tcode{_End_pt == value}.
	
\end{itemdescr}

\rSec1 [\iotwod.curveto.observers]{\tcode{curve_to} observers}

\indexlibrary{\idxcode{curve_to}!\idxcode{control_point_1}}
\indexlibrary{\idxcode{control_point_1}!\idxcode{curve_to}}
\begin{itemdecl}
    point control_point_1() const noexcept;
\end{itemdecl}
\begin{itemdescr}
	\pnum
	\returns
	\tcode{_Control_pt_1}.

\end{itemdescr}

\indexlibrary{\idxcode{curve_to}!\idxcode{control_point_2}}
\indexlibrary{\idxcode{control_point_2}!\idxcode{curve_to}}
\begin{itemdecl}
    point control_point_2() const noexcept;
\end{itemdecl}
\begin{itemdescr}
	\pnum
	\returns
	\tcode{_Control_pt_2}.

\end{itemdescr}

\indexlibrary{\idxcode{curve_to}!\idxcode{end_point}}
\indexlibrary{\idxcode{end_point}!\idxcode{curve_to}}
\begin{itemdecl}
    point end_point() const noexcept;
\end{itemdecl}
\begin{itemdescr}
	\pnum
	\returns
	\tcode{_End_pt}.

\end{itemdescr}

\indexlibrary{\idxcode{curve_to}!\idxcode{type}}
\indexlibrary{\idxcode{type}!\idxcode{curve_to}}
\begin{itemdecl}
    virtual path_data_type type() const noexcept override;
\end{itemdecl}
\begin{itemdescr}
	\pnum
	\returns
	\tcode{path_data_type::curve_to}.

\end{itemdescr}

%!TEX root = io2d.tex
\rSec0 [lineto] {Class \tcode{line_to}}

\rSec1 [lineto.synopsis] {\tcode{line_to} synopsis}

\begin{codeblock}
namespace std { namespace experimental { namespace io2d { inline namespace v1 {
  class line_to : public path_data {
  public:
    // \ref{lineto.cons}, construct/copy/move/destroy:
    line_to() noexcept;
    line_to(const line_to& other) noexcept;
    line_to& operator=(const line_to& other) noexcept;
    line_to(line_to&& other) noexcept;
    line_to& operator=(line_to&& other) noexcept;
    line_to(const vector_2d& pt) noexcept;

    // \ref{lineto.modifiers}, modifiers:
    void to(const vector_2d& pt) noexcept;

    // \ref{lineto.observers}, observers:
    vector_2d to() const noexcept;
    virtual path_data_type type() const noexcept override;
    
  private:
    vector_2d _Data; // \expos
  };
} } } }
\end{codeblock}

\rSec1 [lineto.intro] {\tcode{line_to} Description}

\pnum
\indexlibrary{\idxcode{line_to}}
The class \tcode{line_to} describes an operation on a path geometry collection.

\pnum
If the current path geometry has a current point then this operation creates a line from the current point to the point returned by \tcode{*this.to()} and then sets current point to be the point returned by \tcode{*this.to()}.

\pnum
If there is no current point, then this operation behaves exactly as if this object was a \tcode{move_to} object with the value returned by \tcode{*this.to()} taking the place of the value returned by \tcode{move_to::to()}.

\rSec1 [lineto.cons] {\tcode{line_to} constructors and assignment operators}

\indexlibrary{\idxcode{line_to}!constructor}
\begin{itemdecl}
    line_to() noexcept;
\end{itemdecl}
\begin{itemdescr}
	\pnum
	\effects
	Constructs an object of type \tcode{line_to}.
	
	\pnum
	\postconditions
	\tcode{_Data == vector_2d(0.0, 0.0)}.
\end{itemdescr}

\indexlibrary{\idxcode{line_to}!constructor}
\begin{itemdecl}
    line_to(const vector_2d& pt) noexcept;
\end{itemdecl}
\begin{itemdescr}
	\pnum
	\effects
	Constructs an object of type \tcode{line_to}.
	
	\pnum
	\postconditions
	\tcode{_Data == pt}.
\end{itemdescr}

\rSec1 [lineto.modifiers]{\tcode{line_to} modifiers}

\indexlibrary{\idxcode{line_to}!\idxcode{to}}
\indexlibrary{\idxcode{to}!\idxcode{line_to}}
\begin{itemdecl}
    void to(const vector_2d& pt) noexcept;
\end{itemdecl}
\begin{itemdescr}
	\pnum
	\postconditions
	\tcode{_Data == pt}.
	
\end{itemdescr}

\rSec1 [lineto.observers]{\tcode{line_to} observers}

\indexlibrary{\idxcode{line_to}!\idxcode{to}}
\indexlibrary{\idxcode{to}!\idxcode{line_to}}
\begin{itemdecl}
    vector_2d to() const noexcept;
\end{itemdecl}
\begin{itemdescr}
	\pnum
	\returns
	\tcode{_Data}.

\end{itemdescr}

\indexlibrary{\idxcode{line_to}!\idxcode{type}}
\indexlibrary{\idxcode{type}!\idxcode{line_to}}
\begin{itemdecl}
    virtual path_data_type type() const noexcept override;
\end{itemdecl}
\begin{itemdescr}
	\pnum
	\returns
	\tcode{path_data_type::line_to}.

\end{itemdescr}

%!TEX root = io2d.tex
\rSec0 [\iotwod.moveto] {Class \tcode{move_to}}

\rSec1 [\iotwod.moveto.synopsis] {\tcode{move_to} synopsis}

\begin{codeblock}
namespace std { namespace experimental { namespace io2d { inline namespace v1 {
  class move_to : public path_data {
  public:
    // \ref{\iotwod.moveto.cons}, construct/copy/move/destroy:
    move_to() noexcept;
    move_to(const move_to& other) noexcept;
    move_to& operator=(const move_to& other) noexcept;
    move_to(move_to&& other) noexcept;
    move_to& operator=(move_to&& other) noexcept;
    move_to(const point& pt) noexcept;

    // \ref{\iotwod.moveto.modifiers}, modifiers:
    void to(const point& pt) noexcept;

    // \ref{\iotwod.moveto.observers}, observers:
    point to() const noexcept;
    virtual path_data_type type() const noexcept override;
    
  private:
    point _Data; // \expos
  };
} } } }
\end{codeblock}

\rSec1 [\iotwod.moveto.intro] {\tcode{move_to} Description}

\pnum
\indexlibrary{\idxcode{move_to}}
The class \tcode{move_to} describes a \tcode{path} operation. For a description of its meaning within a \tcode{path}, see the meaning of \tcode{path_data_type::move_to} in Table~\ref{tab:\iotwod.pathdatatype.meanings}.

\rSec1 [\iotwod.moveto.cons] {\tcode{move_to} constructors and assignment operators}

\indexlibrary{\idxcode{move_to}!constructor}
\begin{itemdecl}
    move_to() noexcept;
\end{itemdecl}
\begin{itemdescr}
	\pnum
	\effects
	Constructs an object of type \tcode{move_to}.
	
	\pnum
	\postconditions
	\tcode{_Data == point(0.0, 0.0)}.
\end{itemdescr}

\indexlibrary{\idxcode{move_to}!constructor}
\begin{itemdecl}
    move_to(const point& pt) noexcept;
\end{itemdecl}
\begin{itemdescr}
	\pnum
	\effects
	Constructs an object of type \tcode{move_to}.
	
	\pnum
	\postconditions
	\tcode{_Data == pt}.
\end{itemdescr}

\rSec1 [\iotwod.moveto.modifiers]{\tcode{move_to} modifiers}

\indexlibrary{\idxcode{move_to}!\idxcode{to}}
\indexlibrary{\idxcode{to}!\idxcode{move_to}}
\begin{itemdecl}
    void to(const point& pt) noexcept;
\end{itemdecl}
\begin{itemdescr}
	\pnum
	\postconditions
	\tcode{_Data == pt}.
	
\end{itemdescr}

\rSec1 [\iotwod.moveto.observers]{\tcode{move_to} observers}

\indexlibrary{\idxcode{move_to}!\idxcode{to}}
\indexlibrary{\idxcode{to}!\idxcode{move_to}}
\begin{itemdecl}
    point to() const noexcept;
\end{itemdecl}
\begin{itemdescr}
	\pnum
	\returns
	\tcode{_Data}.

\end{itemdescr}

\indexlibrary{\idxcode{move_to}!\idxcode{type}}
\indexlibrary{\idxcode{type}!\idxcode{move_to}}
\begin{itemdecl}
    virtual path_data_type type() const noexcept override;
\end{itemdecl}
\begin{itemdescr}
	\pnum
	\returns
	\tcode{path_data_type::move_to}.

\end{itemdescr}

%!TEX root = io2d.tex
\rSec0 [pathdataitem.newpath] {Class \tcode{path_factory::path_new_path}}

\pnum
\indexlibrary{\idxcode{path_factory::path_new_path}}
The class \tcode{path_factory::path_new_path} describes an operation on a path group.

\pnum
This operation starts a new path geometry. The new path geometry has no current point.

\rSec1 [pathdataitem.newpath.synopsis] {\tcode{path_factory::path_new_path} synopsis}

\begin{codeblock}
namespace std { namespace experimental { namespace io2d { inline namespace v1 {
  class path_factory::path_new_path {
  public:
    // construct/copy/move/destroy:
    new_path() noexcept;
    new_path(const new_path&) noexcept;
    path_factory::path_new_path& operator=(const new_path&) noexcept;
    new_path(new_path&&) noexcept;
    path_factory::path_new_path& operator=(new_path&&) noexcept;

    // \ref{pathdataitem.newpath.observers}, observers:
    virtual path_data_type type() const noexcept override;
  };
} } } }
\end{codeblock}

\rSec1 [pathdataitem.newpath.observers]{\tcode{path_factory::path_new_path} observers}

\indexlibrary{\idxcode{path_factory::path_new_path}!\idxcode{type}}
\indexlibrary{\idxcode{type}!\idxcode{path_factory::path_new_path}}
\begin{itemdecl}
    virtual path_data_type type() const noexcept override;
\end{itemdecl}
\begin{itemdescr}
	\pnum
	\returns
	\tcode{path_data_type::new_path}.

\end{itemdescr}

%!TEX root = io2d.tex
\rSec0 [pathdataitem.relcurveto] {Class \tcode{path_data_item::rel_curve_to}}

\pnum
\indexlibrary{\idxcode{path_data_item::rel_curve_to}}
The class \tcode{path_data_item::rel_curve_to} describes an operation on a path geometry collection.

\pnum
This operation creates a cubic B\'ezier curve from the current point to the point that is the sum of the current point and the point returned by \tcode{*this.end_point()}, with the first control point being the point that is the sum of the current point and the point returned by \tcode{*this.control_point_1()} and the second control point being the point that is the sum of the current point and the point returned by \tcode{*this.control_point_2()}. It then sets the current point to be the point that is the sum of the current point and the point returned by \tcode{*this.end_point()}.

\pnum
If the current path geometry does not have a current point when this operation is requested the path geometry collection is malformed.

\rSec1 [pathdataitem.relcurveto.synopsis] {\tcode{path_data_item::rel_curve_to} synopsis}

\begin{codeblock}
namespace std { namespace experimental { namespace io2d { inline namespace v1 {
  class path_data_item::rel_curve_to {
  public:
    // \ref{pathdataitem.relcurveto.cons}, construct/copy/move/destroy:
    rel_curve_to() noexcept;
    rel_curve_to(const rel_curve_to&) noexcept;
    path_data_item::rel_curve_to& operator=(const rel_curve_to&) noexcept;
    rel_curve_to(rel_curve_to&&) noexcept;
    path_data_item::rel_curve_to& operator=(rel_curve_to&&) noexcept;
    rel_curve_to(const vector_2d& controlPoint1, const vector_2d& controlPoint2,
      const vector_2d& endPoint) noexcept;

    // \ref{pathdataitem.relcurveto.modifiers}, modifiers:
    void control_point_1(const vector_2d& value) noexcept;
    void control_point_2(const vector_2d& value) noexcept;
    void end_point(const vector_2d& value) noexcept;

    // \ref{pathdataitem.relcurveto.observers}, observers:
    vector_2d control_point_1() const noexcept;
    vector_2d control_point_2() const noexcept;
    vector_2d end_point() const noexcept;
    virtual path_data_type type() const noexcept override;
    
  private:
    vector_2d _Control_pt1; // \expos
    vector_2d _Control_pt2; // \expos
    vector_2d _End_pt;      // \expos
  };
} } } }
\end{codeblock}

\rSec1 [pathdataitem.relcurveto.cons] {\tcode{path_data_item::rel_curve_to} constructors and assignment operators}

\indexlibrary{\idxcode{path_data_item::rel_curve_to}!constructor}
\begin{itemdecl}
    rel_curve_to() noexcept;
\end{itemdecl}
\begin{itemdescr}
	\pnum
	\effects
	Constructs an object of type \tcode{path_data_item::rel_curve_to}.
	
	\pnum
	\postconditions
	\tcode{_Control_pt1 == vector_2d(0.0, 0.0)}.

	\tcode{_Control_pt2 == vector_2d(0.0, 0.0)}.

	\tcode{_End_pt == vector_2d(0.0, 0.0)}.
\end{itemdescr}

\indexlibrary{\idxcode{path_data_item::rel_curve_to}!constructor}
\begin{itemdecl}
    rel_curve_to(const vector_2d& controlPoint1, const vector_2d& controlPoint2,
      const vector_2d& endPoint) noexcept;
\end{itemdecl}
\begin{itemdescr}
	\pnum
	\effects
	Constructs an object of type \tcode{path_data_item::rel_curve_to}.
	
	\pnum
	\postconditions
	\tcode{_Control_pt1 == controlPoint1}.

	\tcode{_Control_pt2 == controlPoint2}.

	\tcode{_End_pt == endPoint}.
\end{itemdescr}

\rSec1 [pathdataitem.relcurveto.modifiers]{\tcode{path_data_item::rel_curve_to} modifiers}

\indexlibrary{\idxcode{path_data_item::rel_curve_to}!\idxcode{control_point_1}}
\indexlibrary{\idxcode{control_point_1}!\idxcode{path_data_item::rel_curve_to}}
\begin{itemdecl}
    void control_point_1(const vector_2d& value) noexcept;
\end{itemdecl}
\begin{itemdescr}
	\pnum
	\postconditions
	\tcode{_Control_pt_1 == value}.
\end{itemdescr}

\indexlibrary{\idxcode{path_data_item::rel_curve_to}!\idxcode{control_point_2}}
\indexlibrary{\idxcode{control_point_2}!\idxcode{path_data_item::rel_curve_to}}
\begin{itemdecl}
    void control_point_2(const vector_2d& value) noexcept;
\end{itemdecl}
\begin{itemdescr}
	\pnum
	\postconditions
	\tcode{_Control_pt_2 == value}.
\end{itemdescr}

\indexlibrary{\idxcode{path_data_item::rel_curve_to}!\idxcode{end_point}}
\indexlibrary{\idxcode{end_point}!\idxcode{path_data_item::rel_curve_to}}
\begin{itemdecl}
    void end_point(const vector_2d& value) noexcept;
\end{itemdecl}
\begin{itemdescr}
	\pnum
	\postconditions
	\tcode{_End_pt == value}.
\end{itemdescr}

\rSec1 [pathdataitem.relcurveto.observers]{\tcode{path_data_item::rel_curve_to} observers}

\indexlibrary{\idxcode{path_data_item::rel_curve_to}!\idxcode{control_point_1}}
\indexlibrary{\idxcode{control_point_1}!\idxcode{path_data_item::rel_curve_to}}
\begin{itemdecl}
    vector_2d control_point_1() const noexcept;
\end{itemdecl}
\begin{itemdescr}
	\pnum
	\returns
	\tcode{_Control_pt_1}.
\end{itemdescr}

\indexlibrary{\idxcode{path_data_item::rel_curve_to}!\idxcode{control_point_2}}
\indexlibrary{\idxcode{control_point_2}!\idxcode{path_data_item::rel_curve_to}}
\begin{itemdecl}
    vector_2d control_point_2() const noexcept;
\end{itemdecl}
\begin{itemdescr}
	\pnum
	\returns
	\tcode{_Control_pt_2}.
\end{itemdescr}

\indexlibrary{\idxcode{path_data_item::rel_curve_to}!\idxcode{end_point}}
\indexlibrary{\idxcode{end_point}!\idxcode{path_data_item::rel_curve_to}}
\begin{itemdecl}
    vector_2d end_point() const noexcept;
\end{itemdecl}
\begin{itemdescr}
	\pnum
	\returns
	\tcode{_End_pt}.
\end{itemdescr}

\indexlibrary{\idxcode{path_data_item::rel_curve_to}!\idxcode{type}}
\indexlibrary{\idxcode{type}!\idxcode{path_data_item::rel_curve_to}}
\begin{itemdecl}
    virtual path_data_type type() const noexcept override;
\end{itemdecl}
\begin{itemdescr}
	\pnum
	\returns
	\tcode{path_data_type::rel_curve_to}.
\end{itemdescr}

%!TEX root = io2d.tex
\rSec0 [\iotwod.rellineto] {Class \tcode{rel_line_to}}

\rSec1 [\iotwod.rellineto.synopsis] {\tcode{rel_line_to} synopsis}

\begin{codeblock}
namespace std { namespace experimental { namespace io2d { inline namespace v1 {
  class rel_line_to : public path_data {
  public:
    // \ref{\iotwod.rellineto.cons}, construct/copy/move/destroy:
    line_to() noexcept;
    line_to(const line_to& other) noexcept;
    line_to& operator=(const line_to& other) noexcept;
    line_to(line_to&& other) noexcept;
    line_to& operator=(line_to&& other) noexcept;
    line_to(const point& pt) noexcept;

    // \ref{\iotwod.rellineto.modifiers}, modifiers:
    void to(const point& pt) noexcept;

    // \ref{\iotwod.rellineto.observers}, observers:
    point to() const noexcept;
    virtual path_data_type type() const noexcept override;
    
  private:
    point _Data; // \expos
  };
} } } }
\end{codeblock}

\rSec1 [\iotwod.rellineto.intro] {\tcode{rel_line_to} Description}

\pnum
\indexlibrary{\idxcode{rel_line_to}}
The class \tcode{rel_line_to} describes a \tcode{path} operation. For a description of its meaning within a \tcode{path}, see the meaning of \tcode{path_data_type::rel_line_to} in Table~\ref{tab:\iotwod.pathdatatype.meanings}.

\rSec1 [\iotwod.rellineto.cons] {\tcode{rel_line_to} constructors and assignment operators}

\indexlibrary{\idxcode{rel_line_to}!constructor}
\begin{itemdecl}
    rel_line_to() noexcept;
\end{itemdecl}
\begin{itemdescr}
	\pnum
	\effects
	Constructs an object of type \tcode{rel_line_to}.
	
	\pnum
	\postconditions
	\tcode{_Data == point(0.0, 0.0)}.
\end{itemdescr}

\indexlibrary{\idxcode{rel_line_to}!constructor}
\begin{itemdecl}
    rel_line_to(const point& pt) noexcept;
\end{itemdecl}
\begin{itemdescr}
	\pnum
	\effects
	Constructs an object of type \tcode{rel_line_to}.
	
	\pnum
	\postconditions
	\tcode{_Data == pt}.
\end{itemdescr}

\rSec1 [\iotwod.rellineto.modifiers]{\tcode{rel_line_to} modifiers}

\indexlibrary{\idxcode{rel_line_to}!\idxcode{to}}
\indexlibrary{\idxcode{to}!\idxcode{rel_line_to}}
\begin{itemdecl}
    void to(const point& pt) noexcept;
\end{itemdecl}
\begin{itemdescr}
	\pnum
	\postconditions
	\tcode{_Data == pt}.
	
\end{itemdescr}

\rSec1 [\iotwod.rellineto.observers]{\tcode{rel_line_to} observers}

\indexlibrary{\idxcode{rel_line_to}!\idxcode{to}}
\indexlibrary{\idxcode{to}!\idxcode{rel_line_to}}
\begin{itemdecl}
    point to() const noexcept;
\end{itemdecl}
\begin{itemdescr}
	\pnum
	\returns
	\tcode{_Data}.

\end{itemdescr}

\indexlibrary{\idxcode{rel_line_to}!\idxcode{type}}
\indexlibrary{\idxcode{type}!\idxcode{rel_line_to}}
\begin{itemdecl}
    virtual path_data_type type() const noexcept override;
\end{itemdecl}
\begin{itemdescr}
	\pnum
	\returns
	\tcode{path_data_type::rel_line_to}.

\end{itemdescr}

%!TEX root = io2d.tex
\rSec0 [pathdataitem.relmoveto] {Class \tcode{rel_move_to}}

\rSec1 [pathdataitem.relmoveto.synopsis] {\tcode{rel_move_to} synopsis}

\begin{codeblock}
namespace std { namespace experimental { namespace io2d { inline namespace v1 {
  class rel_move_to : public path_data {
  public:
    // \ref{pathdataitem.relmoveto.cons}, construct/copy/move/destroy:
    rel_move_to() noexcept;
    rel_move_to(const rel_move_to& other) noexcept;
    rel_move_to& operator=(const rel_move_to& other) noexcept;
    rel_move_to(rel_move_to&& other) noexcept;
    rel_move_to& operator=(rel_move_to&& other) noexcept;
    rel_move_to(const vector_2d& pt) noexcept;

    // \ref{pathdataitem.relmoveto.modifiers}, modifiers:
    void to(const vector_2d& pt) noexcept;

    // \ref{pathdataitem.relmoveto.observers}, observers:
    vector_2d to() const noexcept;
    virtual path_data_type type() const noexcept override;
    
  private:
    vector_2d _Data; // \expos
  };
} } } }
\end{codeblock}

\rSec1 [pathdataitem.relmoveto.intro] {\tcode{rel_move_to} Description}

\pnum
\indexlibrary{\idxcode{rel_move_to}}
The class \tcode{rel_move_to} describes an operation on a path geometry collection.

\pnum
This operation starts a new path geometry and sets its current point and last-move-to point to the point that is the sum of the previous path geometry's current point and the point returned by \tcode{*this.to()}.

\pnum
If the existing path geometry does not have a current point when this operation is requested the path geometry collection is malformed.

\rSec1 [pathdataitem.relmoveto.cons] {\tcode{rel_move_to} constructors and assignment operators}

\indexlibrary{\idxcode{rel_move_to}!constructor}
\begin{itemdecl}
    rel_move_to() noexcept;
\end{itemdecl}
\begin{itemdescr}
	\pnum
	\effects
	Constructs an object of type \tcode{rel_move_to}.
	
	\pnum
	\postconditions
	\tcode{_Data == vector_2d(0.0, 0.0)}.
\end{itemdescr}

\indexlibrary{\idxcode{rel_move_to}!constructor}
\begin{itemdecl}
    rel_move_to(const vector_2d& pt) noexcept;
\end{itemdecl}
\begin{itemdescr}
	\pnum
	\effects
	Constructs an object of type \tcode{rel_move_to}.
	
	\pnum
	\postconditions
	\tcode{_Data == pt}.
\end{itemdescr}

\rSec1 [pathdataitem.relmoveto.modifiers]{\tcode{rel_move_to} modifiers}

\indexlibrary{\idxcode{rel_move_to}!\idxcode{to}}
\indexlibrary{\idxcode{to}!\idxcode{rel_move_to}}
\begin{itemdecl}
    void to(const vector_2d& pt) noexcept;
\end{itemdecl}
\begin{itemdescr}
	\pnum
	\postconditions
	\tcode{_Data == pt}.
	
\end{itemdescr}

\rSec1 [pathdataitem.relmoveto.observers]{\tcode{rel_move_to} observers}

\indexlibrary{\idxcode{rel_move_to}!\idxcode{to}}
\indexlibrary{\idxcode{to}!\idxcode{rel_move_to}}
\begin{itemdecl}
    vector_2d to() const noexcept;
\end{itemdecl}
\begin{itemdescr}
	\pnum
	\returns
	\tcode{_Data}.

\end{itemdescr}

\indexlibrary{\idxcode{rel_move_to}!\idxcode{type}}
\indexlibrary{\idxcode{type}!\idxcode{rel_move_to}}
\begin{itemdecl}
    virtual path_data_type type() const noexcept override;
\end{itemdecl}
\begin{itemdescr}
	\pnum
	\returns
	\tcode{path_data_type::rel_move_to}.

\end{itemdescr}

%%!TEX root = io2d.tex
\rSec0 [pathdataitem.get] {Class \tcode{path_data_item::get} member function template specializations}

\rSec1 [pathdataitem.get.synopsis] {\tcode{path_data_item::get} synopsis}

\begin{codeblock}
namespace std { namespace experimental { namespace io2d { inline namespace v1 {
  template <>
  path_data_item::arc path_data_item::get() const;
  template <>
  path_data_item::arc path_data_item::get(error_code& ec) const noexcept;
  
  template <>
  path_data_item::arc_negative path_data_item::get() const;
  template <>
  path_data_item::arc_negative path_data_item::get(error_code& ec) const 
    noexcept;
  
  template <>
  inline path_data_item::change_matrix path_data_item::get() const;
  template <>
  path_data_item::change_matrix path_data_item::get(error_code& ec) const 
    noexcept;
  
  template <>
  path_data_item::change_origin path_data_item::get() const;
  template <>
  path_data_item::change_origin path_data_item::get(error_code& ec) const 
    noexcept;
  
  template <>
  path_data_item::close_path path_data_item::get() const;
  template <>
  path_data_item::close_path path_data_item::get(error_code& ec) const noexcept;
  
  template <>
  path_data_item::curve_to path_data_item::get() const;
  template <>
  path_data_item::curve_to path_data_item::get(error_code& ec) const noexcept;

  template <>
  path_data_item::rel_curve_to path_data_item::get() const;
  template <>
  path_data_item::rel_curve_to path_data_item::get(error_code& ec) const 
    noexcept;
  
  template <>
  path_data_item::new_sub_path path_data_item::get() const;
  template <>
  path_data_item::new_sub_path path_data_item::get(error_code& ec) const 
    noexcept;

  template <>
  path_data_item::line_to path_data_item::get() const;
  template <>
  path_data_item::line_to path_data_item::get(error_code& ec) const noexcept;

  template <>
  path_data_item::move_to path_data_item::get() const;
  template <>
  path_data_item::move_to path_data_item::get(error_code& ec) const noexcept;

  template <>
  path_data_item::rel_line_to path_data_item::get() const;
  template <>
  path_data_item::rel_line_to path_data_item::get(error_code& ec) const 
    noexcept;

  template <>
  path_data_item::rel_move_to path_data_item::get() const;
  template <>
  path_data_item::rel_move_to path_data_item::get(error_code& ec) const 
    noexcept;
} } } }
\end{codeblock}

\rSec1 [pathdataitem.get.specializations]{\tcode{path_data_item::get} specializations}

\indexlibrary{\idxcode{path_data_item}!\idxcode{get}}
\indexlibrary{\idxcode{get}!\idxcode{path_data_item}}
\begin{itemdecl}
template <>
path_data_item::arc path_data_item::get() const;
template <>
path_data_item::arc path_data_item::get(error_code& ec) const noexcept;
\end{itemdecl}
\begin{itemdescr}
\pnum
\returns
A copy of the stored \tcode{path_data_item::arc} object.

\pnum
If an error occurs and the function was called with an \tcode{error_code\&} argument, returns \tcode{path_data_item::arc\{ \}}.

\pnum
\throws
As specified in Error reporting (\ref{\iotwod.err.report}).

\pnum
\errors
\tcode{errc::operation_not_permitted} if \tcode{!this->has_data()}.

\pnum
\tcode{errc::invalid_argument} if \tcode{this->type() != path_data_type::arc}.
\end{itemdescr}

\indexlibrary{\idxcode{path_data_item}!\idxcode{get}}
\indexlibrary{\idxcode{get}!\idxcode{path_data_item}}
\begin{itemdecl}
template <>
path_data_item::arc_negative path_data_item::get() const;
template <>
path_data_item::arc_negative path_data_item::get(error_code& ec) const noexcept;
\end{itemdecl}
\begin{itemdescr}
\pnum
\returns
A copy of the stored \tcode{path_data_item::arc_negative} object.

\pnum
If an error occurs and the function was called with an \tcode{error_code\&} argument, returns \tcode{path_data_item::arc_negative\{ \}}.

\pnum
\throws
As specified in Error reporting (\ref{\iotwod.err.report}).

\pnum
\errors
\tcode{errc::operation_not_permitted} if \tcode{!this->has_data()}.

\pnum
\tcode{errc::invalid_argument} if \tcode{this->type() != path_data_type::arc_negative}.
\end{itemdescr}

\indexlibrary{\idxcode{path_data_item}!\idxcode{get}}
\indexlibrary{\idxcode{get}!\idxcode{path_data_item}}
\begin{itemdecl}
template <>
inline path_data_item::change_matrix path_data_item::get() const;
template <>
path_data_item::change_matrix path_data_item::get(error_code& ec) const 
noexcept;
\end{itemdecl}
\begin{itemdescr}
\pnum
\returns
A copy of the stored \tcode{path_data_item::change_matrix} object.

\pnum
If an error occurs and the function was called with an \tcode{error_code\&} argument, returns \tcode{path_data_item::change_matrix\{ \}}.

\pnum
\throws
As specified in Error reporting (\ref{\iotwod.err.report}).

\pnum
\errors
\tcode{errc::operation_not_permitted} if \tcode{!this->has_data()}.

\pnum
\tcode{errc::invalid_argument} if \tcode{this->type() != path_data_type::change_matrix}.
\end{itemdescr}

\indexlibrary{\idxcode{path_data_item}!\idxcode{get}}
\indexlibrary{\idxcode{get}!\idxcode{path_data_item}}
\begin{itemdecl}
template <>
path_data_item::change_origin path_data_item::get() const;
template <>
path_data_item::change_origin path_data_item::get(error_code& ec) const 
  noexcept;
\end{itemdecl}
\begin{itemdescr}
\pnum
\returns
A copy of the stored \tcode{path_data_item::change_origin} object.

\pnum
If an error occurs and the function was called with an \tcode{error_code\&} argument, returns \tcode{path_data_item::change_origin\{ \}}.

\pnum
\throws
As specified in Error reporting (\ref{\iotwod.err.report}).

\pnum
\errors
\tcode{errc::operation_not_permitted} if \tcode{!this->has_data()}.

\pnum
\tcode{errc::invalid_argument} if \tcode{this->type() != path_data_type::change_origin}.
\end{itemdescr}

\indexlibrary{\idxcode{path_data_item}!\idxcode{get}}
\indexlibrary{\idxcode{get}!\idxcode{path_data_item}}
\begin{itemdecl}
template <>
path_data_item::close_path path_data_item::get() const;
template <>
path_data_item::close_path path_data_item::get(error_code& ec) const noexcept;
\end{itemdecl}
\begin{itemdescr}
\pnum
\returns
A copy of the stored \tcode{path_data_item::close_path} object.

\pnum
If an error occurs and the function was called with an \tcode{error_code\&} argument, returns \tcode{path_data_item::close_path\{ \}}.

\pnum
\throws
As specified in Error reporting (\ref{\iotwod.err.report}).

\pnum
\errors
\tcode{errc::operation_not_permitted} if \tcode{!this->has_data()}.

\pnum
\tcode{errc::invalid_argument} if \tcode{this->type() != path_data_type::close_path}.
\end{itemdescr}

\indexlibrary{\idxcode{path_data_item}!\idxcode{get}}
\indexlibrary{\idxcode{get}!\idxcode{path_data_item}}
\begin{itemdecl}
template <>
path_data_item::rel_curve_to path_data_item::get() const;
template <>
path_data_item::rel_curve_to path_data_item::get(error_code& ec) const noexcept;
\end{itemdecl}
\begin{itemdescr}
\pnum
\returns
A copy of the stored \tcode{path_data_item::rel_curve_to} object.

\pnum
If an error occurs and the function was called with an \tcode{error_code\&} argument, returns \tcode{path_data_item::rel_curve_to\{ \}}.

\pnum
\throws
As specified in Error reporting (\ref{\iotwod.err.report}).

\pnum
\errors
\tcode{errc::operation_not_permitted} if \tcode{!this->has_data()}.

\pnum
\tcode{errc::invalid_argument} if \tcode{this->type() != path_data_type::rel_curve_to}.
\end{itemdescr}

\indexlibrary{\idxcode{path_data_item}!\idxcode{get}}
\indexlibrary{\idxcode{get}!\idxcode{path_data_item}}
\begin{itemdecl}
template <>
path_data_item::new_sub_path path_data_item::get() const;
template <>
path_data_item::new_sub_path path_data_item::get(error_code& ec) const noexcept;
\end{itemdecl}
\begin{itemdescr}
\pnum
\returns
A copy of the stored \tcode{path_data_item::new_sub_path} object.

\pnum
If an error occurs and the function was called with an \tcode{error_code\&} argument, returns \tcode{path_data_item::new_sub_path\{ \}}.

\pnum
\throws
As specified in Error reporting (\ref{\iotwod.err.report}).

\pnum
\errors
\tcode{errc::operation_not_permitted} if \tcode{!this->has_data()}.

\pnum
\tcode{errc::invalid_argument} if \tcode{this->type() != path_data_type::new_sub_path}.
\end{itemdescr}

\indexlibrary{\idxcode{path_data_item}!\idxcode{get}}
\indexlibrary{\idxcode{get}!\idxcode{path_data_item}}
\begin{itemdecl}
template <>
path_data_item::line_to path_data_item::get() const;
template <>
path_data_item::line_to path_data_item::get(error_code& ec) const noexcept;
\end{itemdecl}
\begin{itemdescr}
\pnum
\returns
A copy of the stored \tcode{path_data_item::line_to} object.

\pnum
If an error occurs and the function was called with an \tcode{error_code\&} argument, returns \tcode{path_data_item::line_to\{ \}}.

\pnum
\throws
As specified in Error reporting (\ref{\iotwod.err.report}).

\pnum
\errors
\tcode{errc::operation_not_permitted} if \tcode{!this->has_data()}.

\pnum
\tcode{errc::invalid_argument} if \tcode{this->type() != path_data_type::line_to}.
\end{itemdescr}

\indexlibrary{\idxcode{path_data_item}!\idxcode{get}}
\indexlibrary{\idxcode{get}!\idxcode{path_data_item}}
\begin{itemdecl}
template <>
path_data_item::move_to path_data_item::get() const;
template <>
path_data_item::move_to path_data_item::get(error_code& ec) const noexcept;
\end{itemdecl}
\begin{itemdescr}
\pnum
\returns
A copy of the stored \tcode{path_data_item::move_to} object.

\pnum
If an error occurs and the function was called with an \tcode{error_code\&} argument, returns \tcode{path_data_item::move_to\{ \}}.

\pnum
\throws
As specified in Error reporting (\ref{\iotwod.err.report}).

\pnum
\errors
\tcode{errc::operation_not_permitted} if \tcode{!this->has_data()}.

\pnum
\tcode{errc::invalid_argument} if \tcode{this->type() != path_data_type::move_to}.
\end{itemdescr}

\indexlibrary{\idxcode{path_data_item}!\idxcode{get}}
\indexlibrary{\idxcode{get}!\idxcode{path_data_item}}
\begin{itemdecl}
template <>
path_data_item::rel_line_to path_data_item::get() const;
template <>
path_data_item::rel_line_to path_data_item::get(error_code& ec) const noexcept;
\end{itemdecl}
\begin{itemdescr}
\pnum
\returns
A copy of the stored \tcode{path_data_item::rel_line_to} object.

\pnum
If an error occurs and the function was called with an \tcode{error_code\&} argument, returns \tcode{path_data_item::rel_line_to\{ \}}.

\pnum
\throws
As specified in Error reporting (\ref{\iotwod.err.report}).

\pnum
\errors
\tcode{errc::operation_not_permitted} if \tcode{!this->has_data()}.

\pnum
\tcode{errc::invalid_argument} if \tcode{this->type() != path_data_type::rel_line_to}.
\end{itemdescr}

\indexlibrary{\idxcode{path_data_item}!\idxcode{get}}
\indexlibrary{\idxcode{get}!\idxcode{path_data_item}}
\begin{itemdecl}
template <>
path_data_item::rel_move_to path_data_item::get() const;
template <>
path_data_item::rel_move_to path_data_item::get(error_code& ec) const noexcept;
\end{itemdecl}
\begin{itemdescr}
\pnum
\returns
A copy of the stored \tcode{path_data_item::rel_move_to} object.

\pnum
If an error occurs and the function was called with an \tcode{error_code\&} argument, returns \tcode{path_data_item::rel_move_to\{ \}}.

\pnum
\throws
As specified in Error reporting (\ref{\iotwod.err.report}).

\pnum
\errors
\tcode{errc::operation_not_permitted} if \tcode{!this->has_data()}.

\pnum
\tcode{errc::invalid_argument} if \tcode{this->type() != path_data_type::rel_move_to}.
\end{itemdescr}

\addtocounter{SectionDepthBase}{-1}
\addtocounter{SectionDepthBase}{-1}
