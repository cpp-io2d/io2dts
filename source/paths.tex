%!TEX root = io2d.tex

\rSec0 [paths] {Paths}

\pnum
Paths define geometric objects which can be stroked (Table~\ref{tab:surface.rendering.operations}), filled, masked, and used to define or modify a Clip Area (Table~\ref{tab:surface.state.listing}).

\pnum
Paths are created using a \tcode{path_factory} object, which stores a path group. 

\pnum
Paths provide vector graphics functionality. As such they are particularly useful in situations where an application is intended to run on a variety of platforms whose output devices (\ref{displaysurface.intro}) span a large gamut of sizes, both in terms of measurement units and in terms of a horizontal and vertical pixel count, in that order. For example, a pixel count expressed as 1280x720 means that there are 1280 horizontal pixels per row of pixels and 720 vertical pixels per column of pixels for a total of 921600 pixels.
%
%\pnum
%For output devices, the measurement size of a pixel is determined by the physical size of the output device. Many output devices represent pixels as having the same horizontal and vertical measurement sizes. As such, when they display a rendered image which does not have the same horizontal to vertical pixel ratio as the output device, it 

\pnum
A path may contain degenerate path segments because of special rules, which are set forth below.

\pnum
A \tcode{path_group} object is an immutable resource wrapper containing a path group (\ref{pathgroup}). A \tcode{path_group} object is created from a \tcode{path_factory} object. It can also be default constructed, in which case the \tcode{path_group} object contains no paths.

\rSec1 [paths.processing] {Processing paths}

\pnum
This section is normative. It describes how to convert the path group of a properly formed \tcode{path_factory} object from a collection of \tcode{path_factory::path_data} objects to a collection of \tcode{path_factory::path_data} objects which have had their points transformed in accordance with the origin and transformation matrix of the \tcode{path_factory} object and any \tcode{path_factory::path_change_origin} and \tcode{path_factory::path_change_matrix} objects in the path group of the \tcode{path_factory} object.

%that consists entirely of lines and cubic \bezierlocal curves. The coordinates contained in these lines and cubic \bezierlocal curves are transformed using a default constructed origin of type \tcode{vector_2d} and a default constructed transformation matrix of type \tcode{matrix_2d}. These are modified when a path instruction that changes the origin or the transformation matrix is encountered.
%todo Eliminate native_path_group and replace it with a vector<path_factory::path_data>. Need to make sure that closed path and open path are denoted properly.

%\pnum
%The \tcode{native_path_group} class, described below, is informative. It is used to demonstrate how to perform this process. 
%
%\pnum
%The \tcode{native_path_group} class has the following state data, the types of which are unspecified:
%
%\begin{libreqtab2}
%	{\tcode{native_path_group} state data}
%	{tab:paths.processing.natpathgroup}
%	\\ \topline
%	\lhdr{Name}
%	& \rhdr{Use}
%	\\ \capsep
%	\endfirsthead
%	\continuedcaption\\
%	\hline
%	\lhdr{Name}
%	& \rhdr{Use}
%	\\ \capsep
%	\endhead
%	Current Point
%	& The start point for a path segment that is added to the Current Path.
%	\\
%	Close Point
%	& The start point for the initial path segment in the Current Path.
%	\\
%	Current Path
%	& The path to which path segments are added.
%	\\
%	Collection
%	& The collection of all paths added to the \tcode{native_path_group} object. A new path that is added to the collection is added to the end of the collection.
%	\\
%\end{libreqtab2}
%
%\begin{codeblock}
%	// \textit{This class is }\expos
%	class native_path_group {
%		public:
%		void current_point(const vector_2d& pt) noexcept;
%		void close_point(const vector_2d& pt) noexcept;
%		void line_to(const vector_2d& pt) noexcept;
%		void curve_to(const vector_2d& control1, const vector_2d& control2,
%		const vector_2d& endPt) noexcept;
%		void close_path() noexcept;
%	};
%\end{codeblock}
%
%\begin{itemdecl}
%	void current_point(const vector_2d& pt) noexcept;
%\end{itemdecl}
%\begin{itemdescr}
%	\pnum
%	\effects
%	If the Collection contains no paths, a new path is created, added to the Collection, and set as the Current Path.  The Current Point is set to the value of \tcode{pt}.
%	
%	\pnum
%	Otherwise, if the last member function called was \tcode{current_point}, sets \tcode{pt} as the Current Point.
%	
%	\pnum
%	Otherwise, unless the last member function called was \tcode{close_path}, the Current Path becomes an open path. Then a new path is created, added to the Collection, and set as the Current Path. The Current Point is set to the value of \tcode{pt}.
%	
%	\pnum
%	Otherwise, a new path is created, added to the Collection, and set as the Current Path. The Current Point is set to the value of \tcode{pt}.
%\end{itemdescr}
%
%\begin{itemdecl}
%	void close_point(const vector_2d& pt) noexcept;
%\end{itemdecl}
%\begin{itemdescr}
%	\pnum
%	\preconditions
%	There is a Current Path.
%	
%	\pnum
%	\effects
%	Sets \tcode{pt} as the Close Point.
%\end{itemdescr}
%
%\begin{itemdecl}
%	void line_to(const vector_2d& pt) noexcept;
%\end{itemdecl}
%\begin{itemdescr}
%	\pnum
%	\preconditions
%	There is a Current Path.
%	
%	\pnum
%	\effects
%	Creates a line segment from the Current Point to \tcode{pt} and adds it to the Current Path.
%\end{itemdescr}
%
%\begin{itemdecl}
%	void curve_to(const vector_2d& cpt1, const vector_2d& cpt2,
%	const vector_2d& endPt) noexcept;
%\end{itemdecl}
%\begin{itemdescr}
%	\pnum
%	\preconditions
%	There is a Current Path.
%	
%	\pnum
%	\effects
%	Creates a cubic \bezierlocal curve from the Current Point to \tcode{endPt} using \tcode{cpt1} as the first control point and \tcode{cpt2} as the second control point and adds it to the Current Path.
%\end{itemdescr}
%
%\begin{itemdecl}
%	void close_path() noexcept;
%\end{itemdecl}
%\begin{itemdescr}
%	\pnum
%	\preconditions
%	There is a Current Path.
%	
%	\pnum
%	\effects
%	Creates a line segment from the Current Point to the Close Point.
%	
%	\pnum
%	The Current Path becomes a closed path.
%	
%\end{itemdescr}
%
%\pnum
%\enternote
%A path geometry graphics resource that only supports rendering triangles is possible. The triangles would be used to form lines and to approximate curves. This description assumes the existence of a path geometry graphics resource that performs those actions where needed.
%\exitnote
%
\pnum
The following code shows how to properly process a \tcode{path_factory} object \tcode{p} and store the results into a \tcode{vector<path_factory::path_data>}:

\begin{codeblock}
  #include <cmath>
  #include <vector>
  #include <variant>
  #include <experimental/io2d>
  using namespace std;
  using namespace std::experimental::io2d;
  
  matrix_2d m;
  vector_2d origin;
  vector_2d currentPoint; // Tracks the untransformed current point.
  bool hasCurrentPoint = false;
  vector_2d closePoint;   // Tracks the transformed close point.
  vector<path_factory::path_data> v;
  
  for (auto val : p) {
    std::visit([&](auto&& item) {
      using T = std::remove_cv_t<std::remove_reference_t<decltype item>>;

      if constexpr(std::is_same_v<T, path_factory::path_abs_move>) {
        currentPoint = item.to();
        auto pt = m.transform_point(currentPoint - origin) + origin;
%        n.current_point(pt);
        hasCurrentPoint = true;
        v.emplace_back(in_place_type_t<path_factory::path_abs_move>, pt);
        closePoint = pt
%        n.close_point(pt);
      }
      else if constexpr(std::is_same_v<T, path_factory::path_abs_line>) {
        currentPoint = item.to();
        auto pt = m.transform_point(currentPoint - origin) + origin;
        if (hasCurrentPoint) {
%          n.line_to(pt);
          v.emplace_back(in_place_type_t<path_factory::path_abs_line>, pt);
        }
        else {
%          n.current_point(pt);
          v.emplace_back(in_place_type_t<path_factory::path_abs_move>, pt);
          v.emplace_back(in_place_type_t<path_factory::path_abs_line>, pt);
          hasCurrentPoint = true;
%          currentPoint = pt;
          closePoint = pt;
%          n.close_point(pt);
        }
      }
      else if constexpr(std::is_same_v<T, path_factory::path_abs_curve>) {
        auto pt1 = m.transform_point(item.control_point_1() - origin) + origin;
        auto pt2 = m.transform_point(item.control_point_2() - origin) + origin;
        auto pt3 = m.transform_point(item.end_point() - origin) + origin;
        if (!hasCurrentPoint) {
          currentPoint = item.control_point_1();
%          n.current_point(pt1);
          v.emplace_back(in_place_type_t<path_factory::path_abs_move>, pt1);
          hasCurrentPoint = true;
%          currentPoint = pt1;
          closePoint = pt1;
%          n.close_point(pt1);
        }
%       n.curve_to(pt1, pt2, pt3);
        v.emplace_back(in_place_type_t<path_factory::path_abs_curve_to>, pt1,
          pt2, pt3);
 
        currentPoint = item.end_point();
      }
      else if constexpr(std::is_same_v<T, path_factory::path_new_path>) {
        hasCurrentPoint = false;
        v.emplace_back(in_place_type_t<path_factory::path_new_path>);
      }
      else if constexpr(std::is_same_v<T, path_factory::path_close_path>) {
        if (hasCurrentPoint) {
%          n.close_path();
%          n.current_point(closePoint);
          v.emplace_back(in_place_type_t<path_factory::path_close_path>);
          v.emplace_back(in_place_type_t<path_factory::path_abs_move>,
            closePoint);
          auto invM = matrix_2d{m}.invert();
          // Need the untransformed value for currentPoint.
          currentPoint = invM.transform_point(closePoint - origin) + origin;
        }
      }
      else if constexpr(std::is_same_v<T, path_factory::path_rel_move>) {
        currentPoint = item.to() + currentPoint;
        auto pt = m.transform_point(currentPoint - origin) + origin;
%        n.current_point(pt);
        v.emplace_back(in_place_type_t<path_factory::path_abs_move>, pt);
        hasCurrentPoint = true;
        closePoint = pt    
        n.close_point(pt);
      }
      else if constexpr(std::is_same_v<T, path_factory::path_rel_line>) {
        currentPoint = item.to() + currentPoint;
        auto pt = m.transform_point(currentPoint - origin) + origin;
%        n.line_to(pt);
        v.emplace_back(in_place_type_t<path_factory::path_abs_line>, pt);
      }
      else if constexpr(std::is_same_v<T, path_factory::path_rel_curve>) {
        auto pt1 = m.transform_point(item.control_point_1() + currentPoint -
        origin) + origin;
        auto pt2 = m.transform_point(item.control_point_2() + currentPoint -
        origin) + origin;
        auto pt3 = m.transform_point(item.end_point() + currentPoint - origin) +
        origin;
%        n.curve_to(pt1, pt2, pt3);
        v.emplace_back(in_place_type_t<path_factory::path_abs_curve>, pt1, pt2, pt3);
        currentPoint = item.end_point() + currentPoint;
      }
      else if constexpr(std::is_same_v<T, path_factory::path_arc_clockwise>) {
        auto ctr = item.center();
        auto rad = item.radius();
        auto ang1 = item.angle_1();
        auto ang2 = item.angle_2();
        while(ang2 < ang1) {
          ang2 += twopi;
        }
        vector_2d pt0, pt1, pt2, pt3;
        int bezCount = 1;
        double theta = ang2 - ang1;
        double phi;
        while (theta >= halfpi) {
          theta /= 2.0;
          bezCount += bezCount;
        }
        phi = theta / 2.0;
        auto cosPhi = cos(phi);
        auto sinPhi = sin(phi);
        pt0.x(cosPhi);
        pt0.y(-sinPhi);
        pt3.x(pt0.x());
        pt3.y(-pt0.y());
        pt1.x((4.0 - cosPhi) / 3.0);
        pt1.y(-(((1.0 - cosPhi) * (3.0 - cosPhi)) / (3.0 * sinPhi)));
        pt2.x(pt1.x());
        pt2.y(-pt1.y());
        phi = -phi;
        auto rotCwFn = [](const vector_2d& pt, double a) -> vector_2d {
          return { pt.x() * cos(a) + pt.y() * sin(a),
            -(pt.x() * -(sin(a)) + pt.y() * cos(a)) };
        };
        pt0 = rotCwFn(pt0, phi);
        pt1 = rotCwFn(pt1, phi);
        pt2 = rotCwFn(pt2, phi);
        pt3 = rotCwFn(pt3, phi);
        
        auto currTheta = ang1;
        const auto startPt =
        ctr + rotCwFn({ pt0.x() * rad, pt0.y() * rad }, currTheta);
        if (hasCurrentPoint) {
          currentPoint = startPt;
          auto pt = m.transform_point(currentPoint - origin) + origin;
%          n.line_to(pt);
          v.emplace_back(in_place_type_t<path_factory::path_abs_line>, pt);
        }
        else {
          currentPoint = startPt;
          auto pt = m.transform_point(currentPoint - origin) + origin;
%          n.current_point(pt);
          v.emplace_back(in_place_type_t<path_factory::path_abs_move>, pt);
          hasCurrentPoint = true;
          closePt = pt;
%          n.close_point(pt);
        }
        for (; bezCount > 0; bezCount--) {
          auto cpt1 = ctr + rotCwFn({ pt1.x() * rad, pt1.y() * rad }, currTheta);
          auto cpt2 = ctr + rotCwFn({ pt2.x() * rad, pt2.y() * rad },
            currTheta);
          auto cpt3 = ctr + rotCwFn({ pt3.x() * rad, pt3.y() * rad },
            currTheta);
          currentPoint = cpt3;
          cpt1 = m.transform_point(cpt1 - origin) + origin;
          cpt2 = m.transform_point(cpt2 - origin) + origin;
          cpt3 = m.transform_point(cpt3 - origin) + origin;
%          n.curve_to(cpt1, cpt2, cpt3);
          v.emplace_back(in_place_type_t<path_factory::path_abs_curve>, cpt1,
            cpt2, cpt3);
          currTheta += theta;
        }
      }
      else if constexpr(std::is_same_v<T, path_factory::path_arc_counterclockwise>) {
      {
        auto ctr = item.center();
        auto rad = item.radius();
        auto ang1 = item.angle_1();
        auto ang2 = item.angle_2();
        while(ang2 > ang1) {
          ang2 -= twopi;
        }
        vector_2d pt0, pt1, pt2, pt3;
        int bezCount = 1;
        double theta = ang1 - ang2;
        double phi;
        while (theta >= halfpi) {
          theta /= 2.0;
          bezCount += bezCount;
        }
        phi = theta / 2.0;
        auto cosPhi = cos(phi);
        auto sinPhi = sin(phi);
        pt0.x(cosPhi);
        pt0.y(-sinPhi);
        pt3.x(pt0.x());
        pt3.y(-pt0.y());
        pt1.x((4.0 - cosPhi) / 3.0);
        pt1.y(-(((1.0 - cosPhi) * (3.0 - cosPhi)) / (3.0 * sinPhi)));
        pt2.x(pt1.x());
        pt2.y(-pt1.y());
        auto rotCwFn = [](const vector_2d& pt, double a) -> vector_2d {
          return { pt.x() * cos(a) + pt.y() * sin(a),
            -(pt.x() * -(sin(a)) + pt.y() * cos(a)) };
        };
        pt0 = rotCwFn(pt0, phi);
        pt1 = rotCwFn(pt1, phi);
        pt2 = rotCwFn(pt2, phi);
        pt3 = rotCwFn(pt3, phi);
        auto shflPt = pt3;
        pt3 = pt0;
        pt0 = shflPt;
        shflPt = pt2;
        pt2 = pt1;
        pt1 = shflPt;
        auto currTheta = ang1;
        const auto startPt =
        ctr + rotCwFn({ pt0.x() * rad, pt0.y() * rad }, currTheta);
        if (hasCurrentPoint) {
          currentPoint = startPt;
          auto pt = m.transform_point(currentPoint - origin) + origin;
%          n.line_to(pt);
          v.emplace_back(in_place_type_t<path_factory::path_abs_line>, pt);
        }
        else {
          currentPoint = startPt;
          auto pt = m.transform_point(currentPoint - origin) + origin;
%          n.current_point(pt);
          v.emplace_back(in_place_type_t<path_factory::path_abs_move>, pt);
          hasCurrentPoint = true;
          closePt = pt;
%          n.close_point(pt);
        }
        for (; bezCount > 0; bezCount--) {
          auto cpt1 = ctr + rotCwFn({ pt1.x() * rad, pt1.y() * rad },
            currTheta);
          auto cpt2 = ctr + rotCwFn({ pt2.x() * rad, pt2.y() * rad },
            currTheta);
          auto cpt3 = ctr + rotCwFn({ pt3.x() * rad, pt3.y() * rad },
            currTheta);
          currentPoint = cpt3;
          cpt1 = m.transform_point(cpt1 - origin) + origin;
          cpt2 = m.transform_point(cpt2 - origin) + origin;
          cpt3 = m.transform_point(cpt3 - origin) + origin;
%          n.curve_to(cpt1, cpt2, cpt3);
          v.emplace_back(in_place_type_t<path_factory::path_curve_to>, cpt1,
            cpt2, cpt3);
          currTheta -= theta;
        }
      }
      else if constexpr(std::is_same_v<T, path_factory::path_change_matrix>) {
        m = item.matrix();
      }
      else if constexpr(std::is_same_v<T, path_factory::path_change_origin>) {
        origin = item.origin();
      }
    }, val);
  }
\end{codeblock}

%\rSec1 [pathgeometries.strokerules] {Stroking path geometries}
%
%\pnum
%The following rules shall apply when a Stroking operation (\ref{surface.stroking}) is carried out on a path geometry.
%
%\begin{enumerate}
%\item If the path geometry only contains a degenerate path segment, then if the \tcode{line_cap} value is
%\end{enumerate}
%
%\begin{enumerate}
%  \item Except as otherwise specified in these rules, the start point and end point of a path segment shall be rendered as specified by the meaning of the surface's current \tcode{line_cap} value (\ref{linecap}).
%  
%  \item If the end point of a path segment \textit{a} is set as the current point and is then used as the start point of another path segment, \textit{b}, the point where \tcode{a}'s end point meets \tcode{b}'s start point shall be rendered as specified by the meaning of the surface's current \tcode{line_join} value (\ref{linejoin}).
%  
%  \item ***FIXME***
%\end{enumerate}
%
%\rSec1 [pathgeometries.fillrules] {Filling path geometries}
%
%\pnum
%***FIXME***

\addtocounter{SectionDepthBase}{1}
%%!TEX root = io2d.tex
\rSec0 [\iotwod.pathgeometries] {Path geometries}

\rSec1 [\iotwod.pathgeometries.overview] {Overview of path geometries}

\pnum
Path geometries are most easily formed using a \tcode{path_factory} object.

\pnum
They may also be formed by directly creating and manipulating a \tcode{vector<path_data_item>} object.

\pnum
A path geometry may contain degenerate path segments.

\pnum
There are special rules concerning the rendering of degenerate path segments. As such they shall be added to a path geometry when requested and shall not be removed from a path geometry when processing it.

\pnum
A \tcode{path} object is an immutable resource wrapper containing zero or more native path geometries (\ref{\iotwod.path}).

\rSec1 [\iotwod.pathgeometries.processing] {Processing path geometries}

\pnum
This section describes how to process a properly formed \tcode{vector<path_data_item>} object into native path geometries that are ready to be rendered.

\pnum
For purposes of exposition, it is assumed that there is an object for forming native path geometries with the following non-normative interface:

\begin{codeblock}
// \expos
class native_geometry_collection {
public:
  void current_point(const vector_2d& pt) noexcept;
  void close_point(const vector_2d& pt) noexcept;
  void line_to(const vector_2d& pt) noexcept;
  void curve_to(const vector_2d& control1, const vector_2d& control2,
    const vector_2d& endPt) noexcept;
  void close_path() noexcept;
};
// \expos
\end{codeblock}

\begin{itemdecl}
  void current_point(const vector_2d& pt) noexcept;
\end{itemdecl}
\begin{itemdescr}
	\pnum
	\effects
	Stores \tcode{pt} as the current point.

\end{itemdescr}

\begin{itemdecl}
  void close_point(const vector_2d& pt) noexcept;
\end{itemdecl}
\begin{itemdescr}
	\pnum
	\effects
	Stores \tcode{pt} as the close point.

\end{itemdescr}

\begin{itemdecl}
  void line_to(const vector_2d& pt) noexcept;
\end{itemdecl}
\begin{itemdescr}
	\pnum
	\effects
	Creates a line from the current point to \tcode{pt}.

\end{itemdescr}

\begin{itemdecl}
  void curve_to(const vector_2d& cpt1, const vector_2d& cpt2,
    const vector_2d& endPt) noexcept;
\end{itemdecl}
\begin{itemdescr}
	\pnum
	\effects
	Creates a cubic B\'ezier curve from the current point to \tcode{endPt} using \tcode{cpt1} as the first control point and \tcode{cpt2} as the second control point.

\end{itemdescr}

\begin{itemdecl}
  void close_path() noexcept;
\end{itemdecl}
\begin{itemdescr}
	\pnum
	\effects
	Creates a line from the current point to the close point. If this geometry is stroked, the point where the close point meets the line or curve that began the path geometry shall be rendered as a join in accordance with the current \tcode{line_join} value.

\end{itemdescr}

\pnum
\enternote
It is possible to work with a native path geometry that only supports rendering triangles by using the triangles to form lines and to approximate curves. This description of path geometry processing assumes the existence of a native path geometry that performs those actions where needed.
\exitnote

\pnum
The following code shows how to properly process \tcode{vector<path_data_item> p} and store the results into \tcode{native_geometry_collection n}:

\begin{codeblock}
const double pi =     3.1415926535897932384626433832795;
const double halfpi = 1.57079632679489661923132169163985;
const double twopi =  6.283185307179586476925286766559;
matrix_2d m;
vector_2d origin;
vector_2d currentPoint;
bool hasCurrentPoint = false;
vector_2d closePoint;

for (const auto& item : p) {
  switch(item.type()) {
  case path_data_type::move_to:
  {
    currentPoint = item.get<move_to>().to();
    auto pt = m.transform_point(currentPoint - origin) + origin;
    n.current_point(pt);
    hasCurrentPoint = true;
    closePoint = pt
    n.close_point(pt);
  } break;
  case path_data_type::line_to:
  {
    currentPoint = item.get<line_to>().to();
    auto pt = m.transform_point(currentPoint - origin) + origin;
    if (hasCurrentPoint) {
      n.line_to(pt);
    }
    else {
      n.current_point(pt);
      hasCurrentPoint = true;
      closePoint = pt;
      n.close_point(pt);
    }
  } break;
  case path_data_type::curve_to:
  {
    auto cd = item.get<curve_to>();
    auto pt1 = m.transform_point(cd.control_point_1() - origin) + origin;
    auto pt2 = m.transform_point(cd.control_point_2() - origin) + origin;
    auto pt3 = m.transform_point(cd.end_point() - origin) + origin;
    if (!hasCurrentPoint) {
      currentPoint = cd.control_point_1();
      n.current_point(pt1);
      hasCurrentPoint = true;
      closePoint = pt1;
      n.close_point(pt1);
    }
    n.curve_to(pt1, pt2, pt3);
    currentPoint = cd.end_point();
  } break;
  case path_data_type::new_sub_path:
  {
    hasCurrentPoint = false;
  } break;
  case path_data_type::close_path:
  {
    if (!hasCurrentPoint) {
      break;
    }
    n.close_path();
    n.current_point(closePoint);
    // Invert can error so use correct overload; here is the throw version.
    auto invM = matrix_2d{m}.invert();
    // Need the untransformed value for currentPoint.
    currentPoint = invM.transform_point(closePoint - origin) + origin;
  } break;
  case path_data_type::rel_move_to:
  {
    // If !hasCurrentPoint, error is io2d_error::no_current_point;
    currentPoint = item.get<rel_move_to>().to() + currentPoint;
    auto pt = m.transform_point(currentPoint - origin) + origin;
    n.current_point(pt);
    hasCurrentPoint = true;
    closePoint = pt    
    n.close_point(pt);
  } break;
  case path_data_type::rel_line_to:
  {
    // If !hasCurrentPoint, error is io2d_error::no_current_point;
    currentPoint = item.get<rel_line_to>().to() + currentPoint;
    auto pt = m.transform_point(currentPoint - origin) + origin;
    n.line_to(pt);
  } break;
  case path_data_type::rel_curve_to:
  {
    // If !hasCurrentPoint, error is io2d_error::no_current_point;
    auto cd = item.get<rel_curve_to>();
    auto pt1 = m.transform_point(cd.control_point_1() + currentPoint -
      origin) + origin;
    auto pt2 = m.transform_point(cd.control_point_2() + currentPoint -
      origin) + origin;
    auto pt3 = m.transform_point(cd.end_point() + currentPoint - origin) +
      origin;
    n.curve_to(pt1, pt2, pt3);
    currentPoint = cd.end_point() + currentPoint;
  } break;
  case path_data_type::arc:
  {
    auto ad = item.get<arc>();
    auto ctr = ad.center();
    auto rad = ad.radius();
    auto ang1 = ad.angle_1();
    auto ang2 = ad.angle_2();
    while(ang2 < ang1) {
      ang2 += twopi;
    }
    vector_2d pt0, pt1, pt2, pt3;
    int bezCount = 1;
    double theta = ang2 - ang1;
    double phi;
    while (theta >= halfpi) {
      theta /= 2.0;
      bezCount += bezCount;
    }
    phi = theta / 2.0;
    auto cosPhi = cos(phi);
    auto sinPhi = sin(phi);
    pt0.x(cosPhi);
    pt0.y(-sinPhi);
    pt3.x(pt0.x());
    pt3.y(-pt0.y());
    pt1.x((4.0 - cosPhi) / 3.0);
    pt1.y(-(((1.0 - cosPhi) * (3.0 - cosPhi)) / (3.0 * sinPhi)));
    pt2.x(pt1.x());
    pt2.y(-pt1.y());
    phi = -phi;
    auto rotCwFn = [](const vector_2d& pt, double a) -> vector_2d {
      return { pt.x() * cos(a) + pt.y() * sin(a),
        -(pt.x() * -(sin(a)) + pt.y() * cos(a)) };
    };
    pt0 = rotCwFn(pt0, phi);
    pt1 = rotCwFn(pt1, phi);
    pt2 = rotCwFn(pt2, phi);
    pt3 = rotCwFn(pt3, phi);
    
    auto currTheta = ang1;
    const auto startPt =
      ctr + rotCwFn({ pt0.x() * rad, pt0.y() * rad }, currTheta);
    if (hasCurrentPoint) {
      currentPoint = startPt;
      auto pt = m.transform_point(currentPoint - origin) + origin;
      n.line_to(pt);
    }
    else {
      currentPoint = startPt;
      auto pt = m.transform_point(currentPoint - origin) + origin;
      n.current_point(pt);
      hasCurrentPoint = true;
      closePt = pt;
      n.close_point(pt);
    }
    for (; bezCount > 0; bezCount--) {
      auto cpt1 = ctr + rotCwFn({ pt1.x() * rad, pt1.y() * rad }, currTheta);
      auto cpt2 = ctr + rotCwFn({ pt2.x() * rad, pt2.y() * rad }, currTheta);
      auto cpt3 = ctr + rotCwFn({ pt3.x() * rad, pt3.y() * rad }, currTheta);
      currentPoint = cpt3;
      cpt1 = m.transform_point(cpt1 - origin) + origin;
      cpt2 = m.transform_point(cpt2 - origin) + origin;
      cpt3 = m.transform_point(cpt3 - origin) + origin;
      n.curve_to(cpt1, cpt2, cpt3);
      currTheta += theta;
    }
  } break;
  case path_data_type::arc_negative:
  {
    auto ad = item.get<arc_negative>();
    auto ctr = ad.center();
    auto rad = ad.radius();
    auto ang1 = ad.angle_1();
    auto ang2 = ad.angle_2();
    while(ang2 > ang1) {
      ang2 -= twopi;
    }
    vector_2d pt0, pt1, pt2, pt3;
    int bezCount = 1;
    double theta = ang1 - ang2;
    double phi;
    while (theta >= halfpi) {
      theta /= 2.0;
      bezCount += bezCount;
    }
    phi = theta / 2.0;
    auto cosPhi = cos(phi);
    auto sinPhi = sin(phi);
    pt0.x(cosPhi);
    pt0.y(-sinPhi);
    pt3.x(pt0.x());
    pt3.y(-pt0.y());
    pt1.x((4.0 - cosPhi) / 3.0);
    pt1.y(-(((1.0 - cosPhi) * (3.0 - cosPhi)) / (3.0 * sinPhi)));
    pt2.x(pt1.x());
    pt2.y(-pt1.y());
    auto rotCwFn = [](const vector_2d& pt, double a) -> vector_2d {
      return { pt.x() * cos(a) + pt.y() * sin(a),
        -(pt.x() * -(sin(a)) + pt.y() * cos(a)) };
    };
    pt0 = rotCwFn(pt0, phi);
    pt1 = rotCwFn(pt1, phi);
    pt2 = rotCwFn(pt2, phi);
    pt3 = rotCwFn(pt3, phi);
    auto shflPt = pt3;
    pt3 = pt0;
    pt0 = shflPt;
    shflPt = pt2;
    pt2 = pt1;
    pt1 = shflPt;
    auto currTheta = ang1;
    const auto startPt =
      ctr + rotCwFn({ pt0.x() * rad, pt0.y() * rad }, currTheta);
    if (hasCurrentPoint) {
      currentPoint = startPt;
      auto pt = m.transform_point(currentPoint - origin) + origin;
      n.line_to(pt);
    }
    else {
      currentPoint = startPt;
      auto pt = m.transform_point(currentPoint - origin) + origin;
      n.current_point(pt);
      hasCurrentPoint = true;
      closePt = pt;
      n.close_point(pt);
    }
    for (; bezCount > 0; bezCount--) {
      auto cpt1 = ctr + rotCwFn({ pt1.x() * rad, pt1.y() * rad }, currTheta);
      auto cpt2 = ctr + rotCwFn({ pt2.x() * rad, pt2.y() * rad }, currTheta);
      auto cpt3 = ctr + rotCwFn({ pt3.x() * rad, pt3.y() * rad }, currTheta);
      currentPoint = cpt3;
      cpt1 = m.transform_point(cpt1 - origin) + origin;
      cpt2 = m.transform_point(cpt2 - origin) + origin;
      cpt3 = m.transform_point(cpt3 - origin) + origin;
      n.curve_to(cpt1, cpt2, cpt3);
      currTheta -= theta;
    }
  } break;
  case path_data_type::change_matrix:
  {
    m = item.get<change_matrix>().matrix();
  } break;
  case path_data_type::change_origin:
  {
    origin = item.get<change_origin>().origin();
  } break;
  }
}
\end{codeblock}

\rSec1 [\iotwod.pathgeometries.strokerules] {Stroking path geometries}

\pnum
The following rules shall apply when a \tcode{path} object is stroked by a call to \tcode{surface::stroke} or \tcode{surface::stroke_immediate}.

\begin{enumerate}
	\item Except as otherwise specified in these rules, the start point and end point of a path segment shall be rendered as specified by the meaning of the surface's current \tcode{line_cap} value (\ref{\iotwod.linecap}).
	
	\item If the end point of a path segment \textit{a} is set as the current point and is then used as the start point of another path segment, \textit{b}, the point where \tcode{a}'s end point meets \tcode{b}'s start point shall be rendered as specified by the meaning of the surface's current \tcode{line_join} value (\ref{\iotwod.linejoin}).
	
	\item ***FIXME***
\end{enumerate}

\rSec1 [\iotwod.pathgeometries.fillrules] {Filling path geometries}

\pnum
***FIXME***

%%!TEX root = io2d.tex
\rSec0 [pathdatatype] {Enum class \tcode{path_data_type}}

\rSec1 [pathdatatype.summary] {\tcode{path_data_type} Summary}

\pnum
The \tcode{path_data_type} enum class specifies the polymorphic type of a 
\tcode{path_data} object.
See Table~\ref{tab:pathdatatype.meanings} for the meaning of each
\tcode{path_data_type} enumerator.

\rSec1 [pathdatatype.synopsis] {\tcode{path_data_type} Synopsis}

\begin{codeblock}
namespace std { namespace experimental { namespace io2d { inline namespace v1 {
  enum class path_data_type {
    move_to,
    line_to,
    curve_to,
    new_sub_path,
    close_path,
    rel_move_to,
    rel_line_to,
    rel_curve_to,
    arc,
    arc_negative,
    change_matrix,
    change_origin
  };
} } } }
\end{codeblock}

\rSec1 [pathdatatype.enumerators] {\tcode{path_data_type} Enumerators}

\begin{libreqtab2}
 {\tcode{path_data_type} enumerator meanings}
 {tab:pathdatatype.meanings}
 \\ \topline
 \lhdr{Enumerator}
 & \rhdr{Meaning}
 \\ \capsep
 \endfirsthead
 \continuedcaption\\
 \hline
 \lhdr{Enumerator}
 & \rhdr{Meaning}
 \\ \capsep
 \endhead
 \tcode{move_to}
 & The object is of type \tcode{move_to}.
 \\
 \tcode{line_to}
 & The object is of type \tcode{line_to}.
 \\
 \tcode{curve_to}
 & The object is of type \tcode{curve_to}.
 \\
 \tcode{new_sub_path}
 & The object is of type \tcode{new_sub_path}.
 \\
 \tcode{close_path}
 & The object is of type \tcode{close_path}.
 \\
 \tcode{rel_move_to}
 & The object is of type \tcode{rel_move_to}.
 \\
 \tcode{rel_line_to}
 & The object is of type \tcode{rel_line_to}.
 \\
 \tcode{rel_curve_to}
 & The object is of type \tcode{rel_curve_to}.
 \\
 \tcode{arc}
 & The object is of type \tcode{arc}.
 \\
 \tcode{arc_negative}
 & The object is of type \tcode{arc_negative}.
 \\
 \tcode{change_matrix}
 & The object is of type \tcode{change_matrix}.
 \\
 \tcode{change_origin}
 & The object is of type \tcode{change_origin}.
 \\
\end{libreqtab2}

%%!TEX root = io2d.tex
\rSec0 [pathdataitem] {Class \tcode{path_data_item}}

\rSec1 [pathdataitem.synopsis] {\tcode{path_data_item} synopsis}

\begin{codeblock}
namespace std { namespace experimental { namespace io2d { inline namespace v1 {
  class path_data_item {
  public:
    // \ref{pathdataitem.cons}, construct/copy/move/destroy:
    path_data_item() noexcept;
    path_data_item(const path_data_item& other) noexcept;
    path_data_item& operator=(const path_data_item& other) noexcept;
    path_data_item(path_data_item&& other) noexcept;
    path_data_item& operator=(path_data_item&& other) noexcept;
    path_data_item(const arc& value) noexcept;
    path_data_item(const arc_negative& value) noexcept;
    path_data_item(const change_matrix& value) noexcept;
    path_data_item(const change_origin& value) noexcept;
    path_data_item(const close_path& value) noexcept;
    path_data_item(const curve_to& value) noexcept;
    path_data_item(const rel_curve_to& value) noexcept;
    path_data_item(const new_sub_path& value) noexcept;
    path_data_item(const line_to& value) noexcept;
    path_data_item(const move_to& value) noexcept;
    path_data_item(const rel_line_to& value) noexcept;
    path_data_item(const rel_move_to& value) noexcept;

    // \ref{pathdataitem.modifiers}, modifiers:
    void assign(const arc& value) noexcept;
    void assign(const arc_negative& value) noexcept;
    void assign(const change_matrix& value) noexcept;
    void assign(const change_origin& value) noexcept;
    void assign(const close_path& value) noexcept;
    void assign(const curve_to& value) noexcept;
    void assign(const rel_curve_to& value) noexcept;
    void assign(const new_sub_path& value) noexcept;
    void assign(const line_to& value) noexcept;
    void assign(const move_to& value) noexcept;
    void assign(const rel_line_to& value) noexcept;
    void assign(const rel_move_to& value) noexcept;

    // \ref{pathdataitem.observers}, observers:
    bool has_data() const noexcept;
    path_data_type type() const;
    path_data_type type(::std::error_code& ec) const noexcept;

    template <class T>
    T get() const;
    template <class T>
    T get(::std::error_code& ec) const noexcept;

    template <>
    arc get() const;
    template <>
    arc get(::std::error_code& ec) const noexcept;
    template <>
    arc_negative get() const;
    template <>
    arc_negative get(::std::error_code& ec) const noexcept;
    template <>
    change_matrix get() const;
    template <>
    change_matrix get(::std::error_code& ec) const noexcept;
    template <>
    change_origin get() const;
    template <>
    change_origin get(::std::error_code& ec) const noexcept;
    template <>
    close_path get() const;
    template <>
    close_path get(::std::error_code& ec) const noexcept;
    template <>
    curve_to get() const;
    template <>
    curve_to get(::std::error_code& ec) const noexcept;
    template <>
    rel_curve_to get() const;
    template <>
    rel_curve_to get(::std::error_code& ec) const noexcept;
    template <>
    new_sub_path get() const;
    template <>
    new_sub_path get(::std::error_code& ec) const noexcept;
    template <>
    line_to get() const;
    template <>
    line_to get(::std::error_code& ec) const noexcept;
    template <>
    move_to get() const;
    template <>
    move_to get(::std::error_code& ec) const noexcept;
    template <>
    rel_line_to get() const;
    template <>
    rel_line_to get(::std::error_code& ec) const noexcept;
    template <>
    rel_move_to get() const;
    template <>
    rel_move_to get(::std::error_code& ec) const noexcept;

  private:
    bool _Has_data;         // \expos
    union {
      struct {
        double centerX;
        double centerY;
        double radius;
        double angle1;
        double angle2;
      } arc;
      struct {
        double m00;
        double m01;
        double m10;
        double m11;
        double m20;
        double m21;
      } matrix;
      struct {
        double cpt1x;
        double cpt1y;
        double cpt2x;
        double cpt2y;
        double eptx;
        double epty;
      } curve;
      struct {
        double x;
        double y;
      } point;
    } _Data;               // \expos

    path_data_type _Type;  // \expos
  };
} } } }
\end{codeblock}

\rSec1 [pathdataitem.intro] {\tcode{path_data_item} Description}

\pnum
\indexlibrary{\idxcode{path_data_item}}
The class \tcode{path_data_item} describes an opaque container capable of storing and retrieving an object of a type derived from \tcode{path_data}.

\rSec1 [pathdataitem.cons] {\tcode{path_data_item} constructors and assignment operators}

\indexlibrary{\idxcode{path_data_item}!constructor}
\begin{itemdecl}
    path_data_item() noexcept;
\end{itemdecl}
\begin{itemdescr}
	\pnum
	\effects
	Constructs an object of type \tcode{path_data_item}.
	
	\pnum
	\postconditions
	\tcode{_Has_data == false}.

\end{itemdescr}

\indexlibrary{\idxcode{path_data_item}!constructor}
\begin{itemdecl}
    path_data_item(const arc& value) noexcept;
\end{itemdecl}
\begin{itemdescr}
	\pnum
	\effects
	Constructs an object of type \tcode{path_data_item}.
	
	\pnum
	\postconditions
	\tcode{_Has_data == true}.
	
	\pnum
	\tcode{_Type == path_data_type::arc}.
	
	\pnum
	\tcode{_Data.arc.centerX == value.center().x()}.
	
	\pnum
	\tcode{_Data.arc.centerY == value.center().y()}.
	
	\pnum
	\tcode{_Data.arc.radius == value.radius()}.
	
	\pnum
	\tcode{_Data.arc.angle1 == value.angle_1()}.
	
	\pnum
	\tcode{_Data.arc.angle2 == value.angle_2()}.
	
\end{itemdescr}

\indexlibrary{\idxcode{path_data_item}!constructor}
\begin{itemdecl}
    path_data_item(const arc_negative& value) noexcept;
\end{itemdecl}
\begin{itemdescr}
	\pnum
	\effects
	Constructs an object of type \tcode{path_data_item}.
	
	\pnum
	\postconditions
	\tcode{_Has_data == true}.
	
	\pnum
	\tcode{_Type == path_data_type::arc_negative}.
	
	\pnum
	\tcode{_Data.arc.centerX == value.center().x()}.
	
	\pnum
	\tcode{_Data.arc.centerY == value.center().y()}.
	
	\pnum
	\tcode{_Data.arc.radius == value.radius()}.
	
	\pnum
	\tcode{_Data.arc.angle1 == value.angle_1()}.
	
	\pnum
	\tcode{_Data.arc.angle2 == value.angle_2()}.

\end{itemdescr}

\indexlibrary{\idxcode{path_data_item}!constructor}
\begin{itemdecl}
    path_data_item(const change_matrix& value) noexcept;
\end{itemdecl}
\begin{itemdescr}
	\pnum
	\effects
	Constructs an object of type \tcode{path_data_item}.
	
	\pnum
	\postconditions
	\tcode{_Has_data == true}.
	
	\pnum
	\tcode{_Type == path_data_type::change_matrix}.
	
	\pnum
	\tcode{_Data.matrix.m00 == value.matrix().m00()}.
	
	\pnum
	\tcode{_Data.matrix.m01 == value.matrix().m01()}.
	
	\pnum
	\tcode{_Data.matrix.m10 == value.matrix().m10()}.
	
	\pnum
	\tcode{_Data.matrix.m11 == value.matrix().m11()}.
	
	\pnum
	\tcode{_Data.matrix.m20 == value.matrix().m20()}.
	
	\pnum
	\tcode{_Data.matrix.m21 == value.matrix().m21()}.
	
\end{itemdescr}

\indexlibrary{\idxcode{path_data_item}!constructor}
\begin{itemdecl}
    path_data_item(const change_origin& value) noexcept;
\end{itemdecl}
\begin{itemdescr}
	\pnum
	\effects
	Constructs an object of type \tcode{path_data_item}.
	
	\pnum
	\postconditions
	\tcode{_Has_data == true}.
	
	\pnum
	\tcode{_Type == path_data_type::change_origin}.

	\pnum
	\tcode{_Data.point.x == value.origin().x()}.	

	\pnum
	\tcode{_Data.point.y == value.origin().y()}.	
	
\end{itemdescr}

\indexlibrary{\idxcode{path_data_item}!constructor}
\begin{itemdecl}
    path_data_item(const close_path& value) noexcept;
\end{itemdecl}
\begin{itemdescr}
	\pnum
	\effects
	Constructs an object of type \tcode{path_data_item}.
	
	\pnum
	\postconditions
	\tcode{_Has_data == true}.
	
	\pnum
	\tcode{_Type == path_data_type::close_path}.
	
\end{itemdescr}

\indexlibrary{\idxcode{path_data_item}!constructor}
\begin{itemdecl}
    path_data_item(const curve_to& value) noexcept;
\end{itemdecl}
\begin{itemdescr}
	\pnum
	\effects
	Constructs an object of type \tcode{path_data_item}.
	
	\pnum
	\postconditions
	\tcode{_Has_data == true}.
	
	\pnum
	\tcode{_Type == path_data_type::curve_to}.
	
	\pnum
	\tcode{_Data.curve.cpt1x == value.control_point_1().x()}.
	
	\pnum
	\tcode{_Data.curve.cpt1y == value.control_point_1().y()}.
	
	\pnum
	\tcode{_Data.curve.cpt2x == value.control_point_2().x()}.
	
	\pnum
	\tcode{_Data.curve.cpt2y == value.control_point_2().y()}.
	
	\pnum
	\tcode{_Data.curve.eptx == value.end_point().x()}.
	
	\pnum
	\tcode{_Data.curve.epty == value.end_point().y()}.
	
\end{itemdescr}

\indexlibrary{\idxcode{path_data_item}!constructor}
\begin{itemdecl}
    path_data_item(const rel_curve_to& value) noexcept;
\end{itemdecl}
\begin{itemdescr}
	\pnum
	\effects
	Constructs an object of type \tcode{path_data_item}.
	
	\pnum
	\postconditions
	\tcode{_Has_data == true}.
	
	\pnum
	\tcode{_Type == path_data_type::rel_curve_to}.
	
	\pnum
	\tcode{_Data.curve.cpt1x == value.control_point_1().x()}.
	
	\pnum
	\tcode{_Data.curve.cpt1y == value.control_point_1().y()}.
	
	\pnum
	\tcode{_Data.curve.cpt2x == value.control_point_2().x()}.
	
	\pnum
	\tcode{_Data.curve.cpt2y == value.control_point_2().y()}.
	
	\pnum
	\tcode{_Data.curve.eptx == value.end_point().x()}.
	
	\pnum
	\tcode{_Data.curve.epty == value.end_point().y()}.

\end{itemdescr}

\indexlibrary{\idxcode{path_data_item}!constructor}
\begin{itemdecl}
    path_data_item(const new_sub_path& value) noexcept;
\end{itemdecl}
\begin{itemdescr}
	\pnum
	\effects
	Constructs an object of type \tcode{path_data_item}.
	
	\pnum
	\postconditions
	\tcode{_Has_data == true}.
	
	\pnum
	\tcode{_Type == path_data_type::new_sub_path}.
	
\end{itemdescr}

\indexlibrary{\idxcode{path_data_item}!constructor}
\begin{itemdecl}
    path_data_item(const line_to& value) noexcept;
\end{itemdecl}
\begin{itemdescr}
	\pnum
	\effects
	Constructs an object of type \tcode{path_data_item}.
	
	\pnum
	\postconditions
	\tcode{_Has_data == true}.
	
	\pnum
	\tcode{_Type == path_data_type::line_to}.

	\pnum
	\tcode{_Data.point.x == value.to().x()}.	

	\pnum
	\tcode{_Data.point.y == value.to().y()}.	
	
\end{itemdescr}

\indexlibrary{\idxcode{path_data_item}!constructor}
\begin{itemdecl}
    path_data_item(const move_to& value) noexcept;
\end{itemdecl}
\begin{itemdescr}
	\pnum
	\effects
	Constructs an object of type \tcode{path_data_item}.
	
	\pnum
	\postconditions
	\tcode{_Has_data == true}.
	
	\pnum
	\tcode{_Type == path_data_type::move_to}.

	\pnum
	\tcode{_Data.point.x == value.to().x()}.	

	\pnum
	\tcode{_Data.point.y == value.to().y()}.	
	
\end{itemdescr}

\indexlibrary{\idxcode{path_data_item}!constructor}
\begin{itemdecl}
    path_data_item(const rel_line_to& value) noexcept;
\end{itemdecl}
\begin{itemdescr}
	\pnum
	\effects
	Constructs an object of type \tcode{path_data_item}.
	
	\pnum
	\postconditions
	\tcode{_Has_data == true}.
	
	\pnum
	\tcode{_Type == path_data_type::rel_line_to}.

	\pnum
	\tcode{_Data.point.x == value.to().x()}.	

	\pnum
	\tcode{_Data.point.y == value.to().y()}.	
	
\end{itemdescr}

\indexlibrary{\idxcode{path_data_item}!constructor}
\begin{itemdecl}
    path_data_item(const rel_move_to& value) noexcept;
\end{itemdecl}
\begin{itemdescr}
	\pnum
	\effects
	Constructs an object of type \tcode{path_data_item}.
	
	\pnum
	\postconditions
	\tcode{_Has_data == true}.
	
	\pnum
	\tcode{_Type == path_data_type::rel_move_to}.

	\pnum
	\tcode{_Data.point.x == value.to().x()}.	

	\pnum
	\tcode{_Data.point.y == value.to().y()}.	
	
\end{itemdescr}

\rSec1 [pathdataitem.modifiers] {\tcode{path_data_item} modifiers}

\indexlibrary{\idxcode{path_data_item}!\idxcode{assign}}
\indexlibrary{\idxcode{assign}!\idxcode{path_data_item}}
\begin{itemdecl}
    void assign(const arc& value) noexcept;
\end{itemdecl}
\begin{itemdescr}
	\pnum
	\postconditions
	\tcode{_Has_data == true}.
	
	\pnum
	\tcode{_Type == path_data_type::arc}.
	
	\pnum
	\tcode{_Data.arc.centerX == value.center().x()}.
	
	\pnum
	\tcode{_Data.arc.centerY == value.center().y()}.
	
	\pnum
	\tcode{_Data.arc.radius == value.radius()}.
	
	\pnum
	\tcode{_Data.arc.angle1 == value.angle_1()}.
	
	\pnum
	\tcode{_Data.arc.angle2 == value.angle_2()}.
	
\end{itemdescr}

\indexlibrary{\idxcode{path_data_item}!\idxcode{assign}}
\indexlibrary{\idxcode{assign}!\idxcode{path_data_item}}
\begin{itemdecl}
    void assign(const arc_negative& value) noexcept;
\end{itemdecl}
\begin{itemdescr}
	\pnum
	\postconditions
	\tcode{_Has_data == true}.
	
	\pnum
	\tcode{_Type == path_data_type::arc_negative}.
	
	\pnum
	\tcode{_Data.arc.centerX == value.center().x()}.
	
	\pnum
	\tcode{_Data.arc.centerY == value.center().y()}.
	
	\pnum
	\tcode{_Data.arc.radius == value.radius()}.
	
	\pnum
	\tcode{_Data.arc.angle1 == value.angle_1()}.
	
	\pnum
	\tcode{_Data.arc.angle2 == value.angle_2()}.
	
\end{itemdescr}

\indexlibrary{\idxcode{path_data_item}!\idxcode{assign}}
\indexlibrary{\idxcode{assign}!\idxcode{path_data_item}}
\begin{itemdecl}
    void assign(const change_matrix& value) noexcept;
\end{itemdecl}
\begin{itemdescr}
	\pnum
	\postconditions
	\tcode{_Has_data == true}.
	
	\pnum
	\tcode{_Type == path_data_type::change_matrix}.
	
	\pnum
	\tcode{_Data.matrix.m00 == value.matrix().m00()}.
	
	\pnum
	\tcode{_Data.matrix.m01 == value.matrix().m01()}.
	
	\pnum
	\tcode{_Data.matrix.m10 == value.matrix().m10()}.
	
	\pnum
	\tcode{_Data.matrix.m11 == value.matrix().m11()}.
	
	\pnum
	\tcode{_Data.matrix.m20 == value.matrix().m20()}.
	
	\pnum
	\tcode{_Data.matrix.m21 == value.matrix().m21()}.
	
\end{itemdescr}

\indexlibrary{\idxcode{path_data_item}!\idxcode{assign}}
\indexlibrary{\idxcode{assign}!\idxcode{path_data_item}}
\begin{itemdecl}
    void assign(const change_origin& value) noexcept;
\end{itemdecl}
\begin{itemdescr}
	\pnum
	\postconditions
	\tcode{_Has_data == true}.
	
	\pnum
	\tcode{_Type == path_data_type::change_origin}.

	\pnum
	\tcode{_Data.point.x == value.origin().x()}.	

	\pnum
	\tcode{_Data.point.y == value.origin().y()}.	
	
\end{itemdescr}

\indexlibrary{\idxcode{path_data_item}!\idxcode{assign}}
\indexlibrary{\idxcode{assign}!\idxcode{path_data_item}}
\begin{itemdecl}
    void assign(const close_path& value) noexcept;
\end{itemdecl}
\begin{itemdescr}
	\pnum
	\postconditions
	\tcode{_Has_data == true}.
	
	\pnum
	\tcode{_Type == path_data_type::close_path}.

\end{itemdescr}

\indexlibrary{\idxcode{path_data_item}!\idxcode{assign}}
\indexlibrary{\idxcode{assign}!\idxcode{path_data_item}}
\begin{itemdecl}
    void assign(const curve_to& value) noexcept;
\end{itemdecl}
\begin{itemdescr}
	\pnum
	\postconditions
	\tcode{_Has_data == true}.
	
	\pnum
	\tcode{_Type == path_data_type::curve_to}.
	
	\pnum
	\tcode{_Data.curve.cpt1x == value.control_point_1().x()}.
	
	\pnum
	\tcode{_Data.curve.cpt1y == value.control_point_1().y()}.
	
	\pnum
	\tcode{_Data.curve.cpt2x == value.control_point_2().x()}.
	
	\pnum
	\tcode{_Data.curve.cpt2y == value.control_point_2().y()}.
	
	\pnum
	\tcode{_Data.curve.eptx == value.end_point().x()}.
	
	\pnum
	\tcode{_Data.curve.epty == value.end_point().y()}.
	
\end{itemdescr}

\indexlibrary{\idxcode{path_data_item}!\idxcode{assign}}
\indexlibrary{\idxcode{assign}!\idxcode{path_data_item}}
\begin{itemdecl}
    void assign(const rel_curve_to& value) noexcept;
\end{itemdecl}
\begin{itemdescr}
	\pnum
	\postconditions
	\tcode{_Has_data == true}.
	
	\pnum
	\tcode{_Type == path_data_type::rel_curve_to}.
	
	\pnum
	\tcode{_Data.curve.cpt1x == value.control_point_1().x()}.
	
	\pnum
	\tcode{_Data.curve.cpt1y == value.control_point_1().y()}.
	
	\pnum
	\tcode{_Data.curve.cpt2x == value.control_point_2().x()}.
	
	\pnum
	\tcode{_Data.curve.cpt2y == value.control_point_2().y()}.
	
	\pnum
	\tcode{_Data.curve.eptx == value.end_point().x()}.
	
	\pnum
	\tcode{_Data.curve.epty == value.end_point().y()}.
	
\end{itemdescr}

\indexlibrary{\idxcode{path_data_item}!\idxcode{assign}}
\indexlibrary{\idxcode{assign}!\idxcode{path_data_item}}
\begin{itemdecl}
    void assign(const new_sub_path& value) noexcept;
\end{itemdecl}
\begin{itemdescr}
	\pnum
	\postconditions
	\tcode{_Has_data == true}.
	
	\pnum
	\tcode{_Type == path_data_type::new_sub_path}.
	
\end{itemdescr}

\indexlibrary{\idxcode{path_data_item}!\idxcode{assign}}
\indexlibrary{\idxcode{assign}!\idxcode{path_data_item}}
\begin{itemdecl}
    void assign(const line_to& value) noexcept;
\end{itemdecl}
\begin{itemdescr}
	\pnum
	\postconditions
	\tcode{_Has_data == true}.
	
	\pnum
	\tcode{_Type == path_data_type::line_to}.

	\pnum
	\tcode{_Data.point.x == value.to().x()}.	

	\pnum
	\tcode{_Data.point.y == value.to().y()}.	
	
\end{itemdescr}

\indexlibrary{\idxcode{path_data_item}!\idxcode{assign}}
\indexlibrary{\idxcode{assign}!\idxcode{path_data_item}}
\begin{itemdecl}
    void assign(const move_to& value) noexcept;
\end{itemdecl}
\begin{itemdescr}
	\pnum
	\postconditions
	\tcode{_Has_data == true}.
	
	\pnum
	\tcode{_Type == path_data_type::move_to}.

	\pnum
	\tcode{_Data.point.x == value.to().x()}.	

	\pnum
	\tcode{_Data.point.y == value.to().y()}.	
	
\end{itemdescr}

\indexlibrary{\idxcode{path_data_item}!\idxcode{assign}}
\indexlibrary{\idxcode{assign}!\idxcode{path_data_item}}
\begin{itemdecl}
    void assign(const rel_line_to& value) noexcept;
\end{itemdecl}
\begin{itemdescr}
	\pnum
	\postconditions
	\tcode{_Has_data == true}.
	
	\pnum
	\tcode{_Type == path_data_type::rel_line_to}.

	\pnum
	\tcode{_Data.point.x == value.to().x()}.	

	\pnum
	\tcode{_Data.point.y == value.to().y()}.	
	
\end{itemdescr}

\indexlibrary{\idxcode{path_data_item}!\idxcode{assign}}
\indexlibrary{\idxcode{assign}!\idxcode{path_data_item}}
\begin{itemdecl}
    void assign(const rel_move_to& value) noexcept;
\end{itemdecl}
\begin{itemdescr}
	\pnum
	\postconditions
	\tcode{_Has_data == true}.
	
	\pnum
	\tcode{_Type == path_data_type::rel_move_to}.

	\pnum
	\tcode{_Data.point.x == value.to().x()}.	

	\pnum
	\tcode{_Data.point.y == value.to().y()}.	
	
\end{itemdescr}

\rSec1 [pathdataitem.observers] {\tcode{path_data_item} observers}

\indexlibrary{\idxcode{path_data_item}!\idxcode{has_data}}
\indexlibrary{\idxcode{has_data}!\idxcode{path_data_item}}
\begin{itemdecl}
    bool has_data() const noexcept;
\end{itemdecl}
\begin{itemdescr}
	\pnum
	\returns
	\tcode{_Has_data}.
	
\end{itemdescr}

\indexlibrary{\idxcode{path_data_item}!\idxcode{type}}
\indexlibrary{\idxcode{type}!\idxcode{path_data_item}}
\begin{itemdecl}
    path_data_type type() const;
    path_data_type type(::std::error_code& ec) const noexcept;
\end{itemdecl}
\begin{itemdescr}
	\pnum
	\preconditions
	\tcode{_Has_data == true}.
	
	\pnum
	\returns
	\tcode{_Type}.
	
	\pnum
	\throws
	As specified in Error reporting (\ref{\iotwod.err.report}).
	
	\pnum
	\errors
	\tcode{errc::operation_not_permitted} if the preconditions are violated.

\end{itemdescr}

\indexlibrary{\idxcode{path_data_item}!\idxcode{get}}
\indexlibrary{\idxcode{get}!\idxcode{path_data_item}}
\begin{itemdecl}
    template <>
    arc get() const;
    template <>
    arc get(::std::error_code& ec) const noexcept;
\end{itemdecl}
\begin{itemdescr}
	\pnum
	\preconditions
	\tcode{_Has_data == true}.
	
	\pnum
	\tcode{_Type == path_data_type::arc}.
	
	\pnum
	\returns
	\tcode{arc\{vector_2d\{ _Data.arc.centerX, _Data.arc.centerY \},
	  _Data.arc.radius, _Data.arc.angle1, _Data.arc.angle2 \}}.
	
	\pnum
	\throws
	As specified in Error reporting (\ref{\iotwod.err.report}).
	
	\pnum
	\remarks
	If a non-throwing error occurs, this function returns a default constructed object of its specified return type.
	
	\pnum
	\errors
	\tcode{errc::operation_not_permitted} if \tcode{!_Has_data}.
	
	\pnum
	\tcode{errc::invalid_argument} if \tcode{_Type != path_data_type::arc}.

\end{itemdescr}

\indexlibrary{\idxcode{path_data_item}!\idxcode{get}}
\indexlibrary{\idxcode{get}!\idxcode{path_data_item}}
\begin{itemdecl}
    template <>
    arc_negative get() const;
    template <>
    arc_negative get(::std::error_code& ec) const noexcept;
\end{itemdecl}
\begin{itemdescr}
	\pnum
	\preconditions
	\tcode{_Has_data == true}.
	
	\pnum
	\tcode{_Type == path_data_type::arc_negative}.
	
	\pnum
	\returns
	\tcode{arc_negative\{vector_2d\{ _Data.arc.centerX, _Data.arc.centerY \},
	  _Data.arc.radius, _Data.arc.angle1, _Data.arc.angle2 \}}.
	
	\pnum
	\throws
	As specified in Error reporting (\ref{\iotwod.err.report}).
	
	\pnum
	\remarks
	If a non-throwing error occurs, this function returns a default constructed object of its specified return type.
	
	\pnum
	\errors
	\tcode{errc::operation_not_permitted} if \tcode{!_Has_data}.
	
	\pnum
	\tcode{errc::invalid_argument} if \tcode{_Type != path_data_type::arc_negative}.

\end{itemdescr}

\indexlibrary{\idxcode{path_data_item}!\idxcode{get}}
\indexlibrary{\idxcode{get}!\idxcode{path_data_item}}
\begin{itemdecl}
    template <>
    change_matrix get() const;
    template <>
    change_matrix get(::std::error_code& ec) const noexcept;
\end{itemdecl}
\begin{itemdescr}
	\pnum
	\preconditions
	\tcode{_Has_data == true}.
	
	\pnum
	\tcode{_Type == path_data_type::change_matrix}.
	
	\pnum
	\returns
	\tcode{change_matrix\{ matrix_2d\{ _Data.matrix.m00, _Data.matrix.m01, _Data.matrix.m10, _Data.matrix.m11, _Data.matrix.m20, _Data.matrix.m21 \} \}}.
	
	\pnum
	\throws
	As specified in Error reporting (\ref{\iotwod.err.report}).
	
	\pnum
	\remarks
	If a non-throwing error occurs, this function returns a default constructed object of its specified return type.
	
	\pnum
	\errors
	\tcode{errc::operation_not_permitted} if \tcode{!_Has_data}.
	
	\pnum
	\tcode{errc::invalid_argument} if \tcode{_Type != path_data_type::change_matrix}.

\end{itemdescr}

\indexlibrary{\idxcode{path_data_item}!\idxcode{get}}
\indexlibrary{\idxcode{get}!\idxcode{path_data_item}}
\begin{itemdecl}
    template <>
    change_origin get() const;
    template <>
    change_origin get(::std::error_code& ec) const noexcept;
\end{itemdecl}
\begin{itemdescr}
	\pnum
	\preconditions
	\tcode{_Has_data == true}.
	
	\pnum
	\tcode{_Type == path_data_type::change_origin}.
	
	\pnum
	\returns
	\tcode{change_origin\{ vector_2d\{ _Data.point.x, _Data.point.y \} \}}.
	
	\pnum
	\throws
	As specified in Error reporting (\ref{\iotwod.err.report}).
	
	\pnum
	\remarks
	If a non-throwing error occurs, this function returns a default constructed object of its specified return type.
	
	\pnum
	\errors
	\tcode{errc::operation_not_permitted} if \tcode{!_Has_data}.
	
	\pnum
	\tcode{errc::invalid_argument} if \tcode{_Type != path_data_type::change_origin}.

\end{itemdescr}

\indexlibrary{\idxcode{path_data_item}!\idxcode{get}}
\indexlibrary{\idxcode{get}!\idxcode{path_data_item}}
\begin{itemdecl}
    template <>
    close_path get() const;
    template <>
    close_path get(::std::error_code& ec) const noexcept;
\end{itemdecl}
\begin{itemdescr}
	\pnum
	\preconditions
	\tcode{_Has_data == true}.
	
	\pnum
	\tcode{_Type == path_data_type::close_path}.
	
	\pnum
	\returns
	\tcode{close_path\{ \}}.
	
	\pnum
	\throws
	As specified in Error reporting (\ref{\iotwod.err.report}).
	
	\pnum
	\remarks
	If a non-throwing error occurs, this function returns a default constructed object of its specified return type.
	
	\pnum
	\errors
	\tcode{errc::operation_not_permitted} if \tcode{!_Has_data}.
	
	\pnum
	\tcode{errc::invalid_argument} if \tcode{_Type != path_data_type::close_path}.

\end{itemdescr}

\indexlibrary{\idxcode{path_data_item}!\idxcode{get}}
\indexlibrary{\idxcode{get}!\idxcode{path_data_item}}
\begin{itemdecl}
    template <>
    curve_to get() const;
    template <>
    curve_to get(::std::error_code& ec) const noexcept;
\end{itemdecl}
\begin{itemdescr}
	\pnum
	\preconditions
	\tcode{_Has_data == true}.
	
	\pnum
	\tcode{_Type == path_data_type::curve_to}.
	
	\pnum
	\returns
	\tcode{curve_to\{ vector_2d\{ _Data.curve.cpt1x, _Data.curve.cpt1y \}, vector_2d\{ _Data.curve.cpt2x, _Data.curve.cpt2y \}, vector_2d\{ _Data.curve.eptx, _Data.curve.epty \} \}}.
	
	\pnum
	\throws
	As specified in Error reporting (\ref{\iotwod.err.report}).
	
	\pnum
	\remarks
	If a non-throwing error occurs, this function returns a default constructed object of its specified return type.
	
	\pnum
	\errors
	\tcode{errc::operation_not_permitted} if \tcode{!_Has_data}.
	
	\pnum
	\tcode{errc::invalid_argument} if \tcode{_Type != path_data_type::curve_to}.

\end{itemdescr}

\indexlibrary{\idxcode{path_data_item}!\idxcode{get}}
\indexlibrary{\idxcode{get}!\idxcode{path_data_item}}
\begin{itemdecl}
    template <>
    rel_curve_to get() const;
    template <>
    rel_curve_to get(::std::error_code& ec) const noexcept;
\end{itemdecl}
\begin{itemdescr}
	\pnum
	\preconditions
	\tcode{_Has_data == true}.
	
	\pnum
	\tcode{_Type == path_data_type::rel_curve_to}.
	
	\pnum
	\returns
	\tcode{rel_curve_to\{ vector_2d\{ _Data.curve.cpt1x, _Data.curve.cpt1y \}, vector_2d\{ _Data.curve.cpt2x, _Data.curve.cpt2y \}, vector_2d\{ _Data.curve.eptx, _Data.curve.epty \} \}}.
	
	\pnum
	\throws
	As specified in Error reporting (\ref{\iotwod.err.report}).
	
	\pnum
	\remarks
	If a non-throwing error occurs, this function returns a default constructed object of its specified return type.
	
	\pnum
	\errors
	\tcode{errc::operation_not_permitted} if \tcode{!_Has_data}.
	
	\pnum
	\tcode{errc::invalid_argument} if \tcode{_Type != path_data_type::rel_curve_to}.

\end{itemdescr}

\indexlibrary{\idxcode{path_data_item}!\idxcode{get}}
\indexlibrary{\idxcode{get}!\idxcode{path_data_item}}
\begin{itemdecl}
    template <>
    new_sub_path get() const;
    template <>
    new_sub_path get(::std::error_code& ec) const noexcept;
\end{itemdecl}
\begin{itemdescr}
	\pnum
	\preconditions
	\tcode{_Has_data == true}.
	
	\pnum
	\tcode{_Type == path_data_type::new_sub_path}.
	
	\pnum
	\returns
	\tcode{new_sub_path\{ \}}.
	
	\pnum
	\throws
	As specified in Error reporting (\ref{\iotwod.err.report}).
	
	\pnum
	\remarks
	If a non-throwing error occurs, this function returns a default constructed object of its specified return type.
	
	\pnum
	\errors
	\tcode{errc::operation_not_permitted} if \tcode{!_Has_data}.
	
	\pnum
	\tcode{errc::invalid_argument} if \tcode{_Type != path_data_type::new_sub_path}.

\end{itemdescr}

\indexlibrary{\idxcode{path_data_item}!\idxcode{get}}
\indexlibrary{\idxcode{get}!\idxcode{path_data_item}}
\begin{itemdecl}
    template <>
    line_to get() const;
    template <>
    line_to get(::std::error_code& ec) const noexcept;
\end{itemdecl}
\begin{itemdescr}
	\pnum
	\preconditions
	\tcode{_Has_data == true}.
	
	\pnum
	\tcode{_Type == path_data_type::line_to}.
	
	\pnum
	\returns
	\tcode{line_to\{ vector_2d\{ _Data.point.x, _Data.point.y \} \}}.
	
	\pnum
	\throws
	As specified in Error reporting (\ref{\iotwod.err.report}).
	
	\pnum
	\remarks
	If a non-throwing error occurs, this function returns a default constructed object of its specified return type.
	
	\pnum
	\errors
	\tcode{errc::operation_not_permitted} if \tcode{!_Has_data}.
	
	\pnum
	\tcode{errc::invalid_argument} if \tcode{_Type != path_data_type::line_to}.

\end{itemdescr}

\indexlibrary{\idxcode{path_data_item}!\idxcode{get}}
\indexlibrary{\idxcode{get}!\idxcode{path_data_item}}
\begin{itemdecl}
    template <>
    move_to get() const;
    template <>
    move_to get(::std::error_code& ec) const noexcept;
\end{itemdecl}
\begin{itemdescr}
	\pnum
	\preconditions
	\tcode{_Has_data == true}.
	
	\pnum
	\tcode{_Type == path_data_type::move_to}.
	
	\pnum
	\returns
	\tcode{move_to\{ vector_2d\{ _Data.point.x, _Data.point.y \} \}}.
	
	\pnum
	\throws
	As specified in Error reporting (\ref{\iotwod.err.report}).
	
	\pnum
	\remarks
	If a non-throwing error occurs, this function returns a default constructed object of its specified return type.
	
	\pnum
	\errors
	\tcode{errc::operation_not_permitted} if \tcode{!_Has_data}.
	
	\pnum
	\tcode{errc::invalid_argument} if \tcode{_Type != path_data_type::move_to}.

\end{itemdescr}

\indexlibrary{\idxcode{path_data_item}!\idxcode{get}}
\indexlibrary{\idxcode{get}!\idxcode{path_data_item}}
\begin{itemdecl}
    template <>
    rel_line_to get() const;
    template <>
    rel_line_to get(::std::error_code& ec) const noexcept;
\end{itemdecl}
\begin{itemdescr}
	\pnum
	\preconditions
	\tcode{_Has_data == true}.
	
	\pnum
	\tcode{_Type == path_data_type::rel_line_to}.
	
	\pnum
	\returns
	\tcode{rel_line_to\{ vector_2d\{ _Data.point.x, _Data.point.y \} \}}.
	
	\pnum
	\throws
	As specified in Error reporting (\ref{\iotwod.err.report}).
	
	\pnum
	\remarks
	If a non-throwing error occurs, this function returns a default constructed object of its specified return type.
	
	\pnum
	\errors
	\tcode{errc::operation_not_permitted} if \tcode{!_Has_data}.
	
	\pnum
	\tcode{errc::invalid_argument} if \tcode{_Type != path_data_type::rel_line_to}.

\end{itemdescr}

\indexlibrary{\idxcode{path_data_item}!\idxcode{get}}
\indexlibrary{\idxcode{get}!\idxcode{path_data_item}}
\begin{itemdecl}
    template <>
    rel_move_to get() const;
    template <>
    rel_move_to get(::std::error_code& ec) const noexcept;
\end{itemdecl}
\begin{itemdescr}
	\pnum
	\preconditions
	\tcode{_Has_data == true}.
	
	\pnum
	\tcode{_Type == path_data_type::rel_move_to}.
	
	\pnum
	\returns
	\tcode{rel_move_to\{ vector_2d\{ _Data.point.x, _Data.point.y \} \}}.
	
	\pnum
	\throws
	As specified in Error reporting (\ref{\iotwod.err.report}).
	
	\pnum
	\remarks
	If a non-throwing error occurs, this function returns a default constructed object of its specified return type.
	
	\pnum
	\errors
	\tcode{errc::operation_not_permitted} if \tcode{!_Has_data}.
	
	\pnum
	\tcode{errc::invalid_argument} if \tcode{_Type != path_data_type::rel_move_to}.

\end{itemdescr}

%!TEX root = io2d.tex
\rSec0 [path] {Class \tcode{path}}
%%%%% Rename path to path_group so that a path group contains paths rather than path geometries. Rework all working accordingly and eliminate "sub path" since it is now just "path".
\pnum
\indexlibrary{\idxcode{path}}
The class \tcode{path} contains a path geometry graphics resource that is usable with a \tcode{surface}-derived object.

\pnum
A \tcode{path} object is constructed from the path geometry collection data of a \tcode{path_factory} object. The path geometries of its path geometry graphics resource are immutable, however its path geometry graphics resource can be changed using copy assignment or move assignment.

\pnum
An \tcode{path} object can be default constructed. Default construction of a \tcode{path} object results in a \tcode{path} object which has a path geometry graphics resource that contains no path geometries.

\pnum
When a \tcode{path} object is set on a \tcode{surface} object using 
\tcode{surface::path}, the geometric paths represented by it can be 
stroked or filled.

%\pnum
%A \tcode{path} object shall be usable with any \tcode{surface} or \tcode{surface}-derived object.
%
\rSec1 [path.synopsis] {\tcode{path} synopsis}

\begin{codeblock}
namespace std { namespace experimental { namespace io2d { inline namespace v1 {
  class path {
    public:
    // \ref{path.cons}, construct/copy/destroy:
    explicit path(const path_factory& pb);
    path(const path_factory& pb, error_code& ec) noexcept;
  };
} } } }
\end{codeblock}

\rSec1 [path.cons] {\tcode{path} constructors and assignment operators}

\indexlibrary{\idxcode{path}!constructor}
\begin{itemdecl}
    explicit path(const path_factory& pb);
    path(const path_factory& pb, error_code& ec) noexcept;
\end{itemdecl}
\begin{itemdescr}
	\pnum
	\effects
	Constructs an object of class \tcode{path}. Implementations shall create a path geometry graphics resource from the path geometries contained in \tcode{pb.data_ref()} as if they followed the procedure set forth in \ref{pathgeometries.processing}.

	\pnum
	\throws
	As specified in Error reporting (\ref{\iotwod.err.report}).

	\pnum
	\remarks
	It is unspecified whether a \tcode{path} object shall require further processing when it is passed as an argument to a \tcode{surface} or \tcode{surface}-derived object.
	
	\pnum
	Implementations should avoid or minimize the need for further processing of a \tcode{path} object after it has been constructed.

	\pnum
	\errors
	\tcode{errc::not_enough_memory} if there was a failure to allocate memory.
	
%	\pnum
%	\tcode{io2d_error::no_current_point} if, when processing the path geometries, an operation was encountered which required a current point and the current path geometry had no current point.
%	
%	\pnum
%	\tcode{io2d_error::invalid_matrix} if, when processing the path geometries, an operation was encountered which required the current transformation matrix to be invertible and the matrix was not invertible.
	
\end{itemdescr}

%!TEX root = io2d.tex
\rSec0 [pathfactory] {Class \tcode{path_factory}}

\rSec1 [pathfactory.synopsis] {\tcode{path_factory} synopsis}

\begin{codeblock}
namespace std { namespace experimental { namespace io2d { inline namespace v1 {
  class path_factory {
    // \ref{pathfactory.cons}, construct/copy/destroy:
    path_factory() noexcept;
    path_factory(const path_factory& other);
    path_factory& operator=(const path_factory& other);
    path_factory(path_factory&& other) noexcept;
    path_factory& operator=(path_factory&& other) noexcept;
    
    // \ref{pathfactory.modifiers}, modifiers:
    void append(const path_factory& p);
    void append(const path_factory& p, error_code& ec) noexcept;
    void append(const vector<path_data_item>& p);
    void append(const vector<path_data_item>& p, error_code& ec) noexcept;
    void new_sub_path();
    void new_sub_path(error_code& ec) noexcept;
    void close_path();
    void close_path(error_code& ec) noexcept;
    void arc(const vector_2d& center, double radius, double angle1,
      double angle2);
    void arc(const vector_2d& center, double radius, double angle1,
      double angle2, error_code& ec) noexcept;
    void arc_negative(const vector_2d& center, double radius, double angle1,
      double angle2);
    void arc_negative(const vector_2d& center, double radius, double angle1,
      double angle2, error_code& ec) noexcept;
    void curve_to(const vector_2d& pt0, const vector_2d& pt1,
      const vector_2d& pt2);
    void curve_to(const vector_2d& pt0, const vector_2d& pt1,
      const vector_2d& pt2, error_code& ec) noexcept;
    void line_to(const vector_2d& pt);
    void line_to(const vector_2d& pt, error_code& ec) noexcept;
    void move_to(const vector_2d& pt);
    void move_to(const vector_2d& pt, error_code& ec) noexcept;
    void rectangle(const std::experimental::io2d::rectangle& r);
    void rectangle(const std::experimental::io2d::rectangle& r,
      error_code& ec) noexcept;
    void rel_curve_to(const vector_2d& dpt0, const vector_2d& dpt1,
      const vector_2d& dpt2);
    void rel_curve_to(const vector_2d& dpt0, const vector_2d& dpt1,
      const vector_2d& dpt2, error_code& ec) noexcept;
    void rel_line_to(const vector_2d& dpt);
    void rel_line_to(const vector_2d& dpt, error_code& ec) noexcept;
    void rel_move_to(const vector_2d& dpt);
    void rel_move_to(const vector_2d& dpt, error_code& ec) noexcept;
    void transform_matrix(const matrix_2d& m);
    void transform_matrix(const matrix_2d& m, error_code& ec) noexcept;
    void origin(const vector_2d& pt);
    void origin(const vector_2d& pt, error_code& ec) noexcept;
    void clear() noexcept;
    
    // \ref{pathfactory.observers}, observers:
    std::experimental::io2d::rectangle path_extents() const;
    std::experimental::io2d::rectangle path_extents(error_code& ec) const noexcept;
    bool has_current_point() const noexcept;
    vector_2d current_point() const;
    vector_2d current_point(error_code& ec) const noexcept;
    matrix_2d transform_matrix() const noexcept;
    vector_2d origin() const noexcept;
    vector<path_data_item> data() const;
    vector<path_data_item> data(error_code& ec) const noexcept;
    path_data_item data_item(unsigned int index) const;
    path_data_item data_item(unsigned int index, error_code& ec) const noexcept;
    const vector<path_data_item>& data_ref() const noexcept;

  private:
    vector<path_data_item> _Data;  // \expos
    bool _Has_current_point;       // \expos
    vector_2d _Current_point;      // \expos
    vector_2d _Last_move_to_point; // \expos
    matrix_2d _Transform_matrix;   // \expos
    vector_2d _Origin;             // \expos
  };
} } } }
\end{codeblock}

\rSec1 [pathfactory.intro] {\tcode{path_factory} Description}

\pnum
\indexlibrary{\idxcode{path_factory}}
The \tcode{path_factory} class is a factory class used in creating path geometry collection data from which \tcode{path} objects are created.

\rSec1 [pathfactory.cons] {\tcode{path_factory} constructors and 
assignment operators}

\indexlibrary{\idxcode{path_factory}!constructor}
\begin{itemdecl}
    path_factory();
\end{itemdecl}
\begin{itemdescr}
	\pnum
	\effects
	Constructs an object of type \tcode{path_factory}.
	
	\pnum
	\postconditions
	\tcode{_Data.empty() == true}.
	
	\pnum
	\tcode{_Has_current_point == false}.
	
	\pnum
	\tcode{_Transform_matrix == matrix_2d::init_identity()}.
	
	\pnum
	\tcode{_Origin == vector_2d{ }}.
	
\end{itemdescr}

\rSec1 [pathfactory.modifiers] {\tcode{path_factory} modifiers}

\indexlibrary{\idxcode{path_factory}!\idxcode{}}
\indexlibrary{\idxcode{}!\idxcode{path_factory}}
\begin{itemdecl}
    void append(const path_factory& p);
    void append(const path_factory& p, error_code& ec) noexcept;
\end{itemdecl}
\begin{itemdescr}
	\pnum
	\postconditions
	
\end{itemdescr}

\indexlibrary{\idxcode{path_factory}!\idxcode{}}
\indexlibrary{\idxcode{}!\idxcode{path_factory}}
\begin{itemdecl}
    void append(const vector<path_data_item>& p);
    void append(const vector<path_data_item>& p, error_code& ec) noexcept;
\end{itemdecl}
\begin{itemdescr}
	\pnum
	\postconditions
	
\end{itemdescr}

\indexlibrary{\idxcode{path_factory}!\idxcode{new_sub_path}}
\indexlibrary{\idxcode{new_sub_path}!\idxcode{path_factory}}
\begin{itemdecl}
    void new_sub_path();
    void new_sub_path(error_code& ec) noexcept;
\end{itemdecl}
\begin{itemdescr}
	\pnum
	\effects
	\tcode{_Data.emplace_back(std::experimental::io2d::new_sub_path())}.
	
	\pnum
	\tcode{_Has_current_point = false}.
	
	\pnum
	\throws
	As specified in Error reporting (\ref{\iotwod.err.report}).

	\pnum
	\remarks
	In the event of an error, the object shall not be modified.

	\pnum
	\errors
	\tcode{errc::not_enough_memory} if the attempt to add the \tcode{path_data_item} failed.
	
\end{itemdescr}

\indexlibrary{\idxcode{path_factory}!\idxcode{close_path}}
\indexlibrary{\idxcode{close_path}!\idxcode{path_factory}}
\begin{itemdecl}
    void close_path();
    void close_path(error_code& ec) noexcept;
\end{itemdecl}
\begin{itemdescr}
	\pnum
	\effects
	If \tcode{_Has_current_point == true}:
	\begin{itemize}
	\item \tcode{_Data.emplace_back(std::experimental::io2d::close_path())}.
	
	\item \tcode{_Current_point = _Last_move_to_point}.
	\end{itemize}
	
	\pnum
	\throws
	As specified in Error reporting (\ref{\iotwod.err.report}).

	\pnum
	\remarks
	In the event of an error, the object shall not be modified.

	\pnum
	\errors
	\tcode{errc::not_enough_memory} if the attempt to add the \tcode{path_data_item} failed.
	
\end{itemdescr}

\indexlibrary{\idxcode{path_factory}!\idxcode{arc}}
\indexlibrary{\idxcode{arc}!\idxcode{path_factory}}
\begin{itemdecl}
    void arc(const vector_2d& center, double radius, double angle1,
      double angle2);
    void arc(const vector_2d& center, double radius, double angle1,
      double angle2, error_code& ec) noexcept;
\end{itemdecl}
\begin{itemdescr}
	\pnum
	\effects
	\tcode{_Data.emplace_back(std::experimental::io2d::arc(center, radius, angle1, angle2))}.
	
	\pnum
	\tcode{_Current_point == vector_2\{ radius * cos(angle2), -(radius * -sin(angle2)) \} + center}.
	
	\pnum
	If \tcode{_Has_current_point == false}:
	\begin{itemize}
	\item \tcode{_Last_move_to_point == vector_2\{ radius * cos(angle1), -(radius * -sin(angle1)) \} + center}.
	
	\item \tcode{_Has_current_point == true}.
	\end{itemize}
	
	\pnum
	\throws
	As specified in Error reporting (\ref{\iotwod.err.report}).

	\pnum
	\remarks
	In the event of an error, the object shall not be modified.

	\pnum
	\errors
	\tcode{errc::not_enough_memory} if the attempt to add the \tcode{path_data_item} failed.
	
\end{itemdescr}

\indexlibrary{\idxcode{path_factory}!\idxcode{arc_negative}}
\indexlibrary{\idxcode{arc_negative}!\idxcode{path_factory}}
\begin{itemdecl}
    void arc_negative(const vector_2d& center, double radius, double angle1,
      double angle2);
    void arc_negative(const vector_2d& center, double radius, double angle1,
      double angle2, error_code& ec) noexcept;
\end{itemdecl}
\begin{itemdescr}
	\pnum
	\effects
	\tcode{_Data.emplace_back(std::experimental::io2d::arc_negative(center, radius, angle1, angle2))}.
	
	\pnum
	\tcode{_Current_point = vector_2\{ radius * cos(angle1), radius * -sin(angle1) \} + center}.
	
	\pnum
	If \tcode{_Has_current_point == false}:
	\begin{itemize}
	\item \tcode{_Last_move_to_point = vector_2\{ radius * cos(angle2), radius * -sin(angle2) \} + center}.
	
	\item \tcode{_Has_current_point = true}.
	\end{itemize}
	
	\pnum
	\throws
	As specified in Error reporting (\ref{\iotwod.err.report}).

	\pnum
	\remarks
	In the event of an error, the object shall not be modified.

	\pnum
	\errors
	\tcode{errc::not_enough_memory} if the attempt to add the \tcode{path_data_item} failed.
	
\end{itemdescr}

\indexlibrary{\idxcode{path_factory}!\idxcode{curve_to}}
\indexlibrary{\idxcode{curve_to}!\idxcode{path_factory}}
\begin{itemdecl}
    void curve_to(const vector_2d& pt0, const vector_2d& pt1,
      const vector_2d& pt2);
    void curve_to(const vector_2d& pt0, const vector_2d& pt1,
      const vector_2d& pt2, error_code& ec) noexcept;
\end{itemdecl}
\begin{itemdescr}
	\pnum
	\effects
	If \tcode{_Has_current_point == false}:
	\begin{itemize}
	\item \tcode{_Data.reserve(_Data.size() + 2U)}.
	
	\item \tcode{*this.move_to(pt0)}.
	\end{itemize}
	
	\pnum
	\tcode{_Data.emplace_back(std::experimental::io2d::curve_to(pt0, pt1, pt2))}.
	
	\pnum
	\throws
	As specified in Error reporting (\ref{\iotwod.err.report}).

	\pnum
	\remarks
	In the event of an error, the object shall not be modified.

	\pnum
	\errors
	\tcode{errc::not_enough_memory} if the attempt to add the \tcode{path_data_item} failed.
	
\end{itemdescr}

\indexlibrary{\idxcode{path_factory}!\idxcode{line_to}}
\indexlibrary{\idxcode{line_to}!\idxcode{path_factory}}
\begin{itemdecl}
    void line_to(const vector_2d& pt);
    void line_to(const vector_2d& pt, error_code& ec) noexcept;
\end{itemdecl}
\begin{itemdescr}
	\pnum
	\effects
	\tcode{_Data.emplace_back(std::experimental::io2d::line_to(pt))}.
	
	\pnum
	If \tcode{_Has_current_point == false}:
	\begin{itemize}
	\item \tcode{_Last_move_to_point = pt}.
	
	\item \tcode{_Has_current_point = true}.
	\end{itemize}
	
	\pnum
	\tcode{_Current_point = pt}.
	
	\pnum
	\throws
	As specified in Error reporting (\ref{\iotwod.err.report}).

	\pnum
	\remarks
	In the event of an error, the object shall not be modified.

	\pnum
	\errors
	\tcode{errc::not_enough_memory} if the attempt to add the \tcode{path_data_item} failed.
	
\end{itemdescr}

\indexlibrary{\idxcode{path_factory}!\idxcode{move_to}}
\indexlibrary{\idxcode{move_to}!\idxcode{path_factory}}
\begin{itemdecl}
    void move_to(const vector_2d& pt);
    void move_to(const vector_2d& pt, error_code& ec) noexcept;
\end{itemdecl}
\begin{itemdescr}
	\pnum
	\effects
	\tcode{_Data.emplace_back(std::experimental::io2d::move_to(pt))}.
	
	\pnum
	\tcode{_Has_current_point = true}.
	
	\pnum
	\tcode{_Current_point = pt}.
	
	\pnum
	\throws
	As specified in Error reporting (\ref{\iotwod.err.report}).

	\pnum
	\remarks
	In the event of an error, the object shall not be modified.

	\pnum
	\errors
	\tcode{errc::not_enough_memory} if the attempt to add the \tcode{path_data_item} failed.
	
\end{itemdescr}

\indexlibrary{\idxcode{path_factory}!\idxcode{rectangle}}
\indexlibrary{\idxcode{rectangle}!\idxcode{path_factory}}
\begin{itemdecl}
    void rectangle(const std::experimental::io2d::rectangle& r);
    void rectangle(const std::experimental::io2d::rectangle& r,
      error_code& ec) noexcept;
\end{itemdecl}
\begin{itemdescr}
	\pnum
	\effects
	\begin{enumerate}
	\item \tcode{_Data.reserve(_Data.size() + 5U)}.

	\item \tcode{*this.move_to(\{ r.x(), r.y() \})}.
	
	\item \tcode{*this.rel_line_to(\{ r.width(), 0.0 \})}.
	
	\item \tcode{*this.rel_line_to(\{ 0.0, r.height() \})}.
	
	\item \tcode{*this.rel_line_to(\{ -r.width(), 0.0 \})}.
	
	\item \tcode{*this.close_path()}.
	\end{enumerate}
	
	\pnum
	\throws
	As specified in Error reporting (\ref{\iotwod.err.report}).

	\pnum
	\remarks
	In the event of an error, the object shall not be modified.

	\pnum
	\errors
	\tcode{errc::not_enough_memory} if the attempt to add the \tcode{path_data_item} failed.
	
\end{itemdescr}

\indexlibrary{\idxcode{path_factory}!\idxcode{rel_curve_to}}
\indexlibrary{\idxcode{rel_curve_to}!\idxcode{path_factory}}
\begin{itemdecl}
    void rel_curve_to(const vector_2d& dpt0, const vector_2d& dpt1,
      const vector_2d& dpt2);
    void rel_curve_to(const vector_2d& dpt0, const vector_2d& dpt1,
      const vector_2d& dpt2, error_code& ec) noexcept;
\end{itemdecl}
\begin{itemdescr}
	\pnum
	\preconditions
	\tcode{_Has_current_point == true}.

	\pnum
	\effects
	\tcode{_Data.emplace_back(std::experimental::io2d::rel_curve_to(dpt0, dpt1, dpt2))}.
	
	\pnum
	\tcode{_Current_point = dpt2 + _Current_point}.
	
	\pnum
	\throws
	As specified in Error reporting (\ref{\iotwod.err.report}).

	\pnum
	\remarks
	In the event of an error, the object shall not be modified.

	\pnum
	\errors
	\tcode{errc::not_enough_memory} if the attempt to add the \tcode{path_data_item} failed.
	
	\pnum
	\tcode{io2d_error::no_current_point} if the preconditions are violated.
	
\end{itemdescr}

\indexlibrary{\idxcode{path_factory}!\idxcode{rel_line_to}}
\indexlibrary{\idxcode{rel_line_to}!\idxcode{path_factory}}
\begin{itemdecl}
    void rel_line_to(const vector_2d& dpt);
    void rel_line_to(const vector_2d& dpt, error_code& ec) noexcept;
\end{itemdecl}
\begin{itemdescr}
	\pnum
	\preconditions
	\tcode{_Has_current_point == true}.

	\pnum
	\effects
	\tcode{_Data.emplace_back(std::experimental::io2d::rel_line_to(pt))}.
	
	\pnum
	\tcode{_Current_point = dpt + _Current_point}.
	
	\pnum
	\throws
	As specified in Error reporting (\ref{\iotwod.err.report}).

	\pnum
	\remarks
	In the event of an error, the object shall not be modified.

	\pnum
	\errors
	\tcode{errc::not_enough_memory} if the attempt to add the \tcode{path_data_item} failed.
	
	\pnum
	\tcode{io2d_error::no_current_point} if the preconditions are violated.
	
\end{itemdescr}

\indexlibrary{\idxcode{path_factory}!\idxcode{rel_move_to}}
\indexlibrary{\idxcode{rel_move_to}!\idxcode{path_factory}}
\begin{itemdecl}
    void rel_move_to(const vector_2d& dpt);
    void rel_move_to(const vector_2d& dpt, error_code& ec) noexcept;
\end{itemdecl}
\begin{itemdescr}
	\pnum
	\preconditions
	\tcode{_Has_current_point == true}.

	\pnum
	\effects
	\tcode{_Data.emplace_back(std::experimental::io2d::rel_move_to(dpt))}.
	
	\pnum
	\tcode{_Current_point = dpt + _Current_point}.
	
	\pnum
	\throws
	As specified in Error reporting (\ref{\iotwod.err.report}).

	\pnum
	\remarks
	In the event of an error, the object shall not be modified.

	\pnum
	\errors
	\tcode{errc::not_enough_memory} if the attempt to add the \tcode{path_data_item} failed.
	
	\pnum
	\tcode{io2d_error::no_current_point} if the preconditions are violated.
	
\end{itemdescr}

\indexlibrary{\idxcode{path_factory}!\idxcode{transform_matrix}}
\indexlibrary{\idxcode{transform_matrix}!\idxcode{path_factory}}
\begin{itemdecl}
    void transform_matrix(const matrix_2d& m);
    void transform_matrix(const matrix_2d& m, error_code& ec) noexcept;
\end{itemdecl}
\begin{itemdescr}
	\pnum
	\effects
	\tcode{_Data.emplace_back(std::experimental::io2d::change_matrix(m))}.
	
	\pnum
	\tcode{_Transform_matrix = m}.
	
	\pnum
	\throws
	As specified in Error reporting (\ref{\iotwod.err.report}).

	\pnum
	\remarks
	In the event of an error, the object shall not be modified.

	\pnum
	\errors
	\tcode{errc::not_enough_memory} if the attempt to add the \tcode{path_data_item} failed.
	
\end{itemdescr}

\indexlibrary{\idxcode{path_factory}!\idxcode{origin}}
\indexlibrary{\idxcode{origin}!\idxcode{path_factory}}
\begin{itemdecl}
    void origin(const vector_2d& pt);
    void origin(const vector_2d& pt, error_code& ec) noexcept;
\end{itemdecl}
\begin{itemdescr}
	\pnum
	\effects
	\tcode{_Data.emplace_back(std::experimental::io2d::change_origin(pt)))}.
	
	\pnum
	\tcode{_Origin = pt}.
	
	\pnum
	\postconditions
	\tcode{_Origin == pt}.
	
	\pnum
	\throws
	As specified in Error reporting (\ref{\iotwod.err.report}).

	\pnum
	\remarks
	In the event of an error, the object shall not be modified.

	\pnum
	\errors
	\tcode{errc::not_enough_memory} if the attempt to add the \tcode{path_data_item} failed.
	
\end{itemdescr}

\indexlibrary{\idxcode{path_factory}!\idxcode{clear}}
\indexlibrary{\idxcode{clear}!\idxcode{path_factory}}
\begin{itemdecl}
    void clear() noexcept;
\end{itemdecl}
\begin{itemdescr}
	\pnum
	\postconditions
	\tcode{_Data.empty() == true}.
	
	\pnum
	\tcode{_Has_current_point == false}.
	
	\pnum
	\tcode{_Transform_matrix == matrix_2d::init_identity()}.
	
	\pnum
	\tcode{_Origin == vector_2d\{ \}}.

\end{itemdescr}

\rSec1 [pathfactory.observers] {\tcode{path_factory} observers}

\indexlibrary{\idxcode{path_factory}!\idxcode{path_extents}}
\indexlibrary{\idxcode{path_extents}!\idxcode{path_factory}}
\begin{itemdecl}
    std::experimental::io2d::rectangle path_extents() const;
    std::experimental::io2d::rectangle path_extents(error_code& ec) const noexcept;
\end{itemdecl}
\begin{itemdescr}
	\pnum
	\returns
	A \tcode{rectangle} object which contains the extents of the \term{path segments}, including \term{degenerate path segments}, in \tcode{_Data} when it is processed as described in \ref{pathgeometries.processing}.
	\enternote
	By using path segments, this description intentionally omits points established by \tcode{move_to} and \tcode{rel_move_to} operations from the extents value except where those points are subsequently used in defining a path segment.
	\exitnote

	\pnum
	\throws
	As specified in Error reporting (\ref{\iotwod.err.report}).

	\pnum
	\errors
	\tcode{io2d_error::invalid_matrix} if \tcode{_Data} includes a \tcode{change_matrix} operation which establishes a non-invertible \tcode{matrix_2d} as the transformation matrix and that matrix must subsequently be inverted in order to process the path geometries.
\end{itemdescr}

\indexlibrary{\idxcode{path_factory}!\idxcode{has_current_point}}
\indexlibrary{\idxcode{has_current_point}!\idxcode{path_factory}}
\begin{itemdecl}
    bool has_current_point() const noexcept;
\end{itemdecl}
\begin{itemdescr}
	\pnum
	\returns
	\tcode{_Has_current_point}.

\end{itemdescr}

\indexlibrary{\idxcode{path_factory}!\idxcode{current_point}}
\indexlibrary{\idxcode{current_point}!\idxcode{path_factory}}
\begin{itemdecl}
    vector_2d current_point() const;
    vector_2d current_point(error_code& ec) const noexcept;
\end{itemdecl}
\begin{itemdescr}
	\pnum
	\preconditions
	\tcode{_Has_current_point == true}.
	
	\pnum
	\returns
	\tcode{_Current_point}.

	\pnum
	\throws
	As specified in Error reporting (\ref{\iotwod.err.report}).

	\pnum
	\errors
	\tcode{io2d_error::no_current_point} if the preconditions are violated.
	
\end{itemdescr}

\indexlibrary{\idxcode{path_factory}!\idxcode{transform_matrix}}
\indexlibrary{\idxcode{transform_matrix}!\idxcode{path_factory}}
\begin{itemdecl}
    matrix_2d transform_matrix() const noexcept;
\end{itemdecl}
\begin{itemdescr}
	\pnum
	\returns
	\tcode{_Transform_matrix}.

\end{itemdescr}

\indexlibrary{\idxcode{path_factory}!\idxcode{origin}}
\indexlibrary{\idxcode{origin}!\idxcode{path_factory}}
\begin{itemdecl}
    vector_2d origin() const noexcept;
\end{itemdecl}
\begin{itemdescr}
	\pnum
	\returns
	\tcode{_Origin}.

\end{itemdescr}

\indexlibrary{\idxcode{path_factory}!\idxcode{data}}
\indexlibrary{\idxcode{data}!\idxcode{path_factory}}
\begin{itemdecl}
    vector<path_data_item> data() const;
    vector<path_data_item> data(error_code& ec) const noexcept;
\end{itemdecl}
\begin{itemdescr}
	\pnum
	\returns
	A copy of \tcode{_Data}.

	\pnum
	\throws
	As specified in Error reporting (\ref{\iotwod.err.report}).

	\pnum
	\errors
	\tcode{errc::not_enough_memory} if there was a failure to allocate memory.
	
\end{itemdescr}

\indexlibrary{\idxcode{path_factory}!\idxcode{data_item}}
\indexlibrary{\idxcode{data_item}!\idxcode{path_factory}}
\begin{itemdecl}
    path_data_item data_item(unsigned int index) const;
    path_data_item data_item(unsigned int index, error_code& ec) const noexcept;
\end{itemdecl}
\begin{itemdescr}
	\pnum
	\preconditions
	\tcode{_Data.size() > index}.
	
	\pnum
	\returns
	\tcode{_Data.at(index)}.
	
	\pnum
	\throws
	As specified in Error reporting (\ref{\iotwod.err.report}).

	\pnum
	\errors
	\tcode{io2d_error::invalid_index} if \tcode{index} violated the preconditions.

\end{itemdescr}

\indexlibrary{\idxcode{path_factory}!\idxcode{data_ref}}
\indexlibrary{\idxcode{data_ref}!\idxcode{path_factory}}
\begin{itemdecl}
    const vector<path_data_item>& data_ref() const noexcept;
\end{itemdecl}
\begin{itemdescr}
	\pnum
	\returns
	\tcode{_Data}.

\end{itemdescr}

\addtocounter{SectionDepthBase}{1}
%%!TEX root = io2d.tex
\rSec0 [pathdataitem.pathdata] {Class \tcode{path_data_item::path_data}}

\pnum
\indexlibrary{\idxcode{path_data_item::path_data}}
The class \tcode{path_data_item::path_data} serves as an abstract base class for classes that describe operations performed on path geometries.

\rSec1 [pathdataitem.pathdata.synopsis] {\tcode{path_data_item::path_data} synopsis}

\begin{codeblock}
namespace std { namespace experimental { namespace io2d { inline namespace v1 {
  class path_data_item::path_data {
  public:
    // \ref{pathdataitem.pathdata.cons}, construct/copy/move/destroy:
    path_data() noexcept;
    path_data(const path_data& other) noexcept;
    path_data& operator=(const path_data& other) noexcept;
    path_data(path_data&& other) noexcept;
    path_data& operator=(path_data&& other) noexcept;
    virtual ~path_data() noexcept;

    // \ref{pathdataitem.pathdata.observers}, observers:
    virtual path_data_type type() const noexcept = 0;
  };
} } } }
\end{codeblock}

\rSec1 [pathdataitem.pathdata.cons] {\tcode{path_data_item::path_data} constructors and assignment operators}

\indexlibrary{\idxcode{path_data_item::path_data}!destructor}
\begin{itemdecl}
    virtual ~path_data() noexcept;
\end{itemdecl}
\begin{itemdescr}
	\pnum
	\effects
	Destroys an object of type \tcode{path_data}.
	
\end{itemdescr}

\rSec1 [pathdataitem.pathdata.observers]{\tcode{path_data_item::path_data} observers}

\indexlibrary{\idxcode{path_data_item::path_data}!\idxcode{type}}
\indexlibrary{\idxcode{type}!\idxcode{path_data_item::path_data}}
\begin{itemdecl}
    virtual path_data_type type() const noexcept = 0;
\end{itemdecl}
\begin{itemdescr}
	\pnum
	\returns
	The \tcode{path_data_type} of the \tcode{path_data}-derived object.
	
	\pnum
	\realnote
	This is used for casting to the correct type when iterating through a collection of \tcode{path_data} objects.
\end{itemdescr}

%!TEX root = io2d.tex
\rSec0 [\iotwod.arc] {Class \tcode{arc}}

\rSec1 [\iotwod.arc.general] {In general}

\pnum
\indexlibrary{\idxcode{arc}}%
The class \tcode{arc} describes a figure item that is a segment.

\pnum
It has a \term{radius} of type \tcode{basic_point_2d}, a \term{rotation} of type \tcode{float}, and a \term{start angle} of type \tcode{float}.

\pnum
It forms a portion of the circumference of a circle. The centre of the circle is implied by the start point, the radius and the start angle of the arc.

\rSec1 [\iotwod.arc.cons] {\tcode{arc} constructors}

\indexlibrary{\idxcode{arc}!constructor}%
\begin{itemdecl}
arc() noexcept;
\end{itemdecl}
\begin{itemdescr}
\pnum
\effects
Equivalent to: \tcode{arc\{ basic_point_2d(10.0f, 10.0f), pi<float>, pi<float> \};}.
\end{itemdescr}

\indexlibrary{\idxcode{arc}!constructor}%
\begin{itemdecl}
arc(const basic_point_2d<typename GraphicsSurfaces::graphics_math_type>& rad,
  float rot, float sang) noexcept;
\end{itemdecl}
\begin{itemdescr}
\pnum
\effects
Constructs an object of type \tcode{arc}.

\pnum
The radius is \tcode{rad}.

\pnum
The rotation is \tcode{rot}.

\pnum
The start angle is \tcode{sang}.
\end{itemdescr}

\indexlibrary{\idxcode{arc}!constructor}%
\begin{itemdecl}
arc(const arc& other);
arc(arc&& other) noexcept;
\end{itemdecl}
\begin{itemdescr}
\pnum
\effects
Constructs an object of type \tcode{arc}. In the second form, other is left in a valid state with an unspecified value.

\pnum
The radius is \tcode{other.radius()}.

\pnum
The rotation is \tcode{other.rotation()}.

\pnum
The start angle is \tcode{other.start_angle()}.
\end{itemdescr}

\rSec1 [\iotwod.arc.assign] {\tcode{arc} assignment operators}

\indexlibrary{\idxcode{arc}!assignment}%
\begin{itemdecl}
arc& operator=(const arc& other);
\end{itemdecl}
\begin{itemdescr}
\pnum
\effects
If \tcode{*this} and \tcode{other} are not the same object, modifies \tcode{*this} such that \tcode{*this.radius()} is \tcode{other.radius()}, \tcode{*this.rotation()} is \tcode{other.rotation()} and \tcode{*this.start_angle()} is \tcode{other.start_angle()}

\pnum
If \tcode{*this} and \tcode{other} are the same object, the member has no effect.

\pnum
\returns
\tcode{*this}
\end{itemdescr}

\indexlibrary{\idxcode{arc}!assignment}%
\begin{itemdecl}
arc& operator=(arc&& other) noexcept;
\end{itemdecl}
\begin{itemdescr}
\pnum
\effects
<TODO>

\pnum
\returns
\tcode{*this}
\end{itemdescr}

\rSec1 [\iotwod.arc.modifiers]{\tcode{arc} modifiers}

\indexlibrarymember{radius}{arc}%
\begin{itemdecl}
void radius(const basic_point_2d<typename GraphicsSurfaces::graphics_math_type>& rad) noexcept;
\end{itemdecl}
\begin{itemdescr}
\pnum
\effects
The radius is \tcode{rad}.
\end{itemdescr}

\indexlibrarymember{rotation}{arc}%
\begin{itemdecl}
constexpr void rotation(float rot) noexcept;
\end{itemdecl}
\begin{itemdescr}
\pnum
\effects
The rotation is \tcode{rot}.
\end{itemdescr}

\indexlibrarymember{start_angle}{arc}%
\begin{itemdecl}
void start_angle(float sang) noexcept;
\end{itemdecl}
\begin{itemdescr}
\pnum
\effects
The start angle is \tcode{sang}.
\end{itemdescr}

\rSec1 [\iotwod.arc.observers]{\tcode{arc} observers}

\indexlibrarymember{radius}{arc}%
\begin{itemdecl}
basic_point_2d<typename GraphicsSurfaces::graphics_math_type> radius() const noexcept;
\end{itemdecl}
\begin{itemdescr}
\pnum
\returns
The radius.
\end{itemdescr}

\indexlibrarymember{rotation}{arc}%
\begin{itemdecl}
float rotation() const noexcept;
\end{itemdecl}
\begin{itemdescr}
\pnum
\returns
The rotation.
\end{itemdescr}

\indexlibrarymember{start_angle}{arc}%
\begin{itemdecl}
float start_angle() const noexcept;
\end{itemdecl}
\begin{itemdescr}
\pnum
\returns
The start angle.
\end{itemdescr}

\indexlibrarymember{center}{arc}%
\begin{itemdecl}
basic_point_2d<typename GraphicsSurfaces::graphics_math_type> center(const basic_point_2d<typename
  GraphicsSurfaces::graphics_math_type>& cpt, const basic_matrix_2d<typename
  GraphicsSurfaces::graphics_math_type>& m = basic_matrix_2d<typename
  GraphicsSurfaces::graphics_math_type>{}) const noexcept;
\end{itemdecl}
\begin{itemdescr}
\pnum
\returns
As-if:
\begin{codeblock}
auto lmtx = m;
lmtx.m20 = 0.0f;
lmtx.m21 = 0.0f;
auto centerOffset = point_for_angle(two_pi<float> - start_angle(), radius());
centerOffset.y = -centerOffset.y;
return cpt - centerOffset * lmtx;
\end{codeblock}
\end{itemdescr}

\indexlibrarymember{start_angle}{arc}%
\begin{itemdecl}
basic_point_2d<typename GraphicsSurfaces::graphics_math_type> end_pt(const basic_point_2d<typename
  GraphicsSurfaces::graphics_math_type>& cpt, const basic_matrix_2d<typename
  GraphicsSurfaces::graphics_math_type>& m = basic_matrix_2d<typename
  GraphicsSurfaces::graphics_math_type>{}) const noexcept;
\end{itemdecl}
\begin{itemdescr}
\pnum
\returns
As-if:
\begin{codeblock}
auto lmtx = m;
auto tfrm = matrix_2d::init_rotate(start_angle() + rotation());
lmtx.m20 = 0.0f;
lmtx.m21 = 0.0f;
auto pt = (radius() * tfrm);
pt.y = -pt.y;
return cpt + pt * lmtx;
\end{codeblock}
\end{itemdescr}

\rSec1 [\iotwod.arc.ops]{\tcode{arc} operators}

\indexlibrarymember{operator==}{arc}%
\begin{itemdecl}
template <class GraphicsSurfaces>
bool operator==(const typename basic_figure_items<GraphicsSurfaces>::arc& lhs,
  const typename basic_figure_items<GraphicsSurfaces>::arc& rhs) noexcept;
\end{itemdecl}
\begin{itemdescr}
\pnum
\returns
\begin{codeblock}
lhs.radius() == rhs.radius() && lhs.rotation() == rhs.rotation() &&
lhs.start_angle() && rhs.start_angle()
\end{codeblock}
\end{itemdescr}

%!TEX root = io2d.tex
\rSec0 [pathdataitem.arcnegative] {Class \tcode{path_factory::path_arc_counterclockwise}}

\pnum
\indexlibrary{\idxcode{path_factory::path_arc_counterclockwise}}
The class \tcode{path_factory::path_arc_counterclockwise} describes an operation on a path group.

\pnum
This operation creates a circular arc with counterclockwise rotation.

\pnum
The unit for the values passed to and returned by \tcode{path_factory::path_arc_counterclockwise::angle_1} and \tcode{path_factory::path_arc_counterclockwise::angle_2} is the radian.

\pnum
The arc's \term{start point} is \tcode{vector_2d\{ *this.radius() * cos(*this.angle_1(), *this.radius() * -sin(*this.angle_1()) \} + *this.center()}.

\pnum
Its \term{end point} is \tcode{vector_2d\{ *this.radius() * cos(*this.angle_2(), *this.radius() * -sin(*this.angle_2()) \} + *this.center()}.

\pnum
If the current path geometry has a current point, a line is created from the current point to the start point before this arc operation is processed. Otherwise the start point is set as the current point and last-move-to point of the current path geometry.

\pnum
The arc rotates around the point returned by \tcode{*this.center()}.

\pnum
The arc begins at its start point and proceeds counterclockwise until it reaches its end point.

\pnum
The current point is set be to the arc's end point at the end of this operation.

\pnum
For purposes of determining whether a point is on the arc, if the value returned by \tcode{*this.angle_2()} is greater than the value returned by \tcode{*this.angle_1()} then the value returned by \tcode{*this.angle_2()} shall be continuously decremented by $2\pi$ until it is less than the value returned by \tcode{*this.angle_1()}.

\rSec1 [pathdataitem.arcnegative.synopsis] {\tcode{path_factory::path_arc_counterclockwise} synopsis}

\begin{codeblock}
namespace std { namespace experimental { namespace io2d { inline namespace v1 {
  class path_factory::path_arc_counterclockwise {
  public:
    // \ref{pathdataitem.arcnegative.cons}, construct/copy/move/destroy:
    arc_negative() noexcept;
    arc_negative(const arc_negative&) noexcept;
    path_factory::path_arc_counterclockwise& operator=(const arc_negative&) noexcept;
    arc_negative(arc_negative&&) noexcept;
    path_factory::path_arc_counterclockwise& operator=(arc_negative&&) noexcept;
    arc_negative(const vector_2d& ctr, double rad, double angle1,
      double angle2) noexcept;

    // \ref{pathdataitem.arcnegative.modifiers}, modifiers:
    void center(const vector_2d& value) noexcept;
    void radius(double value) noexcept;
    void angle_1(double radians) noexcept;
    void angle_2(double radians) noexcept;

    // \ref{pathdataitem.arcnegative.observers}, observers:
    vector_2d center() const noexcept;
    double radius() const noexcept;
    double angle_1() const noexcept;
    double angle_2() const noexcept;
    virtual path_data_type type() const noexcept override;

  private:
    vector_2d _Center;   // \expos
    double _Radius;  // \expos
    double _Angle_1; // \expos
    double _Angle_2; // \expos
  };
} } } }
\end{codeblock}

\rSec1 [pathdataitem.arcnegative.cons] {\tcode{path_factory::path_arc_counterclockwise} constructors and assignment operators}

\indexlibrary{\idxcode{path_factory::path_arc_counterclockwise}!constructor}
\begin{itemdecl}
    arc_negative() noexcept;
\end{itemdecl}
\begin{itemdescr}
	\pnum
	\effects
	Constructs an object of type \tcode{path_factory::path_arc_counterclockwise}.
	
	\pnum
	\postconditions
	\tcode{_Center == vector_2d(0.0, 0.0)}.

	\tcode{_Radius == 0.0}.

	\tcode{_Angle_1 == 0.0}.

	\tcode{_Angle_2 == 0.0}.
\end{itemdescr}

\indexlibrary{\idxcode{path_factory::path_arc_counterclockwise}!constructor}
\begin{itemdecl}
    arc_negative(const vector_2d& ctr, double rad, double angle1,
      double angle2) noexcept;
\end{itemdecl}
\begin{itemdescr}
	\pnum
	\effects
	Constructs an object of type \tcode{path_factory::path_arc_counterclockwise}.
	
	\pnum
	\postconditions
	\tcode{_Center == ctr}.

	\tcode{_Radius == rad}.

	\tcode{_Angle_1 == angle1}.

	\tcode{_Angle_2 == angle2}.
\end{itemdescr}

\rSec1 [pathdataitem.arcnegative.modifiers]{\tcode{path_factory::path_arc_counterclockwise} modifiers}

\indexlibrary{\idxcode{path_factory::path_arc_counterclockwise}!\idxcode{center}}
\indexlibrary{\idxcode{center}!\idxcode{path_factory::path_arc_counterclockwise}}
\begin{itemdecl}
    void center(const vector_2d& value) noexcept;
\end{itemdecl}
\begin{itemdescr}
	\pnum
	\postconditions
	\tcode{_Center == value}.
\end{itemdescr}

\indexlibrary{\idxcode{path_factory::path_arc_counterclockwise}!\idxcode{radius}}
\indexlibrary{\idxcode{radius}!\idxcode{path_factory::path_arc_counterclockwise}}
\begin{itemdecl}
    void radius(double value) noexcept;
\end{itemdecl}
\begin{itemdescr}
	\pnum
	\postconditions
	\tcode{_Radius == value}.
\end{itemdescr}

\indexlibrary{\idxcode{path_factory::path_arc_counterclockwise}!\idxcode{angle_1}}
\indexlibrary{\idxcode{angle_1}!\idxcode{path_factory::path_arc_counterclockwise}}
\begin{itemdecl}
    void angle_1(double value) noexcept;
\end{itemdecl}
\begin{itemdescr}
	\pnum
	\postconditions
	\tcode{_Angle_1 == value}.
\end{itemdescr}

\indexlibrary{\idxcode{path_factory::path_arc_counterclockwise}!\idxcode{angle_2}}
\indexlibrary{\idxcode{angle_2}!\idxcode{path_factory::path_arc_counterclockwise}}
\begin{itemdecl}
    void angle_2(double value) noexcept;
\end{itemdecl}
\begin{itemdescr}
	\pnum
	\postconditions
	\tcode{_Angle_2 == value}.
\end{itemdescr}

\rSec1 [pathdataitem.arcnegative.observers]{\tcode{path_factory::path_arc_counterclockwise} observers}

\indexlibrary{\idxcode{path_factory::path_arc_counterclockwise}!\idxcode{center}}
\indexlibrary{\idxcode{center}!\idxcode{path_factory::path_arc_counterclockwise}}
\begin{itemdecl}
    vector_2d center() const noexcept;
\end{itemdecl}
\begin{itemdescr}
	\pnum
	\returns
	\tcode{_Center}.
\end{itemdescr}

\indexlibrary{\idxcode{path_factory::path_arc_counterclockwise}!\idxcode{radius}}
\indexlibrary{\idxcode{radius}!\idxcode{path_factory::path_arc_counterclockwise}}
\begin{itemdecl}
    double radius() const noexcept;
\end{itemdecl}
\begin{itemdescr}
	\pnum
	\returns
	\tcode{_Radius}.
\end{itemdescr}

\indexlibrary{\idxcode{path_factory::path_arc_counterclockwise}!\idxcode{angle_1}}
\indexlibrary{\idxcode{angle_1}!\idxcode{path_factory::path_arc_counterclockwise}}
\begin{itemdecl}
    double angle_1() const noexcept;
\end{itemdecl}
\begin{itemdescr}
	\pnum
	\returns
	\tcode{_Angle_1}.
\end{itemdescr}

\indexlibrary{\idxcode{path_factory::path_arc_counterclockwise}!\idxcode{angle_2}}
\indexlibrary{\idxcode{angle_2}!\idxcode{path_factory::path_arc_counterclockwise}}
\begin{itemdecl}
    double angle_2() const noexcept;
\end{itemdecl}
\begin{itemdescr}
	\pnum
	\returns
	\tcode{_Angle_2}.
\end{itemdescr}

\indexlibrary{\idxcode{path_factory::path_arc_counterclockwise}!\idxcode{type}}
\indexlibrary{\idxcode{type}!\idxcode{path_factory::path_arc_counterclockwise}}
\begin{itemdecl}
    virtual path_data_type type() const noexcept override;
\end{itemdecl}
\begin{itemdescr}
	\pnum
	\returns
	\tcode{path_data_type::arc_negative}.
\end{itemdescr}

%!TEX root = io2d.tex
\rSec0 [pathfactory.pathchangematrix] {Class \tcode{path_factory::path_change_matrix}}

\rSec1 [pathfactory.pathchangematrix.synopsis] {\tcode{path_factory::path_change_matrix} synopsis}

\pnum
\indexlibrary{\idxcode{path_factory::path_change_matrix}}
The class \tcode{path_factory::path_change_matrix} describes an operation on a path group.

\pnum
This operation changes the transformation matrix for a path group to be the value returned by \tcode{*this.matrix()}. As shown in \ref{paths.processing}, the new transformation matrix does not affect any operations that came before this operation. It is only used in processing operations that come after it. It continues to be used until another \tcode{path_factory::path_change_matrix} object is encountered or the end of the path group is reached.

\begin{codeblock}
namespace std { namespace experimental { namespace io2d { inline namespace v1 {
  class path_factory::path_change_matrix {
  public:
    // \ref{pathfactory.pathchangematrix.cons}, construct/copy/move/destroy:
    change_matrix() noexcept;
    change_matrix(const change_matrix&) noexcept;
    path_factory::path_change_matrix& operator=(const change_matrix&) noexcept;
    change_matrix(change_matrix&&) noexcept;
    path_factory::path_change_matrix& operator=(change_matrix&&) noexcept;
    explicit change_matrix(const matrix_2d& m) noexcept;

    // \ref{pathfactory.pathchangematrix.modifiers}, modifiers:
    void matrix(const matrix_2d& value) noexcept;

    // \ref{pathfactory.pathchangematrix.observers}, observers:
    matrix_2d matrix() const noexcept;
    virtual path_data_type type() const noexcept override;
    
  private:
    matrix_2d _Matrix; // \expos
  };
} } } }
\end{codeblock}

\rSec1 [pathfactory.pathchangematrix.cons] {\tcode{path_factory::path_change_matrix} constructors and assignment operators}

\indexlibrary{\idxcode{path_factory::path_change_matrix}!constructor}
\begin{itemdecl}
    change_matrix() noexcept;
\end{itemdecl}
\begin{itemdescr}
	\pnum
	\effects
	Constructs an object of type \tcode{path_factory::path_change_matrix}.
	
	\pnum
	\postconditions
	\tcode{_Matrix == matrix_2d\{\}}.
\end{itemdescr}

\indexlibrary{\idxcode{path_factory::path_change_matrix}!constructor}
\begin{itemdecl}
    explicit change_matrix(const matrix_2d& m) noexcept;
\end{itemdecl}
\begin{itemdescr}
	\pnum
	\effects
	Constructs an object of type \tcode{path_factory::path_change_matrix}.
	
	\pnum
	\postconditions
	\tcode{_Matrix == m}.
\end{itemdescr}

\rSec1 [pathfactory.pathchangematrix.modifiers]{\tcode{path_factory::path_change_matrix} modifiers}

\indexlibrary{\idxcode{path_factory::path_change_matrix}!\idxcode{matrix}}
\indexlibrary{\idxcode{matrix}!\idxcode{path_factory::path_change_matrix}}
\begin{itemdecl}
    void matrix(const matrix_2d& value) noexcept;
\end{itemdecl}
\begin{itemdescr}
	\pnum
	\postconditions
	\tcode{_Matrix == value}.
\end{itemdescr}

\rSec1 [pathfactory.pathchangematrix.observers]{\tcode{path_factory::path_change_matrix} observers}

\indexlibrary{\idxcode{path_factory::path_change_matrix}!\idxcode{matrix}}
\indexlibrary{\idxcode{matrix}!\idxcode{path_factory::path_change_matrix}}
\begin{itemdecl}
    matrix_2d matrix() const noexcept;
\end{itemdecl}
\begin{itemdescr}
	\pnum
	\returns
	\tcode{_Matrix}.
\end{itemdescr}

\indexlibrary{\idxcode{path_factory::path_change_matrix}!\idxcode{type}}
\indexlibrary{\idxcode{type}!\idxcode{path_factory::path_change_matrix}}
\begin{itemdecl}
    virtual path_data_type type() const noexcept override;
\end{itemdecl}
\begin{itemdescr}
	\pnum
	\returns
	\tcode{path_data_type::change_matrix}.
\end{itemdescr}

%!TEX root = io2d.tex
\rSec0 [pathdataitem.changeorigin] {Class \tcode{path_factory::path_change_origin}}

\pnum
\indexlibrary{\idxcode{path_factory::path_change_origin}}
The class \tcode{path_factory::path_change_origin} describes an operation on a path geometry collection.

\pnum
This operation changes the origin point for a path geometry collection to be the value returned by \tcode{*this.origin()}. As shown in \ref{pathgeometries.processing}, the new origin point does not affect any operations that came before this operation. It is only used in processing operations that come after it. It continues to be used until another \tcode{path_factory::path_change_origin} object is encountered or the end of the path geometry collection is reached.

\rSec1 [pathdataitem.changeorigin.synopsis] {\tcode{path_factory::path_change_origin} synopsis}

\begin{codeblock}
namespace std { namespace experimental { namespace io2d { inline namespace v1 {
  class path_factory::path_change_origin {
  public:
    // \ref{pathdataitem.changeorigin.cons}, construct/copy/move/destroy:
    change_origin() noexcept;
    change_origin(const change_origin&) noexcept;
    path_factory::path_change_origin& operator=(const change_origin&) noexcept;
    change_origin(change_origin&&) noexcept;
    path_factory::path_change_origin& operator=(change_origin&&) noexcept;
    explicit change_origin(const vector_2d& pt) noexcept;

    // \ref{pathdataitem.changeorigin.modifiers}, modifiers:
    void origin(const vector_2d& value) noexcept;

    // \ref{pathdataitem.changeorigin.observers}, observers:
    vector_2d origin() const noexcept;
    virtual path_data_type type() const noexcept override;
    
  private:
    vector_2d _Data; // \expos
  };
} } } }
\end{codeblock}

\rSec1 [pathdataitem.changeorigin.cons] {\tcode{path_factory::path_change_origin} constructors and assignment operators}

\indexlibrary{\idxcode{path_factory::path_change_origin}!constructor}
\begin{itemdecl}
    change_origin() noexcept;
\end{itemdecl}
\begin{itemdescr}
	\pnum
	\effects
	Constructs an object of type \tcode{path_factory::path_change_origin}.
	
	\pnum
	\postconditions
	\tcode{_Data == vector_2d(0.0, 0.0)}.
\end{itemdescr}

\indexlibrary{\idxcode{path_factory::path_change_origin}!constructor}
\begin{itemdecl}
    explicit change_origin(const vector_2d& pt) noexcept;
\end{itemdecl}
\begin{itemdescr}
	\pnum
	\effects
	Constructs an object of type \tcode{path_factory::path_change_origin}.
	
	\pnum
	\postconditions
	\tcode{_Data == pt}.
\end{itemdescr}

\rSec1 [pathdataitem.changeorigin.modifiers]{\tcode{path_factory::path_change_origin} modifiers}

\indexlibrary{\idxcode{path_factory::path_change_origin}!\idxcode{origin}}
\indexlibrary{\idxcode{origin}!\idxcode{path_factory::path_change_origin}}
\begin{itemdecl}
    void origin(const vector_2d& value) noexcept;
\end{itemdecl}
\begin{itemdescr}
	\pnum
	\postconditions
	\tcode{_Data == value}.
\end{itemdescr}

\rSec1 [pathdataitem.changeorigin.observers]{\tcode{change_origin} observers}

\indexlibrary{\idxcode{path_factory::path_change_origin}!\idxcode{origin}}
\indexlibrary{\idxcode{origin}!\idxcode{path_factory::path_change_origin}}
\begin{itemdecl}
    vector_2d origin() const noexcept;
\end{itemdecl}
\begin{itemdescr}
	\pnum
	\returns
	\tcode{_Data}.
\end{itemdescr}

\indexlibrary{\idxcode{path_factory::path_move_to}!\idxcode{type}}
\indexlibrary{\idxcode{type}!\idxcode{path_factory::path_move_to}}
\begin{itemdecl}
    virtual path_data_type type() const noexcept override;
\end{itemdecl}
\begin{itemdescr}
	\pnum
	\returns
	\tcode{path_data_type::change_origin}.
\end{itemdescr}

%!TEX root = io2d.tex
\rSec0 [pathfactory.pathclosepath] {Class \tcode{path_factory::path_close_path}}

%\pnum
%\indexlibrary{\idxcode{path_factory::path_close_path}}
%The class \tcode{path_factory::path_close_path} describes a path instruction that affects the interpretation of a path factory's path group. It is described in terms of its effect on the evaluation of the path group. 
%
%\pnum
%If the current point in the path group contains a value. If it does, this instruction creates a line from the current point to the path group's last-move-to point. It then sets the path group's current point and last-move-to point to the value of the previous path geometry's last-move-to point.
%
%\pnum
%If there is no current point, then this operation does nothing.
%\enternote
%Because this operation does nothing if there is no current point, there is no need to track whether or not a path geometry has a valid last-move-to point. This operation is the only operation that uses the last-move-to point and all operations that establish a current point for a path geometry also establish a valid last-move-to point for that path geometry.
%\exitnote
%
\rSec1 [pathfactory.pathclosepath.synopsis] {\tcode{path_factory::path_close_path} synopsis}

\begin{codeblock}
namespace std { namespace experimental { namespace io2d { inline namespace v1 {
  class path_factory::path_close_path {
  };
} } } }
\end{codeblock}

\enternote
This class is a path instruction that contains no data. It exists to enable certain operations within a path group.
\exitnote

%!TEX root = io2d.tex
\rSec0 [pathdataitem.curveto] {Class \tcode{path_data_item::curve_to}}

\pnum
\indexlibrary{\idxcode{path_data_item::curve_to}}
The class \tcode{path_data_item::curve_to} describes an operation on a path geometry collection.

\pnum
If the current path geometry has no current point, then this operation behaves exactly as if this object was preceded by a \tcode{path_data_item::move_to} object constructed with the value returned by \tcode{*this.control_point_1()} as its argument.

\pnum
This operation creates a cubic B\'ezier curve from the current point to the point returned by \tcode{*this.end_point()}, with the first control point being the point returned by \tcode{*this.control_point_1()} and the second control point being the point returned by \tcode{*this.control_point_2()}. It then sets the current point to be the point returned by \tcode{*this.end_point()}.

\rSec1 [pathdataitem.curveto.synopsis] {\tcode{path_data_item::curve_to} synopsis}

\begin{codeblock}
namespace std { namespace experimental { namespace io2d { inline namespace v1 {
  class path_data_item::curve_to : public path_data_item::path_data {
  public:
    // \ref{pathdataitem.curveto.cons}, construct/copy/move/destroy:
    curve_to() noexcept;
    curve_to(const curve_to&) noexcept;
    path_data_item::curve_to& operator=(const curve_to&) noexcept;
    curve_to(curve_to&&) noexcept;
    path_data_item::curve_to& operator=(curve_to&&) noexcept;
    curve_to(const vector_2d& controlPoint1, const vector_2d& controlPoint2,
      const vector_2d& endPoint) noexcept;

    // \ref{pathdataitem.curveto.modifiers}, modifiers:
    void control_point_1(const vector_2d& value) noexcept;
    void control_point_2(const vector_2d& value) noexcept;
    void end_point(const vector_2d& value) noexcept;


    // \ref{pathdataitem.curveto.observers}, observers:
    vector_2d control_point_1() const noexcept;
    vector_2d control_point_2() const noexcept;
    vector_2d end_point() const noexcept;
    virtual path_data_type type() const noexcept override;
    
  private:
    vector_2d _Control_pt1; // \expos
    vector_2d _Control_pt2; // \expos
    vector_2d _End_pt;      // \expos
  };
} } } }
\end{codeblock}

\rSec1 [pathdataitem.curveto.cons] {\tcode{path_data_item::curve_to} constructors and assignment operators}

\indexlibrary{\idxcode{path_data_item::curve_to}!constructor}
\begin{itemdecl}
    curve_to() noexcept;
\end{itemdecl}
\begin{itemdescr}
	\pnum
	\effects
	Constructs an object of type \tcode{path_data_item::curve_to}.
	
	\pnum
	\postconditions
	\tcode{_Control_pt1 == vector_2d(0.0, 0.0)}.

	\tcode{_Control_pt2 == vector_2d(0.0, 0.0)}.

	\tcode{_End_pt == vector_2d(0.0, 0.0)}.
\end{itemdescr}

\indexlibrary{\idxcode{path_data_item::curve_to}!constructor}
\begin{itemdecl}
    curve_to(const vector_2d& controlPoint1, const vector_2d& controlPoint2,
      const vector_2d& endPoint) noexcept;
\end{itemdecl}
\begin{itemdescr}
	\pnum
	\effects
	Constructs an object of type \tcode{path_data_item::curve_to}.
	
	\pnum
	\postconditions
	\tcode{_Control_pt1 == controlPoint1}.

	\tcode{_Control_pt2 == controlPoint2}.

	\tcode{_End_pt == endPoint}.
\end{itemdescr}

\rSec1 [pathdataitem.curveto.modifiers]{\tcode{path_data_item::curve_to} modifiers}

\indexlibrary{\idxcode{path_data_item::curve_to}!\idxcode{control_point_1}}
\indexlibrary{\idxcode{control_point_1}!\idxcode{path_data_item::curve_to}}
\begin{itemdecl}
    void control_point_1(const vector_2d& value) noexcept;
\end{itemdecl}
\begin{itemdescr}
	\pnum
	\postconditions
	\tcode{_Control_pt_1 == value}.
\end{itemdescr}

\indexlibrary{\idxcode{path_data_item::curve_to}!\idxcode{control_point_2}}
\indexlibrary{\idxcode{control_point_2}!\idxcode{path_data_item::curve_to}}
\begin{itemdecl}
    void control_point_2(const vector_2d& value) noexcept;
\end{itemdecl}
\begin{itemdescr}
	\pnum
	\postconditions
	\tcode{_Control_pt_2 == value}.
\end{itemdescr}

\indexlibrary{\idxcode{path_data_item::curve_to}!\idxcode{end_point}}
\indexlibrary{\idxcode{end_point}!\idxcode{path_data_item::curve_to}}
\begin{itemdecl}
    void end_point(const vector_2d& value) noexcept;
\end{itemdecl}
\begin{itemdescr}
	\pnum
	\postconditions
	\tcode{_End_pt == value}.
\end{itemdescr}

\rSec1 [pathdataitem.curveto.observers]{\tcode{path_data_item::curve_to} observers}

\indexlibrary{\idxcode{path_data_item::curve_to}!\idxcode{control_point_1}}
\indexlibrary{\idxcode{control_point_1}!\idxcode{path_data_item::curve_to}}
\begin{itemdecl}
    vector_2d control_point_1() const noexcept;
\end{itemdecl}
\begin{itemdescr}
	\pnum
	\returns
	\tcode{_Control_pt_1}.
\end{itemdescr}

\indexlibrary{\idxcode{path_data_item::curve_to}!\idxcode{control_point_2}}
\indexlibrary{\idxcode{control_point_2}!\idxcode{path_data_item::curve_to}}
\begin{itemdecl}
    vector_2d control_point_2() const noexcept;
\end{itemdecl}
\begin{itemdescr}
	\pnum
	\returns
	\tcode{_Control_pt_2}.
\end{itemdescr}

\indexlibrary{\idxcode{path_data_item::curve_to}!\idxcode{end_point}}
\indexlibrary{\idxcode{end_point}!\idxcode{path_data_item::curve_to}}
\begin{itemdecl}
    vector_2d end_point() const noexcept;
\end{itemdecl}
\begin{itemdescr}
	\pnum
	\returns
	\tcode{_End_pt}.
\end{itemdescr}

\indexlibrary{\idxcode{path_data_item::curve_to}!\idxcode{type}}
\indexlibrary{\idxcode{type}!\idxcode{path_data_item::curve_to}}
\begin{itemdecl}
    virtual path_data_type type() const noexcept override;
\end{itemdecl}
\begin{itemdescr}
	\pnum
	\returns
	\tcode{path_data_type::curve_to}.
\end{itemdescr}

%!TEX root = io2d.tex
\rSec0 [\iotwod.lineto] {Class \tcode{line_to}}

\rSec1 [\iotwod.lineto.synopsis] {\tcode{line_to} synopsis}

\begin{codeblock}
namespace std { namespace experimental { namespace io2d { inline namespace v1 {
  class line_to : public path_data {
  public:
    // \ref{\iotwod.lineto.cons}, construct/copy/move/destroy:
    line_to() noexcept;
    line_to(const line_to& other) noexcept;
    line_to& operator=(const line_to& other) noexcept;
    line_to(line_to&& other) noexcept;
    line_to& operator=(line_to&& other) noexcept;
    line_to(const vector_2d& pt) noexcept;

    // \ref{\iotwod.lineto.modifiers}, modifiers:
    void to(const vector_2d& pt) noexcept;

    // \ref{\iotwod.lineto.observers}, observers:
    vector_2d to() const noexcept;
    virtual path_data_type type() const noexcept override;
    
  private:
    vector_2d _Data; // \expos
  };
} } } }
\end{codeblock}

\rSec1 [\iotwod.lineto.intro] {\tcode{line_to} Description}

\pnum
\indexlibrary{\idxcode{line_to}}
The class \tcode{line_to} describes a \tcode{path} operation. For a description of its meaning within a \tcode{path}, see the meaning of \tcode{path_data_type::line_to} in Table~\ref{tab:\iotwod.pathdatatype.meanings}.

\rSec1 [\iotwod.lineto.cons] {\tcode{line_to} constructors and assignment operators}

\indexlibrary{\idxcode{line_to}!constructor}
\begin{itemdecl}
    line_to() noexcept;
\end{itemdecl}
\begin{itemdescr}
	\pnum
	\effects
	Constructs an object of type \tcode{line_to}.
	
	\pnum
	\postconditions
	\tcode{_Data == vector_2d(0.0, 0.0)}.
\end{itemdescr}

\indexlibrary{\idxcode{line_to}!constructor}
\begin{itemdecl}
    line_to(const vector_2d& pt) noexcept;
\end{itemdecl}
\begin{itemdescr}
	\pnum
	\effects
	Constructs an object of type \tcode{line_to}.
	
	\pnum
	\postconditions
	\tcode{_Data == pt}.
\end{itemdescr}

\rSec1 [\iotwod.lineto.modifiers]{\tcode{line_to} modifiers}

\indexlibrary{\idxcode{line_to}!\idxcode{to}}
\indexlibrary{\idxcode{to}!\idxcode{line_to}}
\begin{itemdecl}
    void to(const vector_2d& pt) noexcept;
\end{itemdecl}
\begin{itemdescr}
	\pnum
	\postconditions
	\tcode{_Data == pt}.
	
\end{itemdescr}

\rSec1 [\iotwod.lineto.observers]{\tcode{line_to} observers}

\indexlibrary{\idxcode{line_to}!\idxcode{to}}
\indexlibrary{\idxcode{to}!\idxcode{line_to}}
\begin{itemdecl}
    vector_2d to() const noexcept;
\end{itemdecl}
\begin{itemdescr}
	\pnum
	\returns
	\tcode{_Data}.

\end{itemdescr}

\indexlibrary{\idxcode{line_to}!\idxcode{type}}
\indexlibrary{\idxcode{type}!\idxcode{line_to}}
\begin{itemdecl}
    virtual path_data_type type() const noexcept override;
\end{itemdecl}
\begin{itemdescr}
	\pnum
	\returns
	\tcode{path_data_type::line_to}.

\end{itemdescr}

%!TEX root = io2d.tex
\rSec0 [pathdataitem.moveto] {Class \tcode{path_factory::path_move_to}}

\pnum
\indexlibrary{\idxcode{path_factory::path_move_to}}
The class \tcode{path_factory::path_move_to} describes an operation on a path group.

\pnum
This operation starts a new path geometry and sets its current point and last-move-to point to the value of \tcode{*this.to()}.

\rSec1 [pathdataitem.moveto.synopsis] {\tcode{path_factory::path_move_to} synopsis}

\begin{codeblock}
namespace std { namespace experimental { namespace io2d { inline namespace v1 {
  class path_factory::path_move_to {
  public:
    // \ref{pathdataitem.moveto.cons}, construct/copy/move/destroy:
    move_to() noexcept;
    move_to(const move_to&) noexcept;
    path_factory::path_move_to& operator=(const move_to&) noexcept;
    move_to(move_to&&) noexcept;
    path_factory::path_move_to& operator=(move_to&&) noexcept;
    explicit move_to(const vector_2d& pt) noexcept;

    // \ref{pathdataitem.moveto.modifiers}, modifiers:
    void to(const vector_2d& pt) noexcept;

    // \ref{pathdataitem.moveto.observers}, observers:
    vector_2d to() const noexcept;
    virtual path_data_type type() const noexcept override;
    
  private:
    vector_2d _Data; // \expos
  };
} } } }
\end{codeblock}

\rSec1 [pathdataitem.moveto.cons] {\tcode{path_factory::path_move_to} constructors and assignment operators}

\indexlibrary{\idxcode{path_factory::path_move_to}!constructor}
\begin{itemdecl}
    move_to() noexcept;
\end{itemdecl}
\begin{itemdescr}
	\pnum
	\effects
	Constructs an object of type \tcode{path_factory::path_move_to}.
	
	\pnum
	\postconditions
	\tcode{_Data == vector_2d(0.0, 0.0)}.
\end{itemdescr}

\indexlibrary{\idxcode{path_factory::path_move_to}!constructor}
\begin{itemdecl}
    explicit move_to(const vector_2d& pt) noexcept;
\end{itemdecl}
\begin{itemdescr}
	\pnum
	\effects
	Constructs an object of type \tcode{path_factory::path_move_to}.
	
	\pnum
	\postconditions
	\tcode{_Data == pt}.
\end{itemdescr}

\rSec1 [pathdataitem.moveto.modifiers]{\tcode{path_factory::path_move_to} modifiers}

\indexlibrary{\idxcode{path_factory::path_move_to}!\idxcode{to}}
\indexlibrary{\idxcode{to}!\idxcode{path_factory::path_move_to}}
\begin{itemdecl}
    void to(const vector_2d& pt) noexcept;
\end{itemdecl}
\begin{itemdescr}
	\pnum
	\postconditions
	\tcode{_Data == pt}.
\end{itemdescr}

\rSec1 [pathdataitem.moveto.observers]{\tcode{path_factory::path_move_to} observers}

\indexlibrary{\idxcode{path_factory::path_move_to}!\idxcode{to}}
\indexlibrary{\idxcode{to}!\idxcode{path_factory::path_move_to}}
\begin{itemdecl}
    vector_2d to() const noexcept;
\end{itemdecl}
\begin{itemdescr}
	\pnum
	\returns
	\tcode{_Data}.
\end{itemdescr}

\indexlibrary{\idxcode{path_factory::path_move_to}!\idxcode{type}}
\indexlibrary{\idxcode{type}!\idxcode{path_factory::path_move_to}}
\begin{itemdecl}
    virtual path_data_type type() const noexcept override;
\end{itemdecl}
\begin{itemdescr}
	\pnum
	\returns
	\tcode{path_data_type::move_to}.
\end{itemdescr}

%!TEX root = io2d.tex
\rSec0 [pathdataitem.newpath] {Class \tcode{path_factory::path_new_path}}

\pnum
\indexlibrary{\idxcode{path_factory::path_new_path}}
The class \tcode{path_factory::path_new_path} describes an operation on a path group.

\pnum
This operation starts a new path geometry. The new path geometry has no current point.

\rSec1 [pathdataitem.newpath.synopsis] {\tcode{path_factory::path_new_path} synopsis}

\begin{codeblock}
namespace std { namespace experimental { namespace io2d { inline namespace v1 {
  class path_factory::path_new_path {
  public:
    // construct/copy/move/destroy:
    new_path() noexcept;
    new_path(const new_path&) noexcept;
    path_factory::path_new_path& operator=(const new_path&) noexcept;
    new_path(new_path&&) noexcept;
    path_factory::path_new_path& operator=(new_path&&) noexcept;

    // \ref{pathdataitem.newpath.observers}, observers:
    virtual path_data_type type() const noexcept override;
  };
} } } }
\end{codeblock}

\rSec1 [pathdataitem.newpath.observers]{\tcode{path_factory::path_new_path} observers}

\indexlibrary{\idxcode{path_factory::path_new_path}!\idxcode{type}}
\indexlibrary{\idxcode{type}!\idxcode{path_factory::path_new_path}}
\begin{itemdecl}
    virtual path_data_type type() const noexcept override;
\end{itemdecl}
\begin{itemdescr}
	\pnum
	\returns
	\tcode{path_data_type::new_path}.

\end{itemdescr}

%!TEX root = io2d.tex
\rSec0 [pathfactory.pathrelcurve] {Class \tcode{path_factory::path_rel_curve}}

\pnum
\indexlibrary{\idxcode{path_factory::path_rel_curve}}
The class \tcode{path_factory::path_rel_curve} describes a path segment that is a cubic \bezierlocal curve.

\pnum
It has a first control point of type \tcode{vector_2d}, a second control point of type \tcode{vector_2d}, and an end point of type \tcode{vector_2d}.

\pnum
All of its points are relative to the most recently established current point.

\rSec1 [pathfactory.pathrelcurve.synopsis] {\tcode{path_factory::path_rel_curve} synopsis}

\begin{codeblock}
namespace std { namespace experimental { namespace io2d { inline namespace v1 {
  class path_factory::path_rel_curve {
  public:
    // \ref{pathfactory.pathrelcurve.cons}, construct
    path_rel_curve(const vector_2d& cp1, const vector_2d& cp2,
      const vector_2d& ep) noexcept;

    // \ref{pathfactory.pathrelcurve.modifiers}, modifiers:
    void control_point_1(const vector_2d& cp) noexcept;
    void control_point_2(const vector_2d& cp) noexcept;
    void end_point(const vector_2d& ep) noexcept;

    // \ref{pathfactory.pathrelcurve.observers}, observers:
    vector_2d control_point_1() const noexcept;
    vector_2d control_point_2() const noexcept;
    vector_2d end_point() const noexcept;
  };
} } } }
\end{codeblock}

\rSec1 [pathfactory.pathrelcurve.cons] {\tcode{path_factory::path_rel_curve} constructors}
\indexlibrary{\idxcode{path_factory::path_rel_curve_to}!constructor}
\begin{itemdecl}
    path_rel_curve(const vector_2d& cp1, const vector_2d& cp2,
      const vector_2d& ep) noexcept;
\end{itemdecl}
\begin{itemdescr}
	\pnum
	\effects
	Constructs an object of type \tcode{path_factory::path_rel_curve}.
	
	\pnum
	The first control point shall be set to the value of \tcode{cp1}.
	
	\pnum
	The second control point shall be set to the value of \tcode{cp2}.
	
	\pnum
	The end point shall be set to the value of \tcode{ep}.
\end{itemdescr}

\rSec1 [pathfactory.pathrelcurve.modifiers]{\tcode{path_factory::path_rel_curve} modifiers}

\indexlibrary{\idxcode{path_factory::path_rel_curve}!\idxcode{control_point_1}}
\indexlibrary{\idxcode{control_point_1}!\idxcode{path_factory::path_rel_curve}}
\begin{itemdecl}
    void control_point_1(const vector_2d& cp) noexcept;
\end{itemdecl}
\begin{itemdescr}
	\pnum
	\effects
	The first control point shall be set to the value of \tcode{cp}.
\end{itemdescr}

\indexlibrary{\idxcode{path_factory::path_rel_curve}!\idxcode{control_point_2}}
\indexlibrary{\idxcode{control_point_2}!\idxcode{path_factory::path_rel_curve}}
\begin{itemdecl}
    void control_point_2(const vector_2d& value) noexcept;
\end{itemdecl}
\begin{itemdescr}
	\pnum
	\effects
	The second control point shall be set to the value of \tcode{cp}.
\end{itemdescr}

\indexlibrary{\idxcode{path_factory::path_rel_curve}!\idxcode{end_point}}
\indexlibrary{\idxcode{end_point}!\idxcode{path_factory::path_rel_curve}}
\begin{itemdecl}
    void end_point(const vector_2d& value) noexcept;
\end{itemdecl}
\begin{itemdescr}
	\pnum
	\effects
	The end point shall be set to the value of \tcode{ep}.
\end{itemdescr}

\rSec1 [pathfactory.pathrelcurve.observers]{\tcode{path_factory::path_rel_curve} observers}

\indexlibrary{\idxcode{path_factory::path_rel_curve}!\idxcode{control_point_1}}
\indexlibrary{\idxcode{control_point_1}!\idxcode{path_factory::path_rel_curve}}
\begin{itemdecl}
    vector_2d control_point_1() const noexcept;
\end{itemdecl}
\begin{itemdescr}
	\pnum
	\returns
	The value of the first control point.
\end{itemdescr}

\indexlibrary{\idxcode{path_factory::path_rel_curve}!\idxcode{control_point_2}}
\indexlibrary{\idxcode{control_point_2}!\idxcode{path_factory::path_rel_curve}}
\begin{itemdecl}
    vector_2d control_point_2() const noexcept;
\end{itemdecl}
\begin{itemdescr}
	\pnum
	\returns
	The value of the second control point.
\end{itemdescr}

\indexlibrary{\idxcode{path_factory::path_rel_curve}!\idxcode{end_point}}
\indexlibrary{\idxcode{end_point}!\idxcode{path_factory::path_rel_curve}}
\begin{itemdecl}
    vector_2d end_point() const noexcept;
\end{itemdecl}
\begin{itemdescr}
	\pnum
	\returns
	The value of the end point.
\end{itemdescr}

%!TEX root = io2d.tex
\rSec0 [pathdataitem.rellineto] {Class \tcode{path_factory::path_rel_line_to}}

\pnum
\indexlibrary{\idxcode{path_factory::path_rel_line_to}}
The class \tcode{path_factory::path_rel_line_to} describes an operation on a path geometry collection.

\pnum
This operation creates a line from the current point to the point that is the sum of the current point and the point returned by \tcode{*this.to()}. It then sets current point to be the sum of the current point and the point returned by \tcode{*this.to()}.

\pnum
If the current path geometry does not have a current point when this operation is requested the path geometry collection is malformed.

\rSec1 [pathdataitem.rellineto.synopsis] {\tcode{path_factory::path_rel_line_to} synopsis}

\begin{codeblock}
namespace std { namespace experimental { namespace io2d { inline namespace v1 {
  class path_factory::path_rel_line_to {
  public:
    // \ref{pathdataitem.rellineto.cons}, construct/copy/move/destroy:
    rel_line_to() noexcept;
    rel_line_to(const line_to&) noexcept;
    path_factory::path_rel_line_to& operator=(const line_to&) noexcept;
    rel_line_to(line_to&&) noexcept;
    path_factory::path_rel_line_to& operator=(line_to&&) noexcept;
    explicit rel_line_to(const vector_2d& pt) noexcept;

    // \ref{pathdataitem.rellineto.modifiers}, modifiers:
    void to(const vector_2d& pt) noexcept;

    // \ref{pathdataitem.rellineto.observers}, observers:
    vector_2d to() const noexcept;
    virtual path_data_type type() const noexcept override;
    
  private:
    vector_2d _Data; // \expos
  };
} } } }
\end{codeblock}

\rSec1 [pathdataitem.rellineto.cons] {\tcode{path_factory::path_rel_line_to} constructors and assignment operators}

\indexlibrary{\idxcode{path_factory::path_rel_line_to}!constructor}
\begin{itemdecl}
    rel_line_to() noexcept;
\end{itemdecl}
\begin{itemdescr}
	\pnum
	\effects
	Constructs an object of type \tcode{path_factory::path_rel_line_to}.
	
	\pnum
	\postconditions
	\tcode{_Data == vector_2d(0.0, 0.0)}.
\end{itemdescr}

\indexlibrary{\idxcode{path_factory::path_rel_line_to}!constructor}
\begin{itemdecl}
    explicit rel_line_to(const vector_2d& pt) noexcept;
\end{itemdecl}
\begin{itemdescr}
	\pnum
	\effects
	Constructs an object of type \tcode{path_factory::path_rel_line_to}.
	
	\pnum
	\postconditions
	\tcode{_Data == pt}.
\end{itemdescr}

\rSec1 [pathdataitem.rellineto.modifiers]{\tcode{path_factory::path_rel_line_to} modifiers}

\indexlibrary{\idxcode{path_factory::path_rel_line_to}!\idxcode{to}}
\indexlibrary{\idxcode{to}!\idxcode{path_factory::path_rel_line_to}}
\begin{itemdecl}
    void to(const vector_2d& pt) noexcept;
\end{itemdecl}
\begin{itemdescr}
	\pnum
	\postconditions
	\tcode{_Data == pt}.
\end{itemdescr}

\rSec1 [pathdataitem.rellineto.observers]{\tcode{path_factory::path_rel_line_to} observers}

\indexlibrary{\idxcode{path_factory::path_rel_line_to}!\idxcode{to}}
\indexlibrary{\idxcode{to}!\idxcode{path_factory::path_rel_line_to}}
\begin{itemdecl}
    vector_2d to() const noexcept;
\end{itemdecl}
\begin{itemdescr}
	\pnum
	\returns
	\tcode{_Data}.
\end{itemdescr}

\indexlibrary{\idxcode{path_factory::path_rel_line_to}!\idxcode{type}}
\indexlibrary{\idxcode{type}!\idxcode{path_factory::path_rel_line_to}}
\begin{itemdecl}
    virtual path_data_type type() const noexcept override;
\end{itemdecl}
\begin{itemdescr}
	\pnum
	\returns
	\tcode{path_data_type::rel_line_to}.
\end{itemdescr}

%!TEX root = io2d.tex
\rSec0 [pathdataitem.relmoveto] {Class \tcode{path_factory::path_rel_move_to}}

\pnum
\indexlibrary{\idxcode{path_factory::path_rel_move_to}}
The class \tcode{path_factory::path_rel_move_to} describes an operation on a path geometry collection.

\pnum
This operation starts a new path geometry and sets its current point and last-move-to point to the point that is the sum of the previous path geometry's current point and the point returned by \tcode{*this.to()}.

\pnum
If the existing path geometry does not have a current point when this operation is requested the path geometry collection is malformed.

\rSec1 [pathdataitem.relmoveto.synopsis] {\tcode{path_factory::path_rel_move_to} synopsis}

\begin{codeblock}
namespace std { namespace experimental { namespace io2d { inline namespace v1 {
  class path_factory::path_rel_move_to {
  public:
    // \ref{pathdataitem.relmoveto.cons}, construct/copy/move/destroy:
    rel_move_to() noexcept;
    rel_move_to(const rel_move_to&) noexcept;
    path_factory::path_rel_move_to& operator=(const rel_move_to&) noexcept;
    rel_move_to(rel_move_to&&) noexcept;
    path_factory::path_rel_move_to& operator=(rel_move_to&&) noexcept;
    explicit rel_move_to(const vector_2d& pt) noexcept;

    // \ref{pathdataitem.relmoveto.modifiers}, modifiers:
    void to(const vector_2d& pt) noexcept;

    // \ref{pathdataitem.relmoveto.observers}, observers:
    vector_2d to() const noexcept;
    virtual path_data_type type() const noexcept override;
    
  private:
    vector_2d _Data; // \expos
  };
} } } }
\end{codeblock}

\rSec1 [pathdataitem.relmoveto.cons] {\tcode{path_factory::path_rel_move_to} constructors and assignment operators}

\indexlibrary{\idxcode{path_factory::path_rel_move_to}!constructor}
\begin{itemdecl}
    rel_move_to() noexcept;
\end{itemdecl}
\begin{itemdescr}
	\pnum
	\effects
	Constructs an object of type \tcode{path_factory::path_rel_move_to}.
	
	\pnum
	\postconditions
	\tcode{_Data == vector_2d(0.0, 0.0)}.
\end{itemdescr}

\indexlibrary{\idxcode{path_factory::path_rel_move_to}!constructor}
\begin{itemdecl}
    explicit rel_move_to(const vector_2d& pt) noexcept;
\end{itemdecl}
\begin{itemdescr}
	\pnum
	\effects
	Constructs an object of type \tcode{path_factory::path_rel_move_to}.
	
	\pnum
	\postconditions
	\tcode{_Data == pt}.
\end{itemdescr}

\rSec1 [pathdataitem.relmoveto.modifiers]{\tcode{path_factory::path_rel_move_to} modifiers}

\indexlibrary{\idxcode{path_factory::path_rel_move_to}!\idxcode{to}}
\indexlibrary{\idxcode{to}!\idxcode{path_factory::path_rel_move_to}}
\begin{itemdecl}
    void to(const vector_2d& pt) noexcept;
\end{itemdecl}
\begin{itemdescr}
	\pnum
	\postconditions
	\tcode{_Data == pt}.
\end{itemdescr}

\rSec1 [pathdataitem.relmoveto.observers]{\tcode{path_factory::path_rel_move_to} observers}

\indexlibrary{\idxcode{path_factory::path_rel_move_to}!\idxcode{to}}
\indexlibrary{\idxcode{to}!\idxcode{path_factory::path_rel_move_to}}
\begin{itemdecl}
    vector_2d to() const noexcept;
\end{itemdecl}
\begin{itemdescr}
	\pnum
	\returns
	\tcode{_Data}.
\end{itemdescr}

\indexlibrary{\idxcode{path_factory::path_rel_move_to}!\idxcode{type}}
\indexlibrary{\idxcode{type}!\idxcode{path_factory::path_rel_move_to}}
\begin{itemdecl}
    virtual path_data_type type() const noexcept override;
\end{itemdecl}
\begin{itemdescr}
	\pnum
	\returns
	\tcode{path_data_type::rel_move_to}.
\end{itemdescr}

%%!TEX root = io2d.tex
\rSec0 [pathdataitem.get] {Class \tcode{path_data_item::get} member function template specializations}

\rSec1 [pathdataitem.get.synopsis] {\tcode{path_data_item::get} synopsis}

\begin{codeblock}
namespace std { namespace experimental { namespace io2d { inline namespace v1 {
  template <>
  path_data_item::arc path_data_item::get() const;
  template <>
  path_data_item::arc path_data_item::get(error_code& ec) const noexcept;
  
  template <>
  path_data_item::arc_negative path_data_item::get() const;
  template <>
  path_data_item::arc_negative path_data_item::get(error_code& ec) const 
    noexcept;
  
  template <>
  inline path_data_item::change_matrix path_data_item::get() const;
  template <>
  path_data_item::change_matrix path_data_item::get(error_code& ec) const 
    noexcept;
  
  template <>
  path_data_item::change_origin path_data_item::get() const;
  template <>
  path_data_item::change_origin path_data_item::get(error_code& ec) const 
    noexcept;
  
  template <>
  path_data_item::close_path path_data_item::get() const;
  template <>
  path_data_item::close_path path_data_item::get(error_code& ec) const noexcept;
  
  template <>
  path_data_item::curve_to path_data_item::get() const;
  template <>
  path_data_item::curve_to path_data_item::get(error_code& ec) const noexcept;

  template <>
  path_data_item::rel_curve_to path_data_item::get() const;
  template <>
  path_data_item::rel_curve_to path_data_item::get(error_code& ec) const 
    noexcept;
  
  template <>
  path_data_item::new_sub_path path_data_item::get() const;
  template <>
  path_data_item::new_sub_path path_data_item::get(error_code& ec) const 
    noexcept;

  template <>
  path_data_item::line_to path_data_item::get() const;
  template <>
  path_data_item::line_to path_data_item::get(error_code& ec) const noexcept;

  template <>
  path_data_item::move_to path_data_item::get() const;
  template <>
  path_data_item::move_to path_data_item::get(error_code& ec) const noexcept;

  template <>
  path_data_item::rel_line_to path_data_item::get() const;
  template <>
  path_data_item::rel_line_to path_data_item::get(error_code& ec) const 
    noexcept;

  template <>
  path_data_item::rel_move_to path_data_item::get() const;
  template <>
  path_data_item::rel_move_to path_data_item::get(error_code& ec) const 
    noexcept;
} } } }
\end{codeblock}

\rSec1 [pathdataitem.get.specializations]{\tcode{path_data_item::get} specializations}

\indexlibrary{\idxcode{path_data_item}!\idxcode{get}}
\indexlibrary{\idxcode{get}!\idxcode{path_data_item}}
\begin{itemdecl}
template <>
path_data_item::arc path_data_item::get() const;
template <>
path_data_item::arc path_data_item::get(error_code& ec) const noexcept;
\end{itemdecl}
\begin{itemdescr}
\pnum
\returns
A copy of the stored \tcode{path_data_item::arc} object.

\pnum
If an error occurs and the function was called with an \tcode{error_code\&} argument, returns \tcode{path_data_item::arc\{ \}}.

\pnum
\throws
As specified in Error reporting (\ref{\iotwod.err.report}).

\pnum
\errors
\tcode{errc::operation_not_permitted} if \tcode{!this->has_data()}.

\pnum
\tcode{errc::invalid_argument} if \tcode{this->type() != path_data_type::arc}.
\end{itemdescr}

\indexlibrary{\idxcode{path_data_item}!\idxcode{get}}
\indexlibrary{\idxcode{get}!\idxcode{path_data_item}}
\begin{itemdecl}
template <>
path_data_item::arc_negative path_data_item::get() const;
template <>
path_data_item::arc_negative path_data_item::get(error_code& ec) const noexcept;
\end{itemdecl}
\begin{itemdescr}
\pnum
\returns
A copy of the stored \tcode{path_data_item::arc_negative} object.

\pnum
If an error occurs and the function was called with an \tcode{error_code\&} argument, returns \tcode{path_data_item::arc_negative\{ \}}.

\pnum
\throws
As specified in Error reporting (\ref{\iotwod.err.report}).

\pnum
\errors
\tcode{errc::operation_not_permitted} if \tcode{!this->has_data()}.

\pnum
\tcode{errc::invalid_argument} if \tcode{this->type() != path_data_type::arc_negative}.
\end{itemdescr}

\indexlibrary{\idxcode{path_data_item}!\idxcode{get}}
\indexlibrary{\idxcode{get}!\idxcode{path_data_item}}
\begin{itemdecl}
template <>
inline path_data_item::change_matrix path_data_item::get() const;
template <>
path_data_item::change_matrix path_data_item::get(error_code& ec) const 
noexcept;
\end{itemdecl}
\begin{itemdescr}
\pnum
\returns
A copy of the stored \tcode{path_data_item::change_matrix} object.

\pnum
If an error occurs and the function was called with an \tcode{error_code\&} argument, returns \tcode{path_data_item::change_matrix\{ \}}.

\pnum
\throws
As specified in Error reporting (\ref{\iotwod.err.report}).

\pnum
\errors
\tcode{errc::operation_not_permitted} if \tcode{!this->has_data()}.

\pnum
\tcode{errc::invalid_argument} if \tcode{this->type() != path_data_type::change_matrix}.
\end{itemdescr}

\indexlibrary{\idxcode{path_data_item}!\idxcode{get}}
\indexlibrary{\idxcode{get}!\idxcode{path_data_item}}
\begin{itemdecl}
template <>
path_data_item::change_origin path_data_item::get() const;
template <>
path_data_item::change_origin path_data_item::get(error_code& ec) const 
  noexcept;
\end{itemdecl}
\begin{itemdescr}
\pnum
\returns
A copy of the stored \tcode{path_data_item::change_origin} object.

\pnum
If an error occurs and the function was called with an \tcode{error_code\&} argument, returns \tcode{path_data_item::change_origin\{ \}}.

\pnum
\throws
As specified in Error reporting (\ref{\iotwod.err.report}).

\pnum
\errors
\tcode{errc::operation_not_permitted} if \tcode{!this->has_data()}.

\pnum
\tcode{errc::invalid_argument} if \tcode{this->type() != path_data_type::change_origin}.
\end{itemdescr}

\indexlibrary{\idxcode{path_data_item}!\idxcode{get}}
\indexlibrary{\idxcode{get}!\idxcode{path_data_item}}
\begin{itemdecl}
template <>
path_data_item::close_path path_data_item::get() const;
template <>
path_data_item::close_path path_data_item::get(error_code& ec) const noexcept;
\end{itemdecl}
\begin{itemdescr}
\pnum
\returns
A copy of the stored \tcode{path_data_item::close_path} object.

\pnum
If an error occurs and the function was called with an \tcode{error_code\&} argument, returns \tcode{path_data_item::close_path\{ \}}.

\pnum
\throws
As specified in Error reporting (\ref{\iotwod.err.report}).

\pnum
\errors
\tcode{errc::operation_not_permitted} if \tcode{!this->has_data()}.

\pnum
\tcode{errc::invalid_argument} if \tcode{this->type() != path_data_type::close_path}.
\end{itemdescr}

\indexlibrary{\idxcode{path_data_item}!\idxcode{get}}
\indexlibrary{\idxcode{get}!\idxcode{path_data_item}}
\begin{itemdecl}
template <>
path_data_item::rel_curve_to path_data_item::get() const;
template <>
path_data_item::rel_curve_to path_data_item::get(error_code& ec) const noexcept;
\end{itemdecl}
\begin{itemdescr}
\pnum
\returns
A copy of the stored \tcode{path_data_item::rel_curve_to} object.

\pnum
If an error occurs and the function was called with an \tcode{error_code\&} argument, returns \tcode{path_data_item::rel_curve_to\{ \}}.

\pnum
\throws
As specified in Error reporting (\ref{\iotwod.err.report}).

\pnum
\errors
\tcode{errc::operation_not_permitted} if \tcode{!this->has_data()}.

\pnum
\tcode{errc::invalid_argument} if \tcode{this->type() != path_data_type::rel_curve_to}.
\end{itemdescr}

\indexlibrary{\idxcode{path_data_item}!\idxcode{get}}
\indexlibrary{\idxcode{get}!\idxcode{path_data_item}}
\begin{itemdecl}
template <>
path_data_item::new_sub_path path_data_item::get() const;
template <>
path_data_item::new_sub_path path_data_item::get(error_code& ec) const noexcept;
\end{itemdecl}
\begin{itemdescr}
\pnum
\returns
A copy of the stored \tcode{path_data_item::new_sub_path} object.

\pnum
If an error occurs and the function was called with an \tcode{error_code\&} argument, returns \tcode{path_data_item::new_sub_path\{ \}}.

\pnum
\throws
As specified in Error reporting (\ref{\iotwod.err.report}).

\pnum
\errors
\tcode{errc::operation_not_permitted} if \tcode{!this->has_data()}.

\pnum
\tcode{errc::invalid_argument} if \tcode{this->type() != path_data_type::new_sub_path}.
\end{itemdescr}

\indexlibrary{\idxcode{path_data_item}!\idxcode{get}}
\indexlibrary{\idxcode{get}!\idxcode{path_data_item}}
\begin{itemdecl}
template <>
path_data_item::line_to path_data_item::get() const;
template <>
path_data_item::line_to path_data_item::get(error_code& ec) const noexcept;
\end{itemdecl}
\begin{itemdescr}
\pnum
\returns
A copy of the stored \tcode{path_data_item::line_to} object.

\pnum
If an error occurs and the function was called with an \tcode{error_code\&} argument, returns \tcode{path_data_item::line_to\{ \}}.

\pnum
\throws
As specified in Error reporting (\ref{\iotwod.err.report}).

\pnum
\errors
\tcode{errc::operation_not_permitted} if \tcode{!this->has_data()}.

\pnum
\tcode{errc::invalid_argument} if \tcode{this->type() != path_data_type::line_to}.
\end{itemdescr}

\indexlibrary{\idxcode{path_data_item}!\idxcode{get}}
\indexlibrary{\idxcode{get}!\idxcode{path_data_item}}
\begin{itemdecl}
template <>
path_data_item::move_to path_data_item::get() const;
template <>
path_data_item::move_to path_data_item::get(error_code& ec) const noexcept;
\end{itemdecl}
\begin{itemdescr}
\pnum
\returns
A copy of the stored \tcode{path_data_item::move_to} object.

\pnum
If an error occurs and the function was called with an \tcode{error_code\&} argument, returns \tcode{path_data_item::move_to\{ \}}.

\pnum
\throws
As specified in Error reporting (\ref{\iotwod.err.report}).

\pnum
\errors
\tcode{errc::operation_not_permitted} if \tcode{!this->has_data()}.

\pnum
\tcode{errc::invalid_argument} if \tcode{this->type() != path_data_type::move_to}.
\end{itemdescr}

\indexlibrary{\idxcode{path_data_item}!\idxcode{get}}
\indexlibrary{\idxcode{get}!\idxcode{path_data_item}}
\begin{itemdecl}
template <>
path_data_item::rel_line_to path_data_item::get() const;
template <>
path_data_item::rel_line_to path_data_item::get(error_code& ec) const noexcept;
\end{itemdecl}
\begin{itemdescr}
\pnum
\returns
A copy of the stored \tcode{path_data_item::rel_line_to} object.

\pnum
If an error occurs and the function was called with an \tcode{error_code\&} argument, returns \tcode{path_data_item::rel_line_to\{ \}}.

\pnum
\throws
As specified in Error reporting (\ref{\iotwod.err.report}).

\pnum
\errors
\tcode{errc::operation_not_permitted} if \tcode{!this->has_data()}.

\pnum
\tcode{errc::invalid_argument} if \tcode{this->type() != path_data_type::rel_line_to}.
\end{itemdescr}

\indexlibrary{\idxcode{path_data_item}!\idxcode{get}}
\indexlibrary{\idxcode{get}!\idxcode{path_data_item}}
\begin{itemdecl}
template <>
path_data_item::rel_move_to path_data_item::get() const;
template <>
path_data_item::rel_move_to path_data_item::get(error_code& ec) const noexcept;
\end{itemdecl}
\begin{itemdescr}
\pnum
\returns
A copy of the stored \tcode{path_data_item::rel_move_to} object.

\pnum
If an error occurs and the function was called with an \tcode{error_code\&} argument, returns \tcode{path_data_item::rel_move_to\{ \}}.

\pnum
\throws
As specified in Error reporting (\ref{\iotwod.err.report}).

\pnum
\errors
\tcode{errc::operation_not_permitted} if \tcode{!this->has_data()}.

\pnum
\tcode{errc::invalid_argument} if \tcode{this->type() != path_data_type::rel_move_to}.
\end{itemdescr}

\addtocounter{SectionDepthBase}{-1}
\addtocounter{SectionDepthBase}{-1}
