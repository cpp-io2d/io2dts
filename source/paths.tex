%!TEX root = io2d.tex

\rSec0 [paths] {Paths}

\pnum
Paths define geometric objects which can be stroked (Table~\ref{tab:surface.rendering.operations}), filled, masked, and used to define or modify a Clip Area (Table~\ref{tab:surface.state.listing}).

\pnum
Paths are created using a \tcode{path_factory} object, which stores a path group. 

\pnum
Paths provide vector graphics functionality. As such they are particularly useful in situations where an application is intended to run on a variety of platforms whose output devices (\ref{displaysurface.intro}) span a large gamut of sizes, both in terms of measurement units and in terms of a horizontal and vertical pixel count, in that order. For example, a pixel count expressed as 1280x720 means that there are 1280 horizontal pixels per row of pixels and 720 vertical pixels per column of pixels for a total of 921600 pixels.
%
%\pnum
%For output devices, the measurement size of a pixel is determined by the physical size of the output device. Many output devices represent pixels as having the same horizontal and vertical measurement sizes. As such, when they display a rendered image which does not have the same horizontal to vertical pixel ratio as the output device, it 

\pnum
A path may contain degenerate path segments because of special rules, which are set forth below.

\pnum
A \tcode{path_group} object is an immutable resource wrapper containing a path group (\ref{pathgroup}). A \tcode{path_group} object is created from a \tcode{path_factory} object. It can also be default constructed, in which case the \tcode{path_group} object contains no paths.

\rSec1 [paths.processing] {Processing paths}

\pnum
This section is normative. It describes how to convert the path group of a properly formed \tcode{path_factory} object from a collection of \tcode{path_factory::path_data} objects to a collection of \tcode{path_factory::path_data} objects which have had their points transformed in accordance with the origin and transformation matrix of the \tcode{path_factory} object and any \tcode{path_factory::path_change_origin} and \tcode{path_factory::path_change_matrix} objects in the path group of the \tcode{path_factory} object.

\pnum
The following code shows how to properly process a \tcode{path_factory} object \tcode{p} and store the results into a \tcode{vector<path_factory::path_data>}:

\begin{codeblock}
  #include <cmath>
  #include <vector>
  #include <variant>
  #include <experimental/io2d>
  using namespace std;
  using namespace std::experimental::io2d;
  
  matrix_2d m;
  vector_2d origin;
  vector_2d currentPoint; // Tracks the untransformed current point.
  bool hasCurrentPoint = false;
  vector_2d closePoint;   // Tracks the transformed close point.
  vector<path_factory::path_data> v;
  
  for (auto val : p) {
    std::visit([&](auto&& item) {
      using T = std::remove_cv_t<std::remove_reference_t<decltype item>>;

      if constexpr(std::is_same_v<T, path_factory::path_abs_move>) {
        currentPoint = item.to();
        auto pt = m.transform_point(currentPoint - origin) + origin;
        hasCurrentPoint = true;
        v.emplace_back(in_place_type_t<path_factory::path_abs_move>, pt);
        closePoint = pt
      }
      else if constexpr(std::is_same_v<T, path_factory::path_abs_line>) {
        currentPoint = item.to();
        auto pt = m.transform_point(currentPoint - origin) + origin;
        if (hasCurrentPoint) {
          v.emplace_back(in_place_type_t<path_factory::path_abs_line>, pt);
        }
        else {
          v.emplace_back(in_place_type_t<path_factory::path_abs_move>, pt);
          v.emplace_back(in_place_type_t<path_factory::path_abs_line>, pt);
          hasCurrentPoint = true;
          closePoint = pt;
        }
      }
      else if constexpr(std::is_same_v<T, path_factory::path_abs_cubic_curve>) {
        auto pt1 = m.transform_point(item.control_point_1() - origin) + origin;
        auto pt2 = m.transform_point(item.control_point_2() - origin) + origin;
        auto pt3 = m.transform_point(item.end_point() - origin) + origin;
        if (!hasCurrentPoint) {
          currentPoint = item.control_point_1();
          v.emplace_back(in_place_type_t<path_factory::path_abs_move>, pt1);
          hasCurrentPoint = true;
          closePoint = pt1;
        }
        v.emplace_back(in_place_type_t<path_factory::path_abs_cubic_curve>, pt1,
          pt2, pt3);
        currentPoint = item.end_point();
      }
      else if constexpr(std::is_same_v<T,
        path_factory::path_abs_quadratic_curve>) {
      	auto pt1 = m.transform_point(item.control_point() - origin) + origin;
      	auto pt2 = m.transform_point(item.end_point() - origin) + origin;
      	if (!hasCurrentPoint) {
      		currentPoint = item.control_point();
      		v.emplace_back(in_place_type_t<path_factory::path_abs_move>, pt1);
      		hasCurrentPoint = true;
      		closePoint = pt1;
      	}
      	v.emplace_back(in_place_type_t<path_factory::path_abs_quadratic_curve>,
      	  pt1, pt2);
      	currentPoint = item.end_point();
      }
      else if constexpr(std::is_same_v<T, path_factory::path_new_path>) {
        hasCurrentPoint = false;
        v.emplace_back(in_place_type_t<path_factory::path_new_path>);
      }
      else if constexpr(std::is_same_v<T, path_factory::path_close_path>) {
        if (hasCurrentPoint) {
          v.emplace_back(in_place_type_t<path_factory::path_close_path>);
          v.emplace_back(in_place_type_t<path_factory::path_abs_move>,
            closePoint);
          auto invM = matrix_2d{m}.invert();
          // Need to assign the untransformed closePoint value to currentPoint.
          currentPoint = invM.transform_point(closePoint - origin) + origin;
        }
      }
      else if constexpr(std::is_same_v<T, path_factory::path_rel_move>) {
        currentPoint = item.to() + currentPoint;
        auto pt = m.transform_point(currentPoint - origin) + origin;
        v.emplace_back(in_place_type_t<path_factory::path_abs_move>, pt);
        hasCurrentPoint = true;
        closePoint = pt    
        n.close_point(pt);
      }
      else if constexpr(std::is_same_v<T, path_factory::path_rel_line>) {
        currentPoint = item.to() + currentPoint;
        auto pt = m.transform_point(currentPoint - origin) + origin;
        v.emplace_back(in_place_type_t<path_factory::path_abs_line>, pt);
      }
      else if constexpr(std::is_same_v<T, path_factory::path_rel_cubic_curve>) {
        auto pt1 = m.transform_point(item.control_point_1() + currentPoint -
        origin) + origin;
        auto pt2 = m.transform_point(item.control_point_2() + currentPoint -
        origin) + origin;
        auto pt3 = m.transform_point(item.end_point() + currentPoint - origin) +
          origin;
        v.emplace_back(in_place_type_t<path_factory::path_abs_cubic_curve>,
          pt1, pt2, pt3);
        currentPoint = item.end_point() + currentPoint;
      }
      else if constexpr(std::is_same_v<T,
      path_factory::path_rel_quadratic_curve>) {
      	auto pt1 = m.transform_point(item.control_point() + currentPoint -
      	  origin) + origin;
      	auto pt2 = m.transform_point(item.end_point() + currentPoint -
      	  origin) + origin;
      	v.emplace_back(in_place_type_t<path_factory::path_rel_quadratic_curve>,
      	pt1, pt2);
      	currentPoint = item.end_point() + currentPoint;
      }
      else if constexpr(std::is_same_v<T, path_factory::path_arc_clockwise>) {
        auto ctr = item.center();
        auto rad = item.radius();
        auto ang1 = item.angle_1();
        auto ang2 = item.angle_2();
        while(ang2 < ang1) {
          ang2 += two_pi<double>;
        }
        vector_2d pt0, pt1, pt2, pt3;
        int bezCount = 1;
        double theta = ang2 - ang1;
        double phi;
        while (theta >= halfpi) {
          theta /= 2.0;
          bezCount += bezCount;
        }
        phi = theta / 2.0;
        auto cosPhi = cos(phi);
        auto sinPhi = sin(phi);
        pt0.x(cosPhi);
        pt0.y(-sinPhi);
        pt3.x(pt0.x());
        pt3.y(-pt0.y());
        pt1.x((4.0 - cosPhi) / 3.0);
        pt1.y(-(((1.0 - cosPhi) * (3.0 - cosPhi)) / (3.0 * sinPhi)));
        pt2.x(pt1.x());
        pt2.y(-pt1.y());
        phi = -phi;
        auto rotCwFn = [](const vector_2d& pt, double a) -> vector_2d {
          return { pt.x() * cos(a) + pt.y() * sin(a),
            -(pt.x() * -(sin(a)) + pt.y() * cos(a)) };
        };
        pt0 = rotCwFn(pt0, phi);
        pt1 = rotCwFn(pt1, phi);
        pt2 = rotCwFn(pt2, phi);
        pt3 = rotCwFn(pt3, phi);
        
        auto currTheta = ang1;
        const auto startPt =
        ctr + rotCwFn({ pt0.x() * rad, pt0.y() * rad }, currTheta);
        if (hasCurrentPoint) {
          currentPoint = startPt;
          auto pt = m.transform_point(currentPoint - origin) + origin;
          v.emplace_back(in_place_type_t<path_factory::path_abs_line>, pt);
        }
        else {
          currentPoint = startPt;
          auto pt = m.transform_point(currentPoint - origin) + origin;
          v.emplace_back(in_place_type_t<path_factory::path_abs_move>, pt);
          hasCurrentPoint = true;
          closePt = pt;
        }
        for (; bezCount > 0; bezCount--) {
          auto cpt1 = ctr + rotCwFn({ pt1.x() * rad, pt1.y() * rad }, currTheta);
          auto cpt2 = ctr + rotCwFn({ pt2.x() * rad, pt2.y() * rad },
            currTheta);
          auto cpt3 = ctr + rotCwFn({ pt3.x() * rad, pt3.y() * rad },
            currTheta);
          currentPoint = cpt3;
          cpt1 = m.transform_point(cpt1 - origin) + origin;
          cpt2 = m.transform_point(cpt2 - origin) + origin;
          cpt3 = m.transform_point(cpt3 - origin) + origin;
          v.emplace_back(in_place_type_t<path_factory::path_abs_cubic_curve>, cpt1,
            cpt2, cpt3);
          currTheta += theta;
        }
      }
      else if constexpr(std::is_same_v<T, path_factory::path_arc_counterclockwise>) {
      {
        auto ctr = item.center();
        auto rad = item.radius();
        auto ang1 = item.angle_1();
        auto ang2 = item.angle_2();
        while(ang2 > ang1) {
          ang2 -= two_pi<double>;
        }
        vector_2d pt0, pt1, pt2, pt3;
        int bezCount = 1;
        double theta = ang1 - ang2;
        double phi;
        while (theta >= halfpi) {
          theta /= 2.0;
          bezCount += bezCount;
        }
        phi = theta / 2.0;
        auto cosPhi = cos(phi);
        auto sinPhi = sin(phi);
        pt0.x(cosPhi);
        pt0.y(-sinPhi);
        pt3.x(pt0.x());
        pt3.y(-pt0.y());
        pt1.x((4.0 - cosPhi) / 3.0);
        pt1.y(-(((1.0 - cosPhi) * (3.0 - cosPhi)) / (3.0 * sinPhi)));
        pt2.x(pt1.x());
        pt2.y(-pt1.y());
        auto rotCwFn = [](const vector_2d& pt, double a) -> vector_2d {
          return { pt.x() * cos(a) + pt.y() * sin(a),
            -(pt.x() * -(sin(a)) + pt.y() * cos(a)) };
        };
        pt0 = rotCwFn(pt0, phi);
        pt1 = rotCwFn(pt1, phi);
        pt2 = rotCwFn(pt2, phi);
        pt3 = rotCwFn(pt3, phi);
        auto shflPt = pt3;
        pt3 = pt0;
        pt0 = shflPt;
        shflPt = pt2;
        pt2 = pt1;
        pt1 = shflPt;
        auto currTheta = ang1;
        const auto startPt =
          ctr + rotCwFn({ pt0.x() * rad, pt0.y() * rad }, currTheta);
        if (hasCurrentPoint) {
          currentPoint = startPt;
          auto pt = m.transform_point(currentPoint - origin) + origin;
          v.emplace_back(in_place_type_t<path_factory::path_abs_line>, pt);
        }
        else {
          currentPoint = startPt;
          auto pt = m.transform_point(currentPoint - origin) + origin;
          v.emplace_back(in_place_type_t<path_factory::path_abs_move>, pt);
          hasCurrentPoint = true;
          closePt = pt;
        }
        for (; bezCount > 0; bezCount--) {
          auto cpt1 = ctr + rotCwFn({ pt1.x() * rad, pt1.y() * rad },
            currTheta);
          auto cpt2 = ctr + rotCwFn({ pt2.x() * rad, pt2.y() * rad },
            currTheta);
          auto cpt3 = ctr + rotCwFn({ pt3.x() * rad, pt3.y() * rad },
            currTheta);
          currentPoint = cpt3;
          cpt1 = m.transform_point(cpt1 - origin) + origin;
          cpt2 = m.transform_point(cpt2 - origin) + origin;
          cpt3 = m.transform_point(cpt3 - origin) + origin;
          v.emplace_back(in_place_type_t<path_factory::path_cubic_curve_to>, cpt1,
            cpt2, cpt3);
          currTheta -= theta;
        }
      }
      else if constexpr(std::is_same_v<T, path_factory::path_change_matrix>) {
        m = item.matrix();
      }
      else if constexpr(std::is_same_v<T, path_factory::path_change_origin>) {
        origin = item.origin();
      }
    }, val);
  }
\end{codeblock}

%\rSec1 [paths.strokerules] {Stroking paths}
%
%\pnum
%The following rules shall apply when a Stroking operation (\ref{surface.stroking}) is carried out on a path.
%
%\begin{enumerate}
%\item If the path only contains a degenerate path segment, then if the \tcode{line_cap} value is
%\end{enumerate}
%
%\begin{enumerate}
%  \item Except as otherwise specified in these rules, the start point and end point of a path segment shall be rendered as specified by the meaning of the surface's current \tcode{line_cap} value (\ref{linecap}).
%  
%  \item If the end point of a path segment \textit{a} is set as the current point and is then used as the start point of another path segment, \textit{b}, the point where \tcode{a}'s end point meets \tcode{b}'s start point shall be rendered as specified by the meaning of the surface's current \tcode{line_join} value (\ref{linejoin}).
%  
%  \item ***FIXME***
%\end{enumerate}
%
%\rSec1 [paths.fillrules] {Filling paths}
%
%\pnum

\addtocounter{SectionDepthBase}{1}
%!TEX root = io2d.tex
\rSec0 [path] {Class \tcode{path}}
%%%%% Rename path to path_group so that a path group contains paths rather than path geometries. Rework all working accordingly and eliminate "sub path" since it is now just "path".
\pnum
\indexlibrary{\idxcode{path}}
The class \tcode{path} contains a path geometry graphics resource that is usable with a \tcode{surface}-derived object.

\pnum
A \tcode{path} object is constructed from the path geometry collection data of a \tcode{path_factory} object. The path geometries of its path geometry graphics resource are immutable, however its path geometry graphics resource can be changed using copy assignment or move assignment.

\pnum
An \tcode{path} object can be default constructed. Default construction of a \tcode{path} object results in a \tcode{path} object which has a path geometry graphics resource that contains no path geometries.

\pnum
When a \tcode{path} object is set on a \tcode{surface} object using 
\tcode{surface::path}, the geometric paths represented by it can be 
stroked or filled.

%\pnum
%A \tcode{path} object shall be usable with any \tcode{surface} or \tcode{surface}-derived object.
%
\rSec1 [path.synopsis] {\tcode{path} synopsis}

\begin{codeblock}
namespace std { namespace experimental { namespace io2d { inline namespace v1 {
  class path {
    public:
    // \ref{path.cons}, construct/copy/destroy:
    explicit path(const path_factory& pb);
    path(const path_factory& pb, error_code& ec) noexcept;
  };
} } } }
\end{codeblock}

\rSec1 [path.cons] {\tcode{path} constructors and assignment operators}

\indexlibrary{\idxcode{path}!constructor}
\begin{itemdecl}
    explicit path(const path_factory& pb);
    path(const path_factory& pb, error_code& ec) noexcept;
\end{itemdecl}
\begin{itemdescr}
	\pnum
	\effects
	Constructs an object of class \tcode{path}. Implementations shall create a path geometry graphics resource from the path geometries contained in \tcode{pb.data_ref()} as if they followed the procedure set forth in \ref{pathgeometries.processing}.

	\pnum
	\throws
	As specified in Error reporting (\ref{\iotwod.err.report}).

	\pnum
	\remarks
	It is unspecified whether a \tcode{path} object shall require further processing when it is passed as an argument to a \tcode{surface} or \tcode{surface}-derived object.
	
	\pnum
	Implementations should avoid or minimize the need for further processing of a \tcode{path} object after it has been constructed.

	\pnum
	\errors
	\tcode{errc::not_enough_memory} if there was a failure to allocate memory.
	
%	\pnum
%	\tcode{io2d_error::no_current_point} if, when processing the path geometries, an operation was encountered which required a current point and the current path geometry had no current point.
%	
%	\pnum
%	\tcode{io2d_error::invalid_matrix} if, when processing the path geometries, an operation was encountered which required the current transformation matrix to be invertible and the matrix was not invertible.
	
\end{itemdescr}

%!TEX root = io2d.tex
\rSec0 [pathfactory] {Class \tcode{path_factory}}

\rSec1 [pathfactory.synopsis] {\tcode{path_factory} synopsis}

\begin{codeblock}
namespace std { namespace experimental { namespace io2d { inline namespace v1 {
  class path_factory {
    // \ref{pathfactory.cons}, construct/copy/destroy:
    path_factory() noexcept;
    path_factory(const path_factory& other);
    path_factory& operator=(const path_factory& other);
    path_factory(path_factory&& other) noexcept;
    path_factory& operator=(path_factory&& other) noexcept;
    
    // \ref{pathfactory.modifiers}, modifiers:
    void append(const path_factory& p);
    void append(const path_factory& p, error_code& ec) noexcept;
    void append(const vector<path_data_item>& p);
    void append(const vector<path_data_item>& p, error_code& ec) noexcept;
    void new_sub_path();
    void new_sub_path(error_code& ec) noexcept;
    void close_path();
    void close_path(error_code& ec) noexcept;
    void arc(const vector_2d& center, double radius, double angle1,
      double angle2);
    void arc(const vector_2d& center, double radius, double angle1,
      double angle2, error_code& ec) noexcept;
    void arc_negative(const vector_2d& center, double radius, double angle1,
      double angle2);
    void arc_negative(const vector_2d& center, double radius, double angle1,
      double angle2, error_code& ec) noexcept;
    void curve_to(const vector_2d& pt0, const vector_2d& pt1,
      const vector_2d& pt2);
    void curve_to(const vector_2d& pt0, const vector_2d& pt1,
      const vector_2d& pt2, error_code& ec) noexcept;
    void line_to(const vector_2d& pt);
    void line_to(const vector_2d& pt, error_code& ec) noexcept;
    void move_to(const vector_2d& pt);
    void move_to(const vector_2d& pt, error_code& ec) noexcept;
    void rectangle(const std::experimental::io2d::rectangle& r);
    void rectangle(const std::experimental::io2d::rectangle& r,
      error_code& ec) noexcept;
    void rel_curve_to(const vector_2d& dpt0, const vector_2d& dpt1,
      const vector_2d& dpt2);
    void rel_curve_to(const vector_2d& dpt0, const vector_2d& dpt1,
      const vector_2d& dpt2, error_code& ec) noexcept;
    void rel_line_to(const vector_2d& dpt);
    void rel_line_to(const vector_2d& dpt, error_code& ec) noexcept;
    void rel_move_to(const vector_2d& dpt);
    void rel_move_to(const vector_2d& dpt, error_code& ec) noexcept;
    void transform_matrix(const matrix_2d& m);
    void transform_matrix(const matrix_2d& m, error_code& ec) noexcept;
    void origin(const vector_2d& pt);
    void origin(const vector_2d& pt, error_code& ec) noexcept;
    void clear() noexcept;
    
    // \ref{pathfactory.observers}, observers:
    std::experimental::io2d::rectangle path_extents() const;
    std::experimental::io2d::rectangle path_extents(error_code& ec) const noexcept;
    bool has_current_point() const noexcept;
    vector_2d current_point() const;
    vector_2d current_point(error_code& ec) const noexcept;
    matrix_2d transform_matrix() const noexcept;
    vector_2d origin() const noexcept;
    vector<path_data_item> data() const;
    vector<path_data_item> data(error_code& ec) const noexcept;
    path_data_item data_item(unsigned int index) const;
    path_data_item data_item(unsigned int index, error_code& ec) const noexcept;
    const vector<path_data_item>& data_ref() const noexcept;

  private:
    vector<path_data_item> _Data;  // \expos
    bool _Has_current_point;       // \expos
    vector_2d _Current_point;      // \expos
    vector_2d _Last_move_to_point; // \expos
    matrix_2d _Transform_matrix;   // \expos
    vector_2d _Origin;             // \expos
  };
} } } }
\end{codeblock}

\rSec1 [pathfactory.intro] {\tcode{path_factory} Description}

\pnum
\indexlibrary{\idxcode{path_factory}}
The \tcode{path_factory} class is a factory class used in creating path geometry collection data from which \tcode{path} objects are created.

\rSec1 [pathfactory.cons] {\tcode{path_factory} constructors and 
assignment operators}

\indexlibrary{\idxcode{path_factory}!constructor}
\begin{itemdecl}
    path_factory();
\end{itemdecl}
\begin{itemdescr}
	\pnum
	\effects
	Constructs an object of type \tcode{path_factory}.
	
	\pnum
	\postconditions
	\tcode{_Data.empty() == true}.
	
	\pnum
	\tcode{_Has_current_point == false}.
	
	\pnum
	\tcode{_Transform_matrix == matrix_2d::init_identity()}.
	
	\pnum
	\tcode{_Origin == vector_2d{ }}.
	
\end{itemdescr}

\rSec1 [pathfactory.modifiers] {\tcode{path_factory} modifiers}

\indexlibrary{\idxcode{path_factory}!\idxcode{}}
\indexlibrary{\idxcode{}!\idxcode{path_factory}}
\begin{itemdecl}
    void append(const path_factory& p);
    void append(const path_factory& p, error_code& ec) noexcept;
\end{itemdecl}
\begin{itemdescr}
	\pnum
	\postconditions
	
\end{itemdescr}

\indexlibrary{\idxcode{path_factory}!\idxcode{}}
\indexlibrary{\idxcode{}!\idxcode{path_factory}}
\begin{itemdecl}
    void append(const vector<path_data_item>& p);
    void append(const vector<path_data_item>& p, error_code& ec) noexcept;
\end{itemdecl}
\begin{itemdescr}
	\pnum
	\postconditions
	
\end{itemdescr}

\indexlibrary{\idxcode{path_factory}!\idxcode{new_sub_path}}
\indexlibrary{\idxcode{new_sub_path}!\idxcode{path_factory}}
\begin{itemdecl}
    void new_sub_path();
    void new_sub_path(error_code& ec) noexcept;
\end{itemdecl}
\begin{itemdescr}
	\pnum
	\effects
	\tcode{_Data.emplace_back(std::experimental::io2d::new_sub_path())}.
	
	\pnum
	\tcode{_Has_current_point = false}.
	
	\pnum
	\throws
	As specified in Error reporting (\ref{\iotwod.err.report}).

	\pnum
	\remarks
	In the event of an error, the object shall not be modified.

	\pnum
	\errors
	\tcode{errc::not_enough_memory} if the attempt to add the \tcode{path_data_item} failed.
	
\end{itemdescr}

\indexlibrary{\idxcode{path_factory}!\idxcode{close_path}}
\indexlibrary{\idxcode{close_path}!\idxcode{path_factory}}
\begin{itemdecl}
    void close_path();
    void close_path(error_code& ec) noexcept;
\end{itemdecl}
\begin{itemdescr}
	\pnum
	\effects
	If \tcode{_Has_current_point == true}:
	\begin{itemize}
	\item \tcode{_Data.emplace_back(std::experimental::io2d::close_path())}.
	
	\item \tcode{_Current_point = _Last_move_to_point}.
	\end{itemize}
	
	\pnum
	\throws
	As specified in Error reporting (\ref{\iotwod.err.report}).

	\pnum
	\remarks
	In the event of an error, the object shall not be modified.

	\pnum
	\errors
	\tcode{errc::not_enough_memory} if the attempt to add the \tcode{path_data_item} failed.
	
\end{itemdescr}

\indexlibrary{\idxcode{path_factory}!\idxcode{arc}}
\indexlibrary{\idxcode{arc}!\idxcode{path_factory}}
\begin{itemdecl}
    void arc(const vector_2d& center, double radius, double angle1,
      double angle2);
    void arc(const vector_2d& center, double radius, double angle1,
      double angle2, error_code& ec) noexcept;
\end{itemdecl}
\begin{itemdescr}
	\pnum
	\effects
	\tcode{_Data.emplace_back(std::experimental::io2d::arc(center, radius, angle1, angle2))}.
	
	\pnum
	\tcode{_Current_point == vector_2\{ radius * cos(angle2), -(radius * -sin(angle2)) \} + center}.
	
	\pnum
	If \tcode{_Has_current_point == false}:
	\begin{itemize}
	\item \tcode{_Last_move_to_point == vector_2\{ radius * cos(angle1), -(radius * -sin(angle1)) \} + center}.
	
	\item \tcode{_Has_current_point == true}.
	\end{itemize}
	
	\pnum
	\throws
	As specified in Error reporting (\ref{\iotwod.err.report}).

	\pnum
	\remarks
	In the event of an error, the object shall not be modified.

	\pnum
	\errors
	\tcode{errc::not_enough_memory} if the attempt to add the \tcode{path_data_item} failed.
	
\end{itemdescr}

\indexlibrary{\idxcode{path_factory}!\idxcode{arc_negative}}
\indexlibrary{\idxcode{arc_negative}!\idxcode{path_factory}}
\begin{itemdecl}
    void arc_negative(const vector_2d& center, double radius, double angle1,
      double angle2);
    void arc_negative(const vector_2d& center, double radius, double angle1,
      double angle2, error_code& ec) noexcept;
\end{itemdecl}
\begin{itemdescr}
	\pnum
	\effects
	\tcode{_Data.emplace_back(std::experimental::io2d::arc_negative(center, radius, angle1, angle2))}.
	
	\pnum
	\tcode{_Current_point = vector_2\{ radius * cos(angle1), radius * -sin(angle1) \} + center}.
	
	\pnum
	If \tcode{_Has_current_point == false}:
	\begin{itemize}
	\item \tcode{_Last_move_to_point = vector_2\{ radius * cos(angle2), radius * -sin(angle2) \} + center}.
	
	\item \tcode{_Has_current_point = true}.
	\end{itemize}
	
	\pnum
	\throws
	As specified in Error reporting (\ref{\iotwod.err.report}).

	\pnum
	\remarks
	In the event of an error, the object shall not be modified.

	\pnum
	\errors
	\tcode{errc::not_enough_memory} if the attempt to add the \tcode{path_data_item} failed.
	
\end{itemdescr}

\indexlibrary{\idxcode{path_factory}!\idxcode{curve_to}}
\indexlibrary{\idxcode{curve_to}!\idxcode{path_factory}}
\begin{itemdecl}
    void curve_to(const vector_2d& pt0, const vector_2d& pt1,
      const vector_2d& pt2);
    void curve_to(const vector_2d& pt0, const vector_2d& pt1,
      const vector_2d& pt2, error_code& ec) noexcept;
\end{itemdecl}
\begin{itemdescr}
	\pnum
	\effects
	If \tcode{_Has_current_point == false}:
	\begin{itemize}
	\item \tcode{_Data.reserve(_Data.size() + 2U)}.
	
	\item \tcode{*this.move_to(pt0)}.
	\end{itemize}
	
	\pnum
	\tcode{_Data.emplace_back(std::experimental::io2d::curve_to(pt0, pt1, pt2))}.
	
	\pnum
	\throws
	As specified in Error reporting (\ref{\iotwod.err.report}).

	\pnum
	\remarks
	In the event of an error, the object shall not be modified.

	\pnum
	\errors
	\tcode{errc::not_enough_memory} if the attempt to add the \tcode{path_data_item} failed.
	
\end{itemdescr}

\indexlibrary{\idxcode{path_factory}!\idxcode{line_to}}
\indexlibrary{\idxcode{line_to}!\idxcode{path_factory}}
\begin{itemdecl}
    void line_to(const vector_2d& pt);
    void line_to(const vector_2d& pt, error_code& ec) noexcept;
\end{itemdecl}
\begin{itemdescr}
	\pnum
	\effects
	\tcode{_Data.emplace_back(std::experimental::io2d::line_to(pt))}.
	
	\pnum
	If \tcode{_Has_current_point == false}:
	\begin{itemize}
	\item \tcode{_Last_move_to_point = pt}.
	
	\item \tcode{_Has_current_point = true}.
	\end{itemize}
	
	\pnum
	\tcode{_Current_point = pt}.
	
	\pnum
	\throws
	As specified in Error reporting (\ref{\iotwod.err.report}).

	\pnum
	\remarks
	In the event of an error, the object shall not be modified.

	\pnum
	\errors
	\tcode{errc::not_enough_memory} if the attempt to add the \tcode{path_data_item} failed.
	
\end{itemdescr}

\indexlibrary{\idxcode{path_factory}!\idxcode{move_to}}
\indexlibrary{\idxcode{move_to}!\idxcode{path_factory}}
\begin{itemdecl}
    void move_to(const vector_2d& pt);
    void move_to(const vector_2d& pt, error_code& ec) noexcept;
\end{itemdecl}
\begin{itemdescr}
	\pnum
	\effects
	\tcode{_Data.emplace_back(std::experimental::io2d::move_to(pt))}.
	
	\pnum
	\tcode{_Has_current_point = true}.
	
	\pnum
	\tcode{_Current_point = pt}.
	
	\pnum
	\throws
	As specified in Error reporting (\ref{\iotwod.err.report}).

	\pnum
	\remarks
	In the event of an error, the object shall not be modified.

	\pnum
	\errors
	\tcode{errc::not_enough_memory} if the attempt to add the \tcode{path_data_item} failed.
	
\end{itemdescr}

\indexlibrary{\idxcode{path_factory}!\idxcode{rectangle}}
\indexlibrary{\idxcode{rectangle}!\idxcode{path_factory}}
\begin{itemdecl}
    void rectangle(const std::experimental::io2d::rectangle& r);
    void rectangle(const std::experimental::io2d::rectangle& r,
      error_code& ec) noexcept;
\end{itemdecl}
\begin{itemdescr}
	\pnum
	\effects
	\begin{enumerate}
	\item \tcode{_Data.reserve(_Data.size() + 5U)}.

	\item \tcode{*this.move_to(\{ r.x(), r.y() \})}.
	
	\item \tcode{*this.rel_line_to(\{ r.width(), 0.0 \})}.
	
	\item \tcode{*this.rel_line_to(\{ 0.0, r.height() \})}.
	
	\item \tcode{*this.rel_line_to(\{ -r.width(), 0.0 \})}.
	
	\item \tcode{*this.close_path()}.
	\end{enumerate}
	
	\pnum
	\throws
	As specified in Error reporting (\ref{\iotwod.err.report}).

	\pnum
	\remarks
	In the event of an error, the object shall not be modified.

	\pnum
	\errors
	\tcode{errc::not_enough_memory} if the attempt to add the \tcode{path_data_item} failed.
	
\end{itemdescr}

\indexlibrary{\idxcode{path_factory}!\idxcode{rel_curve_to}}
\indexlibrary{\idxcode{rel_curve_to}!\idxcode{path_factory}}
\begin{itemdecl}
    void rel_curve_to(const vector_2d& dpt0, const vector_2d& dpt1,
      const vector_2d& dpt2);
    void rel_curve_to(const vector_2d& dpt0, const vector_2d& dpt1,
      const vector_2d& dpt2, error_code& ec) noexcept;
\end{itemdecl}
\begin{itemdescr}
	\pnum
	\preconditions
	\tcode{_Has_current_point == true}.

	\pnum
	\effects
	\tcode{_Data.emplace_back(std::experimental::io2d::rel_curve_to(dpt0, dpt1, dpt2))}.
	
	\pnum
	\tcode{_Current_point = dpt2 + _Current_point}.
	
	\pnum
	\throws
	As specified in Error reporting (\ref{\iotwod.err.report}).

	\pnum
	\remarks
	In the event of an error, the object shall not be modified.

	\pnum
	\errors
	\tcode{errc::not_enough_memory} if the attempt to add the \tcode{path_data_item} failed.
	
	\pnum
	\tcode{io2d_error::no_current_point} if the preconditions are violated.
	
\end{itemdescr}

\indexlibrary{\idxcode{path_factory}!\idxcode{rel_line_to}}
\indexlibrary{\idxcode{rel_line_to}!\idxcode{path_factory}}
\begin{itemdecl}
    void rel_line_to(const vector_2d& dpt);
    void rel_line_to(const vector_2d& dpt, error_code& ec) noexcept;
\end{itemdecl}
\begin{itemdescr}
	\pnum
	\preconditions
	\tcode{_Has_current_point == true}.

	\pnum
	\effects
	\tcode{_Data.emplace_back(std::experimental::io2d::rel_line_to(pt))}.
	
	\pnum
	\tcode{_Current_point = dpt + _Current_point}.
	
	\pnum
	\throws
	As specified in Error reporting (\ref{\iotwod.err.report}).

	\pnum
	\remarks
	In the event of an error, the object shall not be modified.

	\pnum
	\errors
	\tcode{errc::not_enough_memory} if the attempt to add the \tcode{path_data_item} failed.
	
	\pnum
	\tcode{io2d_error::no_current_point} if the preconditions are violated.
	
\end{itemdescr}

\indexlibrary{\idxcode{path_factory}!\idxcode{rel_move_to}}
\indexlibrary{\idxcode{rel_move_to}!\idxcode{path_factory}}
\begin{itemdecl}
    void rel_move_to(const vector_2d& dpt);
    void rel_move_to(const vector_2d& dpt, error_code& ec) noexcept;
\end{itemdecl}
\begin{itemdescr}
	\pnum
	\preconditions
	\tcode{_Has_current_point == true}.

	\pnum
	\effects
	\tcode{_Data.emplace_back(std::experimental::io2d::rel_move_to(dpt))}.
	
	\pnum
	\tcode{_Current_point = dpt + _Current_point}.
	
	\pnum
	\throws
	As specified in Error reporting (\ref{\iotwod.err.report}).

	\pnum
	\remarks
	In the event of an error, the object shall not be modified.

	\pnum
	\errors
	\tcode{errc::not_enough_memory} if the attempt to add the \tcode{path_data_item} failed.
	
	\pnum
	\tcode{io2d_error::no_current_point} if the preconditions are violated.
	
\end{itemdescr}

\indexlibrary{\idxcode{path_factory}!\idxcode{transform_matrix}}
\indexlibrary{\idxcode{transform_matrix}!\idxcode{path_factory}}
\begin{itemdecl}
    void transform_matrix(const matrix_2d& m);
    void transform_matrix(const matrix_2d& m, error_code& ec) noexcept;
\end{itemdecl}
\begin{itemdescr}
	\pnum
	\effects
	\tcode{_Data.emplace_back(std::experimental::io2d::change_matrix(m))}.
	
	\pnum
	\tcode{_Transform_matrix = m}.
	
	\pnum
	\throws
	As specified in Error reporting (\ref{\iotwod.err.report}).

	\pnum
	\remarks
	In the event of an error, the object shall not be modified.

	\pnum
	\errors
	\tcode{errc::not_enough_memory} if the attempt to add the \tcode{path_data_item} failed.
	
\end{itemdescr}

\indexlibrary{\idxcode{path_factory}!\idxcode{origin}}
\indexlibrary{\idxcode{origin}!\idxcode{path_factory}}
\begin{itemdecl}
    void origin(const vector_2d& pt);
    void origin(const vector_2d& pt, error_code& ec) noexcept;
\end{itemdecl}
\begin{itemdescr}
	\pnum
	\effects
	\tcode{_Data.emplace_back(std::experimental::io2d::change_origin(pt)))}.
	
	\pnum
	\tcode{_Origin = pt}.
	
	\pnum
	\postconditions
	\tcode{_Origin == pt}.
	
	\pnum
	\throws
	As specified in Error reporting (\ref{\iotwod.err.report}).

	\pnum
	\remarks
	In the event of an error, the object shall not be modified.

	\pnum
	\errors
	\tcode{errc::not_enough_memory} if the attempt to add the \tcode{path_data_item} failed.
	
\end{itemdescr}

\indexlibrary{\idxcode{path_factory}!\idxcode{clear}}
\indexlibrary{\idxcode{clear}!\idxcode{path_factory}}
\begin{itemdecl}
    void clear() noexcept;
\end{itemdecl}
\begin{itemdescr}
	\pnum
	\postconditions
	\tcode{_Data.empty() == true}.
	
	\pnum
	\tcode{_Has_current_point == false}.
	
	\pnum
	\tcode{_Transform_matrix == matrix_2d::init_identity()}.
	
	\pnum
	\tcode{_Origin == vector_2d\{ \}}.

\end{itemdescr}

\rSec1 [pathfactory.observers] {\tcode{path_factory} observers}

\indexlibrary{\idxcode{path_factory}!\idxcode{path_extents}}
\indexlibrary{\idxcode{path_extents}!\idxcode{path_factory}}
\begin{itemdecl}
    std::experimental::io2d::rectangle path_extents() const;
    std::experimental::io2d::rectangle path_extents(error_code& ec) const noexcept;
\end{itemdecl}
\begin{itemdescr}
	\pnum
	\returns
	A \tcode{rectangle} object which contains the extents of the \term{path segments}, including \term{degenerate path segments}, in \tcode{_Data} when it is processed as described in \ref{pathgeometries.processing}.
	\enternote
	By using path segments, this description intentionally omits points established by \tcode{move_to} and \tcode{rel_move_to} operations from the extents value except where those points are subsequently used in defining a path segment.
	\exitnote

	\pnum
	\throws
	As specified in Error reporting (\ref{\iotwod.err.report}).

	\pnum
	\errors
	\tcode{io2d_error::invalid_matrix} if \tcode{_Data} includes a \tcode{change_matrix} operation which establishes a non-invertible \tcode{matrix_2d} as the transformation matrix and that matrix must subsequently be inverted in order to process the path geometries.
\end{itemdescr}

\indexlibrary{\idxcode{path_factory}!\idxcode{has_current_point}}
\indexlibrary{\idxcode{has_current_point}!\idxcode{path_factory}}
\begin{itemdecl}
    bool has_current_point() const noexcept;
\end{itemdecl}
\begin{itemdescr}
	\pnum
	\returns
	\tcode{_Has_current_point}.

\end{itemdescr}

\indexlibrary{\idxcode{path_factory}!\idxcode{current_point}}
\indexlibrary{\idxcode{current_point}!\idxcode{path_factory}}
\begin{itemdecl}
    vector_2d current_point() const;
    vector_2d current_point(error_code& ec) const noexcept;
\end{itemdecl}
\begin{itemdescr}
	\pnum
	\preconditions
	\tcode{_Has_current_point == true}.
	
	\pnum
	\returns
	\tcode{_Current_point}.

	\pnum
	\throws
	As specified in Error reporting (\ref{\iotwod.err.report}).

	\pnum
	\errors
	\tcode{io2d_error::no_current_point} if the preconditions are violated.
	
\end{itemdescr}

\indexlibrary{\idxcode{path_factory}!\idxcode{transform_matrix}}
\indexlibrary{\idxcode{transform_matrix}!\idxcode{path_factory}}
\begin{itemdecl}
    matrix_2d transform_matrix() const noexcept;
\end{itemdecl}
\begin{itemdescr}
	\pnum
	\returns
	\tcode{_Transform_matrix}.

\end{itemdescr}

\indexlibrary{\idxcode{path_factory}!\idxcode{origin}}
\indexlibrary{\idxcode{origin}!\idxcode{path_factory}}
\begin{itemdecl}
    vector_2d origin() const noexcept;
\end{itemdecl}
\begin{itemdescr}
	\pnum
	\returns
	\tcode{_Origin}.

\end{itemdescr}

\indexlibrary{\idxcode{path_factory}!\idxcode{data}}
\indexlibrary{\idxcode{data}!\idxcode{path_factory}}
\begin{itemdecl}
    vector<path_data_item> data() const;
    vector<path_data_item> data(error_code& ec) const noexcept;
\end{itemdecl}
\begin{itemdescr}
	\pnum
	\returns
	A copy of \tcode{_Data}.

	\pnum
	\throws
	As specified in Error reporting (\ref{\iotwod.err.report}).

	\pnum
	\errors
	\tcode{errc::not_enough_memory} if there was a failure to allocate memory.
	
\end{itemdescr}

\indexlibrary{\idxcode{path_factory}!\idxcode{data_item}}
\indexlibrary{\idxcode{data_item}!\idxcode{path_factory}}
\begin{itemdecl}
    path_data_item data_item(unsigned int index) const;
    path_data_item data_item(unsigned int index, error_code& ec) const noexcept;
\end{itemdecl}
\begin{itemdescr}
	\pnum
	\preconditions
	\tcode{_Data.size() > index}.
	
	\pnum
	\returns
	\tcode{_Data.at(index)}.
	
	\pnum
	\throws
	As specified in Error reporting (\ref{\iotwod.err.report}).

	\pnum
	\errors
	\tcode{io2d_error::invalid_index} if \tcode{index} violated the preconditions.

\end{itemdescr}

\indexlibrary{\idxcode{path_factory}!\idxcode{data_ref}}
\indexlibrary{\idxcode{data_ref}!\idxcode{path_factory}}
\begin{itemdecl}
    const vector<path_data_item>& data_ref() const noexcept;
\end{itemdecl}
\begin{itemdescr}
	\pnum
	\returns
	\tcode{_Data}.

\end{itemdescr}

\addtocounter{SectionDepthBase}{1}
%!TEX root = io2d.tex
\rSec0 [pathfactory.pathnewpath] {Class \tcode{path_factory::path_new_path}}

\pnum
\indexlibrary{\idxcode{path_factory::path_new_path}}
The class \tcode{path_factory::path_new_path} describes a path operation that creates a new path and makes the previous path, if any, an open path unless it was closed by \tcode{path_factory::path_close_path}.

\pnum
The new path has no current point.

\rSec1 [pathfactory.pathnewpath.synopsis] {\tcode{path_factory::path_new_path} synopsis}

\begin{codeblock}
namespace std { namespace experimental { namespace io2d { inline namespace v1 {
  class path_factory::path_new_path {
  };
} } } }
\end{codeblock}

%!TEX root = io2d.tex
\rSec0 [pathfactory.pathclosepath] {Class \tcode{path_factory::path_close_path}}

%\pnum
%\indexlibrary{\idxcode{path_factory::path_close_path}}
%The class \tcode{path_factory::path_close_path} describes a path instruction that affects the interpretation of a path factory's path group. It is described in terms of its effect on the evaluation of the path group. 
%
%\pnum
%If the current point in the path group contains a value. If it does, this instruction creates a line from the current point to the path group's last-move-to point. It then sets the path group's current point and last-move-to point to the value of the previous path geometry's last-move-to point.
%
%\pnum
%If there is no current point, then this operation does nothing.
%\enternote
%Because this operation does nothing if there is no current point, there is no need to track whether or not a path geometry has a valid last-move-to point. This operation is the only operation that uses the last-move-to point and all operations that establish a current point for a path geometry also establish a valid last-move-to point for that path geometry.
%\exitnote
%
\rSec1 [pathfactory.pathclosepath.synopsis] {\tcode{path_factory::path_close_path} synopsis}

\begin{codeblock}
namespace std { namespace experimental { namespace io2d { inline namespace v1 {
  class path_factory::path_close_path {
  };
} } } }
\end{codeblock}

\enternote
This class is a path instruction that contains no data. It exists to enable certain operations within a path group.
\exitnote

%!TEX root = io2d.tex
\rSec0 [absmove] {Class \tcode{abs_move}}

\pnum
\indexlibrary{\idxcode{abs_move}}
The class \tcode{abs_move} describes a path operation that creates a new path and makes the previous path, if any, an open path unless it was closed by \tcode{close_path}.

\pnum
It has an end point of type \tcode{vector_2d}.

\pnum
The end point is also the start point of the new path and its last-move-to point.

\rSec1 [absmove.synopsis] {\tcode{abs_move} synopsis}

\begin{codeblock}
namespace std { namespace experimental { namespace io2d { inline namespace v1 {
  namespace path_data {
    class abs_move {
    public:
      // \ref{absmove.cons}, construct:
      explicit abs_move(const vector_2d& pt) noexcept;

      // \ref{absmove.modifiers}, modifiers:
      void to(const vector_2d& pt) noexcept;

      // \ref{absmove.observers}, observers:
      vector_2d to() const noexcept;
    };
  };
} } } }
\end{codeblock}

\rSec1 [absmove.cons] {\tcode{abs_move} constructors}

\indexlibrary{\idxcode{abs_move}!constructor}
\begin{itemdecl}
    explicit abs_move(const vector_2d& pt) noexcept;
\end{itemdecl}
\begin{itemdescr}
	\pnum
	\effects
	Constructs an object of type \tcode{abs_move}.
	
	\pnum
	The end point shall be set to the value of \tcode{pt}.
\end{itemdescr}

\rSec1 [absmove.modifiers]{\tcode{abs_move} modifiers}

\indexlibrary{\idxcode{abs_move}!\idxcode{to}}
\indexlibrary{\idxcode{to}!\idxcode{abs_move}}
\begin{itemdecl}
    void to(const vector_2d& pt) noexcept;
\end{itemdecl}
\begin{itemdescr}
	\pnum
	\effects
	The end point shall be set to the value of \tcode{pt}.
\end{itemdescr}

\rSec1 [absmove.observers]{\tcode{abs_move} observers}

\indexlibrary{\idxcode{abs_move}!\idxcode{to}}
\indexlibrary{\idxcode{to}!\idxcode{abs_move}}
\begin{itemdecl}
    vector_2d to() const noexcept;
\end{itemdecl}
\begin{itemdescr}
	\pnum
	\returns
	The value of the end point.
\end{itemdescr}

%!TEX root = io2d.tex
\rSec0 [relmove] {Class \tcode{rel_move}}

\pnum
\indexlibrary{\idxcode{rel_move}}
The class \tcode{rel_move} describes a path operation that creates a new path and makes the previous path, if any, an open path unless it was closed by \tcode{close_path}.

\pnum
It has an end point of type \tcode{vector_2d}.

\pnum
Its end point is relative to the most recently established current point.

\pnum
The relative end point is also the start point of the new path and its last-move-to point.

\rSec1 [relmove.synopsis] {\tcode{rel_move} synopsis}

\begin{codeblock}
namespace std { namespace experimental { namespace io2d { inline namespace v1 {
  namespace path_data {
    class rel_move {
    public:
      // \ref{relmove.cons}, construct:
      constexpr rel_move() noexcept;
      constexpr explicit rel_move(const vector_2d& pt) noexcept;
      constexpr rel_move(const rel_move&) noexcept = default;
      constexpr rel_move& operator=(const rel_move&) noexcept = default;
      rel_move(rel_move&&) noexcept = default;
      rel_move& operator=(rel_move&&) noexcept = default;

      // \ref{relmove.modifiers}, modifiers:
      void to(const vector_2d& pt) noexcept;

      // \ref{relmove.observers}, observers:
      vector_2d to() const noexcept;
    };
  };
} } } }
\end{codeblock}

\rSec1 [relmove.cons] {\tcode{rel_move} constructors}

\indexlibrary{\idxcode{rel_move}!constructor}
\begin{itemdecl}
constexpr rel_move() noexcept;
\end{itemdecl}
\begin{itemdescr}
\pnum
\effects
Constructs an object of type \tcode{rel_move}.

\pnum
The end point shall be set to the value \tcode{vector_2d\{0.0, 0.0\}}.
\end{itemdescr}

\indexlibrary{\idxcode{rel_move}!constructor}
\begin{itemdecl}
constexpr explicit rel_move(const vector_2d& pt) noexcept;
\end{itemdecl}
\begin{itemdescr}
\pnum
\effects
Constructs an object of type \tcode{rel_move}.

\pnum
The end point shall be set to the value of \tcode{pt}.
\end{itemdescr}

\rSec1 [relmove.modifiers]{\tcode{rel_move} modifiers}

\indexlibrary{\idxcode{rel_move}!\idxcode{to}}
\indexlibrary{\idxcode{to}!\idxcode{rel_move}}
\begin{itemdecl}
void to(const vector_2d& pt) noexcept;
\end{itemdecl}
\begin{itemdescr}
\pnum
\effects
The end point shall be set to the value of \tcode{pt}.
\end{itemdescr}

\rSec1 [relmove.observers]{\tcode{rel_move} observers}

\indexlibrary{\idxcode{rel_move}!\idxcode{to}}
\indexlibrary{\idxcode{to}!\idxcode{rel_move}}
\begin{itemdecl}
constexpr vector_2d to() const noexcept;
\end{itemdecl}
\begin{itemdescr}
\pnum
\returns
The value of the end point.
\end{itemdescr}

%!TEX root = io2d.tex
\rSec0 [absline] {Class \tcode{abs_line}}

\pnum
\indexlibrary{\idxcode{abs_line}}
The class \tcode{abs_line} describes a path segment that is a line.

\pnum
It has an end point of type \tcode{vector_2d}.

\rSec1 [absline.synopsis] {\tcode{abs_line} synopsis}

\begin{codeblock}
namespace std { namespace experimental { namespace io2d { inline namespace v1 {
  namespace path_data {
    class abs_line {
    public:
      // \ref{absline.cons}, construct:
      constexpr abs_line() noexcept;
      constexpr explicit abs_line(const vector_2d& pt) noexcept;

      // \ref{absline.modifiers}, modifiers:
      constexpr void to(const vector_2d& pt) noexcept;

      // \ref{absline.observers}, observers:
      constexpr vector_2d to() const noexcept;
    };
  };
} } } }
\end{codeblock}

\rSec1 [absline.cons] {\tcode{abs_line} constructors and assignment operators}

\indexlibrary{\idxcode{abs_line}!constructor}
\begin{itemdecl}
constexpr abs_line() noexcept;
\end{itemdecl}
\begin{itemdescr}
\pnum
\effects
Constructs an object of type \tcode{abs_line}.

\pnum
The end point shall be set to the value of \tcode{vector_2d\{0.0, 0.0\}}.
\end{itemdescr}

\indexlibrary{\idxcode{abs_line}!constructor}
\begin{itemdecl}
constexpr explicit abs_line(const vector_2d& pt) noexcept;
\end{itemdecl}
\begin{itemdescr}
\pnum
\effects
Constructs an object of type \tcode{abs_line}.

\pnum
The end point shall be set to the value of \tcode{pt}.
\end{itemdescr}

\rSec1 [absline.modifiers]{\tcode{abs_line} modifiers}

\indexlibrary{\idxcode{abs_line}!\idxcode{to}}
\begin{itemdecl}
constexpr void to(const vector_2d& pt) noexcept;
\end{itemdecl}
\begin{itemdescr}
\pnum
\effects
The end point shall be set to the value of \tcode{pt}.
\end{itemdescr}

\rSec1 [absline.observers]{\tcode{abs_line} observers}

\indexlibrary{\idxcode{abs_line}!\idxcode{to}}
\begin{itemdecl}
constexpr vector_2d to() const noexcept;
\end{itemdecl}
\begin{itemdescr}
\pnum
\returns
The value of the end point.
\end{itemdescr}

%!TEX root = io2d.tex
\rSec0 [\iotwod.relline] {Class \tcode{rel_line}}

\pnum
\indexlibrary{\idxcode{rel_line}}%
The class \tcode{rel_line} describes a path item that is a path segment.

\pnum
It has an \term{end point} of type \tcode{vector_2d}.

\rSec1 [\iotwod.relline.synopsis] {\tcode{rel_line} synopsis}

\begin{codeblock}
namespace std::experimental::io2d::v1 {
  namespace path_data {
    class rel_line {
    public:
      // \ref{\iotwod.relline.cons}, construct:
      constexpr rel_line() noexcept;
      constexpr explicit rel_line(const vector_2d& pt) noexcept;

      // \ref{\iotwod.relline.modifiers}, modifiers:
      constexpr void to(const vector_2d& pt) noexcept;

      // \ref{\iotwod.relline.observers}, observers:
      constexpr vector_2d to() const noexcept;
    };
    
    // \ref{\iotwod.relline.ops}, operators:
    constexpr bool operator==(const rel_line& lhs, const rel_line& rhs) 
      noexcept;
    constexpr bool operator!=(const rel_line& lhs, const rel_line& rhs) 
      noexcept;
  }
}
\end{codeblock}

\rSec1 [\iotwod.relline.cons] {\tcode{rel_line} constructors}

\indexlibrary{\idxcode{rel_line}!constructor}%
\begin{itemdecl}
constexpr rel_line() noexcept;
\end{itemdecl}
\begin{itemdescr}
\pnum
\effects
Equivalent to: \tcode{rel_line\{ vector_2d() \};}
\end{itemdescr}

\indexlibrary{\idxcode{rel_line}!constructor}%
\begin{itemdecl}
constexpr explicit rel_line(const vector_2d& pt) noexcept;
\end{itemdecl}
\begin{itemdescr}
\pnum
\effects
Constructs an object of type \tcode{rel_line}.

\pnum
The end point is \tcode{pt}.
\end{itemdescr}

\rSec1 [\iotwod.relline.modifiers]{\tcode{rel_line} modifiers}

\indexlibrarymember{rel_line}{to}
\begin{itemdecl}
constexpr void to(const vector_2d& pt) noexcept;
\end{itemdecl}
\begin{itemdescr}
\pnum
\effects
The end point is \tcode{pt}.
\end{itemdescr}

\rSec1 [\iotwod.relline.observers]{\tcode{rel_line} observers}

\indexlibrary{\idxcode{rel_line}!\idxcode{to}}%
\begin{itemdecl}
constexpr vector_2d to() const noexcept;
\end{itemdecl}
\begin{itemdescr}
\pnum
\returns
The end point.
\end{itemdescr}

\rSec1 [\iotwod.relline.ops]{\tcode{rel_line} operators}

\indexlibrarymember{operator==}{rel_line}%
\begin{itemdecl}
constexpr bool operator==(const rel_line& lhs, const rel_line& rhs) noexcept;
\end{itemdecl}
\begin{itemdescr}
\pnum
\returns
\tcode{lhs.to() == rhs.to()}.
\end{itemdescr}

%!TEX root = io2d.tex
\rSec0 [\iotwod.abscubiccurve] {Class template \tcode{basic_figure_items<GraphicsSurfaces>::abs_cubic_curve}}

\rSec1 [\iotwod.abscubiccurve.intro] {Overview}

\pnum
\indexlibrary{\idxcode{abs_cubic_curve}}%
The class \tcode{basic_figure_items<GraphicsSurfaces>::abs_cubic_curve} describes a figure item that is a segment.

\pnum
It has a \term{first control point} of type \tcode{basic_point_2d<GraphicsSurfaces::graphics_math_type>}, a \term{second control point} of type \tcode{basic_point_2d<GraphicsSurfaces::graphics_math_type>}, and an \tcode{end point} of type \tcode{basic_point_2d<GraphicsSurfaces::graphics_math_type>}.

\pnum
The data are stored in an object of type \tcode{typename GraphicsSurfaces::paths::abs_cubic_curve_data_type}. It is accessible using the \tcode{data} member functions.

\rSec1 [\iotwod.abscubiccurve.synopsis] {Synopsis}
\begin{codeblock}
namespace @\fullnamespace{}@ {
  template <class GraphicsSurfaces>
  class basic_figure_items<GraphicsSurfaces>::abs_cubic_curve {
  public:
    using graphics_math_type = typename GraphicsSurfaces::graphics_math_type;
    using data_type =
      typename GraphicsSurfaces::paths::abs_cubic_curve_data_type;

    // \ref{\iotwod.abscubiccurve.ctor}, construct:
    abs_cubic_curve();
    abs_cubic_curve(const basic_point_2d<graphics_math_type>& cpt1,
       const basic_point_2d<graphics_math_type>& cpt2,
       const basic_point_2d<graphics_math_type>& ept) noexcept;
    abs_cubic_curve(const abs_cubic_curve& other) = default;
    abs_cubic_curve(abs_cubic_curve&& other) noexcept = default;

    // assign:
    abs_cubic_curve& operator=(const abs_cubic_curve& other) = default;
    abs_cubic_curve& operator=(abs_cubic_curve&& other) noexcept = default;

    // \ref{\iotwod.abscubiccurve.acc}, accessors:
    const data_type& data() const noexcept;
    data_type& data() noexcept;

    // \ref{\iotwod.abscubiccurve.mod}, modifiers:
    void control_pt1(const basic_point_2d<graphics_math_type>& cpt) noexcept;
    void control_pt2(const basic_point_2d<graphics_math_type>& cpt) noexcept;
    void end_pt(const basic_point_2d<graphics_math_type>& ept) noexcept;

    // \ref{\iotwod.abscubiccurve.obs}, observers:
    basic_point_2d<graphics_math_type> control_pt1() const noexcept;
    basic_point_2d<graphics_math_type> control_pt2() const noexcept;
    basic_point_2d<graphics_math_type> end_pt() const noexcept;
  };

  // \ref{\iotwod.abscubiccurve.eq}, equality operators:
  template <class GraphicsSurfaces>
  bool operator==(
    const typename basic_figure_items<GraphicsSurfaces>::abs_cubic_curve& lhs,
    const typename basic_figure_items<GraphicsSurfaces>::abs_cubic_curve& rhs) 
    noexcept;  
  template <class GraphicsSurfaces>
  bool operator!=(
    const typename basic_figure_items<GraphicsSurfaces>::abs_cubic_curve& lhs,
    const typename basic_figure_items<GraphicsSurfaces>::abs_cubic_curve& rhs) 
    noexcept;  
}
\end{codeblock}

\rSec1 [\iotwod.abscubiccurve.ctor] {Constructors}%

\indexlibrary{\idxcode{abs_cubic_curve}!constructor}%
\begin{itemdecl}
abs_cubic_curve() noexcept;
\end{itemdecl}
\begin{itemdescr}
\pnum
\effects
Equivalent to \tcode{abs_cubic_curve\{ basic_point_2d(), basic_point_2d(), basic_point_2d() \}}.
\end{itemdescr}

\indexlibrary{\idxcode{abs_cubic_curve}!constructor}%
\begin{itemdecl}
abs_cubic_curve(const basic_point_2d<typename GraphicsSurfaces::graphics_math_type>& cpt1,
  const basic_point_2d<typename GraphicsSurfaces::graphics_math_type>& cpt2,
  const basic_point_2d<typename GraphicsSurfaces::graphics_math_type>& ept) noexcept;
\end{itemdecl}
\begin{itemdescr}
\pnum
\effects Constructs an object of type \tcode{abs_cubic_curve}.

\pnum
\remarks The first control point is \tcode{cpt1}.

\pnum
\remarks The second control point is \tcode{cpt2}.

\pnum
\remarks The end point is \tcode{ept}.
\end{itemdescr}

\rSec1 [\iotwod.abscubiccurve.acc] {Accessors}%

\indexlibrarymember{data}{abs_cubic_curve}%
\begin{itemdecl}
const data_type& data() const noexcept;
data_type& data() noexcept;
\end{itemdecl}
\begin{itemdescr}
\pnum
\returns A reference to the \tcode{rel_matrix} object's data object (See: \ref{\iotwod.abscubiccurve.intro}).
\end{itemdescr}

\rSec1 [\iotwod.abscubiccurve.mod] {Modifiers}

\indexlibrarymember{control_pt1}{abs_cubic_curve}%
\begin{itemdecl}
void control_pt1(const basic_point_2d<typename
  GraphicsSurfaces::graphics_math_type>& cpt) noexcept;
\end{itemdecl}
\begin{itemdescr}
\pnum
\effects
The first control point is \tcode{cpt}.
\end{itemdescr}

\indexlibrarymember{control_pt2}{abs_cubic_curve}%
\begin{itemdecl}
void control_pt2(const basic_point_2d<typename
  GraphicsSurfaces::graphics_math_type>& cpt) noexcept;
\end{itemdecl}
\begin{itemdescr}
\pnum
\effects
The second control point is \tcode{cpt}.
\end{itemdescr}

\indexlibrarymember{end_pt}{abs_cubic_curve}%
\begin{itemdecl}
void end_pt(const basic_point_2d<typename GraphicsSurfaces::graphics_math_type>& ept) noexcept;
\end{itemdecl}
\begin{itemdescr}
\pnum
\effects
The end point is \tcode{ept}.
\end{itemdescr}

\rSec1 [\iotwod.abscubiccurve.obs] {Observers}

\indexlibrarymember{control_pt1}{abs_cubic_curve}%
\begin{itemdecl}
basic_point_2d<graphics_math_type> control_pt1() const noexcept;
\end{itemdecl}
\begin{itemdescr}
\pnum
\returns The first control point.
\end{itemdescr}

\indexlibrarymember{control_pt2}{abs_cubic_curve}%
\begin{itemdecl}
basic_point_2d<graphics_math_type> control_pt2() const noexcept;
\end{itemdecl}
\begin{itemdescr}
\pnum
\returns The second control point.
\end{itemdescr}

\indexlibrarymember{end_pt}{abs_cubic_curve}%
\begin{itemdecl}
basic_point_2d<graphics_math_type> end_pt() const noexcept;
\end{itemdecl}
\begin{itemdescr}
\pnum
\returns The end point.
\end{itemdescr}

\rSec1 [\iotwod.abscubiccurve.eq] {Equality operators}%

\indexlibrarymember{operator==}{abs_cubic_curve}%
\begin{itemdecl}
template <class GraphicsSurfaces>
bool operator==(
  const typename basic_figure_items<GraphicsSurfaces>::abs_cubic_curve& lhs,
  const typename basic_figure_items<GraphicsSurfaces>::abs_cubic_curve& rhs) 
  noexcept;
\end{itemdecl}
\begin{itemdescr}
\pnum
\returns
\tcode{lhs.control_pt1() == rhs.control_pt1() \&\& lhs.control_pt2() == rhs.control_pt2() \&\& lhs.end_pt() == rhs.end_pt()}.
\end{itemdescr}

\indexlibrarymember{operator!=}{abs_cubic_curve}%
\begin{itemdecl}
template <class GraphicsSurfaces>
bool operator!=(
  const typename basic_figure_items<GraphicsSurfaces>::abs_cubic_curve& lhs,
  const typename basic_figure_items<GraphicsSurfaces>::abs_cubic_curve& rhs) 
  noexcept;
\end{itemdecl}
\begin{itemdescr}
\pnum
\returns
\tcode{lhs.control_pt1() != rhs.control_pt1() || lhs.control_pt2() != rhs.control_pt2() || lhs.end_pt() != rhs.end_pt()}.
\end{itemdescr}

%!TEX root = io2d.tex
\rSec0 [\iotwod.relcubiccurve] {Class template \tcode{basic_figure_items<GraphicsSurfaces>::rel_cubic_curve}}

\rSec1 [\iotwod.relcubiccurve.intro] {Overview}

\pnum
\indexlibrary{\idxcode{rel_cubic_curve}}%
The class \tcode{basic_figure_items<GraphicsSurfaces>::rel_cubic_curve} describes a figure item that is a segment.

\pnum
It has a \term{first control point} of type \tcode{basic_point_2d<GraphicsSurfaces::graphics_math_type>}, a \term{second control point} of type \tcode{basic_point_2d<GraphicsSurfaces::graphics_math_type>}, and an \tcode{end point} of type \tcode{basic_point_2d<GraphicsSurfaces::graphics_math_type>}.

\pnum
The data are stored in an object of type \tcode{typename GraphicsSurfaces::paths::rel_cubic_curve_data_type}. It is accessible using the \tcode{data} member functions.

\rSec1 [\iotwod.relcubiccurve.synopsis] {Synopsis}
\begin{codeblock}
namespace std::experimemtal::io2d::v1 {
  template <class GraphicsSurfaces>
  class basic_figure_items<GraphicsSurfaces>::rel_cubic_curve {
  public:
    using graphics_math_type = typename GraphicsSurfaces::graphics_math_type;
    using data_type =
      typename GraphicsSurfaces::paths::rel_cubic_curve_data_type;

    // \ref{\iotwod.relcubiccurve.ctor}, construct:
    rel_cubic_curve();
    rel_cubic_curve(const basic_point_2d<graphics_math_type>& cpt1,
       const basic_point_2d<graphics_math_type>& cpt2,
       const basic_point_2d<graphics_math_type>& ept) noexcept;
    rel_cubic_curve(const rel_cubic_curve& other) = default;
    rel_cubic_curve(rel_cubic_curve&& other) noexcept = default;

    // assign:
    rel_cubic_curve& operator=(const rel_cubic_curve& other) = default;
    rel_cubic_curve& operator=(rel_cubic_curve&& other) noexcept = default;

    // \ref{\iotwod.relcubiccurve.acc}, accessors:
    const data_type& data() const noexcept;
    data_type& data() noexcept;

    // \ref{\iotwod.relcubiccurve.mod}, modifiers:
    void control_pt1(const basic_point_2d<graphics_math_type>& cpt) noexcept;
    void control_pt2(const basic_point_2d<graphics_math_type>& cpt) noexcept;
    void end_pt(const basic_point_2d<graphics_math_type>& ept) noexcept;

    // \ref{\iotwod.relcubiccurve.obs}, observers:
    basic_point_2d<graphics_math_type> control_pt1() const noexcept;
    basic_point_2d<graphics_math_type> control_pt2() const noexcept;
    basic_point_2d<graphics_math_type> end_pt() const noexcept;
  };

  // \ref{\iotwod.relcubiccurve.eq}, equality operators:
  template <class GraphicsSurfaces>
  bool operator==(
    const typename basic_figure_items<GraphicsSurfaces>::rel_cubic_curve& lhs,
    const typename basic_figure_items<GraphicsSurfaces>::rel_cubic_curve& rhs) 
    noexcept;  
  template <class GraphicsSurfaces>
  bool operator!=(
    const typename basic_figure_items<GraphicsSurfaces>::rel_cubic_curve& lhs,
    const typename basic_figure_items<GraphicsSurfaces>::rel_cubic_curve& rhs) 
    noexcept;  
}
\end{codeblock}

\rSec1 [\iotwod.relcubiccurve.ctor] {Constructors}%

\indexlibrary{\idxcode{rel_cubic_curve}!constructor}%
\begin{itemdecl}
rel_cubic_curve() noexcept;
\end{itemdecl}
\begin{itemdescr}
\pnum
\effects
Equivalent to \tcode{rel_cubic_curve\{ basic_point_2d(), basic_point_2d(), basic_point_2d() \}}.
\end{itemdescr}

\indexlibrary{\idxcode{rel_cubic_curve}!constructor}%
\begin{itemdecl}
rel_cubic_curve(const basic_point_2d<typename GraphicsSurfaces::graphics_math_type>& cpt1,
  const basic_point_2d<typename GraphicsSurfaces::graphics_math_type>& cpt2,
  const basic_point_2d<typename GraphicsSurfaces::graphics_math_type>& ept) noexcept;
\end{itemdecl}
\begin{itemdescr}
\pnum
\effects Constructs an object of type \tcode{rel_cubic_curve}.

\pnum
\remarks The first control point is \tcode{cpt1}.

\pnum
\remarks The second control point is \tcode{cpt2}.

\pnum
\remarks The end point is \tcode{ept}.
\end{itemdescr}

\rSec1 [\iotwod.relcubiccurve.acc] {Accessors}%

\indexlibrarymember{data}{rel_cubic_curve}%
\begin{itemdecl}
const data_type& data() const noexcept;
data_type& data() noexcept;
\end{itemdecl}
\begin{itemdescr}
\pnum
\returns A reference to the \tcode{rel_matrix} object's data object (See: \ref{\iotwod.relcubiccurve.intro}).
\end{itemdescr}

\rSec1 [\iotwod.relcubiccurve.mod] {Modifiers}

\indexlibrarymember{control_pt1}{rel_cubic_curve}%
\begin{itemdecl}
void control_pt1(const basic_point_2d<typename
  GraphicsSurfaces::graphics_math_type>& cpt) noexcept;
\end{itemdecl}
\begin{itemdescr}
\pnum
\effects
The first control point is \tcode{cpt}.
\end{itemdescr}

\indexlibrarymember{control_pt2}{rel_cubic_curve}%
\begin{itemdecl}
void control_pt2(const basic_point_2d<typename
  GraphicsSurfaces::graphics_math_type>& cpt) noexcept;
\end{itemdecl}
\begin{itemdescr}
\pnum
\effects
The second control point is \tcode{cpt}.
\end{itemdescr}

\indexlibrarymember{end_pt}{rel_cubic_curve}%
\begin{itemdecl}
void end_pt(const basic_point_2d<typename GraphicsSurfaces::graphics_math_type>& ept) noexcept;
\end{itemdecl}
\begin{itemdescr}
\pnum
\effects
The end point is \tcode{ept}.
\end{itemdescr}

\rSec1 [\iotwod.relcubiccurve.obs] {Observers}

\indexlibrarymember{control_pt1}{rel_cubic_curve}%
\begin{itemdecl}
basic_point_2d<graphics_math_type> control_pt1() const noexcept;
\end{itemdecl}
\begin{itemdescr}
\pnum
\returns The first control point.
\end{itemdescr}

\indexlibrarymember{control_pt2}{rel_cubic_curve}%
\begin{itemdecl}
basic_point_2d<graphics_math_type> control_pt2() const noexcept;
\end{itemdecl}
\begin{itemdescr}
\pnum
\returns The second control point.
\end{itemdescr}

\indexlibrarymember{end_pt}{rel_cubic_curve}%
\begin{itemdecl}
basic_point_2d<graphics_math_type> end_pt() const noexcept;
\end{itemdecl}
\begin{itemdescr}
\pnum
\returns The end point.
\end{itemdescr}

\rSec1 [\iotwod.relcubiccurve.eq] {Equality operators}%

\indexlibrarymember{operator==}{rel_cubic_curve}%
\begin{itemdecl}
template <class GraphicsSurfaces>
bool operator==(
  const typename basic_figure_items<GraphicsSurfaces>::rel_cubic_curve& lhs,
  const typename basic_figure_items<GraphicsSurfaces>::rel_cubic_curve& rhs) 
  noexcept;
\end{itemdecl}
\begin{itemdescr}
\pnum
\returns
\tcode{lhs.control_pt1() == rhs.control_pt1() \&\& lhs.control_pt2() == rhs.control_pt2() \&\& lhs.end_pt() == rhs.end_pt()}.
\end{itemdescr}

\indexlibrarymember{operator!=}{rel_cubic_curve}%
\begin{itemdecl}
template <class GraphicsSurfaces>
bool operator!=(
  const typename basic_figure_items<GraphicsSurfaces>::rel_cubic_curve& lhs,
  const typename basic_figure_items<GraphicsSurfaces>::rel_cubic_curve& rhs) 
  noexcept;
\end{itemdecl}
\begin{itemdescr}
\pnum
\returns
\tcode{lhs.control_pt1() != rhs.control_pt1() || lhs.control_pt2() != rhs.control_pt2() || lhs.end_pt() != rhs.end_pt()}.
\end{itemdescr}

%!TEX root = io2d.tex
\rSec0 [\iotwod.absquadraticcurve] {Class \tcode{abs_quadratic_curve}}

\pnum
\indexlibrary{\idxcode{abs_quadratic_curve}}%
The class \tcode{abs_quadratic_curve} describes a figure item that is a segment.

\pnum
It has a \term{control point} of type \tcode{basic_point_2d} and an \term{end point} of type \tcode{basic_point_2d}.

\rSec1 [\iotwod.absquadraticcurve.cons] {\tcode{abs_quadratic_curve} constructors}

\indexlibrary{\idxcode{abs_quadratic_curve}!constructor}%
\begin{itemdecl}
abs_quadratic_curve() noexcept;
\end{itemdecl}
\begin{itemdescr}
\pnum
\effects
Equivalent to: \tcode{abs_quadratic_curve\{ basic_point_2d(), basic_point_2d() \};}
\end{itemdescr}

\indexlibrary{\idxcode{abs_quadratic_curve}!constructor}%
\begin{itemdecl}
abs_quadratic_curve(const basic_point_2d<typename GraphicsSurfaces::graphics_math_type>& cpt,
  const basic_point_2d<typename GraphicsSurfaces::graphics_math_type>& ept) noexcept;
\end{itemdecl}
\begin{itemdescr}
\pnum
\effects
Constructs an object of type \tcode{abs_quadratic_curve}.

\pnum
The control point is \tcode{cpt}.

\pnum
The end point is \tcode{ept}.
\end{itemdescr}

\indexlibrary{\idxcode{abs_quadratic_curve}!constructor}%
\begin{itemdecl}
abs_quadratic_curve(const abs_quadratic_curve& other);
abs_quadratic_curve(abs_quadratic_curve&& other) noexcept;
\end{itemdecl}
\begin{itemdescr}
\pnum
\effects
Constructs an object of type \tcode{abs_quadratic_curve}. In the second form, other is left in a valid state with an unspecified value.

\pnum
The control point is \tcode{other.control_pt()}.

\pnum
The end point is \tcode{other.end_pt()}.
\end{itemdescr}

\rSec1 [\iotwod.absquadraticcurve.assign] {\tcode{abs_quadratic_curve} assignment operators}

\indexlibrary{\idxcode{abs_quadratic_curve}!assignment}%
\begin{itemdecl}
abs_quadratic_curve& operator=(const abs_quadratic_curve& other);
\end{itemdecl}
\begin{itemdescr}
\pnum
\effects
If \tcode{*this} and \tcode{other} are not the same object, modifies \tcode{*this} such that \tcode{*this.control_pt()} is \tcode{other.control_pt()} and \tcode{*this.end_pt()} is \tcode{other.end_pt()}

\pnum
If \tcode{*this} and \tcode{other} are the same object, the member has no effect.

\pnum
\returns
\tcode{*this}
\end{itemdescr}

\indexlibrary{\idxcode{abs_quadratic_curve}!assignment}%
\begin{itemdecl}
abs_quadratic_curve& operator=(abs_quadratic_curve&& other) noexcept;
\end{itemdecl}
\begin{itemdescr}
\pnum
\effects
<TODO>

\pnum
\returns
\tcode{*this}
\end{itemdescr}

\rSec1 [\iotwod.absquadraticcurve.modifiers]{\tcode{abs_quadratic_curve} modifiers}

\indexlibrarymember{control_pt}{abs_quadratic_curve}%
\begin{itemdecl}
void control_pt(const basic_point_2d<typename GraphicsSurfaces::graphics_math_type>& cpt) noexcept;
\end{itemdecl}
\begin{itemdescr}
\pnum
\effects
The control point is \tcode{cpt}.
\end{itemdescr}

\indexlibrarymember{end_pt}{abs_quadratic_curve}%
\begin{itemdecl}
void end_pt(const basic_point_2d<typename GraphicsSurfaces::graphics_math_type>& ept) noexcept;
\end{itemdecl}
\begin{itemdescr}
\pnum
\effects
The end point is \tcode{ept}.
\end{itemdescr}

\rSec1 [\iotwod.absquadraticcurve.observers]{\tcode{abs_quadratic_curve} observers}

\indexlibrarymember{control_pt}{abs_quadratic_curve}%
\begin{itemdecl}
basic_point_2d<typename GraphicsSurfaces::graphics_math_type> control_pt() const noexcept;
\end{itemdecl}
\begin{itemdescr}
\pnum
\returns
The control point.
\end{itemdescr}

\indexlibrarymember{end_pt}{abs_quadratic_curve}%
\begin{itemdecl}
basic_point_2d<typename GraphicsSurfaces::graphics_math_type> end_pt() const noexcept;
\end{itemdecl}
\begin{itemdescr}
\pnum
\returns
The end point.
\end{itemdescr}

\rSec1 [\iotwod.absquadraticcurve.ops]{\tcode{abs_quadratic_curve} operators}

\indexlibrarymember{operator==}{abs_quadratic_curve}%
\begin{itemdecl}
template <class GraphicsSurfaces>
bool operator==(const typename basic_figure_items<GraphicsSurfaces>::abs_quadratic_curve& lhs,
  const typename basic_figure_items<GraphicsSurfaces>::abs_quadratic_curve& rhs) noexcept;
\end{itemdecl}
\begin{itemdescr}
\pnum
\returns
\tcode{lhs.control_pt() == rhs.control_pt() \&\& lhs.end_pt() == rhs.end_pt()}.
\end{itemdescr}

%!TEX root = io2d.tex
\rSec0 [\iotwod.relquadraticcurve] {Class \tcode{rel_quadratic_curve}}

\pnum
\indexlibrary{\idxcode{rel_quadratic_curve}}%
The class \tcode{rel_quadratic_curve} describes a figure item that is a segment.

\pnum
It has a \term{control point} of type \tcode{basic_point_2d} and an \term{end point} of type \tcode{basic_point_2d}.

\rSec1 [\iotwod.relquadraticcurve.cons] {\tcode{rel_quadratic_curve} constructors}

\indexlibrary{\idxcode{rel_quadratic_curve}!constructor}%
\begin{itemdecl}
rel_quadratic_curve() noexcept;
\end{itemdecl}
\begin{itemdescr}
\pnum
\effects
Equivalent to: \tcode{rel_quadratic_curve\{ basic_point_2d(), basic_point_2d() \};}
\end{itemdescr}

\indexlibrary{\idxcode{rel_quadratic_curve}!constructor}%
\begin{itemdecl}
rel_quadratic_curve(const basic_point_2d<typename GraphicsSurfaces::graphics_math_type>& cpt,
  const basic_point_2d<typename GraphicsSurfaces::graphics_math_type>& ept) noexcept;
\end{itemdecl}
\begin{itemdescr}
\pnum
\effects
Constructs an object of type \tcode{rel_quadratic_curve}.

\pnum
The control point is \tcode{cpt}.

\pnum
The end point is \tcode{ept}.
\end{itemdescr}

\indexlibrary{\idxcode{rel_quadratic_curve}!constructor}%
\begin{itemdecl}
rel_quadratic_curve(const rel_quadratic_curve& other);
rel_quadratic_curve(rel_quadratic_curve&& other) noexcept;
\end{itemdecl}
\begin{itemdescr}
\pnum
\effects
Constructs an object of type \tcode{rel_quadratic_curve}. In the second form, other is left in a valid state with an unspecified value.

\pnum
The control point is \tcode{other.control_pt()}.

\pnum
The end point is \tcode{other.end_pt()}.
\end{itemdescr}

\rSec1 [\iotwod.relquadraticcurve.assign] {\tcode{rel_quadratic_curve} assignment operators}

\indexlibrary{\idxcode{rel_quadratic_curve}!assignment}%
\begin{itemdecl}
rel_quadratic_curve& operator=(const rel_quadratic_curve& other);
\end{itemdecl}
\begin{itemdescr}
\pnum
\effects
If \tcode{*this} and \tcode{other} are not the same object, modifies \tcode{*this} such that \tcode{*this.control_pt()} is \tcode{other.control_pt()} and \tcode{*this.end_pt()} is \tcode{other.end_pt()}

\pnum
If \tcode{*this} and \tcode{other} are the same object, the member has no effect.

\pnum
\returns
\tcode{*this}
\end{itemdescr}

\indexlibrary{\idxcode{rel_quadratic_curve}!assignment}%
\begin{itemdecl}
rel_quadratic_curve& operator=(rel_quadratic_curve&& other) noexcept;
\end{itemdecl}
\begin{itemdescr}
\pnum
\effects
<TODO>

\pnum
\returns
\tcode{*this}
\end{itemdescr}

\rSec1 [\iotwod.relquadraticcurve.modifiers]{\tcode{rel_quadratic_curve} modifiers}

\indexlibrarymember{control_pt}{rel_quadratic_curve}%
\begin{itemdecl}
void control_pt(const basic_point_2d<typename GraphicsSurfaces::graphics_math_type>& cpt) noexcept;
\end{itemdecl}
\begin{itemdescr}
\pnum
\effects
The control point is \tcode{cpt}.
\end{itemdescr}

\indexlibrarymember{end_pt}{rel_quadratic_curve}%
\begin{itemdecl}
void end_pt(const basic_point_2d<typename GraphicsSurfaces::graphics_math_type>& ept) noexcept;
\end{itemdecl}
\begin{itemdescr}
\pnum
\effects
The end point is \tcode{ept}.
\end{itemdescr}

\rSec1 [\iotwod.relquadraticcurve.observers]{\tcode{rel_quadratic_curve} observers}

\indexlibrarymember{control_pt}{rel_quadratic_curve}%
\begin{itemdecl}
basic_point_2d<typename GraphicsSurfaces::graphics_math_type> control_pt() const noexcept;
\end{itemdecl}
\begin{itemdescr}
\pnum
\returns
The control point.
\end{itemdescr}

\indexlibrarymember{end_pt}{rel_quadratic_curve}%
\begin{itemdecl}
basic_point_2d<typename GraphicsSurfaces::graphics_math_type> end_pt() const noexcept;
\end{itemdecl}
\begin{itemdescr}
\pnum
\returns
The end point.
\end{itemdescr}

\rSec1 [\iotwod.relquadraticcurve.ops]{\tcode{rel_quadratic_curve} operators}

\indexlibrarymember{operator==}{rel_quadratic_curve}%
\begin{itemdecl}
template <class GraphicsSurfaces>
bool operator==(const typename basic_figure_items<GraphicsSurfaces>::rel_quadratic_curve& lhs,
  const typename basic_figure_items<GraphicsSurfaces>::rel_quadratic_curve& rhs) noexcept;
\end{itemdecl}
\begin{itemdescr}
\pnum
\returns
\tcode{lhs.control_pt() == rhs.control_pt() \&\& lhs.end_pt() == rhs.end_pt()}.
\end{itemdescr}

%!TEX root = io2d.tex
\rSec0 [arcclockwise] {Class \tcode{arc_clockwise}}

\pnum
\indexlibrary{\idxcode{arc_clockwise}}
The class \tcode{arc_clockwise} describes a path segment that is a circular arc with clockwise rotation.

\pnum
It has a Circle of type \tcode{circle}, a First Angle of type \tcode{double}, and a Second Angle of type \tcode{double}.

\pnum
The values for the First Angle and Second Angle are in radians.

\pnum
\begin{note}
Although the value of the First Angle may be greater than the value of the Second Angle, when processed as described in \ref{paths.processing}, \tcode{two_pi<double>} is added to the Second Angle until the value of the Second Angle is greater than or equal to the value of the First Angle.
\end{note}

\rSec1 [arcclockwise.synopsis] {\tcode{arc_clockwise} synopsis}

\begin{codeblock}
namespace std { namespace experimental { namespace io2d { inline namespace v1 {
  namespace path_data {
    class arc_clockwise {
    public:
      // \ref{arcclockwise.cons}, construct/copy/move/destroy:
      constexpr arc_clockwise() noexcept;
      constexpr arc_clockwise(const experimental::io2d::circle& c,
        double angle1, double angle2) noexcept;
      constexpr arc_clockwise(const vector_2d& ctr, double rad,
        double angle1, double angle2) noexcept;

      // \ref{arcclockwise.modifiers}, modifiers:
      constexpr void circle(const experimental::io2d::circle& c) noexcept;
      constexpr void center(const vector_2d& ctr) noexcept;
      constexpr void radius(double r) noexcept;
      constexpr void angle_1(double radians) noexcept;
      constexpr void angle_2(double radians) noexcept;

      // \ref{arcclockwise.observers}, observers:
      constexpr experimental::io2d::circle circle() const noexcept;
      constexpr vector_2d center() const noexcept;
      constexpr double radius() const noexcept;
      constexpr double angle_1() const noexcept;
      constexpr double angle_2() const noexcept;
    };
  };
} } } }
\end{codeblock}

\rSec1 [arcclockwise.cons] {\tcode{arc_clockwise} constructors and assignment operators}

\indexlibrary{\idxcode{arc_clockwise}!constructor}
\begin{itemdecl}
constexpr arc_clockwise() noexcept;
\end{itemdecl}
\begin{itemdescr}
\pnum
\effects
Constructs an object of type \tcode{arc_clockwise}.

\pnum
The Circle shall be set to the value of \tcode{experimental::io2d::circle\{ \}}.

\pnum
The First Angle shall be set to the value of \tcode{0.0}.

\pnum
The Second Angle shall be set to the value of \tcode{0.0}.
\end{itemdescr}

\indexlibrary{\idxcode{arc_clockwise}!constructor}
\begin{itemdecl}
constexpr arc_clockwise(const experimental::io2d::circle& c, double angle1,
  double angle2) noexcept;
\end{itemdecl}
\begin{itemdescr}
\pnum
\effects
Constructs an object of type \tcode{arc_clockwise}.

\pnum
The Circle shall be set to the value of \tcode{c}.

\pnum
The First Angle shall be set to the value of \tcode{angle1}.

\pnum
The Second Angle shall be set to the value of \tcode{angle2}.
\end{itemdescr}

\indexlibrary{\idxcode{arc_clockwise}!constructor}
\begin{itemdecl}
constexpr arc_clockwise(const vector_2d& ctr, double rad, double angle1,
  double angle2) noexcept;
\end{itemdecl}
\begin{itemdescr}
\pnum
\effects
Constructs an object of type \tcode{arc_clockwise}.

\pnum
The Circle's Center (\ref{circle.intro}) shall be set to the value of \tcode{ctr}.

\pnum
The Circle's Radius (\ref{circle.intro}) shall be set to the value of \tcode{rad}.

\pnum
The First Angle shall be set to the value of \tcode{angle1}.

\pnum
The Second Angle shall be set to the value of \tcode{angle2}.
\end{itemdescr}

\rSec1 [arcclockwise.modifiers]{\tcode{arc_clockwise} modifiers}

\indexlibrary{\idxcode{arc_clockwise}!\idxcode{circle}}
\begin{itemdecl}
constexpr void circle(const experimental::io2d::circle& c) noexcept;
\end{itemdecl}
\begin{itemdescr}
\pnum
\effects
The Circle shall be set to the value of \tcode{c}.
\end{itemdescr}

\indexlibrary{\idxcode{arc_clockwise}!\idxcode{center}}
\begin{itemdecl}
constexpr void center(const vector_2d& ctr) noexcept;
\end{itemdecl}
\begin{itemdescr}
\pnum
\effects
The Circle's Center (\ref{circle.intro}) shall be set to the value of \tcode{ctr}.
\end{itemdescr}

\indexlibrary{\idxcode{arc_clockwise}!\idxcode{radius}}
\begin{itemdecl}
constexpr void radius(double r) noexcept;
\end{itemdecl}
\begin{itemdescr}
\pnum
\effects
The Circle's Radius (\ref{circle.intro}) shall be set to the value of \tcode{r}.
\end{itemdescr}

\indexlibrary{\idxcode{arc_clockwise}!\idxcode{angle_1}}
\begin{itemdecl}
constexpr void angle_1(double radians) noexcept;
\end{itemdecl}
\begin{itemdescr}
\pnum
\effects
The First Angle shall be set to the value of \tcode{radians}.
\end{itemdescr}

\indexlibrary{\idxcode{arc_clockwise}!\idxcode{angle_2}}
\begin{itemdecl}
constexpr void angle_2(double radians) noexcept;
\end{itemdecl}
\begin{itemdescr}
\pnum
\effects
The Second Angle shall be set to the value of \tcode{radians}.
\end{itemdescr}

\rSec1 [arcclockwise.observers]{\tcode{arc_clockwise} observers}

\indexlibrary{\idxcode{arc_clockwise}!\idxcode{circle}}
\begin{itemdecl}
constexpr experimental::io2d::circle circle() const noexcept;
\end{itemdecl}
\begin{itemdescr}
\pnum
\returns
The value of the Circle.
\end{itemdescr}

\indexlibrary{\idxcode{arc_clockwise}!\idxcode{center}}
\begin{itemdecl}
constexpr vector_2d center() const noexcept;
\end{itemdecl}
\begin{itemdescr}
\pnum
\returns
The value of the Circle's Center (\ref{circle.intro}).
\end{itemdescr}

\indexlibrary{\idxcode{arc_clockwise}!\idxcode{radius}}
\begin{itemdecl}
constexpr double radius() const noexcept;
\end{itemdecl}
\begin{itemdescr}
\pnum
\returns
The value of the Circle's Radius (\ref{circle.intro}).
\end{itemdescr}

\indexlibrary{\idxcode{arc_clockwise}!\idxcode{angle_1}}
\begin{itemdecl}
constexpr double angle_1() const noexcept;
\end{itemdecl}
\begin{itemdescr}
\pnum
\returns
The value of the First Angle.
\end{itemdescr}

\indexlibrary{\idxcode{arc_clockwise}!\idxcode{angle_2}}
\begin{itemdecl}
constexpr double angle_2() const noexcept;
\end{itemdecl}
\begin{itemdescr}
\pnum
\returns
The value of the Second Angle.
\end{itemdescr}

%!TEX root = io2d.tex
\rSec0 [arccounterclockwise] {Class \tcode{arc_counterclockwise}}

\pnum
\indexlibrary{\idxcode{arc_counterclockwise}}
The class \tcode{arc_counterclockwise} describes a path segment that is a circular arc with counterclockwise rotation.

\pnum
It has a Circle of type \tcode{circle}, a First Angle of type \tcode{double}, and a Second Angle of type \tcode{double}.

\pnum
The values for the First Angle and Second Angle are in radians.

\pnum
\enternote
Although the value of the Second Angle may be greater than the value of the First Angle, when processed as described in \ref{paths.processing}, \tcode{two_pi<double>} is subtracted from the Second Angle until the value of the First Angle is greater than or equal to the value of the Second Angle.
\exitnote

\rSec1 [arccounterclockwise.synopsis] {\tcode{arc_counterclockwise} synopsis}

\begin{codeblock}
namespace std { namespace experimental { namespace io2d { inline namespace v1 {
  namespace path_data {
    class arc_counterclockwise {
    public:
      // \ref{arccounterclockwise.cons}, construct:
      constexpr arc_counterclockwise() noexcept;
      constexpr arc_counterclockwise(const experimental::io2d::circle& c,
        double angle1, double angle2) noexcept;
      constexpr arc_counterclockwise(const vector_2d& ctr, double rad,
        double angle1, double angle2) noexcept;
      constexpr arc_counterclockwise(const arc_clockwise&) noexcept = default;
      constexpr arc_counterclockwise& operator=(const arc_clockwise&&)
        noexcept = default;
      arc_counterclockwise(arc_counterclockwise&&) noexcept = default;
      arc_counterclockwise& operator=(arc_counterclockwise&&)
        noexcept = default;

      // \ref{arccounterclockwise.modifiers}, modifiers:
      void circle(const experimental::io2d::circle& c) noexcept;
      void center(const vector_2d& ctr) noexcept;
      void radius(double r) noexcept;
      void angle_1(double radians) noexcept;
      void angle_2(double radians) noexcept;

      // \ref{arccounterclockwise.observers}, observers:
      constexpr experimental::io2d::circle circle() const noexcept;
      constexpr vector_2d center() const noexcept;
      constexpr double radius() const noexcept;
      constexpr double angle_1() const noexcept;
      constexpr double angle_2() const noexcept;
    };
  };
} } } }
\end{codeblock}

\rSec1 [arccounterclockwise.cons] {\tcode{arc_counterclockwise} constructors and assignment operators}

\indexlibrary{\idxcode{arc_counterclockwise}!constructor}
\begin{itemdecl}
constexpr arc_counterclockwise() noexcept;
\end{itemdecl}
\begin{itemdescr}
\pnum
\effects
Constructs an object of type \tcode{arc_counterclockwise}.

\pnum
The Circle shall be set to the value of \tcode{experimental::io2d::circle\{ \}}.

\pnum
The First Angle shall be set to the value of \tcode{0.0}.

\pnum
The Second Angle shall be set to the value of \tcode{0.0}.
\end{itemdescr}

\indexlibrary{\idxcode{arc_counterclockwise}!constructor}
\begin{itemdecl}
constexpr arc_counterclockwise(const experimental::io2d::circle& c,
  double angle1, double angle2) noexcept;
\end{itemdecl}
\begin{itemdescr}
\pnum
\effects
Constructs an object of type \tcode{arc_counterclockwise}.
arc_counterclockwise
\pnum
The Circle shall be set to the value of \tcode{c}.

\pnum
The First Angle shall be set to the value of \tcode{angle1}.

\pnum
The Second Angle shall be set to the value of \tcode{angle2}.
\end{itemdescr}

\indexlibrary{\idxcode{arc_counterclockwise}!constructor}
\begin{itemdecl}
constexpr arc_counterclockwise(const vector_2d& ctr, double rad, double angle1,
  double angle2) noexcept;
\end{itemdecl}
\begin{itemdescr}
\pnum
\effects
Constructs an object of type \tcode{arc_counterclockwise}.

\pnum
The Circle's Center (\ref{circle.intro}) shall be set to the value of \tcode{ctr}.

\pnum
The Circle's Radius (\ref{circle.intro}) shall be set to the value of \tcode{rad}.

\pnum
The First Angle shall be set to the value of \tcode{angle1}.

\pnum
The Second Angle shall be set to the value of \tcode{angle2}.
\end{itemdescr}

\rSec1 [arccounterclockwise.modifiers]{\tcode{arc_counterclockwise} modifiers}

\indexlibrary{\idxcode{arc_counterclockwise}!\idxcode{circle}}
\indexlibrary{\idxcode{circle}!\idxcode{arc_counterclockwise}}
\begin{itemdecl}
void circle(const experimental::io2d::circle& c) noexcept;
\end{itemdecl}
\begin{itemdescr}
\pnum
\effects
The Circle shall be set to the value of \tcode{c}.
\end{itemdescr}

\indexlibrary{\idxcode{arc_counterclockwise}!\idxcode{center}}
\indexlibrary{\idxcode{center}!\idxcode{arc_counterclockwise}}
\begin{itemdecl}
void center(const vector_2d& ctr) noexcept;
\end{itemdecl}
\begin{itemdescr}
\pnum
\effects
The Circle's Center (\ref{circle.intro}) shall be set to the value of \tcode{ctr}.
\end{itemdescr}

\indexlibrary{\idxcode{arc_counterclockwise}!\idxcode{radius}}
\indexlibrary{\idxcode{radius}!\idxcode{arc_counterclockwise}}
\begin{itemdecl}
void radius(double r) noexcept;
\end{itemdecl}
\begin{itemdescr}
\pnum
\effects
The Circle's Radius (\ref{circle.intro}) shall be set to the value of \tcode{r}.
\end{itemdescr}

\indexlibrary{\idxcode{arc_counterclockwise}!\idxcode{angle_1}}
\indexlibrary{\idxcode{angle_1}!\idxcode{arc_counterclockwise}}
\begin{itemdecl}
void angle_1(double radians) noexcept;
\end{itemdecl}
\begin{itemdescr}
\pnum
\effects
The First Angle shall be set to the value of \tcode{radians}.
\end{itemdescr}

\indexlibrary{\idxcode{arc_counterclockwise}!\idxcode{angle_2}}
\indexlibrary{\idxcode{angle_2}!\idxcode{arc_counterclockwise}}
\begin{itemdecl}
void angle_2(double radians) noexcept;
\end{itemdecl}
\begin{itemdescr}
\pnum
\effects
The Second Angle shall be set to the value of \tcode{radians}.
\end{itemdescr}

\rSec1 [arccounterclockwise.observers]{\tcode{arc_counterclockwise} observers}

\indexlibrary{\idxcode{arc_counterclockwise}!\idxcode{circle}}
\indexlibrary{\idxcode{circle}!\idxcode{arc_counterclockwise}}
\begin{itemdecl}
constexpr experimental::io2d::circle circl() const noexcept;
\end{itemdecl}
\begin{itemdescr}
\pnum
\returns
The value of the Circle.
\end{itemdescr}

\indexlibrary{\idxcode{arc_counterclockwise}!\idxcode{center}}
\indexlibrary{\idxcode{center}!\idxcode{arc_counterclockwise}}
\begin{itemdecl}
constexpr vector_2d center() const noexcept;
\end{itemdecl}
\begin{itemdescr}
\pnum
\returns
The value of the Circle's Center (\ref{circle.intro}).
\end{itemdescr}

\indexlibrary{\idxcode{arc_counterclockwise}!\idxcode{radius}}
\indexlibrary{\idxcode{radius}!\idxcode{arc_counterclockwise}}
\begin{itemdecl}
constexpr double radius() const noexcept;
\end{itemdecl}
\begin{itemdescr}
\pnum
\returns
The value of the Circle's Radius (\ref{circle.intro}).
\end{itemdescr}

\indexlibrary{\idxcode{arc_counterclockwise}!\idxcode{angle_1}}
\indexlibrary{\idxcode{angle_1}!\idxcode{arc_counterclockwise}}
\begin{itemdecl}
constexpr double angle_1() const noexcept;
\end{itemdecl}
\begin{itemdescr}
\pnum
\returns
The value of the First Angle.
\end{itemdescr}

\indexlibrary{\idxcode{arc_counterclockwise}!\idxcode{angle_2}}
\indexlibrary{\idxcode{angle_2}!\idxcode{arc_counterclockwise}}
\begin{itemdecl}
constexpr double angle_2() const noexcept;
\end{itemdecl}
\begin{itemdescr}
\pnum
\returns
The value of the Second Angle.
\end{itemdescr}

%!TEX root = io2d.tex
\rSec0 [pathfactory.pathchangematrix] {Class \tcode{path_factory::path_change_matrix}}

\rSec1 [pathfactory.pathchangematrix.synopsis] {\tcode{path_factory::path_change_matrix} synopsis}

\pnum
\indexlibrary{\idxcode{path_factory::path_change_matrix}}
The class \tcode{path_factory::path_change_matrix} describes an operation on a path group.

\pnum
This operation changes the transformation matrix for a path group to be the value returned by \tcode{*this.matrix()}. As shown in \ref{paths.processing}, the new transformation matrix does not affect any operations that came before this operation. It is only used in processing operations that come after it. It continues to be used until another \tcode{path_factory::path_change_matrix} object is encountered or the end of the path group is reached.

\begin{codeblock}
namespace std { namespace experimental { namespace io2d { inline namespace v1 {
  class path_factory::path_change_matrix {
  public:
    // \ref{pathfactory.pathchangematrix.cons}, construct/copy/move/destroy:
    change_matrix() noexcept;
    change_matrix(const change_matrix&) noexcept;
    path_factory::path_change_matrix& operator=(const change_matrix&) noexcept;
    change_matrix(change_matrix&&) noexcept;
    path_factory::path_change_matrix& operator=(change_matrix&&) noexcept;
    explicit change_matrix(const matrix_2d& m) noexcept;

    // \ref{pathfactory.pathchangematrix.modifiers}, modifiers:
    void matrix(const matrix_2d& value) noexcept;

    // \ref{pathfactory.pathchangematrix.observers}, observers:
    matrix_2d matrix() const noexcept;
    virtual path_data_type type() const noexcept override;
    
  private:
    matrix_2d _Matrix; // \expos
  };
} } } }
\end{codeblock}

\rSec1 [pathfactory.pathchangematrix.cons] {\tcode{path_factory::path_change_matrix} constructors and assignment operators}

\indexlibrary{\idxcode{path_factory::path_change_matrix}!constructor}
\begin{itemdecl}
    change_matrix() noexcept;
\end{itemdecl}
\begin{itemdescr}
	\pnum
	\effects
	Constructs an object of type \tcode{path_factory::path_change_matrix}.
	
	\pnum
	\postconditions
	\tcode{_Matrix == matrix_2d\{\}}.
\end{itemdescr}

\indexlibrary{\idxcode{path_factory::path_change_matrix}!constructor}
\begin{itemdecl}
    explicit change_matrix(const matrix_2d& m) noexcept;
\end{itemdecl}
\begin{itemdescr}
	\pnum
	\effects
	Constructs an object of type \tcode{path_factory::path_change_matrix}.
	
	\pnum
	\postconditions
	\tcode{_Matrix == m}.
\end{itemdescr}

\rSec1 [pathfactory.pathchangematrix.modifiers]{\tcode{path_factory::path_change_matrix} modifiers}

\indexlibrary{\idxcode{path_factory::path_change_matrix}!\idxcode{matrix}}
\indexlibrary{\idxcode{matrix}!\idxcode{path_factory::path_change_matrix}}
\begin{itemdecl}
    void matrix(const matrix_2d& value) noexcept;
\end{itemdecl}
\begin{itemdescr}
	\pnum
	\postconditions
	\tcode{_Matrix == value}.
\end{itemdescr}

\rSec1 [pathfactory.pathchangematrix.observers]{\tcode{path_factory::path_change_matrix} observers}

\indexlibrary{\idxcode{path_factory::path_change_matrix}!\idxcode{matrix}}
\indexlibrary{\idxcode{matrix}!\idxcode{path_factory::path_change_matrix}}
\begin{itemdecl}
    matrix_2d matrix() const noexcept;
\end{itemdecl}
\begin{itemdescr}
	\pnum
	\returns
	\tcode{_Matrix}.
\end{itemdescr}

\indexlibrary{\idxcode{path_factory::path_change_matrix}!\idxcode{type}}
\indexlibrary{\idxcode{type}!\idxcode{path_factory::path_change_matrix}}
\begin{itemdecl}
    virtual path_data_type type() const noexcept override;
\end{itemdecl}
\begin{itemdescr}
	\pnum
	\returns
	\tcode{path_data_type::change_matrix}.
\end{itemdescr}

%!TEX root = io2d.tex
\rSec0 [pathdataitem.changeorigin] {Class \tcode{path_factory::path_change_origin}}

\pnum
\indexlibrary{\idxcode{path_factory::path_change_origin}}
The class \tcode{path_factory::path_change_origin} describes an operation on a path geometry collection.

\pnum
This operation changes the origin point for a path geometry collection to be the value returned by \tcode{*this.origin()}. As shown in \ref{pathgeometries.processing}, the new origin point does not affect any operations that came before this operation. It is only used in processing operations that come after it. It continues to be used until another \tcode{path_factory::path_change_origin} object is encountered or the end of the path geometry collection is reached.

\rSec1 [pathdataitem.changeorigin.synopsis] {\tcode{path_factory::path_change_origin} synopsis}

\begin{codeblock}
namespace std { namespace experimental { namespace io2d { inline namespace v1 {
  class path_factory::path_change_origin {
  public:
    // \ref{pathdataitem.changeorigin.cons}, construct/copy/move/destroy:
    change_origin() noexcept;
    change_origin(const change_origin&) noexcept;
    path_factory::path_change_origin& operator=(const change_origin&) noexcept;
    change_origin(change_origin&&) noexcept;
    path_factory::path_change_origin& operator=(change_origin&&) noexcept;
    explicit change_origin(const vector_2d& pt) noexcept;

    // \ref{pathdataitem.changeorigin.modifiers}, modifiers:
    void origin(const vector_2d& value) noexcept;

    // \ref{pathdataitem.changeorigin.observers}, observers:
    vector_2d origin() const noexcept;
    virtual path_data_type type() const noexcept override;
    
  private:
    vector_2d _Data; // \expos
  };
} } } }
\end{codeblock}

\rSec1 [pathdataitem.changeorigin.cons] {\tcode{path_factory::path_change_origin} constructors and assignment operators}

\indexlibrary{\idxcode{path_factory::path_change_origin}!constructor}
\begin{itemdecl}
    change_origin() noexcept;
\end{itemdecl}
\begin{itemdescr}
	\pnum
	\effects
	Constructs an object of type \tcode{path_factory::path_change_origin}.
	
	\pnum
	\postconditions
	\tcode{_Data == vector_2d(0.0, 0.0)}.
\end{itemdescr}

\indexlibrary{\idxcode{path_factory::path_change_origin}!constructor}
\begin{itemdecl}
    explicit change_origin(const vector_2d& pt) noexcept;
\end{itemdecl}
\begin{itemdescr}
	\pnum
	\effects
	Constructs an object of type \tcode{path_factory::path_change_origin}.
	
	\pnum
	\postconditions
	\tcode{_Data == pt}.
\end{itemdescr}

\rSec1 [pathdataitem.changeorigin.modifiers]{\tcode{path_factory::path_change_origin} modifiers}

\indexlibrary{\idxcode{path_factory::path_change_origin}!\idxcode{origin}}
\indexlibrary{\idxcode{origin}!\idxcode{path_factory::path_change_origin}}
\begin{itemdecl}
    void origin(const vector_2d& value) noexcept;
\end{itemdecl}
\begin{itemdescr}
	\pnum
	\postconditions
	\tcode{_Data == value}.
\end{itemdescr}

\rSec1 [pathdataitem.changeorigin.observers]{\tcode{change_origin} observers}

\indexlibrary{\idxcode{path_factory::path_change_origin}!\idxcode{origin}}
\indexlibrary{\idxcode{origin}!\idxcode{path_factory::path_change_origin}}
\begin{itemdecl}
    vector_2d origin() const noexcept;
\end{itemdecl}
\begin{itemdescr}
	\pnum
	\returns
	\tcode{_Data}.
\end{itemdescr}

\indexlibrary{\idxcode{path_factory::path_move_to}!\idxcode{type}}
\indexlibrary{\idxcode{type}!\idxcode{path_factory::path_move_to}}
\begin{itemdecl}
    virtual path_data_type type() const noexcept override;
\end{itemdecl}
\begin{itemdescr}
	\pnum
	\returns
	\tcode{path_data_type::change_origin}.
\end{itemdescr}

\addtocounter{SectionDepthBase}{-1}
\addtocounter{SectionDepthBase}{-1}
