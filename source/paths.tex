%!TEX root = io2d.tex

\rSec0 [\iotwod.paths] {Paths}

\rSec1 [\iotwod.paths.overview]{Overview of paths}

\pnum
Paths define geometric objects which can be stroked (Table~\ref{tab:\iotwod.surface.rendering.operations}), filled, masked, and used to define a clip area (See: \ref{\iotwod.clipprops.summary}.

\pnum
A path contains zero or more figures.

\pnum
A figure is composed of at least one segment.

\pnum
A figure may contain degenerate segments. When a path is rendered in certain rendering and composing operations, degenerate segments can produce observable behavior.
\begin{example}
When a degenerate segment is rendered in a stroke rendering and composing operation (see \ref{\iotwod.surface.stroking}), the \tcode{line_cap} value contained in its \tcode{stroke_props} argument can result in a degenerate segment producing observable behavior in the form of a circle or square, or some variation thereof.
\end{example}

\pnum
Paths provide vector graphics functionality. As such they are particularly useful in situations where an application is intended to run on a variety of platforms whose output devices (\ref{\iotwod.displaysurface.intro}) span a large gamut of sizes, both in terms of measurement units and in terms of a horizontal and vertical pixel count, in that order.

\pnum
An \tcode{interpreted_path} object is an immutable resource wrapper containing a path (\ref{\iotwod.pathgroup}). An \tcode{interpreted_path} object is created by interpreting the path contained in a \tcode{path_builder} object. It can also be default constructed, in which case the \tcode{interpreted_path} object contains no figures.
\begin{note}
\tcode{interpreted_path} objects provide significant optimization opportunities for implementations. Because they are immutable and opaque, they are intended to be used to store a path in the most efficient representation available.
\end{note}

%!TEX root = io2d.tex

\rSec1 [\iotwod.paths.example]{Path examples (Informative)}

\rSec2 [\iotwod.paths.example.intro] {Overview}

\pnum
Paths are composed of zero or more figures. The following examples show the basics of how paths work in practice.

\pnum
Every example is placed within the following code at the indicated spot. This code is shown here once to avoid repetition:

\begin{codeblock}
#include <@\iotwodheader{}@>

using namespace std;
using namespace @\namespacenoinlinev{}@;

int main() {
  auto imgSfc = make_image_surface(format::argb32, 300, 200);
  brush backBrush{ rgba_color::black };
  brush foreBrush{ rgba_color::white };
  render_props aliased{ antialias::none };
  path_builder pb{};
  imgSfc.paint(backBrush);
  
  // Example code goes here.

  // Example code ends.
  
  imgSfc.save(filesystem::path("example.png"), image_file_format::png);
  return 0;
}
\end{codeblock}

\rSec2 [\iotwod.paths.examples.one] {Example 1}

\pnum
Example 1 consists of a single figure, forming a trapezoid:

\begin{codeblock}
  pb.new_figure({ 80.0f, 20.0f }); // Begins the figure.
  pb.line({ 220.0f, 20.0f }); // Creates a line from the [80, 20] to [220, 20].
  pb.rel_line({ 60.0f, 160.0f }); // Line from [220, 20] to
    // [220 + 60, 160 + 20]. The "to" point is relative to the starting point.
  pb.rel_line({ -260.0f, 0.0f }); // Line from [280, 180] to 
    // [280 - 260, 180 + 0].
  pb.close_figure(); // Creates a line from [20, 180] to [80, 20] 
    // (the new figure point), which makes this a closed figure.
  imgSfc.stroke(foreBrush, pb, nullopt, nullopt, nullopt, aliased);
\end{codeblock}

\begin{importgraphiciotwod}
{Example 1 result}
{fig:pathsexample1}
{pathexample01.png}
\end{importgraphiciotwod}

\FloatBarrier

\rSec2 [\iotwod.paths.examples.two] {Example 2}

\pnum
Example 2 consists of two figures. The first is a rectangular open figure (on the left) and the second is a rectangular closed figure (on the right):

\begin{codeblock}
  pb.new_figure({ 20.0f, 20.0f }); // Begin the first figure.
  pb.rel_line({ 100.0f, 0.0f });
  pb.rel_line({ 0.0f, 160.0f });
  pb.rel_line({ -100.0f, 0.0f });
  pb.rel_line({ 0.0f, -160.0f });
  
  pb.new_figure({ 180.0f, 20.0f }); // End the first figure and begin the 
                                    // second figure.
  pb.rel_line({ 100.0f, 0.0f });
  pb.rel_line({ 0.0f, 160.0f });
  pb.rel_line({ -100.0f, 0.0f });
  pb.close_figure(); // End the second figure.
  imgSfc.stroke(foreBrush, pb, nullopt, stroke_props{ 10.0f }, nullopt, 
    aliased);
\end{codeblock}

\begin{importgraphiciotwod}
{Example 2 result}
{fig:pathsexample2}
{pathexample02.png}
\end{importgraphiciotwod}

\FloatBarrier

\pnum
The resulting image from example 2 shows the difference between an open figure and a closed figure. Each figure begins and ends at the same point. The difference is that with the closed figure, that the rendering of the point where the initial segment and final segment meet is controlled by the \tcode{line_join} value in the \tcode{stroke_props} class, which in this case is the default value of \tcode{line_join::miter}. In the open figure, the rendering of that point receives no special treatment such that each segment at that point is rendered using the \tcode{line_cap} value in the \tcode{stroke_props} class, which in this case is the default value of \tcode{line_cap::none}.

\pnum
That difference between rendering as a \tcode{line_join} versus rendering as two \tcode{line_cap}s is what causes the notch to appear in the open segment. Segments are rendered such that half of the stroke width is rendered on each side of the point being evaluated. With no line cap, each segment begins and ends exactly at the point specified.

\pnum
So for the open figure, the first line begins at \tcode{point_2d\{ 20.0f, 20.0f \}} and the last line ends there. Given the stroke width of \tcode{10.0f}, the visible result for the first line is a rectangle with an upper left corner of \tcode{point_2d\{ 20.0f, 15.0f \}} and a lower right corner of \tcode{point_2d\{ 120.0f, 25.0f \}}. The last line appears as a rectangle with an upper left corner of \tcode{point_2d\{ 15.0f, 20.0f \}} and a lower right corner of \tcode{point_2d\{ 25.0f, 180.0f \}}. This produces the appearance of a square gap between \tcode{point_2d\{ 15.0f, 15.0f \}} and \tcode{point_2d\{20.0f, 20.0f \}}.

\pnum
For the closed figure, adjusting for the coordinate differences, the rendering facts are the same as for the open figure except for one key difference: the point where the first line and last line meet is rendered as a line join rather than two line caps, which, given the default value of \tcode{line_join::miter}, produces a miter, adding that square area to the rendering result.

\rSec2 [\iotwod.paths.examples.three] {Example 3}

\pnum
Example 3 demonstrates open and closed figures each containing either a quadratic curve or a cubic curve.

\begin{codeblock}
pb.new_figure({ 20.0f, 20.0f });
pb.rel_quadratic_curve({ 60.0f, 120.0f }, { 60.0f, -120.0f });
pb.rel_new_figure({ 20.0f, 0.0f });
pb.rel_quadratic_curve({ 60.0f, 120.0f }, { 60.0f, -120.0f });
pb.close_figure();
pb.new_figure({ 20.0f, 150.0f });
pb.rel_cubic_curve({ 40.0f, -120.0f }, { 40.0f, 120.0f * 2.0f },
  { 40.0f, -120.0f });
pb.rel_new_figure({ 20.0f, 0.0f });
pb.rel_cubic_curve({ 40.0f, -120.0f }, { 40.0f, 120.0f * 2.0f },
  { 40.0f, -120.0f });
pb.close_figure();
imgSfc.stroke(foreBrush, pb, nullopt, nullopt, nullopt, aliased);
\end{codeblock}

\begin{importgraphiciotwod}
{Path example 3}
{paths:example3}
{pathexample03.png}
\end{importgraphiciotwod}

\FloatBarrier

\pnum
\begin{note}
\tcode{pb.quadratic_curve(\{ 80.0f, 140.0f \}, \{ 140.0f, 20.0f \});} would be the absolute equivalent of the first curve in example 3.
\end{note}

\rSec2 [\iotwod.paths.examples.four] {Example 4}

\pnum
Example 4 shows how to draw "C++" using figures.

\pnum
For the "C", it is created using an arc. A scaling matrix is used to make it  slightly elliptical. It is also desirable that the arc has a fixed center point, \tcode{point_2d\{ 85.0f, 100.0f \}}. The inverse of the scaling matrix is used in combination with the \tcode{point_for_angle} function to determine the point at which the arc should begin in order to get achieve this fixed center point. The "C" is then stroked.

\pnum
Unlike the "C", which is created using an open figure that is stroked, each "+" is created using a closed figure that is filled. To avoid filling the "C", \tcode{pb.clear();} is called to empty the container. The first "+" is created using a series of lines and is then filled.

\pnum
Taking advantage of the fact that \tcode{path_builder} is a container, rather than create a brand new figure for the second "+", a translation matrix is applied by inserting a \tcode{figure_items::change_matrix} figure item before the \tcode{figure_items::new_figure} object in the existing plus, reverting back to the old matrix immediately after the  and then filling it again.

\begin{codeblock}
// Create the "C".
const matrix_2d scl = matrix_2d::init_scale({ 0.9f, 1.1f });
auto pt = scl.inverse().transform_pt({ 85.0f, 100.0f }) +
  point_for_angle(half_pi<float> / 2.0f, 50.0f);
pb.matrix(scl);
pb.new_figure(pt);
pb.arc({ 50.0f, 50.0f }, three_pi_over_two<float>, half_pi<float> / 2.0f);
imgSfc.stroke(foreBrush, pb, nullopt, stroke_props{ 10.0f });
// Create the first "+".
pb.clear();
pb.new_figure({ 130.0f, 105.0f });
pb.rel_line({ 0.0f, -10.0f });
pb.rel_line({ 25.0f, 0.0f });
pb.rel_line({ 0.0f, -25.0f });
pb.rel_line({ 10.0f, 0.0f });
pb.rel_line({ 0.0f, 25.0f });
pb.rel_line({ 25.0f, 0.0f });
pb.rel_line({ 0.0f, 10.0f });
pb.rel_line({ -25.0f, 0.0f });
pb.rel_line({ 0.0f, 25.0f });
pb.rel_line({ -10.0f, 0.0f });
pb.rel_line({ 0.0f, -25.0f });
pb.close_figure();
imgSfc.fill(foreBrush, pb);
// Create the second "+".
pb.insert(pb.begin(), figure_items::change_matrix(
  matrix_2d::init_translate({ 80.0f, 0.0f })));
imgSfc.fill(foreBrush, pb);
\end{codeblock}

\begin{importgraphiciotwod}
{Path example 4}
{paths:example4}
{pathexample04.png}
\end{importgraphiciotwod}

\FloatBarrier
%
%\rSec2 [\iotwod.paths.examples.five] {Example 5}
%
%\pnum
%Example 5 shows the difference between filling a figure and then stroking it versus stroking that figure then filling it.
%
%\begin{codeblock}
%brush blueBrush{ rgba_color::blue };
%stroke_props ten{ 10.0f };
%pb.new_figure({ 30.0f, 30.0f });
%pb.rel_line({ 105.0f, 0.0f });
%pb.rel_line({ 0.0f, 140.0f });
%pb.rel_line({ -105.0f, 0.0f });
%pb.close_figure();
%imgSfc.stroke(foreBrush, pb, nullopt, ten);
%imgSfc.fill(blueBrush, pb);
%pb.insert(pb.begin(),
%  figure_items::change_matrix(matrix_2d::init_translate({ 135.0f, 0.0f })));
%imgSfc.fill(blueBrush, pb);
%imgSfc.stroke(foreBrush, pb, nullopt, ten);
%\end{codeblock}
%
%\begin{importgraphiciotwod}
%{Path example 5}
%{paths:example5}
%{pathexample05.png}
%\end{importgraphiciotwod}
%
%\FloatBarrier
%
%\pnum
%As can be seen, a figure is filled exactly to the lines of the figure.


\rSec1 [\iotwod.paths.items] {Figure items}

\addtocounter{SectionDepthBase}{2}
%!TEX root = io2d.tex

\rSec0 [\iotwod.paths.figureitems.intro] {Introduction}

\pnum
The nested classes within the class template \tcode{basic_figure_items} describe figure items.

\pnum
A figure begins with an \tcode{abs_new_figure} or \tcode{rel_new_figure} object. A figure ends when:

\begin{itemize}
\item a \tcode{close_figure} object is encountered;
\item a \tcode{abs_new_figure} or \tcode{rel_new_figure} object is encountered; or
\item there are no more figure items in the path.
\end{itemize}

\pnum
The \tcode{basic_path_builder} class is a sequential container that contains a path. It provides a simple interface for building a path but a path can be created using any container that stores \tcode{basic_figure_items::figure_item} objects.

\rSec0 [\iotwod.paths.figureitems.synopsis] {Synopsis}

\begin{codeblock}
namespace std::experimental::io2d::v1 {
  template <class GraphicsSurfaces>
  class basic_figure_items {
  public:
    class abs_new_figure;
    class rel_new_figure;
    class close_figure;
    class abs_matrix;
    class rel_matrix;
    class revert_matrix;
    class abs_cubic_curve;
    class abs_line;
    class abs_quadratic_curve;
    class arc;
    class rel_cubic_curve;
    class rel_line;
    class rel_quadratic_curve;

    using figure_item = variant<abs_cubic_curve, abs_line, abs_matrix,
      abs_new_figure, abs_quadratic_curve, arc, close_figure, rel_cubic_curve, 
      rel_line, rel_matrix, rel_new_figure, rel_quadratic_curve, revert_matrix>;

    class abs_new_figure {
    public:
      // \ref{\iotwod.absnewfigure.ctor}, construct:
      abs_new_figure() noexcept;
      explicit abs_new_figure(const basic_point_2d<typename
        GraphicsSurfaces::graphics_math_type>& pt);
      abs_new_figure(const abs_matrix& other) noexcept;
      abs_new_figure(abs_matrix&& other) noexcept;

      // assign:
      abs_new_figure& operator=(const abs_new_figure& other);
      abs_new_figure& operator=(abs_new_figure&& other) noexcept;
	
      // \ref{\iotwod.absnewfigure.mod}, modifiers:
      void at(const basic_point_2d<typename
        GraphicsSurfaces::graphics_math_type>& pt) noexcept;
	
      // \ref{\iotwod.absnewfigure.obs}, observers:
      basic_point_2d<typename
        GraphicsSurfaces::graphics_math_type> at() const noexcept;
    };

    class rel_new_figure {
    public:
      // \ref{\iotwod.relnewfigure.ctor}, construct:
      rel_new_figure() noexcept;
      explicit rel_new_figure(const basic_point_2d<typename
        GraphicsSurfaces::graphics_math_type>& pt);
      rel_new_figure(const rel_matrix& other) noexcept;
      rel_new_figure(rel_matrix&& other) noexcept;

      // assign:
      rel_new_figure& operator=(const rel_new_figure& other);
      rel_new_figure& operator=(rel_new_figure&& other) noexcept;
	
      // \ref{\iotwod.relnewfigure.mod}, modifiers:
      void at(const basic_point_2d<typename
        GraphicsSurfaces::graphics_math_type>& pt) noexcept;
	
      // \ref{\iotwod.relnewfigure.obs}, observers:
      basic_point_2d<typename
        GraphicsSurfaces::graphics_math_type> at() const noexcept;
    };

    class close_figure {
    public:
      // construct:
      revert_matrix() noexcept;
    };

    class abs_matrix {
    public:
      // \ref{\iotwod.absmatrix.ctor}, construct:
      abs_matrix() noexcept;
      explicit abs_matrix(const basic_matrix_2d<typename
        GraphicsSurfaces::graphics_math_type>& m) noexcept;
      abs_matrix(const abs_matrix& other);
      abs_matrix(abs_matrix&& other) noexcept;

      // assign:
      abs_matrix& operator=(const abs_matrix& other);
      abs_matrix& operator=(abs_matrix&& other) noexcept;

      // \ref{\iotwod.absmatrix.mod}, modifiers:
      void matrix(const basic_matrix_2d<typename
        GraphicsSurfaces::graphics_math_type>& m) noexcept;
	
      // \ref{\iotwod.absmatrix.obs}, observers:
      basic_matrix_2d<typename
        GraphicsSurfaces::graphics_math_type> matrix() const noexcept;
    };

    class rel_matrix {
    public:
      // \ref{\iotwod.relmatrix.ctor}, construct:
      rel_matrix() noexcept;
      explicit rel_matrix(const basic_matrix_2d<typename
        GraphicsSurfaces::graphics_math_type>& m) noexcept;
      rel_matrix(const rel_matrix& other);
      rel_matrix(rel_matrix&& other) noexcept;

      // assign:
      rel_matrix& operator=(const rel_matrix& other);
      rel_matrix& operator=(rel_matrix&& other) noexcept;

      // \ref{\iotwod.relmatrix.mod}, modifiers:
      void matrix(const basic_matrix_2d<typename
        GraphicsSurfaces::graphics_math_type>& m) noexcept;
	
      // \ref{\iotwod.relmatrix.obs}, observers:
      basic_matrix_2d<typename
        GraphicsSurfaces::graphics_math_type> matrix() const noexcept;
    };

    class revert_matrix {
    public:
      // construct:
      revert_matrix() noexcept;
    };

    class abs_line {
    public:
      // \ref{\iotwod.absline.ctor}, construct:
      abs_line() noexcept;
      explicit abs_line(const basic_point_2d<typename
        GraphicsSurfaces::graphics_math_type>& pt) noexcept;
      abs_line(const abs_line& other);
      abs_line(abs_line&& other) noexcept;

      // assign:
      abs_line& operator=(const abs_line& other);
      abs_line& operator=(abs_line&& other) noexcept;

      // \ref{\iotwod.absline.mod}, modifiers:
      void to(const basic_point_2d<typename GraphicsSurfaces::graphics_math_type>& pt) noexcept;

      // \ref{\iotwod.absline.obs}, observers:
      basic_point_2d<typename GraphicsSurfaces::graphics_math_type> to() const noexcept;
    };

    class rel_line {
    public:
      // \ref{\iotwod.relline.ctor}, construct:
      rel_line() noexcept;
      explicit rel_line(const basic_point_2d<typename
        GraphicsSurfaces::graphics_math_type>& pt) noexcept;
      rel_line(const rel_line& other);
      rel_line(rel_line&& other) noexcept;

      // assign:
      rel_line& operator=(rel_line&& other) noexcept;
      rel_line& operator=(const rel_line& other);

      // \ref{\iotwod.relline.mod}, modifiers:
      void to(const basic_point_2d<typename GraphicsSurfaces::graphics_math_type>& pt) noexcept;

      // \ref{\iotwod.relline.obs}, observers:
      basic_point_2d<typename GraphicsSurfaces::graphics_math_type> to() const noexcept;
    };

    class abs_quadratic_curve {
    public:
      // \ref{\iotwod.absquadraticcurve.ctor}, construct:
      abs_quadratic_curve() noexcept;
      abs_quadratic_curve(const basic_point_2d<typename
        GraphicsSurfaces::graphics_math_type>& cpt, const basic_point_2d<typename
        GraphicsSurfaces::graphics_math_type>& ept) noexcept;
      abs_quadratic_curve(const abs_quadratic_curve& other);
      abs_quadratic_curve(abs_quadratic_curve&& other) noexcept;

      // assign:
      abs_quadratic_curve& operator=(abs_quadratic_curve&& other) noexcept;
      abs_quadratic_curve& operator=(const abs_quadratic_curve& other);

      // \ref{\iotwod.absquadraticcurve.mod}, modifiers:
      void control_pt(const basic_point_2d<typename
        GraphicsSurfaces::graphics_math_type>& cpt) noexcept;
      void end_pt(const basic_point_2d<typename
        GraphicsSurfaces::graphics_math_type>& ept) noexcept;

      // \ref{\iotwod.absquadraticcurve.obs}, observers:
      basic_point_2d<typename GraphicsSurfaces::graphics_math_type> control_pt() const noexcept;
      basic_point_2d<typename GraphicsSurfaces::graphics_math_type> end_pt() const noexcept;
    };

    class rel_quadratic_curve {
    public:
      // \ref{\iotwod.relquadraticcurve.ctor}, construct:
      rel_quadratic_curve() noexcept;
      rel_quadratic_curve(const basic_point_2d<typename
        GraphicsSurfaces::graphics_math_type>& cpt, const basic_point_2d<typename
        GraphicsSurfaces::graphics_math_type>& ept) noexcept;
      rel_quadratic_curve(const rel_quadratic_curve& other);
      rel_quadratic_curve(rel_quadratic_curve&& other) noexcept;

      // assign:
      rel_quadratic_curve& operator=(rel_quadratic_curve&& other) noexcept;
      rel_quadratic_curve& operator=(const rel_quadratic_curve& other);

      // \ref{\iotwod.relquadraticcurve.mod}, modifiers:
      void control_pt(const basic_point_2d<typename
        GraphicsSurfaces::graphics_math_type>& cpt) noexcept;
      void end_pt(const basic_point_2d<typename
        GraphicsSurfaces::graphics_math_type>& ept) noexcept;

      // \ref{\iotwod.relquadraticcurve.obs}, observers:
      basic_point_2d<typename GraphicsSurfaces::graphics_math_type> control_pt() const noexcept;
      basic_point_2d<typename GraphicsSurfaces::graphics_math_type> end_pt() const noexcept;
    };

    class abs_cubic_curve {
    public:
      // \ref{\iotwod.abscubiccurve.ctor}, construct:
      abs_cubic_curve() noexcept;
      abs_cubic_curve(const basic_point_2d<typename GraphicsSurfaces::graphics_math_type>& cpt1,
        const basic_point_2d<typename GraphicsSurfaces::graphics_math_type>& cpt2,
        const basic_point_2d<typename GraphicsSurfaces::graphics_math_type>& ept) noexcept;
      abs_cubic_curve(const abs_cubic_curve& other);
      abs_cubic_curve(abs_cubic_curve&& other) noexcept;

      // assign:
      abs_cubic_curve& operator=(const abs_cubic_curve& other);
      abs_cubic_curve& operator=(abs_cubic_curve&& other) noexcept;

      // \ref{\iotwod.abscubiccurve.mod}, modifiers:
      void control_pt1(const basic_point_2d<typename
        GraphicsSurfaces::graphics_math_type>& cpt) noexcept;
      void control_pt2(const basic_point_2d<typename
        GraphicsSurfaces::graphics_math_type>& cpt) noexcept;
      void end_pt(const basic_point_2d<typename
        GraphicsSurfaces::graphics_math_type>& ept) noexcept;

      // \ref{\iotwod.abscubiccurve.obs}, observers:
      basic_point_2d<typename GraphicsSurfaces::graphics_math_type> control_pt1() const noexcept;
      basic_point_2d<typename GraphicsSurfaces::graphics_math_type> control_pt2() const noexcept;
      basic_point_2d<typename GraphicsSurfaces::graphics_math_type> end_pt() const noexcept;
    };

    class rel_cubic_curve {
    public:
      // \ref{\iotwod.relcubiccurve.ctor}, construct:
      rel_cubic_curve() noexcept;
      rel_cubic_curve(const basic_point_2d<typename GraphicsSurfaces::graphics_math_type>& cpt1,
      const basic_point_2d<typename GraphicsSurfaces::graphics_math_type>& cpt2,
      const basic_point_2d<typename GraphicsSurfaces::graphics_math_type>& ept) noexcept;
      rel_cubic_curve(const rel_cubic_curve& other);
      rel_cubic_curve(rel_cubic_curve&& other) noexcept;

      // assign:
      rel_cubic_curve& operator=(const rel_cubic_curve& other);
      rel_cubic_curve& operator=(rel_cubic_curve&& other) noexcept;

      // \ref{\iotwod.relcubiccurve.mod}, modifiers:
      void control_pt1(const basic_point_2d<typename
        GraphicsSurfaces::graphics_math_type>& cpt) noexcept;
      void control_pt2(const basic_point_2d<typename
        GraphicsSurfaces::graphics_math_type>& cpt) noexcept;
      void end_pt(const basic_point_2d<typename
        GraphicsSurfaces::graphics_math_type>& ept) noexcept;

      // \ref{\iotwod.relcubiccurve.obs}, observers:
      basic_point_2d<typename GraphicsSurfaces::graphics_math_type> control_pt1() const noexcept;
      basic_point_2d<typename GraphicsSurfaces::graphics_math_type> control_pt2() const noexcept;
      basic_point_2d<typename GraphicsSurfaces::graphics_math_type> end_pt() const noexcept;
    };

    class arc {
    public:
      // \ref{\iotwod.arc.ctor}, construct:
      arc() noexcept;
      arc(const basic_point_2d<typename GraphicsSurfaces::graphics_math_type>& rad, float rot, float sang) noexcept;
      arc(const arc& other);
      arc(arc&& other) noexcept;

      // assign:
      arc& operator=(const arc& other);
      arc& operator=(arc&& other) noexcept;

      // \ref{\iotwod.arc.mod}, modifiers:
      void radius(const basic_point_2d<typename GraphicsSurfaces::graphics_math_type>& rad) noexcept;
      void rotation(float rot) noexcept;
      void start_angle(float sang) noexcept;

      // \ref{\iotwod.arc.obs}, observers:
      basic_point_2d<typename GraphicsSurfaces::graphics_math_type> radius() const noexcept;
      float rotation() const noexcept;
      float start_angle() const noexcept;
      basic_point_2d<typename GraphicsSurfaces::graphics_math_type> center(const basic_point_2d<typename
        GraphicsSurfaces::graphics_math_type>& cpt, const basic_matrix_2d<typename
        GraphicsSurfaces::graphics_math_type>& m = basic_matrix_2d<typename
        GraphicsSurfaces::graphics_math_type>{}) const noexcept;
      basic_point_2d<typename GraphicsSurfaces::graphics_math_type> end_pt(const basic_point_2d<typename
        GraphicsSurfaces::graphics_math_type>& cpt, const basic_matrix_2d<typename
        GraphicsSurfaces::graphics_math_type>& m = basic_matrix_2d<typename
        GraphicsSurfaces::graphics_math_type>{}) const noexcept;
    };

    using figure_item = variant<abs_cubic_curve, abs_line, abs_matrix, abs_new_figure,
      abs_quadratic_curve, arc, close_figure, rel_cubic_curve, rel_line, rel_matrix,
      rel_new_figure, rel_quadratic_curve, revert_matrix>;
  };

  // \ref{\iotwod.absmatrix.eq}, abs_matrix operators:
  template <class GraphicsSurfaces>
  bool operator==(
    const typename basic_figure_items<GraphicsSurfaces>::abs_matrix& lhs,
    const typename basic_figure_items<GraphicsSurfaces>::abs_matrix& rhs) 
    noexcept;
  template <class GraphicsSurfaces>
  bool operator!=(
    const typename basic_figure_items<GraphicsSurfaces>::abs_matrix& lhs,
    const typename basic_figure_items<GraphicsSurfaces>::abs_matrix& rhs) 
    noexcept;

  // \ref{\iotwod.relmatrix.eq}, rel_matrix operators:
  template <class GraphicsSurfaces>
  bool operator==(
    const typename basic_figure_items<GraphicsSurfaces>::rel_matrix& lhs,
    const typename basic_figure_items<GraphicsSurfaces>::rel_matrix& rhs) 
    noexcept;
  template <class GraphicsSurfaces>
  bool operator!=(
    const typename basic_figure_items<GraphicsSurfaces>::rel_matrix& lhs,
    const typename basic_figure_items<GraphicsSurfaces>::rel_matrix& rhs) 
    noexcept;

  // \ref{\iotwod.revertmatrix.eq}, revert_matrix operators:
  template <class GraphicsSurfaces>
  bool operator==(
    const typename basic_figure_items<GraphicsSurfaces>::revert_matrix& lhs,
    const typename basic_figure_items<GraphicsSurfaces>::revert_matrix& rhs) 
    noexcept;
  template <class GraphicsSurfaces>
  bool operator!=(
    const typename basic_figure_items<GraphicsSurfaces>::revert_matrix& lhs,
    const typename basic_figure_items<GraphicsSurfaces>::revert_matrix& rhs) 
    noexcept;

  // \ref{\iotwod.absline.eq}, abs_line operators:
  template <class GraphicsSurfaces>
  bool operator==(
    const typename basic_figure_items<GraphicsSurfaces>::abs_line& lhs,
    const typename basic_figure_items<GraphicsSurfaces>::abs_line& rhs) 
    noexcept;
  template <class GraphicsSurfaces>
  bool operator!=(
    const typename basic_figure_items<GraphicsSurfaces>::abs_line& lhs,
    const typename basic_figure_items<GraphicsSurfaces>::abs_line& rhs) 
    noexcept;

  // \ref{\iotwod.relline.eq}, rel_line operators:
  template <class GraphicsSurfaces>
  bool operator==(
    const typename basic_figure_items<GraphicsSurfaces>::rel_line& lhs,
    const typename basic_figure_items<GraphicsSurfaces>::rel_line& rhs) 
    noexcept;
  template <class GraphicsSurfaces>
  bool operator!=(
    const typename basic_figure_items<GraphicsSurfaces>::rel_line& lhs,
    const typename basic_figure_items<GraphicsSurfaces>::rel_line& rhs) 
    noexcept;

  // \ref{\iotwod.absquadraticcurve.eq}, abs_quadratic_curve operators:
  template <class GraphicsSurfaces>
  bool operator==(const typename
    basic_figure_items<GraphicsSurfaces>::abs_quadratic_curve& lhs,
    const typename basic_figure_items<GraphicsSurfaces>::abs_quadratic_curve& 
    rhs) noexcept;
  template <class GraphicsSurfaces>
  bool operator!=(const typename 
    basic_figure_items<GraphicsSurfaces>::abs_quadratic_curve& lhs,
    const typename basic_figure_items<GraphicsSurfaces>::abs_quadratic_curve& 
    rhs) noexcept;

  // \ref{\iotwod.relquadraticcurve.eq}, rel_quadratic_curve operators:
  template <class GraphicsSurfaces>
  bool operator==(const typename 
    basic_figure_items<GraphicsSurfaces>::rel_quadratic_curve& lhs,
    const typename basic_figure_items<GraphicsSurfaces>::rel_quadratic_curve& 
    rhs) noexcept;
  template <class GraphicsSurfaces>
  bool operator!=(const typename 
    basic_figure_items<GraphicsSurfaces>::rel_quadratic_curve& lhs,
    const typename basic_figure_items<GraphicsSurfaces>::rel_quadratic_curve& 
    rhs) noexcept;

  // \ref{\iotwod.abscubiccurve.eq}, abs_cubic_curve operators:
  template <class GraphicsSurfaces>
  bool operator==(const typename 
    basic_figure_items<GraphicsSurfaces>::abs_cubic_curve& lhs,
    const typename basic_figure_items<GraphicsSurfaces>::abs_cubic_curve& rhs) 
    noexcept;
  template <class GraphicsSurfaces>
  bool operator!=(const typename 
    basic_figure_items<GraphicsSurfaces>::abs_cubic_curve& lhs,
    const typename basic_figure_items<GraphicsSurfaces>::abs_cubic_curve& rhs) 
    noexcept;

  // \ref{\iotwod.relcubiccurve.eq}, rel_cubic_curve operators:
  template <class GraphicsSurfaces>
  bool operator==(const typename 
    basic_figure_items<GraphicsSurfaces>::rel_cubic_curve& lhs,
    const typename basic_figure_items<GraphicsSurfaces>::rel_cubic_curve& rhs) 
    noexcept;
  template <class GraphicsSurfaces>
  bool operator!=(const typename 
    basic_figure_items<GraphicsSurfaces>::rel_cubic_curve& lhs,
    const typename basic_figure_items<GraphicsSurfaces>::rel_cubic_curve& rhs) 
    noexcept;

  // \ref{\iotwod.arc.eq}, arc operators:
  template <class GraphicsSurfaces>
  bool operator==(const typename basic_figure_items<GraphicsSurfaces>::arc& lhs,
    const typename basic_figure_items<GraphicsSurfaces>::arc& rhs) noexcept;
  template <class GraphicsSurfaces>
  bool operator!=(const typename basic_figure_items<GraphicsSurfaces>::arc& lhs,
    const typename basic_figure_items<GraphicsSurfaces>::arc& rhs) noexcept;
}
\end{codeblock}
%!TEX root = io2d.tex
\rSec0 [\iotwod.absnewfigure] {Class \tcode{abs_new_figure}}

\pnum
\indexlibrary{\idxcode{abs_new_figure}}%
The class \tcode{abs_new_figure} describes a figure item that is a new figure command.

\pnum
It has an \term{at point} of type \tcode{point_2d}.

\rSec1 [\iotwod.absnewfigure.cons] {\tcode{abs_new_figure} constructors and assignment operators}%

\indexlibrary{\idxcode{abs_new_figure}!constructor}%
\begin{itemdecl}
abs_new_figure();
\end{itemdecl}
\begin{itemdescr}
\pnum
\effects
Equivalent to: \tcode{abs_new_figure\{basic_point_2d<typename GraphicsSurfaces::graphics_math_type>()\};}
\end{itemdescr}

\indexlibrary{\idxcode{abs_new_figure}!constructor}%
\begin{itemdecl}
explicit abs_new_figure(const basic_point_2d<typename
  GraphicsSurfaces::graphics_math_type>& pt);
\end{itemdecl}
\begin{itemdescr}
\pnum
\effects
Constructs an object of type \tcode{abs_new_figure}.

\pnum
The at point is \tcode{pt}.
\end{itemdescr}

\indexlibrary{\idxcode{abs_new_figure}!constructor}%
\begin{itemdecl}
abs_new_figure(const abs_new_figure& other);
abs_new_figure(abs_new_figure&& other) noexcept;
\end{itemdecl}
\begin{itemdescr}
\pnum
\effects
Constructs an object of type \tcode{abs_new_figure}. In the second form, other is left in a valid state with an unspecified value.
	
\pnum
The at point is \tcode{other.at()}.
\end{itemdescr}

\indexlibrary{\idxcode{abs_new_figure}!assignment}%
\begin{itemdecl}
abs_new_figure& operator=(const abs_new_figure& other);
\end{itemdecl}
\begin{itemdescr}
\pnum
\effects
If \tcode{*this} and \tcode{other} are not the same object, modifies \tcode{*this} such that \tcode{*this.at()} is \tcode{other.at()}

\pnum
If \tcode{*this} and \tcode{other} are the same object, the member has no effect.
	
\pnum
\returns
\tcode{*this}
\end{itemdescr}

\indexlibrary{\idxcode{abs_new_figure}!assignment}%
\begin{itemdecl}
abs_new_figure& operator=(abs_new_figure&& other) noexcept;
\end{itemdecl}
\begin{itemdescr}
\pnum
\effects
<TODO>

\pnum
\returns
\tcode{*this}
\end{itemdescr}

\rSec1 [\iotwod.absnewfigure.modifiers]{\tcode{abs_new_figure} modifiers}%

\indexlibrarymember{at}{abs_new_figure}%
\begin{itemdecl}
void at(const basic_point_2d<typename GraphicsSurfaces::graphics_math_type>& pt) noexcept;
\end{itemdecl}
\begin{itemdescr}
\pnum
\effects
The at point is \tcode{pt}.
\end{itemdescr}

\rSec1 [\iotwod.absnewfigure.observers]{\tcode{abs_new_figure} observers}%

\indexlibrarymember{at}{abs_new_figure}%
\begin{itemdecl}
basic_point_2d<typename GraphicsSurfaces::graphics_math_type> at() const noexcept;
\end{itemdecl}
\begin{itemdescr}
\pnum
\returns
The at point.
\end{itemdescr}

\rSec1 [\iotwod.absnewfigure.ops]{\tcode{abs_new_figure} operators}%

\indexlibrarymember{operator==}{abs_new_figure}%
\begin{itemdecl}
bool operator==(const abs_new_figure& lhs, const abs_new_figure& rhs) noexcept;
\end{itemdecl}
\begin{itemdescr}
\pnum
\returns
\tcode{lhs.at() == rhs.at()}.
\end{itemdescr}

%!TEX root = io2d.tex
\rSec0 [\iotwod.relnewpath] {Class \tcode{rel_new_path}}%

\pnum
\indexlibrary{\idxcode{rel_new_path}}%
The class \tcode{rel_new_path} describes a path item that creates a new path and makes the previous path an open path unless it was made a closed path by a \tcode{close_path} object.

\pnum
This is a relative path item.

\pnum
It has an \term{at point} of type \tcode{vector_2d}.

\pnum
When interpreting a path group, the path's last-move-to-point and current point are set to the value of the at point added to the path's current point.

\rSec1 [\iotwod.relnewpath.synopsis] {\tcode{rel_new_path} synopsis}%

\begin{codeblock}
namespace std::experimental::io2d::v1 {
  namespace path_data {
    class rel_new_path {
    public:
      // \ref{\iotwod.relnewpath.cons}, construct:
      constexpr rel_new_path() noexcept;
      constexpr explicit rel_new_path(const vector_2d& pt) noexcept;

      // \ref{\iotwod.relnewpath.modifiers}, modifiers:
      constexpr void at(const vector_2d& pt) noexcept;

      // \ref{\iotwod.relnewpath.observers}, observers:
      constexpr vector_2d at() const noexcept;
    };
    
    // \ref{\iotwod.relnewpath.nonmember}, non-members:
    bool operator==(const rel_new_path& lhs, const rel_new_path& rhs) noexcept;
    bool operator!=(const rel_new_path& lhs, const rel_new_path& rhs) noexcept;
  }
}
\end{codeblock}

\rSec1 [\iotwod.relnewpath.cons] {\tcode{rel_new_path} constructors}%

\indexlibrary{\idxcode{rel_new_path}!constructor}%
\begin{itemdecl}
constexpr rel_new_path() noexcept;
\end{itemdecl}
\begin{itemdescr}
\pnum
\effects
Equivalent to: \tcode{rel_new_path\{ vector_2d() \};}
\end{itemdescr}

\indexlibrary{\idxcode{rel_new_path}!constructor}%
\begin{itemdecl}
constexpr explicit rel_new_path(const vector_2d& pt) noexcept;
\end{itemdecl}
\begin{itemdescr}
\pnum
\effects
Constructs an object of type \tcode{rel_new_path}.

\pnum
The at point is \tcode{pt}.
\end{itemdescr}

\rSec1 [\iotwod.relnewpath.modifiers]{\tcode{rel_new_path} modifiers}%

\indexlibrarymember{at}{rel_new_path}%
\begin{itemdecl}
constexpr void at(const vector_2d& pt) noexcept;
\end{itemdecl}
\begin{itemdescr}
\pnum
\effects
The at point is \tcode{pt}.
\end{itemdescr}

\rSec1 [\iotwod.relnewpath.observers]{\tcode{rel_new_path} observers}%

\indexlibrarymember{at}{rel_new_path}%
\begin{itemdecl}
constexpr vector_2d at() const noexcept;
\end{itemdecl}
\begin{itemdescr}
\pnum
\returns
The at point.
\end{itemdescr}

\rSec1 [\iotwod.relnewpath.nonmember]{Non-member functions}%

\indexlibrarymember{operator==}{rel_new_path}%
\begin{itemdecl}
constexpr bool operator==(const rel_new_path& lhs, const rel_new_path& rhs) 
  noexcept;
\end{itemdecl}
\begin{itemdescr}
\pnum
\returns
\tcode{lhs.at() == rhs.at()}.
\end{itemdescr}

\indexlibrarymember{operator!=}{rel_new_path}%
\begin{itemdecl}
constexpr bool operator!=(const rel_new_path& lhs, const rel_new_path& rhs) 
  noexcept;
\end{itemdecl}
\begin{itemdescr}
\pnum
\returns
\tcode{!(lhs == rhs)}.
\end{itemdescr}

%!TEX root = io2d.tex
\rSec0 [closepath] {Class \tcode{close_path}}

\pnum
\indexlibrary{\idxcode{close_path}}
This class is a path instruction that creates a closed path within a path group.

\pnum
It has an end point of type \tcode{vector_2d}.

\rSec1 [closepath.synopsis] {\tcode{close_path} synopsis}

\begin{codeblock}
namespace std { namespace experimental { namespace io2d { inline namespace v1 {
  namespace path_data {
    class close_path {
      // \ref{closepath.cons}, construct:
      constexpr close_path() noexcept;
      constexpr explicit close_path(const vector_2d& to) noexcept;

      // \ref{closepath.modifiers}, modifiers:
      constexpr void to(const vector_2d& value) noexcept;

      // \ref{closepath.observers}, observers:
      constexpr vector_2d to() const noexcept;
    };
  };
} } } }
\end{codeblock}

\rSec1 [closepath.cons] {\tcode{close_path} constructors}

\indexlibrary{\idxcode{close_path}!constructor}
\begin{itemdecl}
constexpr close_path() noexcept;
\end{itemdecl}
\begin{itemdescr}
\pnum
\effects
Constructs an object of type \tcode{close_path}.

\pnum
The end point shall be set to the value of \tcode{vector_2d\{\}}.
\end{itemdescr}

\indexlibrary{\idxcode{close_path}!constructor}
\begin{itemdecl}
constexpr explicit close_path(const vector_2d& pt) noexcept;
\end{itemdecl}
\begin{itemdescr}
\pnum
\effects
Constructs an object of type \tcode{close_path}.

\pnum
The end point shall be set to the value of \tcode{pt}.
\end{itemdescr}

\rSec1 [closepath.modifiers]{\tcode{abs_move} modifiers}

\indexlibrary{\idxcode{close_path}!\idxcode{to}}
\begin{itemdecl}
constexpr void to(const vector_2d& pt) noexcept;
\end{itemdecl}
\begin{itemdescr}
\pnum
\effects
The end point shall be set to the value of \tcode{pt}.
\end{itemdescr}

\rSec1 [closepath.observers]{\tcode{abs_move} observers}

\indexlibrary{\idxcode{close_path}!\idxcode{to}}
\begin{itemdecl}
constexpr vector_2d to() const noexcept;
\end{itemdecl}
\begin{itemdescr}
\pnum
\returns
The value of the end point.
\end{itemdescr}

%!TEX root = io2d.tex
\rSec0 [\iotwod.absmatrix] {Class \tcode{abs_matrix}}%

\pnum
\indexlibrary{\idxcode{abs_matrix}}%
The class \tcode{abs_matrix} describes a figure item that is a path command.

\pnum
It has a transform matrix of type \tcode{basic_matrix_2d}.

\rSec1 [\iotwod.absmatrix.cons] {\tcode{abs_matrix} constructors and assignment operators}

\indexlibrary{\idxcode{abs_matrix}!constructor}%
\begin{itemdecl}
abs_matrix() noexcept;
\end{itemdecl}
\begin{itemdescr}
\pnum
\effects
Equivalent to: \tcode{abs_matrix\{ basic_matrix_2d() \};}
\end{itemdescr}

\indexlibrary{\idxcode{abs_matrix}!constructor}%
\begin{itemdecl}
explicit abs_matrix(const basic_matrix_2d<typename
  GraphicsSurfaces::graphics_math_type>& m) noexcept;
\end{itemdecl}
\begin{itemdescr}
\pnum
\requires
\tcode{m.is_invertible()} is \tcode{true}.

\pnum
\effects
Constructs an object of type \tcode{abs_matrix}.

\pnum
The transform matrix is \tcode{m}.
\end{itemdescr}

\indexlibrary{\idxcode{abs_matrix}!constructor}%
\begin{itemdecl}
abs_matrix(const abs_matrix& other);
abs_matrix(abs_matrix&& other) noexcept;
\end{itemdecl}
\begin{itemdescr}
\pnum
\effects
Constructs an object of type \tcode{abs_matrix}. In the second form, other is left in a valid state with an unspecified value.
\end{itemdescr}

\indexlibrary{\idxcode{abs_matrix}!assignment}%
\begin{itemdecl}
abs_matrix& operator=(const abs_matrix& other);
\end{itemdecl}
\begin{itemdescr}
\pnum
\effects
If \tcode{*this} and \tcode{other} are not the same object, modifies \tcode{*this} such that \tcode{*this.matrix()} is \tcode{other.matrix()}

\pnum
If \tcode{*this} and \tcode{other} are the same object, the member has no effect.

\pnum
\returns
\tcode{*this}
\end{itemdescr}

\indexlibrary{\idxcode{abs_matrix}!assignment}%
\begin{itemdecl}
abs_matrix& operator=(abs_matrix&& other) noexcept;
\end{itemdecl}
\begin{itemdescr}
\pnum
\effects
<TODO>

\pnum
\returns
\tcode{*this}
\end{itemdescr}

\rSec1 [\iotwod.absmatrix.modifiers]{\tcode{abs_matrix} modifiers}

\indexlibrarymember{matrix}{abs_matrix}%
\begin{itemdecl}
void matrix(const basic_matrix_2d<typename GraphicsSurfaces::graphics_math_type>& m) noexcept;
\end{itemdecl}
\begin{itemdescr}
\pnum
\requires
\tcode{m.is_invertible()} is \tcode{true}.

\pnum
\effects
The transform matrix is \tcode{m}.
\end{itemdescr}

\rSec1 [\iotwod.absmatrix.observers]{\tcode{abs_matrix} observers}

\indexlibrarymember{matrix}{abs_matrix}%
\begin{itemdecl}
basic_matrix_2d<typename GraphicsSurfaces::graphics_math_type> matrix() const noexcept;
\end{itemdecl}
\begin{itemdescr}
\pnum
\returns
The transform matrix.
\end{itemdescr}

\rSec1 [\iotwod.absmatrix.ops]{\tcode{abs_matrix} operators}

\indexlibrarymember{operator==}{abs_matrix}%
\begin{itemdecl}
template <class GraphicsSurfaces>
bool operator==(const typename basic_figure_items<GraphicsSurfaces>::abs_matrix& lhs,
  const typename basic_figure_items<GraphicsSurfaces>::abs_matrix& rhs) noexcept;
\end{itemdecl}
\begin{itemdescr}
\pnum
\returns
\tcode{lhs.matrix() == rhs.matrix()}.
\end{itemdescr}

%!TEX root = io2d.tex
\rSec0 [\iotwod.relmatrix] {Class template \tcode{basic_figure_items<GraphicsSurfaces>::rel_matrix}}

\rSec1 [\iotwod.relmatrix.intro] {Overview}

\pnum
\indexlibrary{\idxcode{rel_matrix}}%
The class template \tcode{basic_figure_items<GraphicsSurfaces>::rel_matrix} describes a figure item that is a path command.

\pnum
It has a transform matrix of type \tcode{basic_matrix_2d<GraphicsSurfaces::graphics_math_type>}.

\pnum
The data are stored in an object of type \tcode{typename GraphicsSurfaces::paths::rel_matrix_data_type}. It is accessible using the \tcode{data} member functions.

\rSec1 [\iotwod.relmatrix.synopsis] {Synopsis}
\begin{codeblock}
namespace std::experimemtal::io2d::v1 {
  template <class GraphicsSurfaces>
  class basic_figure_items<GraphicsSurfaces>::rel_matrix {
  public:
    using graphics_math_type = typename GraphicsSurfaces::graphics_math_type;
    using data_type =
      typename GraphicsSurfaces::paths::rel_matrix_data_type;

    // \ref{\iotwod.relmatrix.ctor}, construct:
    rel_matrix();
    explicit rel_matrix(const basic_matrix_2d<graphics_math_type>& m) noexcept;
    rel_matrix(const rel_matrix& other) = default;
    rel_matrix(rel_matrix&& other) noexcept = default;

    // assign:
    rel_matrix& operator=(const rel_matrix& other) = default;
    rel_matrix& operator=(rel_matrix&& other) noexcept = default;

    // \ref{\iotwod.relmatrix.acc}, accessors:
    const data_type& data() const noexcept;
    data_type& data() noexcept;

    // \ref{\iotwod.relmatrix.mod}, modifiers:
    void matrix(const basic_matrix_2d<graphics_math_type>& m) noexcept;

    // \ref{\iotwod.relmatrix.obs}, observers:
    basic_matrix_2d<graphics_math_type> matrix() const noexcept;
  };
  
  // \ref{\iotwod.relmatrix.eq}, equality operators:
  template <class GraphicsSurfaces>
  bool operator==(
    const typename basic_figure_items<GraphicsSurfaces>::rel_matrix& lhs,
    const typename basic_figure_items<GraphicsSurfaces>::rel_matrix& rhs) 
    noexcept;  
  template <class GraphicsSurfaces>
  bool operator!=(
    const typename basic_figure_items<GraphicsSurfaces>::rel_matrix& lhs,
    const typename basic_figure_items<GraphicsSurfaces>::rel_matrix& rhs) 
    noexcept;  
}
\end{codeblock}

\rSec1 [\iotwod.relmatrix.ctor] {Constructors}%

\indexlibrary{\idxcode{rel_matrix}!constructor}%
\begin{itemdecl}
rel_matrix() noexcept;
\end{itemdecl}
\begin{itemdescr}
\pnum
\effects Equivalent to: \tcode{rel_matrix\{ basic_matrix_2d() \};} 

\pnum
\postconditions \tcode{data() == GraphicsSurfaces::paths::create_rel_matrix()}.
\end{itemdescr}

\indexlibrary{\idxcode{rel_matrix}!constructor}%
\begin{itemdecl}
explicit rel_matrix(const basic_matrix_2d<graphics_math_type>& m) noexcept;
\end{itemdecl}
\begin{itemdescr}
\pnum
\requires \tcode{m.is_invertible()} is \tcode{true}.

\pnum
\effects Constructs an object of type \tcode{rel_matrix}.

\pnum
\remarks The transform matrix is \tcode{m}.
\end{itemdescr}

\rSec1 [\iotwod.relmatrix.acc] {Accessors}%

\indexlibrarymember{data}{rel_matrix}%
\begin{itemdecl}
const data_type& data() const noexcept;
data_type& data() noexcept;
\end{itemdecl}
\begin{itemdescr}
\pnum
\returns A reference to the \tcode{rel_matrix} object's data object (See: \ref{\iotwod.relmatrix.intro}).
\end{itemdescr}

\rSec1 [\iotwod.relmatrix.mod] {Modifiers}%

\indexlibrarymember{matrix}{rel_matrix}%
\begin{itemdecl}
void matrix(const basic_matrix_2d<tgraphics_math_type>& m) noexcept;
\end{itemdecl}
\begin{itemdescr}
\pnum
\requires \tcode{m.is_invertible()} is \tcode{true}.

\pnum
\effects The transform matrix is \tcode{m}.
\end{itemdescr}

\rSec1 [\iotwod.relmatrix.obs]{Observers}%

\indexlibrarymember{matrix}{rel_matrix}%
\begin{itemdecl}
basic_matrix_2d<typename GraphicsSurfaces::graphics_math_type> matrix() const noexcept;
\end{itemdecl}
\begin{itemdescr}
\pnum
\returns The transform matrix.
\end{itemdescr}

\rSec1 [\iotwod.relmatrix.eq] {Equality operators}%

\indexlibrarymember{operator==}{rel_matrix}%
\begin{itemdecl}
template <class GraphicsSurfaces>
bool operator==(
  const typename basic_figure_items<GraphicsSurfaces>::rel_matrix& lhs,
  const typename basic_figure_items<GraphicsSurfaces>::rel_matrix& rhs)
  noexcept;
\end{itemdecl}
\begin{itemdescr}
\pnum
\returns \tcode{lhs.matrix() == rhs.matrix()}.
\end{itemdescr}

\indexlibrarymember{operator!=}{rel_matrix}%
\begin{itemdecl}
template <class GraphicsSurfaces>
bool operator!=(
  const typename basic_figure_items<GraphicsSurfaces>::rel_matrix& lhs,
  const typename basic_figure_items<GraphicsSurfaces>::rel_matrix& rhs)
  noexcept;
\end{itemdecl}
\begin{itemdescr}
\pnum
\returns \tcode{lhs.matrix() != rhs.matrix()}.
\end{itemdescr}

%!TEX root = io2d.tex
\rSec0 [\iotwod.revertmatrix] {Class template \tcode{basic_figure_items<GraphicsSurfaces>::revert_matrix}}

\rSec1 [\iotwod.revertmatrix.intro] {Overview}

\pnum
\indexlibrary{\idxcode{revert_matrix}}
The class template \tcode{basic_figure_items<GraphicsSurfaces>::revert_matrix} describes a figure item that is a path command.

\rSec1 [\iotwod.revertmatrix.synopsis] {Synopsis}
\begin{codeblock}
namespace std::experimemtal::io2d::v1 {
  template <class GraphicsSurfaces>
  class basic_figure_items<GraphicsSurfaces>::revert_matrix {
  public:
    // construct:
    revert_matrix() = default;
    revert_matrix(const revert_matrix& other) = default;
    revert_matrix(revert_matrix&& other) noexcept = default;

    // assign:
    revert_matrix& operator=(const revert_matrix& other) = default;
    revert_matrix& operator=(revert_matrix&& other) noexcept = default;
  };

  // \ref{\iotwod.revertmatrix.eq}, equality operators:
  template <class GraphicsSurfaces>
  bool operator==(
    const typename basic_figure_items<GraphicsSurfaces>::revert_matrix& lhs,
    const typename basic_figure_items<GraphicsSurfaces>::revert_matrix& rhs) 
    noexcept;  
  template <class GraphicsSurfaces>
  bool operator!=(
    const typename basic_figure_items<GraphicsSurfaces>::revert_matrix& lhs,
    const typename basic_figure_items<GraphicsSurfaces>::revert_matrix& rhs) 
    noexcept;
}
\end{codeblock}

\rSec1 [\iotwod.revertmatrix.eq] {Equality operators}%

\indexlibrarymember{operator==}{revert_matrix}%
\begin{itemdecl}
template <class GraphicsSurfaces>
bool operator==(
  const typename basic_figure_items<GraphicsSurfaces>::revert_matrix& lhs,
  const typename basic_figure_items<GraphicsSurfaces>::revert_matrix& rhs)
  noexcept;
\end{itemdecl}
\begin{itemdescr}
\pnum
\returns \tcode{true}.
\end{itemdescr}

\indexlibrarymember{operator!=}{revert_matrix}%
\begin{itemdecl}
template <class GraphicsSurfaces>
bool operator!=(
  const typename basic_figure_items<GraphicsSurfaces>::revert_matrix& lhs,
  const typename basic_figure_items<GraphicsSurfaces>::revert_matrix& rhs)
  noexcept;
\end{itemdecl}
\begin{itemdescr}
\pnum
\returns \tcode{false}.
\end{itemdescr}

%!TEX root = io2d.tex
\rSec0 [absline] {Class \tcode{abs_line}}

\pnum
\indexlibrary{\idxcode{abs_line}}
The class \tcode{abs_line} describes a path segment that is a line.

\pnum
It has an end point of type \tcode{vector_2d}.

\rSec1 [absline.synopsis] {\tcode{abs_line} synopsis}

\begin{codeblock}
namespace std { namespace experimental { namespace io2d { inline namespace v1 {
  namespace path_data {
    class abs_line {
    public:
      // \ref{absline.cons}, construct:
      constexpr abs_line() noexcept;
      constexpr explicit abs_line(const vector_2d& pt) noexcept;
      constexpr abs_line(const abs_line&) noexcept = default;
      constexpr abs_line& operator=(const abs_line&) noexcept = default;
      abs_line(abs_line&&) noexcept = default;
      abs_line& operator=(abs_line&&) noexcept = default;

      // \ref{absline.modifiers}, modifiers:
      void to(const vector_2d& pt) noexcept;

      // \ref{absline.observers}, observers:
      constexpr vector_2d to() const noexcept;
    };
  };
} } } }
\end{codeblock}

\rSec1 [absline.cons] {\tcode{abs_line} constructors and assignment operators}

\indexlibrary{\idxcode{abs_line}!constructor}
\begin{itemdecl}
constexpr abs_line() noexcept;
\end{itemdecl}
\begin{itemdescr}
\pnum
\effects
Constructs an object of type \tcode{abs_line}.

\pnum
The end point shall be set to the value of \tcode{vector_2d\{0.0, 0.0\}}.
\end{itemdescr}

\indexlibrary{\idxcode{abs_line}!constructor}
\begin{itemdecl}
constexpr explicit abs_line(const vector_2d& pt) noexcept;
\end{itemdecl}
\begin{itemdescr}
\pnum
\effects
Constructs an object of type \tcode{abs_line}.

\pnum
The end point shall be set to the value of \tcode{pt}.
\end{itemdescr}

\rSec1 [absline.modifiers]{\tcode{abs_line} modifiers}

\indexlibrary{\idxcode{abs_line}!\idxcode{to}}
\indexlibrary{\idxcode{to}!\idxcode{abs_line}}
\begin{itemdecl}
void to(const vector_2d& pt) noexcept;
\end{itemdecl}
\begin{itemdescr}
\pnum
\effects
The end point shall be set to the value of \tcode{pt}.
\end{itemdescr}

\rSec1 [absline.observers]{\tcode{abs_line} observers}

\indexlibrary{\idxcode{abs_line}!\idxcode{to}}
\indexlibrary{\idxcode{to}!\idxcode{abs_line}}
\begin{itemdecl}
constexpr vector_2d to() const noexcept;
\end{itemdecl}
\begin{itemdescr}
\pnum
\returns
The value of the end point.
\end{itemdescr}

%!TEX root = io2d.tex
\rSec0 [\iotwod.relline] {Class \tcode{rel_line}}

\rSec1 [\iotwod.relline.intro] {Overview}

\pnum
\indexlibrary{\idxcode{rel_line}}%
The class \tcode{basic_figure_items<GraphicsSurfaces>::rel_line} describes a figure item that is a segment.

\pnum
It has an \term{end point} of type \tcode{basic_point_2d<GraphicsSurfaces::graphics_math_type>}.

\pnum
The data are stored in an object of type \tcode{typename GraphicsSurfaces::paths::rel_line_data_type}. It is accessible using the \tcode{data} member functions.

\rSec1 [\iotwod.relline.synopsis] {Synopsis}
\begin{codeblock}
namespace std::experimemtal::io2d::v1 {
  template <class GraphicsSurfaces>
  class basic_figure_items<GraphicsSurfaces>::rel_line {
  public:
    using graphics_math_type = typename GraphicsSurfaces::graphics_math_type;
    using data_type =
      typename GraphicsSurfaces::paths::rel_line_data_type;

    // \ref{\iotwod.relline.ctor}, construct:
    rel_line();
    explicit rel_line(const basic_point_2d<graphics_math_type>& pt);
    rel_line(const rel_line& other) = default;
    rel_line(rel_line&& other) noexcept = default;

    // assign:
    rel_line& operator=(const rel_line& other) = default;
    rel_line& operator=(rel_line&& other) noexcept = default;

    // \ref{\iotwod.relline.acc}, accessors:
    const data_type& data() const noexcept;
    data_type& data() noexcept;

    // \ref{\iotwod.relline.mod}, modifiers:
    void at(const basic_point_2d<graphics_math_type>& pt) noexcept;

    // \ref{\iotwod.relline.obs}, observers:
    basic_point_2d<graphics_math_type> at() const noexcept;
  };
  
  // \ref{\iotwod.relline.eq}, equality operators:
  template <class GraphicsSurfaces>
  bool operator==(
    const typename basic_figure_items<GraphicsSurfaces>::rel_line& lhs,
    const typename basic_figure_items<GraphicsSurfaces>::rel_line& rhs) 
    noexcept;  
  template <class GraphicsSurfaces>
  bool operator!=(
    const typename basic_figure_items<GraphicsSurfaces>::rel_line& lhs,
    const typename basic_figure_items<GraphicsSurfaces>::rel_line& rhs) 
    noexcept;  
}
\end{codeblock}

\rSec1 [\iotwod.relline.ctor] {Constructors}%

\indexlibrary{\idxcode{rel_line}!constructor}%
\begin{itemdecl}
rel_line() noexcept;
\end{itemdecl}
\begin{itemdescr}
\pnum
\effects Equivalent to: \tcode{rel_line\{ basic_point_2d() \};}
\end{itemdescr}

\indexlibrary{\idxcode{rel_line}!constructor}%
\begin{itemdecl}
explicit rel_line(const basic_point_2d<typename
  GraphicsSurfaces::graphics_math_type>& pt) noexcept;
\end{itemdecl}
\begin{itemdescr}
\pnum
\effects Constructs an object of type \tcode{rel_line}.

\pnum
\remarks The end point is \tcode{pt}.
\end{itemdescr}

\rSec1 [\iotwod.relline.acc] {Accessors}%

\indexlibrarymember{data}{rel_line}%
\begin{itemdecl}
const data_type& data() const noexcept;
data_type& data() noexcept;
\end{itemdecl}
\begin{itemdescr}
\pnum
\returns A reference to the \tcode{rel_line} object's data object (See: \ref{\iotwod.relline.intro}).
\end{itemdescr}

\rSec1 [\iotwod.relline.mod] {Modifiers}%

\indexlibrarymember{to}{rel_line}%
\begin{itemdecl}
void to(const basic_point_2d<graphics_math_type>& pt) noexcept;
\end{itemdecl}
\begin{itemdescr}
\pnum
\effects The end point is \tcode{pt}.
\end{itemdescr}

\rSec1 [\iotwod.relline.obs] {Observers} 

\indexlibrarymember{to}{rel_line}%
\begin{itemdecl}
basic_point_2d<graphics_math_type> to() const noexcept;
\end{itemdecl}
\begin{itemdescr}
\pnum
\returns The end point.
\end{itemdescr}

\rSec1 [\iotwod.relline.ops] {Equality operators}%

\indexlibrarymember{operator==}{rel_line}%
\begin{itemdecl}
template <class GraphicsSurfaces>
bool operator==(
  const typename basic_figure_items<GraphicsSurfaces>::rel_line& lhs,
  const typename basic_figure_items<GraphicsSurfaces>::rel_line& rhs)
  noexcept;
\end{itemdecl}
\begin{itemdescr}
\pnum
\returns
\tcode{lhs.to() == rhs.to()}.
\end{itemdescr}

\indexlibrarymember{operator!=}{rel_line}%
\begin{itemdecl}
template <class GraphicsSurfaces>
bool operator!=(
  const typename basic_figure_items<GraphicsSurfaces>::rel_line& lhs,
  const typename basic_figure_items<GraphicsSurfaces>::rel_line& rhs)
  noexcept;
\end{itemdecl}
\begin{itemdescr}
\pnum
\returns \tcode{lhs.to() != rhs.to()}.
\end{itemdescr}

%!TEX root = io2d.tex
\rSec0 [pathfactory.pathabsquadraticcurve] {Class \tcode{path_factory::path_abs_quadratic_curve}}

\pnum
\indexlibrary{\idxcode{path_factory::path_abs_quadratic_curve}}
The class \tcode{path_factory::path_abs_quadratic_curve} describes a path segment that is a quadratic \bezierlocal curve.

\pnum
It has a control point of type \tcode{vector_2d} and an end point of type \tcode{vector_2d}.

\rSec1 [pathfactory.pathabsquadraticcurve.synopsis] {\tcode{path_factory::path_abs_quadratic_curve} synopsis}

\begin{codeblock}
namespace std { namespace experimental { namespace io2d { inline namespace v1 {
  class path_factory::path_abs_cubic_curve {
  public:
    // \ref{pathfactory.pathabsquadraticcurve.cons}, construct:
    path_abs_quadratic_curve(const vector_2d& cp, const vector_2d& ep) noexcept;

    // \ref{pathfactory.pathabsquadraticcurve.modifiers}, modifiers:
    void control_point(const vector_2d& cp) noexcept;
    void end_point(const vector_2d& ep) noexcept;


    // \ref{pathfactory.pathabsquadraticcurve.observers}, observers:
    vector_2d control_point() const noexcept;
    vector_2d end_point() const noexcept;
  };
} } } }
\end{codeblock}

\rSec1 [pathfactory.pathabsquadraticcurve.cons] {\tcode{path_factory::path_abs_quadratic_curve} constructors}

\indexlibrary{\idxcode{path_factory::path_abs_quadratic_curve}!constructor}
\begin{itemdecl}
    path_abs_quadratic_curve(const vector_2d& cp, const vector_2d& ep) noexcept;
\end{itemdecl}
\begin{itemdescr}
	\pnum
	\effects
	Constructs an object of type \tcode{path_factory::path_abs_cubic_curve}.
	
	\pnum
	The control point shall be set to the value of \tcode{cp}.
	
	\pnum
	The end point shall be set to the value of \tcode{ep}.
\end{itemdescr}

\rSec1 [pathfactory.pathabsquadraticcurve.modifiers]{\tcode{path_factory::path_abs_cubic_curve} modifiers}

\indexlibrary{\idxcode{path_factory::path_abs_quadratic_curve}!\idxcode{control_point}}
\indexlibrary{\idxcode{control_point}!\idxcode{path_factory::path_abs_quadratic_curve}}
\begin{itemdecl}
    void control_point(const vector_2d& cp) noexcept;
\end{itemdecl}
\begin{itemdescr}
	\pnum
	\effects
	The control point shall be set to the value of \tcode{cp}.
\end{itemdescr}

\indexlibrary{\idxcode{path_factory::path_abs_quadratic_curve}!\idxcode{end_point}}
\indexlibrary{\idxcode{end_point}!\idxcode{path_factory::path_abs_quadratic_curve}}
\begin{itemdecl}
    void end_point(const vector_2d& ep) noexcept;
\end{itemdecl}
\begin{itemdescr}
	\pnum
	\effects
	The end point shall be set to the value of \tcode{ep}.
\end{itemdescr}

\rSec1 [pathfactory.pathabsquadraticcurve.observers]{\tcode{path_factory::path_abs_quadratic_curve} observers}

\indexlibrary{\idxcode{path_factory::path_abs_cubic_curve}!\idxcode{control_point}}
\indexlibrary{\idxcode{control_point}!\idxcode{path_factory::path_abs_quadratic_curve}}
\begin{itemdecl}
    vector_2d control_point() const noexcept;
\end{itemdecl}
\begin{itemdescr}
	\pnum
	\returns
	The value of the control point.
\end{itemdescr}

\indexlibrary{\idxcode{path_factory::path_abs_quadratic_curve}!\idxcode{end_point}}
\indexlibrary{\idxcode{end_point}!\idxcode{path_factory::path_abs_quadratic_curve}}
\begin{itemdecl}
    vector_2d end_point() const noexcept;
\end{itemdecl}
\begin{itemdescr}
	\pnum
	\returns
	The value of the end point.
\end{itemdescr}

%!TEX root = io2d.tex
\rSec0 [\iotwod.relquadraticcurve] {Class template \tcode{basic_figure_items<GraphicsSurfaces>::rel_quadratic_curve}}

\rSec1 [\iotwod.relquadraticcurve.intro] {Overview}

\pnum
\indexlibrary{\idxcode{rel_quadratic_curve}}%
The class \tcode{basic_figure_items<GraphicsSurfaces>::rel_quadratic_curve} describes a figure item that is a segment.

\pnum
It has a \term{control point} of type \tcode{basic_point_2d<GraphicsSurfaces::graphics_math_type>} and an \term{end point} of type \tcode{basic_point_2d<GraphicsSurfaces::graphics_math_type>}.

\pnum
The data are stored in an object of type \tcode{typename GraphicsSurfaces::paths::rel_quadratic_curve_data_type}. It is accessible using the \tcode{data} member functions.

\rSec1 [\iotwod.relquadraticcurve.synopsis] {Synopsis}
\begin{codeblock}
namespace std::experimemtal::io2d::v1 {
  template <class GraphicsSurfaces>
  class basic_figure_items<GraphicsSurfaces>::rel_quadratic_curve {
  public:
    using graphics_math_type = typename GraphicsSurfaces::graphics_math_type;
    using data_type =
      typename GraphicsSurfaces::paths::rel_quadratic_curve_data_type;

    // \ref{\iotwod.relquadraticcurve.ctor}, construct:
    rel_quadratic_curve();
    rel_quadratic_curve(const basic_point_2d<graphics_math_type>& cpt,
      const basic_point_2d<graphics_math_type>& ept);
    rel_quadratic_curve(const rel_quadratic_curve& other) = default;
    rel_quadratic_curve(rel_quadratic_curve&& other) noexcept = default;

    // assign:
    rel_quadratic_curve& operator=(const rel_quadratic_curve& other) = default;
    rel_quadratic_curve& operator=(rel_quadratic_curve&& other) noexcept = default;

    // \ref{\iotwod.relquadraticcurve.acc}, accessors:
    const data_type& data() const noexcept;
    data_type& data() noexcept;

    // \ref{\iotwod.relquadraticcurve.mod}, modifiers:
    void control_pt(const basic_point_2d<graphics_math_type>& cpt) noexcept;
    void end_pt(const basic_point_2d<graphics_math_type>& ept) noexcept;

    // \ref{\iotwod.relquadraticcurve.obs}, observers:
    basic_point_2d<graphics_math_type> control_pt() const noexcept;
    basic_point_2d<graphics_math_type> end_pt() const noexcept;
  };

  // \ref{\iotwod.relquadraticcurve.eq}, equality operators:
  template <class GraphicsSurfaces>
  bool operator==(
    const typename basic_figure_items<GraphicsSurfaces>::rel_quadratic_curve& lhs,
    const typename basic_figure_items<GraphicsSurfaces>::rel_quadratic_curve& rhs) 
    noexcept;  
  template <class GraphicsSurfaces>
  bool operator!=(
    const typename basic_figure_items<GraphicsSurfaces>::rel_quadratic_curve& lhs,
    const typename basic_figure_items<GraphicsSurfaces>::rel_quadratic_curve& rhs) 
    noexcept;  
}
\end{codeblock}

\rSec1 [\iotwod.relquadraticcurve.ctor] {Constructors}%

\indexlibrary{\idxcode{rel_quadratic_curve}!constructor}%
\begin{itemdecl}
rel_quadratic_curve() noexcept;
\end{itemdecl}
\begin{itemdescr}
\pnum
\effects Equivalent to: \tcode{rel_quadratic_curve\{ basic_point_2d(), basic_point_2d() \};}
\end{itemdescr}

\indexlibrary{\idxcode{rel_quadratic_curve}!constructor}%
\begin{itemdecl}
rel_quadratic_curve(const basic_point_2d<typename GraphicsSurfaces::graphics_math_type>& cpt,
  const basic_point_2d<typename GraphicsSurfaces::graphics_math_type>& ept) noexcept;
\end{itemdecl}
\begin{itemdescr}
\pnum
\effects Constructs an object of type \tcode{rel_quadratic_curve}.

\pnum
\remarks The control point is \tcode{cpt}.

\pnum
\remarks The end point is \tcode{ept}.
\end{itemdescr}

\rSec1 [\iotwod.relquadraticcurve.acc] {Accessors}%

\indexlibrarymember{data}{rel_quadratic_curve}%
\begin{itemdecl}
const data_type& data() const noexcept;
data_type& data() noexcept;
\end{itemdecl}
\begin{itemdescr}
\pnum
\returns A reference to the \tcode{rel_quadratic_curve} object's data object (See: \ref{\iotwod.relquadraticcurve.intro}).
\end{itemdescr}

\rSec1 [\iotwod.relquadraticcurve.mod]{Modifiers}%

\indexlibrarymember{control_pt}{rel_quadratic_curve}%
\begin{itemdecl}
void control_pt(const basic_point_2d<graphics_math_type>& cpt) noexcept;
\end{itemdecl}
\begin{itemdescr}
\pnum
\effects The control point is \tcode{cpt}.
\end{itemdescr}

\indexlibrarymember{end_pt}{rel_quadratic_curve}%
\begin{itemdecl}
void end_pt(const basic_point_2d<graphics_math_type>& ept) noexcept;
\end{itemdecl}
\begin{itemdescr}
\pnum
\effects The end point is \tcode{ept}.
\end{itemdescr}

\rSec1 [\iotwod.relquadraticcurve.obs] {Observers}

\indexlibrarymember{control_pt}{rel_quadratic_curve}%
\begin{itemdecl}
basic_point_2d<graphics_math_type> control_pt() const noexcept;
\end{itemdecl}
\begin{itemdescr}
\pnum
\returns The control point.
\end{itemdescr}

\indexlibrarymember{end_pt}{rel_quadratic_curve}%
\begin{itemdecl}
basic_point_2d<graphics_math_type> end_pt() const noexcept;
\end{itemdecl}
\begin{itemdescr}
\pnum
\returns The end point.
\end{itemdescr}

\rSec1 [\iotwod.relquadraticcurve.eq] {Equality operators}%

\indexlibrarymember{operator==}{rel_quadratic_curve}%
\begin{itemdecl}
template <class GraphicsSurfaces>
bool operator==(
  const typename basic_figure_items<GraphicsSurfaces>::rel_quadratic_curve& lhs,
  const typename basic_figure_items<GraphicsSurfaces>::rel_quadratic_curve& rhs) 
  noexcept;
\end{itemdecl}
\begin{itemdescr}
\pnum
\returns
\tcode{lhs.control_pt() == rhs.control_pt() \&\& lhs.end_pt() == rhs.end_pt()}.
\end{itemdescr}

\indexlibrarymember{operator!=}{rel_quadratic_curve}%
\begin{itemdecl}
template <class GraphicsSurfaces>
bool operator!=(
  const typename basic_figure_items<GraphicsSurfaces>::rel_quadratic_curve& lhs,
  const typename basic_figure_items<GraphicsSurfaces>::rel_quadratic_curve& rhs) 
  noexcept;
\end{itemdecl}
\begin{itemdescr}
\pnum
\returns
\tcode{lhs.control_pt() != rhs.control_pt() || lhs.end_pt() != rhs.end_pt()}.
\end{itemdescr}

%!TEX root = io2d.tex
\rSec0 [\iotwod.abscubiccurve] {Class template \tcode{basic_figure_items<GraphicsSurfaces>::abs_cubic_curve}}

\rSec1 [\iotwod.abscubiccurve.intro] {Overview}

\pnum
\indexlibrary{\idxcode{abs_cubic_curve}}%
The class \tcode{basic_figure_items<GraphicsSurfaces>::abs_cubic_curve} describes a figure item that is a segment.

\pnum
It has a \term{first control point} of type \tcode{basic_point_2d<GraphicsSurfaces::graphics_math_type>}, a \term{second control point} of type \tcode{basic_point_2d<GraphicsSurfaces::graphics_math_type>}, and an \tcode{end point} of type \tcode{basic_point_2d<GraphicsSurfaces::graphics_math_type>}.

\pnum
The data are stored in an object of type \tcode{typename GraphicsSurfaces::paths::abs_cubic_curve_data_type}. It is accessible using the \tcode{data} member functions.

\rSec1 [\iotwod.abscubiccurve.synopsis] {Synopsis}
\begin{codeblock}
namespace @\fullnamespace{}@ {
  template <class GraphicsSurfaces>
  class basic_figure_items<GraphicsSurfaces>::abs_cubic_curve {
  public:
    using graphics_math_type = typename GraphicsSurfaces::graphics_math_type;
    using data_type =
      typename GraphicsSurfaces::paths::abs_cubic_curve_data_type;

    // \ref{\iotwod.abscubiccurve.ctor}, construct:
    abs_cubic_curve();
    abs_cubic_curve(const basic_point_2d<graphics_math_type>& cpt1,
       const basic_point_2d<graphics_math_type>& cpt2,
       const basic_point_2d<graphics_math_type>& ept) noexcept;
    abs_cubic_curve(const abs_cubic_curve& other) = default;
    abs_cubic_curve(abs_cubic_curve&& other) noexcept = default;

    // assign:
    abs_cubic_curve& operator=(const abs_cubic_curve& other) = default;
    abs_cubic_curve& operator=(abs_cubic_curve&& other) noexcept = default;

    // \ref{\iotwod.abscubiccurve.acc}, accessors:
    const data_type& data() const noexcept;
    data_type& data() noexcept;

    // \ref{\iotwod.abscubiccurve.mod}, modifiers:
    void control_pt1(const basic_point_2d<graphics_math_type>& cpt) noexcept;
    void control_pt2(const basic_point_2d<graphics_math_type>& cpt) noexcept;
    void end_pt(const basic_point_2d<graphics_math_type>& ept) noexcept;

    // \ref{\iotwod.abscubiccurve.obs}, observers:
    basic_point_2d<graphics_math_type> control_pt1() const noexcept;
    basic_point_2d<graphics_math_type> control_pt2() const noexcept;
    basic_point_2d<graphics_math_type> end_pt() const noexcept;
  };

  // \ref{\iotwod.abscubiccurve.eq}, equality operators:
  template <class GraphicsSurfaces>
  bool operator==(
    const typename basic_figure_items<GraphicsSurfaces>::abs_cubic_curve& lhs,
    const typename basic_figure_items<GraphicsSurfaces>::abs_cubic_curve& rhs) 
    noexcept;  
  template <class GraphicsSurfaces>
  bool operator!=(
    const typename basic_figure_items<GraphicsSurfaces>::abs_cubic_curve& lhs,
    const typename basic_figure_items<GraphicsSurfaces>::abs_cubic_curve& rhs) 
    noexcept;  
}
\end{codeblock}

\rSec1 [\iotwod.abscubiccurve.ctor] {Constructors}%

\indexlibrary{\idxcode{abs_cubic_curve}!constructor}%
\begin{itemdecl}
abs_cubic_curve() noexcept;
\end{itemdecl}
\begin{itemdescr}
\pnum
\effects
Equivalent to \tcode{abs_cubic_curve\{ basic_point_2d(), basic_point_2d(), basic_point_2d() \}}.
\end{itemdescr}

\indexlibrary{\idxcode{abs_cubic_curve}!constructor}%
\begin{itemdecl}
abs_cubic_curve(const basic_point_2d<typename GraphicsSurfaces::graphics_math_type>& cpt1,
  const basic_point_2d<typename GraphicsSurfaces::graphics_math_type>& cpt2,
  const basic_point_2d<typename GraphicsSurfaces::graphics_math_type>& ept) noexcept;
\end{itemdecl}
\begin{itemdescr}
\pnum
\effects Constructs an object of type \tcode{abs_cubic_curve}.

\pnum
\remarks The first control point is \tcode{cpt1}.

\pnum
\remarks The second control point is \tcode{cpt2}.

\pnum
\remarks The end point is \tcode{ept}.
\end{itemdescr}

\rSec1 [\iotwod.abscubiccurve.acc] {Accessors}%

\indexlibrarymember{data}{abs_cubic_curve}%
\begin{itemdecl}
const data_type& data() const noexcept;
data_type& data() noexcept;
\end{itemdecl}
\begin{itemdescr}
\pnum
\returns A reference to the \tcode{rel_matrix} object's data object (See: \ref{\iotwod.abscubiccurve.intro}).
\end{itemdescr}

\rSec1 [\iotwod.abscubiccurve.mod] {Modifiers}

\indexlibrarymember{control_pt1}{abs_cubic_curve}%
\begin{itemdecl}
void control_pt1(const basic_point_2d<typename
  GraphicsSurfaces::graphics_math_type>& cpt) noexcept;
\end{itemdecl}
\begin{itemdescr}
\pnum
\effects
The first control point is \tcode{cpt}.
\end{itemdescr}

\indexlibrarymember{control_pt2}{abs_cubic_curve}%
\begin{itemdecl}
void control_pt2(const basic_point_2d<typename
  GraphicsSurfaces::graphics_math_type>& cpt) noexcept;
\end{itemdecl}
\begin{itemdescr}
\pnum
\effects
The second control point is \tcode{cpt}.
\end{itemdescr}

\indexlibrarymember{end_pt}{abs_cubic_curve}%
\begin{itemdecl}
void end_pt(const basic_point_2d<typename GraphicsSurfaces::graphics_math_type>& ept) noexcept;
\end{itemdecl}
\begin{itemdescr}
\pnum
\effects
The end point is \tcode{ept}.
\end{itemdescr}

\rSec1 [\iotwod.abscubiccurve.obs] {Observers}

\indexlibrarymember{control_pt1}{abs_cubic_curve}%
\begin{itemdecl}
basic_point_2d<graphics_math_type> control_pt1() const noexcept;
\end{itemdecl}
\begin{itemdescr}
\pnum
\returns The first control point.
\end{itemdescr}

\indexlibrarymember{control_pt2}{abs_cubic_curve}%
\begin{itemdecl}
basic_point_2d<graphics_math_type> control_pt2() const noexcept;
\end{itemdecl}
\begin{itemdescr}
\pnum
\returns The second control point.
\end{itemdescr}

\indexlibrarymember{end_pt}{abs_cubic_curve}%
\begin{itemdecl}
basic_point_2d<graphics_math_type> end_pt() const noexcept;
\end{itemdecl}
\begin{itemdescr}
\pnum
\returns The end point.
\end{itemdescr}

\rSec1 [\iotwod.abscubiccurve.eq] {Equality operators}%

\indexlibrarymember{operator==}{abs_cubic_curve}%
\begin{itemdecl}
template <class GraphicsSurfaces>
bool operator==(
  const typename basic_figure_items<GraphicsSurfaces>::abs_cubic_curve& lhs,
  const typename basic_figure_items<GraphicsSurfaces>::abs_cubic_curve& rhs) 
  noexcept;
\end{itemdecl}
\begin{itemdescr}
\pnum
\returns
\tcode{lhs.control_pt1() == rhs.control_pt1() \&\& lhs.control_pt2() == rhs.control_pt2() \&\& lhs.end_pt() == rhs.end_pt()}.
\end{itemdescr}

\indexlibrarymember{operator!=}{abs_cubic_curve}%
\begin{itemdecl}
template <class GraphicsSurfaces>
bool operator!=(
  const typename basic_figure_items<GraphicsSurfaces>::abs_cubic_curve& lhs,
  const typename basic_figure_items<GraphicsSurfaces>::abs_cubic_curve& rhs) 
  noexcept;
\end{itemdecl}
\begin{itemdescr}
\pnum
\returns
\tcode{lhs.control_pt1() != rhs.control_pt1() || lhs.control_pt2() != rhs.control_pt2() || lhs.end_pt() != rhs.end_pt()}.
\end{itemdescr}

%!TEX root = io2d.tex
\rSec0 [relcubiccurve] {Class \tcode{rel_cubic_curve}}

\pnum
\indexlibrary{\idxcode{rel_cubic_curve}}
The class \tcode{rel_cubic_curve} describes a path segment that is a cubic \bezierlocal curve.

\pnum
It has a first control point of type \tcode{vector_2d}, a second control point of type \tcode{vector_2d}, and an end point of type \tcode{vector_2d}.

\pnum
All of its points are relative to the most recently established current point.

\rSec1 [relcubiccurve.synopsis] {\tcode{rel_cubic_curve} synopsis}

\begin{codeblock}
namespace std { namespace experimental { namespace io2d { inline namespace v1 {
  namespace path_data {
    class rel_cubic_curve {
    public:
      // \ref{relcubiccurve.cons}, construct
      rel_cubic_curve(const vector_2d& cp1, const vector_2d& cp2,
        const vector_2d& ep) noexcept;

      // \ref{relcubiccurve.modifiers}, modifiers:
      void control_point_1(const vector_2d& cp) noexcept;
      void control_point_2(const vector_2d& cp) noexcept;
      void end_point(const vector_2d& ep) noexcept;

      // \ref{relcubiccurve.observers}, observers:
      vector_2d control_point_1() const noexcept;
      vector_2d control_point_2() const noexcept;
      vector_2d end_point() const noexcept;
    };
  };
} } } }
\end{codeblock}

\rSec1 [relcubiccurve.cons] {\tcode{rel_cubic_curve} constructors}
\indexlibrary{\idxcode{rel_cubic_curve_to}!constructor}
\begin{itemdecl}
    rel_cubic_curve(const vector_2d& cp1, const vector_2d& cp2,
      const vector_2d& ep) noexcept;
\end{itemdecl}
\begin{itemdescr}
	\pnum
	\effects
	Constructs an object of type \tcode{rel_cubic_curve}.
	
	\pnum
	The first control point shall be set to the value of \tcode{cp1}.
	
	\pnum
	The second control point shall be set to the value of \tcode{cp2}.
	
	\pnum
	The end point shall be set to the value of \tcode{ep}.
\end{itemdescr}

\rSec1 [relcubiccurve.modifiers]{\tcode{rel_cubic_curve} modifiers}

\indexlibrary{\idxcode{rel_cubic_curve}!\idxcode{control_point_1}}
\indexlibrary{\idxcode{control_point_1}!\idxcode{rel_cubic_curve}}
\begin{itemdecl}
    void control_point_1(const vector_2d& cp) noexcept;
\end{itemdecl}
\begin{itemdescr}
	\pnum
	\effects
	The first control point shall be set to the value of \tcode{cp}.
\end{itemdescr}

\indexlibrary{\idxcode{rel_cubic_curve}!\idxcode{control_point_2}}
\indexlibrary{\idxcode{control_point_2}!\idxcode{rel_cubic_curve}}
\begin{itemdecl}
    void control_point_2(const vector_2d& value) noexcept;
\end{itemdecl}
\begin{itemdescr}
	\pnum
	\effects
	The second control point shall be set to the value of \tcode{cp}.
\end{itemdescr}

\indexlibrary{\idxcode{rel_cubic_curve}!\idxcode{end_point}}
\indexlibrary{\idxcode{end_point}!\idxcode{rel_cubic_curve}}
\begin{itemdecl}
    void end_point(const vector_2d& value) noexcept;
\end{itemdecl}
\begin{itemdescr}
	\pnum
	\effects
	The end point shall be set to the value of \tcode{ep}.
\end{itemdescr}

\rSec1 [relcubiccurve.observers]{\tcode{rel_cubic_curve} observers}

\indexlibrary{\idxcode{rel_cubic_curve}!\idxcode{control_point_1}}
\indexlibrary{\idxcode{control_point_1}!\idxcode{rel_cubic_curve}}
\begin{itemdecl}
    vector_2d control_point_1() const noexcept;
\end{itemdecl}
\begin{itemdescr}
	\pnum
	\returns
	The value of the first control point.
\end{itemdescr}

\indexlibrary{\idxcode{rel_cubic_curve}!\idxcode{control_point_2}}
\indexlibrary{\idxcode{control_point_2}!\idxcode{rel_cubic_curve}}
\begin{itemdecl}
    vector_2d control_point_2() const noexcept;
\end{itemdecl}
\begin{itemdescr}
	\pnum
	\returns
	The value of the second control point.
\end{itemdescr}

\indexlibrary{\idxcode{rel_cubic_curve}!\idxcode{end_point}}
\indexlibrary{\idxcode{end_point}!\idxcode{rel_cubic_curve}}
\begin{itemdecl}
    vector_2d end_point() const noexcept;
\end{itemdecl}
\begin{itemdescr}
	\pnum
	\returns
	The value of the end point.
\end{itemdescr}

%!TEX root = io2d.tex
\rSec0 [\iotwod.arc] {Class \tcode{arc}}

\rSec1 [\iotwod.arc.general] {In general}

\pnum
\indexlibrary{\idxcode{arc}}%
The class \tcode{arc} describes a path item that is a path segment.

\pnum
It has a \term{radius} of type \tcode{vector_2d}, a \term{rotation} of type \tcode{float}, and a \term{start angle} of type \tcode{float}.

\rSec1 [\iotwod.arc.synopsis] {\tcode{arc} synopsis}

\begin{codeblock}
namespace std::experimental::io2d::v1 {
  namespace path_data {
    class arc {
    public:
      // \ref{\iotwod.arc.cons}, construct/copy/move/destroy:
      constexpr arc() noexcept;
      constexpr arc(const vector_2d& rad,
        float rot, float sang) noexcept;

      // \ref{\iotwod.arc.modifiers}, modifiers:
      constexpr void radius(const vector_2d& rad) noexcept;
      constexpr void rotation(float rot) noexcept;
      constexpr void start_angle(float radians) noexcept;

      // \ref{\iotwod.arc.observers}, observers:
      constexpr vector_2d radius() const noexcept;
      constexpr float rotation() const noexcept;
      constexpr float start_angle() const noexcept;
      vector_2d center(const vector_2d& cpt, const matrix_2d& m = matrix_2d{}) 
        const noexcept;
      vector_2d end_pt(const vector_2d& cpt, const matrix_2d& m = matrix_2d{}) 
        const noexcept;
    };
    
    // \ref{\iotwod.arc.nonmember}, non-members
    constexpr bool operator==(const arc& lhs, const arc& rhs) noexcept;
    constexpr bool operator!=(const arc& lhs, const arc& rhs) noexcept;
  }
}
\end{codeblock}

\rSec1 [\iotwod.arc.cons] {\tcode{arc} constructors}

\indexlibrary{\idxcode{arc}!constructor}%
\begin{itemdecl}
constexpr arc() noexcept;
\end{itemdecl}
\begin{itemdescr}
\pnum
\effects
Equivalent to: \tcode{arc\{ vector_2d(10.0f, 10.0f), pi<float>, pi<float> \};}.
\end{itemdescr}

\indexlibrary{\idxcode{arc}!constructor}%
\begin{itemdecl}
constexpr arc(const vector_2d& rad, float rot,
  float start_angle = pi<float>) noexcept;
\end{itemdecl}
\begin{itemdescr}
\pnum
\effects
Constructs an object of type \tcode{arc}.

\pnum
The radius is \tcode{rad}.

\pnum
The rotation is \tcode{rot}.

\pnum
The start angle is \tcode{sang}.
\end{itemdescr}

\rSec1 [\iotwod.arc.modifiers]{\tcode{arc} modifiers}

\indexlibrarymember{radius}{arc}%
\begin{itemdecl}
constexpr void radius(const vector_2d& rad) noexcept;
\end{itemdecl}
\begin{itemdescr}
\pnum
\effects
The radius is \tcode{rad}.
\end{itemdescr}

\indexlibrarymember{rotation}{arc}%
\begin{itemdecl}
constexpr void rotation(float rot) noexcept;
\end{itemdecl}
\begin{itemdescr}
\pnum
\effects
The rotation is \tcode{rot}.
\end{itemdescr}

\indexlibrarymember{start_angle}{arc}%
\begin{itemdecl}
constexpr void start_angle(float sang) noexcept;
\end{itemdecl}
\begin{itemdescr}
\pnum
\effects
The start angle is \tcode{sang}.
\end{itemdescr}

\rSec1 [\iotwod.arc.observers]{\tcode{arc} observers}

\indexlibrarymember{radius}{arc}%
\begin{itemdecl}
constexpr vector_2d radius() const noexcept;
\end{itemdecl}
\begin{itemdescr}
\pnum
\returns
The radius.
\end{itemdescr}

\indexlibrarymember{rotation}{arc}%
\begin{itemdecl}
constexpr float rotation() const noexcept;
\end{itemdecl}
\begin{itemdescr}
\pnum
\returns
The rotation.
\end{itemdescr}

\indexlibrarymember{start_angle}{arc}%
\begin{itemdecl}
constexpr float start_angle() const noexcept;
\end{itemdecl}
\begin{itemdescr}
\pnum
\returns
The start angle.
\end{itemdescr}

\indexlibrarymember{center}{arc}%
\begin{itemdecl}
vector_2d center(const vector_2d& cpt, const matrix_2d& m = matrix_2d{})
  const noexcept;
\end{itemdecl}
\begin{itemdescr}
\pnum
\returns
As-if:
\begin{codeblock}
auto lmtx = m;
lmtx.m20(0.0f); lmtx.m21(0.0f); // Eliminate translation.
auto centerOffset = point_for_angle(two_pi<float> - _Start_angle, _Radius);
centerOffset.y(-centerOffset.y());
return cpt - centerOffset * lmtx;
\end{codeblock}
\end{itemdescr}

\indexlibrarymember{start_angle}{arc}%
\begin{itemdecl}
vector_2d end_pt(const vector_2d& cpt, const matrix_2d& m = matrix_2d{})
  const noexcept;
\end{itemdecl}
\begin{itemdescr}
\pnum
\returns
As-if:
\begin{codeblock}
auto lmtx = m;
auto tfrm = matrix_2d::init_rotate(_Start_angle + _Rotation);
lmtx.m20(0.0f); lmtx.m21(0.0f); // Eliminate translation.
auto pt = (_Radius * tfrm);
pt.y(-pt.y());
return cpt + pt * lmtx;
\end{codeblock}
\end{itemdescr}

\rSec1 [\iotwod.arc.nonmember]{Non-member functions}

\indexlibrarymember{operator==}{arc}%
\begin{itemdecl}
constexpr bool operator==(const arc& lhs, const arc& rhs) noexcept;
\end{itemdecl}
\begin{itemdescr}
\pnum
\returns
\begin{codeblock}
lhs.radius() == rhs.radius() && lhs.rotation() == rhs.rotation() &&
lhs.start_angle() && rhs.start_angle()
\end{codeblock}
\end{itemdescr}

\indexlibrarymember{operator!=}{arc}%
\begin{itemdecl}
constexpr bool operator!=(const arc& lhs, const arc& rhs) noexcept;
\end{itemdecl}
\begin{itemdescr}
\pnum
\returns
\tcode{!(lhs == rhs)}.
\end{itemdescr}

%!TEX root = io2d.tex

\rSec0 [paths.interpretation]{Path group interpretation}

\pnum
A path group is \term{interpreted} when its path items are evaluated sequentially starting with the first path item in accordance with the following requirements. The data contained in a \tcode{path_group} object is as-if it were the data of a path group that has been interpreted.

\pnum
When interpreting a path group, the initial value of its path group origin is \tcode{vector_2d\{ \}} and the initial value of its path group tranformation matrix is \tcode{matrix_2d\{ \}}.

\pnum
Defining the path group's path group origin as \tcode{o} and its path group transformation matrix as \tcode{m}, when path items are interpreted, each point \tcode{pt} in a path group is evaluated as-if its value is the return value of \tcode{m.transform_point(pt - o) + o;}. Where the path item is a relative path item, the value of \tcode{pt} is the value of the point after the current point has been added to it. This paragraph is not applicable to the point contained in a \tcode{change_origin} object.

\pnum
When a path group is interpreted, if a path item sets the value of the current point, defining \tcode{cpt} as the point that the path item specifies as the new value for the current point, the value of the current point is \tcode{cpt}. Where the path item is a relative path item, the value of the current point is the result of adding \tcode{cpt} to the current point.

\pnum
\begin{note}
Certain path items, when interpreted, set the value of both the current point and the last-move-to point. If, during interpretation, the path group origin or path group transformation matrix has a value that is different from its initial value, the new values for the current point and the last-move-to point will be determined differently. This is because the previous paragraph only applies when determining the value of the current point.
\end{note}

\pnum
\begin{note}
Because this subclause describes the conditions where the points of relative path items are added to the current point when a path group is interpreted, no mention of any such augmentation is included in the descriptions of relative path items.
\end{note}

%\pnum
%\begin{note}
%The points of path items in a path group are not subject to any 
%When path items are added to a path group, their points, if any, are stored without regard to the path group origin or the path group transformation matrix.
%
%Because of that, when a path group is interpreted, the current point is not  because it is used to calculate the value of points in relative path items before they are interpreted.
%\end{note}
%
\pnum
Except for \tcode{change_origin} and \tcode{change_matrix} objects, if any, the first path item in a path group shall be an \tcode{abs_new_path} object when it is interpreted; no diagnostic is required.
%\begin{note}
%Only an \tcode{abs_new_path} object establishes a non-relative current point for a path. Except for \tcode{change_origin}, \tcode{change_matrix}, and \tcode{abs_new_path} objects, all other path items require a value for the path's current point.
%\end{note}

\addtocounter{SectionDepthBase}{-2}

\addtocounter{SectionDepthBase}{1}
%!TEX root = io2d.tex
\rSec0 [path] {Class \tcode{path}}

\pnum
\indexlibrary{\idxcode{path}}
The \tcode{path} class represents an immutable resource wrapper containing a path geometry graphics resource.

\pnum
When a \tcode{path} object is set on a \tcode{surface} object using 
\tcode{surface::path}, the geometric paths represented by it can be 
stroked or filled.

\pnum
A \tcode{path} object shall be usable with any \tcode{surface} or \tcode{surface}-derived object.

\rSec1 [path.synopsis] {\tcode{path} synopsis}

\begin{codeblock}
namespace std { namespace experimental { namespace io2d { inline namespace v1 {
  class path {
    public:
    // \ref{path.cons}, construct/copy/destroy:
    path() = delete;
    explicit path(const path_factory& pb);
    path(const path_factory& pb, error_code& ec) noexcept;
    explicit path(const vector<path_data_item>& p);
    path(const vector<path_data_item>& p, error_code& ec) noexcept;
    path(const path&) noexcept;
    path& operator=(const path&) noexcept;
    path(path&&) noexcept;
    path& operator=(path&&) noexcept;
  };
} } } }
\end{codeblock}

\rSec1 [path.cons] {\tcode{path} constructors and assignment operators}

\indexlibrary{\idxcode{path}!constructor}
\begin{itemdecl}
    explicit path(const path_factory& pb);
    path(const path_factory& pb, error_code& ec) noexcept;
    explicit path(const vector<path_data_item>& p);
    path(const vector<path_data_item>& p, error_code& ec) noexcept;
\end{itemdecl}
\begin{itemdescr}
	\pnum
	\effects
	Constructs an object of class \tcode{path}. Implementations shall create a path geometry graphics resource from the path geometries contained in \tcode{p} or \tcode{pb.data_ref()} as if they followed the procedure set forth in \ref{pathgeometries.processing}.

	\pnum
	\throws
	As specified in Error reporting (\ref{\iotwod.err.report}).

	\pnum
	\remarks
	It is unspecified whether a \tcode{path} object shall require further processing when it is passed as an argument to a \tcode{surface} or \tcode{surface}-derived object.
	
	\pnum
	Implementations should avoid or minimize the need for further processing of a \tcode{path} object after it has been constructed.

	\pnum
	\errors
	\tcode{errc::not_enough_memory} if there was a failure to allocate memory.
	
	\pnum
	\tcode{io2d_error::no_current_point} if, when processing the path geometries, an operation was encountered which required a current point and the current path geometry had no current point.
	
	\pnum
	\tcode{io2d_error::invalid_matrix} if, when processing the path geometries, an operation was encountered which required the current transformation matrix to be invertible and the matrix was not invertible.
	
\end{itemdescr}

%!TEX root = io2d.tex
\rSec0 [\iotwod.pathbuilder] {Class \tcode{basic_path_builder}}

\pnum
\indexlibrary{\idxcode{basic_path_builder}}%
The class \tcode{basic_path_builder} is a container that stores and manipulates objects of type \tcode{figure_items::figure_item} from which \tcode{interpreted_path} objects are created.

\pnum
A \tcode{basic_path_builder} is a contiguous container. (See [container.requirements.general] in \CppXVII.)

\pnum
The collection of \tcode{figure_items::figure_item} objects in a path builder is referred to as its path.

\rSec1 [\iotwod.pathbuilder.synopsis] {\tcode{basic_path_builder} synopsis}%

\begin{codeblock}
namespace std::experimental::io2d::v1 {
  template <class GraphicsSurfaces,
            class Allocator = ::std::allocator<typename
              basic_figure_items<GraphicsSurfaces>::figure_item>>
  class basic_path_builder {
  public:
    using value_type             = typename basic_figure_items<GraphicsSurfaces>::figure_item;
    using allocator_type         = Allocator;
    using reference              = value_type&;
    using const_reference        = const value_type&;
    using size_type              = @\impdefx{type of \tcode{basic_path_builder::size_type}}@. // See [container.requirements] in \CppXVII.
    using difference_type        = @\impdefx{type of \tcode{basic_path_builder::size_type}}@. // See [container.requirements] in \CppXVII.
    using iterator               = @\impdefx{type of \tcode{basic_path_builder::iterator}}@. // See [container.requirements] in \CppXVII.
    using const_iterator         = @\impdefx{type of \tcode{basic_path_builder::const_iterator}}@. // See [container.requirements] in \CppXVII.
    using reverse_iterator       = std::reverse_iterator<iterator>;
    using const_reverse_iterator = std::reverse_iterator<const_iterator>;

    // \ref{\iotwod.pathbuilder.cons}, construct, copy, move, destroy:
    basic_path_builder() noexcept(noexcept(Allocator()));
    explicit basic_path_builder(const Allocator&) noexcept;
    explicit basic_path_builder(size_type n, const Allocator& = Allocator());
    basic_path_builder(size_type n, const value_type& value, const Allocator& = Allocator());
    template <class InputIterator>
    basic_path_builder(InputIterator first, InputIterator last, const Allocator& = Allocator());
    basic_path_builder(const basic_path_builder& x);
    basic_path_builder(basic_path_builder&&) noexcept;
    basic_path_builder(const basic_path_builder&, const Allocator&);
    basic_path_builder(basic_path_builder&&, const Allocator&);
    basic_path_builder(initializer_list<value_type>, const Allocator& = Allocator());
    ~basic_path_builder();
    basic_path_builder& operator=(const basic_path_builder& x);
    basic_path_builder& operator=(basic_path_builder&& x) noexcept(
      allocator_traits<Allocator>::propagate_on_container_move_assignment::value ||
      allocator_traits<Allocator>::is_always_equal::value);
    basic_path_builder& operator=(initializer_list<value_type>);
    template <class InputIterator>
    void assign(InputIterator first, InputIterator last);
    void assign(size_type n, const value_type& u);
    void assign(initializer_list<value_type>);
    allocator_type get_allocator() const noexcept;

    // \ref{\iotwod.pathbuilder.capacity}, capacity
    bool empty() const noexcept;
    size_type size() const noexcept;
    size_type max_size() const noexcept;
    size_type capacity() const noexcept;
    void resize(size_type sz);
    void resize(size_type sz, const value_type& c);
    void reserve(size_type n);
    void shrink_to_fit();

    // element access:
    reference operator[](size_type n);
    const_reference operator[](size_type n) const;
    const_reference at(size_type n) const;
    reference at(size_type n);
    reference front();
    const_reference front() const;
    reference back();
    const_reference back() const;

    // \ref{\iotwod.pathbuilder.modifiers}, modifiers:
    void new_figure(const basic_point_2d<typename
      GraphicsSurfaces::graphics_math_type>& pt) noexcept;
    void rel_new_figure(const basic_point_2d<typename
      GraphicsSurfaces::graphics_math_type>& pt) noexcept;
    void close_figure() noexcept;
    void matrix(const basic_matrix_2d<typename
      GraphicsSurfaces::graphics_math_type>& m) noexcept;
    void rel_matrix(const basic_matrix_2d<typename
      GraphicsSurfaces::graphics_math_type>& m) noexcept;
    void revert_matrix() noexcept;
    void line(const basic_point_2d<typename GraphicsSurfaces::graphics_math_type>& pt) noexcept;
    void rel_line(const basic_point_2d<typename
      GraphicsSurfaces::graphics_math_type>& dpt) noexcept;
    void quadratic_curve(const basic_point_2d<typename
      GraphicsSurfaces::graphics_math_type>& pt0, const basic_point_2d<typename
      GraphicsSurfaces::graphics_math_type>& pt2) noexcept;
    void rel_quadratic_curve(const basic_point_2d<typename
      GraphicsSurfaces::graphics_math_type>& pt0, const basic_point_2d<typename
      GraphicsSurfaces::graphics_math_type>& pt2) noexcept;
    void cubic_curve(const basic_point_2d<typename
      GraphicsSurfaces::graphics_math_type>& pt0, const basic_point_2d<typename
      GraphicsSurfaces::graphics_math_type>& pt1, const basic_point_2d<typename
      GraphicsSurfaces::graphics_math_type>& pt2) noexcept;
    void rel_cubic_curve(const basic_point_2d<typename
      GraphicsSurfaces::graphics_math_type>& dpt0, const basic_point_2d<typename
      GraphicsSurfaces::graphics_math_type>& dpt1, const basic_point_2d<typename
      GraphicsSurfaces::graphics_math_type>& dpt2) noexcept;
    void arc(const basic_point_2d<typename
      GraphicsSurfaces::graphics_math_type>& rad, float rot, float sang = pi<float>) noexcept;
    template <class... Args>
    reference emplace_back(Args&&... args);
    void push_back(const value_type& x);
    void push_back(value_type&& x);
    void pop_back();
    template <class... Args>
    iterator emplace(const_iterator position, Args&&... args);
    iterator insert(const_iterator position, const value_type& x);
    iterator insert(const_iterator position, value_type&& x);
    iterator insert(const_iterator position, size_type n, const value_type& x);
    template <class InputIterator>
    iterator insert(const_iterator position, InputIterator first, InputIterator last);
    iterator insert(const_iterator position,
    initializer_list<value_type> il);
    iterator erase(const_iterator position);
    iterator erase(const_iterator first, const_iterator last);
    void swap(basic_path_builder&) noexcept(
      allocator_traits<Allocator>::propagate_on_container_swap::value ||
      allocator_traits<Allocator>::is_always_equal::value);
    void clear() noexcept;

    // \ref{\iotwod.pathbuilder.iterators}, iterators:
    iterator begin() noexcept;
    const_iterator begin() const noexcept;
    const_iterator cbegin() const noexcept;
    iterator end() noexcept;
    const_iterator end() const noexcept;
    const_iterator cend() const noexcept;
    reverse_iterator rbegin() noexcept;
    const_reverse_iterator rbegin() const noexcept;
    const_reverse_iterator crbegin() const noexcept;
    reverse_iterator rend() noexcept;
    const_reverse_iterator rend() const noexcept;
    const_reverse_iterator crend() const noexcept;
  };

  template <class GraphicsSurfaces, class Allocator>
  bool operator==(const basic_path_builder<GraphicsSurfaces, Allocator>& lhs,
    const basic_path_builder<GraphicsSurfaces, Allocator>& rhs) noexcept;
  template <class GraphicsSurfaces, class Allocator>
  bool operator!=(const basic_path_builder<GraphicsSurfaces, Allocator>& lhs,
    const basic_path_builder<GraphicsSurfaces, Allocator>& rhs) noexcept;
  template <class GraphicsSurfaces, class Allocator>
  void swap(basic_path_builder<GraphicsSurfaces, Allocator>& lhs,
    basic_path_builder<GraphicsSurfaces, Allocator>& rhs) noexcept(noexcept(lhs.swap(rhs)));
}
\end{codeblock}

\rSec1 [\iotwod.pathbuilder.containerrequirements] {\tcode{basic_path_builder} container requirements}

\pnum
This class is a sequence container, as defined in [containers] in \CppXVII, and all sequence container requirements that apply specifically to \tcode{vector} shall also apply to this class.

\rSec1 [\iotwod.pathbuilder.cons] {\tcode{basic_path_builder} constructors, copy, and assignment}

\indexlibrary{\idxcode{basic_path_builder}!constructor}%
\begin{itemdecl}
basic_path_builder() noexcept(noexcept(Allocator()));
\end{itemdecl}
\begin{itemdescr}
\pnum
\effects
Constructs an empty \tcode{basic_path_builder}.
\end{itemdescr}
	
\indexlibrary{\idxcode{basic_path_builder}!constructor}%
\begin{itemdecl}
explicit basic_path_builder(const Allocator&);
\end{itemdecl}
\begin{itemdescr}
\pnum
\effects
Constructs an empty \tcode{basic_path_builder}, using the specified allocator.

\pnum
\complexity
Constant.
\end{itemdescr}

\indexlibrary{\idxcode{basic_path_builder}!constructor}%
\begin{itemdecl}
explicit basic_path_builder(size_type n, const Allocator& = Allocator());
\end{itemdecl}
\begin{itemdescr}
\pnum
\effects
Constructs a \tcode{basic_path_builder} with \tcode{n} default-inserted elements using the specified allocator.

\pnum
\complexity
Linear in \tcode{n}.
\end{itemdescr}

\indexlibrary{\idxcode{basic_path_builder}!constructor}%
\begin{itemdecl}
basic_path_builder(size_type n, const value_type& value,
  const Allocator& = Allocator());
\end{itemdecl}
\begin{itemdescr}
\pnum
\requires
\tcode{value_type} shall be \tcode{CopyInsertable} into \tcode{*this}.

\pnum
\effects
Constructs a \tcode{basic_path_builder} with n copies of \tcode{value}, using the specified allocator.

\pnum
\complexity
Linear in \tcode{n}.
\end{itemdescr}

\indexlibrary{\idxcode{basic_path_builder}!constructor}%
\begin{itemdecl}
template <class InputIterator>
basic_path_builder(InputIterator first, InputIterator last,
  const Allocator& = Allocator());
\end{itemdecl}
\begin{itemdescr}
\pnum
\effects
Constructs a \tcode{basic_path_builder} equal to the range \range{first}{last}, using the specified allocator.

\pnum
\complexity
Makes only $N$ calls to the copy constructor of \tcode{value_type} (where $N$
is the distance between
\tcode{first}
and
\tcode{last})
and no reallocations if iterators \tcode{first} and \tcode{last} are of forward, bidirectional, or random access categories.
It makes order
\tcode{N}
calls to the copy constructor of
\tcode{value_type}
and order
$\log(N)$
reallocations if they are just input iterators.

\end{itemdescr}

\rSec1 [\iotwod.pathbuilder.capacity] {\tcode{basic_path_builder} capacity}%

\indexlibrarymember{capacity}{basic_path_builder}%
\begin{itemdecl}
size_type capacity() const noexcept;
\end{itemdecl}
\begin{itemdescr}
\pnum
\returns
The total number of elements that the path builder can hold without requiring reallocation.
\end{itemdescr}

\indexlibrarymember{basic_path_builder}{reserve}%
\begin{itemdecl}
void reserve(size_type n);
\end{itemdecl}
\begin{itemdescr}
\pnum
\requires
\tcode{value_type} shall be \tcode{MoveInsertable} into \tcode{*this}.

\pnum
\effects
A directive that informs a path builder of a planned change in size, so that it can manage the storage
allocation accordingly. After \tcode{reserve()}, \tcode{capacity()} is greater or equal to the argument of \tcode{reserve} if
reallocation happens; and equal to the previous value of \tcode{capacity()} otherwise. Reallocation happens
at this point if and only if the current capacity is less than the argument of \tcode{reserve()}. If an exception
is thrown other than by the move constructor of a non-\tcode{CopyInsertable} type, there are no effects.

\pnum
\complexity
It does not change the size of the sequence and takes at most linear time in the size of the
sequence.

\pnum
\throws
\tcode{length_error} if \tcode{n >
max_size()}.\footnote{\tcode{reserve()} uses \tcode{Allocator::allocate()} which
may throw an appropriate exception.}

\pnum
\remarks
Reallocation invalidates all the references, pointers, and iterators
referring to the elements in the sequence.
No reallocation shall take place during insertions that happen
after a call to
\tcode{reserve()}
until the time when an insertion would make the size of the vector
greater than the value of
\tcode{capacity()}.
\end{itemdescr}

\indexlibrarymember{basic_path_builder}{shrink_to_fit}%
\begin{itemdecl}
void shrink_to_fit();
\end{itemdecl}
\begin{itemdescr}
\pnum
\requires
\tcode{value_type} shall be \tcode{MoveInsertable} into \tcode{*this}.

\pnum
\effects
\tcode{shrink_to_fit} is a non-binding request to reduce
\tcode{capacity()} to \tcode{size()}.
\begin{note}
The request is non-binding to allow latitude for
implementation-specific optimizations.
\end{note}
It does not increase \tcode{capacity()}, but may reduce \tcode{capacity()}
by causing reallocation. 
If an exception is thrown other than by the move constructor
of a non-\tcode{CopyInsertable} \tcode{value_type} there are no effects.

\pnum
\complexity Linear in the size of the sequence.

\pnum
\remarks Reallocation invalidates all the references, pointers, and 
iterators referring to the elements in the sequence. If no reallocation 
happens, they remain valid.
\end{itemdescr}

\indexlibrarymember{basic_path_builder}{swap}%
\begin{itemdecl}
void swap(basic_path_builder&)
  noexcept(allocator_traits<Allocator>::propagate_on_container_swap::value ||
  allocator_traits<Allocator>::is_always_equal::value);
\end{itemdecl}
\begin{itemdescr}
\pnum
\effects
Exchanges the contents and
\tcode{capacity()}
of
\tcode{*this}
with that of \tcode{x}.

\pnum
\complexity
Constant time.
\end{itemdescr}

\indexlibrary{basic_path_builder}{resize}%
\begin{itemdecl}
void resize(size_type sz);
\end{itemdecl}
\begin{itemdescr}
\pnum
\effects
If \tcode{sz < size()}, erases the last \tcode{size() - sz} elements
from the sequence. Otherwise, appends \tcode{sz - size()} default-inserted 
elements to the sequence.

\pnum
\requires
\tcode{value_type} shall be
\tcode{MoveInsertable} and \tcode{DefaultInsertable} into \tcode{*this}.

\pnum
\remarks
If an exception is thrown other than by the move constructor of a 
non-\tcode{CopyInsertable}
\tcode{value_type} there are no effects.
\end{itemdescr}

\indexlibrary{basic_path_builder}{resize}%
\begin{itemdecl}
void resize(size_type sz, const value_type& c);
\end{itemdecl}
\begin{itemdescr}
\pnum
\effects
If \tcode{sz < size()}, erases the last \tcode{size() - sz} elements
from the sequence. Otherwise,
appends \tcode{sz - size()} copies of \tcode{c} to the sequence.

\pnum
\requires
\tcode{value_type} shall be \tcode{CopyInsertable} into \tcode{*this}.

\pnum
\remarks
If an exception is thrown there are no effects.
\end{itemdescr}

\rSec1 [\iotwod.pathbuilder.modifiers] {\tcode{basic_path_builder} modifiers}

\indexlibrarymember{basic_path_builder}{new_figure}%
\begin{itemdecl}
void new_figure(point_2d pt) noexcept;
\end{itemdecl}
\begin{itemdescr}
\pnum
\effects
Adds a \tcode{figure_items::figure_item} object constructed from \tcode{figure_items::abs_new_figure(pt)} to the end of the path.
\end{itemdescr}

\indexlibrarymember{basic_path_builder}{rel_new_figure}%
\begin{itemdecl}
void rel_new_figure(point_2d pt) noexcept;
\end{itemdecl}
\begin{itemdescr}
\pnum
\effects
Adds a \tcode{figure_items::figure_item} object constructed from \tcode{figure_items::rel_new_figure(pt)} to the end of the path.
\end{itemdescr}

\indexlibrarymember{basic_path_builder}{close_figure}%
\begin{itemdecl}
void close_figure() noexcept;
\end{itemdecl}
\begin{itemdescr}
\pnum
\requires
The current point contains a value.

\pnum
\effects
Adds a \tcode{figure_items::figure_item} object constructed from \tcode{figure_items::close_figure()} to the end of the path.
\end{itemdescr}

\indexlibrarymember{basic_path_builder}{set_matrix}%
\begin{itemdecl}
void matrix(const matrix_2d& m) noexcept;
\end{itemdecl}
\begin{itemdescr}
\pnum
\requires
The matrix \tcode{m} shall be invertible.

\pnum
\effects
Adds a \tcode{figure_items::figure_item} object constructed from \tcode{(figure_items::abs_matrix(m)} to the end of the path.
\end{itemdescr}

\indexlibrarymember{basic_path_builder}{modify_matrix}%
\begin{itemdecl}
void rel_matrix(const matrix_2d& m) noexcept;
\end{itemdecl}
\begin{itemdescr}
\pnum
\requires
The matrix \tcode{m} shall be invertible.

\pnum
\effects
Adds a \tcode{figure_items::figure_item} object constructed from \tcode{(figure_items::rel_matrix(m)} to the end of the path.
\end{itemdescr}

\indexlibrarymember{basic_path_builder}{revert_matrix}%
\begin{itemdecl}
void revert_matrix() noexcept;
\end{itemdecl}
\begin{itemdescr}
\pnum
\effects
Adds a \tcode{figure_items::figure_item} object constructed from \tcode{(figure_items::revert_matrix()} to the end of the path.
\end{itemdescr}

\indexlibrarymember{basic_path_builder}{line}%
\begin{itemdecl}
void line(point_2d pt) noexcept;
\end{itemdecl}
\begin{itemdescr}
\pnum
Adds a \tcode{figure_items::figure_item} object constructed from \tcode{figure_items::abs_line(pt)} to the end of the path.
\end{itemdescr}

\indexlibrarymember{basic_path_builder}{rel_line}%
\begin{itemdecl}
void rel_line(point_2d dpt) noexcept;
\end{itemdecl}
\begin{itemdescr}
\pnum
\effects
Adds a \tcode{figure_items::figure_item} object constructed from \tcode{figure_items::rel_line(pt)} to the end of the path.
\end{itemdescr}

\indexlibrarymember{basic_path_builder}{quadratic_curve}%
\begin{itemdecl}
void quadratic_curve(point_2d pt0, point_2d pt1) noexcept;
\end{itemdecl}
\begin{itemdescr}
\pnum
\effects
Adds a \tcode{figure_items::figure_item} object constructed from\\ \tcode{figure_items::abs_quadratic_curve(pt0, pt1)} to the end of the path.
\end{itemdescr}

\indexlibrarymember{basic_path_builder}{rel_quadratic_curve}%
\begin{itemdecl}
void rel_quadratic_curve(point_2d dpt0, point_2d dpt1)
  noexcept;
\end{itemdecl}
\begin{itemdescr}
\pnum
\effects
Adds a \tcode{figure_items::figure_item} object constructed from\\ \tcode{figure_items::rel_quadratic_curve(dpt0, dpt1)} to the end of the path.
\end{itemdescr}

\indexlibrarymember{basic_path_builder}{cubic_curve}%
\begin{itemdecl}
void cubic_curve(point_2d pt0, point_2d pt1,
  point_2d pt2) noexcept;
\end{itemdecl}
\begin{itemdescr}
\pnum
\effects
\pnum
Adds a \tcode{figure_items::figure_item} object constructed from \tcode{figure_items::abs_cubic_curve(pt0, pt1, pt2)} to the end of the path.
\end{itemdescr}

\indexlibrarymember{basic_path_builder}{rel_cubic_curve}%
\begin{itemdecl}
void rel_cubic_curve(point_2d dpt0, point_2d dpt1,
  point_2d dpt2) noexcept;
\end{itemdecl}
\begin{itemdescr}
\pnum
\effects
Adds a \tcode{figure_items::figure_item} object constructed from \tcode{figure_items::rel_cubic_curve(dpt0, dpt1, dpt2)} to the end of the path.
\end{itemdescr}

\indexlibrarymember{basic_path_builder}{arc}%
\begin{itemdecl}
void arc(point_2d rad, float rot, float sang) noexcept;
\end{itemdecl}
\begin{itemdescr}
\pnum
\effects
Adds a \tcode{figure_items::figure_item} object constructed from \\ \tcode{figure_items::arc(rad, rot, sang)} to the end of the path.
\end{itemdescr}

\indexlibrarymember{basic_path_builder}{insert}%
\indexlibrarymember{basic_path_builder}{emplace_back}%
\indexlibrarymember{basic_path_builder}{push_back}%
\begin{itemdecl}
iterator insert(const_iterator position, const value_type& x);
iterator insert(const_iterator position, value_type&& x);
iterator insert(const_iterator position, size_type n, const value_type& x);
template <class InputIterator>
iterator insert(const_iterator position, InputIterator first,
  InputIterator last);
iterator insert(const_iterator position, initializer_list<value_type>);
template <class... Args>
reference emplace_back(Args&&... args);
template <class... Args>
iterator emplace(const_iterator position, Args&&... args);
void push_back(const value_type& x);
void push_back(value_type&& x);
\end{itemdecl}

\begin{itemdescr}
\pnum
\remarks
Causes reallocation if the new size is greater than the old capacity.
Reallocation invalidates all the references, pointers, and iterators
referring to the elements in the sequence.
If no reallocation happens, all the iterators and references before the insertion point remain valid.
If an exception is thrown other than by
the copy constructor, move constructor,
assignment operator, or move assignment operator of
\tcode{value_type} or by any \tcode{InputIterator} operation
there are no effects.
If an exception is thrown while inserting a single element at the end and
\tcode{value_type} is \tcode{CopyInsertable} or \tcode{is_nothrow_move_constructible_v<value_type>}
is \tcode{true}, there are no effects.
Otherwise, if an exception is thrown by the move constructor of a non-\tcode{CopyInsertable}
\tcode{value_type}, the effects are unspecified.

\pnum
\complexity
The complexity is linear in the number of elements inserted plus the 
distance to the end of the path builder.
\end{itemdescr}

\indexlibrarymember{erase}{basic_path_builder}%
\indexlibrarymember{pop_back}{basic_path_builder}%
\begin{itemdecl}
iterator erase(const_iterator position);
iterator erase(const_iterator first, const_iterator last);
void pop_back();
\end{itemdecl}

\begin{itemdescr}
\pnum
\effects
Invalidates iterators and references at or after the point of the erase.

\pnum
\complexity
The destructor of \tcode{value_type} is called the number of times equal to 
the number of the elements erased, but the assignment operator
of \tcode{value_type} is called the number of times equal to the number of
elements in the path builder after the erased elements.

\pnum
\throws
Nothing unless an exception is thrown by the copy constructor, move 
constructor, assignment operator, or move assignment operator of
\tcode{value_type}.
\end{itemdescr}

\rSec1 [\iotwod.pathbuilder.iterators] {\tcode{basic_path_builder} iterators}

\indexlibrarymember{begin}{basic_path_builder}%
\indexlibrarymember{cbegin}{basic_path_builder}%
\begin{itemdecl}
iterator begin() noexcept;
const_iterator begin() const noexcept;
const_iterator cbegin() const noexcept;
\end{itemdecl}
\begin{itemdescr}
\pnum
\returns
An iterator referring to the first \tcode{figure_items::figure_item} item in the path.

\pnum
\remarks
Changing a \tcode{figure_items::figure_item} object or otherwise modifying the path in a way that violates the preconditions of that \tcode{figure_items::figure_item} object or of any subsequent \tcode{figure_items::figure_item} object in the path produces undefined behavior when the path is interpreted as described in \ref{\iotwod.paths.interpretation} unless all of the violations are fixed prior to such interpretation.
\end{itemdescr}

\indexlibrarymember{end}{basic_path_builder}%
\indexlibrarymember{cend}{basic_path_builder}%
\begin{itemdecl}
iterator end() noexcept;
const_iterator end() const noexcept;
const_iterator cend() const noexcept;
\end{itemdecl}
\begin{itemdescr}
\pnum
\returns
An iterator which is the past-the-end value.

\pnum
\remarks
Changing a \tcode{figure_items::figure_item} object or otherwise modifying the path in a way that violates the preconditions of that \tcode{figure_items::figure_item} object or of any subsequent \tcode{figure_items::figure_item} object in the path produces undefined behavior when the path is interpreted as described in \ref{\iotwod.paths.interpretation} unless all of the violations are fixed prior to such interpretation.
\end{itemdescr}

\indexlibrarymember{rbegin}{basic_path_builder}%
\indexlibrarymember{crbegin}{basic_path_builder}%
\begin{itemdecl}
reverse_iterator rbegin() noexcept;
const_reverse_iterator rbegin() const noexcept;
const_reverse_iterator crbegin() const noexcept;
\end{itemdecl}
\begin{itemdescr}
\pnum
\returns
An iterator which is semantically equivalent to \tcode{reverse_iterator(end)}.

\pnum
\remarks
Changing a \tcode{figure_items::figure_item} object or otherwise modifying the path in a way that violates the preconditions of that \tcode{figure_items::figure_item} object or of any subsequent \tcode{figure_items::figure_item} object in the path produces undefined behavior when the path is interpreted as described in \ref{\iotwod.paths.interpretation} all of the violations are fixed prior to such interpretation.
\end{itemdescr}

\indexlibrarymember{rend}{basic_path_builder}%
\indexlibrarymember{crend}{basic_path_builder}%
\begin{itemdecl}
reverse_iterator rend() noexcept;
const_reverse_iterator rend() const noexcept;
const_reverse_iterator crend() const noexcept;
\end{itemdecl}
\begin{itemdescr}
\pnum
\returns
An iterator which is semantically equivalent to \tcode{reverse_iterator(begin)}.

\pnum
\remarks
Changing a \tcode{figure_items::figure_item} object or otherwise modifying the path in a way that violates the preconditions of that \tcode{figure_items::figure_item} object or of any subsequent \tcode{figure_items::figure_item} object in the path produces undefined behavior when the path is interpreted as described in \ref{\iotwod.paths.interpretation} unless all of the violations are fixed prior to such interpretation.
\end{itemdescr}

\rSec1[\iotwod.pathbuilder.special] {\tcode{basic_path_builder} specialized algorithms}

\indexlibrary{\idxcode{swap}!\idxcode{basic_path_builder}}%
\begin{itemdecl}
template <class Allocator>
void swap(basic_path_builder<Allocator>& lhs, basic_path_builder<Allocator>& rhs)
  noexcept(noexcept(lhs.swap(rhs)));
\end{itemdecl}
\begin{itemdescr}
\pnum
\effects
As if by \tcode{lhs.swap(rhs)}.
\end{itemdescr}

\addtocounter{SectionDepthBase}{-1}
