%!TEX root = io2d.tex
\rSec0 [\iotwod.displaypt] {Class template \tcode{basic_display_point}}

\rSec1 [\iotwod.displaypt.intro] {\tcode{basic_display_point} description}

\indexlibrary{\idxcode{basic_display_point}}%
\pnum
The class template \tcode{basic_display_point} describes an integral point used to describe certain properties of surfaces.

\pnum
It has an \term{x coordinate} of type \tcode{int} and a \term{y coordinate} of type \tcode{int}.

\pnum
The data are stored in an object of type \tcode{typename GraphicsMath::display_point_data_type}. It is accessible using the \tcode{data} member functions.

\rSec1 [\iotwod.displaypt.synopsis] {\tcode{basic_display_point} synopsis}

\begin{codeblock}
namespace std::experimental::io2d::v1 {
  template <class GraphicsMath>
  class basic_display_point {
  public:
    using data_type = typename GraphicsMath::display_point_data_type;
    
    // \ref{\iotwod.displaypt.cons}, constructors:
    basic_display_point() noexcept;
    basic_display_point(int x, int y) noexcept;
    basic_display_point(const data_type& val) noexcept;

    // \ref{\iotwod.displaypt.accessors}, accessors:
    const data_type& data() const noexcept;
    data_type& data() noexcept;

    // \ref{\iotwod.displaypt.modifiers}, modifiers:
    void x(int val) noexcept;
    void y(int val) noexcept;

    // \ref{\iotwod.displaypt.observers}, observers:
    int x() const noexcept;
    int y() const noexcept;
  };

  // \ref{\iotwod.displaypt.ops}, operators:
  template <class GraphicsMath>
  bool operator==(const basic_display_point<GraphicsMath>& lhs,
    const basic_display_point<GraphicsMath>& rhs) noexcept;
  template <class GraphicsMath>
  bool operator!=(const basic_display_point<GraphicsMath>& lhs,
    const basic_display_point<GraphicsMath>& rhs) noexcept;
}
\end{codeblock}

\rSec1 [\iotwod.displaypt.cons] {\tcode{basic_display_point} constructors}

\indexlibrary{\idxcode{basic_display_point}!constructor}%
\begin{itemdecl}
basic_display_point() noexcept;
\end{itemdecl}
\begin{itemdescr}
\pnum
\effects
Constructs an object of type \tcode{basic_display_point}.

\pnum
\postconditions
\tcode{data() == GraphicsMath::create_display_point()}.

\pnum
\remarks
The x coordinate is \tcode{0} and the y coordinate is \tcode{0}.
\end{itemdescr}

\indexlibrary{\idxcode{basic_display_point}!constructor}%
\begin{itemdecl}
basic_display_point(int x, int y) noexcept;
\end{itemdecl}
\begin{itemdescr}
\pnum
\effects
Constructs an object of type \tcode{basic_display_point}.

\pnum
\postconditions
\tcode{data() == GraphicsMath::create_display_point(x, y)}.

\pnum
\remarks
The x coordinate is \tcode{x} and the y coordinate is \tcode{y}.
\end{itemdescr}

\indexlibrary{\idxcode{basic_display_point}!constructor}%
\begin{itemdecl}
basic_display_point(const data_type& val) noexcept;
\end{itemdecl}
\begin{itemdescr}
\pnum
\effects
Constructs an object of type \tcode{basic_display_point}.

\pnum
\postconditions
\tcode{data() == val}.

\pnum
\remarks
The x coordinate is \tcode{GraphicsMath::x(val)} and the y coordinate is \tcode{GraphicsMath::y(val)}.
\end{itemdescr}

\rSec1 [\iotwod.displaypt.accessors]{\tcode{basic_display_point} accessors}

\indexlibrarymember{data}{basic_display_point}%
\begin{itemdecl}
const data_type& data() const noexcept;
data_type& data() noexcept;
\end{itemdecl}
\begin{itemdescr}
\pnum
\returns
A reference to the \tcode{basic_display_point} object's data object (See: \ref{\iotwod.displaypt.intro}).
\end{itemdescr}

\rSec1 [\iotwod.displaypt.modifiers]{\tcode{basic_display_point} modifiers}

\indexlibrarymember{x}{basic_display_point}%
\begin{itemdecl}
void x(int v) noexcept;
\end{itemdecl}
\begin{itemdescr}
\pnum
\effects
Equivalent to \tcode{GraphicsMath::x(data(), v);}
\end{itemdescr}

\indexlibrarymember{y}{basic_display_point}%
\begin{itemdecl}
void y(int v) noexcept;
\end{itemdecl}
\begin{itemdescr}
\pnum
\effects
Equivalent to \tcode{GraphicsMath::y(data(), v);}
\end{itemdescr}

\rSec1 [\iotwod.displaypt.observers]{\tcode{basic_display_point} observers}

\indexlibrarymember{x}{basic_display_point}%
\begin{itemdecl}
int x() const noexcept;
\end{itemdecl}
\begin{itemdescr}
\pnum
\returns
\tcode{GraphicsMath::x(data())}.
\end{itemdescr}

\indexlibrarymember{y}{basic_display_point}%
\begin{itemdecl}
int y() const noexcept;
\end{itemdecl}
\begin{itemdescr}
\pnum
\returns
\tcode{GraphicsMath::y(data())}.
\end{itemdescr}

\rSec1 [\iotwod.displaypt.ops] {\tcode{basic_display_point} operators}

\indexlibrarymember{operator==}{basic_display_point}%
\begin{itemdecl}
bool operator==(const basic_display_point<GraphicsMath>& lhs,
  const basic_display_point<GraphicsMath>& rhs) noexcept;
\end{itemdecl}
\begin{itemdescr}
\pnum
\returns
\tcode{GraphicsMath::equal(lhs.data(), rhs.data())}.
\end{itemdescr}

\indexlibrarymember{operator!=}{basic_display_point}%
\begin{itemdecl}
bool operator!=(const basic_display_point<GraphicsMath>& lhs,
  const basic_display_point<GraphicsMath>& rhs) noexcept;
\end{itemdecl}
\begin{itemdescr}
\pnum
\returns
\tcode{GraphicsMath::not_equal(lhs.data(), rhs.data())}.
\end{itemdescr}
