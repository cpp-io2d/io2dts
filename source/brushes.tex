%!TEX root = io2d.tex

\rSec0 [\iotwod.brushes] {Brushes}

\rSec1 [\iotwod.brushes.intro] {Overview of brushes}

\pnum
Brushes contain visual data and serve as sources of visual data for rendering and composing operations.

\pnum
There are four types of brushes:
\begin{itemize}
	\item solid color;
	\item linear gradient;
	\item radial gradient; and,
	\item surface.
\end{itemize}

\pnum
Once a brush is created, its visual data is immutable.

\pnum
\begin{note}
While copy and move operations along with a swap operation can change the visual data that a brush contains, the visual data itself is not modified.
\end{note}

\pnum
A brush is used either as a \term{source brush} or a \term{mask brush} (\ref{\iotwod.surface.rendering.brushes}).

\pnum
When a brush is used in a rendering and composing operation, if it is used as a source brush, it has a \tcode{brush_props} object that describes how the brush is interpreted for purposes of sampling. If it is used as a mask brush, it has a \tcode{mask_props} object that describes how the brush is interpreted for purposes of sampling.

\pnum
The \tcode{basic_brush_props} (\ref{\iotwod.brushprops.summary}) and \tcode{basic_mask_props} (\ref{\iotwod.maskprops.summary}) classes each have a \term{wrap mode} and a \term{filter}. The \tcode{basic_brush_props} class also has a \term{brush matrix} and a \term{fill rule}. The \tcode{basic_mask_props} class also has a \term{mask matrix}. Where possible, the terms that are common between the two classes are referenced without regard to whether the brush is being used as a source brush or a mask brush.

%\pnum
%A brush has its own coordinate space.
%
%\pnum
%If it is being used as a source brush, it is the Brush Coordinate Space (\ref{\iotwod.surface.coordinatespaces}).
%
%\pnum
%If it is being used as a mask brush, it is the Mask Coordinate Space (\ref{\iotwod.surface.coordinatespaces}).
%
%\pnum
%It possesses a \tcode{filter} value and a \tcode{wrap_mode} value, which combine to determine the visual data value returned when a composing operation samples a brush.
%
%\pnum
%The \tcode{filter} value determines the value returned by a brush when a composing operation samples a brush. When the visual data of the brush is a pixmap and the point the composing operation requests does not directly correspond to the exact coordinate of a pixel within the pixmap, the \tcode{filter} value is used to determine the value of the visual data that is returned. If the visual data of the brush is not a pixmap, the \tcode{filter} value is irrelevant.
%
%\pnum
%The \tcode{wrap_mode} value controls what happens when a composing operation needs to sample from a point that is outside of the bounds of the brush.
%
\pnum
Solid color brushes are unbounded and as such always produce the same visual data when sampled from, regardless of the requested point.

\pnum
Linear gradient and radial gradient brushes share similarities with each other that are not shared by the other types of brushes. This is discussed in more detail elsewhere (\ref{\iotwod.gradients}).

\pnum
Surface brushes are constructed from a \tcode{basic_image_surface} object. Their visual data is a pixmap, which has implications on sampling from the brush that are not present in the other brush types.

\addtocounter{SectionDepthBase}{1}
%!TEX root = io2d.tex
\rSec0 [gradients] {Gradient brushes}

\rSec1 [gradients.common] {Common properties of gradients}
\pnum
Gradients are formed, in part, from a collection of \tcode{color_stop} objects.

\pnum
The collection of \tcode{color_stop} objects contribute to defining a brush which, when sampled from, returns a value that is interpolated based on those color stops.

\rSec1 [gradients.linear] {Linear gradients}

\pnum
A linear gradient is a type of gradient.

\pnum
A linear gradient has a \term{begin point} and an \term{end point}, each of which are objects of type \tcode{vector_2d}.

\pnum
A linear gradient for which the distance between its begin point and its end point is not greater than \tcode{numeric_limits<double>::epsilon()} is a \term{degenerate linear gradient}.

\pnum
All attempts to sample from a a degenerate linear gradient return the color \tcode{bgra_color::transparent_black()}. The remainder of \ref{gradients} is inapplicable to degenerate linear gradients.

\pnum
The begin point and end point of a linear gradient define a line segment, with a color stop offset value of 0.0 corresponding to the begin point and a color stop offset value of 1.0 corresponding to the end point.

\pnum
Color stop offset values in the range \orange{0.0}{1.0} linearly correspond to points on the line segment.

\pnum
\enterexample
Given a linear gradient with a begin point of \tcode{vector_2d(0.0, 0.0)} and an end point of \tcode{vector_2d(10.0, 5.0)}, a color stop offset value of 0.6 would correspond to the point \tcode{vector_2d(6.0, 3.0)}.
\exitexample

\pnum
To determine the offset value of a point $p$ for a linear gradient, perform the following steps:
\begin{enumeratea}
\item Create a line at the begin point of the linear gradient, the \term{begin line}, and another line at the end point of the linear gradient, the \term{end line}, with each line being perpendicular to the \term{gradient line segment}, which is the line segment delineated by the begin point and the end point.

\item Using the begin line, $p$, and the end line, create a line, the \term{$p$ line}, which is parallel to the gradient line segment.

\item Defining $dp$ as the distance between $p$ and the point where the $p$ line intersects the begin line and $dt$ as the distance between the point where the $p$ line intersects the begin line and the point where the $p$ line intersects the end line, the offset value of $p$ is $dp \div dt$.

\item The offset value shall be negative if
\begin{itemize}
\item $p$ is not on the line segment delineated by the point where the $p$ line intersects the begin line and the point where the $p$ line intersects the end line; and,

\item the distance between $p$ and the point where the $p$ line intersects the begin line is less than the distance between $p$ and the point where the $p$ line intersects the end line.
\end{itemize}
\end{enumeratea}

\rSec1 [gradients.radial] {Radial gradients}

\pnum
A radial gradient is a type of gradient.

\pnum
Aa radial gradient has a \term{start circle} and an \term{end circle}, each of which is defined by a \tcode{circle} object.

\pnum
A radial gradient is a \term{degenerate radial gradient} if:
\begin{itemize}
\item its start circle has a negative radius; or,
\item its end circle has a negative radius; or,
\item the distance between the center point of its start circle and the center point of its end circle is not greater than \tcode{numeric_limits<double>::epsilon()} and the difference between the radius of its start circle and the radius of its end circle is not greater than \tcode{numeric_limits<double>::epsilon()}; or,
\item its start circle has a radius of 0.0 and its end circle has a radius of 0.0.
\end{itemize}

\pnum
All attempts to sample from a \tcode{brush} object created using a degenerate radial gradient return the color \tcode{bgra_color::transparent_black()}. The remainder of \ref{gradients} is inapplicable to degenerate radial gradients.

\pnum
A color stop offset of 0.0 corresponds to all points along the diameter of the start circle or to its center point if it has a radius value of 0.0.

\pnum
A color stop offset of 1.0 corresponds to all points along the diameter of the end circle or to its center point if it has a radius value of 0.0.

\pnum
A radial gradient shall be rendered as a continuous series of interpolated circles defined by the following equations:
\begin{enumeratea}
\item $x(o) = x_{start} + o \times (x_{end} - x_{start})$
\item $y(o) = y_{start} + o \times (y_{end} - y_{start})$
\item $radius(o) = radius_{start} + o \times (radius_{end} - radius_{start})$
\end{enumeratea}
where $o$ is a color stop offset value.

\pnum
The range of potential values for $o$ shall be determined by the Wrap Mode:
\begin{itemize}
\item For \tcode{wrap_mode::none}, the range of potential values for $o$ is $[0,1]$.
\item For all other \tcode{wrap_mode} values, the range of potential values for $o$ is\\ $[$~\tcode{numeric_limits<double>::lowest(),numeric_limits<double>::max()}~$]$.
\end{itemize}

\pnum
The interpolated circles shall be rendered starting from the smallest potential value of $o$.

\pnum
An interpolated circle shall not be rendered if its value for $o$ results in $radius(o)$ evaluating to a negative value.

\rSec1 [gradients.sampling] {Sampling from gradients}

\pnum
For any offset value $o$, its color value shall be determined according to the following rules:

\begin{enumeratea}
\item If there are less than two color stops or if all color stops have the same offset value, then the color value of every offset value shall be \tcode{bgra_color::transparent_black()} and the remainder of these rules are inapplicable.

\item If exactly one color stop has an offset value equal to $o$, $o$'s color value shall be the color value of that color stop and the remainder of these rules are inapplicable.

\item If two or more color stops have an offset value equal to $o$, $o$'s color value shall be the color value of the color stop which has the lowest index value among the set of color stops that have an offset value equal to $o$ and the remainder of \ref{gradients.sampling} is inapplicable.

\item When no color stop has the offset value of \tcode{0.0}, then, defining $n$ to be the offset value that is nearest to \tcode{0.0} among the offset values in the set of all color stops, if $o$ is in the offset range $[0, n)$, $o$'s color value shall be \tcode{bgra_color::transparent_black()} and the remainder of these rules are inapplicable.
\enternote
Since the range described does not include $n$, it does not matter how many color stops have $n$ as their offset value for purposes of this rule.
\exitnote

\item When no color stop has the offset value of \tcode{1.0}, then, defining $n$ to be the offset value that is nearest to \tcode{1.0} among the offset values in the set of all color stops, if $o$ is in the offset range $(n, 1]$, $o$'s color value shall be \tcode{bgra_color::transparent_black()} and the remainder of these rules are inapplicable.
\enternote
Since the range described does not include $n$, it does not matter how many color stops have $n$ as their offset value for purposes of this rule.
\exitnote

\item Each color stop has, at most, two adjacent color stops: one to its left and one to its right.

\item Adjacency of color stops is initially determined by offset values. If two or more color stops have the same offset value then index values are used to determine adjacency as described below.

\item For each color stop \textit{a}, the \term{set of color stops to its left} are those color stops which have an offset value which is closer to \tcode{0.0} than \textit{a}'s offset value.
\enternote
This includes any color stops with an offset value of \tcode{0.0} provided that \textit{a}'s offset value is not \tcode{0.0}.
\exitnote

\item For each color stop \textit{b}, the \term{set of color stops to its right} are those color stops which have an offset value which is closer to \tcode{1.0} than \textit{b}'s offset value.
\enternote
This includes any color stops with an offset value of \tcode{1.0} provided that \textit{b}'s offset value is not \tcode{1.0}.
\exitnote

\item A color stop which has an offset value of \tcode{0.0} does not have an adjacent color stop to its left.

\item A color stop which has an offset value of \tcode{1.0} does not have an adjacent color stop to its right.

\item If a color stop \textit{a}'s set of color stops to its left consists of exactly one color stop, that color stop is the color stop that is adjacent to \textit{a} on its left.

\item If a color stop \textit{b}'s set of color stops to its right consists of exactly one color stop, that color stop is the color stop that is adjacent to \textit{b} on its right.

\item If two or more color stops have the same offset value then the color stop with the lowest index value is the only color stop from that set of color stops which can have a color stop that is adjacent to it on its left and the color stop with the highest index value is the only color stop from that set of color stops which can have a color stop that is adjacent to it on its right. This rule takes precedence over all of the remaining rules.

\item If a color stop can have an adjacent color stop to its left, then the color stop which is adjacent to it to its left is the color stop from the set of color stops to its left which has an offset value which is closest to its offset value. If two or more color stops meet that criteria, then the color stop which is adjacent to it to its left is the color stop which has the highest index value from the set of color stops to its left which are tied for being closest to its offset value.

\item If a color stop can have an adjacent color stop to its right, then the color stop which is adjacent to it to its right is the color stop from the set of color stops to its right which has an offset value which is closest to its offset value. If two or more color stops meet that criteria, then the color stop which is adjacent to it to its right is the color stop which has the lowest index value from the set of color stops to its right which are tied for being closest to its offset value.

\item Where the value of $o$ is in the range $[0,1]$, its color value shall be determined by interpolating between the color stop, $r$, which is the color stop whose offset value is closest to $o$ without being less than $o$ and which can have an adjacent color stop to its left, and the color stop that is adjacent to $r$ on $r$'s left. The acceptable forms of interpolating between color values is set forth later in this section.

\item Where the value of $o$ is outside the range $[0,1]$, its color value depends on the value of Wrap Mode:
	\begin{itemize}
	\item If Wrap Mode is \tcode{wrap_mode::none}, the color value of $o$ shall be \tcode{bgra_color::transparent_black()}.
	
	\item If Wrap Mode is \tcode{wrap_mode::pad}, if $o$ is negative then the color value of $o$ shall be the same as-if the value of $o$ was \tcode{0.0}, otherwise the color value of $o$ shall be the same as-if the value of $o$ was \tcode{1.0}.
	
	\item If Wrap Mode is \tcode{wrap_mode::repeat}, then \tcode{1.0} shall be added to or subtracted from $o$ until $o$ is in the range $[0,1]$, at which point its color value is the color value for the modified value of $o$ as determined by these rules.
	\enterexample
	Given $o == 2.1$, after application of this rule $o == 0.1$ and the color value of $o$ shall be the same value as-if the initial value of $o$ was $0.1$.
	
	Given $o == -0.3$, after application of this rule $o == 0.7$ and the color value of $o$ shall be the same as-if the initial value of $o$ was $0.7$.
	\exitexample
	
	\item If Wrap Mode is \tcode{wrap_mode::reflect}, $o$ shall be set to the absolute value of $o$, then 2.0 shall be subtracted from $o$ until $o$ is in the range $[0,2]$, then if $o$ is in the range $(1,2]$ then $o$ shall be set to \tcode{1.0 - (o - 1.0)}, at which point its color value is the color value for the modified value of $o$ as determined by these rules.
	\enterexample
	Given $o == 2.8$, after application of this rule $o == 0.8$ and the color value of $o$ shall be the same value as-if the initial value of $o$ was $0.8$.
	
	Given $o == 3.6$, after application of this rule $o == 0.4$ and the color value of $o$ shall be the same value as-if the initial value of $o$ was $0.4$.
	
	Given $o == -0.3$, after application of this rule $o == 0.3$ and the color value of $o$ shall be the same as-if the initial value of $o$ was $0.3$.
	
	Given $o == -5.8$, after application of this rule $o == 0.2$ and the color value of $o$ shall be the same as-if the initial value of $o$ was $0.2$.
	\exitexample
	\end{itemize}
\end{enumeratea}

\pnum
It is unspecified whether the interpolation between the color values of two adjacent color stops is performed linearly on each color channel or is performed by a linear color interpolation algorithm implemented in hardware (typically in a graphics processing unit).

\pnum
Implementations shall interpolate between alpha channel values of adjacent color stops linearly except as provided in the following paragraph.

\pnum
A conforming implementation may use the alpha channel interpolation results from a linear color interpolation algorithm implemented in hardware even if those results differ from the results required by the previous paragraph.

%!TEX root = io2d.tex
\rSec0 [\iotwod.wrapmode] {Enum class \tcode{wrap_mode}}

\rSec1 [\iotwod.wrapmode.summary] {\tcode{wrap_mode} summary}

\pnum
The \tcode{wrap_mode} enum class describes how a point's visual data is 
determined if it is outside the bounds of the \term{source brush} (\ref{\iotwod.surface.rendering.brushes}) when sampling.

\pnum
Depending on the source brush's \tcode{filter} value, the visual data of several points may be required to determine the appropriate visual data value for the point that is being sampled. In this case, each point shall be sampled according to the source brush's \tcode{wrap_mode} value with two exceptions:
\begin{enumeratea}
\item If the point to be sampled is within the bounds of the source brush and the source brush's \tcode{wrap_mode} value is \tcode{wrap_mode::none}, then if the source brush's \tcode{filter} value requires that one or more points which are outside of the bounds of the source brush shall be sampled, each of those points shall be sampled as-if the source brush's \tcode{wrap_mode} value is \tcode{wrap_mode::pad} rather than \tcode{wrap_mode::none}.
\item If the point to be sampled is within the bounds of the source brush and the source brush's \tcode{wrap_mode} value is \tcode{wrap_mode::none}, ce Brush and the source brush's \tcode{wrap_mode} value is \tcode{wrap_mode::none}, then if the source brush's \tcode{filter} value requires that one or more points which are inside of the bounds of the source brush shall be sampled, each of those points shall be sampled such that the visual data that is returned shall be the equivalent of \tcode{rgba_color::transparent_black()}.
\end{enumeratea}

\pnum
If a point to be sampled does not have a defined visual data element and the search for the nearest point with defined visual data produces two or more points with defined visual data that are equidistant from the point to be sampled, the returned visual data shall be an \unspecnorm value which is the visual data of one of those equidistant points. Where possible, implementations should choose the among the equidistant points that have an \xaxis value and a \yaxis value that is nearest to \tcode{0.0f}.

\pnum
See Table~\ref{tab:\iotwod.wrap_mode.meanings} for the meaning of each \tcode{wrap_mode} enumerator.

\rSec1 [\iotwod.wrapmode.synopsis] {\tcode{wrap_mode} Synopsis}

\begin{codeblock}
namespace std::experimental::io2d::v1 {
  enum class wrap_mode {
    none,
    repeat,
    reflect,
    pad
  };
}
\end{codeblock}

\rSec1 [\iotwod.wrapmode.enumerators] {\tcode{wrap_mode} Enumerators}
\begin{libreqtab2}
 {\tcode{wrap_mode} enumerator meanings}
 {tab:\iotwod.wrap_mode.meanings}
 \\ \topline
 \lhdr{Enumerator}
 & \rhdr{Meaning}
 \\ \capsep
 \endfirsthead
 \continuedcaption\\
 \hline
 \lhdr{Enumerator}
 & \rhdr{Meaning}
 \\ \capsep
 \endhead
 \tcode{none}
 & If the point to be sampled is outside of the bounds of the source brush, the visual data that is returned shall be the equivalent of \tcode{rgba_color::transparent_black()}.
 \\
 \tcode{repeat}
 & If the point to be sampled is outside of the bounds of the source brush, the visual data that is returned shall be the visual data that would have been returned if the source brush was infinitely large and repeated itself in 
 a left-to-right-left-to-right and top-to-bottom-top-to-bottom fashion.
 \\
 \tcode{reflect}
 & If the point to be sampled is outside of the bounds of the source brush, the visual data that is returned shall be the visual data that would have been returned if the source brush was infinitely large and repeated itself in 
 a left-to-right-to-left-to-right and top-to-bottom-to-top-to-bottom fashion.
 \\
 \tcode{pad}
 & If the point to be sampled is outside of the bounds of the source brush, the visual data that is returned shall be the visual data that would have been returned for the nearest defined point that is in bounds.
 \\
\end{libreqtab2}

%!TEX root = io2d.tex
\rSec0 [filter] {Enum class \tcode{filter}}

\rSec1 [filter.summary] {\tcode{filter} Summary}

\pnum
The \tcode{filter} enum class specifies the type of filter to use when sampling from a \pixmap.

\pnum
Three of the \tcode{filter} enumerators, \tcode{filter::fast}, \tcode{filter::good}, and \tcode{filter::best}, specify desired characteristics of the filter, leaving the choice of a specific filter to the implementation. 

The other two, \tcode{filter::nearest} and \tcode{filter::bilinear}, each specify a particular filter that shall be used.

\pnum
The result of sampling from a \tcode{brush} object \tcode{b} constructed from a \tcode{solid_color_brush_factory} is the same regardless of what filter is used and, as such, in these circumstances implementations should disregard the filter specified by the result of calling \tcode{b.filter()} when sampling from \tcode{b} and instead use an \unspecnorm filter, even if that filter does not correspond to a filter specified by one of the \tcode{filter} enumerators.

\pnum
See Table~\ref{tab:filter.meanings} for the meaning of each
\tcode{filter} enumerator.

\rSec1 [filter.synopsis] {\tcode{filter} Synopsis}

\begin{codeblock}
namespace std { namespace experimental { namespace io2d { inline namespace v1 {
  enum class filter {
    fast,
    good,
    best,
    nearest,
    bilinear
  };
} } } }
\end{codeblock}

\rSec1 [filter.enumerators] {\tcode{filter} Enumerators}
\begin{libreqtab2}
 {\tcode{filter} enumerator meanings}
 {tab:filter.meanings}
 \\ \topline
 \lhdr{Enumerator}
 & \rhdr{Meaning}
 \\ \capsep
 \endfirsthead
 \continuedcaption\\
 \hline
 \lhdr{Enumerator}
 & \rhdr{Meaning}
 \\ \capsep
 \endhead
 \tcode{fast}
 & The filter that corresponds to this value is \impldef{filter!fast}. The implementation shall ensure that the time complexity of the chosen filter is not greater than the time complexity of the filter that corresponds to \tcode{filter::good}.
 \enternote
 By choosing this value, the user is hinting that performance is more important than quality.
 \exitnote
 \\
 \tcode{good}
 & The filter that corresponds to this value is \impldef{filter!good}. The implementation shall ensure that the time complexity of the chosen formula is not greater than the time complexity of the formula for \tcode{filter::best}.
 \enternote
 By choosing this value, the user is hinting that quality and performance are equally important.
 \exitnote
 \\
 \tcode{best}
 & The filter that corresponds to this value is \impldef{filter!best}.
 \enternote
 By choosing this value, the user is hinting that quality is more important 
 than performance.
 \exitnote
 \\
 \tcode{nearest}
 & Nearest-neighbor interpolation filtering shall be used.
% The color of the pixel whose coordinates are nearest to the requested coordinates shall be produced. When two or more pixels are equally near to the requested coordinates, it is unspecified which of the equally near pixels shall be the pixel whose color shall be produced.
 \\
 \tcode{bilinear}
 & Bilinear interpolation filtering shall be used.
% The distance-weighted average of the four nearest pixels is used to create an interpolated color for the destination pixel. If some source pixel values do not exist (e.g. because an edge or corner of the source has been reached) then the current \tcode{extend} should be considered in order to determine the values for the missing source pixels. If a hardware sampler is available and offers bilinear filtering, implementations may use it even if its results do not conform to those produced by the description of bilinear filtering provided by this standard.% Whether mipmapping is used and whether mipmaps are generated for sources that do not have them is \impldef{filter!bilinear mipmapping}.
 \\
\end{libreqtab2}

%!TEX root = io2d.tex
\rSec0 [\iotwod.brushtype] {Enum class \tcode{brush_type}}

\rSec1 [\iotwod.brushtype.summary] {\tcode{brush_type} Summary}

\pnum
The \tcode{brush_type} enum class denotes the type of a \tcode{brush} object.

\pnum
See Table~\ref{tab:\iotwod.brushtype.meanings} for the meaning of each
\tcode{brush_type} enumerator.

\rSec1 [\iotwod.brushtype.synopsis] {\tcode{brush_type} Synopsis}

\begin{codeblock}
namespace std { namespace experimental { namespace io2d { inline namespace v1 {
  enum class brush_type {
    solid_color,
    surface,
    linear,
    radial
  };
} } } }
\end{codeblock}

\rSec1 [\iotwod.brushtype.enumerators] {\tcode{brush_type} Enumerators}
\begin{libreqtab2}
 {\tcode{brush_type} enumerator meanings}
 {tab:\iotwod.brushtype.meanings}
 \\ \topline
 \lhdr{Enumerator}
 & \rhdr{Meaning}
 \\ \capsep
 \endfirsthead
 \continuedcaption\\
 \hline
 \lhdr{Enumerator}
 & \rhdr{Meaning}
 \\ \capsep
 \endhead
 \tcode{solid_color}
 & The \tcode{brush} object is a solid color brush.
 \\
 \tcode{surface}
 & The \tcode{brush} object is a surface brush.
 \\
 \tcode{linear}
 & The \tcode{brush} object is a linear gradient brush.
 \\
 \tcode{radial}
 & The \tcode{brush} object is a radial gradient brush.
 \\
\end{libreqtab2}

%!TEX root = io2d.tex
\rSec0 [colorstops] {Color stops}

\pnum
A \tcode{color_stop_group} is a collection of zero or more \tcode{color_stop} objects which determine the values obtained by sampling a gradient (\ref{gradients}) \tcode{brush}.

\rSec1 [colorstops.colorstop]{Class \tcode{color_stop}}

\pnum
\indexlibrary{\idxcode{color_stop}}
The class \tcode{color_stop} describes a color stop that is used by gradient brushes.

\pnum
It has an offset of type \tcode{double} and a color of type \tcode{rgba_color}.

\rSec2 [colorstops.colorstop.synopsis] {\tcode{color_stop} Synopsis}

\begin{codeblock}
namespace std { namespace experimental { namespace io2d { inline namespace v1 {
  class color_stop {
  public:
  	// \ref{colorstops.colorstop.cons}, construct:
    constexpr color_stop(double o, const rgba_color& c);
    
    // \ref{colorstops.colorstop.modifiers}, modifiers:
    void offset(double val) noexcept;
	void color(const rgba_color& val) noexcept;
	
    // \ref{colorstops.colorstop.observers}, observers:
	double offset() const noexcept;
	rgba_color color() const noexcept;
  };
} } } }
\end{codeblock}

\rSec2 [colorstops.colorstop.cons]{\tcode{color_stop} constructors}

\indexlibrary{\idxcode{color_stop}!constructor}
\begin{itemdecl}
	constexpr color_stop(double o, const rgba_color& c) noexcept;
\end{itemdecl}
\begin{itemdescr}
	\pnum
	\effects
	Constructs a \tcode{color_stop} object.
	
	\pnum
	The offset shall be set to the value of \tcode{o}.
	
	\pnum
	The color shall be set to the value of \tcode{c}.
\end{itemdescr}

\rSec2 [colorstops.colorstop.modifiers]{\tcode{color_stop} modifiers}

\indexlibrary{\idxcode{color_stop}!\idxcode{offset}}
\indexlibrary{\idxcode{offset}!\idxcode{color_stop}}
\begin{itemdecl}
	void offset(double val) noexcept;
\end{itemdecl}
\begin{itemdescr}
	\pnum
	\effects
	The offset shall be set to the value of \tcode{val}.
\end{itemdescr}

\indexlibrary{\idxcode{color_stop}!\idxcode{color}}
\indexlibrary{\idxcode{color}!\idxcode{color_stop}}
\begin{itemdecl}
	void color(double val) noexcept;
\end{itemdecl}
\begin{itemdescr}
	\pnum
	\effects
	The color shall be set to the value of \tcode{val}.
\end{itemdescr}

\rSec2 [colorstops.colorstop.observers]{\tcode{color_stop} observers}

\indexlibrary{\idxcode{color_stop}!\idxcode{offset}}
\indexlibrary{\idxcode{offset}!\idxcode{color_stop}}
\begin{itemdecl}
	double offset() const noexcept;
\end{itemdecl}
\begin{itemdescr}
	\pnum
	\returns
	The value of the offset.
\end{itemdescr}

\indexlibrary{\idxcode{color_stop}!\idxcode{color}}
\indexlibrary{\idxcode{color}!\idxcode{color_stop}}
\begin{itemdecl}
	rgba_color color() const noexcept;
\end{itemdecl}
\begin{itemdescr}
	\pnum
	\returns
	The value of the color.
\end{itemdescr}

\rSec1 [colorstops.colorstopgroup]{Class \tcode{color_stop_group}}

\rSec2 [colorstops.colorstopgroup.synopsis] {\tcode{color_stop_group} Synopsis}

\begin{codeblock}
namespace std { namespace experimental { namespace io2d { inline namespace v1 {
  template <Allocator = allocator<color_stop>>
  class color_stop_group {
  public:
    using value_type      = color_stop;
    using allocator_type  = Allocator;
    using pointer = typename allocator_traits<Allocator>::pointer;
    using const_pointer = typename allocator_traits<Allocator>::const_pointer;
    using reference = value_type&;
    using const_reference = const value_type&;
    using size_type       = @\impdefx{type of \tcode{color_stop_group::size_type}}@. // See [container.requirements] in \cppseventeen.
    using difference_type = @\impdefx{type of \tcode{color_stop_group::size_type}}@. // See [container.requirements] in \cppseventeen.
    using iterator        = @\impdefx{type of \tcode{color_stop_group::iterator}}@. // See [container.requirements] in \cppseventeen.
    using const_iterator  = @\impdefx{type of \tcode{color_stop_group::const_iterator}}@. // See [container.requirements] in \cppseventeen.
    using reverse_iterator       = std::reverse_iterator<iterator>;
    using const_reverse_iterator = std::reverse_iterator<const_iterator>;

    // \ref{colorstops.colorstopgroup.cons}, constructors:    
    color_stop_group() noexcept(noexcept(Allocator())) :
    color_stop_group(Allocator()) { }
    explicit color_stop_group(const Allocator&) noexcept;
    explicit color_stop_group(size_type n, const Allocator& = Allocator());
    color_stop_group(size_type n, const value_type& value,
      const Allocator& = Allocator());
    template <class InputIterator>
    color_stop_group(InputIterator first, InputIterator last,
    const Allocator& = Allocator());
    color_stop_group(const color_stop_group& x);
    color_stop_group(color_stop_group&&) noexcept;
    color_stop_group(const color_stop_group&, const Allocator&);
    color_stop_group(color_stop_group&&, const Allocator&);
    color_stop_group(initializer_list<value_type>,
      const Allocator& = Allocator());
    ~color_stop_group();
    color_stop_group& operator=(const color_stop_group& x);
    color_stop_group& operator=(color_stop_group&& x)
      noexcept(
      allocator_traits<Allocator>::propagate_on_container_move_assignment::value
      || allocator_traits<Allocator>::is_always_equal::value);
    color_stop_group& operator=(initializer_list<value_type>);
    template <class InputIterator>
    void assign(InputIterator first, InputIterator last);
    void assign(size_type n, const value_type& u);
    void assign(initializer_list<value_type>);
    allocator_type get_allocator() const noexcept;
    
    // \ref{colorstops.colorstopgroup.iterators}, iterators:
    iterator begin() noexcept;
    const_iterator begin() const noexcept;
    const_iterator cbegin() const noexcept;
    
    iterator end() noexcept;
    const_iterator end() const noexcept;
    const_iterator cend() const noexcept;
    
    reverse_iterator rbegin() noexcept;
    const_reverse_iterator rbegin() const noexcept;
    const_reverse_iterator crbegin() const noexcept;
    
    reverse_iterator rend() noexcept;
    const_reverse_iterator rend() const noexcept;
    const_reverse_iterator crend() const noexcept;
    
    // \ref{colorstops.colorstopgroup.capacity}, capacity
    bool empty() const noexcept;
    size_type size() const noexcept;
    size_type max_size() const noexcept;
    size_type capacity() const noexcept;
    void resize(size_type sz);
    void resize(size_type sz, const value_type& c);
    void reserve(size_type n);
    void shrink_to_fit();
    
    // element access:
    reference operator[](size_type n);
    const_reference operator[](size_type n) const;
    const_reference at(size_type n) const;
    reference at(size_type n);
    reference front();
    const_reference front() const;
    reference back();
    const_reference back() const;
    
    // \ref{colorstops.colorstopgroup.modifiers}, modifiers:
    template <class... Args>
    reference emplace_back(Args&&... args);
    void push_back(const value_type& x);
    void push_back(value_type&& x);
    void pop_back();
    template <class... Args>
    iterator emplace(const_iterator position, Args&&... args);
    iterator insert(const_iterator position, const value_type& x);
    iterator insert(const_iterator position, value_type&& x);
    iterator insert(const_iterator position, size_type n, const value_type& x);
    template <class InputIterator>
    iterator insert(const_iterator position, InputIterator first,
      InputIterator last);
    iterator insert(const_iterator position,
    initializer_list<value_type> il);
    iterator erase(const_iterator position);
    iterator erase(const_iterator first, const_iterator last);
    void swap(color_stop_group&)
      noexcept(allocator_traits<Allocator>::propagate_on_container_swap::value 
      || allocator_traits<Allocator>::is_always_equal::value);
    void clear() noexcept;
  };

  // \ref{colorstops.colorstopgroup.special}, specialized algorithms:
  template <Allocator>
  void swap(color_stop_group<Allocator>& lhs, color_stop_group<Allocator>& rhs)
    noexcept(noexcept(lhs.swap(rhs)));
} } } }
\end{codeblock}

\rSec1 [colorstops.colorstopgroup.containerrequirements] {\tcode{color_stop_group} container requirements}

\pnum
This class shall be considered a sequence container, as defined in [containers] in \cppseventeen, and all sequence container requirements that apply specifically to \tcode{vector} shall also apply to this class.

\rSec1 [colorstops.colorstopgroup.cons] {\tcode{color_stop_group} constructors, copy, and assignment}

\indexlibrary{\idxcode{color_stop_group}!constructor}
\begin{itemdecl}
	explicit color_stop_group(const Allocator&);
\end{itemdecl}
\begin{itemdescr}
	\pnum
	\effects
	Constructs an empty \tcode{color_stop_group}, using the specified allocator.
	
	\pnum
	\complexity
	Constant.
\end{itemdescr}

\indexlibrary{\idxcode{color_stop_group}!constructor}
\begin{itemdecl}
	explicit color_stop_group(size_type n, const Allocator& = Allocator());
\end{itemdecl}
\begin{itemdescr}
	\pnum
	\effects
	Constructs a \tcode{color_stop_group} with \tcode{n} default-inserted elements using the specified allocator.
	
	\pnum
	\complexity
	Linear in \tcode{n}.
\end{itemdescr}

\indexlibrary{\idxcode{color_stop_group}!constructor}
\begin{itemdecl}
	color_stop_group(size_type n, const value_type& value,
	  const Allocator& = Allocator());
\end{itemdecl}
\begin{itemdescr}
	\pnum
	\requires
	\tcode{value_type} shall be \tcode{CopyInsertable} into \tcode{*this}.
	
	\pnum
	\effects
	Constructs a \tcode{color_stop_group} with n copies of \tcode{value}, using the specified allocator.
	
	\pnum
	\complexity
	Linear in \tcode{n}.
\end{itemdescr}

\indexlibrary{\idxcode{color_stop_group}!constructor}
\begin{itemdecl}
	template <class InputIterator>
	color_stop_group(InputIterator first, InputIterator last,
	  const Allocator& = Allocator());
\end{itemdecl}
\begin{itemdescr}
	\pnum
	\effects
	Constructs a \tcode{color_stop_group} equal to the range \range{first}{last}, using the specified allocator.
	
	\pnum
	\complexity
	Makes only $N$ calls to the copy constructor of \tcode{value_type} (where $N$
	is the distance between
	\tcode{first}
	and
	\tcode{last})
	and no reallocations if iterators \tcode{first} and \tcode{last} are of forward, bidirectional, or random access categories.
	It makes order
	\tcode{N}
	calls to the copy constructor of
	\tcode{value_type}
	and order
	$\log(N)$
	reallocations if they are just input iterators.
	
\end{itemdescr}

\rSec1 [colorstops.colorstopgroup.capacity] {\tcode{color_stop_group} capacity}

\indexlibrary{\idxcode{color_stop_group}!\idxcode{capacity}}
\indexlibrary{\idxcode{capacity}!\idxcode{color_stop_group}}
\begin{itemdecl}
	size_type capacity() const noexcept;
\end{itemdecl}
\begin{itemdescr}
	\pnum
	\returns
	The total number of elements that the color stop group can hold without requiring reallocation.
\end{itemdescr}

\indexlibrary{\idxcode{color_stop_group}!\idxcode{reserve}}
\indexlibrary{\idxcode{reserve}!\idxcode{color_stop_group}}
\begin{itemdecl}
	void reserve(size_type n);
\end{itemdecl}
\begin{itemdescr}
	\pnum
	\requires
	\tcode{value_type} shall be \tcode{MoveInsertable} into \tcode{*this}.
	
	\pnum
	\effects
	A directive that informs a color stop group of a planned change in size, so that it can manage the storage
	allocation accordingly. After \tcode{reserve()}, \tcode{capacity()} is greater or equal to the argument of \tcode{reserve} if
	reallocation happens; and equal to the previous value of \tcode{capacity()} otherwise. Reallocation happens
	at this point if and only if the current capacity is less than the argument of \tcode{reserve()}. If an exception
	is thrown other than by the move constructor of a non-\tcode{CopyInsertable} type, there are no effects.
	
	\pnum
	\complexity
	It does not change the size of the sequence and takes at most linear time in the size of the
	sequence.
	
	\pnum
	\throws
	\tcode{length_error} if \tcode{n >
		max_size()}.\footnote{\tcode{reserve()} uses \tcode{Allocator::allocate()} which
		may throw an appropriate exception.}
	
	\pnum
	\remarks
	Reallocation invalidates all the references, pointers, and iterators
	referring to the elements in the sequence.
	No reallocation shall take place during insertions that happen
	after a call to
	\tcode{reserve()}
	until the time when an insertion would make the size of the vector
	greater than the value of
	\tcode{capacity()}.
\end{itemdescr}

\indexlibrary{\idxcode{color_stop_group}!\idxcode{shrink_to_fit}}
\indexlibrary{\idxcode{shrink_to_fit}!\idxcode{color_stop_group}}
\begin{itemdecl}
	void shrink_to_fit();
\end{itemdecl}
\begin{itemdescr}
	\pnum
	\requires
	\tcode{value_type} shall be \tcode{MoveInsertable} into \tcode{*this}.
	
	\pnum
	\effects
	\tcode{shrink_to_fit} is a non-binding request to reduce
	\tcode{capacity()} to \tcode{size()}.
	\enternote
	The request is non-binding to allow latitude for
	implementation-specific optimizations.
	\exitnote
	It does not increase \tcode{capacity()}, but may reduce \tcode{capacity()}
	by causing reallocation. 
	If an exception is thrown other than by the move constructor
	of a non-\tcode{CopyInsertable} \tcode{value_type} there are no effects.
	
	\pnum
	\complexity Linear in the size of the sequence.
	
	\pnum
	\remarks Reallocation invalidates all the references, pointers, and 
	iterators referring to the elements in the sequence. If no reallocation 
	happens, they remain valid.
\end{itemdescr}

\indexlibrary{\idxcode{color_stop_group}!\idxcode{swap}}
\indexlibrary{\idxcode{swap}!\idxcode{color_stop_group}}
\begin{itemdecl}
	void swap(color_stop_group&)
	noexcept(allocator_traits<Allocator>::propagate_on_container_swap::value ||
	allocator_traits<Allocator>::is_always_equal::value);
\end{itemdecl}
\begin{itemdescr}
	\pnum
	\effects
	Exchanges the contents and
	\tcode{capacity()}
	of
	\tcode{*this}
	with that of \tcode{x}.
	
	\pnum
	\complexity
	Constant time.
\end{itemdescr}

\indexlibrary{\idxcode{color_stop_group}!\idxcode{resize}}
\indexlibrary{\idxcode{resize}!\idxcode{color_stop_group}}
\begin{itemdecl}
	void resize(size_type sz);
\end{itemdecl}
\begin{itemdescr}
	\pnum
	\effects
	If \tcode{sz < size()}, erases the last \tcode{size() - sz} elements
	from the sequence. Otherwise, appends \tcode{sz - size()} default-inserted 
	elements to the sequence.
	
	\pnum
	\requires
	\tcode{value_type} shall be
	\tcode{MoveInsertable} and \tcode{DefaultInsertable} into \tcode{*this}.
	
	\pnum
	\remarks
	If an exception is thrown other than by the move constructor of a 
	non-\tcode{CopyInsertable}
	\tcode{value_type} there are no effects.
\end{itemdescr}

\indexlibrary{\idxcode{color_stop_group}!\idxcode{resize}}
\indexlibrary{\idxcode{resize}!\idxcode{color_stop_group}}
\begin{itemdecl}
	void resize(size_type sz, const value_type& c);
\end{itemdecl}
\begin{itemdescr}
	\pnum
	\effects
	If \tcode{sz < size()}, erases the last \tcode{size() - sz} elements
	from the sequence. Otherwise,
	appends \tcode{sz - size()} copies of \tcode{c} to the sequence.
	
	\pnum
	\requires
	\tcode{value_type} shall be \tcode{CopyInsertable} into \tcode{*this}.
	
	\pnum
	\remarks
	If an exception is thrown there are no effects.
\end{itemdescr}

\rSec1 [colorstops.colorstopgroup.modifiers] {\tcode{color_stop_group} modifiers}

\indexlibrary{\idxcode{color_stop_group}!\idxcode{insert}}
\indexlibrary{\idxcode{insert}!\idxcode{color_stop_group}}
\indexlibrary{\idxcode{color_stop_group}!\idxcode{emplace_back}}
\indexlibrary{\idxcode{emplace_back}!\idxcode{color_stop_group}}
\indexlibrary{\idxcode{color_stop_group}!\idxcode{push_back}}
\indexlibrary{\idxcode{push_back}!\idxcode{color_stop_group}}
\begin{itemdecl}
	iterator insert(const_iterator position, const value_type& x);
	iterator insert(const_iterator position, value_type&& x);
	iterator insert(const_iterator position, size_type n, const value_type& x);
	template <class InputIterator>
	iterator insert(const_iterator position, InputIterator first,
	InputIterator last);
	iterator insert(const_iterator position, initializer_list<value_type>);
	template <class... Args>
	reference emplace_back(Args&&... args);
	template <class... Args>
	iterator emplace(const_iterator position, Args&&... args);
	void push_back(const value_type& x);
	void push_back(value_type&& x);
\end{itemdecl}

\begin{itemdescr}
	\pnum
	\remarks
	Causes reallocation if the new size is greater than the old capacity.
	Reallocation invalidates all the references, pointers, and iterators
	referring to the elements in the sequence.
	If no reallocation happens, all the iterators and references before the insertion point remain valid.
	If an exception is thrown other than by
	the copy constructor, move constructor,
	assignment operator, or move assignment operator of
	\tcode{value_type} or by any \tcode{InputIterator} operation
	there are no effects.
	If an exception is thrown while inserting a single element at the end and
	\tcode{value_type} is \tcode{CopyInsertable} or \tcode{is_nothrow_move_constructible_v<value_type>}
	is \tcode{true}, there are no effects.
	Otherwise, if an exception is thrown by the move constructor of a non-\tcode{CopyInsertable}
	\tcode{value_type}, the effects are unspecified.
	
	\pnum
	\complexity
	The complexity is linear in the number of elements inserted plus the 
	distance to the end of the color stop group.
\end{itemdescr}

\indexlibrary{\idxcode{color_stop_group}!\idxcode{erase}}
\indexlibrary{\idxcode{erase}!\idxcode{color_stop_group}}
\indexlibrary{\idxcode{color_stop_group}!\idxcode{pop_back}}
\indexlibrary{\idxcode{pop_back}!\idxcode{color_stop_group}}
\begin{itemdecl}
	iterator erase(const_iterator position);
	iterator erase(const_iterator first, const_iterator last);
	void pop_back();
\end{itemdecl}

\begin{itemdescr}
	\pnum
	\effects
	Invalidates iterators and references at or after the point of the erase.
	
	\pnum
	\complexity
	The destructor of \tcode{value_type} is called the number of times equal to 
	the number of the elements erased, but the assignment operator
	of \tcode{value_type} is called the number of times equal to the number of
	elements in the vector after the erased elements.
	
	\pnum
	\throws
	Nothing unless an exception is thrown by the copy constructor, move 
	constructor, assignment operator, or move assignment operator of
	\tcode{value_type}.
\end{itemdescr}

\rSec1 [colorstops.colorstopgroup.iterators] {\tcode{color_stop_group} iterators}

\indexlibrary{\idxcode{color_stop_group}!\idxcode{begin}}
\indexlibrary{\idxcode{begin}!\idxcode{color_stop_group}}
\indexlibrary{\idxcode{color_stop_group}!\idxcode{cbegin}}
\indexlibrary{\idxcode{cbegin}!\idxcode{color_stop_group}}
\begin{itemdecl}
	iterator begin() noexcept;
	const_iterator begin() const noexcept;
	const_iterator cbegin() const noexcept;
\end{itemdecl}
\begin{itemdescr}
	\pnum
	\returns
	An iterator referring to the first \tcode{path_data::path_data_types} item in the path group.
	
	\pnum
	\remarks
	Changing a \tcode{path_data::path_data_types} object or otherwise modifying the path group in a way that violates the preconditions of that \tcode{path_data::path_data_types} object or of any subsequent \tcode{path_data::path_data_types} object in the path group shall result in undefined behavior when the path group is processed as described in \ref{paths.processing} unless all of the violations are fixed prior to such processing.
\end{itemdescr}

\indexlibrary{\idxcode{color_stop_group}!\idxcode{end}}
\indexlibrary{\idxcode{end}!\idxcode{color_stop_group}}
\indexlibrary{\idxcode{color_stop_group}!\idxcode{cend}}
\indexlibrary{\idxcode{cend}!\idxcode{color_stop_group}}
\begin{itemdecl}
	iterator end() noexcept;
	const_iterator end() const noexcept;
	const_iterator cend() const noexcept;
\end{itemdecl}
\begin{itemdescr}
	\pnum
	\returns
	An iterator which is the past-the-end value.
\end{itemdescr}

\indexlibrary{\idxcode{color_stop_group}!\idxcode{rbegin}}
\indexlibrary{\idxcode{rbegin}!\idxcode{color_stop_group}}
\indexlibrary{\idxcode{color_stop_group}!\idxcode{crbegin}}
\indexlibrary{\idxcode{crbegin}!\idxcode{color_stop_group}}
\begin{itemdecl}
	reverse_iterator rbegin() noexcept;
	const_reverse_iterator rbegin() const noexcept;
	const_reverse_iterator crbegin() const noexcept;
\end{itemdecl}
\begin{itemdescr}
	\pnum
	\returns
	An iterator which is semantically equivalent to \tcode{reverse_iterator(end)}.
\end{itemdescr}

\indexlibrary{\idxcode{color_stop_group}!\idxcode{rend}}
\indexlibrary{\idxcode{rend}!\idxcode{color_stop_group}}
\indexlibrary{\idxcode{color_stop_group}!\idxcode{crend}}
\indexlibrary{\idxcode{crend}!\idxcode{color_stop_group}}
\begin{itemdecl}
	reverse_iterator rend() noexcept;
	const_reverse_iterator rend() const noexcept;
	const_reverse_iterator crend() const noexcept;
\end{itemdecl}
\begin{itemdescr}
	\pnum
	\returns
	An iterator which is semantically equivalent to \tcode{reverse_iterator(begin)}.
\end{itemdescr}

\rSec1[colorstops.colorstopgroup.special] {\tcode{color_stop_group} specialized algorithms}

\indexlibrary{color_stop_group}{swap}
\indexlibrary{swap}{color_stop_group}
\begin{itemdecl}
	template <class Allocator>
	void swap(color_stop_group<Allocator>& lhs, 
	  color_stop_group<Allocator>& rhs) noexcept(noexcept(lhs.swap(rhs)));
\end{itemdecl}
\begin{itemdescr}
	\pnum
	\effects
	As if by \tcode{lhs.swap(rhs)}.
\end{itemdescr}

%%!TEX root = io2d.tex
\rSec0 [\iotwod.meshpatch] {Class \tcode{mesh_patch}}

\rSec1 [\iotwod.meshpatch.intro] {Overview}

\pnum
\indexlibrary{\idxcode{mesh_patch}}%
The \tcode{mesh_patch} class describes one of several types of patches used in creating \tcode{brush} object's with a brush type of \tcode{brush_type::mesh}. 

\rSec1 [\iotwod.meshpatch.bicubic] {Bicubic tensor-product patch}

\pnum
The most general of the four types of patches is the bicubic tensor-product patch. It is used in creating \tcode{brush} objects of type \tcode{brush_type::mesh}.

\pnum
The patch is described by 16 points and four colors.

\pnum
For purposes of interpreting the formula that describes the patch, the points are arranged as follows: \\
$
~[~[~P03~P13~P23~P33~]~] \\
~[~[~P02~P12~P22~P32~]~] \\
~[~[~P01~P11~P12~P31~]~] \\
~[~[~P00~P10~P20~P30~]~]$

\pnum
The colors are arranged as follows: \\
$
~[~[~C1~C2~]~] \\
~[~[~C0~C3~]~]$

\pnum
The formula that describes the patch is: \\
$
\displaystyle q(u,v) = \sum_{i=0}^{3}\sum_{j=0}^{3}k_{ij}\times{}B_{i}(u)\times{}B_{j}(v)
$ \\
\\
with $k_{ij}$ being the point P\textit{i}\textit{j} above and with $B_{i}(u)$ and $B_{j}(v)$ being Bernstein basis polynomials.

\pnum
\begin{note}
For example, where \textit{i} is 0 and \textit{j} is 2, the point is P02.
\end{note}

\pnum
The inputs $u$ and $v$ are each a value in the range \orange{0}{1}, with $u$ and $v$ representing horizontal and vertical variation, respectively, within a unit square. Unlike in the standard coordinate space, vertical variation here is oriented from bottom to top.

\pnum
When sampling from the patch, the point that results from evaluating the patch at point $(u,v)$ has the color $Cr$, which is determined as follows: \\
$
Cr=(C0\times{}((1 - u)\times{}(1-v)))+(C1\times{}((1-u)\times{}v))+(C2\times{}(u\times{}v))+(C3\times{}(u\times{}(1-v)))$

\rSec1 [\iotwod.meshpatch.forms] {Other forms supported by \tcode{mesh_patch}}

\pnum
The \tcode{mesh_patch} class supports four types of patches. In addition to the 

\rSec1 [\iotwod.meshpatch.synopsis] {\tcode{mesh_patch} synopsis}

\begin{codeblock}
namespace @\fullnamespace{}@ {
  class bicubic_patch {
  public:
  	// \ref{\iotwod.bicubicpatch.cons}, construct:
    constexpr color_stop(float o, const rgba_color& c);
    
    // \ref{\iotwod.bicubicpatch.modifiers}, modifiers:
    constexpr void offset(float val) noexcept;
	constexpr void color(const rgba_color& val) noexcept;
	
    // \ref{\iotwod.bicubicpatch.observers}, observers:
	constexpr float offset() const noexcept;
	constexpr rgba_color color() const noexcept;
  };
}
\end{codeblock}

\rSec1 [\iotwod.bicubicpatch.cons]{\tcode{bicubic_patch} constructors}

\indexlibrary{\idxcode{bicubic_patch}!constructor}%
\begin{itemdecl}
constexpr bicubic_patch(float o, const rgba_color& c) noexcept;
\end{itemdecl}
\begin{itemdescr}
\pnum
\effects
Constructs a \tcode{color_stop} object.

\pnum
The offset shall be set to the value of \tcode{o}.

\pnum
The color shall be set to the value of \tcode{c}.
\end{itemdescr}

\rSec1 [\iotwod.bicubicpatch.modifiers]{\tcode{bicubic_patch} modifiers}

\indexlibrarymember{offset}{bicubic_patch}%
\begin{itemdecl}
constexpr void offset(float val) noexcept;
\end{itemdecl}
\begin{itemdescr}
\pnum
\effects
The offset shall be set to the value of \tcode{val}.
\end{itemdescr}

\rSec1 [\iotwod.bicubicpatch.observers]{\tcode{bicubic_patch} observers}

\indexlibrarymember{offset}{bicubic_patch}%
\begin{itemdecl}
constexpr float offset() const noexcept;
\end{itemdecl}
\begin{itemdescr}
\pnum
\returns
The value of the offset.
\end{itemdescr}

%!TEX root = io2d.tex
\rSec0 [\iotwod.brush] {Class \tcode{brush}}

\rSec1 [\iotwod.brush.intro] {\tcode{brush} summary}

\pnum
\indexlibrary{\idxcode{brush}}%
The class \tcode{brush} describes an opaque wrapper for graphics data.

\pnum
A \tcode{brush} object is usable with any \tcode{surface} or \tcode{surface}-derived object.

\pnum
A \tcode{brush} object's graphics data is immutable. It is observable only by the effect that it produces when the brush is used as a \term{
source brush} or as a \term{mask brush} (\ref{\iotwod.surface.rendering.brushes}).

\pnum
A \tcode{brush} object has a brush type of \tcode{brush_type}, which indicates which type of brush it is (Table~\ref{tab:\iotwod.brushtype.meanings}).

\pnum
As a result of technological limitations and considerations, a \tcode{brush} object's graphics data may have less precision than the data from which it was created.

%\pnum
%\begin{example}
%Several graphics and rendering technologies that are currently widely used typically store individual color and alpha channel data as 8-bit unsigned normalized integer values while the \tcode{float} type that is used by the \tcode{rgba_color} class for individual color and alpha is often a 64-bit value. As such, it is possible for a loss of precision when transforming the 64-bit channel data of an \tcode{rgba_color} object to the 8-bit channel data that is commonly used internally in such graphics and rendering technologies.
%\end{example}
%
\rSec1 [\iotwod.brush.synopsis] {\tcode{brush} synopsis}

\begin{codeblock}
namespace std::experimental::io2d::v1 {
  class brush {
  public:
    // \ref{\iotwod.brush.cons}, construct/copy/move/destroy:
    explicit brush(rgba_color c);
    template <class InputIterator>
    brush(point_2d begin, point_2d end,
      InputIterator first, InputIterator last);
    brush(point_2d begin, point_2d end,
      initializer_list<gradient_stop> il);
    template <class InputIterator>
    brush(const circle& start, const circle& end,
      InputIterator first, InputIterator last);
    brush(const circle& start, const circle& end,
      initializer_list<gradient_stop> il);
    explicit brush(image_surface&& img);

    // \ref{\iotwod.brush.observers}, observers:
    brush_type type() const noexcept;
  };
}
\end{codeblock}

\rSec1 [\iotwod.brush.sampling] {Sampling from a \tcode{brush} object}

\pnum
A \tcode{brush} object is sampled from either as a source brush (\ref{\iotwod.surface.rendering.brushes}) or a mask brush (\ref{\iotwod.surface.rendering.brushes}).

\pnum
If it is being sampled from as a source brush, its \term{wrap mode}, \term{filter}, and \term{brush matrix} are defined by a \tcode{brush_props} object (\ref{\iotwod.surface.rendering.commonstate} and \ref{\iotwod.surface.rendering.statedefaults}).

\pnum
If it is being sampled from as a mask brush, its wrap mode, filter, and \term{mask matrix} are defined by a \tcode{mask_props} object (\ref{\iotwod.surface.rendering.specificstate} and \ref{\iotwod.surface.rendering.statedefaults}).

\pnum
When sampling from a \tcode{brush} object \tcode{b}, the \tcode{brush_type} returned by calling \tcode{b.type()} determines how the results of sampling are determined:
\begin{enumerate}
\item If the result of \tcode{b.type()} is \tcode{brush_type::solid_color} then \tcode{b} is a \term{solid color brush}.
\item If the result of \tcode{b.type()} is \tcode{brush_type::surface} then \tcode{b} is a \term{surface brush}.
\item If the result of \tcode{b.type()} is \tcode{brush_type::linear} then \tcode{b} is a \term{linear gradient brush}.
\item If the result of \tcode{b.type()} is \tcode{brush_type::radial} then \tcode{b} is a \term{radial gradient brush}.
\end{enumerate}

\rSec2 [\iotwod.brush.sampling.color] {Sampling from a solid color brush}

\pnum
When \tcode{b} is a solid color brush, then when sampling from \tcode{b}, the visual data returned is always the visual data used to construct \tcode{b}, regardless of the point which is to be sampled and regardless of the return values of wrap mode, filter, and brush matrix or mask matrix.

\rSec2 [\iotwod.brush.sampling.linear] {Sampling from a linear gradient brush}

\pnum
When \tcode{b} is a linear gradient brush, when sampling point \tcode{pt}, where \tcode{pt} is the return value of calling the \tcode{transform_pt} member function of brush matrix or mask matrix using the requested point, from \tcode{b}, the visual data returned are as specified by \ref{\iotwod.gradients.linear} and \ref{\iotwod.gradients.sampling}.

\rSec2 [\iotwod.brush.sampling.radial] {Sampling from a radial gradient brush}

\pnum
When \tcode{b} is a radial gradient brush, when sampling point \tcode{pt}, where \tcode{pt} is the return value of calling the \tcode{transform_pt} member function of brush matrix or mask matrix using the requested point, from \tcode{b}, the visual data are as specified by \ref{\iotwod.gradients.radial} and \ref{\iotwod.gradients.sampling}.

\rSec2 [\iotwod.brush.sampling.surface] {Sampling from a surface brush}

\pnum
When \tcode{b} is a surface brush, when sampling point \tcode{pt} from \tcode{b}, where \tcode{pt} is the return value of calling the \tcode{transform_pt} member function of the brush matrix or mask matrix using the requested point, the visual data returned are from the point \tcode{pt} in the graphics data of the brush, as modified by the values of wrap mode (\ref{\iotwod.wrapmode}) and filter (\ref{\iotwod.filter}).

\rSec1 [\iotwod.brush.cons] {\tcode{brush} constructors and assignment operators}

\indexlibrary{\idxcode{brush}!constructor}%
\begin{itemdecl}
explicit brush(rgba_color c);
\end{itemdecl}
\begin{itemdescr}
\pnum
\effects
Constructs an object of type \tcode{brush}.

\pnum
The brush's brush type shall be set to the value \tcode{brush_type::solid_color}.

\pnum
The graphics data of the brush are created from the value of \tcode{c}. The visual data format of the graphics data are as-if it is that specified by \tcode{format::argb32}.

\pnum
\remarks
Sampling from this produces the results specified in \ref{\iotwod.brush.sampling.color}.
\end{itemdescr}

\indexlibrary{\idxcode{brush}!constructor}%
\begin{itemdecl}
template <class InputIterator>
brush(point_2d begin, point_2d end,
  InputIterator first, InputIterator last);
\end{itemdecl}
\begin{itemdescr}
\pnum
\effects
Constructs a linear gradient \tcode{brush} object with a begin point of \tcode{begin}, an end point of \tcode{end}, and a sequential series of \tcode{gradient stop} values beginning at {first} and ending at {last - 1}.

\pnum
The brush's brush type is \tcode{brush_type::linear}.

\pnum
\remarks
Sampling from this brush produces the results specified in \ref{\iotwod.brush.sampling.linear}.
\end{itemdescr}

\indexlibrary{\idxcode{brush}!constructor}%
\begin{itemdecl}
brush(point_2d begin, point_2d end,
  initializer_list<gradient_stop> il);
\end{itemdecl}
\begin{itemdescr}
\pnum
\effects
Constructs a linear gradient \tcode{brush} object with a begin point of \tcode{begin}, an end point of \tcode{end}, and the sequential series of \tcode{gradient stop} values in \tcode{il}.

\pnum
The brush's brush type is \tcode{brush_type::linear}.

\pnum
\remarks
Sampling from this brush produces the results specified in \ref{\iotwod.brush.sampling.linear}.
\end{itemdescr}

\indexlibrary{\idxcode{brush}!constructor}%
\begin{itemdecl}
template <class InputIterator>
brush(const circle& start, const circle& end,
  InputIterator first, InputIterator last);
\end{itemdecl}
\begin{itemdescr}
\pnum
\effects
Constructs a radial gradient \tcode{brush} object with a start circle of \tcode{start}, an end circle of \tcode{end},  and a sequential series of \tcode{gradient stop} values beginning at {first} and ending at {last - 1}.

\pnum
The brush's brush type is \tcode{brush_type::radial}.

\pnum
\remarks
Sampling from this brush produces the results specified in \ref{\iotwod.brush.sampling.radial}.
\end{itemdescr}

\indexlibrary{\idxcode{brush}!constructor}%
\begin{itemdecl}
brush(const circle& start, const circle& end,
  initializer_list<gradient_stop> il);
\end{itemdecl}
\begin{itemdescr}
\pnum
\effects
Constructs a radial gradient \tcode{brush} object with a start circle of \tcode{start}, an end circle of \tcode{end}, and the sequential series of \tcode{gradient_stop} values in \tcode{il}.

\pnum
The brush's brush type is \tcode{brush_type::radial}.

\pnum
\remarks
Sampling from this brush produces the results specified in \ref{\iotwod.brush.sampling.radial}.
\end{itemdescr}

\indexlibrary{\idxcode{brush}!constructor}%
\begin{itemdecl}
explicit brush(image_surface&& img);
\end{itemdecl}
\begin{itemdescr}
\pnum
\effects
Constructs an object of type \tcode{brush}.

\pnum
The brush's brush type is \tcode{brush_type::surface}.

\pnum
The graphics data of the brush is as-if it is the raster graphics data of \tcode{img}.

\pnum
\remarks
Sampling from this brush produces the results specified in \ref{\iotwod.brush.sampling.surface}.
\end{itemdescr}

\rSec1 [\iotwod.brush.observers]{\tcode{brush} observers}

\indexlibrary{\idxcode{brush}!\idxcode{type}}%
\begin{itemdecl}
brush_type type() const noexcept;
\end{itemdecl}
\begin{itemdescr}
\pnum
\returns
The brush's brush type.
\end{itemdescr}

\addtocounter{SectionDepthBase}{-1}
