%!TEX root = io2d.tex

\rSec0 [brushes] {Brushes}

\rSec1 [brushes.intro] {Overview of brushes}

\pnum
Brushes contain visual data and serve as sources of visual data for rendering and composing operations.

\pnum
There are four types of brushes:
\begin{itemize}
	\item solid color;
	\item linear gradient;
	\item radial gradient; and,
	\item surface.
\end{itemize}

\pnum
Once a brush is created, its visual data is immutable.

\pnum
\begin{note}
While copy and move operations along with a swap operation can change the visual data that a brush contains, the visual data itself is not modified.
\end{note}

\pnum
A brush is used either as a Source Brush or a Mask Brush (\ref{surface.rendering.brushes}).

\pnum
When a brush is used in a rendering and composing operation, if it is used as a Source Brush, it has a \tcode{brush_props} object that describes how the brush is interpreted for purposes of sampling. If it is used as a Mask Brush, it has a \tcode{mask_props} object that describes how the brush is interpreted for purposes of sampling.

\pnum
The \tcode{brush_props} and \tcode{mask_props} classes both have a Wrap Mode, Filter and Matrix (\ref{\iotwod.brushprops.summary} and \ref{maskprops.summary}). Where necessary, these shall be referenced using those terms without regard to whether the brush is being used as a Source Brush or a Mask Brush.


%\pnum
%A brush has its own coordinate space.
%
%\pnum
%If it is being used as a Source Brush, it is the Brush Coordinate Space (\ref{surface.coordinatespaces}).
%
%\pnum
%If it is being used as a Mask Brush, it is the Mask Coordinate Space (\ref{surface.coordinatespaces}).
%
%\pnum
%It possesses a \tcode{filter} value and a \tcode{wrap_mode} value, which combine to determine the visual data value returned when a composing operation samples a brush.
%
%\pnum
%The \tcode{filter} value determines the value returned by a brush when a composing operation samples a brush. When the visual data of the brush is a pixmap and the point the composing operation requests does not directly correspond to the exact coordinate of a pixel within the pixmap, the \tcode{filter} value is used to determine the value of the visual data that is returned. If the visual data of the brush is not a pixmap, the \tcode{filter} value is irrelevant.
%
%\pnum
%The \tcode{wrap_mode} value controls what happens when a composing operation needs to sample from a point that is outside of the bounds of the brush.
%
\pnum
Solid color brushes are unbounded and as such always produce the same visual data when sampled from, regardless of the requested point.

\pnum
Linear gradient and radial gradient brushes share similarities with each other that are not shared by the other types of brushes. This is discussed in more detail elsewhere (\ref{gradients}).

\pnum
Surface brushes are constructed from an \tcode{image_surface} object. Their visual data is a pixmap, which has implications on sampling from the brush that are not present in the other brush types.

\addtocounter{SectionDepthBase}{1}
%!TEX root = io2d.tex
\rSec0 [\iotwod.gradients] {Gradient brushes}

\rSec1 [\iotwod.gradients.common] {Common properties of gradients}
\pnum
Gradients are formed, in part, from a collection of \tcode{color_stop} objects.

\pnum
The collection of \tcode{color_stop} objects contribute to defining a brush which, when sampled from, returns a value that is interpolated based on those color stops.

\rSec1 [\iotwod.gradients.linear] {Linear gradients}

\pnum
A linear gradient is a type of gradient.

\pnum
A linear gradient has a \term{begin point} and an \term{end point}, each of which are objects of type \tcode{vector_2d}.

\pnum
A linear gradient for which the distance between its begin point and its end point is not greater than \tcode{numeric_limits<double>::epsilon()} is a \term{degenerate linear gradient}.

\pnum
All attempts to sample from a a degenerate linear gradient return the color \tcode{rgba_color::transparent_black()}. The remainder of \ref{\iotwod.gradients} is inapplicable to degenerate linear gradients.

\pnum
The begin point and end point of a linear gradient define a line segment, with a color stop offset value of 0.0 corresponding to the begin point and a color stop offset value of 1.0 corresponding to the end point.

\pnum
Color stop offset values in the range \crange{0.0}{1.0} linearly correspond to points on the line segment.

\pnum
\begin{example}
Given a linear gradient with a begin point of \tcode{vector_2d(0.0, 0.0)} and an end point of \tcode{vector_2d(10.0, 5.0)}, a color stop offset value of 0.6 would correspond to the point \tcode{vector_2d(6.0, 3.0)}.
\end{example}

\pnum
To determine the offset value of a point $p$ for a linear gradient, perform the following steps:
\begin{enumeratea}
\item Create a line at the begin point of the linear gradient, the \term{begin line}, and another line at the end point of the linear gradient, the \term{end line}, with each line being perpendicular to the \term{gradient line segment}, which is the line segment delineated by the begin point and the end point.

\item Using the begin line, $p$, and the end line, create a line, the \term{$p$ line}, which is parallel to the gradient line segment.

\item Defining $dp$ as the distance between $p$ and the point where the $p$ line intersects the begin line and $dt$ as the distance between the point where the $p$ line intersects the begin line and the point where the $p$ line intersects the end line, the offset value of $p$ is $dp \div dt$.

\item The offset value shall be negative if
\begin{itemize}
\item $p$ is not on the line segment delineated by the point where the $p$ line intersects the begin line and the point where the $p$ line intersects the end line; and,

\item the distance between $p$ and the point where the $p$ line intersects the begin line is less than the distance between $p$ and the point where the $p$ line intersects the end line.
\end{itemize}
\end{enumeratea}

\rSec1 [\iotwod.gradients.radial] {Radial gradients}

\pnum
A radial gradient is a type of gradient.

\pnum
Aa radial gradient has a \term{start circle} and an \term{end circle}, each of which is defined by a \tcode{circle} object.

\pnum
A radial gradient is a \term{degenerate radial gradient} if:
\begin{itemize}
\item its start circle has a negative radius; or,
\item its end circle has a negative radius; or,
\item the distance between the center point of its start circle and the center point of its end circle is not greater than \tcode{numeric_limits<double>::epsilon()} and the difference between the radius of its start circle and the radius of its end circle is not greater than \tcode{numeric_limits<double>::epsilon()}; or,
\item its start circle has a radius of 0.0 and its end circle has a radius of 0.0.
\end{itemize}

\pnum
All attempts to sample from a \tcode{brush} object created using a degenerate radial gradient return the color \tcode{rgba_color::transparent_black()}. The remainder of \ref{\iotwod.gradients} is inapplicable to degenerate radial gradients.

\pnum
A color stop offset of 0.0 corresponds to all points along the diameter of the start circle or to its center point if it has a radius value of 0.0.

\pnum
A color stop offset of 1.0 corresponds to all points along the diameter of the end circle or to its center point if it has a radius value of 0.0.

\pnum
A radial gradient shall be rendered as a continuous series of interpolated circles defined by the following equations:
\begin{enumeratea}
\item $x(o) = x_{start} + o \times (x_{end} - x_{start})$
\item $y(o) = y_{start} + o \times (y_{end} - y_{start})$
\item $radius(o) = radius_{start} + o \times (radius_{end} - radius_{start})$
\end{enumeratea}
where $o$ is a color stop offset value.

\pnum
The range of potential values for $o$ shall be determined by the Wrap Mode:
\begin{itemize}
\item For \tcode{wrap_mode::none}, the range of potential values for $o$ is \crange{0}{1}.
\item For all other \tcode{wrap_mode} values, the range of potential values for $o$ is\\ $[$~\tcode{numeric_limits<double>::lowest(),numeric_limits<double>::max()}~$]$.
\end{itemize}

\pnum
The interpolated circles shall be rendered starting from the smallest potential value of $o$.

\pnum
An interpolated circle shall not be rendered if its value for $o$ results in $radius(o)$ evaluating to a negative value.

\rSec1 [\iotwod.gradients.sampling] {Sampling from gradients}

\pnum
For any offset value $o$, its color value shall be determined according to the following rules:

\begin{enumeratea}
\item If there are less than two color stops or if all color stops have the same offset value, then the color value of every offset value shall be \tcode{rgba_color::transparent_black()} and the remainder of these rules are inapplicable.

\item If exactly one color stop has an offset value equal to $o$, $o$'s color value shall be the color value of that color stop and the remainder of these rules are inapplicable.

\item If two or more color stops have an offset value equal to $o$, $o$'s color value shall be the color value of the color stop which has the lowest index value among the set of color stops that have an offset value equal to $o$ and the remainder of \ref{\iotwod.gradients.sampling} is inapplicable.

\item When no color stop has the offset value of \tcode{0.0}, then, defining $n$ to be the offset value that is nearest to \tcode{0.0} among the offset values in the set of all color stops, if $o$ is in the offset range $[0, n)$, $o$'s color value shall be \tcode{rgba_color::transparent_black()} and the remainder of these rules are inapplicable.
\begin{note}
Since the range described does not include $n$, it does not matter how many color stops have $n$ as their offset value for purposes of this rule.
\end{note}

\item When no color stop has the offset value of \tcode{1.0}, then, defining $n$ to be the offset value that is nearest to \tcode{1.0} among the offset values in the set of all color stops, if $o$ is in the offset range $(n, 1]$, $o$'s color value shall be \tcode{rgba_color::transparent_black()} and the remainder of these rules are inapplicable.
\begin{note}
Since the range described does not include $n$, it does not matter how many color stops have $n$ as their offset value for purposes of this rule.
\end{note}

\item Each color stop has, at most, two adjacent color stops: one to its left and one to its right.

\item Adjacency of color stops is initially determined by offset values. If two or more color stops have the same offset value then index values are used to determine adjacency as described below.

\item For each color stop \textit{a}, the \term{set of color stops to its left} are those color stops which have an offset value which is closer to \tcode{0.0} than \textit{a}'s offset value.
\begin{note}
This includes any color stops with an offset value of \tcode{0.0} provided that \textit{a}'s offset value is not \tcode{0.0}.
\end{note}

\item For each color stop \textit{b}, the \term{set of color stops to its right} are those color stops which have an offset value which is closer to \tcode{1.0} than \textit{b}'s offset value.
\begin{note}
This includes any color stops with an offset value of \tcode{1.0} provided that \textit{b}'s offset value is not \tcode{1.0}.
\end{note}

\item A color stop which has an offset value of \tcode{0.0} does not have an adjacent color stop to its left.

\item A color stop which has an offset value of \tcode{1.0} does not have an adjacent color stop to its right.

\item If a color stop \textit{a}'s set of color stops to its left consists of exactly one color stop, that color stop is the color stop that is adjacent to \textit{a} on its left.

\item If a color stop \textit{b}'s set of color stops to its right consists of exactly one color stop, that color stop is the color stop that is adjacent to \textit{b} on its right.

\item If two or more color stops have the same offset value then the color stop with the lowest index value is the only color stop from that set of color stops which can have a color stop that is adjacent to it on its left and the color stop with the highest index value is the only color stop from that set of color stops which can have a color stop that is adjacent to it on its right. This rule takes precedence over all of the remaining rules.

\item If a color stop can have an adjacent color stop to its left, then the color stop which is adjacent to it to its left is the color stop from the set of color stops to its left which has an offset value which is closest to its offset value. If two or more color stops meet that criteria, then the color stop which is adjacent to it to its left is the color stop which has the highest index value from the set of color stops to its left which are tied for being closest to its offset value.

\item If a color stop can have an adjacent color stop to its right, then the color stop which is adjacent to it to its right is the color stop from the set of color stops to its right which has an offset value which is closest to its offset value. If two or more color stops meet that criteria, then the color stop which is adjacent to it to its right is the color stop which has the lowest index value from the set of color stops to its right which are tied for being closest to its offset value.

\item Where the value of $o$ is in the range $[0,1]$, its color value shall be determined by interpolating between the color stop, $r$, which is the color stop whose offset value is closest to $o$ without being less than $o$ and which can have an adjacent color stop to its left, and the color stop that is adjacent to $r$ on $r$'s left. The acceptable forms of interpolating between color values is set forth later in this section.

\item Where the value of $o$ is outside the range $[0,1]$, its color value depends on the value of Wrap Mode:
	\begin{itemize}
	\item If Wrap Mode is \tcode{wrap_mode::none}, the color value of $o$ shall be \tcode{rgba_color::transparent_black()}.
	
	\item If Wrap Mode is \tcode{wrap_mode::pad}, if $o$ is negative then the color value of $o$ shall be the same as-if the value of $o$ was \tcode{0.0}, otherwise the color value of $o$ shall be the same as-if the value of $o$ was \tcode{1.0}.
	
	\item If Wrap Mode is \tcode{wrap_mode::repeat}, then \tcode{1.0} shall be added to or subtracted from $o$ until $o$ is in the range $[0,1]$, at which point its color value is the color value for the modified value of $o$ as determined by these rules.
	\begin{example}
	Given $o == 2.1$, after application of this rule $o == 0.1$ and the color value of $o$ shall be the same value as-if the initial value of $o$ was $0.1$.
	
	Given $o == -0.3$, after application of this rule $o == 0.7$ and the color value of $o$ shall be the same as-if the initial value of $o$ was $0.7$.
	\end{example}
	
	\item If Wrap Mode is \tcode{wrap_mode::reflect}, $o$ shall be set to the absolute value of $o$, then 2.0 shall be subtracted from $o$ until $o$ is in the range $[0,2]$, then if $o$ is in the range $(1,2]$ then $o$ shall be set to \tcode{1.0 - (o - 1.0)}, at which point its color value is the color value for the modified value of $o$ as determined by these rules.
	\begin{example}
	Given $o == 2.8$, after application of this rule $o == 0.8$ and the color value of $o$ shall be the same value as-if the initial value of $o$ was $0.8$.
	
	Given $o == 3.6$, after application of this rule $o == 0.4$ and the color value of $o$ shall be the same value as-if the initial value of $o$ was $0.4$.
	
	Given $o == -0.3$, after application of this rule $o == 0.3$ and the color value of $o$ shall be the same as-if the initial value of $o$ was $0.3$.
	
	Given $o == -5.8$, after application of this rule $o == 0.2$ and the color value of $o$ shall be the same as-if the initial value of $o$ was $0.2$.
	\end{example}
	\end{itemize}
\end{enumeratea}

\pnum
It is unspecified whether the interpolation between the color values of two adjacent color stops is performed linearly on each color channel or is performed by a linear color interpolation algorithm implemented in hardware (typically in a graphics processing unit).

\pnum
Implementations shall interpolate between alpha channel values of adjacent color stops linearly except as provided in the following paragraph.

\pnum
A conforming implementation may use the alpha channel interpolation results from a linear color interpolation algorithm implemented in hardware even if those results differ from the results required by the previous paragraph.

%!TEX root = io2d.tex
\rSec0 [\iotwod.wrapmode] {Enum class \tcode{wrap_mode}}

\rSec1 [\iotwod.wrapmode.summary] {\tcode{wrap_mode} summary}

\pnum
The \tcode{wrap_mode} enum class describes how a point's visual data is 
determined if it is outside the bounds of the \term{source brush} (\ref{\iotwod.surface.rendering.brushes}) when sampling.

\pnum
Depending on the source brush's \tcode{filter} value, the visual data of several points may be required to determine the appropriate visual data value for the point that is being sampled. In this case, each point is sampled according to the source brush's \tcode{wrap_mode} value with two exceptions:
\begin{enumerate}
\item If the point to be sampled is within the bounds of the source brush and the source brush's \tcode{wrap_mode} value is \tcode{wrap_mode::none}, then if the source brush's \tcode{filter} value requires that one or more points which are outside of the bounds of the source brush be sampled, each of those points is sampled as-if the source brush's \tcode{wrap_mode} value is \tcode{wrap_mode::pad} rather than \tcode{wrap_mode::none}.
\item If the point to be sampled is within the bounds of the source brush and the source brush's \tcode{wrap_mode} value is \tcode{wrap_mode::none}, then if the source brush's \tcode{filter} value requires that one or more points which are inside of the bounds of the source brush be sampled, each of those points is sampled such that the visual data that is returned is the equivalent of \tcode{rgba_color::transparent_black}.
\end{enumerate}

\pnum
If a point to be sampled does not have a defined visual data element and the search for the nearest point with defined visual data produces two or more points with defined visual data that are equidistant from the point to be sampled, the returned visual data shall be an \unspecnorm value which is the visual data of one of those equidistant points. Where possible, implementations should choose the among the equidistant points that have an \xaxis value and a \yaxis value that is nearest to \tcode{0.0f}.

\pnum
See Table~\ref{tab:\iotwod.wrap_mode.meanings} for the meaning of each \tcode{wrap_mode} enumerator.

\rSec1 [\iotwod.wrapmode.synopsis] {\tcode{wrap_mode} synopsis}

\begin{codeblock}
namespace @\fullnamespace{}@ {
  enum class wrap_mode {
    none,
    repeat,
    reflect,
    pad
  };
}
\end{codeblock}

\rSec1 [\iotwod.wrapmode.enumerators] {\tcode{wrap_mode} enumerators}
\begin{libreqtab2}
 {\tcode{wrap_mode} enumerator meanings}
 {tab:\iotwod.wrap_mode.meanings}
 \\ \topline
 \lhdr{Enumerator}
 & \rhdr{Meaning}
 \\ \capsep
 \endfirsthead
 \continuedcaption\\
 \hline
 \lhdr{Enumerator}
 & \rhdr{Meaning}
 \\ \capsep
 \endhead
 \tcode{none}
 & If the point to be sampled is outside of the bounds of the source brush, the visual data that is returned is the equivalent of \tcode{rgba_color::transparent_black}.
 \\
 \tcode{repeat}
 & If the point to be sampled is outside of the bounds of the source brush, the visual data that is returned is the visual data that would have been returned if the source brush was infinitely large and repeated itself in 
 a left-to-right-left-to-right and top-to-bottom-top-to-bottom fashion.
 \\
 \tcode{reflect}
 & If the point to be sampled is outside of the bounds of the source brush, the visual data that is returned is the visual data that would have been returned if the source brush was infinitely large and repeated itself in 
 a left-to-right-to-left-to-right and top-to-bottom-to-top-to-bottom fashion.
 \\
 \tcode{pad}
 & If the point to be sampled is outside of the bounds of the source brush, the visual data that is returned is the visual data that would have been returned for the nearest defined point that is in inside the bounds of the source brush.
 \\
\end{libreqtab2}

%!TEX root = io2d.tex
\rSec0 [\iotwod.filter] {Enum class \tcode{filter}}

\rSec1 [\iotwod.filter.summary] {\tcode{filter} Summary}

\pnum
The \tcode{filter} enum class specifies the type of filter to use when sampling from a \pixmap.

\pnum
Three of the \tcode{filter} enumerators, \tcode{filter::fast}, \tcode{filter::good}, and \tcode{filter::best}, specify desired characteristics of the filter, leaving the choice of a specific filter to the implementation. 

The other two, \tcode{filter::nearest} and \tcode{filter::bilinear}, each specify a particular filter that shall be used.

\pnum
\begin{note}
The only type of brush that has a \pixmap as its \underlyingsurface is a brush with a brush type of \tcode{brush_type::surface}.
\end{note}

\pnum
See Table~\ref{tab:\iotwod.filter.meanings} for the meaning of each
\tcode{filter} enumerator.

\rSec1 [\iotwod.filter.synopsis] {\tcode{filter} Synopsis}

\begin{codeblock}
namespace std { namespace experimental { namespace io2d { inline namespace v1 {
  enum class filter {
    fast,
    good,
    best,
    nearest,
    bilinear
  };
} } } }
\end{codeblock}

\rSec1 [\iotwod.filter.enumerators] {\tcode{filter} Enumerators}
\begin{libreqtab2}
 {\tcode{filter} enumerator meanings}
 {tab:\iotwod.filter.meanings}
 \\ \topline
 \lhdr{Enumerator}
 & \rhdr{Meaning}
 \\ \capsep
 \endfirsthead
 \continuedcaption\\
 \hline
 \lhdr{Enumerator}
 & \rhdr{Meaning}
 \\ \capsep
 \endhead
 \tcode{fast}
 & The filter that corresponds to this value is \impldefplain{filter!fast}. The implementation shall ensure that the time complexity of the chosen filter is not greater than the time complexity of the filter that corresponds to \tcode{filter::good}.
 \begin{note}
 By choosing this value, the user is hinting that performance is more important than quality.
 \end{note}
 \\
 \tcode{good}
 & The filter that corresponds to this value is \impldefplain{filter!good}. The implementation shall ensure that the time complexity of the chosen formula is not greater than the time complexity of the formula for \tcode{filter::best}.
 \begin{note}
 By choosing this value, the user is hinting that quality and performance are equally important.
 \end{note}
 \\
 \tcode{best}
 & The filter that corresponds to this value is \impldefplain{filter!best}.
 \begin{note}
 By choosing this value, the user is hinting that quality is more important 
 than performance.
 \end{note}
 \\
 \tcode{nearest}
 & Nearest-neighbor interpolation filtering shall be used.
% The color of the pixel whose coordinates are nearest to the requested coordinates shall be produced. When two or more pixels are equally near to the requested coordinates, it is unspecified which of the equally near pixels shall be the pixel whose color shall be produced.
 \\
 \tcode{bilinear}
 & Bilinear interpolation filtering shall be used.
% The distance-weighted average of the four nearest pixels is used to create an interpolated color for the destination pixel. If some source pixel values do not exist (e.g. because an edge or corner of the source has been reached) then the current \tcode{wrap_mode} should be considered in order to determine the values for the missing source pixels. If a hardware sampler is available and offers bilinear filtering, implementations may use it even if its results do not conform to those produced by the description of bilinear filtering provided by this standard.% Whether mipmapping is used and whether mipmaps are generated for sources that do not have them is \impldefplain{filter!bilinear mipmapping}.
 \\
\end{libreqtab2}

%!TEX root = io2d.tex
\rSec0 [\iotwod.brushtype] {Enum class \tcode{brush_type}}

\rSec1 [\iotwod.brushtype.summary] {\tcode{brush_type} Summary}

\pnum
The \tcode{brush_type} enum class denotes which brush factory was used 
to form a \tcode{brush} object.

\pnum
See Table~\ref{tab:\iotwod.brushtype.meanings} for the meaning of each
\tcode{brush_type} enumerator.

\rSec1 [\iotwod.brushtype.synopsis] {\tcode{brush_type} Synopsis}

\begin{codeblock}
namespace std { namespace experimental { namespace io2d { inline namespace v1 {
  enum class brush_type {
    solid_color,
    surface,
    linear,
    radial,
    mesh
  };
} } } }
\end{codeblock}

\rSec1 [\iotwod.brushtype.enumerators] {\tcode{brush_type} Enumerators}
\begin{libreqtab2}
 {\tcode{brush_type} enumerator meanings}
 {tab:\iotwod.brushtype.meanings}
 \\ \topline
 \lhdr{Enumerator}
 & \rhdr{Meaning}
 \\ \capsep
 \endfirsthead
 \continuedcaption\\
 \hline
 \lhdr{Enumerator}
 & \rhdr{Meaning}
 \\ \capsep
 \endhead
 \tcode{solid_color}
 & The \tcode{brush} object was created from a \tcode{solid_color_brush_factory} object.
 \\
 \tcode{surface}
 & The \tcode{brush} object was created from a \tcode{surface_brush_factory} object.
 \\
 \tcode{linear}
 & The \tcode{brush} object was created from a \tcode{linear_brush_factory} object.
 \\
 \tcode{radial}
 & The \tcode{brush} object was created from a \tcode{radial_brush_factory} object.
 \\
 \tcode{mesh}
 & The \tcode{brush} object was created from a \tcode{mesh_brush_factory} object.
 \\
\end{libreqtab2}

%!TEX root = io2d.tex
\rSec0 [\iotwod.gradientstop] {Class \tcode{gradient_stop}}

\rSec1 [\iotwod.gradientstop.intro] {Overview}
\pnum
\indexlibrary{\idxcode{gradient_stop}}%
The class \tcode{gradient_stop} describes a gradient stop that is used by gradient brushes.

\pnum
It has an \term{offset} of type \tcode{float} and an \term{offset color} of type \tcode{rgba_color}.

\rSec1 [\iotwod.gradientstop.synopsis] {\tcode{gradient_stop} synopsis}

\begin{codeblock}
namespace std::experimental::io2d::v1 {
  class gradient_stop {
  public:
    // \ref{\iotwod.gradientstop.cons}, construct:
    constexpr gradient_stop() noexcept;
    constexpr gradient_stop(float o, rgba_color c) noexcept;
    
    // \ref{\iotwod.gradientstop.modifiers}, modifiers:
    constexpr void offset(float o) noexcept;
    constexpr void color(rgba_color c) noexcept;
	
    // \ref{\iotwod.gradientstop.observers}, observers:
    constexpr float offset() const noexcept;
    constexpr rgba_color color() const noexcept;
  };
  // \ref{\iotwod.gradientstop.ops}, operators:
  constexpr bool operator==(const gradient_stop& lhs, const gradient_stop& rhs)
    noexcept;
  constexpr bool operator!=(const gradient_stop& lhs, const gradient_stop& rhs)
    noexcept;
}
\end{codeblock}

\rSec1 [\iotwod.gradientstop.cons] {\tcode{gradient_stop} constructors}

\indexlibrary{\idxcode{gradient_stop}!constructor}%
\begin{itemdecl}
constexpr gradient_stop() noexcept;
\end{itemdecl}
\begin{itemdescr}
\pnum
\effects
Equivalent to: \tcode{gradient_stop(0.0f, rgba_color::transparent_black)}.
\end{itemdescr}

\indexlibrary{\idxcode{gradient_stop}!constructor}%
\begin{itemdecl}
constexpr gradient_stop(float o, rgba_color c) noexcept;
\end{itemdecl}
\begin{itemdescr}
\pnum
\requires
\tcode{o >= 0.0f} and \tcode{o <= 1.0f}.

\pnum
\effects
Constructs a \tcode{gradient_stop} object.

\pnum
The offset is \tcode{o}. The offset color is \tcode{c}.
\end{itemdescr}

\rSec1 [\iotwod.gradientstop.modifiers] {\tcode{gradient_stop} modifiers}

\indexlibrarymember{offset}{gradient_stop}%
\begin{itemdecl}
constexpr void offset(float o) noexcept;
\end{itemdecl}
\begin{itemdescr}
\pnum
\requires
\tcode{o >= 0.0f} and \tcode{o <= 1.0f}.

\pnum
\effects
The offset is \tcode{o}.
\end{itemdescr}

\indexlibrarymember{color}{gradient_stop}%
\begin{itemdecl}
constexpr void color(rgba_color c) noexcept;
\end{itemdecl}
\begin{itemdescr}
\pnum
\effects
The offset color is \tcode{c}.
\end{itemdescr}

\rSec1 [\iotwod.gradientstop.observers] {\tcode{gradient_stop} observers}

\indexlibrarymember{offset}{gradient_stop}%
\begin{itemdecl}
constexpr float offset() const noexcept;
\end{itemdecl}
\begin{itemdescr}
\pnum
\returns
The offset.
\end{itemdescr}

\indexlibrarymember{color}{gradient_stop}%
\begin{itemdecl}
constexpr rgba_color color() const noexcept;
\end{itemdecl}
\begin{itemdescr}
\pnum
\returns
The offset color.
\end{itemdescr}

\rSec1 [\iotwod.gradientstop.ops] {\tcode{gradient_stop} operators}

\indexlibrarymember{operator==}{gradient_stop}%
\begin{itemdecl}
constexpr bool operator==(const gradient_stop& lhs, const gradient_stop& rhs)
  noexcept;
\end{itemdecl}
\begin{itemdescr}
\pnum
\returns
\tcode{lhs.offset() == rhs.offset() \&\& lhs.color() == rhs.color();}
\end{itemdescr}

%!TEX root = io2d.tex
\rSec0 [\iotwod.brush] {Class \tcode{brush}}

\rSec1 [\iotwod.brush.intro] {\tcode{brush} summary}

\pnum
\indexlibrary{\idxcode{brush}}%
The class \tcode{brush} describes an opaque wrapper for graphics data.

\pnum
A \tcode{brush} object is usable with any \tcode{surface} or \tcode{surface}-derived object.

\pnum
A \tcode{brush} object's graphics data is immutable. It is observable only by the effect that it produces when the brush is used as a \term{
source brush} or as a \term{mask brush} (\ref{\iotwod.surface.rendering.brushes}).

\pnum
A \tcode{brush} object has a brush type of \tcode{brush_type}, which indicates which type of brush it is (Table~\ref{tab:\iotwod.brushtype.meanings}).

\pnum
As a result of technological limitations and considerations, a \tcode{brush} object's graphics data may have less precision than the data from which it was created.

%\pnum
%\begin{example}
%Several graphics and rendering technologies that are currently widely used typically store individual color and alpha channel data as 8-bit unsigned normalized integer values while the \tcode{float} type that is used by the \tcode{rgba_color} class for individual color and alpha is often a 64-bit value. As such, it is possible for a loss of precision when transforming the 64-bit channel data of an \tcode{rgba_color} object to the 8-bit channel data that is commonly used internally in such graphics and rendering technologies.
%\end{example}
%
\rSec1 [\iotwod.brush.synopsis] {\tcode{brush} synopsis}

\begin{codeblock}
namespace std::experimental::io2d::v1 {
  class brush {
  public:
    // \ref{\iotwod.brush.cons}, construct/copy/move/destroy:
    explicit brush(rgba_color c);
    template <class InputIterator>
    brush(vector_2d begin, vector_2d end,
      InputIterator first, InputIterator last);
    brush(vector_2d begin, vector_2d end,
      initializer_list<color_stop> il);
    template <class InputIterator>
    brush(const circle& start, const circle& end,
      InputIterator first, InputIterator last);
    brush(const circle& start, const circle& end,
      initializer_list<color_stop> il);
    explicit brush(image_surface&& img);

    // \ref{\iotwod.brush.observers}, observers:
    brush_type type() const noexcept;
  };
}
\end{codeblock}

\rSec1 [\iotwod.brush.sampling] {Sampling from a \tcode{brush} object}

\pnum
When sampling from a \tcode{brush} object \tcode{b}, the \tcode{brush_type} returned by calling \tcode{b.type()} shall determine how the results of sampling shall be determined:
\begin{enumerate}
\item If the result of \tcode{b.type()} is \tcode{brush_type::solid_color} then \tcode{b} is a \term{solid color brush}.
\item If the result of \tcode{b.type()} is \tcode{brush_type::surface} then \tcode{b} is a \term{surface brush}.
\item If the result of \tcode{b.type()} is \tcode{brush_type::linear} then \tcode{b} is a \term{linear gradient brush}.
\item If the result of \tcode{b.type()} is \tcode{brush_type::radial} then \tcode{b} is a \term{radial gradient brush}.
\end{enumerate}

\rSec2 [\iotwod.brush.sampling.color] {Sampling from a solid color brush}

\pnum
When \tcode{b} is a solid color brush, then when sampling from \tcode{b}, the visual data returned is always the visual data used to construct \tcode{b}, regardless of the point which is to be sampled and regardless of the return values of wrap mode, filter, and brush matrix or mask matrix.

\rSec2 [\iotwod.brush.sampling.linear] {Sampling from a linear gradient brush}

\pnum
When \tcode{b} is a linear gradient brush, when sampling point \tcode{pt}, where \tcode{pt} is the return value of calling the \tcode{transform_pt} member function of brush matrix or mask matrix using the requested point, from \tcode{b}, the visual data returned are as specified by \ref{\iotwod.gradients.linear} and \ref{\iotwod.gradients.sampling}.

\rSec2 [\iotwod.brush.sampling.radial] {Sampling from a radial gradient brush}

\pnum
When \tcode{b} is a radial gradient brush, when sampling point \tcode{pt}, where \tcode{pt} is the return value of calling the \tcode{transform_pt} member function of brush matrix or mask matrix using the requested point, from \tcode{b}, the visual data are as specified by \ref{\iotwod.gradients.radial} and \ref{\iotwod.gradients.sampling}.

\rSec2 [\iotwod.brush.sampling.surface] {Sampling from a surface brush}

\pnum
When \tcode{b} is a surface brush, when sampling point \tcode{pt}, where \tcode{pt} is the return value of calling the \tcode{transform_pt} member function of brush matrix or mask matrix using the requested point, from \tcode{b}, the visual data returned are from the point \tcode{pt} in the graphics data of the brush, taking into account the values of wrap mode and filter.

\rSec1 [\iotwod.brush.cons] {\tcode{brush} constructors and assignment operators}

\indexlibrary{\idxcode{brush}!constructor}%
\begin{itemdecl}
explicit brush(rgba_color c);
\end{itemdecl}
\begin{itemdescr}
\pnum
\effects
Constructs an object of type \tcode{brush}.

\pnum
The brush's brush type shall be set to the value \tcode{brush_type::solid_color}.

\pnum
The graphics data of the brush are created from the value of \tcode{c}. The visual data format of the graphics data are as-if it is that specified by \tcode{format::argb32}.

\pnum
\remarks
Sampling from this produces the results specified in \ref{\iotwod.brush.sampling.color}.
\end{itemdescr}

\indexlibrary{\idxcode{brush}!constructor}%
\begin{itemdecl}
template <class InputIterator>
brush(vector_2d begin, vector_2d end,
  InputIterator first, InputIterator last);
\end{itemdecl}
\begin{itemdescr}
\pnum
\effects
Constructs a linear gradient \tcode{brush} object with a begin point of \tcode{begin}, an end point of \tcode{end}, and a sequential series of \tcode{color stop} values beginning at {first} and ending at {last - 1}.

\pnum
The brush's brush type is \tcode{brush_type::linear}.

\pnum
\remarks
Sampling from this brush produces the results specified in \ref{\iotwod.brush.sampling.linear}.
\end{itemdescr}

\indexlibrary{\idxcode{brush}!constructor}%
\begin{itemdecl}
brush(vector_2d begin, vector_2d end,
  initializer_list<color_stop> il);
\end{itemdecl}
\begin{itemdescr}
\pnum
\effects
Constructs a linear gradient \tcode{brush} object with a begin point of \tcode{begin}, an end point of \tcode{end}, and the sequential series of \tcode{color stop} values in \tcode{il}.

\pnum
The brush's brush type is \tcode{brush_type::linear}.

\pnum
\remarks
Sampling from this brush produces the results specified in \ref{\iotwod.brush.sampling.linear}.
\end{itemdescr}

\indexlibrary{\idxcode{brush}!constructor}%
\begin{itemdecl}
template <class InputIterator>
brush(const circle& start, const circle& end,
  InputIterator first, InputIterator last);
\end{itemdecl}
\begin{itemdescr}
\pnum
\effects
Constructs a radial gradient \tcode{brush} object with a start circle of \tcode{start}, an end circle of \tcode{end},  and a sequential series of \tcode{color stop} values beginning at {first} and ending at {last - 1}.

\pnum
The brush's brush type is \tcode{brush_type::radial}.

\pnum
\remarks
Sampling from this brush produces the results specified in \ref{\iotwod.brush.sampling.radial}.
\end{itemdescr}

\indexlibrary{\idxcode{brush}!constructor}%
\begin{itemdecl}
brush(const circle& start, const circle& end,
  initializer_list<color_stop> il);
\end{itemdecl}
\begin{itemdescr}
\pnum
\effects
Constructs a radial gradient \tcode{brush} object with a start circle of \tcode{start}, an end circle of \tcode{end}, and the sequential series of \tcode{color stop} values in \tcode{il}.

\pnum
The brush's brush type is \tcode{brush_type::radial}.

\pnum
\remarks
Sampling from this brush produces the results specified in \ref{\iotwod.brush.sampling.radial}.
\end{itemdescr}

\indexlibrary{\idxcode{brush}!constructor}%
\begin{itemdecl}
explicit brush(image_surface&& img);
\end{itemdecl}
\begin{itemdescr}
\pnum
\effects
Constructs an object of type \tcode{brush}.

\pnum
The brush's brush type is \tcode{brush_type::surface}.

\pnum
The graphics data of the brush is as-if it is the raster graphics data of \tcode{img}.

\pnum
\remarks
Sampling from this brush produces the results specified in \ref{\iotwod.brush.sampling.surface}.
\end{itemdescr}

\rSec1 [\iotwod.brush.observers]{\tcode{brush} observers}

\indexlibrary{\idxcode{brush}!\idxcode{type}}%
\begin{itemdecl}
brush_type type() const noexcept;
\end{itemdecl}
\begin{itemdescr}
\pnum
\returns
The brush's brush type.
\end{itemdescr}

%%!TEX root = io2d.tex
\rSec0 [\iotwod.solidcolorbrushfact] {Class \tcode{solid_color_brush_factory}}

\rSec1 [\iotwod.solidcolorbrushfact.synopsis] {\tcode{solid_color_brush_factory} synopsis}

\begin{codeblock}
namespace std { namespace experimental { namespace io2d { inline namespace v1 {
  class solid_color_brush_factory {
  public:
    // \ref{\iotwod.solidcolorbrushfact.cons}, construct/copy/move/destroy:
    solid_color_brush_factory() noexcept;
    solid_color_brush_factory(const solid_color_brush_factory&) noexcept;
    solid_color_brush_factory& operator=(
      const solid_color_brush_factory&) noexcept;
    solid_color_brush_factory(solid_color_brush_factory&& other) noexcept;
    solid_color_brush_factory& operator=(
      solid_color_brush_factory&& other) noexcept;
    solid_color_brush_factory(const rgba_color& color) noexcept;

    // \ref{\iotwod.solidcolorbrushfact.modifiers}, modifiers:
    void color(const rgba_color& value) noexcept;

    // \ref{\iotwod.solidcolorbrushfact.observers}, observers:
    rgba_color color() const noexcept;
    
  private:
    rgba_color _Color;      // \expos
  };
} } } }
\end{codeblock}

\rSec1 [\iotwod.solidcolorbrushfact.intro] {\tcode{solid_color_brush_factory} Description}

\pnum
\indexlibrary{\idxcode{solid_color_brush_factory}}
The class \tcode{solid_color_brush_factory} describes a mutable factory for creating \tcode{brush} objects with uniform color and alpha data.

\rSec1 [\iotwod.solidcolorbrushfact.cons] {\tcode{solid_color_brush_factory} constructors and assignment operators}

\indexlibrary{\idxcode{solid_color_brush_factory}!constructor}
\begin{itemdecl}
    solid_color_brush_factory() noexcept;
\end{itemdecl}
\begin{itemdescr}
	\pnum
	\effects
	Constructs an object of type \tcode{solid_color_brush_factory}.
	
	\pnum
	\postcondition
	\tcode{_Color == rgba_color\{\}}.
	
\end{itemdescr}

\indexlibrary{\idxcode{solid_color_brush_factory}!constructor}
\begin{itemdecl}
    solid_color_brush_factory(const rgba_color& color) noexcept;
\end{itemdecl}
\begin{itemdescr}
	\pnum
	\effects
	Constructs an object of type \tcode{solid_color_brush_factory}. A \tcode{brush} created using this object will will have \tcode{color} as its color.
	
	\pnum
	\postcondition
	\tcode{_Color == color}.
		
\end{itemdescr}

\rSec1 [\iotwod.solidcolorbrushfact.modifiers] {\tcode{solid_color_brush_factory} modifiers}

\indexlibrary{\idxcode{solid_color_brush_factory}!\idxcode{color}}
\indexlibrary{\idxcode{color}!\idxcode{solid_color_brush_factory}}
\begin{itemdecl}
    void color(const rgba_color& value) noexcept;
\end{itemdecl}
\begin{itemdescr}
	\pnum
	\effects
	A \tcode{brush} created using this object will will have \tcode{value} as its color.
	
	\pnum
	\postcondition
	\tcode{_Color == value}.
	
\end{itemdescr}

\rSec1 [\iotwod.solidcolorbrushfact.observers] {\tcode{solid_color_brush_factory} observers}

\indexlibrary{\idxcode{solid_color_brush_factory}!\idxcode{color}}
\indexlibrary{\idxcode{color}!\idxcode{solid_color_brush_factory}}
\begin{itemdecl}
    rgba_color color() const noexcept;
\end{itemdecl}
\begin{itemdescr}
	\pnum
	\returns
	\tcode{_Color}.

\end{itemdescr}

%%!TEX root = io2d.tex
\rSec0 [linearbrushfact] {Class \tcode{linear_brush_factory}}

\rSec1 [linearbrushfact.synopsis] {\tcode{linear_brush_factory} synopsis}

\begin{codeblock}
namespace std { namespace experimental { namespace io2d { inline namespace v1 {
  class linear_brush_factory {
  public:
    // types
    typedef @\impdef@          size_type; // See (\ref{linearbrushfact.sizetype}).
    
    // \ref{linearbrushfact.cons}, construct/copy/move/destroy:
    linear_brush_factory() noexcept;
    linear_brush_factory(const linear_brush_factory&);
    linear_brush_factory& operator=(const linear_brush_factory&);
    linear_brush_factory(linear_brush_factory&& other) noexcept;
    linear_brush_factory& operator=(linear_brush_factory&& other) noexcept;
    linear_brush_factory(const vector_2d& begin, const vector_2d& end) noexcept;

    // \ref{linearbrushfact.modifiers}, modifiers:
    void add_color_stop(double offset, const rgba_color& color);
    void add_color_stop(double offset, const rgba_color& color, 
      error_code& ec) noexcept;
    void color_stop(size_type index, double offset,
      const rgba_color& color);
    void color_stop(size_type index, double offset,
      const rgba_color& color, error_code& ec) noexcept;
    void begin_point(const vector_2d& value) noexcept;
    void end_point(const vector_2d& value) noexcept;

    // \ref{linearbrushfact.observers}, observers:
    size_type color_stop_count() const noexcept;
    tuple<double, rgba_color> color_stop(size_type index) const;
    tuple<double, rgba_color> color_stop(size_type index,
      error_code& ec) const noexcept;
    vector_2d begin_point() const noexcept;
    vector_2d end_point() const noexcept;

  private:
    vector_2d _Begin_point;                             // \expos
    vector_2d _End_point;                               // \expos
    vector<tuple<double, rgba_color>> _Color_stops; // \expos
  };
} } } }
\end{codeblock}

\rSec1 [linearbrushfact.sizetype] {\tcode{linear_brush_factory::size_type}}

\pnum
The type of \tcode{linear_brush_factory::size_type} shall comply with the restrictions specified for the size_type of the collection of color stops that is part of the observable state of a gradient (\ref{gradients.colorstops}).

\rSec1 [linearbrushfact.intro] {\tcode{linear_brush_factory} Description}

\pnum
\indexlibrary{\idxcode{linear_brush_factory}}
The class \tcode{linear_brush_factory} describes a mutable factory for creating \tcode{brush} objects with a linear gradient describing its color and alpha data.

\pnum
For more information about gradients, including linear gradients, see \ref{gradients}.

\rSec1 [linearbrushfact.cons] {\tcode{linear_brush_factory} constructors and assignment operators}

\indexlibrary{\idxcode{linear_brush_factory}!constructor}
\begin{itemdecl}
linear_brush_factory() noexcept;
\end{itemdecl}
\begin{itemdescr}
\pnum
\effects
Constructs an object of type \tcode{linear_brush_factory}.

\pnum
\postconditions
\tcode{_Begin_point == vector_2d\{ \}}.

\pnum
\tcode{_End_point == vector_2d\{ \}}.

\pnum
\tcode{_Color_stops.empty() == true}.
\end{itemdescr}

\indexlibrary{\idxcode{linear_brush_factory}!constructor}
\begin{itemdecl}
linear_brush_factory(const vector_2d& begin, const vector_2d& end) noexcept;
\end{itemdecl}
\begin{itemdescr}
\pnum
\effects
Constructs an object of type \tcode{linear_brush_factory}.

\pnum
\postconditions
\tcode{_Begin_point == begin}.

\pnum
\tcode{_End_point == end}.

\pnum
\tcode{_Color_stops.empty() == true}.
\end{itemdescr}

\rSec1 [linearbrushfact.modifiers] {\tcode{linear_brush_factory} modifiers}

\indexlibrary{\idxcode{linear_brush_factory}!\idxcode{add_color_stop}}
\indexlibrary{\idxcode{add_color_stop}!\idxcode{linear_brush_factory}}
\begin{itemdecl}
void add_color_stop(double offset, const rgba_color& color);
void add_color_stop(double offset, const rgba_color& color, 
  error_code& ec) noexcept;
\end{itemdecl}
\begin{itemdescr}
\pnum
\effects
Adds a color stop with an offset value of \tcode{offset} and a color value of \tcode{color}.

\pnum
\tcode{_Color_stops.push_back(make_tuple(offset, color))}.

\pnum
\throws
As specified in Error reporting (\ref{\iotwod.err.report}).

\pnum
\errors
\tcode{errc::not_enough_memory} if the attempt to add the color stop fails.
\end{itemdescr}

\indexlibrary{\idxcode{linear_brush_factory}!\idxcode{color_stop}}
\indexlibrary{\idxcode{color_stop}!\idxcode{linear_brush_factory}}
\begin{itemdecl}
void color_stop(size_type index, double offset,
  const rgba_color& color);
void color_stop(size_type index, double offset,
  const rgba_color& color, error_code& ec) noexcept;
\end{itemdecl}
\begin{itemdescr}
\pnum
\effects
Replaces the color stop at index \tcode{index} with a color stop with an offset of \tcode{offset} and a color of \tcode{color}.

\pnum
\postconditions
\tcode{_Color_stops.at(index) == make_tuple(offset, color)}.

\pnum
\throws
As specified in Error reporting (\ref{\iotwod.err.report}).

\pnum
\errors
\tcode{io2d_error::invalid_index} if \tcode{_Color_stops.size() <= index}.
\end{itemdescr}

\indexlibrary{\idxcode{linear_brush_factory}!\idxcode{begin_point}}
\indexlibrary{\idxcode{begin_point}!\idxcode{linear_brush_factory}}
\begin{itemdecl}
void begin_point(const vector_2d& value) noexcept;
\end{itemdecl}
\begin{itemdescr}
\pnum
\effects
Sets the begin point to \tcode{value}.

\pnum
\postconditions
\tcode{_Begin_point == value}.
\end{itemdescr}

\indexlibrary{\idxcode{linear_brush_factory}!\idxcode{end_point}}
\indexlibrary{\idxcode{end_point}!\idxcode{linear_brush_factory}}
\begin{itemdecl}
void end_point(const vector_2d& value) noexcept;
\end{itemdecl}
\begin{itemdescr}
\pnum
\effects
Sets the end point to \tcode{value}.

\pnum
\postconditions
\tcode{_End_point == value}.
\end{itemdescr}

\rSec1 [linearbrushfact.observers] {\tcode{color_stop_count} observers}

\indexlibrary{\idxcode{linear_brush_factory}!\idxcode{color_stop_count}}
\indexlibrary{\idxcode{color_stop_count}!\idxcode{linear_brush_factory}}
\begin{itemdecl}
size_type color_stop_count() const noexcept;
\end{itemdecl}
\begin{itemdescr}
\pnum
\returns
\tcode{static_cast<size_type>(_Color_stops.size())}.
\end{itemdescr}

\indexlibrary{\idxcode{linear_brush_factory}!\idxcode{color_stop}}
\indexlibrary{\idxcode{color_stop}!\idxcode{linear_brush_factory}}
\begin{itemdecl}
tuple<double, rgba_color> color_stop(unsigned int index) const;
tuple<double, rgba_color> color_stop(unsigned int index,
  error_code& ec) const noexcept;
\end{itemdecl}
\begin{itemdescr}
\pnum
\returns
\tcode{_Color_stops.at(index)}.

\pnum
\throws
As specified in Error reporting (\ref{\iotwod.err.report}).

\pnum
\errors
\tcode{io2d_error::invalid_index} if \tcode{_Color_stops.size() <= index}.
\end{itemdescr}

\indexlibrary{\idxcode{linear_brush_factory}!\idxcode{begin_point}}
\indexlibrary{\idxcode{begin_point}!\idxcode{linear_brush_factory}}
\begin{itemdecl}
vector_2d begin_point() const noexcept;
\end{itemdecl}
\begin{itemdescr}
\pnum
\returns
\tcode{_Begin_point}.
\end{itemdescr}

\indexlibrary{\idxcode{linear_brush_factory}!\idxcode{end_point}}
\indexlibrary{\idxcode{end_point}!\idxcode{linear_brush_factory}}
\begin{itemdecl}
vector_2d end_point() const noexcept;
\end{itemdecl}
\begin{itemdescr}
\pnum
\returns
\tcode{_End_point}.
\end{itemdescr}

%%!TEX root = io2d.tex
\rSec0 [\iotwod.] {Class \tcode{}}

\rSec1 [\iotwod..synopsis] {\tcode{} synopsis}

\begin{codeblock}
namespace std { namespace experimental { namespace io2d { inline namespace v1 {
    // \ref{\iotwod..cons}, construct/copy/move/destroy:

    // \ref{\iotwod..modifiers}, modifiers:
    
    // \ref{\iotwod..observers}, observers:
    
    // \ref{\iotwod..member.ops}, member operators:
    
// \expos
  
  // \ref{\iotwod..ops}, non-member operators:
} } } }
\end{codeblock}

\rSec1 [\iotwod..intro] {\tcode{} Description}

\pnum
\indexlibrary{\idxcode{}}
The class \tcode{} describes .

\rSec1 [\iotwod..cons] {\tcode{} constructors and assignment operators}

\indexlibrary{\idxcode{}!constructor}
\begin{itemdecl}
\end{itemdecl}
\begin{itemdescr}
	\pnum
	\effects
	Constructs an object of type \tcode{}.
	
	\pnum
	\postconditions
	
\end{itemdescr}

\rSec1 [\iotwod..modifiers] {\tcode{} modifiers}

\indexlibrary{\idxcode{}!\idxcode{}}
\indexlibrary{\idxcode{}!\idxcode{}}
\begin{itemdecl}
\end{itemdecl}
\begin{itemdescr}
	\pnum
	\postconditions
	
\end{itemdescr}

\rSec1 [\iotwod..observers] {\tcode{} observers}

\indexlibrary{\idxcode{}!\idxcode{}}
\indexlibrary{\idxcode{}!\idxcode{}}
\begin{itemdecl}
\end{itemdecl}
\begin{itemdescr}
	\pnum
	\returns

\end{itemdescr}

\rSec1 [\iotwod..member.ops] {\tcode{} member operators}

\indexlibrary{\idxcode{}!\idxcode{}}
\indexlibrary{\idxcode{}!\idxcode{}}
\begin{itemdecl}
\end{itemdecl}
\begin{itemdescr}
	\pnum
	\effects
	
	\pnum
	\returns
	\tcode{*this}.
\end{itemdescr}

\rSec1 [\iotwod..ops] {\tcode{} non-member operators}

\indexlibrary{\idxcode{}!\idxcode{}}
\indexlibrary{\idxcode{}!\idxcode{}}
\begin{itemdecl}
\end{itemdecl}
\begin{itemdescr}
	\pnum
	\returns
\end{itemdescr}

%%%%!TEX root = io2d.tex
\rSec0 [\iotwod.] {Class \tcode{}}

\rSec1 [\iotwod..synopsis] {\tcode{} synopsis}

\begin{codeblock}
namespace std { namespace experimental { namespace io2d { inline namespace v1 {
    // \ref{\iotwod..cons}, construct/copy/move/destroy:

    // \ref{\iotwod..modifiers}, modifiers:
    
    // \ref{\iotwod..observers}, observers:
    
    // \ref{\iotwod..member.ops}, member operators:
    
// \expos
  
  // \ref{\iotwod..ops}, non-member operators:
} } } }
\end{codeblock}

\rSec1 [\iotwod..intro] {\tcode{} Description}

\pnum
\indexlibrary{\idxcode{}}
The class \tcode{} describes .

\rSec1 [\iotwod..cons] {\tcode{} constructors and assignment operators}

\indexlibrary{\idxcode{}!constructor}
\begin{itemdecl}
\end{itemdecl}
\begin{itemdescr}
	\pnum
	\effects
	Constructs an object of type \tcode{}.
	
	\pnum
	\postconditions
	
\end{itemdescr}

\rSec1 [\iotwod..modifiers] {\tcode{} modifiers}

\indexlibrary{\idxcode{}!\idxcode{}}
\indexlibrary{\idxcode{}!\idxcode{}}
\begin{itemdecl}
\end{itemdecl}
\begin{itemdescr}
	\pnum
	\postconditions
	
\end{itemdescr}

\rSec1 [\iotwod..observers] {\tcode{} observers}

\indexlibrary{\idxcode{}!\idxcode{}}
\indexlibrary{\idxcode{}!\idxcode{}}
\begin{itemdecl}
\end{itemdecl}
\begin{itemdescr}
	\pnum
	\returns

\end{itemdescr}

\rSec1 [\iotwod..member.ops] {\tcode{} member operators}

\indexlibrary{\idxcode{}!\idxcode{}}
\indexlibrary{\idxcode{}!\idxcode{}}
\begin{itemdecl}
\end{itemdecl}
\begin{itemdescr}
	\pnum
	\effects
	
	\pnum
	\returns
	\tcode{*this}.
\end{itemdescr}

\rSec1 [\iotwod..ops] {\tcode{} non-member operators}

\indexlibrary{\idxcode{}!\idxcode{}}
\indexlibrary{\idxcode{}!\idxcode{}}
\begin{itemdecl}
\end{itemdecl}
\begin{itemdescr}
	\pnum
	\returns
\end{itemdescr}

%%!TEX root = io2d.tex
\rSec0 [surfacebrushfact] {Class \tcode{surface_brush_factory}}

\rSec1 [surfacebrushfact.synopsis] {\tcode{surface_brush_factory} synopsis}

\begin{codeblock}
namespace std { namespace experimental { namespace io2d { inline namespace v1 {
  class surface_brush_factory {
  public:
    // \ref{surfacebrushfact.cons}, construct/copy/move/destroy:
    surface_brush_factory() noexcept;
    surface_brush_factory(const surface_brush_factory&) = delete;
    surface_brush_factory& operator=(const surface_brush_factory&) = delete;
    surface_brush_factory(surface_brush_factory&& other) noexcept;
    surface_brush_factory& operator=(surface_brush_factory&& other) noexcept;
    surface_brush_factory(experimental::io2d::surface& s);
    surface_brush_factory(experimental::io2d::surface& s, error_code& ec) noexcept;
    
    // \ref{surfacebrushfact.modifiers}, modifiers:
    image_surface surface(experimental::io2d::surface& s);
    void surface(experimental::io2d::surface& s, image_surface& oldSurface, 
      error_code& ec) noexcept;
    void surface(experimental::io2d::surface& s, error_code& ec) noexcept;
    
    // \ref{surfacebrushfact.observers}, observers:
    bool has_surface() const noexcept;
    const image_surface& surface() const;
    
  private:
    unique_ptr<image_surface> _Surface; // \expos
  };
} } } }
\end{codeblock}

\rSec1 [surfacebrushfact.intro] {\tcode{surface_brush_factory} Description}

\pnum
\indexlibrary{\idxcode{surface_brush_factory}}
The class \tcode{surface_brush_factory} describes a mutable factory for creating \tcode{brush} objects with a \tcode{surface} object describing its color and alpha data.

\rSec1 [surfacebrushfact.cons] {\tcode{surface_brush_factory} constructors and assignment operators}

\indexlibrary{\idxcode{surface_brush_factory}!constructor}
\begin{itemdecl}
    surface_brush_factory(experimental::io2d::surface& s);
    surface_brush_factory(experimental::io2d::surface& s, error_code& ec) noexcept;
\end{itemdecl}
\begin{itemdescr}
	\pnum
	\effects
	Constructs an object of type \tcode{surface_brush_factory}. Calls \tcode{s.flush()} then creates a copy of \tcode{s} and stores it as the surface which will be painted by a brush formed using \tcode{*this}.
	
	\pnum
	\postconditions
	\tcode{_Surface} contains a valid pointer to an \tcode{image_surface} object with the same width, height, \tcode{format} as \tcode{s} and a copy of the visual data of \tcode{s} as raster graphics data.
	
	\pnum
	\throws
	As specified in Error reporting (\ref{\iotwod.err.report}).
	
	\pnum
	\remarks
	Excluding any effects of calling \tcode{s.flush()}, \tcode{s} shall not be modified.
	
	\pnum
	\errors
	Any error condition documented for \tcode{surface::flush}.
	
	\pnum
	\tcode{errc::invalid_argument} if \tcode{!s.has_surface_resource()}.
	
	\pnum
	\tcode{io2d_error::surface_finished} if \tcode{s.is_finished()}.
	
	\pnum
	\tcode{errc::not_enough_memory} if an attempted memory allocation failed.
	
\end{itemdescr}

\rSec1 [surfacebrushfact.modifiers]{\tcode{surface_brush_factory} modifiers}

\indexlibrary{\idxcode{surface_brush_factory}!\idxcode{surface}}
\indexlibrary{\idxcode{surface}!\idxcode{surface_brush_factory}}
\begin{itemdecl}
    void surface(experimental::io2d::surface& s);
    void surface(experimental::io2d::surface& s, error_code& ec) noexcept;
\end{itemdecl}
\begin{itemdescr}
	\pnum
	\effects
	Calls \tcode{s.flush()} then creates a copy of \tcode{s} and stores it as the surface which will be painted by a brush formed using \tcode{*this}. The stored surface shall be accessible as an object of type \tcode{image_surface}.
	
	\pnum
	\postconditions
	\tcode{_Surface} contains a valid pointer to an \tcode{image_surface} object with the same width, height, \tcode{format} as \tcode{s} and a copy of the visual data of \tcode{s} as raster graphics data.

	\pnum
	\throws
	As specified in Error reporting (\ref{\iotwod.err.report}).

	\pnum
	\remarks
	Excluding the effects of calling \tcode{s.flush()}, \tcode{s} shall not be modified.
	
	\pnum
	\errors
	Any error condition documented for \tcode{surface::flush} (\ref{surface}).
	
	\pnum
	\tcode{errc::invalid_argument} if \tcode{!s.has_surface_resource()}.
	
	\pnum
	\tcode{io2d_error::surface_finished} if \tcode{s.is_finished()}.

	\pnum
	\tcode{errc::not_enough_memory} if an attempted memory allocation failed.

\end{itemdescr}

\rSec1 [surfacebrushfact.observers]{\tcode{surface_brush_factory} observers}

\indexlibrary{\idxcode{surface_brush_factory}!\idxcode{has_surface}}
\indexlibrary{\idxcode{has_surface}!\idxcode{surface_brush_factory}}
\begin{itemdecl}
    bool has_surface() const noexcept;
\end{itemdecl}
\begin{itemdescr}
	\pnum
	\returns
	\tcode{_Surface.get != nullptr}.

\end{itemdescr}

\indexlibrary{\idxcode{surface_brush_factory}!\idxcode{surface}}
\indexlibrary{\idxcode{surface}!\idxcode{surface_brush_factory}}
\begin{itemdecl}
    const image_surface& surface() const;
\end{itemdecl}
\begin{itemdescr}
	\pnum
	\preconditions
	\tcode{has_surface() == true}.
	
	\pnum
	\returns
	\tcode{_Surface.get()}.
	
\end{itemdescr}

\addtocounter{SectionDepthBase}{-1}
