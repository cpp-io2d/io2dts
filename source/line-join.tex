%!TEX root = io2d.tex
\rSec0 [\iotwod.linejoin] {Enum class \tcode{line_join}}

\rSec1 [\iotwod.linejoin.summary] {\tcode{line_join} Summary}

\pnum
The \tcode{line_join} enum class specifies how the junction of two line 
segments should be rendered when a \tcode{path} is stroked.
See Table~\ref{tab:\iotwod.linejoin.meanings} for the meaning of each
\tcode{} enumerator.

\rSec1 [\iotwod.linejoin.synopsis] {\tcode{line_join} Synopsis}

\begin{codeblock}
namespace std { namespace experimental { namespace drawing { inline namespace 
v1 {
  enum class line_join {
    miter,
    round,
    bevel,
    miter_or_bevel
  };
} } } }
\end{codeblock}

\rSec1 [\iotwod.linejoin.enumerators] {\tcode{line_join} Enumerators}
\begin{libreqtab2}
 {\tcode{line_join} enumerator meanings}
 {tab:\iotwod.linejoin.meanings}
 \\ \topline
 \lhdr{Enumerator}
 & \rhdr{Meaning}
 \\ \capsep
 \endfirsthead
 \continuedcaption\\
 \hline
 \lhdr{Enumerator}
 & \rhdr{Meaning}
 \\ \capsep
 \endhead
 \tcode{miter}
 & Joins will be angular. When the join angle is less than 0.0005 radians, it 
 is \impldef{line_join!miter} whether this value will produce mitered or 
 beveled joins. Specifically, when this value is set implementations may switch 
 to beveled rendering instead of mitered rendering at an \impdef angle value, 
 provided that the value is less than 0.0005 radians and that all joins with 
 angles below the \impdef angle value are rendered as beveled.
 \\
 \tcode{round}
 & Joins will be rounded, with the center of the circle being the join point.
 \\
 \tcode{bevel}
 & Joins will be beveled, with the join cut off at half the line width from the 
 join point. Implementations may vary the cut off distance by an amount that is 
 less than one pixel at each join for aesthetic or technical reasons.
 \\
 \tcode{miter_or_bevel}
 & Joins will be angular or beveled, depending on the current miter limit.
 \\
\end{libreqtab2}
