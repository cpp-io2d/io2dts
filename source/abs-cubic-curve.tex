%!TEX root = io2d.tex
\rSec0 [\iotwod.abscubiccurve] {Class \tcode{abs_cubic_curve}}

\pnum
\indexlibrary{\idxcode{abs_cubic_curve}}%
The class \tcode{abs_cubic_curve} describes a path item that adds a cubic \bezierlocal curve path segment to a path.

\pnum
It has a \term{first control point} of type \tcode{vector_2d}, a \term{second control point} of type \tcode{vector_2d}, and an \tcode{end point} of type \tcode{vector_2d}.

\pnum
When interpreting a path group, after the curve is added to the path, the path's current point is the end point.

\rSec1 [\iotwod.abscubiccurve.synopsis] {\tcode{abs_cubic_curve} synopsis}

\begin{codeblock}
namespace std::experimental::io2d::v1 {
  namespace path_data {
    class abs_cubic_curve {
    public:
      // \ref{\iotwod.abscubiccurve.cons}, construct:
      constexpr abs_cubic_curve() noexcept;
      constexpr abs_cubic_curve(const vector_2d& cpt1, const vector_2d& cpt2,
        const vector_2d& ept) noexcept;

      // \ref{\iotwod.abscubiccurve.modifiers}, modifiers:
      constexpr void control_1(const vector_2d& cpt) noexcept;
      constexpr void control_2(const vector_2d& cpt) noexcept;
      constexpr void end(const vector_2d& ept) noexcept;

      // \ref{\iotwod.abscubiccurve.observers}, observers:
      constexpr vector_2d control_1() const noexcept;
      constexpr vector_2d control_2() const noexcept;
      constexpr vector_2d end() const noexcept;
    };
    
    \ref{\iotwod.abscubiccurve.nonmember}, non-members
    constexpr bool operator==(const abs_cubic_curve& lhs,
      const abs_cubic_curve& rhs) noexcept;
    constexpr bool operator!=(const abs_cubic_curve& lhs,
      const abs_cubic_curve& rhs) noexcept;
  }
}
\end{codeblock}

\rSec1 [\iotwod.abscubiccurve.cons] {\tcode{abs_cubic_curve} constructors}

\indexlibrary{\idxcode{abs_cubic_curve}!constructor}%
\begin{itemdecl}
constexpr abs_cubic_curve() noexcept;
\end{itemdecl}
\begin{itemdescr}
\pnum
\effects
Equivalent to \tcode{abs_cubic_curve\{ vector_2d(), vector_2d(), vector_2d() \}}.
\end{itemdescr}

\indexlibrary{\idxcode{abs_cubic_curve}!constructor}%
\begin{itemdecl}
constexpr abs_cubic_curve(const vector_2d& cpt1, const vector_2d& cpt2,
  const vector_2d& ept) noexcept;
\end{itemdecl}
\begin{itemdescr}
\pnum
\effects
Constructs an object of type \tcode{abs_cubic_curve}.

\pnum
The first control point is \tcode{cpt1}.

\pnum
The second control point is \tcode{cpt2}.

\pnum
The end point is \tcode{ept}.
\end{itemdescr}

\rSec1 [\iotwod.abscubiccurve.modifiers]{\tcode{abs_cubic_curve} modifiers}

\indexlibrarymember{control_1}{abs_cubic_curve}%
\begin{itemdecl}
constexpr void control_1(const vector_2d& cpt) noexcept;
\end{itemdecl}
\begin{itemdescr}
\pnum
\effects
The first control point is \tcode{cpt}.
\end{itemdescr}

\indexlibrarymember{control_2}{abs_cubic_curve}%
\begin{itemdecl}
constexpr void control_2(const vector_2d& cpt) noexcept;
\end{itemdecl}
\begin{itemdescr}
\pnum
\effects
The second control point is \tcode{cpt}.
\end{itemdescr}

\indexlibrarymember{end}{abs_cubic_curve}%
\begin{itemdecl}
constexpr void end(const vector_2d& ept) noexcept;
\end{itemdecl}
\begin{itemdescr}
\pnum
\effects
The end point is \tcode{ept}.
\end{itemdescr}

\rSec1 [\iotwod.abscubiccurve.observers]{\tcode{abs_cubic_curve} observers}

\indexlibrarymember{control_1}{abs_cubic_curve}%
\begin{itemdecl}
constexpr vector_2d control_1() const noexcept;
\end{itemdecl}
\begin{itemdescr}
\pnum
\returns
The first control point.
\end{itemdescr}

\indexlibrarymember{control_2}{abs_cubic_curve}%
\begin{itemdecl}
constexpr vector_2d control_2() const noexcept;
\end{itemdecl}
\begin{itemdescr}
\pnum
\returns
The second control point.
\end{itemdescr}

\indexlibrarymember{abs_cubic_curve}{end}%
\begin{itemdecl}
constexpr vector_2d end() const noexcept;
\end{itemdecl}
\begin{itemdescr}
\pnum
\returns
The end point.
\end{itemdescr}

\rSec1 [\iotwod.abscubiccurve.nonmember]{Non-member functions}

\indexlibrarymember{operator==}{abs_cubic_curve}%
\begin{itemdecl}
constexpr bool operator==(const abs_cubic_curve& lhs,
  const abs_cubic_curve& rhs) noexcept;
\end{itemdecl}
\begin{itemdescr}
\pnum
\returns
\begin{codeblock}
lhs.control_1() == rhs.control_1() && lhs.control_2() == rhs.control_2() &&
lhs.end() && rhs.end()
\end{codeblock}
\end{itemdescr}

\indexlibrarymember{operator!=}{abs_cubic_curve}%
\begin{itemdecl}
constexpr bool operator!=(const abs_cubic_curve& lhs, const abs_cubic_curve& rhs) 
  noexcept;
\end{itemdecl}
\begin{itemdescr}
\pnum
\returns
\tcode{!(lhs == rhs)}.
\end{itemdescr}
