%!TEX root = io2d.tex
\rSec0 [\iotwod.abscubiccurve] {Class template \tcode{basic_figure_items<GraphicsSurfaces>::abs_cubic_curve}}

\rSec1 [\iotwod.abscubiccurve.intro] {Overview}

\pnum
\indexlibrary{\idxcode{abs_cubic_curve}}%
The class \tcode{basic_figure_items<GraphicsSurfaces>::abs_cubic_curve} describes a figure item that is a segment.

\pnum
It has a \term{first control point} of type \tcode{basic_point_2d<GraphicsSurfaces::graphics_math_type>}, a \term{second control point} of type \tcode{basic_point_2d<GraphicsSurfaces::graphics_math_type>}, and an \tcode{end point} of type \tcode{basic_point_2d<GraphicsSurfaces::graphics_math_type>}.

\pnum
The data are stored in an object of type \tcode{typename GraphicsSurfaces::paths::abs_cubic_curve_data_type}. It is accessible using the \tcode{data} member functions.

\rSec1 [\iotwod.abscubiccurve.synopsis] {Synopsis}
\begin{codeblock}
namespace std::experimemtal::io2d::v1 {
  template <class GraphicsSurfaces>
  class basic_figure_items<GraphicsSurfaces>::abs_cubic_curve {
  public:
    using graphics_math_type = typename GraphicsSurfaces::graphics_math_type;
    using data_type =
      typename GraphicsSurfaces::paths::abs_cubic_curve_data_type;

    // \ref{\iotwod.abscubiccurve.ctor}, construct:
    abs_cubic_curve();
    abs_cubic_curve(const basic_point_2d<graphics_math_type>& cpt1,
       const basic_point_2d<graphics_math_type>& cpt2,
       const basic_point_2d<graphics_math_type>& ept) noexcept;
    abs_cubic_curve(const abs_cubic_curve& other) = default;
    abs_cubic_curve(abs_cubic_curve&& other) noexcept = default;

    // assign:
    abs_cubic_curve& operator=(const abs_cubic_curve& other) = default;
    abs_cubic_curve& operator=(abs_cubic_curve&& other) noexcept = default;

    // \ref{\iotwod.abscubiccurve.acc}, accessors:
    const data_type& data() const noexcept;
    data_type& data() noexcept;

    // \ref{\iotwod.abscubiccurve.mod}, modifiers:
    void control_pt1(const basic_point_2d<graphics_math_type>& cpt) noexcept;
    void control_pt2(const basic_point_2d<graphics_math_type>& cpt) noexcept;
    void end_pt(const basic_point_2d<graphics_math_type>& ept) noexcept;

    // \ref{\iotwod.abscubiccurve.obs}, observers:
    basic_point_2d<graphics_math_type> control_pt1() const noexcept;
    basic_point_2d<graphics_math_type> control_pt2() const noexcept;
    basic_point_2d<graphics_math_type> end_pt() const noexcept;
  };

  // \ref{\iotwod.abscubiccurve.eq}, equality operators:
  template <class GraphicsSurfaces>
  bool operator==(
    const typename basic_figure_items<GraphicsSurfaces>::abs_cubic_curve& lhs,
    const typename basic_figure_items<GraphicsSurfaces>::abs_cubic_curve& rhs) 
    noexcept;  
  template <class GraphicsSurfaces>
  bool operator!=(
    const typename basic_figure_items<GraphicsSurfaces>::abs_cubic_curve& lhs,
    const typename basic_figure_items<GraphicsSurfaces>::abs_cubic_curve& rhs) 
    noexcept;  
}
\end{codeblock}

\rSec1 [\iotwod.abscubiccurve.ctor] {Constructors}%

\indexlibrary{\idxcode{abs_cubic_curve}!constructor}%
\begin{itemdecl}
abs_cubic_curve() noexcept;
\end{itemdecl}
\begin{itemdescr}
\pnum
\effects
Equivalent to \tcode{abs_cubic_curve\{ basic_point_2d(), basic_point_2d(), basic_point_2d() \}}.
\end{itemdescr}

\indexlibrary{\idxcode{abs_cubic_curve}!constructor}%
\begin{itemdecl}
abs_cubic_curve(const basic_point_2d<typename GraphicsSurfaces::graphics_math_type>& cpt1,
  const basic_point_2d<typename GraphicsSurfaces::graphics_math_type>& cpt2,
  const basic_point_2d<typename GraphicsSurfaces::graphics_math_type>& ept) noexcept;
\end{itemdecl}
\begin{itemdescr}
\pnum
\effects Constructs an object of type \tcode{abs_cubic_curve}.

\pnum
\remarks The first control point is \tcode{cpt1}.

\pnum
\remarks The second control point is \tcode{cpt2}.

\pnum
\remarks The end point is \tcode{ept}.
\end{itemdescr}

\rSec1 [\iotwod.abscubiccurve.acc] {Accessors}%

\indexlibrarymember{data}{abs_cubic_curve}%
\begin{itemdecl}
const data_type& data() const noexcept;
data_type& data() noexcept;
\end{itemdecl}
\begin{itemdescr}
\pnum
\returns A reference to the \tcode{rel_matrix} object's data object (See: \ref{\iotwod.abscubiccurve.intro}).
\end{itemdescr}

\rSec1 [\iotwod.abscubiccurve.mod] {Modifiers}

\indexlibrarymember{control_pt1}{abs_cubic_curve}%
\begin{itemdecl}
void control_pt1(const basic_point_2d<typename
  GraphicsSurfaces::graphics_math_type>& cpt) noexcept;
\end{itemdecl}
\begin{itemdescr}
\pnum
\effects
The first control point is \tcode{cpt}.
\end{itemdescr}

\indexlibrarymember{control_pt2}{abs_cubic_curve}%
\begin{itemdecl}
void control_pt2(const basic_point_2d<typename
  GraphicsSurfaces::graphics_math_type>& cpt) noexcept;
\end{itemdecl}
\begin{itemdescr}
\pnum
\effects
The second control point is \tcode{cpt}.
\end{itemdescr}

\indexlibrarymember{end_pt}{abs_cubic_curve}%
\begin{itemdecl}
void end_pt(const basic_point_2d<typename GraphicsSurfaces::graphics_math_type>& ept) noexcept;
\end{itemdecl}
\begin{itemdescr}
\pnum
\effects
The end point is \tcode{ept}.
\end{itemdescr}

\rSec1 [\iotwod.abscubiccurve.obs] {Observers}

\indexlibrarymember{control_pt1}{abs_cubic_curve}%
\begin{itemdecl}
basic_point_2d<graphics_math_type> control_pt1() const noexcept;
\end{itemdecl}
\begin{itemdescr}
\pnum
\returns The first control point.
\end{itemdescr}

\indexlibrarymember{control_pt2}{abs_cubic_curve}%
\begin{itemdecl}
basic_point_2d<graphics_math_type> control_pt2() const noexcept;
\end{itemdecl}
\begin{itemdescr}
\pnum
\returns The second control point.
\end{itemdescr}

\indexlibrarymember{end_pt}{abs_cubic_curve}%
\begin{itemdecl}
basic_point_2d<graphics_math_type> end_pt() const noexcept;
\end{itemdecl}
\begin{itemdescr}
\pnum
\returns The end point.
\end{itemdescr}

\rSec1 [\iotwod.abscubiccurve.eq] {Equality operators}%

\indexlibrarymember{operator==}{abs_cubic_curve}%
\begin{itemdecl}
template <class GraphicsSurfaces>
bool operator==(
  const typename basic_figure_items<GraphicsSurfaces>::abs_cubic_curve& lhs,
  const typename basic_figure_items<GraphicsSurfaces>::abs_cubic_curve& rhs) 
  noexcept;
\end{itemdecl}
\begin{itemdescr}
\pnum
\returns
\tcode{lhs.control_pt1() == rhs.control_pt1() \&\& lhs.control_pt2() == rhs.control_pt2() \&\& lhs.end_pt() == rhs.end_pt()}.
\end{itemdescr}

\indexlibrarymember{operator!=}{abs_cubic_curve}%
\begin{itemdecl}
template <class GraphicsSurfaces>
bool operator!=(
  const typename basic_figure_items<GraphicsSurfaces>::abs_cubic_curve& lhs,
  const typename basic_figure_items<GraphicsSurfaces>::abs_cubic_curve& rhs) 
  noexcept;
\end{itemdecl}
\begin{itemdescr}
\pnum
\returns
\tcode{lhs.control_pt1() != rhs.control_pt1() || lhs.control_pt2() != rhs.control_pt2() || lhs.end_pt() != rhs.end_pt()}.
\end{itemdescr}
