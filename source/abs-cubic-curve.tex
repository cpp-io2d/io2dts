%!TEX root = io2d.tex
\rSec0 [abscubiccurve] {Class \tcode{abs_cubic_curve}}

\pnum
\indexlibrary{\idxcode{abs_cubic_curve}}
The class \tcode{abs_cubic_curve} describes a path segment that is a cubic \bezierlocal curve.

\pnum
It has a first control point of type \tcode{vector_2d}, a second control point of type \tcode{vector_2d}, and an end point of type \tcode{vector_2d}.

\rSec1 [abscubiccurve.synopsis] {\tcode{abs_cubic_curve} synopsis}

\begin{codeblock}
namespace std { namespace experimental { namespace io2d { inline namespace v1 {
  namespace path_data {
    class abs_cubic_curve {
    public:
      // \ref{abscubiccurve.cons}, construct:
      abs_cubic_curve(const vector_2d& cp1, const vector_2d& cp2,
        const vector_2d& ep) noexcept;

      // \ref{abscubiccurve.modifiers}, modifiers:
      void control_point_1(const vector_2d& cp) noexcept;
      void control_point_2(const vector_2d& cp) noexcept;
      void end_point(const vector_2d& ep) noexcept;


      // \ref{abscubiccurve.observers}, observers:
      vector_2d control_point_1() const noexcept;
      vector_2d control_point_2() const noexcept;
      vector_2d end_point() const noexcept;
    };
  };
} } } }
\end{codeblock}

\rSec1 [abscubiccurve.cons] {\tcode{abs_cubic_curve} constructors}

\indexlibrary{\idxcode{abs_cubic_curve}!constructor}
\begin{itemdecl}
    abs_cubic_curve(const vector_2d& cp1, const vector_2d& cp2,
      const vector_2d& ep) noexcept;
\end{itemdecl}
\begin{itemdescr}
	\pnum
	\effects
	Constructs an object of type \tcode{abs_cubic_curve}.
	
	\pnum
	The first control point shall be set to the value of \tcode{cp1}.
	
	\pnum
	The second control point shall be set to the value of \tcode{cp2}.
	
	\pnum
	The end point shall be set to the value of \tcode{ep}.
\end{itemdescr}

\rSec1 [abscubiccurve.modifiers]{\tcode{abs_cubic_curve} modifiers}

\indexlibrary{\idxcode{abs_cubic_curve}!\idxcode{control_point_1}}
\indexlibrary{\idxcode{control_point_1}!\idxcode{abs_cubic_curve}}
\begin{itemdecl}
    void control_point_1(const vector_2d& cp) noexcept;
\end{itemdecl}
\begin{itemdescr}
	\pnum
	\effects
	The first control point shall be set to the value of \tcode{cp}.
\end{itemdescr}

\indexlibrary{\idxcode{abs_cubic_curve}!\idxcode{control_point_2}}
\indexlibrary{\idxcode{control_point_2}!\idxcode{abs_cubic_curve}}
\begin{itemdecl}
    void control_point_2(const vector_2d& cp) noexcept;
\end{itemdecl}
\begin{itemdescr}
	\pnum
	\effects
	The second control point shall be set to the value of \tcode{cp}.
\end{itemdescr}

\indexlibrary{\idxcode{abs_cubic_curve}!\idxcode{end_point}}
\indexlibrary{\idxcode{end_point}!\idxcode{abs_cubic_curve}}
\begin{itemdecl}
    void end_point(const vector_2d& ep) noexcept;
\end{itemdecl}
\begin{itemdescr}
	\pnum
	\effects
	The end point shall be set to the value of \tcode{ep}.
\end{itemdescr}

\rSec1 [abscubiccurve.observers]{\tcode{abs_cubic_curve} observers}

\indexlibrary{\idxcode{abs_cubic_curve}!\idxcode{control_point_1}}
\indexlibrary{\idxcode{control_point_1}!\idxcode{abs_cubic_curve}}
\begin{itemdecl}
    vector_2d control_point_1() const noexcept;
\end{itemdecl}
\begin{itemdescr}
	\pnum
	\returns
	The value of the first control point.
\end{itemdescr}

\indexlibrary{\idxcode{abs_cubic_curve}!\idxcode{control_point_2}}
\indexlibrary{\idxcode{control_point_2}!\idxcode{abs_cubic_curve}}
\begin{itemdecl}
    vector_2d control_point_2() const noexcept;
\end{itemdecl}
\begin{itemdescr}
	\pnum
	\returns
	The value of the second control point.
\end{itemdescr}

\indexlibrary{\idxcode{abs_cubic_curve}!\idxcode{end_point}}
\indexlibrary{\idxcode{end_point}!\idxcode{abs_cubic_curve}}
\begin{itemdecl}
    vector_2d end_point() const noexcept;
\end{itemdecl}
\begin{itemdescr}
	\pnum
	\returns
	The value of the end point.
\end{itemdescr}
