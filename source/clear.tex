%!TEX root = io2d.tex

\rSec0 [\iotwod.cmdlists.clear] {Class template \tcode{basic_commands<GraphicsSurfaces>::clear}}

\rSec1 [\iotwod.cmdlists.clear.intro] {Overview}

\pnum
\indexlibrary{\idxcode{clear}}%
The class template \tcode{basic_commands<GraphicsSurfaces>::clear} describes a command that invokes the \tcode{clear} member function of a surface.
%clears a surface by performing a paint rendering and composing operation using the \tcode{compositing_op::clear} composition algorithm with no .

\pnum
It has an \term{optional surface} of type \tcode{optional<reference_wrapper<basic_image_surface<GraphicsSurfaces>>>}. If optional surface has a value, the clear operation is performed on the optional surface instead of the surface that the command is submitted to.

\pnum
If optional surface has a value and the referenced \tcode{basic_image_surface<GraphicsSurfaces>} object has been destroyed or otherwise rendered invalid when a \tcode{basic_command_list<GraphicsSurfaces>} object built using this \tcode{paint} object is used by the program, the effects are undefined.

\pnum
The data are stored in an object of type \tcode{typename GraphicsSurfaces::surfaces::clear_data_type}. It is accessible using the \tcode{data} member functions.

\rSec1 [\iotwod.cmdlists.clear.synopsis] {Synopsis}
\begin{codeblock}
namespace @\fullnamespace{}@ {
  template <class GraphicsSurfaces>
  class basic_commands<GraphicsSurfaces::clear {
  public:
    using graphics_math_type = typename GraphicsSurfaces::graphics_math_type;
    using data_type = typename GraphicsSurfaces::surfaces::clear_data_type;

    // \ref{\iotwod.cmdlists.clear.ctor}, construct:
    clear() noexcept;
    clear(reference_wrapper<basic_image_surface<GraphicsSurfaces>> sfc) 
      noexcept;
    
    // \ref{\iotwod.cmdlists.clear.acc}, accessors:
    const data_type& data() const noexcept;
    data_type& data() noexcept;

    // \ref{\iotwod.cmdlists.clear.mod}, modifiers:
    void surface(
      optional<reference_wrapper<basic_image_surface<GraphicsSurfaces>>> sfc) 
      noexcept;

    // \ref{\iotwod.cmdlists.clear.obs}, observers:
    optional<reference_wrapper<basic_image_surface<GraphicsSurfaces>>> 
      surface() const noexcept;
  };

  // \ref{\iotwod.cmdlists.clear.eq}, equality operators:
  template <class GraphicsSurfaces>
  bool operator==(
    const typename basic_commands<GraphicsSurfaces::clear& lhs,
    const typename basic_commands<GraphicsSurfaces::clear& rhs) 
    noexcept;
  template <class GraphicsSurfaces>
  bool operator!=(
    const typename basic_commands<GraphicsSurfaces::clear& lhs,
    const typename basic_commands<GraphicsSurfaces::clear& rhs) 
    noexcept;
}
\end{codeblock}

\rSec1 [\iotwod.cmdlists.clear.ctor] {Constructors}%

\indexlibrary{\idxcode{clear}!constructor}%
\begin{itemdecl}
clear();
\end{itemdecl}
\begin{itemdescr}
\pnum
\effects Constructs an object of type \tcode{clear}.

\pnum
\postconditions \tcode{data() == GraphicsSurfaces::surfaces::create_clear()}.
\end{itemdescr}

\rSec1 [\iotwod.cmdlists.clear.acc] {Accessors}%

\indexlibrarymember{data}{clear}%
\begin{itemdecl}
const data_type& data() const noexcept;
data_type& data() noexcept;
\end{itemdecl}
\begin{itemdescr}
\pnum
\returns A reference to the \tcode{clear} object's data object (See: \ref{\iotwod.cmdlists.clear.intro}).

\pnum
\remarks The behavior of a program is undefined if the user modifies the data contained in the \tcode{data_type} object returned by this function.
\end{itemdescr}

\rSec1 [\iotwod.cmdlists.clear.mod] {Modifiers}%

\indexlibrarymember{surface}{clear}%
\begin{itemdecl}
void surface(
  optional<reference_wrapper<basic_image_surface<GraphicsSurfaces>>> sfc) 
  noexcept;
\end{itemdecl}
\begin{itemdescr}
\pnum
\effects Calls \tcode{GraphicsSurfaces::surfaces::surface(data(), sfc)}.

\pnum
\remarks The optional surface is \tcode{sfc}.
\end{itemdescr}

\rSec1 [\iotwod.cmdlists.clear.obs] {Observers}%

\indexlibrarymember{surface}{clear}%
\begin{itemdecl}
optional<reference_wrapper<basic_image_surface<GraphicsSurfaces>>> 
  surface() const noexcept;
\end{itemdecl}
\begin{itemdescr}
\pnum
\returns \tcode{GraphicsSurfaces::surfaces::surface(data())}.

\pnum
\remarks
The returned value is the optional surface.
\end{itemdescr}

\rSec1 [\iotwod.cmdlists.clear.eq] {Equality operators}%

\indexlibrarymember{operator==}{clear}%
\begin{itemdecl}
template <class GraphicsSurfaces>
bool operator==(
  const typename basic_commands<GraphicsSurfaces::clear& lhs,
  const typename basic_commands<GraphicsSurfaces::clear& rhs) 
  noexcept;
\end{itemdecl}
\begin{itemdescr}
\pnum
\returns \tcode{GraphicsSurfaces::surfaces::equal(lhs.data(), rhs.data())}.
\end{itemdescr}

\indexlibrarymember{operator!=}{clear}%
\begin{itemdecl}
template <class GraphicsSurfaces>
bool operator!=(
  const typename basic_commands<GraphicsSurfaces::clear& lhs,
  const typename basic_commands<GraphicsSurfaces::clear& rhs) 
  noexcept;
\end{itemdecl}
\begin{itemdescr}
\pnum
\returns \tcode{GraphicsSurfaces::surfaces::not_equal(lhs.data(), rhs.data())}.
\end{itemdescr}
