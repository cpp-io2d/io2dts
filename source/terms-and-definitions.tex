%!TEX root = io2d.tex

\rSec0[\iotwod.defns]{Terms and definitions}

\indextext{definitions|(}%
For the purposes of this document, the following terms and definitions apply.
ISO and IEC maintain terminological databases for use in standardization at the following addresses:
\begin{itemize}
\renewcommand{\labelitemi}{$\bullet$} %% Note: This renewcommand is scoped, i.e. it only affects the items in this itemize block.
\item IEC Electropedia: available at http://www.electropedia.org/
\item ISO Online browsing platform: available at http://www.iso.org/obp
\end{itemize}

\pnum
Terms that are used only in a small portion of this document are defined where they are used and italicized where they are defined.

\def\definition{\definitionx{\section}}%

\indexdefn{standard coordinate space}%
\definition{standard coordinate space}{\iotwod.defns.stndcrdspace}
a Euclidean plane described by a Cartesian coordinate system where the first coordinate is measured along a horizontal axis, called the \xaxis, oriented from left to right, the second coordinate is measured along a vertical axis, called the \yaxis, oriented from top to bottom, and rotation of a point around the origin by a positive value expressed in radians is counterclockwise

\indexdefn{point}%
\definition{point}{\iotwod.defns.point}
\defncontext{point} a coordinate designated by a floating point \xaxis{} value and a floating point \yaxis{} value within the \term{standard coordinate space} (\ref{\iotwod.defns.stndcrdspace})

\indexdefn{point}%
\definition{point}{\iotwod.defns.point.integral}
\defncontext{integral point} a coordinate designated by an integral \xaxis{} value and an integral \yaxis{} value within the \term{standard coordinate space} (\ref{\iotwod.defns.stndcrdspace})

\indexdefn{normalize}%
\definition{normalize}{\iotwod.defns.normalize}
to map a closed set of evenly spaced values in the range $[0, x]$ to an evenly spaced sequence of floating point values in the range $[0, 1]$
\begin{note}
The definition of normalize given is the definition for normalizing unsigned input. Signed normalization, i.e. the mapping of a closed set of evenly spaced values in the range $[-x, x)$ to an evenly spaced sequence of floating point values in the range $[-1, 1]$ is not used in this \documenttypename{}.
\end{note}

% The following may not be needed. Already defined in IEC 60050 723-05-05.
\indexdefn{aspect ratio}%
\definition{aspect ratio}{\iotwod.defns.aspectratio}
the ratio of the width to the height of a rectangular area

\indexdefn{color space}%
\definition{color space}{\iotwod.defns.colorspace}
an unambiguous mapping of values to colorimetric colors

\indexdefn{color stop}%
\definition{color stop}{\iotwod.defns.colorstop}
a tuple composed of a floating point offset value in the range $[0, 1]$ and a color value

\indexdefn{visual data}%
\definition{visual data}{\iotwod.defns.visdata}
data representing color, transparency, or some combination thereof

\indexdefn{graphics data}%
\definition{graphics data}{\iotwod.defns.graphicsdata}
\defncontext{graphics data} \term{visual data} (\ref{\iotwod.defns.visdata}) stored in an unspecified form

\indexdefn{channel}%
\definition{channel}{\iotwod.defns.channel}
%a bounded set of homogeneously-spaced real numbers in the range $[0,1]$
a component of \term{visual data} (\ref{\iotwod.defns.visdata}) with a defined bit size

\indexdefn{color channel}%
\definition{color channel}{\iotwod.defns.colorchannel}
a component of \term{visual data} (\ref{\iotwod.defns.visdata}) representing color

\indexdefn{alpha channel}%
\definition{alpha channel}{\iotwod.defns.alphachannel}
a component of \term{visual data} (\ref{\iotwod.defns.visdata}) representing transparency

\indexdefn{visual data format}%
\definition{visual data format}{\iotwod.defns.visdatafmt}
a specification that defines a total bit size, a set of one or more \term{channels} (\ref{\iotwod.defns.channel}), and each \term{channel}'s role, bit size, and location relative to the upper (high-order) bit
%a specification of visual data channels which defines a total bit size for the format and each channel's role, bit size, and location relative to the upper (high-order) bit
%\begin{note}
%The total bit size may be larger than the sum of the bit sizes of the channels of the format.
%\end{note}

\indexdefn{premultiplied format}%
\definition{premultiplied format}{\iotwod.defns.premultipliedformat}
a format with \term{color channels} (\ref{\iotwod.defns.colorchannel}) and an \term{alpha channel} (\ref{\iotwod.defns.alphachannel}) where each \term{color channel} is \term{normalized} (\ref{\iotwod.defns.normalize}) and then multiplied by the \term{normalized} \term{alpha channel} value
\begin{example}
Given the 32-bit non-premultiplied RGBA pixel with 8 bits per channel \{255, 0, 0, 127\} (half-transparent red), when normalized it would become \{1.0f, 0.0f, 0.0f, 0.5f\}. As such, in premultiplied, normalized format it would become \{0.5f, 0.0f, 0.0f, 0.5f\} as a result of multiplying each of the three color channels by the alpha channel value.
\end{example}

\indexdefn{visual data element}%
\definition{visual data element}{\iotwod.defns.visdataelem}
an item of \term{visual data} (\ref{\iotwod.defns.visdata}) with a defined \term{visual data format} (\ref{\iotwod.defns.visdatafmt})

% The following may not be needed. Already defined in IEC 60050 723-05-31
\indexdefn{pixel}
\definition{pixel}{\iotwod.defns.pixel}
a discrete, rectangular \term{visual data element} (\ref{\iotwod.defns.visdataelem})

\indexdefn{graphics data}%
\indexdefn{graphics data!raster}%
\definition{graphics data}{\iotwod.defns.graphics.raster}
\defncontext{raster graphics data} \term{visual data} (\ref{\iotwod.defns.visdata}) stored as \term{pixels} (\ref{\iotwod.defns.pixel}) that is accessible as-if it was an array of rows of pixels beginning with the pixel at the \term{integral point} $(0,0)$ (\ref{\iotwod.defns.point.integral})

\indexdefn{additive color}%
\definition{additive color}{\iotwod.defns.additivecolor}
a color defined by the emissive intensity of its \term{color channels} (\ref{\iotwod.defns.colorchannel})

\indexdefn{color model}%
\definition{color model}{\iotwod.defns.colormodel}
an ideal, mathematical representation of colors which often uses \term{color channels} (\ref{\iotwod.defns.colorchannel})

\indexdefn{color model}%
\indexdefn{color model!RGB}%
\definition{RGB color model}{\iotwod.defns.rgbcolormodel}
\defncontext{RGB} an \term{additive} (\ref{\iotwod.defns.additivecolor}) \term{color model} (\ref{\iotwod.defns.colormodel}) using red, green, and blue \term{color channels} (\ref{\iotwod.defns.colorchannel})

\indexdefn{color model}%
\indexdefn{color model!RGBA}%
\definition{RGBA color model}{\iotwod.defns.rgbacolormodel}
\defncontext{RGBA} an \term{RGB color model} (\ref{\iotwod.defns.rgbcolormodel}) with an \term{alpha channel} (\ref{\iotwod.defns.alphachannel})

\indexdefn{color space}%
\indexdefn{color space!sRGB}%
\definition{sRGB color space}{\iotwod.defns.srgbcolorspace}
\defncontext{sRGB} the \term{additive} (\ref{\iotwod.defns.additivecolor}) \term{color space} (\ref{\iotwod.defns.colorspace}) defined in IEC 61966-2-1 that is based on an \term{RGB color model} (\ref{\iotwod.defns.rgbcolormodel})

\indexdefn{control point}%
\definition{control point}{\iotwod.defns.controlpt}
the point other than the start point and the end point that is used in defining a curve

\indexdefn{\bezierlocal curve}%
\indexdefn{\bezierlocal curve!quadratic}%
\definition{\bezierlocal curve}{\iotwod.defns.bezier.quadratic}
\defncontext{quadratic} a curve defined by the 
equation $f(t) = (1 - t)^{2} \times P_{0} + 2 \times t \times (1 - t) 
\times P_{1} + t^{2} \times t \times P_{2}$ where t is in the range \crange{0}{1}, $P_{0}$ is the starting point, $P_{1}$ is the 
\term{control point} (\ref{\iotwod.defns.controlpt}), and $P_{2}$ is end point

\indexdefn{\bezierlocal curve}
\indexdefn{\bezierlocal curve!cubic}
\definition{\bezierlocal curve}{\iotwod.defns.bezier.cubic}
\defncontext{cubic} a curve defined by the 
equation $f(t) = (1 - t)^{3} \times P_{0} + 3 \times t \times (1 - t)^{2} 
\times P_{1} + 3 \times t^{2} \times (1 - t) \times P_{2} + t^{3} \times t 
\times P_{3}$ where t is in the range \crange{0}{1}, $P_{0}$ is the starting point, $P_{1}$ is the first \term{control point} (\ref{\iotwod.defns.controlpt}), $P_{2}$ is the second \term{control point}, and $P_{3}$ is the ending point

\indexdefn{path segment}%
\definition{path segment}{\iotwod.defns.pathseg}
a line, \term{\bezierlocal curve} (\ref{\iotwod.defns.bezier.quadratic}, \ref{\iotwod.defns.bezier.cubic}), or arc, each of which has a start point and an end point

\indexdefn{initial path segment}%
\definition{initial path segment}{\iotwod.defns.initialpathseg}
the \term{path segment} (\ref{\iotwod.defns.pathseg}) whose start point is not defined as being the end point of another \term{path segment}
\begin{note}
It is possible for the initial path segment and final path segment to be the same path segment.
\end{note}

\indexdefn{new path point}%
\definition{new path point}{\iotwod.defns.newpathpt}
the point in a path that is the start point of the \term{initial path segment} (\ref{\iotwod.defns.initialpathseg})

\indexdefn{final path segment}%
\definition{final path segment}{\iotwod.defns.finalpathseg}
the \term{path segment} (\ref{\iotwod.defns.pathseg}) whose end point does not define the start point of any other \term{path segment}
\begin{note}
It is possible for the initial path segment and final path segment to be the same path segment.
\end{note}

\indexdefn{current point}%
\definition{current point}{\iotwod.defns.currentpoint}
the point used as the start point of a \term{path segment} (\ref{\iotwod.defns.pathseg})

\indexdefn{open path}%
\definition{open path}{\iotwod.defns.openpath}
a path with one or more \term{path segments} (\ref{\iotwod.defns.pathseg}) where the \term{new path point} (\ref{\iotwod.defns.newpathpt}) is not used to define the end point of the path's \term{final path segment} (\ref{\iotwod.defns.finalpathseg})
\begin{note}
Even if the start point of the initial path segment and the end point of the final path segment are assigned the same coordinates, the path is still an open path since the final path segment's end point is not defined as being the start point of the initial segment but instead merely happens to have the same value as that point.
\end{note}

\indexdefn{closed path}%
\definition{closed path}{\iotwod.defns.closedpath}
a path with one or more \term{path segments} (\ref{\iotwod.defns.pathseg}) where the \term{new path point} (\ref{\iotwod.defns.newpathpt}) is used to define the end point of the path's \term{final path segment} (\ref{\iotwod.defns.finalpathseg})

\indexdefn{degenerate path segment}%
\definition{degenerate path segment}{\iotwod.defns.degenepathseg}
a \term{path segment} (\ref{\iotwod.defns.pathseg}) that has the same values for its start point, end point, and, if any, \term{control points} (\ref{\iotwod.defns.controlpt})

\indexdefn{path instruction}%
\definition{path instruction}{\iotwod.defns.closepathinstruction}
\defncontext{close path instruction} an instruction that creates a line \term{path segment} (\ref{\iotwod.defns.pathseg}) from the current point to the \term{new path point} (\ref{\iotwod.defns.newpathpt})

\indexdefn{path item}%
\definition{path item}{\iotwod.defns.pathitem}
a \term{path segment} (\ref{\iotwod.defns.pathseg}), \term{new path instruction} (\ref{\iotwod.defns.newpathinstruction}), \term{close path instruction} (\ref{\iotwod.defns.closepathinstruction}), or \term{path group instruction} (\ref{\iotwod.defns.pathgrpinstruction})

\indexdefn{path}%
\definition{path}{\iotwod.defns.path}
a collection of \term{path items} (\ref{\iotwod.defns.pathitem}) where the end point of each \term{path segment} (\ref{\iotwod.defns.pathseg}), except the \term{final path segment} (\ref{\iotwod.defns.finalpathseg}), defines the start point of exactly one other \term{path segment} in the collection

\indexdefn{path instruction}%
\definition{path instruction}{\iotwod.defns.newpathinstruction}
\defncontext{new path instruction} an instruction that creates a new \term{path} (\ref{\iotwod.defns.path})

\indexdefn{path group}%
\definition{path group}{\iotwod.defns.pathgroup}
a collection of \term{paths} (\ref{\iotwod.defns.path})

\indexdefn{path group transformation matrix}%
\definition{path group transformation matrix}{\iotwod.defns.pathgrptransform}
an affine transformation matrix used to apply affine transformations to the points in a \term{path group} (\ref{\iotwod.defns.pathgroup})

\indexdefn{path group instruction}%
\definition{path group instruction}{\iotwod.defns.pathgrpinstruction}
an instruction that modifies the \term{path group transformation matrix} (\ref{\iotwod.defns.pathgrptransform})

\indexdefn{degenerate path}%
\definition{degenerate path}{\iotwod.defns.degenpath}
a \term{path} (\ref{\iotwod.defns.path}) composed entirely of a \term{new path instruction} (\ref{\iotwod.defns.newpathinstruction}), zero or more \term{degenerate path segments} (\ref{\iotwod.defns.degenepathseg}), zero or more \term{path group items} (\ref{\iotwod.defns.pathgroup}), and, optionally, a \term{close path instruction} (\ref{\iotwod.defns.closepathinstruction})

\indexdefn{graphics subsystem}%
\definition{graphics subsystem}{\iotwod.defns.graphicssubsystem}
a collection of unspecified operating system and library functionality used to render and display 2D computer graphics

\indexdefn{graphics resource}%
\definition{graphics resource}{\iotwod.defns.graphicsresource}
\defncontext{graphics resource} an object of unspecified type used by an implementation
\begin{note}
By its definition a graphics resource is an implementation detail. Often it will be a graphics subsystem object (e.g. a graphics device or a render target) or an aggregate composed of multiple graphics subsystem objects. However the only requirement placed upon a graphics resource is that the implementation is able to use it to provide the functionality required of the graphics resource.
\end{note}

\indexdefn{graphics resource}%
\indexdefn{graphics resource!graphics data graphics resource}%
\definition{graphics resource}{\iotwod.defns.graphicsresource.graphicsdata}
\defncontext{graphics data graphics resource} an object of unspecified type used by an implementation to provide access to, and allow manipulation of, \term{visual data} (\ref{\iotwod.defns.visdata})

\indexdefn{\pixmap}%
\definition{\pixmap}{\iotwod.defns.pixmap}
a raster \term{graphics data graphics resource} (\ref{\iotwod.defns.graphicsresource.graphicsdata})

\indexdefn{filter}%
\definition{filter}{\iotwod.defns.filter}
a mathematical function that determines the \term{visual data} (\ref{\iotwod.defns.visdata}) value of a point for a \term{graphics data graphics resource} (\ref{\iotwod.defns.graphicsresource.graphicsdata})

\indexdefn{composition algorithm}%
\definition{composition algorithm}{\iotwod.defns.compositionalgorithm}
an algorithm that combines a source \term{visual data element} (\ref{\iotwod.defns.visdataelem}) and a destination \term{visual data element} producing a \term{visual data element} that has the same \term{visual data format} (\ref{\iotwod.defns.visdatafmt}) as the destination \term{visual data element}

\indexdefn{compose}%
\definition{compose}{\iotwod.defns.compose}
to combine part or all of a source \term{graphics data graphics resource} (\ref{\iotwod.defns.graphicsresource.graphicsdata}) with a destination \term{graphics data graphics resource} in the manner specified by a \term{composition algorithm} (\ref{\iotwod.defns.compositionalgorithm})

\indexdefn{composing operation}%
\definition{composing operation}{\iotwod.defns.composingoperation}
an operation that performs \term{composing} (\ref{\iotwod.defns.compose})
%an operation that uses a composition algorithm to combine part or all of a source of visual data capable of being treated as though it were a \pixmap with a \pixmap

\indexdefn{artifact}%
\definition{artifact}{\iotwod.defns.artifact}
an error in the results of the application of a \term{composing operation} (\ref{\iotwod.defns.composingoperation})

\indexdefn{sample}%
\definition{sample}{\iotwod.defns.sample}
to use a \term{filter} (\ref{\iotwod.defns.filter}) to obtain the \term{visual data} (\ref{\iotwod.defns.visdata}) for a given point from a \term{graphics data graphics resource} (\ref{\iotwod.defns.graphicsresource.graphicsdata})

\indexdefn{aliasing}%
\definition{aliasing}{\iotwod.defns.alias}
the presence of visual \term{artifacts} (\ref{\iotwod.defns.artifact}) in the results of rendering due to \term{sampling} (\ref{\iotwod.defns.sample}) imperfections

\indexdefn{anti-aliasing}%
\definition{anti-aliasing}{\iotwod.defns.antialias}
the application of a function or algorithm while \term{composing} (\ref{\iotwod.defns.compose}) to reduce \term{aliasing} (\ref{\iotwod.defns.alias})
\begin{note}
Certain algorithms can produce ``better'' results, i.e. results with fewer artifacts or with less pronounced artifacts, when rendering text with anti-aliasing due to the nature of text rendering. As such, it often makes sense to provide the ability to choose one type of anti-aliasing for text rendering and another for all other rendering and to provide different sets of anti-aliasing types to choose from for each of the two operations.
\end{note}

\indexdefn{graphics state data}%
\definition{graphics state data}{\iotwod.defns.graphicsstatedata}
data which specify how some part of the process of rendering, or of a \term{composing operation} (\ref{\iotwod.defns.composingoperation}), shall be performed in part or in whole

\indexdefn{render}%
\definition{render}{\iotwod.defns.render}
to transform a \term{path group} (\ref{\iotwod.defns.pathgroup}) into graphics data in the manner specified by a set of \term{graphics state data} (\ref{\iotwod.defns.graphicsstatedata})

\indexdefn{rendering operation}%
\definition{rendering operation}{\iotwod.defns.renderingoperation}
an operation that performs \term{rendering} (\ref{\iotwod.defns.render})

\indexdefn{rendering and composing operation}%
\definition{rendering and composing operation}{\iotwod.defns.renderingandcomposingop}
an operation that is either a \term{composing operation} (\ref{\iotwod.defns.composingoperation}), or a \term{rendering operation} (\ref{\iotwod.defns.renderingoperation}) followed by a \term{composing operation}

%%!TEX root = io2d.tex

\indexdefn{path segment}
\definition{path segment}{\iotwod.general.defns.pathsegment}
a line, \bezierlocal curve, or arc, each of which has a start point and an end point

\indexdefn{control point}
\definition{control point}{\iotwod.general.defns.controlpoint}
a point other than the start point and end point that is used in defining a \bezierlocal curve

\indexdefn{degenerate path segment}
\definition{degenerate path segment}{\iotwod.general.defns.degeneratepathsegment}
a path segment that has the same values for its start point, end point, and, if any, control points

\indexdefn{initial path segment}
\definition{initial path segment}{\iotwod.general.defns.initialpathsegment}
a path segment whose start point is not defined as being the end point of another path segment
\enternote
It is possible for the initial path segment and final path segment to be the same path segment.
\exitnote

\indexdefn{final path segment}
\definition{final path segment}{\iotwod.general.defns.finalpathsegment}
a path segment whose end point shall not be used to define the start point of any other path segment
\enternote
It is possible for the initial path segment and final path segment to be the same path segment.
\exitnote

\indexdefn{path instruction}
\definition{path instruction}{\iotwod.general.defns.pathinstruction}
an instruction that creates a new path, closes an existing path, or modifies the interpretation of path segments that follow it

\indexdefn{path}
\definition{path}{\iotwod.general.defns.path}
a collection of path instructions and path segments where the end point of each path segment, except the final path segment, defines the start point of exactly one other path segment in the collection

\indexdefn{current point}
\definition{current point}{\iotwod.general.defns.currentpoint}
a point established by various operations used in creating a path
\enternote
A new path has no current point except as otherwise specified.
\exitnote

\indexdefn{last-move-to point}
\definition{last-move-to point}{\iotwod.general.defns.lastmovetopoint}
the point in a path that is the start point of the initial path segment

\indexdefn{path group}
\definition{path group}{\iotwod.general.defns.pathgroup}
a collection of paths

\indexdefn{closed path}
\definition{closed path}{\iotwod.general.defns.closedpath}
a path with one or more path segments where the last-move-to point is used to define the end point of the path's final path segment

\indexdefn{open path}
\definition{open path}{\iotwod.general.defns.openpath}
a path with one or more path segments where the last-move-to point is not used to define the end point of the path's final path segment
\enternote
Even if the start point of the initial path segment and the end point of the final path segment are assigned the same coordinates, the path is still an open path since the final path segment's end point is not defined as being the start point of the initial segment but instead merely happens to have the same value as that point.
\exitnote

\indexdefn{degenerate path}
\definition{degenerate path}{\iotwod.general.defns.degeneratepath}
a path with only one path segment
\enternote
The path segment is not required to be a degenerate path segment.
\exitnote


%%!TEX root = io2d.tex

\indexdefn{alignment line}
\definition{alignment line}{\iotwod.general.defns.alignmentline}
imaginary line to which most glyph images of a font seem to align \\
\lbrack SOURCE: ISO/IEC 9541-1:2012, definition 3.1 \rbrack

\indexdefn{current position}
\definition{current position}{\iotwod.general.defns.currentposition}
a point on a graphics data graphics resource at which the next glyph representation is to be rendered

\indexdefn{design size}
\definition{design size}{\iotwod.general.defns.designsize}
absolute size at which a font is designed to be used \\
\lbrack SOURCE: ISO/IEC 9541-1:2012, definition 3.3 \rbrack

\indexdefn{escapement}
\definition{escapement}{\iotwod.general.defns.escapement}
movement of the current position on the presentation surface after a glyph representation is rendered

\indexdefn{escapement point}
\definition{escapement point}{\iotwod.general.defns.escapementpoint}
a glyph metric; a point in the glyph's standard coordinate system, to which the current position on the graphics data graphics resource is usually translated, after the glyph representation is rendered

\indexdefn{font}
\definition{font}{\iotwod.general.defns.font}
a collection of glyph images having the same basic design, e.g., \textit{Courier Bold Oblique} \\
\lbrack SOURCE: ISO/IEC 9541-1:2012, definition 3.6 \rbrack

\indexdefn{font family}
\definition{font family}{\iotwod.general.defns.fontfamily}
a collection of fonts of common design, e.g., \textit{Courier, Courier Bold, Courier Bold Oblique} \\
\lbrack SOURCE: ISO/IEC 9541-1:2012, definition 3.7 \rbrack

\indexdefn{font metrics}
\definition{font metrics}{\iotwod.general.defns.fontmetrics}
the set of dimensions and positioning information in a font resource common to all glyph representations contained in that font resource \\
\lbrack SOURCE: ISO/IEC 9541-1:2012, definition 3.8 \rbrack

%\indexdefn{font reference}
%\definition{font reference}{\iotwod.general.defns.fontreference}
%the information about a font resource in an electronic document representation, and possible procedures and operations on that information, which identify or describe the desired font \\
%\lbrack SOURCE: ISO/IEC 9541-1:2012, definition 3.9 \rbrack
%
\indexdefn{font resource}
\definition{font resource}{\iotwod.general.defns.fontresource}
a collection of glyph representations together with descriptive and font metric information which are relevant to the collection of glyph representations as a whole \\
\lbrack SOURCE: ISO/IEC 9541-1:2012, definition 3.10 \rbrack

\indexdefn{font size}
\definition{font size}{\iotwod.general.defns.fontsize}
a scalar reference size relative to which most font metrics, glyph shapes and glyph metrics are specified \\
\lbrack SOURCE: ISO/IEC 9541-1:2012, definition 3.11 \rbrack

\indexdefn{glyph}
\definition{glyph}{\iotwod.general.defns.glyph}
a recognizable abstract graphic symbol which is independent of any specific design \\
\lbrack SOURCE: ISO/IEC 9541-1:2012, definition 3.12 \rbrack

\indexdefn{glyph collection}
\definition{glyph collection}{\iotwod.general.defns.glyphcollection}
an identified set of glyphs \\
\lbrack SOURCE: ISO/IEC 9541-1:2012, definition 3.13 \rbrack

\indexdefn{glyph image}
\definition{glyph image}{\iotwod.general.defns.glyphimage}
an image of a glyph, as obtained from a glyph representation rendered and composed to a graphics data graphics resource

\indexdefn{glyph metrics}
\definition{glyph metrics}{\iotwod.general.defns.glyphmetrics}
the set of information in a glyph representation used for defining the dimensions and positioning of the glyph shape \\
\lbrack SOURCE: ISO/IEC 9541-1:2012, definition 3.16 \rbrack

\indexdefn{glyph representation}
\definition{glyph representation}{\iotwod.general.defns.glyphrepresentation}
the glyph shape and glyph metrics associated with a specific glyph in a font resource \\
\lbrack SOURCE: ISO/IEC 9541-1:2012, definition 3.17 \rbrack

\indexdefn{glyph shape}
\definition{glyph shape}{\iotwod.general.defns.glyphshape}
the set of information in a glyph representation used for defining the shape which represents the glyph \\
\lbrack SOURCE: ISO/IEC 9541-1:2012, definition 3.18 \rbrack

\indexdefn{kern}
\definition{kern}{\iotwod.general.defns.kern}
the extension of a glyph shape beyond its position point or escapement point \\
\lbrack SOURCE: ISO/IEC 9541-1:2012, definition 3.19 \rbrack

\indexdefn{position point}
\definition{position point}{\iotwod.general.defns.positionpoint}
a glyph metric; a point in the glyph's standard coordinate system, usually translated to the current position on the graphics data graphics resource before the glyph shape is rendered

\indexdefn{posture}
\definition{posture}{\iotwod.general.defns.posture}
the extent to which the shape of a glyph or set of glyphs appears to incline, including any consequent design or form change \\
\lbrack SOURCE: ISO/IEC 9541-1:2012, definition 3.22 \rbrack

\indexdefn{proportionate width}
\definition{proportionate width}{\iotwod.general.defns.proportionatewidth}
the ratio of a glyph's or set of glyphs' escapement to font height \\
\lbrack SOURCE: ISO/IEC 9541-1:2012, definition 3.24 \rbrack

\indexdefn{stem}
\definition{stem}{\iotwod.general.defns.stem}
the major stroke of a glyph shape \\
\lbrack SOURCE: ISO/IEC 9541-1:2012, definition 3.25 \rbrack

\indexdefn{weight}
\definition{weight}{\iotwod.general.defns.weight}
the ratio of a glyph's or set of glyphs' stem width to font height \\
\lbrack SOURCE: ISO/IEC 9541-1:2012, definition 3.26 \rbrack

\indexdefn{writing mode}
\definition{writing mode}{\iotwod.general.defns.writingmode}
an identified mode for setting of text in a writing system, usually corresponding to a nominal escapement direction of the glyphs in that mode, i.e., left-to-right, right-to-left or top-to-bottom \\
\lbrack SOURCE: ISO/IEC 9541-1:2012, definition 3.27 \rbrack

\indexdefn{body size}
\definition{body size}{\iotwod.general.defns.bodysize}
the font size, measured along the y axis of the glyph's standard coordinate system

\indexdefn{wrap_modeed body size}
\definition{wrap_modeed body size}{\iotwod.general.defns.wrap_modeedbodysize}
a reference size with two components, measured respectively along the \xaxis and \yaxis of the glyph's standard coordinate system

\indexdefn{design frame}
\definition{design frame}{\iotwod.general.defns.designframe}
dimensional expression that specifies the area inside which a set of glyph images can be designed \\
\lbrack SOURCE: ISO/IEC 9541-1:2012, definition 3.30 \rbrack

\indexdefn{bounding box}
\definition{bounding box}{\iotwod.general.defns.boundingbox}
dimensional expression to specify an actual area that a glyph image occupies within a design frame \\
\lbrack SOURCE: ISO/IEC 9541-1:2012, definition 3.31 \rbrack
%
%\indexdefn{blackness}
%\definition{blackness}{\iotwod.general.defns.blackness}
%the ratio of the blackened area of a glyph image to the wrap_modeed body size area of the glyph image \\
%\lbrack SOURCE: ISO/IEC 9541-1:2012, definition 3.32 \rbrack

%
% end definitions
\indextext{definitions|)}

