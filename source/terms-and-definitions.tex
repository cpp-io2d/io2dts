%!TEX root = io2d.tex

\rSec0[\iotwod.defns]{Terms and definitions}

\indextext{definitions|(}%
For the purposes of this document, the following terms and definitions apply.
ISO and IEC maintain terminological databases for use in standardization at the following addresses:
\begin{itemize}
\renewcommand{\labelitemi}{$\bullet$} %% Note: This renewcommand is scoped, i.e. it only affects the items in this itemize block.
\item IEC Electropedia: available at \url{http://www.electropedia.org/}
\item ISO Online browsing platform: available at \url{http://www.iso.org/obp}
\end{itemize}

\pnum
Terms that are used only in a small portion of this document are defined where they are used and italicized where they are defined.

\indexdefn{point}%
\definition{point}{\iotwod.defns.point}
coordinate designated by a floating-point \xaxis{} value and a floating-point \yaxis{} value

\indexdefn{origin}%
\definition{origin}{\iotwod.defns.origin}
point with an \xaxis{} value of $0$ and a \yaxis{} value of $0$

\indexdefn{standard coordinate space}%
\definition{standard coordinate space}{\iotwod.defns.stndcrdspace}
Euclidean plane described by a Cartesian coordinate system where the \xaxis{} is a horizontal axis oriented from left to right, the \yaxis{} is a vertical axis oriented from top to bottom, and rotation of a point, excluding the origin, around the origin by a positive value in radians is counterclockwise

\indexdefn{integral point}%
\definition{integral point}{\iotwod.defns.point.integral}
point where the \xaxis{} value and the \yaxis{} value are integers

\indexdefn{normalize}%
\definition{normalize}{\iotwod.defns.normalize}
map a closed set of evenly spaced values in the range $[0, x]$ to an evenly spaced sequence of floating-point values in the range $[0, 1]$
\begin{note}
The definition of normalize given is the definition for normalizing unsigned input. Signed normalization, i.e. the mapping of a closed set of evenly spaced values in the range $[-x, x)$ to an evenly spaced sequence of floating-point values in the range $[-1, 1]$ is not used in this \documenttypename{}.
\end{note}

% The following may not be needed. Already defined in IEC 60050 723-05-05.
\indexdefn{aspect ratio}%
\definition{aspect ratio}{\iotwod.defns.aspectratio}
ratio of the width to the height of a rectangular area

\indexdefn{gradient stop}%
\definition{gradient stop}{\iotwod.defns.gradientstop}
point at which a color gradient changes from one color to the next

\indexdefn{visual data}%
\definition{visual data}{\iotwod.defns.visdata}
data representing color, transparency, some other quality, or some combination thereof in a possibly bounded Euclidean plane \begin{note}
The data can vary throughout the plane.
\end{note}

\indexdefn{visual data element}%
\definition{visual data element}{\iotwod.defns.visdataelem}
visual data at a specific point

\indexdefn{channel}%
\definition{channel}{\iotwod.defns.channel}
component of visual data representing color, transparency, or some other quality

\indexdefn{color channel}%
\definition{color channel}{\iotwod.defns.colorchannel}
channel that only represents color

\indexdefn{alpha channel}%
\definition{alpha channel}{\iotwod.defns.alphachannel}
channel that only represents transparency

\indexdefn{visual data format}%
\definition{visual data format}{\iotwod.defns.visdatafmt}
set of all information necessary to transform visual data stored in a particular form into data in a specified, well-defined color model

\indexdefn{premultiplied format}%
\definition{premultiplied format}{\iotwod.defns.premultipliedformat}
visual data format with one or more color channels and an alpha channel where each color channel is normalized and then multiplied by the normalized alpha channel value
\begin{example}
Given the 32-bit non-premultiplied RGBA pixel with 8 bits per channel \{255, 0, 0, 127\} (half-transparent red), when normalized it would become \{1.0f, 0.0f, 0.0f, 0.5f\}. As such, in premultiplied, normalized format it would become \{0.5f, 0.0f, 0.0f, 0.5f\} as a result of multiplying each of the three color channels by the alpha channel value.
\end{example}

% The following may not be needed. Already defined in IEC 60050 723-05-31
\indexdefn{pixel}%
\definition{pixel}{\iotwod.defns.pixel}
discrete visual data element with a particular visual data format

\indexdefn{raster graphics data}%
\definition{raster graphics data}{\iotwod.defns.rastergfxdata}
data comprised of a rectangular array of pixels where the top-left pixel is located at the origin and additional pixels are located at consecutive integral points of increasing value

\indexdefn{vector graphics data}%
\definition{vector graphics data}{\iotwod.defns.vectorgfxdata}
data comprised of zero or more paths together with a sequence of rendering and composing operations and graphics state data that produces continuous visual data

\indexdefn{color model}%
\definition{color model}{\iotwod.defns.colormodel}
ideal, mathematical representation of color

\indexdefn{additive color}%
\definition{additive color}{\iotwod.defns.additivecolor}
color defined by the emissive intensity of its color channels

\indexdefn{color model}%
\indexdefn{color model!RGB}%
\definition{RGB color model}{\iotwod.defns.rgbcolormodel}
color model using additive color comprised of red, green, and blue color channels

\indexdefn{color model}%
\indexdefn{color model!RGBA}%
\definition{RGBA color model}{\iotwod.defns.rgbacolormodel}
RGB color model with an alpha channel

\indexdefn{color space}%
\definition{color space}{\iotwod.defns.colorspace}
systematic mapping of values to colorimetric colors

\indexdefn{color space}%
\indexdefn{color space!sRGB}%
\definition{sRGB color space}{\iotwod.defns.srgbcolorspace}
color space defined in IEC 61966-2-1 that is based on the RGB color model

\indexdefn{filter}%
\definition{filter}{\iotwod.defns.filter}
mathematical function that determines a visual data element

\indexdefn{composition algorithm}%
\definition{composition algorithm}{\iotwod.defns.compositionalgorithm}
algorithm that combines source visual data element and a destination visual data element producing a visual data element that has the same visual data format as the destination visual data element

\indexdefn{compose}%
\definition{compose}{\iotwod.defns.compose}
combine part or all of source visual data with destination visual data in the manner specified by a composition algorithm

\indexdefn{composing operation}%
\definition{composing operation}{\iotwod.defns.composingoperation}
operation that performs composing

\indexdefn{sample}%
\definition{sample}{\iotwod.defns.sample}
use a filter to obtain the visual data of a given point within visual data

\indexdefn{artifact}%
\definition{artifact}{\iotwod.defns.artifact}
error in the results of the application of a composing operation when sampling from raster graphics data due to its visual data being discrete rather than continuous

\indexdefn{aliasing}%
\definition{aliasing}{\iotwod.defns.alias}
presence of artifacts in the results of composing

\indexdefn{anti-aliasing}%
\definition{anti-aliasing}{\iotwod.defns.antialias}
application of a function or algorithm while composing to reduce aliasing
\begin{note}
Certain algorithms can produce ``better'' results, i.e. results with fewer artifacts or with less pronounced artifacts, when rendering text with anti-aliasing due to the nature of text rendering. As such, it often makes sense to provide the ability to choose one type of anti-aliasing for text rendering and another for all other rendering and to provide different sets of anti-aliasing types to choose from for each of the two operations.
\end{note}

%
% path/figure definitions
%

\indexdefn{start point}%
\definition{start point}{\iotwod.defns.startpt}
point that begins a segment

\indexdefn{end point}%
\definition{end point}{\iotwod.defns.endpt}
point that ends a segment

\indexdefn{control point}%
\definition{control point}{\iotwod.defns.controlpt}
point, other than the start point and the end point, that is used in defining a curve

\indexdefn{\bezierlocal curve}%
\indexdefn{\bezierlocal curve!quadratic}%
\definition{\bezierlocal curve}{\iotwod.defns.bezier.quadratic}
\defncontext{quadratic} curve defined by the 
equation $f(t) = (1 - t)^{2} \times P_{0} + 2 \times t \times (1 - t) 
\times P_{1} + t^{2} \times t \times P_{2}$ where t is in the range \crange{0}{1}, $P_{0}$ is the start point, $P_{1}$ is the control point, and $P_{2}$ is end point

\indexdefn{\bezierlocal curve}
\indexdefn{\bezierlocal curve!cubic}
\definition{\bezierlocal curve}{\iotwod.defns.bezier.cubic}
\defncontext{cubic} curve defined by the 
equation $f(t) = (1 - t)^{3} \times P_{0} + 3 \times t \times (1 - t)^{2} 
\times P_{1} + 3 \times t^{2} \times (1 - t) \times P_{2} + t^{3} \times t 
\times P_{3}$ where t is in the range \crange{0}{1}, $P_{0}$ is the start point, $P_{1}$ is the first control point, $P_{2}$ is the second control point, and $P_{3}$ is the end point

\indexdefn{segment}%
\definition{segment}{\iotwod.defns.seg}
line, \bezierlocal curve, or arc

\indexdefn{initial segment}%
\definition{initial segment}{\iotwod.defns.initialseg}
segment in a figure whose start point is not defined as being the end point of another segment in the figure
\begin{note}
It is possible for the initial segment and final segment to be the same segment.
\end{note}

\indexdefn{new figure point}%
\definition{new figure point}{\iotwod.defns.newfigpt}
point that is the start point of the initial segment

\indexdefn{final segment}%
\definition{final segment}{\iotwod.defns.finalseg}
segment in a figure whose end point does not define the start point of any other segment
\begin{note}
It is possible for the initial segment and final segment to be the same segment.
\end{note}

\indexdefn{current point}%
\definition{current point}{\iotwod.defns.currentpt}
point used as the start point of a segment

\indexdefn{open figure}%
\definition{open figure}{\iotwod.defns.openfigure}
figure with one or more segments where the new figure point is not used to define the end point of the figure's final segment
\begin{note}
Even if the start point of the initial segment and the end point of the final segment are assigned the same coordinates, the figure is still an open figure. This is because the final segment's end point is not defined as being the new figure point but instead merely happens to have the same value as that point.
\end{note}

\indexdefn{closed figure}%
\definition{closed figure}{\iotwod.defns.closedfigure}
figure with one or more segments where the new figure point is used to define the end point of the figure's final segment

\indexdefn{degenerate segment}%
\definition{degenerate segment}{\iotwod.defns.degenerateseg}
segment that has the same values for its start point, end point, and, if any, control points

\indexdefn{command}%
\definition{command}{\iotwod.defns.command.closefig}
\defncontext{close figure command} instruction that creates a line segment with a start point of current point and an end point of new figure point

\indexdefn{command}%
\definition{command}{\iotwod.defns.command.newfig}
\defncontext{new figure command} an instruction that creates a new path

\indexdefn{figure item}%
\definition{figure item}{\iotwod.defns.figitem}
segment, new figure command, close figure command, or path command

\indexdefn{figure}%
\definition{figure}{\iotwod.defns.figure}
collection of figure items where the end point of each segment in the collection, except the final segment, defines the start point of exactly one other segment in the collection

\indexdefn{path}%
\definition{path}{\iotwod.defns.path}
collection of figures

\indexdefn{path transformation matrix}%
\definition{path transformation matrix}{\iotwod.defns.pathtransform}
affine transformation matrix used to apply affine transformations to the points in a path

\indexdefn{path command}%
\definition{path command}{\iotwod.defns.pathcommand}
instruction that modifies the path transformation matrix

\indexdefn{degenerate figure}%
\definition{degenerate figure}{\iotwod.defns.degenfigure}
figure containing a new figure command, zero or more degenerate segments, zero or more path commands, and, optionally, a close figure command

\indexdefn{graphics state data}%
\definition{graphics state data}{\iotwod.defns.graphicsstatedata}
data which specify how some part of the process of rendering, or of a composing operation, is performed in part or in whole

\indexdefn{render}%
\definition{render}{\iotwod.defns.render}
transform a path into visual data in the manner specified by a set of graphics state data

\indexdefn{rendering operation}%
\definition{rendering operation}{\iotwod.defns.renderingoperation}
operation that performs rendering

\indexdefn{rendering and composing operation}%
\definition{rendering and composing operation}{\iotwod.defns.renderingandcomposingop}
operation that is either a composing operation, or a rendering operation followed by a composing operation

%%!TEX root = io2d.tex

\indexdefn{path segment}
\definition{path segment}{\iotwod.general.defns.pathsegment}
a line, \bezierlocal curve, or arc, each of which has a start point and an end point

\indexdefn{control point}
\definition{control point}{\iotwod.general.defns.controlpoint}
a point other than the start point and end point that is used in defining a \bezierlocal curve

\indexdefn{degenerate path segment}
\definition{degenerate path segment}{\iotwod.general.defns.degeneratepathsegment}
a path segment that has the same values for its start point, end point, and, if any, control points

\indexdefn{initial path segment}
\definition{initial path segment}{\iotwod.general.defns.initialpathsegment}
a path segment whose start point is not defined as being the end point of another path segment
\enternote
It is possible for the initial path segment and final path segment to be the same path segment.
\exitnote

\indexdefn{final path segment}
\definition{final path segment}{\iotwod.general.defns.finalpathsegment}
a path segment whose end point shall not be used to define the start point of any other path segment
\enternote
It is possible for the initial path segment and final path segment to be the same path segment.
\exitnote

\indexdefn{path instruction}
\definition{path instruction}{\iotwod.general.defns.pathinstruction}
an instruction that creates a new path, closes an existing path, or modifies the interpretation of path segments that follow it

\indexdefn{path}
\definition{path}{\iotwod.general.defns.path}
a collection of path instructions and path segments where the end point of each path segment, except the final path segment, defines the start point of exactly one other path segment in the collection

\indexdefn{current point}
\definition{current point}{\iotwod.general.defns.currentpoint}
a point established by various operations used in creating a path
\enternote
A new path has no current point except as otherwise specified.
\exitnote

\indexdefn{last-move-to point}
\definition{last-move-to point}{\iotwod.general.defns.lastmovetopoint}
the point in a path that is the start point of the initial path segment

\indexdefn{path group}
\definition{path group}{\iotwod.general.defns.pathgroup}
a collection of paths

\indexdefn{closed path}
\definition{closed path}{\iotwod.general.defns.closedpath}
a path with one or more path segments where the last-move-to point is used to define the end point of the path's final path segment

\indexdefn{open path}
\definition{open path}{\iotwod.general.defns.openpath}
a path with one or more path segments where the last-move-to point is not used to define the end point of the path's final path segment
\enternote
Even if the start point of the initial path segment and the end point of the final path segment are assigned the same coordinates, the path is still an open path since the final path segment's end point is not defined as being the start point of the initial segment but instead merely happens to have the same value as that point.
\exitnote

\indexdefn{degenerate path}
\definition{degenerate path}{\iotwod.general.defns.degeneratepath}
a path with only one path segment
\enternote
The path segment is not required to be a degenerate path segment.
\exitnote


%%!TEX root = io2d.tex

\indexdefn{alignment line}
\definition{alignment line}{\iotwod.general.defns.alignmentline}
imaginary line to which most glyph images of a font seem to align \\
\lbrack SOURCE: ISO/IEC 9541-1:2012, definition 3.1 \rbrack

\indexdefn{current position}
\definition{current position}{\iotwod.general.defns.currentposition}
a point on a graphics data graphics resource at which the next glyph representation is to be rendered

\indexdefn{design size}
\definition{design size}{\iotwod.general.defns.designsize}
absolute size at which a font is designed to be used \\
\lbrack SOURCE: ISO/IEC 9541-1:2012, definition 3.3 \rbrack

\indexdefn{escapement}
\definition{escapement}{\iotwod.general.defns.escapement}
movement of the current position on the presentation surface after a glyph representation is rendered

\indexdefn{escapement point}
\definition{escapement point}{\iotwod.general.defns.escapementpoint}
a glyph metric; a point in the glyph's standard coordinate system, to which the current position on the graphics data graphics resource is usually translated, after the glyph representation is rendered

\indexdefn{font}
\definition{font}{\iotwod.general.defns.font}
a collection of glyph images having the same basic design, e.g., \textit{Courier Bold Oblique} \\
\lbrack SOURCE: ISO/IEC 9541-1:2012, definition 3.6 \rbrack

\indexdefn{font family}
\definition{font family}{\iotwod.general.defns.fontfamily}
a collection of fonts of common design, e.g., \textit{Courier, Courier Bold, Courier Bold Oblique} \\
\lbrack SOURCE: ISO/IEC 9541-1:2012, definition 3.7 \rbrack

\indexdefn{font metrics}
\definition{font metrics}{\iotwod.general.defns.fontmetrics}
the set of dimensions and positioning information in a font resource common to all glyph representations contained in that font resource \\
\lbrack SOURCE: ISO/IEC 9541-1:2012, definition 3.8 \rbrack

%\indexdefn{font reference}
%\definition{font reference}{\iotwod.general.defns.fontreference}
%the information about a font resource in an electronic document representation, and possible procedures and operations on that information, which identify or describe the desired font \\
%\lbrack SOURCE: ISO/IEC 9541-1:2012, definition 3.9 \rbrack
%
\indexdefn{font resource}
\definition{font resource}{\iotwod.general.defns.fontresource}
a collection of glyph representations together with descriptive and font metric information which are relevant to the collection of glyph representations as a whole \\
\lbrack SOURCE: ISO/IEC 9541-1:2012, definition 3.10 \rbrack

\indexdefn{font size}
\definition{font size}{\iotwod.general.defns.fontsize}
a scalar reference size relative to which most font metrics, glyph shapes and glyph metrics are specified \\
\lbrack SOURCE: ISO/IEC 9541-1:2012, definition 3.11 \rbrack

\indexdefn{glyph}
\definition{glyph}{\iotwod.general.defns.glyph}
a recognizable abstract graphic symbol which is independent of any specific design \\
\lbrack SOURCE: ISO/IEC 9541-1:2012, definition 3.12 \rbrack

\indexdefn{glyph collection}
\definition{glyph collection}{\iotwod.general.defns.glyphcollection}
an identified set of glyphs \\
\lbrack SOURCE: ISO/IEC 9541-1:2012, definition 3.13 \rbrack

\indexdefn{glyph image}
\definition{glyph image}{\iotwod.general.defns.glyphimage}
an image of a glyph, as obtained from a glyph representation rendered and composed to a graphics data graphics resource

\indexdefn{glyph metrics}
\definition{glyph metrics}{\iotwod.general.defns.glyphmetrics}
the set of information in a glyph representation used for defining the dimensions and positioning of the glyph shape \\
\lbrack SOURCE: ISO/IEC 9541-1:2012, definition 3.16 \rbrack

\indexdefn{glyph representation}
\definition{glyph representation}{\iotwod.general.defns.glyphrepresentation}
the glyph shape and glyph metrics associated with a specific glyph in a font resource \\
\lbrack SOURCE: ISO/IEC 9541-1:2012, definition 3.17 \rbrack

\indexdefn{glyph shape}
\definition{glyph shape}{\iotwod.general.defns.glyphshape}
the set of information in a glyph representation used for defining the shape which represents the glyph \\
\lbrack SOURCE: ISO/IEC 9541-1:2012, definition 3.18 \rbrack

\indexdefn{kern}
\definition{kern}{\iotwod.general.defns.kern}
the extension of a glyph shape beyond its position point or escapement point \\
\lbrack SOURCE: ISO/IEC 9541-1:2012, definition 3.19 \rbrack

\indexdefn{position point}
\definition{position point}{\iotwod.general.defns.positionpoint}
a glyph metric; a point in the glyph's standard coordinate system, usually translated to the current position on the graphics data graphics resource before the glyph shape is rendered

\indexdefn{posture}
\definition{posture}{\iotwod.general.defns.posture}
the extent to which the shape of a glyph or set of glyphs appears to incline, including any consequent design or form change \\
\lbrack SOURCE: ISO/IEC 9541-1:2012, definition 3.22 \rbrack

\indexdefn{proportionate width}
\definition{proportionate width}{\iotwod.general.defns.proportionatewidth}
the ratio of a glyph's or set of glyphs' escapement to font height \\
\lbrack SOURCE: ISO/IEC 9541-1:2012, definition 3.24 \rbrack

\indexdefn{stem}
\definition{stem}{\iotwod.general.defns.stem}
the major stroke of a glyph shape \\
\lbrack SOURCE: ISO/IEC 9541-1:2012, definition 3.25 \rbrack

\indexdefn{weight}
\definition{weight}{\iotwod.general.defns.weight}
the ratio of a glyph's or set of glyphs' stem width to font height \\
\lbrack SOURCE: ISO/IEC 9541-1:2012, definition 3.26 \rbrack

\indexdefn{writing mode}
\definition{writing mode}{\iotwod.general.defns.writingmode}
an identified mode for setting of text in a writing system, usually corresponding to a nominal escapement direction of the glyphs in that mode, i.e., left-to-right, right-to-left or top-to-bottom \\
\lbrack SOURCE: ISO/IEC 9541-1:2012, definition 3.27 \rbrack

\indexdefn{body size}
\definition{body size}{\iotwod.general.defns.bodysize}
the font size, measured along the y axis of the glyph's standard coordinate system

\indexdefn{wrap_modeed body size}
\definition{wrap_modeed body size}{\iotwod.general.defns.wrap_modeedbodysize}
a reference size with two components, measured respectively along the \xaxis and \yaxis of the glyph's standard coordinate system

\indexdefn{design frame}
\definition{design frame}{\iotwod.general.defns.designframe}
dimensional expression that specifies the area inside which a set of glyph images can be designed \\
\lbrack SOURCE: ISO/IEC 9541-1:2012, definition 3.30 \rbrack

\indexdefn{bounding box}
\definition{bounding box}{\iotwod.general.defns.boundingbox}
dimensional expression to specify an actual area that a glyph image occupies within a design frame \\
\lbrack SOURCE: ISO/IEC 9541-1:2012, definition 3.31 \rbrack
%
%\indexdefn{blackness}
%\definition{blackness}{\iotwod.general.defns.blackness}
%the ratio of the blackened area of a glyph image to the wrap_modeed body size area of the glyph image \\
%\lbrack SOURCE: ISO/IEC 9541-1:2012, definition 3.32 \rbrack

%
% end definitions
\indextext{definitions|)}

