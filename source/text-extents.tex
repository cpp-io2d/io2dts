%!TEX root = io2d.tex
\rSec0 [textextents] {Class \tcode{text_extents}}

\rSec1 [textextents.synopsis] {\tcode{text_extents} synopsis}

\begin{codeblock}
namespace std { namespace experimental { namespace io2d { inline namespace v1 {
  class text_extents {
  public:
    // \ref{textextents.cons}, construct/copy/move/destroy:
    text_extents() noexcept;
    text_extents(const text_extents& other) noexcept;
    text_extents& operator=(const text_extents& other) noexcept;
    text_extents(font_extents&& other) noexcept;
    text_extents& operator=(font_extents&& other) noexcept;
    text_extents(double xBearing, double yBearing, double width,
      double height, double xAdvance, double yAdvance) noexcept;

    // \ref{textextents.modifiers}, modifiers:
    void x_bearing(double value) noexcept;
    void y_bearing(double value) noexcept;
    void width(double value) noexcept;
    void height(double value) noexcept;
    void x_advance(double value) noexcept;
    void y_advance(double value) noexcept;

    // \ref{textextents.observers}, observers:
    double x_bearing() const noexcept;
    double y_bearing() const noexcept;
    double width() const noexcept;
    double height() const noexcept;
    double x_advance() const noexcept;
    double y_advance() const noexcept;

  private:
    double _X_bear; // \expos
    double _Y_bear; // \expos
    double _Width;  // \expos
    double _Height; // \expos
    double _X_adv;  // \expos
    double _Y_adv;  // \expos
  };
} } } }
\end{codeblock}

\rSec1 [textextents.intro] {\tcode{text_extents} Description}

\pnum
\indexlibrary{\idxcode{text_extents}}
The class \tcode{text_extents} describes extents for a string.

\pnum
It is used by a \tcode{surface} object to report the extents of a string in the \tcode{surface} object's untransformed coordinate space units if the string were rendered with the currently selected font.

\pnum
\enternote
This object's observable values can be manipulated by library users for their convenience. But since the \tcode{text_extents} object returned by \tcode{surface::text_extents()} is not a reference or a pointer, the changes do not reflect back to the surface or its current font.
\exitnote

\rSec1 [textextents.cons] {\tcode{text_extents} constructors and assignment operators}

\indexlibrary{\idxcode{text_extents}!constructor}
\begin{itemdecl}
    text_extents() noexcept;
\end{itemdecl}
\begin{itemdescr}
	\pnum
	\effects
	Constructs an object of type \tcode{text_extents}.
	
	\pnum
	\postconditions
    \tcode{_X_bear == 0.0}.
    
    \tcode{_Y_bear == 0.0}.
    
    \tcode{_Width == 0.0}.
    
    \tcode{_Height == 0.0}.
    
    \tcode{_X_adv == 0.0}.
    
    \tcode{_Y_adv == 0.0}.

\end{itemdescr}

\indexlibrary{\idxcode{text_extents}!constructor}
\begin{itemdecl}
    text_extents(double xBearing, double yBearing, double width,
      double height, double xAdvance, double yAdvance) noexcept;
\end{itemdecl}
\begin{itemdescr}
	\pnum
	\effects
	Constructs an object of type \tcode{text_extents}.
	
	\pnum
	\postconditions
    \tcode{_X_bear == xBearing}.
    
    \tcode{_Y_bear == yBearing}.
    
    \tcode{_Width == width}.
    
    \tcode{_Height == height}.
    
    \tcode{_X_adv == xAdvance}.
    
    \tcode{_Y_adv == yAdvance}.

\end{itemdescr}

\rSec1 [textextents.modifiers]{\tcode{text_extents} modifiers}

\indexlibrary{\idxcode{text_extents}!\idxcode{x_bearing}}
\indexlibrary{\idxcode{x_bearing}!\idxcode{text_extents}}
\begin{itemdecl}
    void x_bearing(double value) noexcept;
\end{itemdecl}
\begin{itemdescr}
	\pnum
	\postconditions
	\tcode{_X_bear == value}.
\end{itemdescr}

\indexlibrary{\idxcode{text_extents}!\idxcode{y_bearing}}
\indexlibrary{\idxcode{y_bearing}!\idxcode{text_extents}}
\begin{itemdecl}
    void y_bearing(double value) noexcept;
\end{itemdecl}
\begin{itemdescr}
	\pnum
	\postconditions
	\tcode{_Y_bear == value}.
	
\end{itemdescr}

\indexlibrary{\idxcode{text_extents}!\idxcode{width}}
\indexlibrary{\idxcode{width}!\idxcode{text_extents}}
\begin{itemdecl}
    void width(double value) noexcept;
\end{itemdecl}
\begin{itemdescr}
	\pnum
	\postconditions
	\tcode{_Width == value}.
	
\end{itemdescr}
	
\indexlibrary{\idxcode{text_extents}!\idxcode{height}}
\indexlibrary{\idxcode{height}!\idxcode{text_extents}}
\begin{itemdecl}
    void height(double value) noexcept;
\end{itemdecl}
\begin{itemdescr}
	\pnum
	\postconditions
	\tcode{_Height == value}.
	
\end{itemdescr}
	
\indexlibrary{\idxcode{text_extents}!\idxcode{x_advance}}
\indexlibrary{\idxcode{x_advance}!\idxcode{text_extents}}
\begin{itemdecl}
    void x_advance(double value) noexcept;
\end{itemdecl}
\begin{itemdescr}
	\pnum
	\postconditions
	\tcode{_X_adv == value}.
	
\end{itemdescr}
	
\indexlibrary{\idxcode{text_extents}!\idxcode{y_advance}}
\indexlibrary{\idxcode{y_advance}!\idxcode{text_extents}}
\begin{itemdecl}
    void y_advance(double value) noexcept;
\end{itemdecl}
\begin{itemdescr}
	\pnum
	\postconditions
	\tcode{_Y_adv == value}.
	
\end{itemdescr}

\rSec1 [textextents.observers]{\tcode{text_extents} observers}

\indexlibrary{\idxcode{text_extents}!\idxcode{x_bearing}}
\indexlibrary{\idxcode{x_bearing}!\idxcode{text_extents}}
\begin{itemdecl}
    double x_bearing() const noexcept;
\end{itemdecl}
\begin{itemdescr}
	\pnum
	\returns
	\tcode{_X_bear}.
	
	\pnum
	\remarks
	This value is the x axis offset of the leftmost visible part of the text as rendered from the x coordinate of the specified position at which to draw the text.
	
	\pnum
	Leading and trailing spaces can affect this value due to the fact that spaces change the position of other text and thus can change the position of the first visible text that is rendered. 

	\pnum
	\enternote
	Because this value is an offset from the specified position and is given in untransformed units, it remains the same regardless of the position at which the text will be drawn.
	
	\pnum
	This value will typically be negative, zero, or slightly positive depending on the font used and the text being rendered (e.g. scripts that are written right-to-left will normally have a negative \tcode{x_bearing()} value).
	\exitnote
\end{itemdescr}

\indexlibrary{\idxcode{text_extents}!\idxcode{y_bearing}}
\indexlibrary{\idxcode{y_bearing}!\idxcode{text_extents}}
\begin{itemdecl}
    double y_bearing() const noexcept;
\end{itemdecl}
\begin{itemdescr}
	\pnum
	\returns
	\tcode{_Y_bear}.
	
	\pnum
	\remarks
	This value is the y axis offset of the topmost visible part of the text as rendered from the y coordinate of the specified position at which to draw the text. 

	\pnum
	\enternote
	This value may range from negative to positive depending on the font origin chosen by the font designer. Usually this value is negative.
	\exitnote
\end{itemdescr}

\indexlibrary{\idxcode{text_extents}!\idxcode{width}}
\indexlibrary{\idxcode{width}!\idxcode{text_extents}}
\begin{itemdecl}
    double width() const noexcept;
\end{itemdecl}
\begin{itemdescr}
	\pnum
	\returns
	\tcode{_Width}.
	
	\pnum
	\remarks
	This value is the width of the text as rendered from its leftmost visible part to its rightmost visible part.
	
	\pnum
	This value may include a de minimus amount of whitespace, e.g. 1 to 2 pixels when pixels are the coordinate space unit. This allowance is meant to cover discrepancies between expected rendering results and actual results which can arise due to techniques such as font hinting, antialiasing, and subpixel rendering.

\end{itemdescr}

\indexlibrary{\idxcode{text_extents}!\idxcode{height}}
\indexlibrary{\idxcode{height}!\idxcode{text_extents}}
\begin{itemdecl}
    double height() const noexcept;
\end{itemdecl}
\begin{itemdescr}
	\pnum
	\returns
	\tcode{_Height}.
	
	\pnum
	\remarks
	This value is the height of the text as rendered from its topmost visible part to its bottommost visible part.
	
	\pnum
	This value may include a de minimus amount of whitespace, e.g. 1 to 2 pixels when pixels are the coordinate space unit. This allowance is meant to cover discrepancies between expected rendering results and actual results which can arise due to techniques such as font hinting, antialiasing, and subpixel rendering.

\end{itemdescr}

\indexlibrary{\idxcode{text_extents}!\idxcode{x_advance}}
\indexlibrary{\idxcode{x_advance}!\idxcode{text_extents}}
\begin{itemdecl}
    double x_advance() const noexcept;
\end{itemdecl}
\begin{itemdescr}
	\pnum
	\returns
	\tcode{_X_adv}.
	
	\pnum
	\remarks
	This value is amount to add to the x coordinate of the original specified position in order to properly draw text that will immediately follow this text on the same line.
	
	\pnum
	In vertically oriented text, this value will typically be zero.

\end{itemdescr}

\indexlibrary{\idxcode{text_extents}!\idxcode{y_advance}}
\indexlibrary{\idxcode{y_advance}!\idxcode{text_extents}}
\begin{itemdecl}
    double y_advance() const noexcept;
\end{itemdecl}
\begin{itemdescr}
	\pnum
	\returns
	\tcode{_Y_adv}.
	
	\pnum
	\remarks
	This value is amount to add to the y coordinate of the original specified position in order to properly draw text that will immediately follow this text on the same line.
	
	\pnum
	In horizontally oriented text, this value will typically be zero.

\end{itemdescr}
