%!TEX root = io2d.tex
\rSec0 [\iotwod.\vectortwod] {Class \tcode{vector_2d}}

\rSec1 [\iotwod.\vectortwod.synopsis] {\tcode{vector_2d} synopsis}

\begin{codeblock}
namespace std { namespace experimental { namespace io2d { inline namespace v1 {
  class vector_2d {
  public:
    // \ref{\iotwod.\vectortwod.cons}, construct/copy/move/destroy:
    vector_2d() noexcept;
    vector_2d(double x, double y) noexcept;
    vector_2d(const vector_2d&) noexcept;
    vector_2d& operator=(const vector_2d&) noexcept;
    vector_2d(vector_2d&&) noexcept;
    vector_2d& operator=(vector_2d&&) noexcept;

    // \ref{\iotwod.\vectortwod.modifiers}, modifiers:
    void x(double value) noexcept;
    void y(double value) noexcept;
    
    // \ref{\iotwod.\vectortwod.observers}, observers:
    double x() const noexcept;
    double y() const noexcept;
    double length() const noexcept;
    double dot(const vector_2d& other) const noexcept;
    vector_2d to_unit() const noexcept;
    
    // \ref{\iotwod.\vectortwod.member.ops}, member operators:
    vector_2d& operator+=(const vector_2d& rhs) noexcept;
    vector_2d& operator-=(const vector_2d& rhs) noexcept;
    vector_2d& operator*=(double rhs) noexcept;
    
  private:
    double _X; // \expos
    double _Y; // \expos
  };
  
  // \ref{\iotwod.\vectortwod.ops}, non-member operators:
  bool operator==(const vector_2d& lhs, const vector_2d& rhs) noexcept;
  bool operator!=(const vector_2d& lhs, const vector_2d& rhs) noexcept;
  vector_2d operator+(const vector_2d& lhs) noexcept;
  vector_2d operator+(const vector_2d& lhs, const vector_2d& rhs) noexcept;
  vector_2d operator-(const vector_2d& lhs) noexcept;
  vector_2d operator-(const vector_2d& lhs, const vector_2d& rhs) noexcept;
  vector_2d operator*(const vector_2d& lhs, double rhs) noexcept;
  vector_2d operator*(double lhs, const vector_2d& rhs) noexcept;
} } } }
\end{codeblock}

\rSec1 [\iotwod.\vectortwod.intro] {\tcode{vector_2d} Description}

\pnum
\indexlibrary{\idxcode{vector_2d}}
The class \tcode{vector_2d} describes an object that stores a two-dimensional Euclidean vector.

\rSec1 [\iotwod.\vectortwod.cons] {\tcode{vector_2d} constructors and assignment operators}

\indexlibrary{\idxcode{vector_2d}!constructor}
\begin{itemdecl}
vector_2d() noexcept;
\end{itemdecl}
\begin{itemdescr}
	\pnum
	\effects
	Constructs an object of type \tcode{vector_2d}.
	
	\pnum
	\postconditions
	\tcode{_X == 0.0 \&\& _Y == 0.0}.
\end{itemdescr}

\indexlibrary{\idxcode{vector_2d}!constructor}
\begin{itemdecl}
vector_2d(double x, double y) noexcept;
\end{itemdecl}
\begin{itemdescr}
	\pnum
	\effects
	Constructs an object of type \tcode{vector_2d}.
	
	\pnum
	\postconditions
	\tcode{_X == x \&\& _Y == y}.
\end{itemdescr}
	
\rSec1 [\iotwod.\vectortwod.modifiers]{\tcode{vector_2d} modifiers}

\indexlibrary{\idxcode{vector_2d}!\idxcode{x}}
\indexlibrary{\idxcode{x}!\idxcode{vector_2d}}
\begin{itemdecl}
void x(double value) noexcept;
\end{itemdecl}
\begin{itemdescr}
	\pnum
	\postconditions
	\tcode{_X == value}.
	
\end{itemdescr}

\indexlibrary{\idxcode{vector_2d}!\idxcode{y}}
\indexlibrary{\idxcode{y}!\idxcode{vector_2d}}
\begin{itemdecl}
    void y(double value) noexcept;
\end{itemdecl}
\begin{itemdescr}
	\pnum
	\postconditions
	\tcode{_Y == value}.
	
\end{itemdescr}

\rSec1 [\iotwod.\vectortwod.observers]{\tcode{vector_2d} observers}

\indexlibrary{\idxcode{vector_2d}!\idxcode{x}}
\indexlibrary{\idxcode{x}!\idxcode{vector_2d}}
\begin{itemdecl}
    double x() const noexcept;
\end{itemdecl}
\begin{itemdescr}
	\pnum
	\returns
	\tcode{_X}.
\end{itemdescr}

\indexlibrary{\idxcode{vector_2d}!\idxcode{y}}
\indexlibrary{\idxcode{y}!\idxcode{vector_2d}}
\begin{itemdecl}
    double y() const noexcept;
\end{itemdecl}
\begin{itemdescr}
	\pnum
	\returns
	\tcode{_Y}.
\end{itemdescr}

\indexlibrary{\idxcode{vector_2d}!\idxcode{length}}
\indexlibrary{\idxcode{length}!\idxcode{vector_2d}}
\begin{itemdecl}
    double length() const noexcept;
\end{itemdecl}
\begin{itemdescr}
	\pnum
	\returns
	\tcode{sqrt(_X * _X + _Y * _Y)}.
\end{itemdescr}

\indexlibrary{\idxcode{vector_2d}!\idxcode{dot}}
\indexlibrary{\idxcode{dot}!\idxcode{vector_2d}}
\begin{itemdecl}
    double dot(const vector_2d& other) const noexcept;
\end{itemdecl}
\begin{itemdescr}
	\pnum
	\returns
	\tcode{_X * other._X + _Y * other._Y}.
\end{itemdescr}

\indexlibrary{\idxcode{vector_2d}!\idxcode{to_unit}}
\indexlibrary{\idxcode{to_unit}!\idxcode{vector_2d}}
\begin{itemdecl}
    vector_2d to_unit() const noexcept;
\end{itemdecl}
\begin{itemdescr}
	\pnum
	\returns
	\tcode{vector_2d\{ _X / length(), _Y / length()\}}.
\end{itemdescr}

\rSec1 [\iotwod.\vectortwod.member.ops] {\tcode{vector_2d} member operators}

\indexlibrary{\idxcode{vector_2d}!\idxcode{operator+=}}
\indexlibrary{\idxcode{operator+=}!\idxcode{vector_2d}}
\begin{itemdecl}
	vector_2d& operator+=(const vector_2d& rhs) noexcept;
\end{itemdecl}
\begin{itemdescr}
	\pnum
	\effects
	\tcode{*this = *this + rhs}.
	
	\pnum
	\returns
	\tcode{*this}.
\end{itemdescr}

\indexlibrary{\idxcode{vector_2d}!\idxcode{operator-=}}
\indexlibrary{\idxcode{operator-=}!\idxcode{vector_2d}}
\begin{itemdecl}
	vector_2d& operator-=(const vector_2d& rhs) noexcept;
\end{itemdecl}
\begin{itemdescr}
	\pnum
	\effects
	\tcode{*this = *this - rhs}.
	
	\pnum
	\returns
	\tcode{*this}.
\end{itemdescr}

\indexlibrary{\idxcode{vector_2d}!\idxcode{operator*=}}
\indexlibrary{\idxcode{operator*=}!\idxcode{vector_2d}}
\begin{itemdecl}
	vector_2d& operator*=(double rhs) noexcept;
\end{itemdecl}
\begin{itemdescr}
	\pnum
	\effects
	\tcode{*this = *this * rhs}.
	
	\pnum
	\returns
	\tcode{*this}.
\end{itemdescr}

\rSec1 [\iotwod.\vectortwod.ops] {\tcode{vector_2d} non-member operators}

\indexlibrary{\idxcode{vector_2d}!\idxcode{operator==}}
\indexlibrary{\idxcode{operator==}!\idxcode{vector_2d}}
\begin{itemdecl}
	bool operator==(const vector_2d& lhs, const vector_2d& rhs) noexcept;
\end{itemdecl}
\begin{itemdescr}
	\pnum
	\returns
	\tcode{lhs.x() == rhs.x() \&\& lhs.y() == rhs.y()}.
\end{itemdescr}

\indexlibrary{\idxcode{vector_2d}!\idxcode{operator!=}}
\indexlibrary{\idxcode{operator!=}!\idxcode{vector_2d}}
\begin{itemdecl}
	bool operator!=(const vector_2d& lhs, const vector_2d& rhs) noexcept;
\end{itemdecl}
\begin{itemdescr}
	\pnum
	\returns
	\tcode{!(lhs == rhs)}.
\end{itemdescr}

\indexlibrary{\idxcode{vector_2d}!\idxcode{operator+}}
\indexlibrary{\idxcode{operator+}!\idxcode{vector_2d}}
\begin{itemdecl}
vector_2d operator+(const vector_2d& lhs) noexcept;
\end{itemdecl}
\begin{itemdescr}
	\pnum
	\returns
	\tcode{vector_2d(lhs)}.
\end{itemdescr}

\indexlibrary{\idxcode{vector_2d}!\idxcode{operator+}}
\indexlibrary{\idxcode{operator+}!\idxcode{vector_2d}}
\begin{itemdecl}
vector_2d operator+(const vector_2d& lhs, const vector_2d& rhs) noexcept;
\end{itemdecl}
\begin{itemdescr}
	\pnum
	\returns
	\tcode{vector_2d\{ lhs.x() + rhs.x(), lhs.y() + rhs.y() \}}.
\end{itemdescr}

\indexlibrary{\idxcode{vector_2d}!\idxcode{operator-}}
\indexlibrary{\idxcode{operator-}!\idxcode{vector_2d}}
\begin{itemdecl}
vector_2d operator-(const vector_2d& lhs) noexcept;
\end{itemdecl}
\begin{itemdescr}
	\pnum
	\returns
	\tcode{vector_2d\{ -lhs.x(), -lhs.y() \}}.
\end{itemdescr}

\indexlibrary{\idxcode{vector_2d}!\idxcode{operator-}}
\indexlibrary{\idxcode{operator-}!\idxcode{vector_2d}}
\begin{itemdecl}
vector_2d operator-(const vector_2d& lhs, const vector_2d& rhs) noexcept;
\end{itemdecl}
\begin{itemdescr}
	\pnum
	\returns
	\tcode{vector_2d\{ lhs.x() - rhs.x(), lhs.y() - rhs.y() \}}.
\end{itemdescr}

\indexlibrary{\idxcode{vector_2d}!\idxcode{operator*}}
\indexlibrary{\idxcode{operator*}!\idxcode{vector_2d}}
\begin{itemdecl}
vector_2d operator*(const vector_2d& lhs, double rhs) noexcept;
\end{itemdecl}
\begin{itemdescr}
	\pnum
	\returns
	\tcode{vector_2d\{ lhs.x() * rhs, lhs.y() * rhs \}}.
\end{itemdescr}

\indexlibrary{\idxcode{vector_2d}!\idxcode{operator*}}
\indexlibrary{\idxcode{operator*}!\idxcode{vector_2d}}
\begin{itemdecl}
vector_2d operator*(double lhs, const vector_2d& rhs) noexcept;
\end{itemdecl}
\begin{itemdescr}
	\pnum
	\returns
	\tcode{vector_2d\{ lhs * rhs.x(), lhs * rhs.y() \}}.
\end{itemdescr}
