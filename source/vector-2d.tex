%!TEX root = io2d.tex
\rSec0 [\iotwod.\pointtwod] {Class \tcode{basic_point_2d}}

\rSec1 [\iotwod.\pointtwod.intro] {\tcode{basic_point_2d} description}

\indexlibrary{\idxcode{basic_point_2d}}%
\pnum
The class template \tcode{basic_point_2d} is used as both a point and as a two-dimensional Euclidean vector.

\pnum
It has an \term{x coordinate} of type \tcode{float} and a \term{y coordinate} of type \tcode{float}.

\pnum
The data are stored in an object of type \tcode{typename GraphicsMath::point_2d_data_type}. The name of the object is \unspec. It is represented below in the expository \tcode{_Data} variable.

\rSec1 [\iotwod.\pointtwod.synopsis] {\tcode{basic_point_2d} synopsis}

\begin{codeblock}
namespace std::experimental::io2d::v1 {
  template <class GraphicsMath>
  class basic_point_2d {
  public:
    using data_type = typename GraphicsMath::point_2d_data_type;

    // \ref{\iotwod.\pointtwod.cons}, constructors:
    basic_point_2d() noexcept;
    basic_point_2d(float x, float y) noexcept;
    basic_point_2d(const typename GraphicsMath::point_2d_data_type& data) noexcept;

    // \ref{\iotwod.\pointtwod.accessors}, accessors:
    const data_type& data() const noexcept;
    data_type& data() noexcept;
    
    // \ref{\iotwod.\pointtwod.modifiers}, modifiers:
    void x(float val) noexcept;
    void y(float val) noexcept;

    // \ref{\iotwod.\pointtwod.observers}, observers:
    float x() const noexcept;
    float y() const noexcept;
    float dot(const basic_point_2d& other) const noexcept;
    float magnitude() const noexcept;
    float magnitude_squared() const noexcept;
    float angular_direction() const noexcept;
    basic_point_2d to_unit() const noexcept;

    // \ref{\iotwod.\pointtwod.member.ops}, member operators:
    basic_point_2d& operator+=(const basic_point_2d& rhs) noexcept;
    basic_point_2d& operator+=(float rhs) noexcept;
    basic_point_2d& operator-=(const basic_point_2d& rhs) noexcept;
    basic_point_2d& operator-=(float rhs) noexcept;
    basic_point_2d& operator*=(const basic_point_2d& rhs) noexcept;
    basic_point_2d& operator*=(float rhs) noexcept;
    basic_point_2d& operator/=(const basic_point_2d& rhs) noexcept;
    basic_point_2d& operator/=(float rhs) noexcept;
  };

  // \ref{\iotwod.\pointtwod.ops}, non-member operators:
  template <class GraphicsMath>
  bool operator==(const basic_point_2d<GraphicsMath>& lhs,
    const basic_point_2d<GraphicsMath>& rhs) noexcept;
  template <class GraphicsMath>
  bool operator!=(const basic_point_2d<GraphicsMath>& lhs,
    const basic_point_2d<GraphicsMath>& rhs) noexcept;
  template <class GraphicsMath>
  basic_point_2d<GraphicsMath> operator+(const basic_point_2d<GraphicsMath>& val) noexcept;
  template <class GraphicsMath>
  basic_point_2d<GraphicsMath> operator+(const basic_point_2d<GraphicsMath>& lhs,
    const basic_point_2d<GraphicsMath>& rhs) noexcept;
  template <class GraphicsMath>
  basic_point_2d<GraphicsMath> operator-(const basic_point_2d<GraphicsMath>& val) noexcept;
  template <class GraphicsMath>
  basic_point_2d<GraphicsMath> operator-(const basic_point_2d<GraphicsMath>& lhs,
    const basic_point_2d<GraphicsMath>& rhs) noexcept;
  template <class GraphicsMath>
  basic_point_2d<GraphicsMath> operator*(const basic_point_2d<GraphicsMath>& lhs,
    float rhs) noexcept;
  template <class GraphicsMath>
  basic_point_2d<GraphicsMath> operator*(float lhs,
    const basic_point_2d<GraphicsMath>& rhs) noexcept;
  template <class GraphicsMath>
  basic_point_2d<GraphicsMath> operator*(const basic_point_2d<GraphicsMath>& lhs,
    const basic_point_2d<GraphicsMath>& rhs) noexcept;
  template <class GraphicsMath>
  basic_point_2d<GraphicsMath> operator/(const basic_point_2d<GraphicsMath>& lhs,
    float rhs) noexcept;
  template <class GraphicsMath>
  basic_point_2d<GraphicsMath> operator/(float lhs,
    const basic_point_2d<GraphicsMath>& rhs) noexcept;
  template <class GraphicsMath>
  basic_point_2d<GraphicsMath> operator/(const basic_point_2d<GraphicsMath>& lhs,
    const basic_point_2d<GraphicsMath>& rhs) noexcept;
}
\end{codeblock}

\rSec1 [\iotwod.\pointtwod.cons] {\tcode{basic_point_2d} constructors}

\indexlibrary{\idxcode{basic_point_2d}!constructor}%
\begin{itemdecl}
basic_point_2d() noexcept;
\end{itemdecl}
\begin{itemdescr}
\pnum
\effects
Constructs an object of type \tcode{basic_point_2d}.

\postconditions
\pnum
\tcode{data() == GraphicsMath::create_point_2d()}.
\end{itemdescr}

\indexlibrary{\idxcode{basic_point_2d}!constructor}%
\begin{itemdecl}
basic_point_2d(float x, float y) noexcept;
\end{itemdecl}
\begin{itemdescr}
\effects
\pnum
Constructs an object of type \tcode{basic_point_2d}.

\postconditions
\pnum
\tcode{data() == GraphicsMath::create_point_2d(x, y)}.
\end{itemdescr}

\indexlibrary{\idxcode{basic_point_2d}!constructor}%
\begin{itemdecl}
basic_point_2d(const typename GraphicsMath::point_2d_data_type& d) noexcept;
\end{itemdecl}
\begin{itemdescr}
\pnum
\effects
Constructs an object of type \tcode{basic_point_2d}.

\pnum
\postconditions
\tcode{data() == d}.
\end{itemdescr}

\rSec1 [\iotwod.\pointtwod.accessors]{\tcode{basic_point_2d} accessors}

\indexlibrarymember{data}{basic_point_2d}%
\begin{itemdecl}
const data_type& data() const noexcept;
\end{itemdecl}
\begin{itemdescr}
\pnum
\returns
Returns a const reference to the data stored within the \tcode{basic_point_2d} object.
\end{itemdescr}

\begin{itemdecl}
data_type& data() noexcept;
\end{itemdecl}
\begin{itemdescr}
\pnum
\returns
Returns a reference to the data stored within the \tcode{basic_point_2d} object.
\end{itemdescr}

\rSec1 [\iotwod.\pointtwod.modifiers]{\tcode{basic_point_2d} modifiers}

\indexlibrarymember{x}{basic_point_2d}%
\begin{itemdecl}
void x(float val) noexcept;
\end{itemdecl}
\begin{itemdescr}
\pnum
\effects
Equivalent to \tcode{GraphicsMath::x(data(), val);}
\end{itemdescr}

\indexlibrarymember{y}{basic_point_2d}%
\begin{itemdecl}
void y(float val) noexcept;
\end{itemdecl}
\begin{itemdescr}
\pnum
\effects
Equivalent to \tcode{GraphicsMath::y(data(), val);}
\end{itemdescr}

\rSec1 [\iotwod.\pointtwod.observers]{\tcode{basic_point_2d} observers}

\indexlibrarymember{x}{basic_point_2d}%
\begin{itemdecl}
float x() const noexcept;
\end{itemdecl}
\begin{itemdescr}
\pnum \returns \tcode{GraphicsMath::x(data())}.
\end{itemdescr}

\indexlibrarymember{y}{basic_point_2d}%
\begin{itemdecl}
float y() const noexcept;
\end{itemdecl}
\begin{itemdescr}
\pnum \returns \tcode{GraphicsMath::y(data())}.
\end{itemdescr}

\indexlibrarymember{dot}{basic_point_2d}%
\begin{itemdecl}
float dot(const basic_point_2d& other) const noexcept;
\end{itemdecl}
\begin{itemdescr}
\pnum \returns \tcode{GraphicsMath::dot(data(), other)}.
\end{itemdescr}

\indexlibrarymember{magnitude}{basic_point_2d}%
\begin{itemdecl}
float magnitude() const noexcept;
\end{itemdecl}
\begin{itemdescr}
\pnum \returns \tcode{GraphicsMath::magnitude(data())}.
\end{itemdescr}

\indexlibrarymember{magnitude_squared}{basic_point_2d}%
\begin{itemdecl}
float magnitude_squared() const noexcept;
\end{itemdecl}
\begin{itemdescr}
\pnum \returns \tcode{GraphicsMath::magnitude_square(data())}.
\end{itemdescr}

\indexlibrarymember{angular_direction}{basic_point_2d}%
\begin{itemdecl}
float angular_direction() const noexcept;
\end{itemdecl}
\begin{itemdescr}
\pnum \returns \tcode{GraphicsMath::angular_direction(data())}.
\end{itemdescr}

\indexlibrarymember{to_unit}{basic_point_2d}%
\begin{itemdecl}
basic_point_2d to_unit() const noexcept;
\end{itemdecl}
\begin{itemdescr}
\pnum \returns \tcode{GraphicsMath::to_unit(data())}.
\end{itemdescr}

\rSec1 [\iotwod.\pointtwod.member.ops] {\tcode{basic_point_2d} member operators}

\indexlibrarymember{operator+=}{basic_point_2d}%
\begin{itemdecl}
basic_point_2d& operator+=(const basic_point_2d& rhs) noexcept;
\end{itemdecl}
\begin{itemdescr}
\pnum \effects Equivalent to \tcode{data() = GraphicsMath::add(data(), rhs.data());}
	
\pnum \returns \tcode{*this}.
\end{itemdescr}

\indexlibrarymember{operator-=}{basic_point_2d}%
\begin{itemdecl}
basic_point_2d& operator-=(const basic_point_2d& rhs) noexcept;
\end{itemdecl}
\begin{itemdescr}
\pnum \effects Equivalent to \tcode{data() = GraphicsMath::subtract(data(), rhs.data());}

\pnum \returns \tcode{*this}.
\end{itemdescr}

\indexlibrarymember{operator*=}{basic_point_2d}%
\begin{itemdecl}
basic_point_2d& operator*=(float rhs) noexcept;
\end{itemdecl}
\begin{itemdescr}
\pnum
\effects
Equivalent to \tcode{data() = GraphicsMath::multiply(data(), rhs);}

\pnum
\returns
\tcode{*this}.
\end{itemdescr}

\indexlibrarymember{operator*=}{basic_point_2d}%
\begin{itemdecl}
basic_point_2d& operator*=(const basic_point_2d& rhs) noexcept;
\end{itemdecl}
\begin{itemdescr}
\pnum
\effects
Equivalent to \tcode{data() = GraphicsMath::multiply(data(), rhs.data());}

\pnum
\returns
\tcode{*this}.
\end{itemdescr}

\indexlibrarymember{operator/=}{basic_point_2d}%
\begin{itemdecl}
basic_point_2d& operator/=(float rhs) noexcept;
\end{itemdecl}
\begin{itemdescr}
\pnum
\effects
Equivalent to: \tcode{data() = GraphicsMath::divide(data(), rhs);}

\pnum
\returns
\tcode{*this}.
\end{itemdescr}

\indexlibrarymember{operator/=}{basic_point_2d}%
\begin{itemdecl}
basic_point_2d& operator/=(const basic_point_2d& rhs) noexcept;
\end{itemdecl}
\begin{itemdescr}
\pnum
\effects
Equivalent to: \tcode{data() = GraphicsMath::divide(data(), rhs.data());}

\pnum
\returns
\tcode{*this}.
\end{itemdescr}

\rSec1 [\iotwod.\pointtwod.ops] {\tcode{basic_point_2d} non-member operators}

\indexlibrarymember{operator==}{basic_point_2d}%
\begin{itemdecl}
bool operator==(const basic_point_2d& lhs, const basic_point_2d& rhs) noexcept;
\end{itemdecl}
\begin{itemdescr}
\pnum \returns \tcode{GraphicsMath::equal(lhs.data(), rhs.data())}.
\end{itemdescr}

\indexlibrarymember{operator!=}{basic_point_2d}%
\begin{itemdecl}
bool operator!=(const basic_point_2d& lhs, const basic_point_2d& rhs) noexcept;
\end{itemdecl}
\begin{itemdescr}
\pnum \returns \tcode{GraphicsMath::not_equal(lhs.data(), rhs.data())}.
\end{itemdescr}

\indexlibrarymember{operator+}{basic_point_2d}%
\begin{itemdecl}
basic_point_2d operator+(const basic_point_2d& val) noexcept;
\end{itemdecl}
\begin{itemdescr}
\pnum
\returns
\tcode{val}.
\end{itemdescr}

\indexlibrarymember{operator+}{basic_point_2d}%
\begin{itemdecl}
basic_point_2d operator+(const basic_point_2d& lhs, const basic_point_2d& rhs) noexcept;
\end{itemdecl}
\begin{itemdescr}
\pnum
\returns
\tcode{GraphicsMath::add(lhs.data(), rhs.data())}.
\end{itemdescr}

\indexlibrarymember{operator-}{basic_point_2d}%
\begin{itemdecl}
basic_point_2d operator-(const basic_point_2d& val) noexcept;
\end{itemdecl}
\begin{itemdescr}
\pnum
\returns
\tcode{GraphicsMath::negate(val.data())}.
\end{itemdescr}

\indexlibrarymember{operator-}{basic_point_2d}%
\begin{itemdecl}
basic_point_2d operator-(const basic_point_2d& lhs, const basic_point_2d& rhs) noexcept;
\end{itemdecl}
\begin{itemdescr}
\pnum
\returns
\tcode{GraphicsMath::subtract(lhs.data(), rhs.data())}.
\end{itemdescr}

\indexlibrarymember{operator*}{basic_point_2d}%
\begin{itemdecl}
basic_point_2d operator*(const basic_point_2d& lhs, const basic_point_2d& rhs) noexcept;
\end{itemdecl}
\begin{itemdescr}
\pnum
\returns
\tcode{GraphicsMath::multiply(lhs.data(), rhs.data())}.
\end{itemdescr}

\indexlibrarymember{operator*}{basic_point_2d}%
\begin{itemdecl}
basic_point_2d operator*(const basic_point_2d& lhs, float rhs) noexcept;
\end{itemdecl}
\begin{itemdescr}
\pnum
\returns
\tcode{GraphicsMath::multiply(lhs.data(), rhs)}.
\end{itemdescr}

\indexlibrarymember{operator*}{basic_point_2d}%
\begin{itemdecl}
basic_point_2d operator*(float lhs, const basic_point_2d& rhs) noexcept;
\end{itemdecl}
\begin{itemdescr}
\pnum
\returns
\tcode{GraphicsMath::multiply(lhs, rhs.data())}.
\end{itemdescr}

\indexlibrarymember{operator/}{basic_point_2d}%
\begin{itemdecl}
basic_point_2d operator/(const basic_point_2d& lhs, const basic_point_2d& rhs) noexcept;
\end{itemdecl}
\begin{itemdescr}
\pnum
\requires \tcode{rhs.x() != 0.0f} and \tcode{rhs.y() != 0.0f}.

\pnum
\returns
\tcode{GraphicsMath::divide(lhs.data(), rhs.data())}.
\end{itemdescr}

\indexlibrarymember{operator/}{basic_point_2d}%
\begin{itemdecl}
basic_point_2d operator/(const basic_point_2d& lhs, float rhs) noexcept;
\end{itemdecl}
\begin{itemdescr}
\pnum
\requires
\tcode{rhs != 0.0f}.

\pnum
\returns
\tcode{GraphicsMath::divide(lhs.data(), rhs)}.
\end{itemdescr}

\indexlibrarymember{operator/}{basic_point_2d}%
\begin{itemdecl}
basic_point_2d operator/(float lhs, const basic_point_2d& rhs) noexcept;
\end{itemdecl}
\begin{itemdescr}
\pnum
\requires
\tcode{rhs.x() != 0.0f} and \tcode{rhs.y() != 0.0f}.

\pnum
\returns
\tcode{GraphicsMath::divide(lhs, rhs.data())}.
\end{itemdescr}
