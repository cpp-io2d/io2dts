%!TEX root = io2d.tex
\rSec0 [\iotwod.\pointtwod] {Class \tcode{basic_point_2d}}

\rSec1 [\iotwod.\pointtwod.intro] {\tcode{basic_point_2d} description}

\indexlibrary{\idxcode{basic_point_2d}}%
\pnum
The class \tcode{basic_point_2d} is used as both a point and as a two-dimensional Euclidean vector.

\pnum
It has an \term{x coordinate} of type \tcode{float} and a \term{y coordinate} of type \tcode{float}.

\rSec1 [\iotwod.\pointtwod.synopsis] {\tcode{basic_point_2d} synopsis}

\begin{codeblock}
namespace std::experimental::io2d::v1 {
  template <class GraphicsMath>
  class basic_point_2d {
  public:
    // \ref{\iotwod.\pointtwod.cons}, constructors:
    basic_point_2d() noexcept;
    basic_point_2d(float x, float y) noexcept;
    basic_point_2d(const typename GraphicsMath::point_2d_data_type& data) noexcept;

    // \ref{\iotwod.\pointtwod.modifiers}, modifiers:
    void x(float val) noexcept;
    void y(float val) noexcept;

    // \ref{\iotwod.\pointtwod.observers}, observers:
    float x() const noexcept;
    float y() const noexcept;
    float dot(const basic_point_2d& other) const noexcept;
    float magnitude() const noexcept;
    float magnitude_squared() const noexcept;
    float angular_direction() const noexcept;
    basic_point_2d to_unit() const noexcept;

    // \ref{\iotwod.\pointtwod.member.ops}, member operators:
    basic_point_2d& operator+=(const basic_point_2d& rhs) noexcept;
    basic_point_2d& operator+=(float rhs) noexcept;
    basic_point_2d& operator-=(const basic_point_2d& rhs) noexcept;
    basic_point_2d& operator-=(float rhs) noexcept;
    basic_point_2d& operator*=(const basic_point_2d& rhs) noexcept;
    basic_point_2d& operator*=(float rhs) noexcept;
    basic_point_2d& operator/=(const basic_point_2d& rhs) noexcept;
    basic_point_2d& operator/=(float rhs) noexcept;
  };

  // \ref{\iotwod.\pointtwod.ops}, non-member operators:
  template <class GraphicsMath>
  bool operator==(const basic_point_2d<GraphicsMath>& lhs,
    const basic_point_2d<GraphicsMath>& rhs) noexcept;
  template <class GraphicsMath>
  bool operator!=(const basic_point_2d<GraphicsMath>& lhs,
    const basic_point_2d<GraphicsMath>& rhs) noexcept;
  template <class GraphicsMath>
  basic_point_2d<GraphicsMath> operator+(const basic_point_2d<GraphicsMath>& val) noexcept;
  template <class GraphicsMath>
  basic_point_2d<GraphicsMath> operator+(const basic_point_2d<GraphicsMath>& lhs,
    const basic_point_2d<GraphicsMath>& rhs) noexcept;
  template <class GraphicsMath>
  basic_point_2d<GraphicsMath> operator-(const basic_point_2d<GraphicsMath>& val) noexcept;
  template <class GraphicsMath>
  basic_point_2d<GraphicsMath> operator-(const basic_point_2d<GraphicsMath>& lhs,
    const basic_point_2d<GraphicsMath>& rhs) noexcept;
  template <class GraphicsMath>
  basic_point_2d<GraphicsMath> operator*(const basic_point_2d<GraphicsMath>& lhs,
    float rhs) noexcept;
  template <class GraphicsMath>
  basic_point_2d<GraphicsMath> operator*(float lhs,
    const basic_point_2d<GraphicsMath>& rhs) noexcept;
  template <class GraphicsMath>
  basic_point_2d<GraphicsMath> operator*(const basic_point_2d<GraphicsMath>& lhs,
    const basic_point_2d<GraphicsMath>& rhs) noexcept;
  template <class GraphicsMath>
  basic_point_2d<GraphicsMath> operator/(const basic_point_2d<GraphicsMath>& lhs,
    float rhs) noexcept;
  template <class GraphicsMath>
  basic_point_2d<GraphicsMath> operator/(float lhs,
    const basic_point_2d<GraphicsMath>& rhs) noexcept;
  template <class GraphicsMath>
  basic_point_2d<GraphicsMath> operator/(const basic_point_2d<GraphicsMath>& lhs,
    const basic_point_2d<GraphicsMath>& rhs) noexcept;
}
\end{codeblock}

\rSec1 [\iotwod.\pointtwod.cons] {\tcode{basic_point_2d} constructors}

\indexlibrary{\idxcode{basic_point_2d}!constructor}%
\begin{itemdecl}
basic_point_2d() noexcept;
\end{itemdecl}
\begin{itemdescr}
\pnum
\effects
Equivalent to \tcode{basic_point_2d\{ 0.0f, 0.0f \}}.
\end{itemdescr}

\indexlibrary{\idxcode{basic_point_2d}!constructor}%
\begin{itemdecl}
basic_point_2d(float x, float y) noexcept;
\end{itemdecl}
\begin{itemdescr}
\pnum
\effects
Constructs an object of type \tcode{basic_point_2d}.

\pnum
The x coordinate is \tcode{x}.

\pnum
The y coordinate is \tcode{y}.
\end{itemdescr}

\indexlibrary{\idxcode{basic_point_2d}!constructor}%
\begin{itemdecl}
basic_point_2d(const typename GraphicsMath::point_2d_data_type& data) noexcept;
\end{itemdecl}
\begin{itemdescr}
\pnum
\effects
Constructs an object of type \tcode{basic_point_2d}.

\pnum
<TODO>The coordinates are \tcode{data}.
\end{itemdescr}

\rSec1 [\iotwod.\pointtwod.modifiers]{\tcode{basic_point_2d} modifiers}

\indexlibrarymember{x}{basic_point_2d}%
\begin{itemdecl}
void x(float val) noexcept;
\end{itemdecl}
\begin{itemdescr}
\pnum
\effects
<TODO>
\end{itemdescr}

\indexlibrarymember{y}{basic_point_2d}%
\begin{itemdecl}
void y(float val) noexcept;
\end{itemdecl}
\begin{itemdescr}
\pnum
\effects
<TODO>
\end{itemdescr}

\rSec1 [\iotwod.\pointtwod.observers]{\tcode{basic_point_2d} observers}

\indexlibrarymember{x}{basic_point_2d}%
\begin{itemdecl}
float x() const noexcept;
\end{itemdecl}
\begin{itemdescr}
\pnum
\returns
<TODO>
\end{itemdescr}

\indexlibrarymember{y}{basic_point_2d}%
\begin{itemdecl}
float y() const noexcept;
\end{itemdecl}
\begin{itemdescr}
\pnum
\returns
<TODO>
\end{itemdescr}

\indexlibrarymember{dot}{basic_point_2d}%
\begin{itemdecl}
float dot(const basic_point_2d& other) const noexcept;
\end{itemdecl}
\begin{itemdescr}
\pnum
\returns
<TODO>\tcode{x * other.x + y * other.y}.
\end{itemdescr}

\indexlibrarymember{magnitude}{basic_point_2d}%
\begin{itemdecl}
float magnitude() const noexcept;
\end{itemdecl}
\begin{itemdescr}
\pnum
\returns
Equivalent to: \tcode{sqrt(dot(*this));}
\end{itemdescr}

\indexlibrarymember{magnitude_squared}{basic_point_2d}%
\begin{itemdecl}
float magnitude_squared() const noexcept;
\end{itemdecl}
\begin{itemdescr}
\pnum
\returns
Equivalent to: \tcode{dot(*this);}
\end{itemdescr}

\indexlibrarymember{angular_direction}{basic_point_2d}%
\begin{itemdecl}
float angular_direction() const noexcept
\end{itemdecl}
\begin{itemdescr}
\pnum
\returns
<TODO>\tcode{atan2(y, x)} if it is greater than or equal to \tcode{0.0f}.

\pnum
<TODO>Otherwise, \tcode{atan2(y, x) + two_pi<float>}. 

\pnum
\begin{note}
The purpose of adding \tcode{two_pi<float>} if the result is negative is to produce values in the range \range{0.0f}{two_pi<float>}.
\end{note}
\end{itemdescr}

\indexlibrarymember{to_unit}{basic_point_2d}%
\begin{itemdecl}
basic_point_2d to_unit() const noexcept;
\end{itemdecl}
\begin{itemdescr}
\pnum
\returns
<TODO>\tcode{basic_point_2d\{ x / magnitude(), y / magnitude()\}}.
\end{itemdescr}

\rSec1 [\iotwod.\pointtwod.member.ops] {\tcode{basic_point_2d} member operators}

\indexlibrarymember{operator+=}{basic_point_2d}%
\begin{itemdecl}
basic_point_2d& operator+=(const basic_point_2d& rhs) noexcept;
\end{itemdecl}
\begin{itemdescr}
\pnum
\effects
\tcode{*this = *this + rhs}.
	
\pnum
\returns
\tcode{*this}.
\end{itemdescr}

\indexlibrarymember{operator-=}{basic_point_2d}%
\begin{itemdecl}
basic_point_2d& operator-=(const basic_point_2d& rhs) noexcept;
\end{itemdecl}
\begin{itemdescr}
\pnum
\effects
Equivalent to: \tcode{*this = *this - rhs;}.

\pnum
\returns
\tcode{*this}.
\end{itemdescr}

\indexlibrarymember{operator*=}{basic_point_2d}%
\begin{itemdecl}
basic_point_2d& operator*=(float rhs) noexcept;
basic_point_2d& operator*=(const basic_point_2d& rhs) noexcept;
\end{itemdecl}
\begin{itemdescr}
\pnum
\effects
Equivalent to: \tcode{*this = *this * rhs;}.

\pnum
\returns
\tcode{*this}.
\end{itemdescr}

\indexlibrarymember{operator/=}{basic_point_2d}%
\begin{itemdecl}
basic_point_2d& operator/=(float rhs) noexcept;
basic_point_2d& operator/=(const basic_point_2d& rhs) noexcept;
\end{itemdecl}
\begin{itemdescr}
\pnum
\effects
Equivalent to: \tcode{*this = *this / rhs;}.

\pnum
\returns
\tcode{*this}.
\end{itemdescr}

\rSec1 [\iotwod.\pointtwod.ops] {\tcode{basic_point_2d} non-member operators}

\indexlibrarymember{operator==}{basic_point_2d}%
\begin{itemdecl}
bool operator==(const basic_point_2d& lhs, const basic_point_2d& rhs) noexcept;
\end{itemdecl}
\begin{itemdescr}
\pnum
\returns
<TODO>\tcode{lhs.x == rhs.x \&\& lhs.y == rhs.y}.
\end{itemdescr}

\indexlibrarymember{operator!=}{basic_point_2d}%
\begin{itemdecl}
bool operator!=(const basic_point_2d& lhs, const basic_point_2d& rhs) noexcept;
\end{itemdecl}
\begin{itemdescr}
\pnum
\returns
\tcode{!(lhs == rhs)}.
\end{itemdescr}

\indexlibrarymember{operator+}{basic_point_2d}%
\begin{itemdecl}
basic_point_2d operator+(const basic_point_2d& val) noexcept;
\end{itemdecl}
\begin{itemdescr}
\pnum
\returns
\tcode{val}.
\end{itemdescr}

\indexlibrarymember{operator+}{basic_point_2d}%
\begin{itemdecl}
basic_point_2d operator+(const basic_point_2d& lhs, const basic_point_2d& rhs) noexcept;
\end{itemdecl}
\begin{itemdescr}
\pnum
\returns
<TODO>\tcode{basic_point_2d\{ lhs.x + rhs.x, lhs.y + rhs.y \}}.
\end{itemdescr}

\indexlibrarymember{operator-}{basic_point_2d}%
\begin{itemdecl}
basic_point_2d operator-(const basic_point_2d& val) noexcept;
\end{itemdecl}
\begin{itemdescr}
\pnum
\returns
<TODO>\tcode{basic_point_2d\{ -val.x, -val.y \}}.
\end{itemdescr}

\indexlibrarymember{operator-}{basic_point_2d}%
\begin{itemdecl}
basic_point_2d operator-(const basic_point_2d& lhs, const basic_point_2d& rhs) noexcept;
\end{itemdecl}
\begin{itemdescr}
\pnum
\returns
<TODO>\tcode{basic_point_2d\{ lhs.x - rhs.x, lhs.y - rhs.y \}}.
\end{itemdescr}

\indexlibrarymember{operator*}{basic_point_2d}%
\begin{itemdecl}
basic_point_2d operator*(const basic_point_2d& lhs, const basic_point_2d& rhs) noexcept;
\end{itemdecl}
\begin{itemdescr}
\pnum
\returns
<TODO>\tcode{basic_point_2d\{ lhs.x * rhs.x, lhs.y * rhs.y \}}.
\end{itemdescr}

\indexlibrarymember{operator*}{basic_point_2d}%
\begin{itemdecl}
basic_point_2d operator*(const basic_point_2d& lhs, float rhs) noexcept;
\end{itemdecl}
\begin{itemdescr}
\pnum
\returns
<TODO>\tcode{basic_point_2d\{ lhs.x * rhs, lhs.y * rhs \}}.
\end{itemdescr}

\indexlibrarymember{operator*}{basic_point_2d}%
\begin{itemdecl}
basic_point_2d operator*(float lhs, const basic_point_2d& rhs) noexcept;
\end{itemdecl}
\begin{itemdescr}
\pnum
\returns
<TODO>\tcode{basic_point_2d\{ lhs * rhs.x, lhs * rhs.y \}}.
\end{itemdescr}

\indexlibrarymember{operator/}{basic_point_2d}%
\begin{itemdecl}
basic_point_2d operator/(const basic_point_2d& lhs, const basic_point_2d& rhs) noexcept;
\end{itemdecl}
\begin{itemdescr}
\pnum
\requires
<TODO>\tcode{rhs.x} is not \tcode{0.0f} and \tcode{rhs.y} is not \tcode{0.0f}.

\pnum
\returns
<TODO>\tcode{basic_point_2d\{ lhs.x / rhs.x, lhs.y / rhs.y \}}.
\end{itemdescr}

\indexlibrarymember{operator/}{basic_point_2d}%
\begin{itemdecl}
basic_point_2d operator/(const basic_point_2d& lhs, float rhs) noexcept;
\end{itemdecl}
\begin{itemdescr}
\pnum
\requires
\tcode{rhs} is not \tcode{0.0f}.

\pnum
\returns
<TODO>\tcode{basic_point_2d\{ lhs.x / rhs, lhs.y / rhs \}}.
\end{itemdescr}

\indexlibrarymember{operator/}{basic_point_2d}%
\begin{itemdecl}
basic_point_2d operator/(float lhs, const basic_point_2d& rhs) noexcept;
\end{itemdecl}
\begin{itemdescr}
\pnum
\requires
<TODO>\tcode{rhs.x} is not \tcode{0.0f} and \tcode{rhs.y} is not \tcode{0.0f}.

\pnum
\returns
<TODO>\tcode{basic_point_2d\{ lhs / rhs.x, lhs / rhs.y \}}.
\end{itemdescr}
