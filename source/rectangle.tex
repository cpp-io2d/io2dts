%!TEX root = io2d.tex
\rSec0 [\iotwod.bounding_box] {Class \tcode{bounding_box}}

\rSec1 [\iotwod.bounding_box.intro] {\tcode{bounding_box} description}

\indexlibrary{\idxcode{bounding_box}}%
\pnum
The class \tcode{bounding_box} describes a bounding_box.

\pnum
It has an \term{x coordinate} of type \tcode{float}, a \term{y coordinate} of type \tcode{float}, a \term{width} of type \tcode{float}, and a \term{height} of type \tcode{float}.

\rSec1 [\iotwod.bounding_box.synopsis] {\tcode{bounding_box} synopsis}

\begin{codeblock}
namespace std::experimental::io2d::v1 {
  class bounding_box {
  public:
    // \ref{\iotwod.bounding_box.cons}, construct:
    constexpr bounding_box() noexcept;
    constexpr bounding_box(float x, float y, float width, float height)
      noexcept;
    constexpr bounding_box(point_2d tl, point_2d br) noexcept;

    // \ref{\iotwod.bounding_box.modifiers}, modifiers:
    constexpr void x(float val) noexcept;
    constexpr void y(float val) noexcept;
    constexpr void width(float val) noexcept;
    constexpr void height(float val) noexcept;
    constexpr void top_left(point_2d val) noexcept;
    constexpr void bottom_right(point_2d val) noexcept;
    
    // \ref{\iotwod.bounding_box.observers}, observers:
    constexpr float x() const noexcept;
    constexpr float y() const noexcept;
    constexpr float width() const noexcept;
    constexpr float height() const noexcept;
    constexpr point_2d top_left() const noexcept;
    constexpr point_2d bottom_right() const noexcept;
  };
  
  // \ref{\iotwod.bounding_box.ops}, operators:
  constexpr bool operator==(const bounding_box& lhs, const bounding_box& rhs) 
    noexcept;
  constexpr bool operator!=(const bounding_box& lhs, const bounding_box& rhs) 
    noexcept;
}
\end{codeblock}

\rSec1 [\iotwod.bounding_box.cons] {\tcode{bounding_box} constructors}

\indexlibrary{\idxcode{bounding_box}!constructor}%
\begin{itemdecl}
constexpr bounding_box() noexcept;
\end{itemdecl}
\begin{itemdescr}
\pnum
\effects
Equivalent to \tcode{bounding_box\{ 0.0f, 0.0f, 0.0f, 0.0f \}}.
\end{itemdescr}

\indexlibrary{\idxcode{bounding_box}!constructor}%
\begin{itemdecl}
constexpr bounding_box(float x, float y, float w, float h) noexcept;
\end{itemdecl}
\begin{itemdescr}
\pnum
\requires
\tcode{w} is not less than \tcode{0.0f} and \tcode{h} is not less than \tcode{0.0f}.

\pnum
\effects
Constructs an object of type \tcode{bounding_box}.

\pnum
The x coordinate is \tcode{x}. The y coordinate is \tcode{y}. The width is \tcode{w}. The height is \tcode{h}.
\end{itemdescr}

\indexlibrary{\idxcode{bounding_box}!constructor}%
\begin{itemdecl}
constexpr bounding_box(point_2d tl, point_2d br) noexcept;
\end{itemdecl}
\begin{itemdescr}
\pnum
\effects
Constructs an object of type \tcode{bounding_box}.

\pnum
The x coordinate is \tcode{tl.x}. The y coordinate is \tcode{tl.y}. The width is \tcode{max(0.0f, br.x - tl.x)}. The height is \tcode{max(0.0f, br.y - tl.y)}.
\end{itemdescr}

\rSec1 [\iotwod.bounding_box.modifiers]{\tcode{bounding_box} modifiers}

\indexlibrarymember{x}{bounding_box}%
\begin{itemdecl}
constexpr void x(float val) noexcept;
\end{itemdecl}

\begin{itemdescr}
\pnum
\effects
The x coordinate is \tcode{val}.
\end{itemdescr}

\indexlibrarymember{y}{bounding_box}%
\begin{itemdecl}
constexpr void y(float val) noexcept;
\end{itemdecl}
\begin{itemdescr}
\pnum
\effects
The y coordinate is \tcode{val}.
\end{itemdescr}

\indexlibrarymember{width}{bounding_box}%
\begin{itemdecl}
constexpr void width(float val) noexcept;
\end{itemdecl}
\begin{itemdescr}
\pnum
\effects
The width is \tcode{val}.
\end{itemdescr}

\indexlibrarymember{height}{bounding_box}%
\begin{itemdecl}
constexpr void height(float val) noexcept;
\end{itemdecl}
\begin{itemdescr}
\pnum
\effects
The height is \tcode{val}.
\end{itemdescr}

\indexlibrarymember{top_left}{bounding_box}%
\begin{itemdecl}
constexpr void top_left(point_2d val) noexcept;
\end{itemdecl}
\begin{itemdescr}
\pnum
\effects
The x coordinate is \tcode{val.x}.

\effects
The y coordinate is \tcode{val.y}.
\end{itemdescr}

\indexlibrarymember{bottom_right}{bounding_box}%
\begin{itemdecl}
constexpr void bottom_right(point_2d val) noexcept;
\end{itemdecl}
\begin{itemdescr}
\pnum
\effects
The width is \tcode{max(0.0f, val.x - x())}.

\pnum
The height is \tcode{max(0.0f, value.y - y())}.
\end{itemdescr}

\rSec1 [\iotwod.bounding_box.observers]{\tcode{bounding_box} observers}

\indexlibrarymember{x}{bounding_box}%
\begin{itemdecl}
constexpr float x() const noexcept;
\end{itemdecl}
\begin{itemdescr}
\pnum
\returns
The x coordinate.
\end{itemdescr}

\indexlibrarymember{y}{bounding_box}%
\begin{itemdecl}
constexpr float y() const noexcept;
\end{itemdecl}
\begin{itemdescr}
\pnum
\returns
The y coordinate.
\end{itemdescr}

\indexlibrarymember{width}{bounding_box}%
\begin{itemdecl}
constexpr float width() const noexcept;
\end{itemdecl}
\begin{itemdescr}
\pnum
\returns
The width.
\end{itemdescr}

\indexlibrarymember{height}{bounding_box}%
\begin{itemdecl}
constexpr float height() const noexcept;
\end{itemdecl}
\begin{itemdescr}
\pnum
\returns
The height.
\end{itemdescr}

\indexlibrarymember{top_left}{bounding_box}%
\begin{itemdecl}
constexpr point_2d top_left() const noexcept;
\end{itemdecl}
\begin{itemdescr}
\pnum
\returns
A \tcode{point_2d} object constructed with the x coordinate as its first argument and the y coordinate as its second argument.
\end{itemdescr}

\indexlibrarymember{bottom_right}{bounding_box}%
\begin{itemdecl}
constexpr point_2d bottom_right() const noexcept;
\end{itemdecl}
\begin{itemdescr}
\pnum
\returns
A \tcode{point_2d} object constructed with the width added to the x coordinate as its first argument and the height added to the y coordinate as its second argument.
\end{itemdescr}

\rSec1 [\iotwod.bounding_box.ops] {\tcode{bounding_box} operators}

\indexlibrarymember{operator==}{bounding_box}%
\begin{itemdecl}
constexpr bool operator==(const bounding_box& lhs, const bounding_box& rhs) noexcept;
\end{itemdecl}
\begin{itemdescr}
\pnum
\returns
\begin{codeblock}
lhs.x() == rhs.x() && lhs.y() == rhs.y() &&
lhs.width() == rhs.width() && lhs.height() == rhs.height()
\end{codeblock}
\end{itemdescr}
