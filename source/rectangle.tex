%!TEX root = io2d.tex
\rSec0 [rectangle] {Class \tcode{rectangle}}

\rSec1 [rectangle.intro] {\tcode{rectangle} Description}

\pnum
\indexlibrary{\idxcode{rectangle}}
The class \tcode{rectangle} describes a rectangle.

\pnum
It has an X coordinate of type \tcode{double}, a Y coordinate of type \tcode{double}, a width of type \tcode{double}, and a height of type \tcode{double}.

\rSec1 [rectangle.synopsis] {\tcode{rectangle} synopsis}

\begin{codeblock}
namespace std { namespace experimental { namespace io2d { inline namespace v1 {
  class rectangle {
  public:
    // \ref{rectangle.cons}, construct/copy/move/destroy:
    rectangle(double x, double y, double width, double height) noexcept;
    rectangle(const vector_2d& tl, const vector_2d& br) noexcept;

    // \ref{rectangle.modifiers}, modifiers:
    void x(double val) noexcept;
    void y(double val) noexcept;
    void width(double val) noexcept;
    void height(double val) noexcept;
    void top_left(const vector_2d& val) noexcept;
    void bottom_right(const vector_2d& val) noexcept;
    
    // \ref{rectangle.observers}, observers:
    double x() const noexcept;
    double y() const noexcept;
    double width() const noexcept;
    double height() const noexcept;
    vector_2d top_left() const noexcept;
    vector_2d bottom_right() const noexcept;
  };
} } } }
\end{codeblock}

\rSec1 [rectangle.cons] {\tcode{rectangle} constructors}

\indexlibrary{\idxcode{rectangle}!constructor}
\begin{itemdecl}
rectangle(double x, double y, double w, double h) noexcept;
\end{itemdecl}
\begin{itemdescr}
	\pnum
	\effects
	Constructs an object of type \tcode{rectangle}.
	
	\pnum
	The X coordinate shall be set to the value of \tcode{x}.
	
	\pnum
	The Y coordinate shall be set to the value of \tcode{y}.
	
	\pnum
	The width shall be set to the value of \tcode{w}.
	
	\pnum
	The height shall be set to the value of \tcode{h}.
\end{itemdescr}

\indexlibrary{\idxcode{rectangle}!constructor}
\begin{itemdecl}
rectangle(const vector_2d& tl, const vector_2d& br) noexcept;
\end{itemdecl}
\begin{itemdescr}
	\pnum
	\effects
	Constructs an object of type \tcode{rectangle}.
	
	\pnum
	The X coordinate shall be set to the value of \tcode{tl.x()}.
	
	\pnum
	The Y coordinate shall be set to the value of \tcode{tl.y()}.
	
	\pnum
	The width shall be set to the value of \tcode{max(0.0, br.x() - tl.x())}.
	
	\pnum
	The height shall be set to the value of \tcode{max(0.0, br.y() - tl.y())}.
\end{itemdescr}

\rSec1 [rectangle.modifiers]{\tcode{rectangle} modifiers}

\indexlibrary{\idxcode{rectangle}!\idxcode{x}}
\indexlibrary{\idxcode{x}!\idxcode{rectangle}}
\begin{itemdecl}
void x(double val) noexcept;
\end{itemdecl}

\begin{itemdescr}
	\pnum
	\effects
	The X coordinate shall be set to the value of \tcode{val}.
\end{itemdescr}

\indexlibrary{\idxcode{rectangle}!\idxcode{y}}
\indexlibrary{\idxcode{y}!\idxcode{rectangle}}
\begin{itemdecl}
void y(double value) noexcept;
\end{itemdecl}
\begin{itemdescr}
	\pnum
	\effects
	The Y coordinate shall be set to the value of \tcode{val}.
\end{itemdescr}

\indexlibrary{\idxcode{rectangle}!\idxcode{width}}
\indexlibrary{\idxcode{width}!\idxcode{rectangle}}
\begin{itemdecl}
void width(double value) noexcept;
\end{itemdecl}
\begin{itemdescr}
	\pnum
	\effects
	The width shall be set to the value of \tcode{val}.
\end{itemdescr}

\indexlibrary{\idxcode{rectangle}!\idxcode{height}}
\indexlibrary{\idxcode{height}!\idxcode{rectangle}}
\begin{itemdecl}
void height(double value) noexcept;
\end{itemdecl}
\begin{itemdescr}
	\pnum
	\effects
	The height shall be set to the value of \tcode{val}.
\end{itemdescr}

\indexlibrary{\idxcode{rectangle}!\idxcode{top_left}}
\indexlibrary{\idxcode{top_left}!\idxcode{rectangle}}
\begin{itemdecl}
void top_left(const vector_2d& val) noexcept;
\end{itemdecl}
\begin{itemdescr}
	\pnum
	\effects
	The X coordinate shall be set to the value of \tcode{val.x()}.
	
	\effects
	The Y coordinate shall be set to the value of \tcode{val.y()}.
\end{itemdescr}

\indexlibrary{\idxcode{rectangle}!\idxcode{bottom_right}}
\indexlibrary{\idxcode{bottom_right}!\idxcode{rectangle}}
\begin{itemdecl}
void bottom_right(const vector_2d& val) noexcept;
\end{itemdecl}
\begin{itemdescr}
	\pnum
	\effects
	The width shall be set to the value of \tcode{max(0.0, val.x() - *this.x())}.
	
	\pnum
	The height shall be set to the value of \tcode{max(0.0, value.y() - *this.y())}.
\end{itemdescr}

\rSec1 [rectangle.observers]{\tcode{rectangle} observers}

\indexlibrary{\idxcode{rectangle}!\idxcode{x}}
\indexlibrary{\idxcode{x}!\idxcode{rectangle}}
\begin{itemdecl}
double x() const noexcept;
\end{itemdecl}
\begin{itemdescr}
	\pnum
	\returns
	The value of the X coordinate.
\end{itemdescr}

\indexlibrary{\idxcode{rectangle}!\idxcode{y}}
\indexlibrary{\idxcode{y}!\idxcode{rectangle}}
\begin{itemdecl}
double y() const noexcept;
\end{itemdecl}
\begin{itemdescr}
	\pnum
	\returns
	The value of the Y coordinate.
\end{itemdescr}

\indexlibrary{\idxcode{rectangle}!\idxcode{width}}
\indexlibrary{\idxcode{width}!\idxcode{rectangle}}
\begin{itemdecl}
double width() const noexcept;
\end{itemdecl}
\begin{itemdescr}
	\pnum
	\returns
	The value of the width.
\end{itemdescr}

\indexlibrary{\idxcode{rectangle}!\idxcode{height}}
\indexlibrary{\idxcode{height}!\idxcode{rectangle}}
\begin{itemdecl}
double height() const noexcept;
\end{itemdecl}
\begin{itemdescr}
	\pnum
	\returns
	The value of the height.
\end{itemdescr}

\indexlibrary{\idxcode{rectangle}!\idxcode{top_left}}
\indexlibrary{\idxcode{top_left}!\idxcode{rectangle}}
\begin{itemdecl}
vector_2d top_left() const noexcept;
\end{itemdecl}
\begin{itemdescr}
	\pnum
	\returns
	A \tcode{vector_2d} object constructed from the value of the X coordinate as its first argument and the value of the Y coordinate as its second argument.
\end{itemdescr}

\indexlibrary{\idxcode{rectangle}!\idxcode{bottom_right}}
\indexlibrary{\idxcode{bottom_right}!\idxcode{rectangle}}
\begin{itemdecl}
vector_2d bottom_right() const noexcept;
\end{itemdecl}
\begin{itemdescr}
	\pnum
	\returns
	A \tcode{vector_2d} object constructed from the value of the width added to the value of the X coordinate as its first argument and the value of the height added to the value of the Y coordinate as its second argument.
\end{itemdescr}
