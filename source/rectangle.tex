%!TEX root = io2d.tex
\rSec0 [\iotwod.rectangle] {Class \tcode{rectangle}}

\rSec1 [\iotwod.rectangle.intro] {\tcode{rectangle} description}

\indexlibrary{\idxcode{rectangle}}%
\pnum
The class \tcode{rectangle} describes a rectangle.

\pnum
It has an \term{x coordinate} of type \tcode{float}, a \term{y coordinate} of type \tcode{float}, a \term{width} of type \tcode{float}, and a \term{height} of type \tcode{float}.

\rSec1 [\iotwod.rectangle.synopsis] {\tcode{rectangle} synopsis}

\begin{codeblock}
namespace std::experimental::io2d::v1 {
  class rectangle {
  public:
    // \ref{\iotwod.rectangle.cons}, construct:
    constexpr rectangle() noexcept;
    constexpr rectangle(float x, float y, float width, float height)
      noexcept;
    constexpr rectangle(const vector_2d& tl, const vector_2d& br) noexcept;

    // \ref{\iotwod.rectangle.modifiers}, modifiers:
    constexpr void x(float val) noexcept;
    constexpr void y(float val) noexcept;
    constexpr void width(float val) noexcept;
    constexpr void height(float val) noexcept;
    constexpr void top_left(const vector_2d& val) noexcept;
    constexpr void bottom_right(const vector_2d& val) noexcept;
    
    // \ref{\iotwod.rectangle.observers}, observers:
    constexpr float x() const noexcept;
    constexpr float y() const noexcept;
    constexpr float width() const noexcept;
    constexpr float height() const noexcept;
    constexpr vector_2d top_left() const noexcept;
    constexpr vector_2d bottom_right() const noexcept;
  };
  
  // \ref{\iotwod.rectangle.ops}, operators:
  constexpr bool operator==(const rectangle& lhs, const rectangle& rhs) 
    noexcept;
  constexpr bool operator!=(const rectangle& lhs, const rectangle& rhs) 
    noexcept;
}
\end{codeblock}

\rSec1 [\iotwod.rectangle.cons] {\tcode{rectangle} constructors}

\indexlibrary{\idxcode{rectangle}!constructor}%
\begin{itemdecl}
constexpr rectangle() noexcept;
\end{itemdecl}
\begin{itemdescr}
\pnum
\effects
Equivalent to \tcode{rectangle\{ 0.0f, 0.0f, 0.0f, 0.0f \}}.
\end{itemdescr}

\indexlibrary{\idxcode{rectangle}!constructor}%
\begin{itemdecl}
constexpr rectangle(float x, float y, float w, float h) noexcept;
\end{itemdecl}
\begin{itemdescr}
\pnum
\requires
\tcode{w} is not less than \tcode{0.0f} and \tcode{h} is not less than \tcode{0.0f}.

\pnum
\effects
Constructs an object of type \tcode{rectangle}.

\pnum
The x coordinate is \tcode{x}. The y coordinate is \tcode{y}. The width is \tcode{w}. The height is \tcode{h}.
\end{itemdescr}

\indexlibrary{\idxcode{rectangle}!constructor}%
\begin{itemdecl}
constexpr rectangle(const vector_2d& tl, const vector_2d& br) noexcept;
\end{itemdecl}
\begin{itemdescr}
\pnum
\effects
Constructs an object of type \tcode{rectangle}.

\pnum
The x coordinate is \tcode{tl.x()}. The y coordinate is \tcode{tl.y()}. The width is \tcode{max(0.0f, br.x() - tl.x())}. The height is \tcode{max(0.0f, br.y() - tl.y())}.
\end{itemdescr}

\rSec1 [\iotwod.rectangle.modifiers]{\tcode{rectangle} modifiers}

\indexlibrarymember{x}{rectangle}%
\begin{itemdecl}
constexpr void x(float val) noexcept;
\end{itemdecl}

\begin{itemdescr}
\pnum
\effects
The x coordinate is \tcode{val}.
\end{itemdescr}

\indexlibrarymember{y}{rectangle}%
\begin{itemdecl}
constexpr void y(float val) noexcept;
\end{itemdecl}
\begin{itemdescr}
\pnum
\effects
The y coordinate is \tcode{val}.
\end{itemdescr}

\indexlibrarymember{width}{rectangle}%
\begin{itemdecl}
constexpr void width(float val) noexcept;
\end{itemdecl}
\begin{itemdescr}
\pnum
\effects
The width is \tcode{val}.
\end{itemdescr}

\indexlibrarymember{height}{rectangle}%
\begin{itemdecl}
constexpr void height(float val) noexcept;
\end{itemdecl}
\begin{itemdescr}
\pnum
\effects
The height is \tcode{val}.
\end{itemdescr}

\indexlibrarymember{top_left}{rectangle}%
\begin{itemdecl}
constexpr void top_left(const vector_2d& val) noexcept;
\end{itemdecl}
\begin{itemdescr}
\pnum
\effects
The x coordinate is \tcode{val.x()}.

\effects
The y coordinate is \tcode{val.y()}.
\end{itemdescr}

\indexlibrarymember{bottom_right}{rectangle}%
\begin{itemdecl}
constexpr void bottom_right(const vector_2d& val) noexcept;
\end{itemdecl}
\begin{itemdescr}
\pnum
\effects
The width is \tcode{max(0.0f, val.x() - x())}.

\pnum
The height is \tcode{max(0.0f, value.y() - y())}.
\end{itemdescr}

\rSec1 [\iotwod.rectangle.observers]{\tcode{rectangle} observers}

\indexlibrarymember{x}{rectangle}%
\begin{itemdecl}
constexpr float x() const noexcept;
\end{itemdecl}
\begin{itemdescr}
\pnum
\returns
The x coordinate.
\end{itemdescr}

\indexlibrarymember{y}{rectangle}%
\begin{itemdecl}
constexpr float y() const noexcept;
\end{itemdecl}
\begin{itemdescr}
\pnum
\returns
The y coordinate.
\end{itemdescr}

\indexlibrarymember{width}{rectangle}%
\begin{itemdecl}
constexpr float width() const noexcept;
\end{itemdecl}
\begin{itemdescr}
\pnum
\returns
The width.
\end{itemdescr}

\indexlibrarymember{height}{rectangle}%
\begin{itemdecl}
constexpr float height() const noexcept;
\end{itemdecl}
\begin{itemdescr}
\pnum
\returns
The height.
\end{itemdescr}

\indexlibrarymember{top_left}{rectangle}%
\begin{itemdecl}
constexpr vector_2d top_left() const noexcept;
\end{itemdecl}
\begin{itemdescr}
\pnum
\returns
A \tcode{vector_2d} object constructed with the x coordinate as its first argument and the y coordinate as its second argument.
\end{itemdescr}

\indexlibrarymember{bottom_right}{rectangle}%
\begin{itemdecl}
constexpr vector_2d bottom_right() const noexcept;
\end{itemdecl}
\begin{itemdescr}
\pnum
\returns
A \tcode{vector_2d} object constructed with the width added to the x coordinate as its first argument and the height added to the y coordinate as its second argument.
\end{itemdescr}

\rSec1 [\iotwod.rectangle.ops] {\tcode{rectangle} operators}

\indexlibrarymember{operator==}{rectangle}%
\begin{itemdecl}
constexpr bool operator==(const rectangle& lhs, const rectangle& rhs) noexcept;
\end{itemdecl}
\begin{itemdescr}
\pnum
\returns
\begin{codeblock}
lhs.x() == rhs.x() && lhs.y() == rhs.y() &&
lhs.width() == rhs.width() && lhs.height() == rhs.height()
\end{codeblock}
\end{itemdescr}

\indexlibrarymember{operator!=}{rectangle}%
\begin{itemdecl}
constexpr bool operator!=(const rectangle& lhs, const rectangle& rhs) noexcept;
\end{itemdecl}
\begin{itemdescr}
\pnum
\returns
\tcode{!(lhs == rhs)}.
\end{itemdescr}
