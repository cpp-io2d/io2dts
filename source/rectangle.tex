%!TEX root = io2d.tex
\rSec0 [\iotwod.bounding_box] {Class \tcode{basic_bounding_box}}

\rSec1 [\iotwod.bounding_box.intro] {\tcode{basic_bounding_box} description}

\indexlibrary{\idxcode{basic_bounding_box}}%
\pnum
The class template \tcode{basic_bounding_box} describes a bounding_box.

\pnum
It has an \term{x coordinate} of type \tcode{float}, a \term{y coordinate} of type \tcode{float}, a \term{width} of type \tcode{float}, and a \term{height} of type \tcode{float}.

\pnum
The data are stored in an object of type \tcode{typename GraphicsMath::bounding_box_data_type}. It is accessible using the \tcode{data} member functions.

\rSec1 [\iotwod.bounding_box.synopsis] {\tcode{basic_bounding_box} synopsis}

\begin{codeblock}
namespace std::experimental::io2d::v1 {
  template <class GraphicsMath>
  class basic_bounding_box {
  public:
    using data_type = typename GraphicsMath::bounding_box_data_type;
    
    // \ref{\iotwod.bounding_box.cons}, constructors:
    basic_bounding_box() noexcept;
    basic_bounding_box(float x, float y, float width, float height) noexcept;
    basic_bounding_box(const basic_point_2d<GraphicsMath>& tl,
      const basic_point_2d<GraphicsMath>& br) noexcept;
    basic_bounding_box(const data_type& val) noexcept;

    // \ref{\iotwod.bounding_box.accessors}, accessors:
    const data_type& data() const noexcept;
    data_type& data() noexcept;

    // \ref{\iotwod.bounding_box.modifiers}, modifiers:
    void x(float val) noexcept;
    void y(float val) noexcept;
    void width(float val) noexcept;
    void height(float val) noexcept;
    void top_left(const basic_point_2d<GraphicsMath>& val) noexcept;
    void bottom_right(const basic_point_2d<GraphicsMath>& val) noexcept;

    // \ref{\iotwod.bounding_box.observers}, observers:
    float x() const noexcept;
    float y() const noexcept;
    float width() const noexcept;
    float height() const noexcept;
    basic_point_2d<GraphicsMath> top_left() const noexcept;
    basic_point_2d<GraphicsMath> bottom_right() const noexcept;
  };

  // \ref{\iotwod.bounding_box.ops}, operators:
  template <class GraphicsMath>
  bool operator==(const basic_bounding_box<GraphicsMath>& lhs,
    const basic_bounding_box<GraphicsMath>& rhs) noexcept;
  template <class GraphicsMath>
  bool operator!=(const basic_bounding_box<GraphicsMath>& lhs,
    const basic_bounding_box<GraphicsMath>& rhs) noexcept;
}
\end{codeblock}

\rSec1 [\iotwod.bounding_box.cons] {\tcode{basic_bounding_box} constructors}

\indexlibrary{\idxcode{basic_bounding_box}!constructor}%
\begin{itemdecl}
basic_bounding_box() noexcept;
\end{itemdecl}
\begin{itemdescr}
\pnum
\effects
Constructs an object of type \tcode{basic_bounding_box}.

\pnum
\postconditions
\tcode{data() == GraphicsMath::create_bounding_box()}.
\end{itemdescr}

\indexlibrary{\idxcode{basic_bounding_box}!constructor}%
\begin{itemdecl}
basic_bounding_box(float x, float y, float w, float h) noexcept;
\end{itemdecl}
\begin{itemdescr}
\pnum
\requires
\tcode{w} is not less than \tcode{0.0f} and \tcode{h} is not less than \tcode{0.0f}.

\pnum
\effects
Constructs an object of type \tcode{basic_bounding_box}.

\pnum
\postconditions
\tcode{data() == GraphicsMath::create_bounding_box(x, y, w, h)}.
\end{itemdescr}

\indexlibrary{\idxcode{basic_bounding_box}!constructor}%
\begin{itemdecl}
basic_bounding_box(const basic_point_2d<GraphicsMath>& tl,
  const basic_point_2d<GraphicsMath>& br) noexcept;
\end{itemdecl}
\begin{itemdescr}
\pnum
\effects
Constructs an object of type \tcode{basic_bounding_box}.

\pnum
\postconditions
\tcode{data() == GraphicsMath::create_bounding_box(tl.data(), br.data()}.
\end{itemdescr}

\rSec1 [\iotwod.bounding_box.accessors]{\tcode{basic_bounding_box} accessors}

\indexlibrarymember{data}{basic_bounding_box}%
\begin{itemdecl}
const data_type& data() const noexcept;
data_type& data() noexcept;
\end{itemdecl}
\begin{itemdescr}
\pnum
\returns
A reference to the \tcode{basic_bounding_box} object's data object (See: \ref{\iotwod.bounding_box.intro}).
\end{itemdescr}

\rSec1 [\iotwod.bounding_box.modifiers]{\tcode{basic_bounding_box} modifiers}

\indexlibrarymember{x}{basic_bounding_box}%
\begin{itemdecl}
void x(float v) noexcept;
\end{itemdecl}
\begin{itemdescr}
\pnum
\effects
Equivalent to \tcode{GraphicsMath::x(data(), v);}
\end{itemdescr}

\indexlibrarymember{y}{basic_bounding_box}%
\begin{itemdecl}
void y(float v) noexcept;
\end{itemdecl}
\begin{itemdescr}
\pnum
\effects
Equivalent to \tcode{GraphicsMath::y(data(), v);}
\end{itemdescr}

\indexlibrarymember{width}{basic_bounding_box}%
\begin{itemdecl}
void width(float v) noexcept;
\end{itemdecl}
\begin{itemdescr}
\pnum
\effects
Equivalent to \tcode{GraphicsMath::width(data(), v);}
\end{itemdescr}

\indexlibrarymember{height}{basic_bounding_box}%
\begin{itemdecl}
void height(float val) noexcept;
\end{itemdecl}
\begin{itemdescr}
\pnum
\effects
Equivalent to \tcode{GraphicsMath::height(data(), v);}
\end{itemdescr}

\indexlibrarymember{top_left}{basic_bounding_box}%
\begin{itemdecl}
void top_left(const basic_point_2d<GraphicsMath>& v) noexcept;
\end{itemdecl}
\begin{itemdescr}
\pnum
\effects
Equivalent to \tcode{GraphicsMath::top_left(data(), v.data());}
\end{itemdescr}

\indexlibrarymember{bottom_right}{basic_bounding_box}%
\begin{itemdecl}
void bottom_right(const basic_point_2d<GraphicsMath>& v) noexcept;
\end{itemdecl}
\begin{itemdescr}
\pnum
\effects
Equivalent to \tcode{GraphicsMath::bottom_right(data(), v.data());}
\end{itemdescr}

\rSec1 [\iotwod.bounding_box.observers]{\tcode{basic_bounding_box} observers}

\indexlibrarymember{x}{basic_bounding_box}%
\begin{itemdecl}
float x() const noexcept;
\end{itemdecl}
\begin{itemdescr}
\pnum
\returns
\tcode{GraphicsMath::x(data())}.
\end{itemdescr}

\indexlibrarymember{y}{basic_bounding_box}%
\begin{itemdecl}
float y() const noexcept;
\end{itemdecl}
\begin{itemdescr}
\pnum
\returns
\tcode{GraphicsMath::y(data())}.
\end{itemdescr}

\indexlibrarymember{width}{basic_bounding_box}%
\begin{itemdecl}
float width() const noexcept;
\end{itemdecl}
\begin{itemdescr}
\pnum
\returns
\tcode{GraphicsMath::width(data())}.
\end{itemdescr}

\indexlibrarymember{height}{basic_bounding_box}%
\begin{itemdecl}
float height() const noexcept;
\end{itemdecl}
\begin{itemdescr}
\pnum
\returns
\tcode{GraphicsMath::height(data())}.
\end{itemdescr}

\indexlibrarymember{top_left}{basic_bounding_box}%
\begin{itemdecl}
basic_point_2d<GraphicsMath> top_left() const noexcept;
\end{itemdecl}
\begin{itemdescr}
\pnum
\returns
\tcode{basic_point_2d<GraphicsMath>(GraphicsMath::top_left(data()))}.
\end{itemdescr}

\indexlibrarymember{bottom_right}{basic_bounding_box}%
\begin{itemdecl}
basic_point_2d<GraphicsMath> bottom_right() const noexcept;
\end{itemdecl}
\begin{itemdescr}
\pnum
\returns
\tcode{basic_point_2d<GraphicsMath>(GraphicsMath::bottom_right(data()))}.
\end{itemdescr}

\rSec1 [\iotwod.bounding_box.ops] {\tcode{basic_bounding_box} operators}

\indexlibrarymember{operator==}{basic_bounding_box}%
\begin{itemdecl}
bool operator==(const basic_bounding_box<GraphicsMath>& lhs,
  const basic_bounding_box<GraphicsMath>& rhs) noexcept;
\end{itemdecl}
\begin{itemdescr}
\pnum
\returns
\tcode{GraphicsMath::equal(lhs.data(), rhs.data())}.
\end{itemdescr}

\indexlibrarymember{operator!=}{basic_bounding_box}%
\begin{itemdecl}
bool operator!=(const basic_bounding_box<GraphicsMath>& lhs,
  const basic_bounding_box<GraphicsMath>& rhs) noexcept;
\end{itemdecl}
\begin{itemdescr}
\pnum
\returns
\tcode{GraphicsMath::not_equal(lhs.data(), rhs.data())}.
\end{itemdescr}
