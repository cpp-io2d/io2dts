%!TEX root = io2d.tex
\rSec0 [rectangle] {Class \tcode{rectangle}}

\rSec1 [rectangle.intro] {\tcode{rectangle} description}

\indexlibrary{\idxcode{rectangle}}
\pnum
The class \tcode{rectangle} describes a rectangle.

\pnum
It has an x coordinate of type \tcode{double}, a y coordinate of type \tcode{double}, a width of type \tcode{double}, and a height of type \tcode{double}.

\rSec1 [rectangle.synopsis] {\tcode{rectangle} synopsis}

\begin{codeblock}
namespace std { namespace experimental { namespace io2d { inline namespace v1 {
  class rectangle {
  public:
    // \ref{rectangle.cons}, construct/copy/move/destroy:
    constexpr rectangle() noexcept;
    constexpr rectangle(double x, double y, double width, double height)
      noexcept;
    constexpr rectangle(const vector_2d& tl, const vector_2d& br) noexcept;

    // \ref{rectangle.modifiers}, modifiers:
    constexpr void x(double val) noexcept;
    constexpr void y(double val) noexcept;
    constexpr void width(double val) noexcept;
    constexpr void height(double val) noexcept;
    constexpr void top_left(const vector_2d& val) noexcept;
    constexpr void bottom_right(const vector_2d& val) noexcept;
    
    // \ref{rectangle.observers}, observers:
    constexpr double x() const noexcept;
    constexpr double y() const noexcept;
    constexpr double width() const noexcept;
    constexpr double height() const noexcept;
    constexpr vector_2d top_left() const noexcept;
    constexpr vector_2d bottom_right() const noexcept;
  };
} } } }
\end{codeblock}

\rSec1 [rectangle.cons] {\tcode{rectangle} constructors}

\indexlibrary{\idxcode{rectangle}!constructor}
\begin{itemdecl}
constexpr rectangle() noexcept;
\end{itemdecl}
\begin{itemdescr}
\pnum
\effects
Constructs an object of type \tcode{rectangle}.

\pnum
The x coordinate, y coordinate, width, and height shall each be set to the value \tcode{0.0}.
\end{itemdescr}

\indexlibrary{\idxcode{rectangle}!constructor}
\begin{itemdecl}
constexpr rectangle(double x, double y, double w, double h) noexcept;
\end{itemdecl}
\begin{itemdescr}
\pnum
\effects
Constructs an object of type \tcode{rectangle}.

\pnum
The x coordinate shall be set to the value of \tcode{x}.

\pnum
The y coordinate shall be set to the value of \tcode{y}.

\pnum
The width shall be set to the value of \tcode{w}.

\pnum
The height shall be set to the value of \tcode{h}.
\end{itemdescr}

\indexlibrary{\idxcode{rectangle}!constructor}
\begin{itemdecl}
constexpr rectangle(const vector_2d& tl, const vector_2d& br) noexcept;
\end{itemdecl}
\begin{itemdescr}
\pnum
\effects
Constructs an object of type \tcode{rectangle}.

\pnum
The x coordinate shall be set to the value of \tcode{tl.x()}.

\pnum
The y coordinate shall be set to the value of \tcode{tl.y()}.

\pnum
The width shall be set to the value of \tcode{max(0.0, br.x() - tl.x())}.

\pnum
The height shall be set to the value of \tcode{max(0.0, br.y() - tl.y())}.
\end{itemdescr}

\rSec1 [rectangle.modifiers]{\tcode{rectangle} modifiers}

\indexlibrary{\idxcode{rectangle}!\idxcode{x}}
\begin{itemdecl}
constexpr void x(double val) noexcept;
\end{itemdecl}

\begin{itemdescr}
\pnum
\effects
The x coordinate shall be set to the value of \tcode{val}.
\end{itemdescr}

\indexlibrary{\idxcode{rectangle}!\idxcode{y}}
\begin{itemdecl}
constexpr void y(double value) noexcept;
\end{itemdecl}
\begin{itemdescr}
\pnum
\effects
The y coordinate shall be set to the value of \tcode{val}.
\end{itemdescr}

\indexlibrary{\idxcode{rectangle}!\idxcode{width}}
\begin{itemdecl}
constexpr void width(double value) noexcept;
\end{itemdecl}
\begin{itemdescr}
\pnum
\effects
The width shall be set to the value of \tcode{val}.
\end{itemdescr}

\indexlibrary{\idxcode{rectangle}!\idxcode{height}}
\begin{itemdecl}
constexpr void height(double value) noexcept;
\end{itemdecl}
\begin{itemdescr}
\pnum
\effects
The height shall be set to the value of \tcode{val}.
\end{itemdescr}

\indexlibrary{\idxcode{rectangle}!\idxcode{top_left}}
\begin{itemdecl}
constexpr void top_left(const vector_2d& val) noexcept;
\end{itemdecl}
\begin{itemdescr}
\pnum
\effects
The x coordinate shall be set to the value of \tcode{val.x()}.

\effects
The y coordinate shall be set to the value of \tcode{val.y()}.
\end{itemdescr}

\indexlibrary{\idxcode{rectangle}!\idxcode{bottom_right}}
\begin{itemdecl}
constexpr void bottom_right(const vector_2d& val) noexcept;
\end{itemdecl}
\begin{itemdescr}
\pnum
\effects
The width shall be set to the value of \tcode{max(0.0, val.x() - *this.x())}.

\pnum
The height shall be set to the value of \tcode{max(0.0, value.y() - *this.y())}.
\end{itemdescr}

\rSec1 [rectangle.observers]{\tcode{rectangle} observers}

\indexlibrary{\idxcode{rectangle}!\idxcode{x}}
\begin{itemdecl}
constexpr double x() const noexcept;
\end{itemdecl}
\begin{itemdescr}
\pnum
\returns
The value of the x coordinate.
\end{itemdescr}

\indexlibrary{\idxcode{rectangle}!\idxcode{y}}
\begin{itemdecl}
constexpr double y() const noexcept;
\end{itemdecl}
\begin{itemdescr}
\pnum
\returns
The value of the y coordinate.
\end{itemdescr}

\indexlibrary{\idxcode{rectangle}!\idxcode{width}}
\begin{itemdecl}
constexpr double width() const noexcept;
\end{itemdecl}
\begin{itemdescr}
\pnum
\returns
The value of the width.
\end{itemdescr}

\indexlibrary{\idxcode{rectangle}!\idxcode{height}}
\begin{itemdecl}
constexpr double height() const noexcept;
\end{itemdecl}
\begin{itemdescr}
\pnum
\returns
The value of the height.
\end{itemdescr}

\indexlibrary{\idxcode{rectangle}!\idxcode{top_left}}
\begin{itemdecl}
constexpr vector_2d top_left() const noexcept;
\end{itemdecl}
\begin{itemdescr}
\pnum
\returns
A \tcode{vector_2d} object constructed from the value of the x coordinate as its first argument and the value of the y coordinate as its second argument.
\end{itemdescr}

\indexlibrary{\idxcode{rectangle}!\idxcode{bottom_right}}
\begin{itemdecl}
constexpr vector_2d bottom_right() const noexcept;
\end{itemdecl}
\begin{itemdescr}
\pnum
\returns
A \tcode{vector_2d} object constructed from the value of the width added to the value of the x coordinate as its first argument and the value of the height added to the value of the y coordinate as its second argument.
\end{itemdescr}
