%!TEX root = io2d.tex
\rSec0 [\iotwod.bounding_box] {Class \tcode{basic_bounding_box}}

\rSec1 [\iotwod.bounding_box.intro] {\tcode{basic_bounding_box} description}

\indexlibrary{\idxcode{basic_bounding_box}}%
\pnum
The class template \tcode{basic_bounding_box} describes a bounding_box.

\pnum
It has an \term{x coordinate} of type \tcode{float}, a \term{y coordinate} of type \tcode{float}, a \term{width} of type \tcode{float}, and a \term{height} of type \tcode{float}.

\rSec1 [\iotwod.bounding_box.synopsis] {\tcode{basic_bounding_box} synopsis}

\begin{codeblock}
namespace std::experimental::io2d::v1 {
  template <class GraphicsMath>
  class basic_bounding_box {
  public:
    // \ref{\iotwod.bounding_box.cons}, constructors:
    basic_bounding_box() noexcept;
    basic_bounding_box(float x, float y, float width, float height) noexcept;
    basic_bounding_box(const basic_point_2d<GraphicsMath>& tl,
      const basic_point_2d<GraphicsMath>& br) noexcept;
    basic_bounding_box(const typename GraphicsMath::bounding_box_data_type& val) noexcept;

    // \ref{\iotwod.bounding_box.modifiers}, modifiers:
    void x(float val) noexcept;
    void y(float val) noexcept;
    void width(float val) noexcept;
    void height(float val) noexcept;
    void top_left(const basic_point_2d<GraphicsMath>& val) noexcept;
    void bottom_right(const basic_point_2d<GraphicsMath>& val) noexcept;

    // \ref{\iotwod.bounding_box.observers}, observers:
    float x() const noexcept;
    float y() const noexcept;
    float width() const noexcept;
    float height() const noexcept;
    basic_point_2d<GraphicsMath> top_left() const noexcept;
    basic_point_2d<GraphicsMath> bottom_right() const noexcept;
  };

  // \ref{\iotwod.bounding_box.ops}, operators:
  template <class GraphicsMath>
  bool operator==(const basic_bounding_box<GraphicsMath>& lhs,
    const basic_bounding_box<GraphicsMath>& rhs) noexcept;
  template <class GraphicsMath>
  bool operator!=(const basic_bounding_box<GraphicsMath>& lhs,
    const basic_bounding_box<GraphicsMath>& rhs) noexcept;
}
\end{codeblock}

\rSec1 [\iotwod.bounding_box.cons] {\tcode{basic_bounding_box} constructors}

\indexlibrary{\idxcode{basic_bounding_box}!constructor}%
\begin{itemdecl}
basic_bounding_box() noexcept;
\end{itemdecl}
\begin{itemdescr}
\pnum
\effects
Equivalent to \tcode{basic_bounding_box\{ 0.0f, 0.0f, 0.0f, 0.0f \}}.
\end{itemdescr}

\indexlibrary{\idxcode{basic_bounding_box}!constructor}%
\begin{itemdecl}
basic_bounding_box(float x, float y, float w, float h) noexcept;
\end{itemdecl}
\begin{itemdescr}
\pnum
\requires
\tcode{w} is not less than \tcode{0.0f} and \tcode{h} is not less than \tcode{0.0f}.

\pnum
\effects
Constructs an object of type \tcode{basic_bounding_box}.

\pnum
The x coordinate is \tcode{x}. The y coordinate is \tcode{y}. The width is \tcode{w}. The height is \tcode{h}.
\end{itemdescr}

\indexlibrary{\idxcode{basic_bounding_box}!constructor}%
\begin{itemdecl}
basic_bounding_box(const basic_point_2d<GraphicsMath>& tl,
  const basic_point_2d<GraphicsMath>& br) noexcept;
\end{itemdecl}
\begin{itemdescr}
\pnum
\effects
Constructs an object of type \tcode{basic_bounding_box}.

\pnum
<TODO>The x coordinate is \tcode{tl.x}. The y coordinate is \tcode{tl.y}. The width is \tcode{max(0.0f, br.x - tl.x)}. The height is \tcode{max(0.0f, br.y - tl.y)}.
\end{itemdescr}

\rSec1 [\iotwod.bounding_box.modifiers]{\tcode{basic_bounding_box} modifiers}

\indexlibrarymember{x}{basic_bounding_box}%
\begin{itemdecl}
void x(float val) noexcept;
\end{itemdecl}

\begin{itemdescr}
\pnum
\effects
The x coordinate is \tcode{val}.
\end{itemdescr}

\indexlibrarymember{y}{basic_bounding_box}%
\begin{itemdecl}
void y(float val) noexcept;
\end{itemdecl}
\begin{itemdescr}
\pnum
\effects
The y coordinate is \tcode{val}.
\end{itemdescr}

\indexlibrarymember{width}{basic_bounding_box}%
\begin{itemdecl}
void width(float val) noexcept;
\end{itemdecl}
\begin{itemdescr}
\pnum
\effects
The width is \tcode{val}.
\end{itemdescr}

\indexlibrarymember{height}{basic_bounding_box}%
\begin{itemdecl}
void height(float val) noexcept;
\end{itemdecl}
\begin{itemdescr}
\pnum
\effects
The height is \tcode{val}.
\end{itemdescr}

\indexlibrarymember{top_left}{basic_bounding_box}%
\begin{itemdecl}
void top_left(const basic_point_2d<GraphicsMath>& val) noexcept;
\end{itemdecl}
\begin{itemdescr}
\pnum
\effects
<TODO>The x coordinate is \tcode{val.x}.

\pnum
<TODO>The y coordinate is \tcode{val.y}.
\end{itemdescr}

\indexlibrarymember{bottom_right}{basic_bounding_box}%
\begin{itemdecl}
void bottom_right(const basic_point_2d<GraphicsMath>& val) noexcept;
\end{itemdecl}
\begin{itemdescr}
\pnum
\effects
<TODO>The width is \tcode{max(0.0f, val.x - x())}.

\pnum
<TODO>The height is \tcode{max(0.0f, value.y - y())}.
\end{itemdescr}

\rSec1 [\iotwod.bounding_box.observers]{\tcode{basic_bounding_box} observers}

\indexlibrarymember{x}{basic_bounding_box}%
\begin{itemdecl}
float x() const noexcept;
\end{itemdecl}
\begin{itemdescr}
\pnum
\returns
The x coordinate.
\end{itemdescr}

\indexlibrarymember{y}{basic_bounding_box}%
\begin{itemdecl}
float y() const noexcept;
\end{itemdecl}
\begin{itemdescr}
\pnum
\returns
The y coordinate.
\end{itemdescr}

\indexlibrarymember{width}{basic_bounding_box}%
\begin{itemdecl}
float width() const noexcept;
\end{itemdecl}
\begin{itemdescr}
\pnum
\returns
The width.
\end{itemdescr}

\indexlibrarymember{height}{basic_bounding_box}%
\begin{itemdecl}
float height() const noexcept;
\end{itemdecl}
\begin{itemdescr}
\pnum
\returns
The height.
\end{itemdescr}

\indexlibrarymember{top_left}{basic_bounding_box}%
\begin{itemdecl}
basic_point_2d<GraphicsMath> top_left() const noexcept;
\end{itemdecl}
\begin{itemdescr}
\pnum
\returns
A \tcode{basic_point_2d<GraphicsMath>} object constructed with the x coordinate as its first argument and the y coordinate as its second argument.
\end{itemdescr}

\indexlibrarymember{bottom_right}{basic_bounding_box}%
\begin{itemdecl}
basic_point_2d<GraphicsMath> bottom_right() const noexcept;
\end{itemdecl}
\begin{itemdescr}
\pnum
\returns
A \tcode{basic_point_2d<GraphicsMath>} object constructed with the width added to the x coordinate as its first argument and the height added to the y coordinate as its second argument.
\end{itemdescr}

\rSec1 [\iotwod.bounding_box.ops] {\tcode{basic_bounding_box} operators}

\indexlibrarymember{operator==}{basic_bounding_box}%
\begin{itemdecl}
bool operator==(const basic_bounding_box<GraphicsMath>& lhs,
  const basic_bounding_box<GraphicsMath>& rhs) noexcept;
\end{itemdecl}
\begin{itemdescr}
\pnum
\returns
<TODO>
\begin{codeblock}
lhs.x() == rhs.x() && lhs.y() == rhs.y() &&
lhs.width() == rhs.width() && lhs.height() == rhs.height()
\end{codeblock}
\end{itemdescr}
