%!TEX root = io2d.tex
\rSec0 [rectangle] {Class \tcode{rectangle}}

\rSec1 [rectangle.synopsis] {\tcode{rectangle} synopsis}

\begin{codeblock}
namespace std { namespace experimental { namespace io2d { inline namespace v1 {
  class rectangle {
  public:
    // \ref{rectangle.cons}, construct/copy/move/destroy:
    rectangle() noexcept;
    rectangle(double x, double y, double width, double height) noexcept;
    rectangle(const vector_2d& tl, const vector_2d& br) noexcept;
    rectangle(const rectangle& r) noexcept;
    rectangle& operator=(const rectangle& r) noexcept;
    rectangle(rectangle&& r) noexcept;
    rectangle& operator=(rectangle&& r) noexcept;

    // \ref{rectangle.modifiers}, modifiers:
    void x(double value) noexcept;
    void y(double value) noexcept;
    void width(double value) noexcept;
    void height(double value) noexcept;
    void top_left(const vector_2d& value) noexcept;
    void bottom_right(const vector_2d& value) noexcept;
    
    // \ref{rectangle.observers}, observers:
    double x() const noexcept;
    double y() const noexcept;
    double width() const noexcept;
    double height() const noexcept;
    vector_2d top_left() const noexcept;
    vector_2d bottom_right() const noexcept;
    
  private:
    double _X;      // \expos
    double _Y;      // \expos
    double _Width;  // \expos
    double _Height; // \expos
  };
} } } }
\end{codeblock}

\rSec1 [rectangle.intro] {\tcode{rectangle} Description}

\pnum
\indexlibrary{\idxcode{rectangle}}
The class \tcode{rectangle} describes an object that represents a rectangle.

\rSec1 [rectangle.cons] {\tcode{rectangle} constructors and assignment operators}

\indexlibrary{\idxcode{rectangle}!constructor}
\begin{itemdecl}
rectangle() noexcept;
\end{itemdecl}
\begin{itemdescr}
	\pnum
	\effects
	Constructs an object of type \tcode{rectangle}.
	
	\pnum
	\postconditions
	\tcode{_X == 0.0}.
	
	\pnum
	\tcode{_Y == 0.0}.
	
	\pnum
	\tcode{_Width == 0.0}.
	
	\pnum
	\tcode{_Height == 0.0}.

\end{itemdescr}

\indexlibrary{\idxcode{rectangle}!constructor}
\begin{itemdecl}
rectangle(double x, double y, double w, double h) noexcept;
\end{itemdecl}
\begin{itemdescr}
	\pnum
	\effects
	Constructs an object of type \tcode{rectangle}.
	
	\pnum
	\postconditions
	\tcode{_X == x}.
	
	\pnum
	\tcode{_Y == y}.
	
	\pnum
	\tcode{_Width == w}.
	
	\pnum
	\tcode{_Height == h}.
	
\end{itemdescr}

\indexlibrary{\idxcode{rectangle}!constructor}
\begin{itemdecl}
rectangle(const vector_2d& tl, const vector_2d& br) noexcept;
\end{itemdecl}
\begin{itemdescr}
	\pnum
	\effects
	Constructs an object of type \tcode{rectangle} from a top-left coordinate and a bottom-right coordinate.
	
	\pnum
	\postconditions
	\tcode{_X == tl.x()}.
	
	\pnum
	\tcode{_Y == tl.y()}.
	
	\pnum
	\tcode{_Width == max(0.0, br.x() - tl.x())}.
	
	\pnum
	\tcode{_Height == max(0.0, br.y() - tl.y())}.
	
\end{itemdescr}

\rSec1 [rectangle.modifiers]{\tcode{rectangle} modifiers}

\indexlibrary{\idxcode{rectangle}!\idxcode{x}}
\indexlibrary{\idxcode{x}!\idxcode{rectangle}}
\begin{itemdecl}
void x(double value) noexcept;
\end{itemdecl}

\begin{itemdescr}
	\pnum
	\postconditions
	\tcode{_X == value}.

\end{itemdescr}

\indexlibrary{\idxcode{rectangle}!\idxcode{y}}
\indexlibrary{\idxcode{y}!\idxcode{rectangle}}
\begin{itemdecl}
void y(double value) noexcept;
\end{itemdecl}
\begin{itemdescr}
	\pnum
	\postconditions
	\tcode{_Y == value}.

\end{itemdescr}

\indexlibrary{\idxcode{rectangle}!\idxcode{width}}
\indexlibrary{\idxcode{width}!\idxcode{rectangle}}
\begin{itemdecl}
void width(double value) noexcept;
\end{itemdecl}
\begin{itemdescr}
	\pnum
	\postconditions
	\tcode{_Width == value}.

\end{itemdescr}

\indexlibrary{\idxcode{rectangle}!\idxcode{height}}
\indexlibrary{\idxcode{height}!\idxcode{rectangle}}
\begin{itemdecl}
void height(double value) noexcept;
\end{itemdecl}
\begin{itemdescr}
	\pnum
	\postconditions
	\tcode{_Height == value}.

\end{itemdescr}

\indexlibrary{\idxcode{rectangle}!\idxcode{top_left}}
\indexlibrary{\idxcode{top_left}!\idxcode{rectangle}}
\begin{itemdecl}
void top_left(const vector_2d& value) noexcept;
\end{itemdecl}
\begin{itemdescr}
	\pnum
	\postconditions
	\tcode{_X == value.x()}.
	
	\pnum
	\tcode{_Y == value.y()}.
	
\end{itemdescr}

\indexlibrary{\idxcode{rectangle}!\idxcode{bottom_right}}
\indexlibrary{\idxcode{bottom_right}!\idxcode{rectangle}}
\begin{itemdecl}
void bottom_right(const vector_2d& value) noexcept;
\end{itemdecl}
\begin{itemdescr}
	\pnum
	\postconditions
	\tcode{_Width == max(0.0, value.x() - _X)}.
	
	\pnum
	\tcode{_Height == max(0.0, value.y() - _Y)}.
	
\end{itemdescr}

\rSec1 [rectangle.observers]{\tcode{rectangle} observers}

\indexlibrary{\idxcode{rectangle}!\idxcode{x}}
\indexlibrary{\idxcode{x}!\idxcode{rectangle}}
\begin{itemdecl}
double x() const noexcept;
\end{itemdecl}
\begin{itemdescr}
	\pnum
	\returns
	\tcode{_X}.

\end{itemdescr}

\indexlibrary{\idxcode{rectangle}!\idxcode{y}}
\indexlibrary{\idxcode{y}!\idxcode{rectangle}}
\begin{itemdecl}
double y() const noexcept;
\end{itemdecl}
\begin{itemdescr}
	\pnum
	\returns
	\tcode{_Y}.

\end{itemdescr}

\indexlibrary{\idxcode{rectangle}!\idxcode{width}}
\indexlibrary{\idxcode{width}!\idxcode{rectangle}}
\begin{itemdecl}
double width() const noexcept;
\end{itemdecl}
\begin{itemdescr}
	\pnum
	\returns
	\tcode{_Width}.

\end{itemdescr}

\indexlibrary{\idxcode{rectangle}!\idxcode{height}}
\indexlibrary{\idxcode{height}!\idxcode{rectangle}}
\begin{itemdecl}
double height() const noexcept;
\end{itemdecl}
\begin{itemdescr}
	\pnum
	\returns
	\tcode{_Height}.

\end{itemdescr}

\indexlibrary{\idxcode{rectangle}!\idxcode{top_left}}
\indexlibrary{\idxcode{top_left}!\idxcode{rectangle}}
\begin{itemdecl}
vector_2d top_left() const noexcept;
\end{itemdecl}
\begin{itemdescr}
	\pnum
	\returns
	\tcode{vector_2d\{ _X, _Y \}}.
	
\end{itemdescr}

\indexlibrary{\idxcode{rectangle}!\idxcode{bottom_right}}
\indexlibrary{\idxcode{bottom_right}!\idxcode{rectangle}}
\begin{itemdecl}
vector_2d bottom_right() const noexcept;
\end{itemdecl}
\begin{itemdescr}
	\pnum
	\returns
	\tcode{vector_2d\{ _X + _Width, _Y + _Height \}}.

\end{itemdescr}
