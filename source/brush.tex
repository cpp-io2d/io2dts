%!TEX root = io2d.tex
\rSec0 [brush] {Class \tcode{brush}}

\rSec1 [brush.intro] {\tcode{brush} Description}

\pnum
\indexlibrary{\idxcode{brush}}
The class \tcode{brush} describes an opaque wrapper for a graphics data graphics resource.

\pnum
A \tcode{brush} object is usable with any \tcode{surface} or \tcode{surface}-derived object.

\pnum
A \tcode{brush} object's graphics data is immutable. It is observable only by the effect that it produces when the brush is used as a Source Brush or as a Mask Brush (\ref{surface.rendering.brushes}).

\pnum
A \tcode{brush} object has an immutable \tcode{brush_type} value which indicates which type of brush it is (Table~\ref{tab:brushtype.meanings}).

\pnum
As a result of technological limitations and considerations, \tcode{brush} object's graphics data can have less precision than the data from which it was created.

\pnum
\enterexample
Several graphics and rendering technologies that are currently widely used typically store individual color and alpha channel data as 8-bit unsigned normalized integer values while the \tcode{double} type that is used by the \tcode{rgba_color} class for individual color and alpha is often a 64-bit value. As such, it is possible for a loss of precision when transforming the 64-bit channel data of an \tcode{rgba_color} object to the 8-bit channel data that is commonly used internally in such graphics and rendering technologies.
\exitexample

\rSec1 [brush.synopsis] {\tcode{brush} synopsis}

\begin{codeblock}
namespace std { namespace experimental { namespace io2d { inline namespace v1 {
  class brush {
  public:
    // \ref{brush.cons}, construct/copy/move/destroy:
    explicit brush(const rgba_color& c);
    template <class InputIterator>
    brush(const vector_2d& begin, const vector_2d& end,
      InputIterator first, InputIterator last);
    brush(const vector_2d& begin, const vector_2d& end,
      initializer_list<color_stop> il);
    template <class InputIterator>
    brush(const circle& start, const circle& end,
      InputIterator first, InputIterator last);
    brush(const circle& start, const circle& end,
      initializer_list<color_stop> il);
    explicit brush(image_surface&& img);

    // \ref{brush.observers}, observers:
    brush_type type() const noexcept;
  };
} } } }
\end{codeblock}

\rSec1 [brush.sampling] {Sampling from a \tcode{brush} object}

\pnum
When sampling from a \tcode{brush} object \tcode{b}, the \tcode{brush_type} returned by calling \tcode{b.type()} shall determine how the results of sampling shall be determined:
\begin{enumerate}
\item If the result of \tcode{b.type()} is \tcode{brush_type::solid_color} then \tcode{b} is a \term{solid color brush}.
\item If the result of \tcode{b.type()} is \tcode{brush_type::surface} then \tcode{b} is a \term{surface brush}.
\item If the result of \tcode{b.type()} is \tcode{brush_type::linear} then \tcode{b} is a \term{linear gradient brush}.
\item If the result of \tcode{b.type()} is \tcode{brush_type::radial} then \tcode{b} is a \term{radial gradient brush}.
\end{enumerate}

\rSec2 [brush.sampling.color] {Sampling from a color brush}

\pnum
When \tcode{b} is a color brush, then when sampling from \tcode{b}, the visual data returned shall always be the visual data equivalent \tcode{rgba_code} which was passed in when \tcode{b} was created, regardless of the point which is to be sampled and regardless of the return values of \tcode{b.wrap_mode()}, \tcode{b.filter()}, and \tcode{b.matrix()}.

\rSec2 [brush.sampling.linear] {Sampling from a linear gradient brush}

\pnum
When \tcode{b} is a linear gradient brush, then when sampling from \tcode{b}, the visual data returned shall be from the point \tcode{pt} in the rendered linear gradient, where \tcode{pt} is the return value when passing the point to be sampled to \tcode{b.matrix().transform_coords} and the rendered linear gradient is created as specified by \ref{gradients.linear} and \ref{gradients.sampling}, taking into account the value of \tcode{b.wrap_mode()}.

\rSec2 [brush.sampling.radial] {Sampling from a radial gradient brush}

\pnum
When \tcode{b} is a radial gradient brush, then when sampling from \tcode{b}, the visual data returned shall be from the point \tcode{pt} in the rendered radial gradient, where \tcode{pt} is the return value when passing the point to be sampled to \tcode{b.matrix().transform_coords} and the rendered radial gradient is created as specified by \ref{gradients.radial} and \ref{gradients.sampling}, taking into account the value of \tcode{b.wrap_mode()}.

\rSec2 [brush.sampling.surface] {Sampling from a surface brush}

\pnum
When \tcode{b} is a surface brush, then when sampling from \tcode{b}, the visual data returned shall be from the point \tcode{pt} in the graphics data of the brush, where \tcode{pt} is the return value when passing the point to be sampled to \tcode{b.matrix().transform_coords}, taking into account the value of \tcode{b.wrap_mode()} and \tcode{b.filter()}.

\rSec1 [brush.cons] {\tcode{brush} constructors and assignment operators}

\indexlibrary{\idxcode{brush}!constructor}
\begin{itemdecl}
brush(const rgba_color& c);
brush(const rgba_color& c, error_code& ec) noexcept;
\end{itemdecl}
\begin{itemdescr}
\pnum
\effects
Constructs an object of type \tcode{brush}.

\pnum
The brush shall be a color brush.

\pnum
The brush's brush type shall be set to the value \tcode{brush_type::solid_color}.

\pnum
The brush's wrap mode shall be set to the value \tcode{experimental::io2d::wrap_mode::none}.

\pnum
The brush's filter shall be set to the value \tcode{experimental::io2d::filter::fast}.

\pnum
The brush's transformation matrix shall be set to the value \tcode{matrix_2d::init_identity()}.

\pnum
The graphics data of the brush shall be created from the return value of \tcode{f.color()}. The visual data format of the graphics data shall be as if it is that specified by \tcode{format::argb}.

\pnum
\remarks
Sampling from this brush shall produce the results specified in \ref{brush.sampling.color}.

\pnum
\throws
As specified in Error reporting (\ref{\iotwod.err.report}).

\pnum
\errors
\tcode{errc::not_enough_memory} if there was a failure to allocate memory.

\tcode{io2d_error::invalid_status} if there was a failure to allocate a resource other than memory.
\end{itemdescr}

\indexlibrary{\idxcode{brush}!constructor}
\begin{itemdecl}
template <class Allocator>
brush(const vector_2d& begin, const vector_2d& end,
  const color_stop_group<Allocator>& csg);
template <class Allocator>
brush(const vector_2d& begin, const vector_2d& end,
  const color_stop_group<Allocator>& csg, error_code& ec) noexcept;
\end{itemdecl}
\begin{itemdescr}
\pnum
\effects
Constructs an object of type \tcode{brush}.

\pnum
The brush shall be a linear gradient brush.

\pnum
The brush's brush type shall be set to the value \tcode{brush_type::linear}.

\pnum
The brush's wrap mode shall be set to the value \tcode{experimental::io2d::wrap_mode::none}.

\pnum
The brush's filter shall be set to the value \tcode{experimental::io2d::filter::fast}.

\pnum
The brush's transformation matrix shall be set to the value \tcode{matrix_2d::init_identity()}.

\pnum
The graphics data of the brush is nominally as specified the introductory paragraphs of \ref{gradients} and in \ref{gradients.linear}. Its color stops shall be the values contained in \tcode{csg}. However because the graphics data is not directly observable, it is \unspecnorm what data is stored and how it is stored, provided that the results of sampling from the brush are the same as if the brush's graphics data was stored as specified in the introductory paragraphs of \ref{gradients} and in \ref{gradients.linear}.

\pnum
\remarks
Sampling from this brush shall produce the results specified in \ref{brush.sampling.linear}.

\pnum
\throws
As specified in Error reporting (\ref{\iotwod.err.report}).

\pnum
\errors
\tcode{errc::not_enough_memory} if there was a failure to allocate memory.

\tcode{io2d_error::invalid_status} if there was a failure to allocate a resource other than memory.
\end{itemdescr}

\indexlibrary{\idxcode{brush}!constructor}
\begin{itemdecl}
template <class InputIterator>
brush(const circle& start, const circle& end,
  InputIterator first, InputIterator last);
\end{itemdecl}
\begin{itemdescr}
\pnum
\effects
Constructs an object of type \tcode{brush}.

\pnum
The brush is a radial gradient brush.

\pnum

\pnum
The brush's brush type is \tcode{brush_type::radial}.

\pnum
The graphics data of the brush is nominally as specified the introductory paragraphs of \ref{gradients} and in \ref{gradients.radial}. Its color stops shall be the values contained in \tcode{csg}. However because the graphics data is not directly observable, it is \unspecnorm what data is stored and how it is stored, provided that the results of sampling from the brush are the same as if the brush's graphics data was stored as specified in the introductory paragraphs of \ref{gradients} and in \ref{gradients.radial}.

\pnum
\remarks
Sampling from this brush shall produce the results specified in \ref{brush.sampling.radial}.
\end{itemdescr}

\indexlibrary{\idxcode{brush}!constructor}
\begin{itemdecl}
explicit brush(image_surface&& img);
\end{itemdecl}
\begin{itemdescr}
\pnum
\pnum
\effects
Constructs an object of type \tcode{brush}.

\pnum
The brush is a surface brush.

\pnum
The brush's brush type is \tcode{brush_type::surface}.

\pnum
The graphics data of the brush is nominally the same as the graphics data of the \underlyingimagesurface of \tcode{img}. However because the graphics data is not directly observable, it is \unspecnorm what data is stored and how it is stored.

\pnum
The graphics data of the brush shall be the \underlyingimagesurface of \tcode{img}.

\pnum
\remarks
Sampling from this brush shall produce the results specified in \ref{brush.sampling.surface}.
\end{itemdescr}

\rSec1 [brush.observers]{\tcode{brush} observers}

\indexlibrary{\idxcode{brush}!\idxcode{type}}
\begin{itemdecl}
brush_type type() const noexcept;
\end{itemdecl}
\begin{itemdescr}
\pnum
\returns
The brush's brush type.
\end{itemdescr}
