%!TEX root = io2d.tex
\rSec0 [\iotwod.brush] {Class \tcode{brush}}

\rSec1 [\iotwod.brush.synopsis] {\tcode{brush} synopsis}

\begin{codeblock}
namespace std { namespace experimental { namespace io2d { inline namespace v1 {
  class brush {
  public:
    // construct/copy/move/destroy:
    brush() = delete;
    brush(const brush&) noexcept;
    brush& operator=(const brush&) noexcept;
    brush(brush&& other) noexcept;
    brush& operator=(brush&& other) noexcept;

    // \ref{\iotwod.brush.modifiers}, modifiers:
    void extend(::std::experimental::io2d::extend e) noexcept;
    void filter(::std::experimental::io2d::filter f) noexcept;
    void matrix(const matrix_2d& m) noexcept;

    // \ref{\iotwod.brush.observers}, observers:
    ::std::experimental::io2d::extend extend() const noexcept;
    ::std::experimental::io2d::filter filter() const noexcept;
    matrix_2d matrix() const noexcept;
    brush_type type() const noexcept;
  };

    
    
// \expos
  
} } } }
\end{codeblock}

\rSec1 [\iotwod.brush.intro] {\tcode{brush} Description}

\pnum
\indexlibrary{\idxcode{brush}}
The class \tcode{brush} describes .

\rSec1 [\iotwod.brush.modifiers]{\tcode{brush} modifiers}

\indexlibrary{\idxcode{brush}!\idxcode{}}
\indexlibrary{\idxcode{}!\idxcode{brush}}
\begin{itemdecl}
\end{itemdecl}
\begin{itemdescr}
	\pnum
	\postconditions
	
\end{itemdescr}

\rSec1 [\iotwod.brush.observers]{\tcode{brush} observers}

\indexlibrary{\idxcode{brush}!\idxcode{}}
\indexlibrary{\idxcode{}!\idxcode{brush}}
\begin{itemdecl}
\end{itemdecl}
\begin{itemdescr}
	\pnum
	\returns

\end{itemdescr}
