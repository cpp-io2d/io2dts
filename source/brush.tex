%!TEX root = io2d.tex
\rSec0 [\iotwod.brush] {Class \tcode{basic_brush}}

\rSec1 [\iotwod.brush.intro] {\tcode{basic_brush} summary}

\pnum
\indexlibrary{\idxcode{basic_brush}}%
The class template \tcode{basic_brush} describes an opaque wrapper for graphics data. It takes one type parameter, which is a \graphicssurfacestemplparamnospace.

\pnum
A \tcode{basic_brush} object is usable with any \tcode{basic_image_surface} \tcode{basic_output_surface}, and \tcode{basic_unmanaged_output_surface} object provided that they have the \graphicssurfacestemplparam as the \tcode{basic_brush} object.

\pnum
A \tcode{basic_brush} object's graphics data is immutable. It is observable only by the effect that it produces when the brush is used as a \term{source brush} or as a \term{mask brush} (\ref{\iotwod.surface.rendering.brushes}).

\pnum
A \tcode{basic_brush} object has a brush type of \tcode{brush_type}, which indicates which type of brush it is (Table~\ref{tab:\iotwod.brushtype.meanings}).

\pnum
As a result of technological limitations and considerations, a \tcode{basic_brush} object's graphics data may have less precision than the data from which it was created.

\pnum
The data are stored in an object of type \tcode{typename GraphicsMath::brushes::brush_data_type}.
%\pnum
%\begin{example}
%Several graphics and rendering technologies that are currently widely used typically store individual color and alpha channel data as 8-bit unsigned normalized integer values while the \tcode{float} type that is used by the \tcode{rgba_color} class for individual color and alpha is often a 64-bit value. As such, it is possible for a loss of precision when transforming the 64-bit channel data of an \tcode{rgba_color} object to the 8-bit channel data that is commonly used internally in such graphics and rendering technologies.
%\end{example}
%
\rSec1 [\iotwod.brush.synopsis] {\tcode{basic_brush} synopsis}

\begin{codeblock}
namespace std::experimental::io2d::v1 {
template <class GraphicsSurfaces>
class basic_brush {
  public:
    using graphics_math_type = typename GraphicsSurfaces::graphics_math_type;
    using data_type = typename GraphicsSurfaces::brushes::brush_data_type;
    explicit basic_brush(const rgba_color& c);

    template <class InputIterator>
    basic_brush(const basic_point_2d<graphics_math_type>& begin,
      const basic_point_2d<graphics_math_type>& end,
      InputIterator first, InputIterator last);

    basic_brush(const basic_point_2d<graphics_math_type>& begin,
      const basic_point_2d<graphics_math_type>& end,
      ::std::initializer_list<gradient_stop> il);

    template <class InputIterator>
    basic_brush(const basic_circle<graphics_math_type>& start,
      const basic_circle<graphics_math_type>& end,
      InputIterator first, InputIterator last);

    basic_brush(const basic_circle<graphics_math_type>& start,
      const basic_circle<graphics_math_type>& end,
      ::std::initializer_list<gradient_stop> il);

    basic_brush(basic_image_surface<GraphicsSurfaces>&& img);
    
    const data_type& data() const noexcept;
    brush_type type() const noexcept;
  };
}
\end{codeblock}

\rSec1 [\iotwod.brush.sampling] {Sampling from a \tcode{basic_brush} object}

\pnum
A \tcode{basic_brush} object is sampled from either as a source brush (\ref{\iotwod.surface.rendering.brushes}) or a mask brush (\ref{\iotwod.surface.rendering.brushes}).

\pnum
If it is being sampled from as a source brush, its \term{wrap mode}, \term{filter}, and \term{brush matrix} are defined by a \tcode{basic_brush_props} object (\ref{\iotwod.surface.rendering.commonstate} and \ref{\iotwod.surface.rendering.statedefaults}).

\pnum
If it is being sampled from as a mask brush, its wrap mode, filter, and \term{mask matrix} are defined by a \tcode{basic_mask_props} object (\ref{\iotwod.surface.rendering.specificstate} and \ref{\iotwod.surface.rendering.statedefaults}).

\pnum
When sampling from a \tcode{basic_brush} object \tcode{b}, the \tcode{brush_type} returned by calling \tcode{b.type()} determines how the results of sampling are determined:
\begin{enumerate}
\item If the result of \tcode{b.type()} is \tcode{brush_type::solid_color} then \tcode{b} is a \term{solid color brush}.
\item If the result of \tcode{b.type()} is \tcode{brush_type::surface} then \tcode{b} is a \term{surface brush}.
\item If the result of \tcode{b.type()} is \tcode{brush_type::linear} then \tcode{b} is a \term{linear gradient brush}.
\item If the result of \tcode{b.type()} is \tcode{brush_type::radial} then \tcode{b} is a \term{radial gradient brush}.
\end{enumerate}

\rSec2 [\iotwod.brush.sampling.color] {Sampling from a solid color brush}

\pnum
When \tcode{b} is a solid color brush, then when sampling from \tcode{b}, the visual data returned is always the visual data used to construct \tcode{b}, regardless of the point which is to be sampled and regardless of the return values of wrap mode, filter, and brush matrix or mask matrix.

\rSec2 [\iotwod.brush.sampling.linear] {Sampling from a linear gradient brush}

\pnum
When \tcode{b} is a linear gradient brush, when sampling point \tcode{pt}, where \tcode{pt} is the return value of calling the \tcode{transform_pt} member function of brush matrix or mask matrix using the requested point, from \tcode{b}, the visual data returned are as specified by \ref{\iotwod.gradients.linear} and \ref{\iotwod.gradients.sampling}.

\rSec2 [\iotwod.brush.sampling.radial] {Sampling from a radial gradient brush}

\pnum
When \tcode{b} is a radial gradient brush, when sampling point \tcode{pt}, where \tcode{pt} is the return value of calling the \tcode{transform_pt} member function of brush matrix or mask matrix using the requested point, from \tcode{b}, the visual data are as specified by \ref{\iotwod.gradients.radial} and \ref{\iotwod.gradients.sampling}.

\rSec2 [\iotwod.brush.sampling.surface] {Sampling from a surface brush}

\pnum
When \tcode{b} is a surface brush, when sampling point \tcode{pt} from \tcode{b}, where \tcode{pt} is the return value of calling the \tcode{transform_pt} member function of the brush matrix or mask matrix using the requested point, the visual data returned are from the point \tcode{pt} in the graphics data of the brush, as modified by the values of wrap mode (\ref{\iotwod.wrapmode}) and filter (\ref{\iotwod.filter}).

\rSec1 [\iotwod.brush.cons] {\tcode{basic_brush} constructors and assignment operators}

\indexlibrary{\idxcode{basic_brush}!constructor}%
\begin{itemdecl}
explicit basic_brush(const rgba_color& c);<TODO><Implementation mismatch>
\end{itemdecl}
\begin{itemdescr}
\pnum
\effects
Constructs an object of type \tcode{basic_brush}.

\pnum
\postconditions
\tcode{data() == GraphicsSurfaces::brushes::create_brush(c)}.

\pnum
\remarks
The visual data format of the graphics data are as-if it is that specified by \tcode{format::argb32}.

\pnum
Sampling from the brush produces the results specified in \ref{\iotwod.brush.sampling.color}.
\end{itemdescr}

\indexlibrary{\idxcode{basic_brush}!constructor}%
\begin{itemdecl}
template <class InputIterator>
basic_brush(const basic_point_2d<graphics_math_type>& begin,
  const basic_point_2d<graphics_math_type>& end,
  InputIterator first, InputIterator last);
\end{itemdecl}
\begin{itemdescr}
\pnum
\effects
Constructs an object of type \tcode{basic_brush}.

\pnum
\postconditions
\tcode{data() == GraphicsSurfaces::brushes::create_brush(begin, end, first, last)}.

\pnum
\remarks
Sampling from this brush produces the results specified in \ref{\iotwod.brush.sampling.linear}.
\end{itemdescr}

\indexlibrary{\idxcode{basic_brush}!constructor}%
\begin{itemdecl}
basic_brush(const basic_point_2d<graphics_math_type>& begin,
  const basic_point_2d<graphics_math_type>& end,
  initializer_list<gradient_stop> il);
\end{itemdecl}
\begin{itemdescr}
\pnum
\effects
Constructs an object of type \tcode{basic_brush}.

\pnum
\postconditions
\tcode{data() == GraphicsSurfaces::brushes::create_brush(begin, end, il)}.

\pnum
\remarks
Sampling from this brush produces the results specified in \ref{\iotwod.brush.sampling.linear}.
\end{itemdescr}

\indexlibrary{\idxcode{basic_brush}!constructor}%
\begin{itemdecl}
template <class InputIterator>
basic_brush(const basic_circle<graphics_math_type>& start,
  const basic_circle<graphics_math_type>& end,
  InputIterator first, InputIterator last);
\end{itemdecl}
\begin{itemdescr}
\pnum
\effects
Constructs an object of type \tcode{basic_brush}.

\pnum
\postconditions
\tcode{data() == GraphicsSurfaces::brushes::create_brush(start, end, first, last)}.

\pnum
\remarks
Sampling from this brush produces the results specified in \ref{\iotwod.brush.sampling.radial}.
\end{itemdescr}

\indexlibrary{\idxcode{basic_brush}!constructor}%
\begin{itemdecl}
basic_brush(const basic_circle<graphics_math_type>& start,
  const basic_circle<graphics_math_type>& end,
  initializer_list<gradient_stop> il);
\end{itemdecl}
\begin{itemdescr}
\pnum
\effects
Constructs an object of type \tcode{basic_brush}.

\pnum
\postconditions
\tcode{data() == GraphicsSurfaces::brushes::create_brush(start, end, il)}.

\pnum
\remarks
Sampling from this brush produces the results specified in \ref{\iotwod.brush.sampling.radial}.
\end{itemdescr}

\indexlibrary{\idxcode{basic_brush}!constructor}%
\begin{itemdecl}
basic_brush(basic_image_surface<GraphicsSurfaces>&& img);
\end{itemdecl}
\begin{itemdescr}
\pnum
\effects
Constructs an object of type \tcode{basic_brush}.

\pnum
\postconditions
\tcode{data() == GraphicsSurfaces::brushes::create_brush(move(img))}.

\pnum
Sampling from this brush produces the results specified in \ref{\iotwod.brush.sampling.surface}.
\end{itemdescr}

\rSec1 [\iotwod.brush.observers]{\tcode{basic_brush} observers}

\indexlibrarymember{data}{basic_brush}
\begin{itemdecl}
const data_type& data() const noexcept;
\end{itemdecl}
\begin{itemdescr}
\pnum
\returns
A reference to the \tcode{basic_brush} object's data object (See \ref{\iotwod.brush.intro}).

\pnum
\remarks
This function shall only be called by other functions of this library; no diagnostic is required.
\end{itemdescr}

\indexlibrary{\idxcode{basic_brush}!\idxcode{type}}%
\begin{itemdecl}
brush_type type() const noexcept;
\end{itemdecl}
\begin{itemdescr}
\pnum
\returns
\tcode{GraphicsSurfaces::brushes::get_brush_type(data())}.
\end{itemdescr}
