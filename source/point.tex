%!TEX root = io2d.tex
\rSec0 [\iotwod.point] {Class \tcode{point}}

\rSec1 [\iotwod.point.synopsis] {\tcode{point} synopsis}

\begin{codeblock}
namespace std { namespace experimental { namespace io2d { inline namespace v1 {
  class point {
  public:
    // \ref{\iotwod.point.cons}, construct/copy/move/destroy:
    point() noexcept;
    point(double x, double y) noexcept;
    point(const point&) noexcept;
    point& operator=(const point&) noexcept;
    point(point&&) noexcept;
    point& operator=(point&&) noexcept;

    // \ref{\iotwod.point.modifiers}, modifiers:
    void x(double val) noexcept;
    void y(double val) noexcept;
    
    // \ref{\iotwod.point.observers}, observers:
    double x() const noexcept;
    double y() const noexcept;
    
    // \ref{\iotwod.point.member.ops}, member operators:
    point& operator+=(const point& rhs) noexcept;
    point& operator-=(const point& rhs) noexcept;
    
  private:
    double _X; // \expos
    double _Y; // \expos
  };
  
  // \ref{\iotwod.point.ops}, non-member operators:
  bool operator==(const point& lhs, const point& rhs) noexcept;
  bool operator!=(const point& lhs, const point& rhs) noexcept;
  point operator+(const point& lhs) noexcept;
  point operator+(const point& lhs, const point& rhs) noexcept;
  point operator-(const point& lhs) noexcept;
  point operator-(const point& lhs, const point& rhs) noexcept;
} } } }
\end{codeblock}

\rSec1 [\iotwod.point.intro] {\tcode{point} Description}

\pnum
\indexlibrary{\idxcode{point}}%
The class \tcode{point} describes an object that stores a two-dimensional coordinate.

\rSec1 [\iotwod.point.cons] {\tcode{point} constructors and assignment operators}

\indexlibrary{\idxcode{point}!constructor}
\begin{itemdecl}
point() noexcept;
\end{itemdecl}
\begin{itemdescr}
	\pnum
	\effects
	Constructs an object of type \tcode{point}.
	
	\pnum
	\postconditions
	\tcode{_X == 0.0 \&\& _Y == 0.0}.
\end{itemdescr}

\indexlibrary{\idxcode{point}!constructor}
\begin{itemdecl}
point(double x, double y) noexcept;
\end{itemdecl}
\begin{itemdescr}
	\pnum
	\effects
	Constructs an object of type \tcode{point}.
	
	\pnum
	\postconditions
	\tcode{_X == x \&\& _Y == y}.
\end{itemdescr}
	
\rSec1 [\iotwod.point.modifiers]{\tcode{point} modifiers}

\indexlibrary{\idxcode{point}!\idxcode{x}}%
\indexlibrary{\idxcode{x}!\idxcode{point}}%
\begin{itemdecl}
void x(double val) noexcept;
\end{itemdecl}

\begin{itemdescr}
	\pnum
	\postconditions
	\tcode{_X == val}.
	
\end{itemdescr}

\indexlibrary{\idxcode{point}!\idxcode{y}}%
\indexlibrary{\idxcode{y}!\idxcode{point}}%
\begin{itemdecl}
    void y(double val) noexcept;
\end{itemdecl}
\begin{itemdescr}
	\pnum
	\postconditions
	\tcode{_Y == val}.
	
\end{itemdescr}

\rSec1 [\iotwod.point.observers]{\tcode{point} observers}

\indexlibrary{\idxcode{point}!\idxcode{x}}%
\indexlibrary{\idxcode{x}!\idxcode{point}}%
\begin{itemdecl}
    double x() const noexcept;
\end{itemdecl}
\begin{itemdescr}
	\pnum
	\returns
	\tcode{_X}.
\end{itemdescr}

\indexlibrary{\idxcode{point}!\idxcode{y}}%
\indexlibrary{\idxcode{y}!\idxcode{point}}%
\begin{itemdecl}
    double y() const noexcept;
\end{itemdecl}
\begin{itemdescr}
	\pnum
	\returns
	\tcode{_Y}.
\end{itemdescr}

\rSec1 [\iotwod.point.member.ops] {\tcode{point} member operators}

\indexlibrary{\idxcode{point}!\idxcode{operator+=}}%
\indexlibrary{\idxcode{operator+=}!\idxcode{point}}%
\begin{itemdecl}
	point& operator+=(const point& rhs);
\end{itemdecl}
\begin{itemdescr}
	\pnum
	\effects
	\tcode{*this = *this + rhs}.
	
	\pnum
	\returns
	\tcode{*this}.
\end{itemdescr}

\indexlibrary{\idxcode{point}!\idxcode{operator-=}}%
\indexlibrary{\idxcode{operator-=}!\idxcode{point}}%
\begin{itemdecl}
	point& operator-=(const point& rhs);
\end{itemdecl}
\begin{itemdescr}
	\pnum
	\effects
	\tcode{*this = *this - rhs}.
	
	\pnum
	\returns
	\tcode{*this}.
\end{itemdescr}

\rSec1 [\iotwod.point.ops] {\tcode{point} non-member operators}

\indexlibrary{\idxcode{point}!\idxcode{operator==}}%
\indexlibrary{\idxcode{operator==}!\idxcode{point}}%
\begin{itemdecl}
	bool operator==(const point& lhs, const point& rhs);
\end{itemdecl}
\begin{itemdescr}
	\pnum
	\returns
	\tcode{lhs.x() == rhs.x() \&\& lhs.y() == rhs.y()}.
\end{itemdescr}

\indexlibrary{\idxcode{point}!\idxcode{operator!=}}%
\indexlibrary{\idxcode{operator!=}!\idxcode{point}}%
\begin{itemdecl}
	bool operator!=(const point& lhs, const point& rhs);
\end{itemdecl}
\begin{itemdescr}
	\pnum
	\returns
	\tcode{!(lhs == rhs)}.
\end{itemdescr}

\indexlibrary{\idxcode{point}!\idxcode{operator+}}%
\indexlibrary{\idxcode{operator+}!\idxcode{point}}%
\begin{itemdecl}
point operator+(const point& lhs);
\end{itemdecl}
\begin{itemdescr}
	\pnum
	\returns
	\tcode{point(lhs)}.
\end{itemdescr}

\indexlibrary{\idxcode{point}!\idxcode{operator+}}%
\indexlibrary{\idxcode{operator+}!\idxcode{point}}%
\begin{itemdecl}
point operator+(const point& lhs, const point& rhs);
\end{itemdecl}
\begin{itemdescr}
	\pnum
	\returns
	\tcode{point\{ lhs.x() + rhs.x(), lhs.y() + rhs.y() \}}.
\end{itemdescr}

\indexlibrary{\idxcode{point}!\idxcode{operator-}}%
\indexlibrary{\idxcode{operator-}!\idxcode{point}}%
\begin{itemdecl}
point operator-(const point& lhs);
\end{itemdecl}
\begin{itemdescr}
	\pnum
	\returns
	\tcode{point\{ -lhs.x(), -lhs.y() \}}.
\end{itemdescr}

\indexlibrary{\idxcode{point}!\idxcode{operator-}}%
\indexlibrary{\idxcode{operator-}!\idxcode{point}}%
\begin{itemdecl}
point operator-(const point& lhs, const point& rhs);
\end{itemdecl}
\begin{itemdescr}
	\pnum
	\returns
	\tcode{point\{ lhs.x() - rhs.x(), lhs.y() - rhs.y() \}}.
\end{itemdescr}
