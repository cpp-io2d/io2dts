%!TEX root = io2d.tex
\rSec0 [pathgeometries] {Path geometries}

\rSec1 [pathgeometries.overview] {Overview of path geometries}

\pnum
Path geometries are most easily formed using a \tcode{path_factory} object.

\pnum
They may also be formed by directly creating and manipulating a \tcode{vector<path_data_item>} object.

\pnum
A path geometry may contain degenerate path segments.

\pnum
There are special rules concerning the rendering of degenerate path segments. As such they shall be added to a path geometry when requested and shall not be removed from a path geometry when processing it.

\pnum
A \tcode{path} object is an immutable resource wrapper containing a path geometry graphics resource (\ref{path}).

\rSec1 [pathgeometries.processing] {Processing path geometries}

\pnum
\ref{pathgeometries.processing} describes how a properly formed \tcode{vector<path_data_item>} object shall be interpreted by an implementation in order to convert it into a path geometry graphics resource.

\pnum
The \tcode{native_geometry_collection} class, described below, is used for expository purposes only. It provides a hypothetical interface for forming a path geometry graphics resource object. It is not normative.

\pnum
The \tcode{native_geometry_collection} class has the following state data, the types of which are unspecified:

\begin{libreqtab2}
 {\tcode{native_geometry_collection} state data}
 {tab:pathgeometries.processing.natgeocoll}
 \\ \topline
 \lhdr{Name}
 & \rhdr{Use}
 \\ \capsep
 \endfirsthead
 \continuedcaption\\
 \hline
 \lhdr{Name}
 & \rhdr{Use}
 \\ \capsep
 \endhead
 Current Point
 & The start point for a path segment that is added to the Current Path Geometry.
 \\
 Close Point
 & The start point for the initial path segment in the Current Path Geometry.
 \\
 Current Path Geometry
 & The path geometry to which path segments are added.
 \\
 Collection
 & The collection of all path geometries added to the \tcode{native_geometry_collection} object. A new path geometry that is added to the collection is added to the end of the collection.
 \\
\end{libreqtab2}

\begin{codeblock}
// \textit{This class is }\expos
class native_geometry_collection {
public:
  void current_point(const vector_2d& pt) noexcept;
  void close_point(const vector_2d& pt) noexcept;
  void line_to(const vector_2d& pt) noexcept;
  void curve_to(const vector_2d& control1, const vector_2d& control2,
    const vector_2d& endPt) noexcept;
  void close_path() noexcept;
};
\end{codeblock}

\begin{itemdecl}
  void current_point(const vector_2d& pt) noexcept;
\end{itemdecl}
\begin{itemdescr}
	\pnum
	\effects
	If the last member function called was not \tcode{current_point} or if the Collection contains no path geometries, a new path geometry is created, added to the Collection, and set as the Current Path Geometry.
	
	\pnum
	If a new path geometry is created and the Collection contained at least one path geometry prior to this member function being called, then unless the last member function called was \tcode{close_path}, the previous Current Path Geometry shall be an open path geometry.
	
	\pnum
	Sets \tcode{pt} as the Current Point.
\end{itemdescr}

\begin{itemdecl}
  void close_point(const vector_2d& pt) noexcept;
\end{itemdecl}
\begin{itemdescr}
	\pnum
	\preconditions
	There is a Current Path Geometry.
	
	\pnum
	\effects
	Sets \tcode{pt} as the Close Point.

\end{itemdescr}

\begin{itemdecl}
  void line_to(const vector_2d& pt) noexcept;
\end{itemdecl}
\begin{itemdescr}
	\pnum
	\preconditions
	There is a Current Path Geometry.
	
	\pnum
	\effects
	Creates a line segment from the Current Point to \tcode{pt} and adds it to the Current Path Geometry.

\end{itemdescr}

\begin{itemdecl}
  void curve_to(const vector_2d& cpt1, const vector_2d& cpt2,
    const vector_2d& endPt) noexcept;
\end{itemdecl}
\begin{itemdescr}
	\pnum
	\preconditions
	There is a Current Path Geometry.
	
	\pnum
	\effects
	Creates a cubic B\'ezier curve from the Current Point to \tcode{endPt} using \tcode{cpt1} as the first control point and \tcode{cpt2} as the second control point and adds it to the Current Path Geometry.

\end{itemdescr}

\begin{itemdecl}
  void close_path() noexcept;
\end{itemdecl}
\begin{itemdescr}
	\pnum
	\preconditions
	There is a Current Path Geometry.
	
	\pnum
	\effects
	Creates a line segment from the Current Point to the Close Point.
	
	\pnum
	The Current Path Geometry becomes a closed path geometry.

\end{itemdescr}

\pnum
\enternote
A path geometry graphics resource that only supports rendering triangles is possible. The triangles would be used to form lines and to approximate curves. This description assumes the existence of a path geometry graphics resource that performs those actions where needed.
\exitnote

\pnum
The following code shows how to properly process \tcode{vector<path_data_item> p} and store the results into \tcode{native_geometry_collection n}:

\begin{codeblock}
const double pi =     3.1415926535897932384626433832795;
const double halfpi = 1.57079632679489661923132169163985;
const double twopi =  6.283185307179586476925286766559;
matrix_2d m;
vector_2d origin;
vector_2d currentPoint;
bool hasCurrentPoint = false;
vector_2d closePoint;

for (const auto& item : p) {
  switch(item.type()) {
  case path_data_type::move_to:
  {
    currentPoint = item.get<path_data_item::move_to>().to();
    auto pt = m.transform_point(currentPoint - origin) + origin;
    n.current_point(pt);
    hasCurrentPoint = true;
    closePoint = pt
    n.close_point(pt);
  } break;
  case path_data_type::line_to:
  {
    currentPoint = item.get<path_data_item::line_to>().to();
    auto pt = m.transform_point(currentPoint - origin) + origin;
    if (hasCurrentPoint) {
      n.line_to(pt);
    }
    else {
      n.current_point(pt);
      hasCurrentPoint = true;
      closePoint = pt;
      n.close_point(pt);
    }
  } break;
  case path_data_type::curve_to:
  {
    auto cd = item.get<path_data_item::curve_to>();
    auto pt1 = m.transform_point(cd.control_point_1() - origin) + origin;
    auto pt2 = m.transform_point(cd.control_point_2() - origin) + origin;
    auto pt3 = m.transform_point(cd.end_point() - origin) + origin;
    if (!hasCurrentPoint) {
      currentPoint = cd.control_point_1();
      n.current_point(pt1);
      hasCurrentPoint = true;
      closePoint = pt1;
      n.close_point(pt1);
    }
    n.curve_to(pt1, pt2, pt3);
    currentPoint = cd.end_point();
  } break;
  case path_data_type::new_sub_path:
  {
    hasCurrentPoint = false;
  } break;
  case path_data_type::close_path:
  {
    if (!hasCurrentPoint) {
      break;
    }
    n.close_path();
    n.current_point(closePoint);
    // Invert can error so use correct overload; here is the throw version.
    auto invM = matrix_2d{m}.invert();
    // Need the untransformed value for currentPoint.
    currentPoint = invM.transform_point(closePoint - origin) + origin;
  } break;
  case path_data_type::rel_move_to:
  {
    // If !hasCurrentPoint, error is io2d_error::no_current_point;
    currentPoint = item.get<path_data_item::rel_move_to>().to() + currentPoint;
    auto pt = m.transform_point(currentPoint - origin) + origin;
    n.current_point(pt);
    hasCurrentPoint = true;
    closePoint = pt    
    n.close_point(pt);
  } break;
  case path_data_type::rel_line_to:
  {
    // If !hasCurrentPoint, error is io2d_error::no_current_point;
    currentPoint = item.get<path_data_item::rel_line_to>().to() + currentPoint;
    auto pt = m.transform_point(currentPoint - origin) + origin;
    n.line_to(pt);
  } break;
  case path_data_type::rel_curve_to:
  {
    // If !hasCurrentPoint, error is io2d_error::no_current_point;
    auto cd = item.get<path_data_item::rel_curve_to>();
    auto pt1 = m.transform_point(cd.control_point_1() + currentPoint -
      origin) + origin;
    auto pt2 = m.transform_point(cd.control_point_2() + currentPoint -
      origin) + origin;
    auto pt3 = m.transform_point(cd.end_point() + currentPoint - origin) +
      origin;
    n.curve_to(pt1, pt2, pt3);
    currentPoint = cd.end_point() + currentPoint;
  } break;
  case path_data_type::arc:
  {
    auto ad = item.get<path_factory::path_arc_clockwise>();
    auto ctr = ad.center();
    auto rad = ad.radius();
    auto ang1 = ad.angle_1();
    auto ang2 = ad.angle_2();
    while(ang2 < ang1) {
      ang2 += twopi;
    }
    vector_2d pt0, pt1, pt2, pt3;
    int bezCount = 1;
    double theta = ang2 - ang1;
    double phi;
    while (theta >= halfpi) {
      theta /= 2.0;
      bezCount += bezCount;
    }
    phi = theta / 2.0;
    auto cosPhi = cos(phi);
    auto sinPhi = sin(phi);
    pt0.x(cosPhi);
    pt0.y(-sinPhi);
    pt3.x(pt0.x());
    pt3.y(-pt0.y());
    pt1.x((4.0 - cosPhi) / 3.0);
    pt1.y(-(((1.0 - cosPhi) * (3.0 - cosPhi)) / (3.0 * sinPhi)));
    pt2.x(pt1.x());
    pt2.y(-pt1.y());
    phi = -phi;
    auto rotCwFn = [](const vector_2d& pt, double a) -> vector_2d {
      return { pt.x() * cos(a) + pt.y() * sin(a),
        -(pt.x() * -(sin(a)) + pt.y() * cos(a)) };
    };
    pt0 = rotCwFn(pt0, phi);
    pt1 = rotCwFn(pt1, phi);
    pt2 = rotCwFn(pt2, phi);
    pt3 = rotCwFn(pt3, phi);
    
    auto currTheta = ang1;
    const auto startPt =
      ctr + rotCwFn({ pt0.x() * rad, pt0.y() * rad }, currTheta);
    if (hasCurrentPoint) {
      currentPoint = startPt;
      auto pt = m.transform_point(currentPoint - origin) + origin;
      n.line_to(pt);
    }
    else {
      currentPoint = startPt;
      auto pt = m.transform_point(currentPoint - origin) + origin;
      n.current_point(pt);
      hasCurrentPoint = true;
      closePt = pt;
      n.close_point(pt);
    }
    for (; bezCount > 0; bezCount--) {
      auto cpt1 = ctr + rotCwFn({ pt1.x() * rad, pt1.y() * rad }, currTheta);
      auto cpt2 = ctr + rotCwFn({ pt2.x() * rad, pt2.y() * rad }, currTheta);
      auto cpt3 = ctr + rotCwFn({ pt3.x() * rad, pt3.y() * rad }, currTheta);
      currentPoint = cpt3;
      cpt1 = m.transform_point(cpt1 - origin) + origin;
      cpt2 = m.transform_point(cpt2 - origin) + origin;
      cpt3 = m.transform_point(cpt3 - origin) + origin;
      n.curve_to(cpt1, cpt2, cpt3);
      currTheta += theta;
    }
  } break;
  case path_data_type::arc_negative:
  {
    auto ad = item.get<path_factory::path_arc_counterclockwise>();
    auto ctr = ad.center();
    auto rad = ad.radius();
    auto ang1 = ad.angle_1();
    auto ang2 = ad.angle_2();
    while(ang2 > ang1) {
      ang2 -= twopi;
    }
    vector_2d pt0, pt1, pt2, pt3;
    int bezCount = 1;
    double theta = ang1 - ang2;
    double phi;
    while (theta >= halfpi) {
      theta /= 2.0;
      bezCount += bezCount;
    }
    phi = theta / 2.0;
    auto cosPhi = cos(phi);
    auto sinPhi = sin(phi);
    pt0.x(cosPhi);
    pt0.y(-sinPhi);
    pt3.x(pt0.x());
    pt3.y(-pt0.y());
    pt1.x((4.0 - cosPhi) / 3.0);
    pt1.y(-(((1.0 - cosPhi) * (3.0 - cosPhi)) / (3.0 * sinPhi)));
    pt2.x(pt1.x());
    pt2.y(-pt1.y());
    auto rotCwFn = [](const vector_2d& pt, double a) -> vector_2d {
      return { pt.x() * cos(a) + pt.y() * sin(a),
        -(pt.x() * -(sin(a)) + pt.y() * cos(a)) };
    };
    pt0 = rotCwFn(pt0, phi);
    pt1 = rotCwFn(pt1, phi);
    pt2 = rotCwFn(pt2, phi);
    pt3 = rotCwFn(pt3, phi);
    auto shflPt = pt3;
    pt3 = pt0;
    pt0 = shflPt;
    shflPt = pt2;
    pt2 = pt1;
    pt1 = shflPt;
    auto currTheta = ang1;
    const auto startPt =
      ctr + rotCwFn({ pt0.x() * rad, pt0.y() * rad }, currTheta);
    if (hasCurrentPoint) {
      currentPoint = startPt;
      auto pt = m.transform_point(currentPoint - origin) + origin;
      n.line_to(pt);
    }
    else {
      currentPoint = startPt;
      auto pt = m.transform_point(currentPoint - origin) + origin;
      n.current_point(pt);
      hasCurrentPoint = true;
      closePt = pt;
      n.close_point(pt);
    }
    for (; bezCount > 0; bezCount--) {
      auto cpt1 = ctr + rotCwFn({ pt1.x() * rad, pt1.y() * rad }, currTheta);
      auto cpt2 = ctr + rotCwFn({ pt2.x() * rad, pt2.y() * rad }, currTheta);
      auto cpt3 = ctr + rotCwFn({ pt3.x() * rad, pt3.y() * rad }, currTheta);
      currentPoint = cpt3;
      cpt1 = m.transform_point(cpt1 - origin) + origin;
      cpt2 = m.transform_point(cpt2 - origin) + origin;
      cpt3 = m.transform_point(cpt3 - origin) + origin;
      n.curve_to(cpt1, cpt2, cpt3);
      currTheta -= theta;
    }
  } break;
  case path_data_type::change_matrix:
  {
    m = item.get<path_data_item::change_matrix>().matrix();
  } break;
  case path_data_type::change_origin:
  {
    origin = item.get<path_data_item::change_origin>().origin();
  } break;
  }
}
\end{codeblock}

%\rSec1 [pathgeometries.strokerules] {Stroking path geometries}
%
%\pnum
%The following rules shall apply when a Stroking operation (\ref{surface.stroking}) is carried out on a path geometry.
%
%\begin{enumerate}
%\item If the path geometry only contains a degenerate path segment, then if the \tcode{line_cap} value is
%\end{enumerate}
%
%\begin{enumerate}
%	\item Except as otherwise specified in these rules, the start point and end point of a path segment shall be rendered as specified by the meaning of the surface's current \tcode{line_cap} value (\ref{linecap}).
%	
%	\item If the end point of a path segment \textit{a} is set as the current point and is then used as the start point of another path segment, \textit{b}, the point where \tcode{a}'s end point meets \tcode{b}'s start point shall be rendered as specified by the meaning of the surface's current \tcode{line_join} value (\ref{linejoin}).
%	
%	\item ***FIXME***
%\end{enumerate}
%
%\rSec1 [pathgeometries.fillrules] {Filling path geometries}
%
%\pnum
%***FIXME***
