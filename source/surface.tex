%!TEX root = io2d.tex
\rSec0 [\iotwod.surface] {Overview of surface classes}

\rSec1 [\iotwod.surface.intro] {Surface class templates description}

\pnum
\indexlibrary{\idxcode{surface}}%
There are three surface class templates:
\begin{itemize}
\item \tcode{basic_image_surface}
\item \tcode{basic_output_surface}
\item \tcode{basic_unmanaged_output_surface}
\end{itemize}

\pnum
The surface classes provides an interface for managing a graphics data graphics resource.

\pnum
A surface object is a move-only object.

\pnum
The surface classes modify their graphics resource through rendering and composing operations. They shall provide well-defined semantics for the graphics data graphics resource.

\pnum
The definitions of the rendering and composing operations in \ref{\iotwod.surface.rendering} shall only be applicable when the graphics data graphics resource on which the surface class members operate is a raster graphics data graphics resource. In all other cases, any attempt to invoke the rendering and composing operations shall result in undefined behavior.

\rSec1 [\iotwod.surface.rendering] {Rendering and composing}

\rSec2 [\iotwod.surface.rendering.ops] {Operations}

\pnum
The surface classes provide four fundamental rendering and composing operations:
\begin{libreqtab2}
 {surface rendering and composing operations}
 {tab:\iotwod.surface.rendering.operations}
 \\ \topline
 \lhdr{Operation}
 & \rhdr{Function(s)}
 \\ \capsep
 \endfirsthead
 \continuedcaption\\
 \hline
 \lhdr{Operation}
 & \rhdr{Function(s)}
 \\ \capsep
 \endhead
 Painting
 & \tcode{paint}
 \\
 Filling
 & \tcode{fill}
 \\
 Stroking
 & \tcode{stroke}
 \\
 Masking
 & \tcode{mask}
 \\
\end{libreqtab2}

\pnum
All composing operations shall happen in a linear color space, regardless of the color space of the graphics data that is involved.

\pnum
\begin{note}
While a color space such as sRGB helps produce expected, consistent results when graphics data are viewed by people, composing operations only produce expected results when the channel data in the graphics data involved are uniformly (i.e. linearly) spaced. 
\end{note}

\rSec2 [\iotwod.surface.rendering.brushes] {Rendering and composing brushes}

\pnum
All rendering and composing operations use a \term{source brush} of type \tcode{basic_brush}.

\pnum
The masking operation uses a \term{mask brush} of type \tcode{basic_brush}.

\rSec2 [\iotwod.surface.rendering.sourcepath] {Rendering and composing source path}

\pnum
In addition to brushes (\ref{\iotwod.surface.rendering.brushes}), all rendering and composing operation except for painting and masking use a \term{source path}. The source path is either a \tcode{basic_path_builder<Allocator>} object or a \tcode{basic_interpreted_path} object. If it is a \tcode{basic_path_builder<Allocator>} object, it is interpreted (\ref{\iotwod.paths.interpretation}) before it is used as the source path.

\rSec2 [\iotwod.surface.rendering.commonstate] {Common state data}

\pnum
All rendering and composing operations use the following state data:

\begin{libreqtab2}
 {\tcode{surface} rendering and composing common state data}
 {tab:\iotwod.surface.rendering.commonstate.listing}
 \\ \topline
 \lhdr{Name}
 & \rhdr{Type}
 \\ \capsep
 \endfirsthead
 \hline
 \lhdr{Name}
 & \rhdr{Type}
 \\ \capsep
 \endhead
 Brush properties
 & \tcode{brush_props}
 \\
 Surface properties
 & \tcode{render_props}
 \\
 Clip properties
 & \tcode{clip_props}
 \\
\end{libreqtab2}

\rSec2 [\iotwod.surface.rendering.specificstate] {Specific state data}

\pnum
In addition to the common state data (\ref{\iotwod.surface.rendering.commonstate}), certain rendering and composing operations use state data that is specific to each of them:

\begin{libiotwodtab3e}
 {surface rendering and composing specific state data}
 {tab:\iotwod.surface.rendering.specificstate.listing}
 \\ \topline
 \lhdr{Operation}
 & \chdr{Name}
 & \rhdr{Type}
 \\ \capsep
 \endfirsthead
 \hline
 \lhdr{Operation}
 & \chdr{Name}
 & \rhdr{Type}
 \\ \capsep
 \endhead
 Stroking
 & Stroke properties
 & \tcode{stroke_props}
 \\
 Stroking
 & Dashes
 & \tcode{dashes}
 \\
 Masking
 & Mask properties
 & \tcode{mask_props}
 \\
\end{libiotwodtab3e}

\rSec2 [\iotwod.surface.rendering.statedefaults] {State data default values}

\pnum
For all rendering and composing operations, the state data objects named above are provided using \tcode{optional<T>} class template arguments.

\pnum
If there is no contained value for a state data object, it is interpreted as-if the \tcode{optional<T>} argument contained a default constructed object of the relevant state data object.

\rSec1 [\iotwod.surface.coordinatespaces] {Standard coordinate spaces}

\pnum
There are four standard coordinate spaces relevant to the rendering and composing operations (\ref{\iotwod.surface.rendering}):
\begin{itemize}
\item the brush coordinate space;
\item the mask coordinate space;
\item the user coordinate space; and
\item the surface coordinate space.
\end{itemize}

\pnum
The \term{brush coordinate space} is the standard coordinate space of the source brush (\ref{\iotwod.surface.rendering.brushes}). Its transformation matrix is the brush properties' brush matrix (\ref{\iotwod.brushprops.summary}).

\pnum
The \term{mask coordinate space} is the standard coordinate space of the mask brush (\ref{\iotwod.surface.rendering.brushes}). Its transformation matrix is the mask properties' mask matrix (\ref{\iotwod.maskprops.summary}).

\pnum
The \term{user coordinate space} is the standard coordinate space of \tcode{basic_interpreted_path} objects. Its transformation matrix is a default-constructed \tcode{basic_matrix_2d}.

\pnum
The \term{surface coordinate space} is the standard coordinate space of the surface object's \underlyingsurface. Its transformation matrix is the surface properties' surface matrix (\ref{\iotwod.renderprops.summary}).

\pnum
Given a point \tcode{pt}, a brush coordinate space transformation matrix \tcode{bcsm}, a mask coordinate space transformation matrix \tcode{mcsm}, a user coordinate space transformation matrix \tcode{ucsm}, and a surface coordinate space transformation matrix \tcode{scsm}, the following table describes how to transform it from each of these standard coordinate spaces to the other standard coordinate spaces:

\begin{libiotwodreqtab3}
 {Point transformations}
 {tab:\iotwod.surface.pointtransforms.listing}
 \\ \topline
 \lhdr{From}
 & \chdr{To}
 & \rhdr{Transform}
 \\ \capsep
 \endfirsthead
 \continuedcaption\\
 \hline
 \lhdr{From}
 & \chdr{To}
 & \rhdr{Transform}
 \\ \capsep
 \endhead
 brush coordinate space
 & mask coordinate space
 & \tcode{mcsm.transform_pt(bcsm.invert().transform_pt(pt))}.
 \\
 brush coordinate space
 & user coordinate space
 & \tcode{bcsm.invert().transform_pt(pt)}.
 \\
 brush coordinate space
 & surface coordinate space
 & \tcode{scsm.transform_pt(bcsm.invert().transform_pt(pt))}.
 \\
 user coordinate space
 & brush coordinate space
 & \tcode{bcsm.transform_pt(pt)}.
 \\
 user coordinate space
 & mask coordinate space
 & \tcode{mcsm.transform_pt(pt)}.
 \\
 user coordinate space
 & surface coordinate space
 & \tcode{scsm.transform_pt(pt)}.
 \\
 surface coordinate space
 & brush coordinate space
 & \tcode{bcsm.transform_pt(scsm.invert().transform_pt(pt))}.
 \\
 surface coordinate space
 & mask coordinate space
 & \tcode{mcsm.transform_pt(scsm.invert().transform_pt(pt))}.
 \\
 surface coordinate space
 & user coordinate space
 & \tcode{scsm.invert().transform_pt(pt)}.
 \\
\end{libiotwodreqtab3}

\rSec1 [\iotwod.surface.painting] {surface painting}

\pnum
When a painting operation is initiated on a surface, the implementation shall produce results as-if the following steps were performed:

\begin{enumerate}
\item For each integral point $sp$ of the \underlyingsurface, determine if $sp$ is within the clip area (\tcode{\iotwod.clipprops.summary}); if so, proceed with the remaining steps.
\item Transform $sp$ from the surface coordinate space (\ref{\iotwod.surface.coordinatespaces}) to the brush coordinate space (Table~\ref{tab:\iotwod.surface.pointtransforms.listing}), resulting in point $bp$.
\item Sample from point $bp$ of the source brush (\ref{\iotwod.surface.rendering.brushes}), combine the resulting visual data with the visual data at point $sp$ in the \underlyingsurface in the manner specified by the surface's current \term{compositing operator} (\ref{\iotwod.renderprops.summary}), and modify the visual data of the \underlyingsurface at point $sp$ to reflect the result produced by application of the compositing operator.
\end{enumerate}

\rSec1 [\iotwod.surface.filling] {surface filling}

\pnum
When a filling operation is initiated on a surface, the implementation shall produce results as-if the following steps were performed:

\begin{enumerate}
\item For each integral point $sp$ of the \underlyingsurface, determine if $sp$ is within the \term{clip area} (\ref{\iotwod.clipprops.summary}); if so, proceed with the remaining steps.
\item Transform $sp$ from the surface coordinate space (\ref{\iotwod.surface.coordinatespaces}) to the user coordinate space (Table~\ref{tab:\iotwod.surface.pointtransforms.listing}), resulting in point $up$.
\item Using the source path (\ref{\iotwod.surface.rendering.sourcepath}) and the fill rule (\ref{\iotwod.brushprops.summary}), determine whether $up$ shall be filled; if so, proceed with the remaining steps.
\item Transform $up$ from the user coordinate space to the brush coordinate space (\ref{\iotwod.surface.coordinatespaces} and Table~\ref{tab:\iotwod.surface.pointtransforms.listing}), resulting in point $bp$.
\item Sample from point $bp$ of the source brush (\ref{\iotwod.surface.rendering.brushes}), combine the resulting visual data with the visual data at point $sp$ in the \underlyingsurface in the manner specified by the surface's current compositing operator (\ref{\iotwod.renderprops.summary}), and modify the visual data of the \underlyingsurface at point $sp$ to reflect the result produced by application of the compositing operator.
\end{enumerate}

\rSec1 [\iotwod.surface.stroking] {surface stroking}

\pnum
When a stroking operation is initiated on a surface, it is carried out for each figure in the source path (\ref{\iotwod.surface.rendering}).

\pnum
The following rules shall apply when a stroking operation is carried out on a figure:
\begin{enumerate}
\item No part of the \underlyingsurface that is outside of the clip area shall be modified.

\item If the figure is a closed figure, then the point where the end point of its final segment meets the start point of the initial segment shall be rendered as specified by the \term{line join} value (see: \ref{\iotwod.strokeprops.summary} and \ref{\iotwod.surface.rendering.specificstate}); otherwise the start point of the initial segment and end point of the final segment shall each by rendered as specified by the line cap value. The remaining meetings between successive end points and start points shall be rendered as specified by the line join value.

\item If the dash pattern (Table~\ref{tab:\iotwod.surface.rendering.specificstate.listing}) has its default value or if its \tcode{vector<float>} member is empty, the segments shall be rendered as a continuous path.

\item If the dash pattern's \tcode{vector<float>} member contains only one value, that value shall be used to define a repeating pattern in which the path is shown then hidden. The ends of each shown portion of the path shall be rendered as specified by the line cap value.

\item If the dash pattern's \tcode{vector<float>} member contains two or more values, the values shall be used to define a pattern in which the figure is alternatively rendered then not rendered for the length specified by the value. The ends of each rendered portion of the figure shall be rendered as specified by the line cap value. If the dash pattern's \tcode{float} member, which specifies an offset value, is not \tcode{0.0f}, the meaning of its value is \impldefplain{dash pattern!offset value}. If a rendered portion of the figure overlaps a not rendered portion of the figure, the rendered portion shall be rendered.
\end{enumerate}

\pnum
When a stroking operation is carried out on a figure, the width of each rendered portion shall be the \term{line width} (see: \ref{\iotwod.strokeprops.summary} and \ref{\iotwod.surface.rendering.specificstate}). Ideally this means that the diameter of the stroke at each rendered point should be equal to the line width. However, because there are an infinite number of points along each rendered portion, implementations may choose an \unspecnorm method of determining minimum distances between points along each rendered portion and the diameter of the stroke between those points shall be the same.
\begin{note}
This concept is sometimes referred to as a tolerance. It allows for a balance between precision and performance, especially in situations where the end result is in a non-exact format such as raster graphics data.
\end{note}

\pnum
After all figures in the path have been rendered but before the rendered result is composed to the \underlyingsurface, the rendered result shall be transformed from the user coordinate space (\ref{\iotwod.surface.coordinatespaces}) to the surface coordinate space (\ref{\iotwod.surface.coordinatespaces}).

\rSec1 [\iotwod.surface.masking] {surface masking}

\pnum
A \term{mask brush} is composed of a graphics data graphics resource, a \tcode{wrap_mode} value, a \tcode{filter} value, and a \tcode{basic_matrix_2d} object.

\pnum
When a masking operation is initiated on a surface, the implementation shall produce results as-if the following steps were performed:

\begin{enumerate}
\item For each integral point $sp$ of the \underlyingsurface, determine if $sp$ is within the clip area (\ref{\iotwod.clipprops.summary}); if so, proceed with the remaining steps.
\item Transform $sp$ from the surface coordinate space (\ref{\iotwod.surface.coordinatespaces}) to the mask coordinate space (Table~\ref{tab:\iotwod.surface.pointtransforms.listing}), resulting in point $mp$.
\item Sample the alpha channel from point $mp$ of the mask brush and store the result in $mac$; if the visual data format of the mask brush does not have an alpha channel, the value of $mac$ shall always be $1.0$.
\item Transform $sp$ from the surface coordinate space to the brush coordinate space, resulting in point $bp$.
\item Sample from point $bp$ of the source brush (\ref{\iotwod.surface.rendering.brushes}), combine the resulting visual data with the visual data at point $sp$ in the \underlyingsurface in the manner specified by the surface's current compositing operator (\ref{\iotwod.renderprops.summary}), multiply each channel of the result produced by application of the compositing operator by $map$ if the visual data format of the \underlyingsurface is a premultiplied format and if not then just multiply the alpha channel of the result by $map$, and modify the visual data of the \underlyingsurface at point $sp$ to reflect the multiplied result.
\end{enumerate}
