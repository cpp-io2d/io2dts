%!TEX root = io2d.tex
\rSec0 [format] {Enum class \tcode{format}}

\rSec1 [format.summary] {\tcode{format} Summary}

\pnum
The \tcode{format} enum class indicates a visual data format. See Table~\ref{tab:format.meanings} for 
the meaning of each \tcode{format} enumerator.

%\pnum
%A pixel is a value composed from one or more channels.
%
%\pnum
%A \term{channel} is a visual data element that represents color data (red, green,
%or blue) or transparency data (alpha). A pixel can be comprised of color data,
%transparency data, or both color and transparency data.
%
\pnum
Unless otherwise specified, a visual data format shall be an unsigned integral
value of the specified bit size in native-endian format.

%\pnum
%Unless otherwise specified, each channel of a pixel shall be treated as an 
%unsigned integral value of the specified bit size at the specified bit location
%within the pixel.
%
\pnum
A channel value of 0x0 means that there is no contribution from that channel. 
As the channel value increases towards the maximum unsigned integral value 
representable by the number of bits of the channel, the contribution from that 
channel also increases, with the maximum value representing the maximum
contribution from that channel.
\enterexample
Given a 5-bit channel representing the color , a value of 0x0 means that the red channel does not 
contribute any value towards the final color of the pixel. A value of 0x1F 
means that the red channel makes its maximum contribution to the final color of 
the pixel.

A
\exitexample

\rSec1 [format.synopsis] {\tcode{format} Synopsis}

\begin{codeblock}
namespace std { namespace experimental { namespace io2d { inline namespace v1 {
  enum class format {
    invalid,
    argb32,
    rgb24,
    a8,
    rgb16_565,
    rgb30
  };
} } } }
\end{codeblock}

\rSec1 [format.enumerators] {\tcode{format} Enumerators}
\begin{libreqtab2}
 {\tcode{format} enumerator meanings}
 {tab:format.meanings}
 \\ \topline
 \lhdr{Enumerator}
 & \rhdr{Meaning}
 \\ \capsep
 \endfirsthead
 \continuedcaption\\
 \hline
 \lhdr{Enumerator}
 & \rhdr{Meaning}
 \\ \capsep
 \endhead
 \tcode{invalid}
 & A previously specified \tcode{format} is unsupported by the implementation.
 \\
 \tcode{argb32}
 & A 32-bit RGB color model pixel format. The upper 8 bits are an alpha channel, 
 followed by an 8-bit red color channel, then an 8-bit green color channel, and 
 finally an 8-bit blue color channel. The value in each channel is an unsigned 
 normalized integer. This is a premultiplied format.
 \\
 \tcode{rgb24}
 & A 32-bit RGB color model pixel format. The upper 8 bits are unused, followed by an 8-bit red 
 color channel, then an 8-bit green color channel, and finally an 8-bit blue color channel. 
 \\
 \tcode{a8}
 & An 8-bit transparency data pixel format. All 8 bits are an alpha channel.
 \\
 \tcode{rgb16_565}
 & A 16-bit RGB color model pixel format. The upper 5 bits are a red color channel,
 followed by a 6-bit green color channel, and finally a 5-bit blue color channel.
 \\
 \tcode{rgb30}
 & A 32-bit RGB color model pixel format. The upper 2 bits are unused, followed by a 10-bit red 
 color channel, a 10-bit green color channel, and finally a 10-bit blue color channel. The value 
 in each channel is an unsigned normalized integer.
 \\
\end{libreqtab2}
