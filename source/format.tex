%!TEX root = io2d.tex
\rSec0 [\iotwod.format] {Enum class \tcode{format}}

\rSec1 [\iotwod.format.summary] {\tcode{format} summary}

\pnum
The \tcode{format} enum class indicates a visual data format. See Table~\ref{tab:\iotwod.format.meanings} for 
the meaning of each \tcode{format} enumerator.

\rSec1 [\iotwod.format.synopsis] {\tcode{format} synopsis}

\begin{codeblock}
namespace std::experimental::io2d::v1 {
  enum class format {
    invalid,
    argb32,
    rgb24,
    a8
  };
}
\end{codeblock}

\rSec1 [\iotwod.format.enumerators] {\tcode{format} enumerators}
\begin{libreqtab2}
 {\tcode{format} enumerator meanings}
 {tab:\iotwod.format.meanings}
 \\ \topline
 \lhdr{Enumerator}
 & \rhdr{Meaning}
 \\ \capsep
 \endfirsthead
 \continuedcaption\\
 \hline
 \lhdr{Enumerator}
 & \rhdr{Meaning}
 \\ \capsep
 \endhead
 \tcode{invalid}
 & A previously specified \tcode{format} is unsupported by the implementation.
 \\ \rowsep
 \tcode{argb32}
 & A 32-bit RGB color model pixel format. There is an 8 bit alpha channel, 
 an 8-bit red color channel, an 8-bit green color channel, and 
 an 8-bit blue color channel. The byte order, interpretation of values within 
 each channel, and whether or not this is a premultiplied format are 
 \impldef{format::argb32}.
 \\ \rowsep
 \tcode{xrgb32}
 & A 32-bit RGB color model pixel format. There is an 8 bit channel that is not 
  used, an 8-bit red color channel, an 8-bit green color channel, and 
  an 8-bit blue color channel. The byte order and interpretation of values 
  within each channel are \impldef{format::xrgb32}.
 \\ \rowsep
 \tcode{a8}
 & An 8-bit transparency data pixel format. All 8 bits are an alpha channel.
 \\
\end{libreqtab2}
