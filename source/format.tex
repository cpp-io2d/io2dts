%!TEX root = io2d.tex
\rSec0 [\iotwod.format] {Enum class \tcode{format}}

\rSec1 [\iotwod.format.summary] {\tcode{format} Summary}

\pnum
The \tcode{format} enum class describes the data format of the pixels of an 
\tcode{image_surface}. See Table~\ref{tab:\iotwod.format.meanings} for 
the meaning of each \tcode{format} enumerator.

\pnum
Unless otherwise specified, each channel of a pixel shall be treated as an 
unsigned integral value.

\pnum
A channel value of 0x0 means that there is no contribution from that channel. 
As the channel value increases towards the maximum unsigned integral value 
representable by the number of bits of the channel, the contribution from that 
channel also increases.
\enterexample
Given a 5-bit red channel, a value of 0x0 means that the red channel does not 
contribute any value towards the final color of the pixel. A value of 0x1F 
means that the red channel makes its maximum contribution to the final color of 
the pixel.
\exitexample

\rSec1 [\iotwod.format.synopsis] {\tcode{format} Synopsis}

\begin{codeblock}
namespace std { namespace experimental { namespace io2d { inline namespace v1 {
  enum class format {
    invalid,
    argb32,
    rgb24,
    a8,
    rgb16_565,
    rgb30
  };
} } } } // namespaces std::experimental::graphics::v1
\end{codeblock}

\rSec1 [\iotwod.format.enumerators] {\tcode{format} Enumerators}
\begin{libreqtab2}
 {\tcode{format} enumerator meanings}
 {tab:\iotwod.format.meanings}
 \\ \topline
 \lhdr{Enumerator}
 & \rhdr{Meaning}
 \\ \capsep
 \endfirsthead
 \continuedcaption\\
 \hline
 \lhdr{Enumerator}
 & \rhdr{Meaning}
 \\ \capsep
 \endhead
 \tcode{invalid}
 & A previously specified \tcode{format} is unsupported by the implementation.
 \\
 \tcode{argb32}
 & A 32-bit pixel format. The upper 8 bits are an alpha channel, 
 followed by an 8-bit red color channel, then an 8-bit green color channel, and 
 finally an 8-bit blue color channel. The value in each channel is an unsigned 
 normalized integer. The endianness of the 32-bit value is 
 \impldef{\tcode{format}!endianness}. This is a premultiplied format.
 \enternote
 It is valid for an implementation to define the endianness of the 32-bit value 
 as being native-endian.
 \exitnote
 \\
 \tcode{rgb24}
 & A 32-bit pixel format. The upper 8 bits are unused, followed by an 8-bit red 
 color channel, then an 8-bit green channel, and finally an 8-bit blue channel. 
 The value in each channel is an unsigned normalized integer.
 \\
 \tcode{a8}
 & An 8-bit pixel format composed of an alpha channel. The value in the channel 
 is an unsigned normalized integer.
 \\
 \tcode{rgb16_565}
 & A 16-bit pixel format composed on a red channel in the upper 5 bits, 
 followed by a 6-bit green channel, and finally a 5-bit blue channel. The value 
 in each channel is an unsigned normalized integer.
 \\
 \tcode{rgb30}
 & A 32-bit pixel format. The upper 2 bits are unused, followed by a 10-bit red 
 channel, a 10-bit green channel, and finally a 10-bit blue channel. The value 
 in each channel is an unsigned normalized integer.
 \\
\end{libreqtab2}
