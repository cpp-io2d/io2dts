%!TEX root = io2d.tex
\rSec0 [closepath] {Class \tcode{close_path}}

\rSec1 [closepath.synopsis] {\tcode{close_path} synopsis}

\begin{codeblock}
namespace std { namespace experimental { namespace io2d { inline namespace v1 {
  class close_path {
  public:
    // construct/copy/move/destroy:
    close_path() noexcept;
    close_path(const close_path& other) noexcept;
    close_path& operator=(const close_path& other) noexcept;
    close_path(close_path&& other) noexcept;
    close_path& operator=(close_path&& other) noexcept;

    // \ref{closepath.observers}, observers:
    virtual path_data_type type() const noexcept override;
  };
} } } }
\end{codeblock}

\rSec1 [closepath.intro] {\tcode{close_path} Description}

\pnum
\indexlibrary{\idxcode{close_path}}
The class \tcode{close_path} describes an operation on a path geometry collection.

\pnum
If the current path geometry has a current point, this operation creates a line from the current point to the last-move-to point. It then starts a new path geometry and sets its current point and last-move-to point to the value of the previous path geometry's last-move-to point.

\pnum
If there is no current point when this operation is requested, then this operation does nothing.
\enternote
Because this operation does nothing if there is no current point, there is no need to track whether or not a path geometry has a valid last-move-to point. This operation is the only operation that uses the last-move-to point and all operations that establish a current point when no current point existed for a path geometry also establish a valid last-move-to point for that path geometry.
\exitnote

\rSec1 [closepath.observers]{\tcode{close_path} observers}

\indexlibrary{\idxcode{close_path}!\idxcode{type}}
\indexlibrary{\idxcode{type}!\idxcode{close_path}}
\begin{itemdecl}
    virtual path_data_type type() const noexcept override;
\end{itemdecl}
\begin{itemdescr}
	\pnum
	\returns
	\tcode{path_data_type::close_path}.

\end{itemdescr}
