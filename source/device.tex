%!TEX root = io2d.tex
\rSec0 [\iotwod.device] {Class \tcode{device}}

\rSec1 [\iotwod.device.synopsis] {\tcode{device} synopsis}

\begin{codeblock}
namespace std { namespace experimental { namespace io2d { inline namespace v1 {
  class device {
  public:
    // See~\ref{\iotwod.req.native}
    typedef @\impdef@ native_handle_type; //     \expos
    native_handle_type native_handle() const noexcept; // \expos

    device() = delete;
    device(const device&) = delete;
    device& operator=(const device&) = delete;
    device(device&& other);
    device& operator=(device&& other);

    // \ref{\iotwod.device.modifiers}, modifiers:
    void flush() noexcept;
    void lock();
    void lock(error_code& ec) noexcept;
    void unlock() noexcept;
  };
} } } }
\end{codeblock}

\rSec1 [\iotwod.device.intro] {\tcode{device} Description}

\pnum
\indexlibrary{\idxcode{device}}
The \tcode{device} class provides access to the underlying rendering and 
presentation technologies, such graphics devices, graphics device contexts, and swap chains.

\pnum
A \tcode{device} object is obtained from a \tcode{surface} or \tcode{surface}-derived object.

\rSec1 [\iotwod.device.modifiers] {\tcode{device} modifiers}

\indexlibrary{\idxcode{device}!\idxcode{flush}}
\indexlibrary{\idxcode{flush}!\idxcode{device}}
\begin{itemdecl}
void flush() noexcept;
\end{itemdecl}
\begin{itemdescr}
	\pnum
	\effects
	The user shall be able to manipulate the underlying rendering and 
	presentation technologies used by the implementation without introducing a 
	race condition.
	
	\pnum
	\postconditions
	Any pending device operations shall be \defn{flushed}, which means that they shall be executed, batched, or otherwise committed to the underlying rendering and presentation technologies.
	
	\pnum 
	Saved device state, if any, shall be restored.

	\pnum
	\remarks
	This function exists primarily to allow the user to take control of the 
	underlying rendering and presentation technologies using an 
	implementation-provided native handle.
	
	\pnum
	The implementation's responsibility is to ensure that the user can safely make changes to the underlying rendering and presentation technologies using a native handle after calling this function.
	
	\pnum
	The implementation is not required to ensure that every last operation has fully completed so long as those operations which are not complete do not prevent safe use of the underlying rendering and presentation technologies.

	\pnum
	If the underlying technologies internally batch operations in a way that allows them to receive and batch further commands without introducing race conditions, the implementation should return as soon as all pending operations have been submitted to the batch queue.
	
	\pnum
	This function should not flush the surface to which the device is bound.
	
	\pnum
	If the implementation does not provide a native handle to the underlying rendering and presentation technologies, this function shall have \impldef{\tcode{device}!\tcode{flush}} behavior. This function should do nothing in the absence of a native handle.
	
	\pnum
	\realnotes
	One reason a user might call this function would be because they wished to 
	render UI controls (something that this library does not provide). As such, the user needs to know that using the underlying rendering system outside of this library will not introduce any race conditions. This function, in combination with locking the device, exists to provide that surety.
\end{itemdescr}

\indexlibrary{\idxcode{device}!\idxcode{lock}}
\indexlibrary{\idxcode{lock}!\idxcode{device}}
\begin{itemdecl}
void lock();
void lock(error_code& ec) noexcept;
\end{itemdecl}
\begin{itemdescr}
	\pnum
	\effects
	Produces all effects of \tcode{m.lock()} from \term{BasicLockable}, 
	30.2.5.2 in \CppXIV. Implementations shall make this function capable of 
	being recursively reentered from the same thread.
	
	\pnum
	\throws
	As described in Error reporting (\ref{\iotwod.err.report}).
	
	\pnum
	\errors
	\tcode{errc::resource_unavailable_try_again} if a lock cannot be obtained.
	\enternote
	One reason this error may occur is if a system limit on the maximum number of times a lock could be recursively acquired would be exceeded.
	\exitnote
\end{itemdescr}

\indexlibrary{\idxcode{device}!\idxcode{unlock}}
\indexlibrary{\idxcode{unlock}!\idxcode{device}}
\begin{itemdecl}
void unlock() noexcept;
\end{itemdecl}
\begin{itemdescr}
	\pnum
	\requires
	Meets all requirements of \tcode{m.unlock()} from \term{BasicLockable}, 
	30.2.5.2 in \CppXIV.
	
	\pnum
	\effects
	Produces all effects of \tcode{m.unlock()} from \term{BasicLockable}, 
	30.2.5.2 in \CppXIV. The lock on \tcode{m} shall not be fully released 
	until \tcode{m.unlock} has been called a number of times equal to the 
	number of times \tcode{m.lock} was successfully called.
	
	\pnum
	\remark
	This function shall not be called more times than \tcode{lock} has been called; no diagnostic is required.

\end{itemdescr}
