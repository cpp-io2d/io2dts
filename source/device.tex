%!TEX root = io2d.tex
\rSec0 [\iotwod.device] {Class \tcode{device}}

\rSec1 [\iotwod.device.intro] {\tcode{device} Description}

\pnum
\indexlibrary{\idxcode{device}}%
The \tcode{device} class provides access to the underlying rendering and 
presentation technologies, such a graphics devices, contexts, and swap chains.

\rSec1 [\iotwod.device.synopsis] {\tcode{device} synopsis}

\begin{codeblock}
namespace std { namespace experimental { namespace io2d { inline namespace v1 {
  class device {
  public:
    // See~\ref{\iotwod.req.native}
    typedef @\impdef@ native_handle_type; // Exposition only.
    native_handle_type native_handle() const; // Exposition only.

    device() = delete;
    device(const device&) = delete;
    device& operator=(const device&) = delete;
    device(device&& other);
    device& operator=(device&& other);

    // \ref{\iotwod.device.modifiers}, modifiers:
    void flush();
    void lock();
    void unlock();
  };
} } } }
\end{codeblock}

\rSec1 [\iotwod.device.modifiers] {\tcode{device} modifiers}

\indexlibrary{\idxcode{device}!\idxcode{flush}}%
\indexlibrary{\idxcode{flush}!\idxcode{device}}%
\begin{itemdecl}
void flush();
\end{itemdecl}
\begin{itemdescr}
	\pnum
	\effects
	The user will be able to manipulate the underlying rendering and 
	presentation technologies used by the implementation without introducing a 
	race condition. This means that any pending device operations will be 
	executed, batched, or otherwise committed to the underlying technologies. 
	Saved device state, if any, shall also be restored before this function 
	returns.

	\pnum
	\remarks
	This function exists primarily to allow the user to take control of the 
	underlying rendering and presentation technologies using an 
	implementation-provided native handle. The implementation's responsibility 
	is to ensure that the user can safely make changes to the state of those 
	underlying technologies. The implementation is not required to ensure that 
	every last operation has fully completed.
	\enternote
	This function flushes the native device or context that the implementation 
	is using to render to a surface. It is not required to flush the surface.
	\exitnote
	
	\pnum
	\realnotes
	One reason a user might call this function would be because they wished to 
	render UI controls (something that this library does not provide). As such, the user needs to know that using the underlying rendering system outside of this library will not introduce any race conditions. This function, in combination with locking the device, exists to provide that surety. If the underlying technologies internally batch operations in a way that allows them to receive and batch further commands without introducing race conditions, the implementation should return as soon as all pending operations have been submitted to the batch queue.
\end{itemdescr}

\indexlibrary{\idxcode{device}!\idxcode{lock}}
\indexlibrary{\idxcode{lock}!\idxcode{device}}
\begin{itemdecl}
void lock();
\end{itemdecl}
\begin{itemdescr}
	\pnum
	\effects
	Produces all effects of \tcode{m.lock()} from \term{BasicLockable}, 
	30.2.5.2 in \CppXIV. Implementations shall make this function capable of 
	being recursively reentered.
	
	\pnum
	\throws
	\impldef{\tcode{device::lock}!exception types}.
	\enternote
	Implementations should avoid throwing exceptions from this function absent 
	a compelling technical reason such as reaching a maximum level of 
	ownership, in which case an exception of type \tcode{::std::system_error} 
	shall be thrown.
	\exitnote
\end{itemdescr}

\indexlibrary{\idxcode{device}!\idxcode{unlock}}
\indexlibrary{\idxcode{unlock}!\idxcode{device}}
\begin{itemdecl}
\end{itemdecl}
\begin{itemdescr}
	\pnum
	\requires
	Meets all requirements of \tcode{m.unlock()} from \term{BasicLockable}, 
	30.2.5.2 in \CppXIV.
	
	\pnum
	\effects
	Produces all effects of \tcode{m.unlock()} from \term{BasicLockable}, 
	30.2.5.2 in \CppXIV. The lock on \tcode{m} shall not be fully released 
	until \tcode{m.unlock} has been called a number of times equal to the 
	number of times \tcode{m.lock} was successfully called.
	
	\pnum
	\throws
	Nothing.
\end{itemdescr}
