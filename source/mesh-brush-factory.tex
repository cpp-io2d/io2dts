%!TEX root = io2d.tex
\rSec0 [] {Class \tcode{}}

\rSec1 [.synopsis] {\tcode{} synopsis}

\begin{codeblock}
namespace std { namespace experimental { namespace io2d { inline namespace v1 {
    // \ref{.cons}, construct/copy/move/destroy:

    // \ref{.modifiers}, modifiers:
    
    // \ref{.observers}, observers:
    
    // \ref{.member.ops}, member operators:
    
// \expos
  
  // \ref{.ops}, non-member operators:
} } } }
\end{codeblock}

\rSec1 [.intro] {\tcode{} Description}

\pnum
\indexlibrary{\idxcode{}}
The class \tcode{} describes .

\rSec1 [.cons] {\tcode{} constructors and assignment operators}

\indexlibrary{\idxcode{}!constructor}
\begin{itemdecl}
\end{itemdecl}
\begin{itemdescr}
	\pnum
	\effects
	Constructs an object of type \tcode{}.
	
	\pnum
	\postconditions
	
\end{itemdescr}

\rSec1 [.modifiers] {\tcode{} modifiers}

\indexlibrary{\idxcode{}!\idxcode{}}
\indexlibrary{\idxcode{}!\idxcode{}}
\begin{itemdecl}
\end{itemdecl}
\begin{itemdescr}
	\pnum
	\postconditions
	
\end{itemdescr}

\rSec1 [.observers] {\tcode{} observers}

\indexlibrary{\idxcode{}!\idxcode{}}
\indexlibrary{\idxcode{}!\idxcode{}}
\begin{itemdecl}
\end{itemdecl}
\begin{itemdescr}
	\pnum
	\returns

\end{itemdescr}

\rSec1 [.member.ops] {\tcode{} member operators}

\indexlibrary{\idxcode{}!\idxcode{}}
\indexlibrary{\idxcode{}!\idxcode{}}
\begin{itemdecl}
\end{itemdecl}
\begin{itemdescr}
	\pnum
	\effects
	
	\pnum
	\returns
	\tcode{*this}.
\end{itemdescr}

\rSec1 [.ops] {\tcode{} non-member operators}

\indexlibrary{\idxcode{}!\idxcode{}}
\indexlibrary{\idxcode{}!\idxcode{}}
\begin{itemdecl}
\end{itemdecl}
\begin{itemdescr}
	\pnum
	\returns
\end{itemdescr}
