%!TEX root = io2d.tex
\rSec0[\iotwod.general]{General}
\rSec1[\iotwod.general.scope]{Scope}
\pnum
\indextext{scope|(}%
This Technical Specification specifies requirements for implementations
of an interface that computer programs written in the \Cpp programming
language may use to render and display raster graphics.%
\indextext{scope|)}

\rSec1[\iotwod.general.refs]{Normative references}

\pnum
\indextext{references!normative|(}%
The following referenced documents are indispensable for the application
of this document. For dated references, only the edition cited applies.
For undated references, the latest edition of the referenced document
(including any amendments) applies.

\begin{itemize}
\item ISO/IEC 14882, \doccite{Programming languages --- \Cpp}
\item ISO/IEC 10646-1:1993, \doccite{Information technology --- Universal 
Multiple-Octet Coded Character Set (UCS) --- Part 1: Architecture and Basic 
Multilingual Plane}
\item ISO/IEC TR 19769:2004, \doccite{Information technology --- Programming 
languages, their environments and system software interfaces --- Extensions for 
the programming language C to support new character data types}
\item ISO 32000-1:2008, \doccite{Document management --- Portable document 
format --- Part 1: PDF 1.7}
\item Tantek \c{C}elik et al., \doccite{CSS Color Module Level 3 --- W3C 
Recommendation 07 June 2011}, 
Copyright~\textcopyright~2011~W3C\textsuperscript{\textregistered} (MIT, ERCIM, 
Keio)
\item ****FIXME**** The remaining references should probably be non-normative 
and in an informative bibliography (ISO/IEC Directives, Part 2, 6.2.2 and 6.4.2)
\item Carl D. Worth, Chris Wilson, et al., \doccite{cairo graphics library --- 
Version 1.12.16}, 
Copyright~\textcopyright~2002~University~of~Southern~California, 
Copyright~\textcopyright~2005~Red~Hat,~Inc., 
Copyright~\textcopyright~2011~Intel~Corporation
\item Thomas Porter, Tom Duff, \doccite{Compositing Digital Images}, 
Copyright~\textcopyright~1984~ACM
\end{itemize}

\pnum
The library described in ISO/IEC TR 19769:2004 is hereinafter called the
\indexdefn{C!Unicode TR}\term{C Unicode TR}.

\pnum
The document CSS Color Module Level 3 --- W3C Recommendation 07 June 2011 is 
hereinafter called the \indexdefn{CSS Colors Specification}\term{CSS Colors 
Specification}.

\rSec1[\iotwod.general.defns]{Terms and definitions}

\pnum
\indextext{definitions|(}%
For the purposes of this document, the following definitions apply.

\indexdefn{aliasing}
\definition{aliasing}{\iotwod.general.defns.alias}
the presence of visual artifacts in the results of rendering due to 
sampling imperfections

\indexdefn{alpha}
\definition{alpha}{\iotwod.general.defns.alpha}
visual data representing transparency

\indexdefn{anti-aliasing}
\definition{anti-aliasing}{\iotwod.general.defns.antialias}
the application of a function or algorithm while rendering to 
reduce aliasing
\enternote
Certain algorithms can produce ``better'' results, i.e. results with less 
artifacts or with less pronounced artifacts, when rendering text with 
anti-aliasing due to the nature of text rendering. As such, it often makes 
sense to provide the ability to choose one type of anti-aliasing for text 
rendering and another for all other rendering and to provide different sets of 
anti-aliasing types to choose from for each of the two operations.
\exitnote

\indexdefn{artifact}
\definition{artifact}{\iotwod.general.defns.artifact}
an error in the results of signal processing due to the input signal and output 
signal not having a one-to-one mapping between their frequencies

\indexdefn{aspect ratio}
\definition{aspect ratio}{\iotwod.general.defns.aspectratio}
the width divided by the height

\indexdefn{B\'ezier curve}
\definition{B\'ezier curve}{\iotwod.general.defns.bezier}
a curve defined by the 
equation $f(t) = (1 - t)^{3} \times P_{0} + 3 \times t \times (1 - t)^{2} 
\times P_{1} + 3 \times t^{2} \times (1 - t) \times P_{2} + t^{3} \times P_{3}$ 
where $0.0 \le t \le 1.0$, $P_{0}$ is the starting point, $P_{1}$ is the first 
control point, $P_{2}$ is the second control point, and $P_{3}$ is the 
ending point
\enternote
This is a the definition of a cubic B\'ezier curve; other B\'eziers curve types 
with different numbers of control points exist but are not relevant in this 
\documenttypename.
\exitnote

\indexdefn{channel}
\definition{channel}{\iotwod.general.defs.channel}
a bounded set of homogeneously-spaced rational numbers

\indexdefn{color stop}
\definition{color stop}{\iotwod.general.defns.colorstop}
a value composed of a floating point offset value in the range $[0, 1]$ and a color

\indexdefn{coordinate space}
\definition{coordinate space}{\iotwod.general.defns.coordinatespace}
a Euclidean plane described by a Cartesian coordinate system where the first coordinate is measured along a horizontal axis, called the \xaxis, oriented from left to right, the second coordinate is measured along a vertical axis, called the \yaxis, oriented from top to bottom, and rotation around the origin by a positive value expressed in radians is clockwise

\indexdefn{filter}
\definition{filter}{\iotwod.general.defns.filter}
a mathematical function that transforms a signal into discrete data

\indexdefn{format}
\definition{format}{\iotwod.general.defns.format}
a well-defined visual data format composed of one or more channels representing 
color data or transparency data

\indexdefn{glyph}
\definition{glyph}{\iotwod.general.defns.glyph}
a ***FIXME***

\indexdefn{graphics}
\indexdefn{graphics!raster}
\definition{graphics}{\iotwod.general.defns.graphics.raster}
<raster graphics> visual data stored as a rectangular grid of pixels

\indexdefn{graphics}
\indexdefn{graphics!vector}
\definition{graphics}{\iotwod.general.defns.graphics.vector}
<vector graphics> visual data stored as geometrical data

\indexdefn{letterbox}
\definition{letterbox}{\iotwod.general.defns.letterbox}
render one surface to another by scaling it, centering it, and filling the 
areas outside of the  ***FIXME***

\indexdefn{normalize}
\definition{normalize}{\iotwod.general.defns.normalize}
to map a closed set of evenly spaced values in the range $[0, x]$ to an evenly spaced sequence of floating point values in the range $[0, 1]$
\enternote
The definition of normalize given is the definition for normalizing unsigned input. Signed normalization, i.e. the mapping of a closed set of evenly spaced values in the range $[-x, x)$ to an evenly spaced sequence of floating point values in the range $[-1, 1]$, also exists but is not used in this \documenttypename.
\exitnote

\indexdefn{pixel}
\definition{pixel}{\iotwod.general.defns.pixel}
a discrete visual data element with a format-dependent composition
\enternote
In raster graphics a pixel is obtained by sampling visual data, whether 
electro-mechanically, e.g. by a digital camera when a picture is taken, or 
manually, e.g. by an artist using a raster graphics editing program to create a 
pixel representation of visual data; the given examples typically result in a 
texture composed of many pixels.
\exitnote

\indexdefn{point}
\definition{point}{\iotwod.general.defns.point}
a coordinate designated by an X component and a Y component within the 
coordinate space of a surface

\indexdefn{premultiplied format}
\definition{premultiplied format}{\iotwod.general.defns.premultipliedformat}
a format with color and alpha where each color channel is normalized and then 
multiplied by the normalized alpha channel value
\enterexample
Given the 32-bit non-premultiplied RGBA pixel with 8 bits per channel \{255, 0, 
0, 127\} (half-transparent red), when normalized it would become \{1.0, 0.0, 
0.0, 0.5\}. As such, in premultiplied, normalized format it would become \{0.5, 
0.0, 0.0, 0.5\} as a result of multiplying each of the three color channels by 
the alpha channel value.
\exitexample

\indexdefn{render}
\definition{render}{\iotwod.general.defns.render}
combine source data and destination data in accordance with state data to 
create result data that replaces the original destination data
\enternote
Depending on the compositing operator and the clip, two of the pieces of state 
data, there may not be any change to the original destination data. 
\exitnote

\indexdefn{sampling}%
\definition{sampling}{\iotwod.general.defns.sampling}
the transformation of continuous or discrete signals to discrete 
data by application of a filter

\indexdefn{signal}
\indexdefn{signal!continuous}
\definition{signal}{\iotwod.general.defns.signal.cont}
<continuous signal> a real function which is defined at every point within a 
2D space and conveys information about one or more properties of the space

\indexdefn{signal}
\indexdefn{signal!discrete}
\definition{signal}{\iotwod.general.defns.signal}
<discrete signal> a real function which is defined at points separated by 
integer intervals within a 2D space and conveys information about one or more 
properties of the space

\indexdefn{texture}
\definition{texture}{\iotwod.general.defns.texture}
***FIXME***

% end definitions
\indextext{definitions|)}

\rSec1[\iotwod.general.compliance]{Implementation compliance}

\pnum
\indextext{conformance requirements|(}%
\indextext{conformance requirements!general|(}%
The set of
\defn{diagnosable rules}
consists of all syntactic and semantic rules in this International
Standard except for those rules containing an explicit notation that
``no diagnostic is required'' or which are described as resulting in
``undefined behavior.''
