%!TEX root = io2d.tex
\rSec0[\iotwod.general]{General}
\rSec1[\iotwod.general.scope]{Scope}
\pnum
\indextext{scope|(}%
This \documenttypename{} specifies requirements for implementations
of an interface that computer programs written in the \Cpp programming
language may use to render and display 2D computer graphics.
\indextext{scope|)}

\rSec1[\iotwod.general.refs]{Normative references}

\pnum
\indextext{references!normative|(}%
The following referenced documents are indispensable for the application
of this document. For dated references, only the edition cited applies.
For undated references, the latest edition of the referenced document
(including any amendments) applies.

\begin{itemize}
\item ISO/IEC 14882, \doccite{Programming languages --- \Cpp}
\item ISO/IEC 10646-1:1993, \doccite{Information technology --- Universal 
Multiple-Octet Coded Character Set (UCS) --- Part 1: Architecture and Basic 
Multilingual Plane}
\item ISO/IEC TR 19769:2004, \doccite{Information technology --- Programming 
languages, their environments and system software interfaces --- Extensions for 
the programming language C to support new character data types}
\item ISO 15076-1, \doccite{Image technology colour management --- Architecture, profile format and data structure --- Part 1: Based on ICC.1:2004-10}
\item IEC 61966-2-1, \doccite{Colour Measurement and Management in Multimedia Systems and Equipment - Part 2-1: Default RGB Colour Space - sRGB}
\item ISO 32000-1:2008, \doccite{Document management --- Portable document 
format --- Part 1: PDF 1.7}
%\item ISO/IEC 9541-1:2012, \doccite{Information technology --- Font information interchange --- Part 1: Architecture}
\item Tantek \c{C}elik et al., \doccite{CSS Color Module Level 3 --- W3C 
Recommendation 07 June 2011}, 
Copyright~\textcopyright~2011~W3C\textsuperscript{\textregistered} (MIT, ERCIM, 
Keio)
\end{itemize}

\pnum
The library described in ISO/IEC TR 19769:2004 is hereinafter called the
\indexdefn{C!Unicode TR}\term{C Unicode TR}.

\pnum
The document CSS Color Module Level 3 --- W3C Recommendation 07 June 2011 is 
hereinafter called the \indexdefn{CSS Colors Specification}\term{CSS Colors 
Specification}.

\rSec1[\iotwod.general.defns]{Terms and definitions}

\pnum
\indextext{definitions|(}
For the purposes of this document, the following definitions apply.

\indexdefn{standard coordinate space}
\definition{standard coordinate space}{\iotwod.general.defns.standardcoordinatespace}
a Euclidean plane described by a Cartesian coordinate system where the first coordinate is measured along a horizontal axis, called the \xaxis, oriented from left to right, the second coordinate is measured along a vertical axis, called the \yaxis, oriented from top to bottom, and rotation of a point around the origin by a positive value expressed in radians is clockwise

\indexdefn{visual data}
\definition{visual data}{\iotwod.general.defns.visualdata}
data representing color, transparency, or some combination thereof

\indexdefn{channel}
\definition{channel}{\iotwod.general.defs.channel}
a bounded set of homogeneously-spaced real numbers in the range $[0,1]$

%\indexdefn{channel}
%\indexdefn{channel!color}
%\definition{color channel}{\iotwod.general.defns.colorchannel}
%a channel representing the intensity of a specific color
%
\indexdefn{visual data format}
\definition{visual data format}{\iotwod.general.defns.visualdataformat}
a specification of visual data channels which defines a total bit size for the format and each channel's role, bit size, and location relative to the upper (high-order) bit
\enternote
The total bit size may be larger than the sum of the bit sizes of all of the channels of the format.
\exitnote

\indexdefn{visual data element}
\definition{visual data element}{\iotwod.general.defns.visualdataelement}
an item of visual data with a defined visual data format

\indexdefn{alpha}
\definition{alpha}{\iotwod.general.defns.alpha}
visual data representing transparency

% The following may not be needed. Already defined in IEC 60050 723-05-31
\indexdefn{pixel}
\definition{pixel}{\iotwod.general.defns.pixel}
a discrete, rectangular visual data element

\indexdefn{aliasing}
\definition{aliasing}{\iotwod.general.defns.alias}
the presence of visual artifacts in the results of rendering due to 
sampling imperfections

\indexdefn{artifact}
\definition{artifact}{\iotwod.general.defns.artifact}
an error in the results of the application of a composing operation 

\indexdefn{anti-aliasing}
\definition{anti-aliasing}{\iotwod.general.defns.antialias}
the application of a function or algorithm while rendering to 
reduce aliasing
\enternote
Certain algorithms can produce ``better'' results, i.e. results with less 
artifacts or with less pronounced artifacts, when rendering text with 
anti-aliasing due to the nature of text rendering. As such, it often makes 
sense to provide the ability to choose one type of anti-aliasing for text 
rendering and another for all other rendering and to provide different sets of 
anti-aliasing types to choose from for each of the two operations.
\exitnote

% The following may not be needed. Already defined in IEC 60050 723-05-05.
\indexdefn{aspect ratio}
\definition{aspect ratio}{\iotwod.general.defns.aspectratio}
the ratio of the width to the height of a rectangular area

\indexdefn{color model}
\definition{color model}{\iotwod.general.defns.colormodel}
an ideal, mathematical representation of colors which often uses color channels

\indexdefn{additive color}
\definition{additive color}{\iotwod.general.defns.additivecolor}
a color defined by the emissive intensity of its color channels

\indexdefn{color model}
\indexdefn{color model!RGB}
\definition{RGB color model}{\iotwod.general.defns.rgbcolormodel}
an additive color model using red, green, and blue color channels

\indexdefn{color model}
\indexdefn{color model!RGBA}
\definition{RGBA color model}{\iotwod.general.defns.rgbacolormodel}
the RGB color model with an alpha channel
\enternote
RGBA is not a proper color model; it is a convenient way to refer to the RGB color model to which an alpha channel has been added.

The interpretation of the alpha channel and its effect on the interpretation of the RGB color channels is intentionally not defined here because it is context-dependent.
\exitnote

\indexdefn{color space}
\definition{color space}{\iotwod.general.defns.colorspace}
an unambiguous mapping of values to colorimetric colors
\enternote
The difference between a color model and a color space is often obscured, and sometimes the terms themselves are mistakenly used interchangeably.

A color model defines color mathematically without regard to how humans actually perceive color. Color models are useful for working with color computationally but, since they deal in ideal colors rather than perceived colors, they fail to provide the information necessary to allow for the uniform display of their colors on different output devices (e.g. LCD monitors, CRT TVs, and printers).

A color space, by contrast, maps unambiguous values to perceived colors. Since the perception of color varies from person to person, color spaces use the science of colorimetry to define those perceived colors in order to obtain uniformity. As such, the uniform display of the colors in a color space on different output devices is possible.
\exitnote

\indexdefn{color space}
\indexdefn{color space!sRGB}
\definition{sRGB color space}{\iotwod.general.defns.srgbcolorspace}
an additive color space defined in IEC 61966-2-1 that is based on the RGB color model

% %The following definition is malformed according to ISO/IEC Directives, Part 2, 2011 Appendix D
%\indexdefn{compose}
%\definition{compose}{\iotwod.general.defns.compose}
%to combine a source of raster graphics data with a raster graphics data graphics resource in the manner specified by a composition algorithm and store the result to the graphics resource
%\enternote
%The 'source of raster graphics' data may be the results of rendering or may be another raster graphics data graphics resource.
%
%Where a statement in the form 'compose X and Y' is encountered, X is the 'source of raster graphics data' and Y is the 'raster graphics data graphics resource'.
%\exitnote
%
\indexdefn{cubic \bezierlocal curve}
\definition{cubic \bezierlocal curve}{\iotwod.general.defns.cubicbezier}
a curve defined by the 
equation $f(t) = (1 - t)^{3} \times P_{0} + 3 \times t \times (1 - t)^{2} 
\times P_{1} + 3 \times t^{2} \times (1 - t) \times P_{2} + t^{3} \times P_{3}$ 
where $0.0 \le t \le 1.0$, $P_{0}$ is the starting point, $P_{1}$ is the first 
control point, $P_{2}$ is the second control point, and $P_{3}$ is the 
ending point

\indexdefn{filter}
\definition{filter}{\iotwod.general.defns.filter}
a mathematical function that determines the visual data value of a point for a graphics data graphics resource

\indexdefn{graphics data}
\definition{graphics data}{\iotwod.general.defns.graphicsdata}
<graphics data> visual data stored in an unspecified form

\indexdefn{graphics data}
\indexdefn{graphics data!raster}
\definition{graphics data}{\iotwod.general.defns.graphics.raster}
<raster graphics data> visual data stored as pixels that is accessible as if it was an array of rows of pixels beginning with the pixel at the integral point $(0,0)$

\indexdefn{graphics resource}
\definition{graphics resource}{\iotwod.general.defns.graphicsresource}
<graphics resource> an object of unspecified type used by an implementation
\enternote
By its definition a graphics resource is an implementation detail. Often it will be a graphics subsystem object (e.g. a graphics device or a render target) or an aggregate composed of multiple graphics subsystem objects. However the only requirement placed upon a graphics resource is that the implementation is able to use it to provide the functionality required of the graphics resource.
\exitnote

\indexdefn{graphics resource}
\indexdefn{graphics resource!graphics data graphics resource}
\definition{graphics resource}{\iotwod.general.defns.graphicsresource.graphicsdata}
<graphics data graphics resource> an object of unspecified type used by an implementation to provide access to and allow manipulation of visual data

%\indexdefn{graphics resource}
%\indexdefn{graphics resource!path geometry graphics resource}
%\definition{graphics resource}{\iotwod.general.defns.graphicsresource.pathgeometry}
%<path geometry graphics resource> an object of unspecified type used by an implementation to store a collection of zero or more path geometries in an unspecified format
%
\indexdefn{\pixmap}
\definition{\pixmap}{\iotwod.general.defns.pixmap}
a raster graphics data graphics resource

\indexdefn{point}
\definition{point}{\iotwod.general.defns.point}
<point> a coordinate designated by a floating point \xaxis{} value and a floating point \yaxis{} value within the standard coordinate space

\indexdefn{point}
\definition{point}{\iotwod.general.defns.point.integral}
<integral point> a coordinate designated by an integral \xaxis{} value and an integral \yaxis{} value within the standard coordinate space

\indexdefn{premultiplied format}
\definition{premultiplied format}{\iotwod.general.defns.premultipliedformat}
a format with color and alpha where each color channel is normalized and then 
multiplied by the normalized alpha channel value
\enterexample
Given the 32-bit non-premultiplied RGBA pixel with 8 bits per channel \{255, 0, 
0, 127\} (half-transparent red), when normalized it would become \{1.0, 0.0, 
0.0, 0.5\}. As such, in premultiplied, normalized format it would become \{0.5, 
0.0, 0.0, 0.5\} as a result of multiplying each of the three color channels by 
the alpha channel value.
\exitexample

\indexdefn{graphics state data}
\definition{graphics state data}{\iotwod.general.defns.graphicsstatedata}
data which specify how some part of the process of rendering or of a composing operation shall be performed in part or in whole

\indexdefn{graphics subsystem}
\definition{graphics subsystem}{\iotwod.general.defns.graphicssubsystem}
collection of unspecified operating system and library functionality used to render and display 2D computer graphics

\indexdefn{normalize}
\definition{normalize}{\iotwod.general.defns.normalize}
to map a closed set of evenly spaced values in the range $[0, x]$ to an evenly spaced sequence of floating point values in the range $[0, 1]$
\enternote
The definition of normalize given is the definition for normalizing unsigned input. Signed normalization, i.e. the mapping of a closed set of evenly spaced values in the range $[-x, x)$ to an evenly spaced sequence of floating point values in the range $[-1, 1]$, also exists but is not used in this \documenttypename{}.
\exitnote

\indexdefn{render}
\definition{render}{\iotwod.general.defns.render}
to transform path geometries or text into graphics data in the manner specified by a set of graphics state data

\indexdefn{rendering operation}
\definition{rendering operation}{\iotwod.general.defns.renderingoperation}
an operation that performs rendering

\indexdefn{compose}
\definition{compose}{\iotwod.general.defns.compose}
to combine part or all of a source graphics data graphics resource with a destination graphics data graphics resource in the manner specified by a composition algorithm

\indexdefn{composing operation}
\definition{composing operation}{\iotwod.general.defns.composingoperation}
an operation that performs composing
%an operation that uses a composition algorithm to combine part or all of a source of visual data capable of being treated as though it were a \pixmap with a \pixmap

\indexdefn{composition algorithm}
\definition{composition algorithm}{\iotwod.general.defns.compositionalgorithm}
an algorithm that combines a source visual data element and a destination visual data element producing a visual data element that has the same visual data format as the destination visual data element

\indexdefn{rendering and composing operation}
\definition{rendering and composing operation}{\iotwod.general.defns.renderingandcomposingop}
an operation that is either a composing operation or a rendering operation followed by a composing operation

\indexdefn{sample}
\definition{sample}{\iotwod.general.defns.sample}
to use a filter to obtain the visual data for a given point from a graphics data graphics resource

\indexdefn{color stop}
\definition{color stop}{\iotwod.general.defns.colorstop}
a tuple composed of a floating point offset value in the range $[0, 1]$ and a color value

%!TEX root = io2d.tex

\indexdefn{path segment}
\definition{path segment}{\iotwod.general.defns.pathsegment}
a line, \bezierlocal curve, or arc, each of which has a start point and an end point

\indexdefn{control point}
\definition{control point}{\iotwod.general.defns.controlpoint}
a point other than the start point and end point that is used in defining a \bezierlocal curve

\indexdefn{degenerate path segment}
\definition{degenerate path segment}{\iotwod.general.defns.degeneratepathsegment}
a path segment that has the same values for its start point, end point, and, if any, control points

\indexdefn{initial path segment}
\definition{initial path segment}{\iotwod.general.defns.initialpathsegment}
a path segment whose start point is not defined as being the end point of another path segment
\enternote
It is possible for the initial path segment and final path segment to be the same path segment.
\exitnote

\indexdefn{final path segment}
\definition{final path segment}{\iotwod.general.defns.finalpathsegment}
a path segment whose end point shall not be used to define the start point of any other path segment
\enternote
It is possible for the initial path segment and final path segment to be the same path segment.
\exitnote

\indexdefn{path instruction}
\definition{path instruction}{\iotwod.general.defns.pathinstruction}
an instruction that creates a new path, closes an existing path, or modifies the interpretation of path segments that follow it

\indexdefn{path}
\definition{path}{\iotwod.general.defns.path}
a collection of path instructions and path segments where the end point of each path segment, except the final path segment, defines the start point of exactly one other path segment in the collection

\indexdefn{current point}
\definition{current point}{\iotwod.general.defns.currentpoint}
a point established by various operations used in creating a path
\enternote
A new path has no current point except as otherwise specified.
\exitnote

\indexdefn{last-move-to point}
\definition{last-move-to point}{\iotwod.general.defns.lastmovetopoint}
the point in a path that is the start point of the initial path segment

\indexdefn{path group}
\definition{path group}{\iotwod.general.defns.pathgroup}
a collection of paths

\indexdefn{closed path}
\definition{closed path}{\iotwod.general.defns.closedpath}
a path with one or more path segments where the last-move-to point is used to define the end point of the path's final path segment

\indexdefn{open path}
\definition{open path}{\iotwod.general.defns.openpath}
a path with one or more path segments where the last-move-to point is not used to define the end point of the path's final path segment
\enternote
Even if the start point of the initial path segment and the end point of the final path segment are assigned the same coordinates, the path is still an open path since the final path segment's end point is not defined as being the start point of the initial segment but instead merely happens to have the same value as that point.
\exitnote

\indexdefn{degenerate path}
\definition{degenerate path}{\iotwod.general.defns.degeneratepath}
a path with only one path segment
\enternote
The path segment is not required to be a degenerate path segment.
\exitnote


%%!TEX root = io2d.tex

\indexdefn{alignment line}
\definition{alignment line}{\iotwod.general.defns.alignmentline}
imaginary line to which most glyph images of a font seem to align \\
\lbrack SOURCE: ISO/IEC 9541-1:2012, definition 3.1 \rbrack

\indexdefn{current position}
\definition{current position}{\iotwod.general.defns.currentposition}
a point on a graphics data graphics resource at which the next glyph representation is to be rendered

\indexdefn{design size}
\definition{design size}{\iotwod.general.defns.designsize}
absolute size at which a font is designed to be used \\
\lbrack SOURCE: ISO/IEC 9541-1:2012, definition 3.3 \rbrack

\indexdefn{escapement}
\definition{escapement}{\iotwod.general.defns.escapement}
movement of the current position on the presentation surface after a glyph representation is rendered

\indexdefn{escapement point}
\definition{escapement point}{\iotwod.general.defns.escapementpoint}
a glyph metric; a point in the glyph's standard coordinate system, to which the current position on the graphics data graphics resource is usually translated, after the glyph representation is rendered

\indexdefn{font}
\definition{font}{\iotwod.general.defns.font}
a collection of glyph images having the same basic design, e.g., \textit{Courier Bold Oblique} \\
\lbrack SOURCE: ISO/IEC 9541-1:2012, definition 3.6 \rbrack

\indexdefn{font family}
\definition{font family}{\iotwod.general.defns.fontfamily}
a collection of fonts of common design, e.g., \textit{Courier, Courier Bold, Courier Bold Oblique} \\
\lbrack SOURCE: ISO/IEC 9541-1:2012, definition 3.7 \rbrack

\indexdefn{font metrics}
\definition{font metrics}{\iotwod.general.defns.fontmetrics}
the set of dimensions and positioning information in a font resource common to all glyph representations contained in that font resource \\
\lbrack SOURCE: ISO/IEC 9541-1:2012, definition 3.8 \rbrack

%\indexdefn{font reference}
%\definition{font reference}{\iotwod.general.defns.fontreference}
%the information about a font resource in an electronic document representation, and possible procedures and operations on that information, which identify or describe the desired font \\
%\lbrack SOURCE: ISO/IEC 9541-1:2012, definition 3.9 \rbrack
%
\indexdefn{font resource}
\definition{font resource}{\iotwod.general.defns.fontresource}
a collection of glyph representations together with descriptive and font metric information which are relevant to the collection of glyph representations as a whole \\
\lbrack SOURCE: ISO/IEC 9541-1:2012, definition 3.10 \rbrack

\indexdefn{font size}
\definition{font size}{\iotwod.general.defns.fontsize}
a scalar reference size relative to which most font metrics, glyph shapes and glyph metrics are specified \\
\lbrack SOURCE: ISO/IEC 9541-1:2012, definition 3.11 \rbrack

\indexdefn{glyph}
\definition{glyph}{\iotwod.general.defns.glyph}
a recognizable abstract graphic symbol which is independent of any specific design \\
\lbrack SOURCE: ISO/IEC 9541-1:2012, definition 3.12 \rbrack

\indexdefn{glyph collection}
\definition{glyph collection}{\iotwod.general.defns.glyphcollection}
an identified set of glyphs \\
\lbrack SOURCE: ISO/IEC 9541-1:2012, definition 3.13 \rbrack

\indexdefn{glyph image}
\definition{glyph image}{\iotwod.general.defns.glyphimage}
an image of a glyph, as obtained from a glyph representation rendered and composed to a graphics data graphics resource

\indexdefn{glyph metrics}
\definition{glyph metrics}{\iotwod.general.defns.glyphmetrics}
the set of information in a glyph representation used for defining the dimensions and positioning of the glyph shape \\
\lbrack SOURCE: ISO/IEC 9541-1:2012, definition 3.16 \rbrack

\indexdefn{glyph representation}
\definition{glyph representation}{\iotwod.general.defns.glyphrepresentation}
the glyph shape and glyph metrics associated with a specific glyph in a font resource \\
\lbrack SOURCE: ISO/IEC 9541-1:2012, definition 3.17 \rbrack

\indexdefn{glyph shape}
\definition{glyph shape}{\iotwod.general.defns.glyphshape}
the set of information in a glyph representation used for defining the shape which represents the glyph \\
\lbrack SOURCE: ISO/IEC 9541-1:2012, definition 3.18 \rbrack

\indexdefn{kern}
\definition{kern}{\iotwod.general.defns.kern}
the extension of a glyph shape beyond its position point or escapement point \\
\lbrack SOURCE: ISO/IEC 9541-1:2012, definition 3.19 \rbrack

\indexdefn{position point}
\definition{position point}{\iotwod.general.defns.positionpoint}
a glyph metric; a point in the glyph's standard coordinate system, usually translated to the current position on the graphics data graphics resource before the glyph shape is rendered

\indexdefn{posture}
\definition{posture}{\iotwod.general.defns.posture}
the extent to which the shape of a glyph or set of glyphs appears to incline, including any consequent design or form change \\
\lbrack SOURCE: ISO/IEC 9541-1:2012, definition 3.22 \rbrack

\indexdefn{proportionate width}
\definition{proportionate width}{\iotwod.general.defns.proportionatewidth}
the ratio of a glyph's or set of glyphs' escapement to font height \\
\lbrack SOURCE: ISO/IEC 9541-1:2012, definition 3.24 \rbrack

\indexdefn{stem}
\definition{stem}{\iotwod.general.defns.stem}
the major stroke of a glyph shape \\
\lbrack SOURCE: ISO/IEC 9541-1:2012, definition 3.25 \rbrack

\indexdefn{weight}
\definition{weight}{\iotwod.general.defns.weight}
the ratio of a glyph's or set of glyphs' stem width to font height \\
\lbrack SOURCE: ISO/IEC 9541-1:2012, definition 3.26 \rbrack

\indexdefn{writing mode}
\definition{writing mode}{\iotwod.general.defns.writingmode}
an identified mode for setting of text in a writing system, usually corresponding to a nominal escapement direction of the glyphs in that mode, i.e., left-to-right, right-to-left or top-to-bottom \\
\lbrack SOURCE: ISO/IEC 9541-1:2012, definition 3.27 \rbrack

\indexdefn{body size}
\definition{body size}{\iotwod.general.defns.bodysize}
the font size, measured along the y axis of the glyph's standard coordinate system

\indexdefn{wrap_modeed body size}
\definition{wrap_modeed body size}{\iotwod.general.defns.wrap_modeedbodysize}
a reference size with two components, measured respectively along the \xaxis and \yaxis of the glyph's standard coordinate system

\indexdefn{design frame}
\definition{design frame}{\iotwod.general.defns.designframe}
dimensional expression that specifies the area inside which a set of glyph images can be designed \\
\lbrack SOURCE: ISO/IEC 9541-1:2012, definition 3.30 \rbrack

\indexdefn{bounding box}
\definition{bounding box}{\iotwod.general.defns.boundingbox}
dimensional expression to specify an actual area that a glyph image occupies within a design frame \\
\lbrack SOURCE: ISO/IEC 9541-1:2012, definition 3.31 \rbrack
%
%\indexdefn{blackness}
%\definition{blackness}{\iotwod.general.defns.blackness}
%the ratio of the blackened area of a glyph image to the wrap_modeed body size area of the glyph image \\
%\lbrack SOURCE: ISO/IEC 9541-1:2012, definition 3.32 \rbrack

%
% end definitions
\indextext{definitions|)}

\rSec1 [\iotwod.general.helloworld] {Hello world example}

\pnum
The following is an example of a complete "hello world" program written using this library:

\begin{codeblock}
#include <experimental/io2d>
using namespace std::experimental::io2d;

int main() {
  auto ds = make_display_surface(640, 480, format::argb32);
  ds.draw_callback([](display_surface& ds) {
    ds.paint(rgba_color::firebrick());
    ds.render_text("Hello world!", { 50.0, 50.0 }, rgba_color::aqua());
  });
  return ds.show();
}
\end{codeblock}
