%!TEX root = io2d.tex
\rSec0 [\iotwod.imagesurface] {Class \tcode{basic_image_surface}}

\rSec1 [\iotwod.imagesurface.summary] {\tcode{basic_image_surface} summary}

\pnum
\indexlibrary{\idxcode{basic_image_surface}}%
The class \tcode{basic_image_surface} provides an interface to a raster graphics data graphics resource.

\pnum
It has a \term{pixel format} of type \tcode{format}, a \term{width} of type \tcode{int}, and a \term{height} of type \tcode{int}.

\pnum
\begin{note}
Because of the functionality it provides and what it can be used for, it is expected that developers familiar with other graphics technologies will think of the \tcode{basic_image_surface} class as being a form of \term{render target}. This is intentional, though this \documenttypename{} does not formally define or use that term to avoid any minor ambiguities and differences in its meaning between the various graphics technologies that do use the term render target.
\end{note}

\rSec1 [\iotwod.imagesurface.synopsis] {\tcode{basic_image_surface} synopsis}

\begin{codeblock}
namespace std::experimental::io2d::v1 {
  template <class GraphicsSurfaces>
  class basic_image_surface {
  public:
    using graphics_math_type = typename GraphicsSurfaces::graphics_math_type;

    // \ref{\iotwod.imagesurface.cons}, construct/copy/move/destroy:
    basic_image_surface(io2d::format fmt, int width, int height);
    basic_image_surface(filesystem::path f, io2d::image_file_format iff, io2d::format fmt);
    basic_image_surface(filesystem::path f, io2d::image_file_format iff, io2d::format fmt,
      error_code& ec) noexcept;
    basic_image_surface(basic_image_surface&&) noexcept;
    basic_image_surface& operator=(basic_image_surface&&) noexcept;

    // \ref{\iotwod.imagesurface.members}, members:
    void save(filesystem::path p, image_file_format i);
    void save(filesystem::path p, image_file_format i, error_code& ec) noexcept;

    // \ref{\iotwod.imagesurface.staticmembers}, static members:
    static basic_display_point<graphics_math_type> max_dimensions() noexcept;

    // \ref{\iotwod.imagesurface.observers}, observers:
    io2d::format format() const noexcept;
    basic_display_point<graphics_math_type> dimensions() const noexcept;
	
    // \ref{\iotwod.imagesurface.mofifiers}, modifiers:
    void clear();
    void flush();
    void flush(error_code& ec) noexcept;
    void mark_dirty();
    void mark_dirty(error_code& ec) noexcept;
    void mark_dirty(const basic_bounding_box<graphics_math_type>& extents);
    void mark_dirty(const basic_bounding_box<graphics_math_type>& extents,
      error_code& ec) noexcept;
    void paint(const basic_brush<GraphicsSurfaces>& b,
      const optional<basic_brush_props<GraphicsSurfaces>>& bp = nullopt,
      const optional<basic_render_props<GraphicsSurfaces>>& rp = nullopt,
      const optional<basic_clip_props<GraphicsSurfaces>>& cl = nullopt);
    template <class Allocator>
    void stroke(const basic_brush<GraphicsSurfaces>& b,
      const basic_path_builder<GraphicsSurfaces, Allocator>& pb,
      const optional<basic_brush_props<GraphicsSurfaces>>& bp = nullopt,
      const optional<basic_stroke_props<GraphicsSurfaces>>& sp = nullopt,
      const optional<basic_dashes<GraphicsSurfaces>>& d = nullopt,
      const optional<basic_render_props<GraphicsSurfaces>>& rp = nullopt,
      const optional<basic_clip_props<GraphicsSurfaces>>& cl = nullopt);
    void stroke(const basic_brush<GraphicsSurfaces>& b,
      const basic_interpreted_path<GraphicsSurfaces>& ip,
      const optional<basic_brush_props<GraphicsSurfaces>>& bp = nullopt,
      const optional<basic_stroke_props<GraphicsSurfaces>>& sp = nullopt,
      const optional<basic_dashes<GraphicsSurfaces>>& d = nullopt,
      const optional<basic_render_props<GraphicsSurfaces>>& rp = nullopt,
      const optional<basic_clip_props<GraphicsSurfaces>>& cl = nullopt);
    template <class Allocator>
    void fill(const basic_brush<GraphicsSurfaces>& b,
      const basic_path_builder<GraphicsSurfaces, Allocator>& pb,
      const optional<basic_brush_props<GraphicsSurfaces>>& bp = nullopt,
      const optional<basic_render_props<GraphicsSurfaces>>& rp = nullopt,
      const optional<basic_clip_props<GraphicsSurfaces>>& cl = nullopt);
    void fill(const basic_brush<GraphicsSurfaces>& b,
      const basic_interpreted_path<GraphicsSurfaces>& ip,
      const optional<basic_brush_props<GraphicsSurfaces>>& bp = nullopt,
      const optional<basic_render_props<GraphicsSurfaces>>& rp = nullopt,
      const optional<basic_clip_props<GraphicsSurfaces>>& cl = nullopt);
    void mask(const basic_brush<GraphicsSurfaces>& b,
      const basic_brush<GraphicsSurfaces>& mb,
      const optional<basic_brush_props<GraphicsSurfaces>>& bp = nullopt,
      const optional<basic_mask_props<GraphicsSurfaces>>& mp = nullopt,
      const optional<basic_render_props<GraphicsSurfaces>>& rp = nullopt,
      const optional<basic_clip_props<GraphicsSurfaces>>& cl = nullopt);
  };

  template <class GraphicsSurfaces>
  basic_image_surface<GraphicsSurfaces> copy_image_surface(
    basic_image_surface<GraphicsSurfaces>& sfc) noexcept;
}
\end{codeblock}

\rSec1 [\iotwod.imagesurface.cons] {\tcode{basic_image_surface} constructors and assignment operators}

\indexlibrary{\idxcode{basic_image_surface}!constructor}%
\begin{itemdecl}
basic_image_surface(io2d::format fmt, int w, int h);
\end{itemdecl}
\begin{itemdescr}
\pnum
\requires
\tcode{w} is greater than \tcode{0} and not greater than \tcode{basic_image_surface::max_width()}.

\pnum
\tcode{h} is greater than \tcode{0} and not greater than \tcode{basic_image_surface::max_height()}.

\pnum
\tcode{fmt} is not \tcode{io2d::format::invalid}.

\pnum
\effects
Constructs an object of type \tcode{basic_image_surface}.

\pnum
The pixel format is \tcode{fmt}, the width is \tcode{w}, and the height is \tcode{h}.
\end{itemdescr}

\indexlibrary{\idxcode{basic_image_surface}!constructor}%
\begin{itemdecl}
basic_image_surface(filesystem::path f, io2d::image_file_format i, io2d::format fmt);
basic_image_surface(filesystem::path f, io2d::image_file_format i, io2d::format fmt,
  error_code& ec) noexcept;
\end{itemdecl}
\begin{itemdescr}
\pnum
\requires
\tcode{f} is a file and its contents are data in either JPEG format, TIFF format or PNG format.

\pnum
\tcode{fmt} is not \tcode{io2d::format::invalid}.

\pnum
\effects
Constructs an object of type \tcode{basic_image_surface}.

\pnum
The data of the \underlyingimagesurface is the raster graphics data that results from processing \tcode{f} into uncompressed raster graphics in the manner specified by the standard that specifies how to transform the contents of data contained in \tcode{f} into raster graphics data and then transforming that raster graphics data into the format specified by \tcode{fmt}.

\pnum
The data of \tcode{f} is processed into uncompressed raster graphics data as specified by the value of \tcode{i}.

\pnum
If \tcode{i} is \tcode{image_file_format::unknown}, implementations may attempt to process the data of \tcode{f} into uncompressed raster graphics data. The manner in which it does so is \unspecnorm. If no uncompressed raster graphics data is produced, the error specified below occurs.

\pnum
\begin{note}
The intent of \tcode{image_file_format::unknown} is to allow implementations to support image file formats that are not required to be supported.
\end{note}

\pnum
If the width of the uncompressed raster graphics data would be less than \tcode{1} or greater than \tcode{basic_image_surface::max_width()} or if the height of the uncompressed raster graphics data would be less than \tcode{1} or greater than \tcode{basic_image_surface::max_height()}, the error specified below occurs.

\pnum
The resulting uncompressed raster graphics data is then transformed into the data format specified by \tcode{fmt}. If the format specified by \tcode{fmt} only contains an alpha channel, the values of the color channels, if any, of the \underlyingimagesurface are \unspecnorm. If the format specified by \tcode{fmt} only contains color channels and the resulting uncompressed raster graphics data is in a premultiplied format, then the value of each color channel for each pixel is be divided by the value of the alpha channel for that pixel. The visual data is then set as the visual data of the \underlyingimagesurface.

\pnum
The width is the width of the uncompressed raster graphics data. The height is the height of the uncompressed raster graphics data.

\pnum
\throws
As specified in Error reporting (\ref{\iotwod.err.report}).

\pnum
\errors
Any error that could result from trying to access \tcode{f}, open \tcode{f} for reading, or reading data from \tcode{f}.

\pnum
\tcode{errc::not_supported} if \tcode{image_file_format::unknown} is passed as an argument and the implementation is unable to determine the file format or does not support saving in the image file format it determined.

\pnum
\tcode{errc::invalid_argument} if \tcode{fmt} is \tcode{io2d::format::invalid}.

\pnum
\tcode{errc::argument_out_of_domain} if the width would be less than \tcode{1}, the width would be greater than \tcode{basic_image_surface::max_width()}, the height would be less than \tcode{1}, or the height would be greater than \tcode{basic_image_surface::max_height()}.
\end{itemdescr}

\rSec1 [\iotwod.imagesurface.members] {\tcode{basic_image_surface} members}

\indexlibrarymember{save}{basic_image_surface}%
\begin{itemdecl}
void save(filesystem::path p, image_file_format i);
void save(filesystem::path p, image_file_format i, error_code& ec) noexcept;
\end{itemdecl}
\begin{itemdescr}
\pnum
\requires
\tcode{p} shall be a valid path to a file. The file need not exist provided that the other components of the path are valid.

\pnum
If the file exists, it shall be writable. If the file does not exist, it shall be possible to create the file at the specified path and then the created file shall be writable.

\pnum
\effects
Any pending rendering and composing operations (\ref{\iotwod.surface.rendering}) are performed.

\pnum
The visual data of the \underlyingimagesurface is written to \tcode{p} in the data format specified by \tcode{i}.

\pnum
If \tcode{i} is \tcode{image_file_format::unknown}, it is \impldefplain{basic_image_surface!save} whether the surface is saved in the image file format, if any, that the implementation associates with \tcode{p.extension()} provided that \tcode{p.has_extension() == true}. If \tcode{p.has_extension() == false}, the implementation does not associate an image file format with \tcode{p.extension()}, or the implementation does not support saving in that image file format, the error specified below occurs.

\pnum
\throws
As specified in Error reporting (\ref{\iotwod.err.report}).

\pnum
\errors
Any error that could result from trying to create \tcode{f}, access \tcode{f}, or write data to \tcode{f}.

\pnum
\tcode{errc::not_supported} if \tcode{image_file_format::unknown} is passed as an argument and the implementation is unable to determine the file format or does not support saving in the image file format it determined.
\end{itemdescr}

\rSec1 [\iotwod.imagesurface.staticmembers] {\tcode{basic_image_surface} static members}

\indexlibrarymember{max_dimensions}{basic_image_surface}%
\begin{itemdecl}
static basic_display_point<graphics_math_type> max_dimensions() noexcept;
\end{itemdecl}
\begin{itemdescr}
\pnum
\returns
<TODO>The maximum height and width for a \tcode{basic_image_surface} object.
\end{itemdescr}

\rSec1 [\iotwod.imagesurface.observers] {\tcode{basic_image_surface} observers}

\indexlibrarymember{format}{basic_image_surface}%
\begin{itemdecl}
io2d::format format() const noexcept;
\end{itemdecl}
\begin{itemdescr}
\pnum
\returns
The pixel format.
\end{itemdescr}

\indexlibrarymember{dimensions}{basic_image_surface}%
\begin{itemdecl}
basic_display_point<graphics_math_type> dimensions() const noexcept;
\end{itemdecl}
\begin{itemdescr}
\pnum
\returns
<TODO>The height and width.
\end{itemdescr}

\rSec1 [\iotwod.imagesurface.mofifiers] {\tcode{basic_image_surface} modifiers}

\indexlibrarymember{clear}{basic_image_surface}%
\begin{itemdecl}
void clear();
\end{itemdecl}
\begin{itemdescr}
\pnum
\effects
<TODO>
\end{itemdescr}

\indexlibrarymember{flush}{basic_image_surface}%
\begin{itemdecl}
void flush();
void flush(error_code& ec) noexcept;
\end{itemdecl}
\begin{itemdescr}
\pnum
\effects
If the implementation does not provide a native handle to the surface's \underlyingsurface, this function does nothing.

\pnum
If the implementation does provide a native handle to the surface's \underlyingsurface, then the implementation performs every action necessary to ensure that all operations on the surface that produce observable effects occur.

\pnum
The implementation performs any other actions necessary to ensure that the surface will be usable again after a call to \tcode{basic_image_surface::mark_dirty}.

\pnum
Once a call to \tcode{basic_image_surface::flush} is made, \tcode{basic_image_surface::mark_dirty} shall be called before any other member function of the surface is called or the surface is used as an argument to any other function.

\pnum
\throws
As specified in Error reporting (\ref{\iotwod.err.report}).

\pnum
\remarks
This function exists to allow the user to take control of the underlying surface using an implementation-provided native handle without introducing a race condition. The implementation's responsibility is to ensure that the user can safely use the underlying surface.

\pnum
\errors
The potential errors are \impldefplain{basic_image_surface::flush errors}.

\pnum
Implementations should avoid producing errors here.

\pnum
If the implementation does not provide a native handle to the \tcode{basic_image_surface} object's \underlyingsurface, this function shall not produce any errors.

\pnum
\begin{note}
There are several purposes for \tcode{basic_image_surface::flush} and \tcode{basic_image_surface::mark_dirty}.

\pnum
One is to allow implementation wide latitude in how they implement the rendering and composing operations (\ref{\iotwod.surface.rendering}), such as batching calls and then sending them to the \underlyingrendandpresenttechs at appropriate times.

\pnum
Another is to give implementations the chance during the call to \tcode{basic_image_surface::flush} to save any internal state that might be modified by the user and then restore it during the call to \tcode{basic_image_surface::mark_dirty}.

\pnum
Other uses of this pair of calls are also possible.
\end{note}
\end{itemdescr}

\indexlibrarymember{mark_dirty}{basic_image_surface}%
\begin{itemdecl}
void mark_dirty();
void mark_dirty(error_code& ec) noexcept;
void mark_dirty(const basic_bounding_box<graphics_math_type>& extents);
void mark_dirty(const basic_bounding_box<graphics_math_type>& extents, error_code& ec) noexcept;
\end{itemdecl}
\begin{itemdescr}
\pnum
\effects
If the implementation does not provide a native handle to the \tcode{basic_image_surface} object's \underlyingsurface, this function shall do nothing.

\pnum
If the implementation does provide a native handle to the \tcode{basic_image_surface} object's \underlyingsurface, then:
\begin{itemize}
\item If called without a \tcode{basic_bounding_box} argument, informs the implementation that external changes using a native handle were potentially made to the entire \underlyingsurface.
\item If called with a \tcode{basic_bounding_box} argument, informs the implementation that external changes using a native handle were potentially made to the \underlyingsurface within the bounds specified by the \term{bounding rectangle} \tcode{basic_bounding_box\{ round(extents.x()), round (extents.y()), round(extents.width()), round(extents.height())\}}. No part of the bounding rectangle shall be outside of the bounds of the \underlyingsurface; no diagnostic is required.
\end{itemize}

\pnum
\throws
As specified in Error reporting (\ref{\iotwod.err.report}).

\pnum
\remarks
After external changes are made to this \tcode{basic_image_surface} object's \underlyingsurface using a native pointer, this function shall be called before using this \tcode{basic_image_surface} object; no diagnostic is required.

\pnum
\errors
The errors, if any, produced by this function are \impldefplain{basic_image_surface!mark_dirty}.

\pnum
If the implementation does not provide a native handle to the \tcode{basic_image_surface} object's \underlyingsurface, this function shall not produce any errors.
\end{itemdescr}

\indexlibrarymember{paint}{basic_image_surface}%
\begin{itemdecl}
void paint(const basic_brush<GraphicsSurfaces>& b,
  const optional<basic_brush_props<GraphicsSurfaces>>& bp = nullopt,
  const optional<basic_render_props<GraphicsSurfaces>>& rp = nullopt,
  const optional<basic_clip_props<GraphicsSurfaces>>& cl = nullopt);
\end{itemdecl}
\begin{itemdescr}
\pnum
\effects
Performs the painting rendering and composing operation as specified by \ref{\iotwod.surface.painting}.

\pnum
The meanings of the parameters are specified by \ref{\iotwod.surface.rendering}.

\pnum
\throws
As specified in Error reporting (\ref{\iotwod.err.report}).

\pnum
\errors
The errors, if any, produced by this function are \impldefplain{basic_image_surface!paint}.
\end{itemdescr}

\indexlibrarymember{stroke}{basic_image_surface}%
\begin{itemdecl}
template <class Allocator>
void stroke(const basic_brush<GraphicsSurfaces>& b,
  const basic_path_builder<GraphicsSurfaces, Allocator>& pb,
  const optional<basic_brush_props<GraphicsSurfaces>>& bp = nullopt,
  const optional<basic_stroke_props<GraphicsSurfaces>>& sp = nullopt,
  const optional<basic_dashes<GraphicsSurfaces>>& d = nullopt,
  const optional<basic_render_props<GraphicsSurfaces>>& rp = nullopt,
  const optional<basic_clip_props<GraphicsSurfaces>>& cl = nullopt);
void stroke(const basic_brush<GraphicsSurfaces>& b,
  const basic_interpreted_path<GraphicsSurfaces>& ip,
  const optional<basic_brush_props<GraphicsSurfaces>>& bp = nullopt,
  const optional<basic_stroke_props<GraphicsSurfaces>>& sp = nullopt,
  const optional<basic_dashes<GraphicsSurfaces>>& d = nullopt,
  const optional<basic_render_props<GraphicsSurfaces>>& rp = nullopt,
  const optional<basic_clip_props<GraphicsSurfaces>>& cl = nullopt);
\end{itemdecl}
\begin{itemdescr}
\pnum
\effects
Performs the stroking rendering and composing operation as specified by \ref{\iotwod.surface.stroking}.

\pnum
The meanings of the parameters are specified by \ref{\iotwod.surface.rendering}.

\pnum
\throws
As specified in Error reporting (\ref{\iotwod.err.report}).

\pnum
\errors
The errors, if any, produced by this function are \impldefplain{basic_image_surface!stroke}.
\end{itemdescr}

\indexlibrarymember{fill}{basic_image_surface}%
\begin{itemdecl}
template <class Allocator>
void fill(const basic_brush<GraphicsSurfaces>& b,
  const basic_path_builder<GraphicsSurfaces, Allocator>& pb,
  const optional<basic_brush_props<GraphicsSurfaces>>& bp = nullopt,
  const optional<basic_render_props<GraphicsSurfaces>>& rp = nullopt,
  const optional<basic_clip_props<GraphicsSurfaces>>& cl = nullopt);
void fill(const basic_brush<GraphicsSurfaces>& b,
  const basic_interpreted_path<GraphicsSurfaces>& ip,
  const optional<basic_brush_props<GraphicsSurfaces>>& bp = nullopt,
  const optional<basic_render_props<GraphicsSurfaces>>& rp = nullopt,
  const optional<basic_clip_props<GraphicsSurfaces>>& cl = nullopt);
\end{itemdecl}
\begin{itemdescr}
\pnum
\effects
Performs the filling rendering and composing operation as specified by \ref{\iotwod.surface.filling}.

\pnum
The meanings of the parameters are specified by \ref{\iotwod.surface.rendering}.

\pnum
\throws
As specified in Error reporting (\ref{\iotwod.err.report}).

\pnum
\errors
The errors, if any, produced by this function are \impldefplain{basic_image_surface!fill}.
\end{itemdescr}

\indexlibrarymember{mask}{basic_image_surface}%
\begin{itemdecl}
void mask(const basic_brush<GraphicsSurfaces>& b,
  const basic_brush<GraphicsSurfaces>& mb,
  const optional<basic_brush_props<GraphicsSurfaces>>& bp = nullopt,
  const optional<basic_mask_props<GraphicsSurfaces>>& mp = nullopt,
  const optional<basic_render_props<GraphicsSurfaces>>& rp = nullopt,
  const optional<basic_clip_props<GraphicsSurfaces>>& cl = nullopt);
\end{itemdecl}
\begin{itemdescr}
\pnum
\effects
Performs the masking rendering and composing operation as specified by \ref{\iotwod.surface.masking}.

\pnum
The meanings of the parameters are specified by \ref{\iotwod.surface.rendering}.

\pnum
\throws
As specified in Error reporting (\ref{\iotwod.err.report}).

\pnum
\errors

The errors, if any, produced by this function are \impldefplain{basic_image_surface!mask}.
\end{itemdescr}
