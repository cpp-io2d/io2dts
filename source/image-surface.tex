%!TEX root = io2d.tex
\rSec0 [\iotwod.imagesurface] {Class \tcode{image_surface}}

\rSec1 [\iotwod.imagesurface.summary] {\tcode{image_surface} summary}

\pnum
\indexlibrary{\idxcode{image_surface}}
The class \tcode{image_surface} derives from the \tcode{surface} class and provides an interface to a raster graphics data graphics resource.

\pnum
\begin{note}
Because of the functionality it provides and what it can be used for, it is expected that developers familiar with other graphics technologies will think of the \tcode{image_surface} class as being a form of \term{render target}. This is intentional, though this \documenttypename{} does not formally define or use that term to avoid any minor ambiguities and differences in its meaning between the various graphics technologies that do use the term render target.
\end{note}

\rSec1 [\iotwod.imagesurface.synopsis] {\tcode{image_surface} synopsis}

\begin{codeblock}
namespace std::experimental::io2d::v1 {
  class image_surface : public surface {
  public:
    // \ref{\iotwod.imagesurface.cons}, construct/copy/move/destroy:
    image_surface() = delete;
    image_surface(experimental::io2d::format fmt, int width, int height);
    image_surface(filesystem::path f, image_file_format i, 
      experimental::io2d::format fmt);
    
    // \ref{\iotwod.imagesurface.members}, members:
    void save(filesystem::path p, image_file_format i);
    
    // \ref{\iotwod.imagesurface.observers}, observers:
    experimental::io2d::format format() const noexcept;
    int width() const noexcept;
    int height() const noexcept;
  };
}
\end{codeblock}

\rSec1 [\iotwod.imagesurface.cons] {\tcode{image_surface} constructors and assignment operators}

\indexlibrary{\idxcode{image_surface}!constructor}
\begin{itemdecl}
image_surface(experimental::io2d::format fmt, int width, int height);
\end{itemdecl}
\begin{itemdescr}
\pnum
\requires
\tcode{w >= 1}.

\pnum
\tcode{h >= 1}.

\pnum
\effects
Constructs an object of type \tcode{image_surface}.

\pnum
\postconditions
\tcode{this->format() == fmt}.

\pnum
\tcode{this->width() == width}.

\pnum
\tcode{this->height() == height}.
\end{itemdescr}

\indexlibrary{\idxcode{image_surface}!constructor}
\begin{itemdecl}
image_surface(filesystem::path f, image_file_format i,
  experimental::io2d::format fmt);
\end{itemdecl}
\begin{itemdescr}
\pnum
\requires
\tcode{f} is a file and its contents are data in either JPEG format or PNG format.

\pnum
\effects
Constructs an object of type \tcode{image_surface}.

\pnum
The data of the \underlyingimagesurface is the raster graphics data that results from processing \tcode{f} into uncompressed raster graphics in the manner specified by the standard that specifies how to transform the contents of data contained in \tcode{f} into raster graphics data and then transforming that raster graphics data into the format specified by \tcode{fmt}.

\pnum
The data of \tcode{f} is processed into uncompressed raster graphics data as specified by the value of \tcode{i}.

\pnum
The resulting uncompressed raster graphics data is then transformed into the data format specified by \tcode{fmt}. If the format specified by \tcode{fmt} only contains an alpha channel, the values of the color channels, if any, of the \underlyingimagesurface are \unspecnorm. If the format specified by \tcode{fmt} only contains color channels and the resulting uncompressed raster graphics data is in a premultiplied format, then the value of each color channel for each pixel shall be divided by the value of the alpha channel for that pixel. The visual data shall then be set as the visual data of the \underlyingimagesurface.

\pnum
\throws
As specified in Error reporting [\iotwod.fs.err.report] in \cppseventeen.

\pnum
\errors
Any error that could result from trying to access \tcode{f}, open \tcode{f} for reading, or reading data from \tcode{f}.

\pnum
Other errors, if any, produced by this function are \impldefplain{image_surface!data}.
\end{itemdescr}

\rSec1 [\iotwod.imagesurface.members] {\tcode{image_surface} members}

\indexlibrary{\idxcode{image_surface}!\idxcode{save}}
\begin{itemdecl}
void save(filesystem::path p, image_file_format i);
\end{itemdecl}
\begin{itemdescr}
\pnum
\requires
\tcode{p} shall be a valid path to a file. The file need not exist provided that the other components of the path are valid.

\pnum
If the file exists, it shall be writable. If the file does not exist, it shall be possible to create the file at the specified path and then the created file shall be writable.

\pnum
\effects
Any pending rendering and composing operations (\ref{\iotwod.surface.rendering}) shall be performed.

\pnum
The visual data of the \underlyingimagesurface is written to \tcode{p} in the data format specified by \tcode{i}.

\pnum
\throws
As specified in Error reporting [\iotwod.fs.err.report] in \cppseventeen.

\pnum
\errors
Any error that could result from trying to create \tcode{f}, access \tcode{f}, or write data to \tcode{f}.

\pnum
Other errors, if any, produced by this function are \impldefplain{image_surface!data}.
\end{itemdescr}

\rSec1 [\iotwod.imagesurface.observers] {\tcode{image_surface} observers}

\indexlibrary{\idxcode{image_surface}!\idxcode{format}}
\indexlibrary{\idxcode{format}!\idxcode{image_surface}}
\begin{itemdecl}
experimental::io2d::format format() const noexcept;
\end{itemdecl}
\begin{itemdescr}
\pnum
\returns
The pixel format of the \tcode{image_surface} object.

\pnum
\remarks
If the \tcode{image_surface} object is invalid, this function shall return \\ \tcode{experimental::io2d::format::invalid}.
\end{itemdescr}

\indexlibrary{\idxcode{image_surface}!\idxcode{width}}
\indexlibrary{\idxcode{width}!\idxcode{image_surface}}
\begin{itemdecl}
int width() const noexcept;
\end{itemdecl}
\begin{itemdescr}
\pnum
\returns
The number of pixels per horizontal line of the \tcode{image_surface} object.

\pnum
\remarks
This function shall return the value \tcode{0} if \\
\tcode{this->format() == experimental::io2d::format::invalid}.
\end{itemdescr}

\indexlibrary{\idxcode{image_surface}!\idxcode{height}}
\indexlibrary{\idxcode{height}!\idxcode{image_surface}}
\begin{itemdecl}
int height() const noexcept;
\end{itemdecl}
\begin{itemdescr}
\pnum
\returns
The number of horizontal lines of pixels in the \tcode{image_surface} object.

\pnum
\remarks
This function shall return the value \tcode{0} if \\
\tcode{this->format() == experimental::io2d::format::invalid}.
\end{itemdescr}
