%!TEX root = io2d.tex
\rSec0 [\iotwod.imagesurface] {Class \tcode{basic_image_surface}}

\rSec1 [\iotwod.imagesurface.summary] {\tcode{basic_image_surface} summary}

\pnum
\indexlibrary{\idxcode{basic_image_surface}}%
The class \tcode{basic_image_surface} provides an interface to raster graphics data.

\pnum
It has a \term{pixel format} of type \tcode{format}, a \term{width} of type \tcode{int}, and a \term{height} of type \tcode{int}.

\pnum
The data are stored in an object of type \tcode{typename GraphicsSurfaces::surfaces::image_surface_data_type}. It is accessible using the \tcode{data} member functions.

\pnum
\begin{note}
Because of the functionality it provides and what it can be used for, it is expected that developers familiar with other graphics technologies will think of the \tcode{basic_image_surface} class as being a form of \term{render target}. This is intentional, though this \documenttypename{} does not formally define or use that term to avoid any minor ambiguities and differences in its meaning between the various graphics technologies that do use the term render target.
\end{note}

\rSec1 [\iotwod.imagesurface.synopsis] {\tcode{basic_image_surface} synopsis}

\begin{codeblock}
namespace @\fullnamespace{}@ {
  template <class GraphicsSurfaces>
  class basic_image_surface {
  public:
    using graphics_math_type = typename GraphicsSurfaces::graphics_math_type;
    using data_type = typename GraphicsSurfaces::image_surface_data_type;

    // \ref{\iotwod.imagesurface.cons}, construct/copy/move/destroy:
    basic_image_surface(io2d::format fmt, int width, int height);
    basic_image_surface(filesystem::path f, io2d::image_file_format iff, io2d::format fmt);
    basic_image_surface(filesystem::path f, io2d::image_file_format iff, io2d::format fmt,
      @\stdqualifier{}@error_code& ec) noexcept;
    basic_image_surface(basic_image_surface&&) noexcept;
    basic_image_surface& operator=(basic_image_surface&&) noexcept;

    // \ref{\iotwod.imagesurface.acc}, accessors:
    const data_type& data() const noexcept;
    data_type& data() noexcept;

    // \ref{\iotwod.imagesurface.members}, members:
    void save(filesystem::path p, image_file_format i);
    void save(filesystem::path p, image_file_format i, @\stdqualifier{}@error_code& ec) noexcept;

    // \ref{\iotwod.imagesurface.staticmembers}, static members:
    static basic_display_point<graphics_math_type> max_dimensions() noexcept;

    // \ref{\iotwod.imagesurface.observers}, observers:
    io2d::format format() const noexcept;
    basic_display_point<graphics_math_type> dimensions() const noexcept;
	
    // \ref{\iotwod.imagesurface.mofifiers}, modifiers:
    void clear();
    void paint(const basic_brush<GraphicsSurfaces>& b,
      const optional<basic_brush_props<GraphicsSurfaces>>& bp = nullopt,
      const optional<basic_render_props<GraphicsSurfaces>>& rp = nullopt,
      const optional<basic_clip_props<GraphicsSurfaces>>& cl = nullopt);
    template <class Allocator>
    void stroke(const basic_brush<GraphicsSurfaces>& b,
      const basic_path_builder<GraphicsSurfaces, Allocator>& pb,
      const optional<basic_brush_props<GraphicsSurfaces>>& bp = nullopt,
      const optional<basic_stroke_props<GraphicsSurfaces>>& sp = nullopt,
      const optional<basic_dashes<GraphicsSurfaces>>& d = nullopt,
      const optional<basic_render_props<GraphicsSurfaces>>& rp = nullopt,
      const optional<basic_clip_props<GraphicsSurfaces>>& cl = nullopt);
    void stroke(const basic_brush<GraphicsSurfaces>& b,
      const basic_interpreted_path<GraphicsSurfaces>& ip,
      const optional<basic_brush_props<GraphicsSurfaces>>& bp = nullopt,
      const optional<basic_stroke_props<GraphicsSurfaces>>& sp = nullopt,
      const optional<basic_dashes<GraphicsSurfaces>>& d = nullopt,
      const optional<basic_render_props<GraphicsSurfaces>>& rp = nullopt,
      const optional<basic_clip_props<GraphicsSurfaces>>& cl = nullopt);
    template <class Allocator>
    void fill(const basic_brush<GraphicsSurfaces>& b,
      const basic_path_builder<GraphicsSurfaces, Allocator>& pb,
      const optional<basic_brush_props<GraphicsSurfaces>>& bp = nullopt,
      const optional<basic_render_props<GraphicsSurfaces>>& rp = nullopt,
      const optional<basic_clip_props<GraphicsSurfaces>>& cl = nullopt);
    void fill(const basic_brush<GraphicsSurfaces>& b,
      const basic_interpreted_path<GraphicsSurfaces>& ip,
      const optional<basic_brush_props<GraphicsSurfaces>>& bp = nullopt,
      const optional<basic_render_props<GraphicsSurfaces>>& rp = nullopt,
      const optional<basic_clip_props<GraphicsSurfaces>>& cl = nullopt);
    void mask(const basic_brush<GraphicsSurfaces>& b,
      const basic_brush<GraphicsSurfaces>& mb,
      const optional<basic_brush_props<GraphicsSurfaces>>& bp = nullopt,
      const optional<basic_mask_props<GraphicsSurfaces>>& mp = nullopt,
      const optional<basic_render_props<GraphicsSurfaces>>& rp = nullopt,
      const optional<basic_clip_props<GraphicsSurfaces>>& cl = nullopt);
    void draw_text(const basic_point_2d<graphics_math_type>& pt,
      const basic_brush<GraphicsSurfaces>& b,
      const basic_font<GraphicsSurfaces>& font, const string& text,
      const optional<basic_text_props<GraphicsSurfaces>>& tp = nullopt,
      const optional<basic_brush_props<GraphicsSurfaces>>& bp = nullopt,
      const optional<basic_stroke_props<GraphicsSurfaces>>& sp = nullopt,
      const optional<basic_dashes<GraphicsSurfaces>>& d = nullopt,
      const optional<basic_render_props<GraphicsSurfaces>>& rp = nullopt,
      const optional<basic_clip_props<GraphicsSurfaces>>& cl = nullopt);
    void draw_text(const basic_bounding_box<graphics_math_type>& bb,
      const basic_brush<GraphicsSurfaces>& b,
      const basic_font<GraphicsSurfaces>& font, const string& text,
      const optional<basic_text_props<GraphicsSurfaces>>& tp = nullopt,
      const optional<basic_brush_props<GraphicsSurfaces>>& bp = nullopt,
      const optional<basic_stroke_props<GraphicsSurfaces>>& sp = nullopt,
      const optional<basic_dashes<GraphicsSurfaces>>& d = nullopt,
      const optional<basic_render_props<GraphicsSurfaces>>& rp = nullopt,
      const optional<basic_clip_props<GraphicsSurfaces>>& cl = nullopt);
	future<void> command_list(const basic_command_list<GraphicsSurfaces>& cl);
  };
}
\end{codeblock}

\rSec1 [\iotwod.imagesurface.cons] {\tcode{basic_image_surface} constructors and assignment operators}

\indexlibrary{\idxcode{basic_image_surface}!constructor}%
\begin{itemdecl}
basic_image_surface(io2d::format fmt, int w, int h);
\end{itemdecl}
\begin{itemdescr}
\pnum
\requires
\tcode{w} is greater than \tcode{0} and not greater than \tcode{basic_image_surface::max_width()}.

\pnum
\tcode{h} is greater than \tcode{0} and not greater than \tcode{basic_image_surface::max_height()}.

\pnum
\tcode{fmt} is not \tcode{io2d::format::invalid}.

\pnum
\effects
Constructs an object of type \tcode{basic_image_surface}.

\pnum
The pixel format is \tcode{fmt}, the width is \tcode{w}, and the height is \tcode{h}.

\pnum
\postconditions \tcode{data() == GraphicsSurfaces::surfaces::create_image_surface(fmt, w, h)}.
\end{itemdescr}

\indexlibrary{\idxcode{basic_image_surface}!constructor}%
\begin{itemdecl}
basic_image_surface(filesystem::path f, io2d::image_file_format i, io2d::format fmt);
basic_image_surface(filesystem::path f, io2d::image_file_format i, io2d::format fmt,
  @\stdqualifier{}@error_code& ec) noexcept;
\end{itemdecl}
\begin{itemdescr}
\pnum
\requires
\tcode{f} is a file and its contents are data in a supported format (see: \ref{\iotwod.imagefileformat}).

\pnum
\tcode{fmt} is not \tcode{io2d::format::invalid}.

\pnum
\effects
Constructs an object of type \tcode{basic_image_surface}.

\pnum
\postconditions If called without an \tcode{error_code\&} argument \tcode{data() == GraphicsSurfaces::surfaces::create_image_surface(f, i, fmt)}, otherwise \tcode{data() == GraphicsSurfaces::surfaces::create_image_surface(f, i, fmt, ec)}.

\pnum
\remarks The raster graphics data is the result of processing \tcode{f} into uncompressed raster graphics in the manner specified by the standard that describes how to transform the contents of data contained in \tcode{f} into raster graphics data and then transforming that transformed raster graphics data into the format specified by \tcode{fmt}.

\pnum
The data of \tcode{f} is processed into uncompressed raster graphics data as specified by the value of \tcode{i}.

\pnum
If \tcode{i} is \tcode{image_file_format::unknown}, implementations may attempt to process the data of \tcode{f} into uncompressed raster graphics data. The manner in which it does so is \unspecnorm. If no uncompressed raster graphics data is produced, the error specified below occurs.

\pnum
\begin{note}
The intent of \tcode{image_file_format::unknown} is to allow implementations to support image file formats that are not required to be supported.
\end{note}

\pnum
If the width of the uncompressed raster graphics data would be less than \tcode{1} or greater than \tcode{basic_image_surface::max_width()} or if the height of the uncompressed raster graphics data would be less than \tcode{1} or greater than \tcode{basic_image_surface::max_height()}, the error specified below occurs.

\pnum
The resulting uncompressed raster graphics data is then transformed into the data format specified by \tcode{fmt}. If the format specified by \tcode{fmt} only contains an alpha channel, the values of the color channels, if any, of the surface's visual data are \unspecnorm. If the format specified by \tcode{fmt} only contains color channels and the resulting uncompressed raster graphics data is in a premultiplied format, then the value of each color channel for each pixel is be divided by the value of the alpha channel for that pixel. The visual data is then set as the visual data of the surface.

\pnum
The width is the width of the uncompressed raster graphics data. The height is the height of the uncompressed raster graphics data.

\pnum
\throws
As specified in Error reporting (\ref{\iotwod.err.report}).

\pnum
\errors
Any error that could result from trying to access \tcode{f}, open \tcode{f} for reading, or reading data from \tcode{f}.

\pnum
\tcode{errc::not_supported} if \tcode{image_file_format::unknown} is passed as an argument and the implementation is unable to determine the file format or does not support saving in the image file format it determined.

\pnum
\tcode{errc::invalid_argument} if \tcode{fmt} is \tcode{io2d::format::invalid}.

\pnum
\tcode{errc::argument_out_of_domain} if the width would be less than \tcode{1}, the width would be greater than \tcode{basic_image_surface::max_width()}, the height would be less than \tcode{1}, or the height would be greater than \tcode{basic_image_surface::max_height()}.
\end{itemdescr}

\rSec1 [\iotwod.imagesurface.acc] {Accessors}%

\indexlibrarymember{data}{basic_image_surface}%
\begin{itemdecl}
const data_type& data() const noexcept;
data_type& data() noexcept;
\end{itemdecl}
\begin{itemdescr}
\pnum
\returns A reference to the \tcode{basic_image_surface} object's data object (See: \ref{\iotwod.imagesurface.summary}).

\pnum
\remarks The behavior of a program is undefined if the user modifies the data contained in the \tcode{data_type} object returned by this function.
\end{itemdescr}


\rSec1 [\iotwod.imagesurface.members] {\tcode{basic_image_surface} members}

\indexlibrarymember{save}{basic_image_surface}%
\begin{itemdecl}
void save(filesystem::path p, image_file_format i);
void save(filesystem::path p, image_file_format i, @\stdqualifier{}@error_code& ec) noexcept;
\end{itemdecl}
\begin{itemdescr}
\pnum
\requires
\tcode{p} shall be a valid path to a file. The file need not exist provided that the other components of the path are valid.

\pnum
If the file exists, it shall be writable. If the file does not exist, it shall be possible to create the file at the specified path and then the created file shall be writable.

\pnum
\effects If called without an \tcode{error_code\&} argument \tcode{GraphicsSurfaces::surfaces::save(p, i)}, otherwise \tcode{GraphicsSurfaces::surfaces::save(p, i, ec)}.

\pnum
\remarks Any pending rendering and composing operations (\ref{\iotwod.surface.rendering}) are performed before the surface's visual data is written to \tcode{p}.

\pnum
The surface's visual data is written to \tcode{p} in the data format specified by \tcode{i}.

\pnum
If \tcode{i} is \tcode{image_file_format::unknown}, it is \impldefplain{basic_image_surface!save} whether the surface is saved in the image file format, if any, that the implementation associates with \tcode{p.extension()} provided that \tcode{p.has_extension() == true}. If \tcode{p.has_extension() == false}, the implementation does not associate an image file format with \tcode{p.extension()}, or the implementation does not support saving in that image file format, the error specified below occurs.

\pnum
\throws
As specified in Error reporting (\ref{\iotwod.err.report}).

\pnum
\errors
Any error that could result from trying to create \tcode{f}, access \tcode{f}, or write data to \tcode{f}.

\pnum
\tcode{errc::not_supported} if \tcode{image_file_format::unknown} is passed as an argument and the implementation is unable to determine the file format or does not support saving in the image file format it determined.
\end{itemdescr}

\rSec1 [\iotwod.imagesurface.staticmembers] {\tcode{basic_image_surface} static members}

\indexlibrarymember{max_dimensions}{basic_image_surface}%
\begin{itemdecl}
static basic_display_point<graphics_math_type> max_dimensions() noexcept;
\end{itemdecl}
\begin{itemdescr}
\pnum
\returns \tcode{GraphicsSurfaces::surfaces::max_dimensions()}.

\pnum
\remarks
The maximum height and width for a \tcode{basic_image_surface} object.
\end{itemdescr}

\rSec1 [\iotwod.imagesurface.observers] {\tcode{basic_image_surface} observers}

\indexlibrarymember{format}{basic_image_surface}%
\begin{itemdecl}
io2d::format format() const noexcept;
\end{itemdecl}
\begin{itemdescr}
\pnum
\returns \tcode{GraphicsSurfaces::surfaces::format(data())}.

\pnum
\remarks
The pixel format.
\end{itemdescr}

\indexlibrarymember{dimensions}{basic_image_surface}%
\begin{itemdecl}
basic_display_point<graphics_math_type> dimensions() const noexcept;
\end{itemdecl}
\begin{itemdescr}
\pnum
\returns \tcode{GraphicsSurfaces::surfaces::dimensions(data())}.

\pnum
\remarks
The pixel dimensions.
\end{itemdescr}

\rSec1 [\iotwod.imagesurface.mofifiers] {\tcode{basic_image_surface} modifiers}

\indexlibrarymember{clear}{basic_image_surface}%
\begin{itemdecl}
void clear();
\end{itemdecl}
\begin{itemdescr}
\pnum
\effects
Equivalent to \tcode{paint(basic_brush<GraphicsSurfaces(rgba_color::white), nullopt, basic_render_props<GraphicsSurfaces>(nearest, basic_matrix_2d<typename GraphicsSurfaces::graphics_math_type>\{\}, compositing_op::clear));}
\end{itemdescr}

\indexlibrarymember{paint}{basic_image_surface}%
\begin{itemdecl}
void paint(const basic_brush<GraphicsSurfaces>& b,
  const optional<basic_brush_props<GraphicsSurfaces>>& bp = nullopt,
  const optional<basic_render_props<GraphicsSurfaces>>& rp = nullopt,
  const optional<basic_clip_props<GraphicsSurfaces>>& cl = nullopt);
\end{itemdecl}
\begin{itemdescr}
\pnum
\effects Calls \tcode{GraphicsSurfaces::surfaces::paint(data(), b, (bp == nullopt ? basic_brush_props<GraphicsSurfaces>() : bp.value()), (rp == nullopt ? basic_render_props<GraphicsSurfaces>() : rp.value()), (cl == nullopt ? basic_clip_props<GraphicsSurfaces>() : cl.value()))}

\pnum
\remarks Performs the painting rendering and composing operation as specified by \ref{\iotwod.surface.painting}.

\pnum
The meanings of the parameters are specified by \ref{\iotwod.surface.rendering}.

\pnum
\throws
As specified in Error reporting (\ref{\iotwod.err.report}).

\pnum
\errors
The errors, if any, produced by this function are \impldefplain{basic_image_surface!paint}.
\end{itemdescr}

\indexlibrarymember{stroke}{basic_image_surface}%
\begin{itemdecl}
template <class Allocator>
void stroke(const basic_brush<GraphicsSurfaces>& b,
  const basic_path_builder<GraphicsSurfaces, Allocator>& pb,
  const optional<basic_brush_props<GraphicsSurfaces>>& bp = nullopt,
  const optional<basic_stroke_props<GraphicsSurfaces>>& sp = nullopt,
  const optional<basic_dashes<GraphicsSurfaces>>& d = nullopt,
  const optional<basic_render_props<GraphicsSurfaces>>& rp = nullopt,
  const optional<basic_clip_props<GraphicsSurfaces>>& cl = nullopt);
\end{itemdecl}
\begin{itemdescr}
\pnum
\effects Calls \tcode{GraphicsSurfaces::surfaces::stroke(data(), b, basic_interpreted_path<GraphicsSurfaces>(pb), (bp == nullopt ? basic_brush_props<GraphicsSurfaces>() : bp.value()), (sp == nullopt ? basic_stroke_props<GraphicsSurfaces>() : sp.value()), (d == nullopt ? basic_dashes<GraphicsSurfaces>() : d.value()), (rp == nullopt ? basic_render_props<GraphicsSurfaces>() : rp.value()), (cl == nullopt ? basic_clip_props<GraphicsSurfaces>() : cl.value()))}.

\pnum
\remarks
Performs the stroking rendering and composing operation as specified by \ref{\iotwod.surface.stroking}.

\pnum
The meanings of the parameters are specified by \ref{\iotwod.surface.rendering}.

\pnum
\throws
As specified in Error reporting (\ref{\iotwod.err.report}).

\pnum
\errors
The errors, if any, produced by this function are \impldefplain{basic_image_surface!stroke}.
\end{itemdescr}

\indexlibrarymember{stroke}{basic_image_surface}%
\begin{itemdecl}
void stroke(const basic_brush<GraphicsSurfaces>& b,
  const basic_interpreted_path<GraphicsSurfaces>& ip,
  const optional<basic_brush_props<GraphicsSurfaces>>& bp = nullopt,
  const optional<basic_stroke_props<GraphicsSurfaces>>& sp = nullopt,
  const optional<basic_dashes<GraphicsSurfaces>>& d = nullopt,
  const optional<basic_render_props<GraphicsSurfaces>>& rp = nullopt,
  const optional<basic_clip_props<GraphicsSurfaces>>& cl = nullopt);
\end{itemdecl}
\begin{itemdescr}
\pnum
\effects Calls \tcode{GraphicsSurfaces::surfaces::stroke(data(), b, ip, (bp == nullopt ? basic_brush_props<GraphicsSurfaces>() : bp.value()), (sp == nullopt ? basic_stroke_props<GraphicsSurfaces>() : sp.value()), (d == nullopt ? basic_dashes<GraphicsSurfaces>() : d.value()), (rp == nullopt ? basic_render_props<GraphicsSurfaces>() : rp.value()), (cl == nullopt ? basic_clip_props<GraphicsSurfaces>() : cl.value()))}.

\pnum
\remarks
Performs the stroking rendering and composing operation as specified by \ref{\iotwod.surface.stroking}.

\pnum
The meanings of the parameters are specified by \ref{\iotwod.surface.rendering}.

\pnum
\throws
As specified in Error reporting (\ref{\iotwod.err.report}).

\pnum
\errors
The errors, if any, produced by this function are \impldefplain{basic_image_surface!stroke}.
\end{itemdescr}

\indexlibrarymember{fill}{basic_image_surface}%
\begin{itemdecl}
template <class Allocator>
void fill(const basic_brush<GraphicsSurfaces>& b,
  const basic_path_builder<GraphicsSurfaces, Allocator>& pb,
  const optional<basic_brush_props<GraphicsSurfaces>>& bp = nullopt,
  const optional<basic_render_props<GraphicsSurfaces>>& rp = nullopt,
  const optional<basic_clip_props<GraphicsSurfaces>>& cl = nullopt);
\end{itemdecl}
\begin{itemdescr}
\pnum
\effects Calls \tcode{GraphicsSurfaces::surfaces::fill(data(), b, basic_interpreted_path<GraphicsSurfaces>(pb), (bp == nullopt ? basic_brush_props<GraphicsSurfaces>() : bp.value()), (rp == nullopt ? basic_render_props<GraphicsSurfaces>() : rp.value()), (cl == nullopt ? basic_clip_props<GraphicsSurfaces>() : cl.value()))}.

\pnum
\remarks
Performs the filling rendering and composing operation as specified by \ref{\iotwod.surface.filling}.

\pnum
The meanings of the parameters are specified by \ref{\iotwod.surface.rendering}.

\pnum
\throws
As specified in Error reporting (\ref{\iotwod.err.report}).

\pnum
\errors
The errors, if any, produced by this function are \impldefplain{basic_image_surface!fill}.
\end{itemdescr}

\indexlibrarymember{fill}{basic_image_surface}%
\begin{itemdecl}
void fill(const basic_brush<GraphicsSurfaces>& b,
  const basic_interpreted_path<GraphicsSurfaces>& ip,
  const optional<basic_brush_props<GraphicsSurfaces>>& bp = nullopt,
  const optional<basic_render_props<GraphicsSurfaces>>& rp = nullopt,
  const optional<basic_clip_props<GraphicsSurfaces>>& cl = nullopt);
\end{itemdecl}
\begin{itemdescr}
\pnum
\effects Calls \tcode{GraphicsSurfaces::surfaces::fill(data(), b, ip, (bp == nullopt ? basic_brush_props<GraphicsSurfaces>() : bp.value()), (rp == nullopt ? basic_render_props<GraphicsSurfaces>() : rp.value()), (cl == nullopt ? basic_clip_props<GraphicsSurfaces>() : cl.value()))}.

\pnum
\remarks
Performs the filling rendering and composing operation as specified by \ref{\iotwod.surface.filling}.

\pnum
The meanings of the parameters are specified by \ref{\iotwod.surface.rendering}.

\pnum
\throws
As specified in Error reporting (\ref{\iotwod.err.report}).

\pnum
\errors
The errors, if any, produced by this function are \impldefplain{basic_image_surface!fill}.
\end{itemdescr}

\indexlibrarymember{mask}{basic_image_surface}%
\begin{itemdecl}
void mask(const basic_brush<GraphicsSurfaces>& b,
  const basic_brush<GraphicsSurfaces>& mb,
  const optional<basic_brush_props<GraphicsSurfaces>>& bp = nullopt,
  const optional<basic_mask_props<GraphicsSurfaces>>& mp = nullopt,
  const optional<basic_render_props<GraphicsSurfaces>>& rp = nullopt,
  const optional<basic_clip_props<GraphicsSurfaces>>& cl = nullopt);
\end{itemdecl}
\begin{itemdescr}
\pnum
\effects Calls \tcode{GraphicsSurfaces::surfaces::mask(data(), b, mb, (bp == nullopt ? basic_brush_props<GraphicsSurfaces>() : bp.value()), (mp == nullopt ? basic_mask_props<GraphicsSurfaces>() : mp.value()), (rp == nullopt ? basic_render_props<GraphicsSurfaces>() : rp.value()), (cl == nullopt ? basic_clip_props<GraphicsSurfaces>() : cl.value()))}.

\pnum
\remarks
Performs the masking rendering and composing operation as specified by \ref{\iotwod.surface.masking}.

\pnum
The meanings of the parameters are specified by \ref{\iotwod.surface.rendering}.

\pnum
\throws
As specified in Error reporting (\ref{\iotwod.err.report}).

\pnum
\errors
The errors, if any, produced by this function are \impldefplain{basic_image_surface!mask}.
\end{itemdescr}

\indexlibrarymember{draw_text}{basic_image_surface}%
\begin{itemdecl}
void draw_text(const basic_point_2d<graphics_math_type>& pt,
  const basic_brush<GraphicsSurfaces>& b,
  const basic_font<GraphicsSurfaces>& font, const string& text,
  const optional<basic_text_props<GraphicsSurfaces>>& tp = nullopt,
  const optional<basic_brush_props<GraphicsSurfaces>>& bp = nullopt,
  const optional<basic_stroke_props<GraphicsSurfaces>>& sp = nullopt,
  const optional<basic_dashes<GraphicsSurfaces>>& d = nullopt,
  const optional<basic_render_props<GraphicsSurfaces>>& rp = nullopt,
  const optional<basic_clip_props<GraphicsSurfaces>>& cl = nullopt);
\end{itemdecl}
\begin{itemdescr}
\pnum
\effects Calls \tcode{GraphicsSurfaces::surfaces::draw_text(data(), pt, b, font, text, tp.value_or(basic_text_props<GraphicsSurfaces>()), bp.value_or(basic_brush_props<GraphicsSurfaces>()), sp.value_or(basic_stroke_props<GraphicsSurfaces>()), d.value_or(basic_dashes<GraphicsSurfaces>()), rp.value_or(basic_render_props<GraphicsSurfaces>()), cl.value_or(basic_clip_props<GraphicsSurfaces>()))}.

\pnum
\remarks
Performs the text rendering rendering and composing operation as specified by \ref{\iotwod.surface.textrendering}.

\pnum
The meanings of the parameters are specified by \ref{\iotwod.surface.rendering}.

\pnum
\throws
As specified in Error reporting (\ref{\iotwod.err.report}).

\pnum
\errors
The errors, if any, produced by this function are \impldefplain{basic_image_surface!draw_text}.
\end{itemdescr}

\indexlibrarymember{draw_text}{basic_image_surface}%
\begin{itemdecl}
void draw_text(const basic_bounding_box<graphics_math_type>& bb,
  const basic_brush<GraphicsSurfaces>& b,
  const basic_font<GraphicsSurfaces>& font, const string& text,
  const optional<basic_text_props<GraphicsSurfaces>>& tp = nullopt,
  const optional<basic_brush_props<GraphicsSurfaces>>& bp = nullopt,
  const optional<basic_stroke_props<GraphicsSurfaces>>& sp = nullopt,
  const optional<basic_dashes<GraphicsSurfaces>>& d = nullopt,
  const optional<basic_render_props<GraphicsSurfaces>>& rp = nullopt,
  const optional<basic_clip_props<GraphicsSurfaces>>& cl = nullopt);
\end{itemdecl}
\begin{itemdescr}
\pnum
\effects Calls \tcode{GraphicsSurfaces::surfaces::draw_text(data(), bb, b, font, text, tp.value_or(basic_text_props<GraphicsSurfaces>()), bp.value_or(basic_brush_props<GraphicsSurfaces>()), sp.value_or(basic_stroke_props<GraphicsSurfaces>()), d.value_or(basic_dashes<GraphicsSurfaces>()), rp.value_or(basic_render_props<GraphicsSurfaces>()), cl.value_or(basic_clip_props<GraphicsSurfaces>()))}.

\pnum
\remarks
Performs the text rendering rendering and composing operation as specified by \ref{\iotwod.surface.textrendering}.

\pnum
The meanings of the parameters are specified by \ref{\iotwod.surface.rendering}.

\pnum
\throws
As specified in Error reporting (\ref{\iotwod.err.report}).

\pnum
\errors
The errors, if any, produced by this function are \impldefplain{basic_image_surface!draw_text}.
\end{itemdescr}

\indexlibrarymember{command_list}{basic_image_surface}%
\begin{itemdecl}
future<void> command_list(const basic_command_list<GraphicsSurfaces>& cl);
\end{itemdecl}
\begin{itemdescr}
\pnum
\effects Calls \tcode{GraphicsSurfaces::surfaces::command_list(data(), cl)}.

\pnum
\returns A \tcode{future<void>} object to inform the user when the command list has completed.

\pnum
\remarks
Submits a command list to be processed by the surface. The command list may run on a separate thread. Users shall be responsible for preventing data races.

\pnum
\begin{note}
The ability of the implementation to run command lists on separate threads provides a number of optimization opportunities. As a byproduct, it introduces the potential for data races.

\pnum
Submitting a command list to a \tcode{basic_image_surface} object and then calling one of its other functions, especially one of the rendering and composing operation functions, before the command list has finished execution is highly likely to introduce data races. Attempting to use that object before the command list has finished execution will, at best, produce erroneous results.

\pnum
Some intended uses for command lists are to allow advanced graphics users who need high performance for their applications to run graphics operations in parallel, to allow users to pre-record various sets of operations and run them on an as-needed basis, and to provide a mechanism where users can be sure that the graphics operations they perform will be batched such that the function is guaranteed to return instantly and allow other work that would not introduce data races to proceed in parallel.

\pnum
As such, it is recommended that users choose to use either command lists or the "direct" API (where surface member functions that perform rendering and composing operations, copying image surfaces, saving image data, etc.).

\pnum
This is not to suggest that the direct API is simplistic and only meant for beginners. For light graphics loads it is easier to use since it does not introduce potentials for race conditions, etc. Further, the direct API can be implemented such that its graphics operations are batched and only run when it is efficient to run them or when when observable behavior requirements force their execution, which will allow them to have reasonably good performance.
\end{note}
\end{itemdescr}
