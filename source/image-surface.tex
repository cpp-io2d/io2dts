%!TEX root = io2d.tex
\rSec0 [\iotwod.imagesurface] {Class \tcode{image_surface}}

\rSec1 [\iotwod.imagesurface.summary] {\tcode{image_surface} summary}

\pnum
\indexlibrary{\idxcode{image_surface}}%
The class \tcode{image_surface} derives from the \tcode{surface} class and provides an interface to a raster graphics data graphics resource.

\pnum
The \tcode{image_surface} class modifies its graphics resource using the rendering and composing operations inherited from the \tcode{surface} class or through the \tcode{image_surface::map} function.

\pnum
It has a \term{pixel format} of type \tcode{format}, a \term{width} of type \tcode{int}, and a \term{height} of type \tcode{int}.

\pnum
\begin{note}
Because of the functionality it provides and what it can be used for, it is expected that developers familiar with other graphics technologies will think of the \tcode{image_surface} class as being a form of \term{render target}. This is intentional, though this \documenttypename{} does not formally define or use that term to avoid any minor ambiguities and differences in its meaning between the various graphics technologies that do use the term render target.
\end{note}

\rSec1 [\iotwod.imagesurface.synopsis] {\tcode{image_surface} synopsis}

\begin{codeblock}
namespace std::experimental::io2d::v1 {
  class image_surface : public surface {
  public:
    // \ref{\iotwod.imagesurface.cons}, construct/copy/move/destroy:
    image_surface(io2d::format fmt, int width, int height);
    image_surface(filesystem::path f, image_file_format i, 
      io2d::format fmt);
    image_surface(filesystem::path f, image_file_format i, 
      io2d::format fmt, error_code& ec) noexcept;
    image_surface(image_surface&&);
    image_surface& operator=(image_surface&&);
    
    // \ref{\iotwod.imagesurface.members}, members:
    void save(filesystem::path p, image_file_format i);
    void save(filesystem::path p, image_file_format i, error_code& ec) noexcept;
    void map(const function<void(mapped_surface&)>& action);
    void map(const function<void(mapped_surface&, error_code&)>& action,
      error_code& ec);
    
    // \ref{\iotwod.imagesurface.staticmembers}, static members:
    static int max_width() const noexcept;
    static int max_height() const noexcept;
    
    // \ref{\iotwod.imagesurface.observers}, observers:
    io2d::format format() const noexcept;
    int width() const noexcept;
    int height() const noexcept;
  };
}
\end{codeblock}

\rSec1 [\iotwod.imagesurface.cons] {\tcode{image_surface} constructors and assignment operators}

\indexlibrary{\idxcode{image_surface}!constructor}%
\begin{itemdecl}
image_surface(io2d::format fmt, int w, int h);
\end{itemdecl}
\begin{itemdescr}
\pnum
\requires
\tcode{w} is greater than \tcode{0} and not greater than \tcode{image_surface::max_width()}.

\pnum
\tcode{h} is greater than \tcode{0} and not greater than \tcode{image_surface::max_height()}.

\pnum
\tcode{fmt} is not \tcode{io2d::format::invalid}.

\pnum
\effects
Constructs an object of type \tcode{image_surface}.

\pnum
The pixel format is \tcode{fmt}, the width is \tcode{w}, and the height is \tcode{h}.
\end{itemdescr}

\indexlibrary{\idxcode{image_surface}!constructor}%
\begin{itemdecl}
image_surface(filesystem::path f, image_file_format i,
  io2d::format fmt);
image_surface(filesystem::path f, image_file_format i,
  io2d::format fmt, error_code& ec) noexcept;
\end{itemdecl}
\begin{itemdescr}
\pnum
\requires
\tcode{f} is a file and its contents are data in either JPEG format or PNG format.

\pnum
\tcode{fmt} is not \tcode{io2d::format::invalid}.

\pnum
\effects
Constructs an object of type \tcode{image_surface}.

\pnum
The data of the \underlyingimagesurface is the raster graphics data that results from processing \tcode{f} into uncompressed raster graphics in the manner specified by the standard that specifies how to transform the contents of data contained in \tcode{f} into raster graphics data and then transforming that raster graphics data into the format specified by \tcode{fmt}.

\pnum
The data of \tcode{f} is processed into uncompressed raster graphics data as specified by the value of \tcode{i}.

\pnum
If \tcode{i} is \tcode{image_file_format::unknown}, implementations may attempt to process the data of \tcode{f} into uncompressed raster graphics data. The manner in which it does so is \unspecnorm. If no uncompressed raster graphics data is produced, the error specified below occurs.

\pnum
\begin{note}
The intent of \tcode{image_file_format::unknown} is to allow implementations to support image file formats that are not required to be supported.
\end{note}

\pnum
If the width of the uncompressed raster graphics data would be less than \tcode{1} or greater than \tcode{image_surface::max_width()} or if the height of the uncompressed raster graphics data would be less than \tcode{1} or greater than \tcode{image_surface::max_height()}, the error specified below occurs.

\pnum
The resulting uncompressed raster graphics data is then transformed into the data format specified by \tcode{fmt}. If the format specified by \tcode{fmt} only contains an alpha channel, the values of the color channels, if any, of the \underlyingimagesurface are \unspecnorm. If the format specified by \tcode{fmt} only contains color channels and the resulting uncompressed raster graphics data is in a premultiplied format, then the value of each color channel for each pixel is be divided by the value of the alpha channel for that pixel. The visual data is then set as the visual data of the \underlyingimagesurface.

\pnum
The width is the width of the uncompressed raster graphics data. The height is the height of the uncompressed raster graphics data.

\pnum
\throws
As specified in Error reporting (\ref{\iotwod.err.report}).

\pnum
\errors
Any error that could result from trying to access \tcode{f}, open \tcode{f} for reading, or reading data from \tcode{f}.

\pnum
\tcode{errc::not_supported} if \tcode{image_file_format::unknown} is passed as an argument and the implementation is unable to determine the file format or does not support saving in the image file format it determined.

\pnum
\tcode{errc::invalid_argument} if \tcode{fmt} is \tcode{io2d::format::invalid}.

\pnum
\tcode{errc::argument_out_of_domain} if the width would be less than \tcode{1}, the width would be greater than \tcode{image_surface::max_width()}, the height would be less than \tcode{1}, or the height would be greater than \tcode{image_surface::max_height()}.
\end{itemdescr}

\rSec1 [\iotwod.imagesurface.members] {\tcode{image_surface} members}

\indexlibrarymember{save}{image_surface}%
\begin{itemdecl}
void save(filesystem::path p, image_file_format i);
void save(filesystem::path p, image_file_format i, error_code& ec) noexcept;
\end{itemdecl}
\begin{itemdescr}
\pnum
\requires
\tcode{p} shall be a valid path to a file. The file need not exist provided that the other components of the path are valid.

\pnum
If the file exists, it shall be writable. If the file does not exist, it shall be possible to create the file at the specified path and then the created file shall be writable.

\pnum
\effects
Any pending rendering and composing operations (\ref{\iotwod.surface.rendering}) are performed.

\pnum
The visual data of the \underlyingimagesurface is written to \tcode{p} in the data format specified by \tcode{i}.

\pnum
If \tcode{i} is \tcode{image_file_format::unknown}, it is \impldefplain{image_surface!save} whether the surface is saved in the image file format, if any, that the implementation associates with \tcode{p.extension()} provided that \tcode{p.has_extension() == true}. If \tcode{p.has_extension() == false}, the implementation does not associate an image file format with \tcode{p.extension()}, or the implementation does not support saving in that image file format, the error specified below occurs.

\pnum
\throws
As specified in Error reporting (\ref{\iotwod.err.report}).

\pnum
\errors
Any error that could result from trying to create \tcode{f}, access \tcode{f}, or write data to \tcode{f}.

\pnum
\tcode{errc::not_supported} if \tcode{image_file_format::unknown} is passed as an argument and the implementation is unable to determine the file format or does not support saving in the image file format it determined.
\end{itemdescr}

\indexlibrarymember{map}{image_surface}%
\begin{itemdecl}
void map(const function<void(mapped_surface&)>& action);
void map(const function<void(mapped_surface&, error_code&)>& action, error_code& ec);
\end{itemdecl}
\begin{itemdescr}
\pnum
\effects
Creates a \tcode{mapped_surface} object and calls \tcode{action} using it.

\pnum
The \tcode{mapped_surface} object is created using \tcode{*this}, which allows direct manipulation of the \underlyingsurface.
	
\pnum
\throws
As specified in Error reporting (\ref{\iotwod.err.report}).
	
\pnum
\remarks
Whether changes are committed to the \underlyingsurface immediately or only when the \tcode{mapped_surface} object is destroyed is \unspecnorm.
	
\pnum
Calling this function on an \tcode{image_surface} object and then calling any function on the \tcode{image_surface} object or using the \tcode{image_surface} object before the call to this function has returned shall result in undefined behavior; no diagnostic is required.
	
\pnum
\errors
\tcode{errc::not_supported} if a \tcode{mapped_surface} object cannot be created for the \tcode{image_surface} object. The \tcode{image_surface} object is not modified if an error occurs.
\end{itemdescr}

\rSec1 [\iotwod.imagesurface.staticmembers] {\tcode{image_surface} static members}

\indexlibrarymember{max_width}{image_surface}%
\begin{itemdecl}
static int max_width() const noexcept;
\end{itemdecl}
\begin{itemdescr}
\pnum
\returns
The maximum width for an \tcode{image_surface} object.
\end{itemdescr}

\indexlibrarymember{max_height}{image_surface}%
\begin{itemdecl}
static int max_height() const noexcept;
\end{itemdecl}
\begin{itemdescr}
\pnum
\returns
The maximum height for an \tcode{image_surface} object.
\end{itemdescr}
    

\rSec1 [\iotwod.imagesurface.observers] {\tcode{image_surface} observers}

\indexlibrarymember{format}{image_surface}%
\begin{itemdecl}
io2d::format format() const noexcept;
\end{itemdecl}
\begin{itemdescr}
\pnum
\returns
The pixel format.
\end{itemdescr}

\indexlibrarymember{width}{image_surface}%
\begin{itemdecl}
int width() const noexcept;
\end{itemdecl}
\begin{itemdescr}
\pnum
\returns
The width.
\end{itemdescr}

\indexlibrarymember{height}{image_surface}%
\begin{itemdecl}
int height() const noexcept;
\end{itemdecl}
\begin{itemdescr}
\pnum
\returns
The height.
\end{itemdescr}
