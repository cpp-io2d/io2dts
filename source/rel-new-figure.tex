%!TEX root = io2d.tex
\rSec0 [\iotwod.relnewfigure] {Class template \tcode{basic_figure_items<GraphicsSurfaces>::rel_new_figure}}%

\rSec1 [\iotwod.relnewfigure.intro] {Overview}

\pnum
\indexlibrary{\idxcode{rel_new_figure}}%
The class template \tcode{basic_figure_items<GraphicsSurfaces>::rel_new_figure} describes a figure item that is a new figure command.

\pnum
It has an \term{at point} of type \tcode{point_2d}.

\pnum
The data are stored in an object of type \tcode{typename GraphicsSurfaces::paths::rel_new_figure_data_type}. It is accessible using the \tcode{data} member functions.

\rSec1 [\iotwod.relnewfigure.synopsis] {Synopsis}
\begin{codeblock}
namespace std::experimemtal::io2d::v1 {
  template <class GraphicsSurfaces>
  class basic_figure_items<GraphicsSurfaces>::rel_new_figure {
  public:
    using graphics_math_type = typename GraphicsSurfaces::graphics_math_type;
    using data_type =
      typename GraphicsSurfaces::paths::rel_new_figure_data_type;

    // \ref{\iotwod.relnewfigure.ctor}, construct:
    rel_new_figure();
    explicit rel_new_figure(const basic_point_2d<graphics_math_type>& pt);
    rel_new_figure(const rel_new_figure& other) = default;
    rel_new_figure(rel_new_figure&& other) noexcept = default;

    // assign:
    rel_new_figure& operator=(const rel_new_figure& other) = default;
    rel_new_figure& operator=(rel_new_figure&& other) noexcept = default;

    // \ref{\iotwod.relnewfigure.acc}, accessors:
    const data_type& data() const noexcept;
    data_type& data() noexcept;

    // \ref{\iotwod.relnewfigure.mod}, modifiers:
    void at(const basic_point_2d<graphics_math_type>& pt) noexcept;

    // \ref{\iotwod.relnewfigure.obs}, observers:
    basic_point_2d<graphics_math_type> at() const noexcept;
  };
  
  // \ref{\iotwod.relnewfigure.eq}, equality operators:
  template <class GraphicsSurfaces>
  bool operator==(
    const typename basic_figure_items<GraphicsSurfaces>::rel_new_figure& lhs,
    const typename basic_figure_items<GraphicsSurfaces>::rel_new_figure& rhs) 
    noexcept;  
  template <class GraphicsSurfaces>
  bool operator!=(
    const typename basic_figure_items<GraphicsSurfaces>::rel_new_figure& lhs,
    const typename basic_figure_items<GraphicsSurfaces>::rel_new_figure& rhs) 
    noexcept;  
}
\end{codeblock}

\rSec1 [\iotwod.relnewfigure.ctor] {Constructors}%

\indexlibrary{\idxcode{rel_new_figure}!constructor}%
\begin{itemdecl}
rel_new_figure() noexcept;
\end{itemdecl}
\begin{itemdescr}
\pnum
\effects Constructs an object of type \tcode{rel_new_figure}.

\pnum
\postconditions \tcode{data() == GraphicsSurfaces::paths::create_rel_new_figure()}.

\pnum
\remarks The at point is \tcode{basic_point_2d<graphics_math_type>()}.
\end{itemdescr}

\indexlibrary{\idxcode{rel_new_figure}!constructor}%
\begin{itemdecl}
explicit rel_new_figure(const basic_point_2d<typename
  GraphicsSurfaces::graphics_math_type>& pt) noexcept;
\end{itemdecl}
\begin{itemdescr}
\pnum
\effects Constructs an object of type \tcode{rel_new_figure}.

\pnum
\postconditions \tcode{data() == GraphicsSurfaces::paths::create_rel_new_figure(pt)}.

\pnum
\remarks The at point is \tcode{pt}.
\end{itemdescr}

\rSec1 [\iotwod.relnewfigure.acc] {Accessors}%

\indexlibrarymember{data}{rel_new_figure}%
\begin{itemdecl}
const data_type& data() const noexcept;
data_type& data() noexcept;
\end{itemdecl}
\begin{itemdescr}
\pnum
\returns A reference to the \tcode{rel_new_figure} object's data object (See: \ref{\iotwod.relnewfigure.intro}).
\end{itemdescr}

\rSec1 [\iotwod.relnewfigure.mod] {Modifiers}%

\indexlibrarymember{at}{rel_new_figure}%
\begin{itemdecl}
void at(const basic_point_2d<graphics_math_type>& pt) noexcept;
\end{itemdecl}
\begin{itemdescr}
\pnum
\effects Calls \tcode{GraphicsSurfaces::paths::at(data(), pt)}.

\pnum
\remarks The at point is \tcode{pt}.
\end{itemdescr}

\rSec1 [\iotwod.relnewfigure.obs]{Observers}%

\indexlibrarymember{at}{rel_new_figure}%
\begin{itemdecl}
basic_point_2d<graphics_math_type> at() const noexcept;
\end{itemdecl}
\begin{itemdescr}
\pnum
\returns \tcode{GraphicsSurfaces::paths::at(data())}.

\pnum
\remarks
The returned value is the at point.
\end{itemdescr}

\rSec1 [\iotwod.relnewfigure.eq] {Equality operators}%

\indexlibrarymember{operator==}{rel_new_figure}%
\begin{itemdecl}
template <class GraphicsSurfaces>
bool operator==(
  const typename basic_figure_items<GraphicsSurfaces>::rel_new_figure& lhs,
  const typename basic_figure_items<GraphicsSurfaces>::rel_new_figure& rhs)
  noexcept;
\end{itemdecl}
\begin{itemdescr}
\pnum
\returns \tcode{GraphicsSurfaces::paths::equal(lhs, rhs)}.
\end{itemdescr}

\indexlibrarymember{operator!=}{rel_new_figure}%
\begin{itemdecl}
template <class GraphicsSurfaces>
bool operator!=(
  const typename basic_figure_items<GraphicsSurfaces>::rel_new_figure& lhs,
  const typename basic_figure_items<GraphicsSurfaces>::rel_new_figure& rhs) 
  noexcept;
\end{itemdecl}
\begin{itemdescr}
\pnum
\returns \tcode{GraphicsSurfaces::paths::not_equal(lhs, rhs)}.
\end{itemdescr}
