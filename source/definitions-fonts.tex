%!TEX root = io2d.tex

\indexdefn{alignment line}
\definition{alignment line}{\iotwod.general.defns.alignmentline}
imaginary line to which most glyph images of a font seem to align \\
\lbrack SOURCE: ISO/IEC 9541-1:2012, definition 3.1 \rbrack

\indexdefn{current position}
\definition{current position}{\iotwod.general.defns.currentposition}
a point on a graphics data graphics resource at which the next glyph representation is to be rendered

\indexdefn{design size}
\definition{design size}{\iotwod.general.defns.designsize}
absolute size at which a font is designed to be used \\
\lbrack SOURCE: ISO/IEC 9541-1:2012, definition 3.3 \rbrack

\indexdefn{escapement}
\definition{escapement}{\iotwod.general.defns.escapement}
movement of the current position on the presentation surface after a glyph representation is rendered

\indexdefn{escapement point}
\definition{escapement point}{\iotwod.general.defns.escapementpoint}
a glyph metric; a point in the glyph's standard coordinate system, to which the current position on the graphics data graphics resource is usually translated, after the glyph representation is rendered

\indexdefn{font}
\definition{font}{\iotwod.general.defns.font}
a collection of glyph images having the same basic design, e.g., \textit{Courier Bold Oblique} \\
\lbrack SOURCE: ISO/IEC 9541-1:2012, definition 3.6 \rbrack

\indexdefn{font family}
\definition{font family}{\iotwod.general.defns.fontfamily}
a collection of fonts of common design, e.g., \textit{Courier, Courier Bold, Courier Bold Oblique} \\
\lbrack SOURCE: ISO/IEC 9541-1:2012, definition 3.7 \rbrack

\indexdefn{font metrics}
\definition{font metrics}{\iotwod.general.defns.fontmetrics}
the set of dimensions and positioning information in a font resource common to all glyph representations contained in that font resource \\
\lbrack SOURCE: ISO/IEC 9541-1:2012, definition 3.8 \rbrack

%\indexdefn{font reference}
%\definition{font reference}{\iotwod.general.defns.fontreference}
%the information about a font resource in an electronic document representation, and possible procedures and operations on that information, which identify or describe the desired font \\
%\lbrack SOURCE: ISO/IEC 9541-1:2012, definition 3.9 \rbrack
%
\indexdefn{font resource}
\definition{font resource}{\iotwod.general.defns.fontresource}
a collection of glyph representations together with descriptive and font metric information which are relevant to the collection of glyph representations as a whole \\
\lbrack SOURCE: ISO/IEC 9541-1:2012, definition 3.10 \rbrack

\indexdefn{font size}
\definition{font size}{\iotwod.general.defns.fontsize}
a scalar reference size relative to which most font metrics, glyph shapes and glyph metrics are specified \\
\lbrack SOURCE: ISO/IEC 9541-1:2012, definition 3.11 \rbrack

\indexdefn{glyph}
\definition{glyph}{\iotwod.general.defns.glyph}
a recognizable abstract graphic symbol which is independent of any specific design \\
\lbrack SOURCE: ISO/IEC 9541-1:2012, definition 3.12 \rbrack

\indexdefn{glyph collection}
\definition{glyph collection}{\iotwod.general.defns.glyphcollection}
an identified set of glyphs \\
\lbrack SOURCE: ISO/IEC 9541-1:2012, definition 3.13 \rbrack

\indexdefn{glyph image}
\definition{glyph image}{\iotwod.general.defns.glyphimage}
an image of a glyph, as obtained from a glyph representation rendered and composed to a graphics data graphics resource

\indexdefn{glyph metrics}
\definition{glyph metrics}{\iotwod.general.defns.glyphmetrics}
the set of information in a glyph representation used for defining the dimensions and positioning of the glyph shape \\
\lbrack SOURCE: ISO/IEC 9541-1:2012, definition 3.16 \rbrack

\indexdefn{glyph representation}
\definition{glyph representation}{\iotwod.general.defns.glyphrepresentation}
the glyph shape and glyph metrics associated with a specific glyph in a font resource \\
\lbrack SOURCE: ISO/IEC 9541-1:2012, definition 3.17 \rbrack

\indexdefn{glyph shape}
\definition{glyph shape}{\iotwod.general.defns.glyphshape}
the set of information in a glyph representation used for defining the shape which represents the glyph \\
\lbrack SOURCE: ISO/IEC 9541-1:2012, definition 3.18 \rbrack

\indexdefn{kern}
\definition{kern}{\iotwod.general.defns.kern}
the extension of a glyph shape beyond its position point or escapement point \\
\lbrack SOURCE: ISO/IEC 9541-1:2012, definition 3.19 \rbrack

\indexdefn{position point}
\definition{position point}{\iotwod.general.defns.positionpoint}
a glyph metric; a point in the glyph's standard coordinate system, usually translated to the current position on the graphics data graphics resource before the glyph shape is rendered

\indexdefn{posture}
\definition{posture}{\iotwod.general.defns.posture}
the extent to which the shape of a glyph or set of glyphs appears to incline, including any consequent design or form change \\
\lbrack SOURCE: ISO/IEC 9541-1:2012, definition 3.22 \rbrack

\indexdefn{proportionate width}
\definition{proportionate width}{\iotwod.general.defns.proportionatewidth}
the ratio of a glyph's or set of glyphs' escapement to font height \\
\lbrack SOURCE: ISO/IEC 9541-1:2012, definition 3.24 \rbrack

\indexdefn{stem}
\definition{stem}{\iotwod.general.defns.stem}
the major stroke of a glyph shape \\
\lbrack SOURCE: ISO/IEC 9541-1:2012, definition 3.25 \rbrack

\indexdefn{weight}
\definition{weight}{\iotwod.general.defns.weight}
the ratio of a glyph's or set of glyphs' stem width to font height \\
\lbrack SOURCE: ISO/IEC 9541-1:2012, definition 3.26 \rbrack

\indexdefn{writing mode}
\definition{writing mode}{\iotwod.general.defns.writingmode}
an identified mode for setting of text in a writing system, usually corresponding to a nominal escapement direction of the glyphs in that mode, i.e., left-to-right, right-to-left or top-to-bottom \\
\lbrack SOURCE: ISO/IEC 9541-1:2012, definition 3.27 \rbrack

\indexdefn{body size}
\definition{body size}{\iotwod.general.defns.bodysize}
the font size, measured along the y axis of the glyph's standard coordinate system

\indexdefn{tilinged body size}
\definition{tilinged body size}{\iotwod.general.defns.tilingedbodysize}
a reference size with two components, measured respectively along the \xaxis and \yaxis of the glyph's standard coordinate system

\indexdefn{design frame}
\definition{design frame}{\iotwod.general.defns.designframe}
dimensional expression that specifies the area inside which a set of glyph images can be designed \\
\lbrack SOURCE: ISO/IEC 9541-1:2012, definition 3.30 \rbrack

\indexdefn{bounding box}
\definition{bounding box}{\iotwod.general.defns.boundingbox}
dimensional expression to specify an actual area that a glyph image occupies within a design frame \\
\lbrack SOURCE: ISO/IEC 9541-1:2012, definition 3.31 \rbrack
%
%\indexdefn{blackness}
%\definition{blackness}{\iotwod.general.defns.blackness}
%the ratio of the blackened area of a glyph image to the tilinged body size area of the glyph image \\
%\lbrack SOURCE: ISO/IEC 9541-1:2012, definition 3.32 \rbrack
