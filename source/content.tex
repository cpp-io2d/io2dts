%!TEX root = io2d.tex
\rSec0 [\iotwod.content] {Enum class \tcode{content}}

\rSec1 [\iotwod.content.summary] {\tcode{content} Summary}

\pnum
The \tcode{content} enum class describes the type of data that a \tcode{surface}
object contains. See Table~\ref{tab:\iotwod.content.meanings} for the meaning of
each \tcode{} enumerator.

\rSec1 [\iotwod.content.synopsis] {\tcode{content} Synopsis}

\begin{codeblock}
namespace std { namespace experimental { namespace io2d { inline namespace v1 {
  enum class content {
    color,
    alpha,
    color_alpha
  };
} } } } // namespaces std::experimental::io2d::v1
\end{codeblock}

\rSec1 [\iotwod.content.enumerators] {\tcode{content} Enumerators}

\begin{libreqtab2}
 {\tcode{content} value meanings}
 {tab:\iotwod.content.meanings}
 \\ \topline
 \lhdr{Enumerator}
 & \rhdr{Meaning}
 \\ \capsep
 \endfirsthead
 \continuedcaption\\
 \hline
 \lhdr{Enumerator}
 & \rhdr{Meaning}
 \\ \capsep
 \endhead
 \tcode{color}
 & The \tcode{surface} holds opaque color data only.
 \\
 \tcode{alpha}
 & The \tcode{surface} holds alpha (translucency) data only.
 \\
 \tcode{color_alpha}
 & The \tcode{surface} holds both color and alpha data.
 \\
\end{libreqtab2}
