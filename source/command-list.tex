%!TEX root = io2d.tex
\rSec0 [\iotwod.commandlist] {Class template \tcode{basic_command_list}}

\rSec1 [\iotwod.commandlist.intro] {Overview}

\pnum
\indexlibrary{\idxcode{basic_command_list}}%
The class template \tcode{basic_command_list<\graphicssurfacestemplparamnospace{}>} contains the data that results from a back end pre-compiling (interpreting) a sequence of \tcode{basic_commands<\graphicssurfacestemplparamnospace{}>::command_item} objects.

\pnum
This command list may later be executed by one of the surface types. For \tcode{basic_image_surface}, it may be executed on separate thread.

\pnum
The data are stored in an object of type \tcode{typename \graphicssurfacestemplparamnospace{}::surfaces::command_list_data_type}. It is accessible using the \tcode{data} member function.

\rSec1 [\iotwod.commandlist.synopsis] {\tcode{basic_command_list} synopsis}

\begin{codeblock}
namespace @\fullnamespace{}@ {
  template <class @\graphicssurfacestemplparamnospace{}@>
  class basic_command_list {
  public:
    using data_type = typename 
      @\graphicssurfacestemplparamnospace{}@::surfaces::command_list_data_type;
      
    // \ref{\iotwod.commandlist.ctor}, construct:
    basic_command_list() noexcept;
    template <class InputIterator>
    basic_command_list(InputIterator first, InputIterator last);
    explicit basic_command_list(@\stdqualifier{}@initializer_list<typename
      basic_commands<@\graphicssurfacestemplparamnospace{}@>::command_item>> il);    
    
    // \ref{\iotwod.commandlist.acc}, accessors:
    const data_type& data() const noexcept;
  };
}
\end{codeblock}

\rSec1 [\iotwod.commandlist.ctor] {\tcode{basic_command_list} constructors}

\indexlibrary{\idxcode{basic_command_list}!constructor}%
\begin{itemdecl}
basic_command_list() noexcept;
\end{itemdecl}
\begin{itemdescr}
\pnum
\pnum
\effects
Constructs an object of type \tcode{basic_command_list}.

\pnum
\postconditions
\tcode{data() == \graphicssurfacestemplparamnospace{}::surfaces::create_command_list()}.
\end{itemdescr}

\indexlibrary{\idxcode{basic_command_list}!constructor}%
\begin{itemdecl}
explicit basic_command_list(const basic_bounding_box<@\graphicsmathtemplparamnospace{}@>& bb);
\end{itemdecl}
\begin{itemdescr}
\pnum
\effects
Constructs an object of type \tcode{basic_command_list}.

\pnum
\postconditions
\tcode{data() == \graphicssurfacestemplparamnospace{}::surfaces::create_command_list()}.
\end{itemdescr}

\indexlibrary{\idxcode{basic_command_list}!constructor}%
\begin{itemdecl}
template <class InputIterator>
basic_command_list(InputIterator first, InputIterator last);
\end{itemdecl}
\begin{itemdescr}
\pnum
\effects
Constructs an object of type \tcode{basic_command_list}.

\pnum
\postconditions
\tcode{data() == \graphicssurfacestemplparamnospace{}::surfaces::create_command_list(first, last)}.

\pnum
\begin{note}
The contained data is the result of the back end pre-compiling the series of objects of type \tcode{basic_commands<GraphicsSurfaces>::command_item} from \tcode{first} to the last element before \tcode{last}.
\end{note}
\end{itemdescr}

\indexlibrary{\idxcode{basic_command_list}!constructor}%
\begin{itemdecl}
explicit basic_command_list(@\stdqualifier{}@initializer_list<typename
  basic_commands<GraphicsSurfaces>::command_item> il);
\end{itemdecl}
\begin{itemdescr}
\pnum
\effects
Equivalent to: \tcode{basic_command_list\{ il.begin(), il.end() \}}.
\end{itemdescr}

\rSec1 [\iotwod.commandlist.acc] {Accessors}

\indexlibrarymember{data}{basic_command_list}%
\begin{itemdecl}
const data_type& data() const noexcept;
\end{itemdecl}
\begin{itemdescr}
\pnum
\returns A reference to the \tcode{basic_command_list} object's data object (See: \ref{\iotwod.commandlist.intro}).
\end{itemdescr}
