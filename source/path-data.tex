%!TEX root = io2d.tex
\rSec0 [\iotwod.pathdata] {Class \tcode{path_data}}

\rSec1 [\iotwod.pathdata.synopsis] {\tcode{path_data} synopsis}

\begin{codeblock}
namespace std { namespace experimental { namespace io2d { inline namespace v1 {
  class path_data {
  public:
    // \ref{\iotwod.pathdata.cons}, construct/copy/move/destroy:
    path_data() noexcept;
    path_data(const path_data& other) noexcept;
    path_data& operator=(const path_data& other) noexcept;
    path_data(path_data&& other) noexcept;
    path_data& operator=(path_data&& other) noexcept;
    virtual ~path_data() noexcept;

    // \ref{\iotwod.pathdata.observers}, observers:
    virtual path_data_type type() const noexcept = 0;
  };
} } } }
\end{codeblock}

\rSec1 [\iotwod.pathdata.intro] {\tcode{path_data} Description}

\pnum
\indexlibrary{\idxcode{path_data}}
The class \tcode{path_data} serves as an abstract base class for classes that describe operations performed on path geometries.

\rSec1 [\iotwod.pathdata.cons] {\tcode{path_data} constructors and assignment operators}

\indexlibrary{\idxcode{path_data}!destructor}
\begin{itemdecl}
    virtual ~path_data() noexcept;
\end{itemdecl}
\begin{itemdescr}
	\pnum
	\effects
	Destroys an object of type \tcode{path_data}.
	
\end{itemdescr}

\rSec1 [\iotwod.pathdata.observers]{\tcode{path_data} observers}

\indexlibrary{\idxcode{path_data}!\idxcode{type}}
\indexlibrary{\idxcode{type}!\idxcode{path_data}}
\begin{itemdecl}
    virtual path_data_type type() const noexcept = 0;
\end{itemdecl}
\begin{itemdescr}
	\pnum
	\returns
	The \tcode{path_data_type} of the \tcode{path_data}-derived object.
	
	\pnum
	\realnote
	This is used for casting to the correct type when iterating through a \tcode{vector<path_data>} object.
\end{itemdescr}
