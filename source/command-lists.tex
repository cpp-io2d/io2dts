%!TEX root = io2d.tex

\rSec0 [\iotwod.cmdlists] {Command lists}

\rSec1 [\iotwod.cmdlists.overview]{Overview of command lists}

\pnum
Command lists define operations on surfaces, \term{commands}, that can be submitted to a surface.

\pnum
Commands consist of the rendering and composing operations, other operations on surfaces, and a type that allows a user-provided function to run.

\pnum
Command lists provide a mechanism for efficiently processing graphics operations, allowing them to be executed on multiple threads. Additionally, the \tcode{basic_interpreted_command_list} class template allows command lists to be pre-compiled by the back end, which provides optimization possibilities for back ends that use graphics acceleration hardware.

\rSec1 [\iotwod.cmdlists.commands] {Class template \tcode{basic_commands}}

\addtocounter{SectionDepthBase}{2}
%%!TEX root = io2d.tex

\rSec0 [\iotwod.cmdlists.commands.intro] {Introduction}

\pnum
The nested classes within the class template \tcode{basic_commands} describe TODO.


%!TEX root = io2d.tex

\rSec0 [\iotwod.cmdlists.clear] {Class template \tcode{basic_commands<GraphicsSurfaces>::clear}}

\rSec1 [\iotwod.cmdlists.clear.intro] {Overview}

\pnum
\indexlibrary{\idxcode{clear}}%
The class template \tcode{basic_commands<GraphicsSurfaces>::clear} describes a command that invokes the \tcode{clear} member function of a surface.
%clears a surface by performing a paint rendering and composing operation using the \tcode{compositing_op::clear} composition algorithm with no .

\pnum
It has an \term{optional surface} of type \tcode{optional<reference_wrapper<basic_image_surface<GraphicsSurfaces>>>}. If optional surface has a value, the clear operation is performed on the optional surface instead of the surface that the command is submitted to.

\pnum
If optional surface has a value and the referenced \tcode{basic_image_surface<GraphicsSurfaces>} object has been destroyed or otherwise rendered invalid when a \tcode{basic_command_list<GraphicsSurfaces>} object built using this \tcode{paint} object is used by the program, the effects are undefined.

\pnum
The data are stored in an object of type \tcode{typename GraphicsSurfaces::surfaces::clear_data_type}. It is accessible using the \tcode{data} member functions.

\rSec1 [\iotwod.cmdlists.clear.synopsis] {Synopsis}
\begin{codeblock}
namespace @\fullnamespace{}@ {
  template <class GraphicsSurfaces>
  class basic_commands<GraphicsSurfaces::clear {
  public:
    using graphics_math_type = typename GraphicsSurfaces::graphics_math_type;
    using data_type = typename GraphicsSurfaces::surfaces::clear_data_type;

    // \ref{\iotwod.cmdlists.clear.ctor}, construct:
    clear() noexcept;
    clear(reference_wrapper<basic_image_surface<GraphicsSurfaces>> sfc) 
      noexcept;
    
    // \ref{\iotwod.cmdlists.clear.acc}, accessors:
    const data_type& data() const noexcept;
    data_type& data() noexcept;

    // \ref{\iotwod.cmdlists.clear.mod}, modifiers:
    void surface(
      optional<reference_wrapper<basic_image_surface<GraphicsSurfaces>>> sfc) 
      noexcept;

    // \ref{\iotwod.cmdlists.clear.obs}, observers:
    optional<reference_wrapper<basic_image_surface<GraphicsSurfaces>>> 
      surface() const noexcept;
  };

  // \ref{\iotwod.cmdlists.clear.eq}, equality operators:
  template <class GraphicsSurfaces>
  bool operator==(
    const typename basic_commands<GraphicsSurfaces::clear& lhs,
    const typename basic_commands<GraphicsSurfaces::clear& rhs) 
    noexcept;
  template <class GraphicsSurfaces>
  bool operator!=(
    const typename basic_commands<GraphicsSurfaces::clear& lhs,
    const typename basic_commands<GraphicsSurfaces::clear& rhs) 
    noexcept;
}
\end{codeblock}

\rSec1 [\iotwod.cmdlists.clear.ctor] {Constructors}%

\indexlibrary{\idxcode{clear}!constructor}%
\begin{itemdecl}
clear();
\end{itemdecl}
\begin{itemdescr}
\pnum
\effects Constructs an object of type \tcode{clear}.

\pnum
\postconditions \tcode{data() == GraphicsSurfaces::surfaces::create_clear()}.
\end{itemdescr}

\rSec1 [\iotwod.cmdlists.clear.acc] {Accessors}%

\indexlibrarymember{data}{clear}%
\begin{itemdecl}
const data_type& data() const noexcept;
data_type& data() noexcept;
\end{itemdecl}
\begin{itemdescr}
\pnum
\returns A reference to the \tcode{clear} object's data object (See: \ref{\iotwod.cmdlists.clear.intro}).

\pnum
\remarks The behavior of a program is undefined if the user modifies the data contained in the \tcode{data_type} object returned by this function.
\end{itemdescr}

\rSec1 [\iotwod.cmdlists.clear.mod] {Modifiers}%

\indexlibrarymember{surface}{clear}%
\begin{itemdecl}
void surface(
  optional<reference_wrapper<basic_image_surface<GraphicsSurfaces>>> sfc) 
  noexcept;
\end{itemdecl}
\begin{itemdescr}
\pnum
\effects Calls \tcode{GraphicsSurfaces::surfaces::surface(data(), sfc)}.

\pnum
\remarks The optional surface is \tcode{sfc}.
\end{itemdescr}

\rSec1 [\iotwod.cmdlists.clear.obs] {Observers}%

\indexlibrarymember{surface}{clear}%
\begin{itemdecl}
optional<reference_wrapper<basic_image_surface<GraphicsSurfaces>>> 
  surface() const noexcept;
\end{itemdecl}
\begin{itemdescr}
\pnum
\returns \tcode{GraphicsSurfaces::surfaces::surface(data())}.

\pnum
\remarks
The returned value is the optional surface.
\end{itemdescr}

\rSec1 [\iotwod.cmdlists.clear.eq] {Equality operators}%

\indexlibrarymember{operator==}{clear}%
\begin{itemdecl}
template <class GraphicsSurfaces>
bool operator==(
  const typename basic_commands<GraphicsSurfaces::clear& lhs,
  const typename basic_commands<GraphicsSurfaces::clear& rhs) 
  noexcept;
\end{itemdecl}
\begin{itemdescr}
\pnum
\returns \tcode{GraphicsSurfaces::surfaces::equal(lhs.data(), rhs.data())}.
\end{itemdescr}

\indexlibrarymember{operator!=}{clear}%
\begin{itemdecl}
template <class GraphicsSurfaces>
bool operator!=(
  const typename basic_commands<GraphicsSurfaces::clear& lhs,
  const typename basic_commands<GraphicsSurfaces::clear& rhs) 
  noexcept;
\end{itemdecl}
\begin{itemdescr}
\pnum
\returns \tcode{GraphicsSurfaces::surfaces::not_equal(lhs.data(), rhs.data())}.
\end{itemdescr}

%\input{flush}
%\input{mark-dirty}
%!TEX root = io2d.tex

\rSec0 [\iotwod.cmdlists.commands.paint] {Class template \tcode{basic_commands<GraphicsSurfaces>::paint}}

\rSec1 [\iotwod.cmdlists.paint.intro] {Overview}

\pnum
\indexlibrary{\idxcode{paint}}%
The class template \tcode{basic_commands<GraphicsSurfaces>::paint} describes a command that invokes the \tcode{paint} member function of a surface.

\pnum
It has an \term{optional surface} of type \tcode{optional<reference_wrapper<basic_image_surface<GraphicsSurfaces>>>}. If optional surface has a value, the paint operation is performed on the optional surface instead of the surface that the command list is submitted to.

\pnum
If optional surface has a value and the referenced \tcode{basic_image_surface<GraphicsSurfaces>} object has been destroyed or otherwise rendered invalid when a \tcode{basic_command_list<GraphicsSurfaces>} object built using this \tcode{paint} object is used by the program, the effects are undefined.

\pnum
It has a \term{brush} of type \tcode{basic_brush<GraphicsSurfaces}.

\pnum
It has a \term{brush props} of type \tcode{basic_brush_props<GraphicsSurfaces>}.

\pnum
It has a \term{render props} of type \tcode{basic_render_props<GraphicsSurfaces>}.

\pnum
It has a \term{clip props} of type \tcode{basic_clip_props}.

\pnum
The data are stored in an object of type \tcode{typename GraphicsSurfaces::surfaces::paint_data_type}. It is accessible using the \tcode{data} member functions.

\pnum
The data are used as arguments for the invocation of the \tcode{paint} member function of the appropriate surface when a \tcode{basic_command_list<GraphicsSurfaces>} object built using this \tcode{paint} object is used by the program.

\rSec1 [\iotwod.cmdlists.paint.synopsis] {Synopsis}
\begin{codeblock}
namespace @\fullnamespace{}@ {
  template <class GraphicsSurfaces>
  class basic_commands<GraphicsSurfaces::paint {
  public:
    using graphics_math_type = typename GraphicsSurfaces::graphics_math_type;
    using data_type = typename GraphicsSurfaces::surfaces::paint_data_type;

    // \ref{\iotwod.cmdlists.paint.ctor}, construct:
    paint(const basic_brush<GraphicsSurfaces>& b,
      const basic_brush_props<GraphicsSurfaces>& bp = 
      basic_brush_props<GraphicsSurfaces>{},
      const basic_render_props<GraphicsSurfaces>& rp = 
      basic_render_props<GraphicsSurfaces>{},
      const basic_clip_props<GraphicsSurfaces>& cl = 
      basic_clip_props<GraphicsSurfaces>{}) noexcept;
    paint(reference_wrapper<basic_image_surface<GraphicsSurfaces>> sfc,
      const basic_brush<GraphicsSurfaces>& b,
      const basic_brush_props<GraphicsSurfaces>& bp = 
      basic_brush_props<GraphicsSurfaces>{},
      const basic_render_props<GraphicsSurfaces>& rp = 
      basic_render_props<GraphicsSurfaces>{},
      const basic_clip_props<GraphicsSurfaces>& cl = 
      basic_clip_props<GraphicsSurfaces>{}) noexcept;

    // \ref{\iotwod.cmdlists.paint.acc}, accessors:
    const data_type& data() const noexcept;
    data_type& data() noexcept;

    // \ref{\iotwod.cmdlists.paint.mod}, modifiers:
    void surface(
      optional<reference_wrapper<basic_image_surface<GraphicsSurfaces>>> sfc) 
      noexcept;
    void brush(const basic_brush<GraphicsSurfaces>& b) noexcept;
    void brush_props(const basic_brush_props<GraphicsSurfaces>& bp) noexcept;
    void render_props(const basic_render_props<GraphicsSurfaces>& rp) noexcept;
    void clip_props(const basic_clip_props<GraphicsSurfaces>& cl) noexcept;

    // \ref{\iotwod.cmdlists.paint.obs}, observers:
    optional<reference_wrapper<basic_image_surface<GraphicsSurfaces>>> 
      surface() const noexcept;
    basic_brush<GraphicsSurfaces> brush() const noexcept;
    basic_brush_props<GraphicsSurfaces> brush_props() const noexcept;
    basic_render_props<GraphicsSurfaces> render_props() const noexcept;
    basic_clip_props<GraphicsSurfaces> clip_props() const noexcept;
  };

  // \ref{\iotwod.cmdlists.paint.eq}, equality operators:
  template <class GraphicsSurfaces>
  bool operator==(
    const typename basic_commands<GraphicsSurfaces::paint& lhs,
    const typename basic_commands<GraphicsSurfaces::paint& rhs) 
    noexcept;
  template <class GraphicsSurfaces>
  bool operator!=(
    const typename basic_commands<GraphicsSurfaces::paint& lhs,
    const typename basic_commands<GraphicsSurfaces::paint& rhs) 
    noexcept;
}
\end{codeblock}

\rSec1 [\iotwod.cmdlists.paint.ctor] {Constructors}%

\indexlibrary{\idxcode{paint}!constructor}%
\begin{itemdecl}
paint(const basic_brush<GraphicsSurfaces>& b,
  const basic_brush_props<GraphicsSurfaces>& bp = 
  basic_brush_props<GraphicsSurfaces>{},
  const basic_render_props<GraphicsSurfaces>& rp = 
  basic_render_props<GraphicsSurfaces>{},
  const basic_clip_props<GraphicsSurfaces>& cl = 
  basic_clip_props<GraphicsSurfaces>{}) noexcept;
\end{itemdecl}
\begin{itemdescr}
\pnum
\effects Constructs an object of type \tcode{paint}.

\pnum
\postconditions \tcode{data() == GraphicsSurfaces::surfaces::create_paint(b, bp, rp, cl)}.
\end{itemdescr}

\indexlibrary{\idxcode{paint}!constructor}%
\begin{itemdecl}
paint(reference_wrapper<basic_image_surface<GraphicsSurfaces>> sfc,
  const basic_brush<GraphicsSurfaces>& b,
  const basic_brush_props<GraphicsSurfaces>& bp = 
  basic_brush_props<GraphicsSurfaces>{},
  const basic_render_props<GraphicsSurfaces>& rp = 
  basic_render_props<GraphicsSurfaces>{},
  const basic_clip_props<GraphicsSurfaces>& cl = 
  basic_clip_props<GraphicsSurfaces>{}) noexcept;
\end{itemdecl}
\begin{itemdescr}
\pnum
\effects Constructs an object of type \tcode{paint}.

\pnum
\postconditions \tcode{data() == GraphicsSurfaces::surfaces::create_paint(sfc, b, bp, rp, cl)}.
\end{itemdescr}

\rSec1 [\iotwod.cmdlists.paint.acc] {Accessors}%

\indexlibrarymember{data}{paint}%
\begin{itemdecl}
const data_type& data() const noexcept;
data_type& data() noexcept;
\end{itemdecl}
\begin{itemdescr}
\pnum
\returns A reference to the \tcode{paint} object's data object (See: \ref{\iotwod.cmdlists.paint.intro}).

\pnum
\remarks The behavior of a program is undefined if the user modifies the data contained in the \tcode{data_type} object returned by this function.
\end{itemdescr}

\rSec1 [\iotwod.cmdlists.paint.mod] {Modifiers}%

\indexlibrarymember{surface}{paint}%
\begin{itemdecl}
void surface(
  optional<reference_wrapper<basic_image_surface<GraphicsSurfaces>>> sfc) 
  noexcept;
\end{itemdecl}
\begin{itemdescr}
\pnum
\effects Calls \tcode{GraphicsSurfaces::surfaces::surface(data(), sfc)}.

\pnum
\remarks The optional surface is \tcode{sfc}.
\end{itemdescr}

\indexlibrarymember{brush}{paint}%
\begin{itemdecl}
void brush(const basic_brush<GraphicsSurfaces>& b) noexcept;
\end{itemdecl}
\begin{itemdescr}
\pnum
\effects Calls \tcode{GraphicsSurfaces::surfaces::brush(data(), b)}.

\pnum
\remarks The brush is \tcode{b}.
\end{itemdescr}

\indexlibrarymember{brush_props}{paint}%
\begin{itemdecl}
void brush_props(const basic_brush_props<GraphicsSurfaces>& bp) noexcept;
\end{itemdecl}
\begin{itemdescr}
\pnum
\effects Calls \tcode{GraphicsSurfaces::surfaces::brush_props(data(), bp)}.

\pnum
\remarks The brush props is \tcode{bp}.
\end{itemdescr}

\indexlibrarymember{render_props}{paint}%
\begin{itemdecl}
void render_props(const basic_render_props<GraphicsSurfaces>& rp) noexcept;
\end{itemdecl}
\begin{itemdescr}
\pnum
\effects Calls \tcode{GraphicsSurfaces::surfaces::render_props(data(), rp)}.

\pnum
\remarks The render props is \tcode{rp}.
\end{itemdescr}

\indexlibrarymember{clip_props}{paint}%
\begin{itemdecl}
void clip_props(const basic_clip_props<GraphicsSurfaces>& cl) noexcept;
\end{itemdecl}
\begin{itemdescr}
\pnum
\effects Calls \tcode{GraphicsSurfaces::surfaces::clip_props(data(), cl)}.

\pnum
\remarks The clip props is \tcode{cl}.
\end{itemdescr}

\rSec1 [\iotwod.cmdlists.paint.obs] {Observers}%

\indexlibrarymember{surface}{paint}%
\begin{itemdecl}
optional<reference_wrapper<basic_image_surface<GraphicsSurfaces>>> 
  surface() const noexcept;
\end{itemdecl}
\begin{itemdescr}
\pnum
\returns \tcode{GraphicsSurfaces::surfaces::surface(data())}.

\pnum
\remarks
The returned value is the optional surface.
\end{itemdescr}

\indexlibrarymember{brush}{paint}%
\begin{itemdecl}
basic_brush<GraphicsSurfaces> brush() const noexcept;
\end{itemdecl}
\begin{itemdescr}
\pnum
\returns \tcode{GraphicsSurfaces::surfaces::brush(data())}.

\pnum
\remarks The returned value is the brush.
\end{itemdescr}

\indexlibrarymember{brush_props}{paint}%
\begin{itemdecl}
basic_brush_props<GraphicsSurfaces> brush_props() const noexcept;
\end{itemdecl}
\begin{itemdescr}
\pnum
\returns \tcode{GraphicsSurfaces::surfaces::brush_props(data())}.

\pnum
\remarks The returned value is the brush props.
\end{itemdescr}

\indexlibrarymember{render_props}{paint}%
\begin{itemdecl}
basic_render_props<GraphicsSurfaces> render_props() const noexcept;
\end{itemdecl}
\begin{itemdescr}
\pnum
\returns \tcode{GraphicsSurfaces::surfaces::render_props(data())}.

\pnum
\remarks The returned value is the render props.
\end{itemdescr}

\indexlibrarymember{clip_props}{paint}%
\begin{itemdecl}
basic_clip_props<GraphicsSurfaces> clip_props() const noexcept;
\end{itemdecl}
\begin{itemdescr}
\pnum
\returns \tcode{GraphicsSurfaces::surfaces::clip_props(data())}.

\pnum
\remarks The returned value is the clip props.
\end{itemdescr}

\rSec1 [\iotwod.cmdlists.paint.eq] {Equality operators}%

\indexlibrarymember{operator==}{paint}%
\begin{itemdecl}
template <class GraphicsSurfaces>
bool operator==(
  const typename basic_commands<GraphicsSurfaces::paint& lhs,
  const typename basic_commands<GraphicsSurfaces::paint& rhs) 
  noexcept;
\end{itemdecl}
\begin{itemdescr}
\pnum
\returns \tcode{GraphicsSurfaces::surfaces::equal(lhs.data(), rhs.data())}.
\end{itemdescr}

\indexlibrarymember{operator!=}{paint}%
\begin{itemdecl}
template <class GraphicsSurfaces>
bool operator!=(
  const typename basic_commands<GraphicsSurfaces::paint& lhs,
  const typename basic_commands<GraphicsSurfaces::paint& rhs) 
  noexcept;
\end{itemdecl}
\begin{itemdescr}
\pnum
\returns \tcode{GraphicsSurfaces::surfaces::not_equal(lhs.data(), rhs.data())}.
\end{itemdescr}

%!TEX root = io2d.tex

\rSec0 [\iotwod.cmdlists.commands.stroke] {Class template \tcode{basic_commands<GraphicsSurfaces>::stroke}}

\rSec1 [\iotwod.cmdlists.stroke.intro] {Overview}

\pnum
\indexlibrary{\idxcode{stroke}}%
The class template \tcode{basic_commands<GraphicsSurfaces>::stroke} describes a command that invokes the \tcode{stroke} member function of a surface.

\pnum
It has an \term{optional surface} of type \tcode{optional<reference_wrapper<basic_image_surface<GraphicsSurfaces>>>}. If optional surface has a value, the stroke operation is performed on the optional surface instead of the surface that the command is submitted to.

\pnum
If optional surface has a value and the referenced \tcode{basic_image_surface<GraphicsSurfaces>} object has been destroyed or otherwise rendered invalid when a \tcode{basic_command_list<GraphicsSurfaces>} object built using this \tcode{paint} object is used by the program, the effects are undefined.

\pnum
It has a \term{brush} of type \tcode{basic_brush<GraphicsSurfaces>}.

\pnum
It has a \term{path} of type \tcode{basic_interpreted_path<GraphicsSurfaces>}.

\pnum
It has a \term{brush props} of type \tcode{basic_brush_props<GraphicsSurfaces>}.

\pnum
It has a \term{stroke props} of type \tcode{basic_stroke_props<GraphicsSurfaces>}.

\pnum
It has a \term{dashes} of type \tcode{basic_dashes<GraphicsSurfaces>}.

\pnum
It has a \term{render props} of type \tcode{basic_render_props<GraphicsSurfaces>}.

\pnum
It has a \term{clip props} of type \tcode{basic_clip_props}.

\pnum
The data are stored in an object of type \tcode{typename GraphicsSurfaces::surfaces::stroke_data_type}. It is accessible using the \tcode{data} member functions.

\pnum
The data are used as arguments for the invocation of the \tcode{stroke} member function of the appropriate surface when a \tcode{basic_command_list<GraphicsSurfaces>} object built using this \tcode{stroke} object is used by the program.

\rSec1 [\iotwod.cmdlists.stroke.synopsis] {Synopsis}
\begin{codeblock}
namespace @\fullnamespace{}@ {
  template <class GraphicsSurfaces>
  class basic_commands<GraphicsSurfaces::stroke {
  public:
    using graphics_math_type = typename GraphicsSurfaces::graphics_math_type;
    using data_type = typename GraphicsSurfaces::surfaces::stroke_data_type;

    // \ref{\iotwod.cmdlists.stroke.ctor}, construct:
    stroke(const basic_brush<GraphicsSurfaces>& b,
      const basic_interpreted_path<GraphicsSurfaces>& ip,
      const basic_brush_props<GraphicsSurfaces>& bp = 
      basic_brush_props<GraphicsSurfaces>{},
      const basic_stroke_props<GraphicsSurfaces>& sp = 
      basic_stroke_props<GraphicsSurfaces>{},
      const basic_dashes<GraphicsSurfaces>& d = 
      basic_dashes<GraphicsSurfaces>{},
      const basic_render_props<GraphicsSurfaces>& rp = 
      basic_render_props<GraphicsSurfaces>{},
      const basic_clip_props<GraphicsSurfaces>& cl = 
      basic_clip_props<GraphicsSurfaces>{}) noexcept;
    stroke(reference_wrapper<basic_image_surface<GraphicsSurfaces>> sfc,
      const basic_brush<GraphicsSurfaces>& b,
      const basic_interpreted_path<GraphicsSurfaces>& ip,
      const basic_brush_props<GraphicsSurfaces>& bp = 
      basic_brush_props<GraphicsSurfaces>{},
      const basic_stroke_props<GraphicsSurfaces>& sp = 
      basic_stroke_props<GraphicsSurfaces>{},
      const basic_dashes<GraphicsSurfaces>& d =
      basic_dashes<GraphicsSurfaces>{},
      const basic_render_props<GraphicsSurfaces>& rp = 
      basic_render_props<GraphicsSurfaces>{},
      const basic_clip_props<GraphicsSurfaces>& cl = 
      basic_clip_props<GraphicsSurfaces>{}) noexcept;
    
    // \ref{\iotwod.cmdlists.stroke.acc}, accessors:
    const data_type& data() const noexcept;
    data_type& data() noexcept;

    // \ref{\iotwod.cmdlists.stroke.mod}, modifiers:
    void surface(
      optional<reference_wrapper<basic_image_surface<GraphicsSurfaces>>> sfc) 
      noexcept;
    void brush(const basic_brush<GraphicsSurfaces>& b) noexcept;
    void path(const basic_interpreted_path<GraphicsSurfaces>& p) noexcept;
    void brush_props(const basic_brush_props<GraphicsSurfaces>& bp) noexcept;
    void stroke_props(const basic_stroke_props<GraphicsSurfaces>& sp) noexcept;
    void dashes(const basic_dashes<GraphicsSurfaces>& d) noexcept;
    void render_props(const basic_render_props<GraphicsSurfaces>& rp) noexcept;
    void clip_props(const basic_clip_props<GraphicsSurfaces>& cl) noexcept;

    // \ref{\iotwod.cmdlists.stroke.obs}, observers:
    optional<reference_wrapper<basic_image_surface<GraphicsSurfaces>>> 
      surface() const noexcept;
    basic_brush<GraphicsSurfaces> brush() const noexcept;
    basic_interpreted_path<GraphicsSurfaces> path() const noexcept;
    basic_brush_props<GraphicsSurfaces> brush_props() const noexcept;
    basic_stroke_props<GraphicsSurfaces> stroke_props() const noexcept;
    basic_dashes<GraphicsSurfaces> dashes() const noexcept;
    basic_render_props<GraphicsSurfaces> render_props() const noexcept;
    basic_clip_props<GraphicsSurfaces> clip_props() const noexcept;
  };

  // \ref{\iotwod.cmdlists.stroke.eq}, equality operators:
  template <class GraphicsSurfaces>
  bool operator==(
    const typename basic_commands<GraphicsSurfaces::stroke& lhs,
    const typename basic_commands<GraphicsSurfaces::stroke& rhs) 
    noexcept;
  template <class GraphicsSurfaces>
  bool operator!=(
    const typename basic_commands<GraphicsSurfaces::stroke& lhs,
    const typename basic_commands<GraphicsSurfaces::stroke& rhs) 
    noexcept;
}
\end{codeblock}

\rSec1 [\iotwod.cmdlists.stroke.ctor] {Constructors}%

\indexlibrary{\idxcode{stroke}!constructor}%
\begin{itemdecl}
stroke(const basic_brush<GraphicsSurfaces>& b,
  const basic_interpreted_path<GraphicsSurfaces>& ip,
  const basic_brush_props<GraphicsSurfaces>& bp = 
  basic_brush_props<GraphicsSurfaces>{},
  const basic_stroke_props<GraphicsSurfaces>& sp = 
  basic_stroke_props<GraphicsSurfaces>{},
  const basic_dashes<GraphicsSurfaces>& d = 
  basic_dashes<GraphicsSurfaces>{},
  const basic_render_props<GraphicsSurfaces>& rp = 
  basic_render_props<GraphicsSurfaces>{},
  const basic_clip_props<GraphicsSurfaces>& cl = 
  basic_clip_props<GraphicsSurfaces>{}) noexcept;
\end{itemdecl}
\begin{itemdescr}
\pnum
\effects Constructs an object of type \tcode{stroke}.

\pnum
\postconditions \tcode{data() == GraphicsSurfaces::surfaces::create_stroke(b, ip, bp, sp, d, rp, cl)}.
\end{itemdescr}

\indexlibrary{\idxcode{stroke}!constructor}%
\begin{itemdecl}
stroke(reference_wrapper<basic_image_surface<GraphicsSurfaces>> sfc,
  const basic_brush<GraphicsSurfaces>& b,
  const basic_interpreted_path<GraphicsSurfaces>& ip,
  const basic_brush_props<GraphicsSurfaces>& bp = 
  basic_brush_props<GraphicsSurfaces>{},
  const basic_stroke_props<GraphicsSurfaces>& sp = 
  basic_stroke_props<GraphicsSurfaces>{},
  const basic_dashes<GraphicsSurfaces>& d =
  basic_dashes<GraphicsSurfaces>{},
  const basic_render_props<GraphicsSurfaces>& rp = 
  basic_render_props<GraphicsSurfaces>{},
  const basic_clip_props<GraphicsSurfaces>& cl = 
  basic_clip_props<GraphicsSurfaces>{}) noexcept;
\end{itemdecl}
\begin{itemdescr}
\pnum
\effects Constructs an object of type \tcode{stroke}.

\pnum
\postconditions \tcode{data() == GraphicsSurfaces::surfaces::create_stroke(sfc, b, ip, bp, sp, d, rp, cl)}.
\end{itemdescr}

\rSec1 [\iotwod.cmdlists.stroke.acc] {Accessors}%

\indexlibrarymember{data}{stroke}%
\begin{itemdecl}
const data_type& data() const noexcept;
data_type& data() noexcept;
\end{itemdecl}
\begin{itemdescr}
\pnum
\returns A reference to the \tcode{stroke} object's data object (See: \ref{\iotwod.cmdlists.stroke.intro}).

\pnum
\remarks The behavior of a program is undefined if the user modifies the data contained in the \tcode{data_type} object returned by this function.
\end{itemdescr}

\rSec1 [\iotwod.cmdlists.stroke.mod] {Modifiers}%

\indexlibrarymember{surface}{stroke}%
\begin{itemdecl}
void surface(
  optional<reference_wrapper<basic_image_surface<GraphicsSurfaces>>> sfc) 
  noexcept;
\end{itemdecl}
\begin{itemdescr}
\pnum
\effects Calls \tcode{GraphicsSurfaces::surfaces::surface(data(), sfc)}.

\pnum
\remarks The optional surface is \tcode{sfc}.
\end{itemdescr}

\indexlibrarymember{brush}{stroke}%
\begin{itemdecl}
void brush(const basic_brush<GraphicsSurfaces>& b) noexcept;
\end{itemdecl}
\begin{itemdescr}
\pnum
\effects Calls \tcode{GraphicsSurfaces::surfaces::brush(data(), b)}.

\pnum
\remarks The brush is \tcode{b}.
\end{itemdescr}

\indexlibrarymember{path}{stroke}%
\begin{itemdecl}
void path(const basic_interpreted_path<GraphicsSurfaces>& p) noexcept;
\end{itemdecl}
\begin{itemdescr}
\pnum
\effects Calls \tcode{GraphicsSurfaces::surfaces::path(data(), p)}.

\pnum
\remarks The path is \tcode{p}.
\end{itemdescr}

\indexlibrarymember{brush_props}{stroke}%
\begin{itemdecl}
void brush_props(const basic_brush_props<GraphicsSurfaces>& bp) noexcept;
\end{itemdecl}
\begin{itemdescr}
\pnum
\effects Calls \tcode{GraphicsSurfaces::surfaces::brush_props(data(), bp)}.

\pnum
\remarks The brush props is \tcode{bp}.
\end{itemdescr}

\indexlibrarymember{stroke_props}{stroke}%
\begin{itemdecl}
void brush_props(const basic_stroke_props<GraphicsSurfaces>& sp) noexcept;
\end{itemdecl}
\begin{itemdescr}
\pnum
\effects Calls \tcode{GraphicsSurfaces::surfaces::stroke_props(data(), sp)}.

\pnum
\remarks The stroke props is \tcode{sp}.
\end{itemdescr}

\indexlibrarymember{dashes}{stroke}%
\begin{itemdecl}
void dashes(const basic_dashes<GraphicsSurfaces>& d) noexcept;
\end{itemdecl}
\begin{itemdescr}
\pnum
\effects Calls \tcode{GraphicsSurfaces::surfaces::dashes(data(), d)}.

\pnum
\remarks The dashes is \tcode{d}.
\end{itemdescr}

\indexlibrarymember{render_props}{stroke}%
\begin{itemdecl}
void render_props(const basic_render_props<GraphicsSurfaces>& rp) noexcept;
\end{itemdecl}
\begin{itemdescr}
\pnum
\effects Calls \tcode{GraphicsSurfaces::surfaces::render_props(data(), rp)}.

\pnum
\remarks The render props is \tcode{rp}.
\end{itemdescr}

\indexlibrarymember{clip_props}{stroke}%
\begin{itemdecl}
void clip_props(const basic_clip_props<GraphicsSurfaces>& cl) noexcept;
\end{itemdecl}
\begin{itemdescr}
\pnum
\effects Calls \tcode{GraphicsSurfaces::surfaces::clip_props(data(), cl)}.

\pnum
\remarks The clip props is \tcode{cl}.
\end{itemdescr}

\rSec1 [\iotwod.cmdlists.stroke.obs] {Observers}%

\indexlibrarymember{surface}{stroke}%
\begin{itemdecl}
optional<reference_wrapper<basic_image_surface<GraphicsSurfaces>>> 
  surface() const noexcept;
\end{itemdecl}
\begin{itemdescr}
\pnum
\returns \tcode{GraphicsSurfaces::surfaces::surface(data())}.

\pnum
\remarks
The returned value is the optional surface.
\end{itemdescr}

\indexlibrarymember{brush}{stroke}%
\begin{itemdecl}
basic_brush<GraphicsSurfaces> brush() const noexcept;
\end{itemdecl}
\begin{itemdescr}
\pnum
\returns \tcode{GraphicsSurfaces::surfaces::brush(data())}.

\pnum
\remarks The returned value is the brush.
\end{itemdescr}

\indexlibrarymember{path}{stroke}%
\begin{itemdecl}
basic_interpreted_path<GraphicsSurfaces> path() const noexcept;
\end{itemdecl}
\begin{itemdescr}
\pnum
\returns \tcode{GraphicsSurfaces::surfaces::path(data())}.

\pnum
\remarks The returned value is the path.
\end{itemdescr}

\indexlibrarymember{brush_props}{stroke}%
\begin{itemdecl}
basic_brush_props<GraphicsSurfaces> brush_props() const noexcept;
\end{itemdecl}
\begin{itemdescr}
\pnum
\returns \tcode{GraphicsSurfaces::surfaces::brush_props(data())}.

\pnum
\remarks The returned value is the brush props.
\end{itemdescr}

\indexlibrarymember{stroke_props}{stroke}%
\begin{itemdecl}
basic_stroke_props<GraphicsSurfaces> stroke_props() const noexcept;
\end{itemdecl}
\begin{itemdescr}
\pnum
\returns \tcode{GraphicsSurfaces::surfaces::stroke_props(data())}.

\pnum
\remarks The returned value is the stroke props.
\end{itemdescr}

\indexlibrarymember{dashes}{stroke}%
\begin{itemdecl}
basic_dashes<GraphicsSurfaces> dashes() const noexcept;
\end{itemdecl}
\begin{itemdescr}
\pnum
\returns \tcode{GraphicsSurfaces::surfaces::dashes(data())}.

\pnum
\remarks The returned value is the dashes.
\end{itemdescr}

\indexlibrarymember{render_props}{stroke}%
\begin{itemdecl}
basic_render_props<GraphicsSurfaces> render_props() const noexcept;
\end{itemdecl}
\begin{itemdescr}
\pnum
\returns \tcode{GraphicsSurfaces::surfaces::render_props(data())}.

\pnum
\remarks The returned value is the render props.
\end{itemdescr}

\indexlibrarymember{clip_props}{stroke}%
\begin{itemdecl}
basic_clip_props<GraphicsSurfaces> clip_props() const noexcept;
\end{itemdecl}
\begin{itemdescr}
\pnum
\returns \tcode{GraphicsSurfaces::surfaces::clip_props(data())}.

\pnum
\remarks The returned value is the clip props.
\end{itemdescr}

\rSec1 [\iotwod.cmdlists.stroke.eq] {Equality operators}%

\indexlibrarymember{operator==}{stroke}%
\begin{itemdecl}
template <class GraphicsSurfaces>
bool operator==(
  const typename basic_commands<GraphicsSurfaces::stroke& lhs,
  const typename basic_commands<GraphicsSurfaces::stroke& rhs) 
  noexcept;
\end{itemdecl}
\begin{itemdescr}
\pnum
\returns \tcode{GraphicsSurfaces::surfaces::equal(lhs.data(), rhs.data())}.
\end{itemdescr}

\indexlibrarymember{operator!=}{stroke}%
\begin{itemdecl}
template <class GraphicsSurfaces>
bool operator!=(
  const typename basic_commands<GraphicsSurfaces::stroke& lhs,
  const typename basic_commands<GraphicsSurfaces::stroke& rhs) 
  noexcept;
\end{itemdecl}
\begin{itemdescr}
\pnum
\returns \tcode{GraphicsSurfaces::surfaces::not_equal(lhs.data(), rhs.data())}.
\end{itemdescr}

%!TEX root = io2d.tex

\rSec0 [\iotwod.cmdlists.commands.fill] {Class template \tcode{basic_commands<GraphicsSurfaces>::fill}}

\rSec1 [\iotwod.cmdlists.fill.intro] {Overview}

\pnum
\indexlibrary{\idxcode{fill}}%
The class template \tcode{basic_commands<GraphicsSurfaces>::fill} describes a command that invokes the \tcode{fill} member function of a surface.

\pnum
It has an \term{optional surface} of type \tcode{optional<reference_wrapper<basic_image_surface<GraphicsSurfaces>>>}. If optional surface has a value, the fill operation is performed on the optional surface instead of the surface that the command is submitted to.

\pnum
If optional surface has a value and the referenced \tcode{basic_image_surface<GraphicsSurfaces>} object has been destroyed or otherwise rendered invalid when a \tcode{basic_command_list<GraphicsSurfaces>} object built using this \tcode{paint} object is used by the program, the effects are undefined.

\pnum
It has a \term{brush} of type \tcode{basic_brush<GraphicsSurfaces>}.

\pnum
It has a \term{path} of type \tcode{basic_interpreted_path<GraphicsSurfaces>}.

\pnum
It has a \term{brush props} of type \tcode{basic_brush_props<GraphicsSurfaces>}.

\pnum
It has a \term{render props} of type \tcode{basic_render_props<GraphicsSurfaces>}.

\pnum
It has a \term{clip props} of type \tcode{basic_clip_props}.

\pnum
The data are stored in an object of type \tcode{typename GraphicsSurfaces::surfaces::fill_data_type}. It is accessible using the \tcode{data} member functions.

\pnum
The data are used as arguments for the invocation of the \tcode{fill} member function of the appropriate surface when a \tcode{basic_command_list<GraphicsSurfaces>} object built using this \tcode{fill} object is used by the program.

\rSec1 [\iotwod.cmdlists.fill.synopsis] {Synopsis}
\begin{codeblock}
namespace @\fullnamespace{}@ {
  template <class GraphicsSurfaces>
  class basic_commands<GraphicsSurfaces::fill {
  public:
    using graphics_math_type = typename GraphicsSurfaces::graphics_math_type;
    using data_type = typename GraphicsSurfaces::surfaces::fill_data_type;

    // \ref{\iotwod.cmdlists.fill.ctor}, construct:
    fill(const basic_brush<GraphicsSurfaces>& b,
      const basic_interpreted_path<GraphicsSurfaces>& ip,
      const basic_brush_props<GraphicsSurfaces>& bp = 
      basic_brush_props<GraphicsSurfaces>{},
      const basic_render_props<GraphicsSurfaces>& rp = 
      basic_render_props<GraphicsSurfaces>{},
      const basic_clip_props<GraphicsSurfaces>& cl = 
      basic_clip_props<GraphicsSurfaces>{}) noexcept;
    fill(reference_wrapper<basic_image_surface<GraphicsSurfaces>> sfc,
      const basic_brush<GraphicsSurfaces>& b,
      const basic_interpreted_path<GraphicsSurfaces>& ip,
      const basic_brush_props<GraphicsSurfaces>& bp = 
      basic_brush_props<GraphicsSurfaces>{},
      const basic_render_props<GraphicsSurfaces>& rp = 
      basic_render_props<GraphicsSurfaces>{},
      const basic_clip_props<GraphicsSurfaces>& cl = 
      basic_clip_props<GraphicsSurfaces>{}) noexcept;
    
    // \ref{\iotwod.cmdlists.fill.acc}, accessors:
    const data_type& data() const noexcept;
    data_type& data() noexcept;

    // \ref{\iotwod.cmdlists.fill.mod}, modifiers:
    void surface(
      optional<reference_wrapper<basic_image_surface<GraphicsSurfaces>>> sfc) 
      noexcept;
    void brush(const basic_brush<GraphicsSurfaces>& b) noexcept;
    void path(const basic_interpreted_path<GraphicsSurfaces>& p) noexcept;
    void brush_props(const basic_brush_props<GraphicsSurfaces>& bp) noexcept;
    void render_props(const basic_render_props<GraphicsSurfaces>& rp) noexcept;
    void clip_props(const basic_clip_props<GraphicsSurfaces>& cl) noexcept;

    // \ref{\iotwod.cmdlists.fill.obs}, observers:
    optional<reference_wrapper<basic_image_surface<GraphicsSurfaces>>> 
      surface() const noexcept;
    basic_brush<GraphicsSurfaces> brush() const noexcept;
    basic_interpreted_path<GraphicsSurfaces> path() const noexcept;
    basic_brush_props<GraphicsSurfaces> brush_props() const noexcept;
    basic_render_props<GraphicsSurfaces> render_props() const noexcept;
    basic_clip_props<GraphicsSurfaces> clip_props() const noexcept;
  };

  // \ref{\iotwod.cmdlists.fill.eq}, equality operators:
  template <class GraphicsSurfaces>
  bool operator==(
    const typename basic_commands<GraphicsSurfaces::fill& lhs,
    const typename basic_commands<GraphicsSurfaces::fill& rhs) 
    noexcept;
  template <class GraphicsSurfaces>
  bool operator!=(
    const typename basic_commands<GraphicsSurfaces::fill& lhs,
    const typename basic_commands<GraphicsSurfaces::fill& rhs) 
    noexcept;
}
\end{codeblock}

\rSec1 [\iotwod.cmdlists.fill.ctor] {Constructors}%

\indexlibrary{\idxcode{fill}!constructor}%
\begin{itemdecl}
fill(const basic_brush<GraphicsSurfaces>& b,
  const basic_interpreted_path<GraphicsSurfaces>& ip,
  const basic_brush_props<GraphicsSurfaces>& bp = 
  basic_brush_props<GraphicsSurfaces>{},
  const basic_render_props<GraphicsSurfaces>& rp = 
  basic_render_props<GraphicsSurfaces>{},
  const basic_clip_props<GraphicsSurfaces>& cl = 
  basic_clip_props<GraphicsSurfaces>{}) noexcept;
\end{itemdecl}
\begin{itemdescr}
\pnum
\effects Constructs an object of type \tcode{fill}.

\pnum
\postconditions \tcode{data() == GraphicsSurfaces::surfaces::create_fill(b, ip, bp, rp, cl)}.
\end{itemdescr}

\indexlibrary{\idxcode{fill}!constructor}%
\begin{itemdecl}
fill(reference_wrapper<basic_image_surface<GraphicsSurfaces>> sfc,
  const basic_brush<GraphicsSurfaces>& b,
  const basic_interpreted_path<GraphicsSurfaces>& ip,
  const basic_brush_props<GraphicsSurfaces>& bp = 
  basic_brush_props<GraphicsSurfaces>{},
  const basic_render_props<GraphicsSurfaces>& rp = 
  basic_render_props<GraphicsSurfaces>{},
  const basic_clip_props<GraphicsSurfaces>& cl = 
  basic_clip_props<GraphicsSurfaces>{}) noexcept;
\end{itemdecl}
\begin{itemdescr}
\pnum
\effects Constructs an object of type \tcode{fill}.

\pnum
\postconditions \tcode{data() == GraphicsSurfaces::surfaces::create_fill(sfc, b, ip, bp, rp, cl)}.
\end{itemdescr}

\rSec1 [\iotwod.cmdlists.fill.acc] {Accessors}%

\indexlibrarymember{data}{fill}%
\begin{itemdecl}
const data_type& data() const noexcept;
data_type& data() noexcept;
\end{itemdecl}
\begin{itemdescr}
\pnum
\returns A reference to the \tcode{fill} object's data object (See: \ref{\iotwod.cmdlists.fill.intro}).

\pnum
\remarks The behavior of a program is undefined if the user modifies the data contained in the \tcode{data_type} object returned by this function.
\end{itemdescr}

\rSec1 [\iotwod.cmdlists.fill.mod] {Modifiers}%

\indexlibrarymember{surface}{fill}%
\begin{itemdecl}
void surface(
  optional<reference_wrapper<basic_image_surface<GraphicsSurfaces>>> sfc) 
  noexcept;
\end{itemdecl}
\begin{itemdescr}
\pnum
\effects Calls \tcode{GraphicsSurfaces::surfaces::surface(data(), sfc)}.

\pnum
\remarks The optional surface is \tcode{sfc}.
\end{itemdescr}

\indexlibrarymember{brush}{fill}%
\begin{itemdecl}
void brush(const basic_brush<GraphicsSurfaces>& b) noexcept;
\end{itemdecl}
\begin{itemdescr}
\pnum
\effects Calls \tcode{GraphicsSurfaces::surfaces::brush(data(), b)}.

\pnum
\remarks The brush is \tcode{b}.
\end{itemdescr}

\indexlibrarymember{path}{fill}%
\begin{itemdecl}
void path(const basic_interpreted_path<GraphicsSurfaces>& p) noexcept;
\end{itemdecl}
\begin{itemdescr}
\pnum
\effects Calls \tcode{GraphicsSurfaces::surfaces::path(data(), p)}.

\pnum
\remarks The path is \tcode{p}.
\end{itemdescr}

\indexlibrarymember{brush_props}{fill}%
\begin{itemdecl}
void brush_props(const basic_brush_props<GraphicsSurfaces>& bp) noexcept;
\end{itemdecl}
\begin{itemdescr}
\pnum
\effects Calls \tcode{GraphicsSurfaces::surfaces::brush_props(data(), bp)}.

\pnum
\remarks The brush props is \tcode{bp}.
\end{itemdescr}

\indexlibrarymember{render_props}{fill}%
\begin{itemdecl}
void render_props(const basic_render_props<GraphicsSurfaces>& rp) noexcept;
\end{itemdecl}
\begin{itemdescr}
\pnum
\effects Calls \tcode{GraphicsSurfaces::surfaces::render_props(data(), rp)}.

\pnum
\remarks The render props is \tcode{rp}.
\end{itemdescr}

\indexlibrarymember{clip_props}{fill}%
\begin{itemdecl}
void clip_props(const basic_clip_props<GraphicsSurfaces>& cl) noexcept;
\end{itemdecl}
\begin{itemdescr}
\pnum
\effects Calls \tcode{GraphicsSurfaces::surfaces::clip_props(data(), cl)}.

\pnum
\remarks The clip props is \tcode{cl}.
\end{itemdescr}

\rSec1 [\iotwod.cmdlists.fill.obs] {Observers}%

\indexlibrarymember{surface}{fill}%
\begin{itemdecl}
optional<reference_wrapper<basic_image_surface<GraphicsSurfaces>>> 
  surface() const noexcept;
\end{itemdecl}
\begin{itemdescr}
\pnum
\returns \tcode{GraphicsSurfaces::surfaces::surface(data())}.

\pnum
\remarks
The returned value is the optional surface.
\end{itemdescr}

\indexlibrarymember{brush}{fill}%
\begin{itemdecl}
basic_brush<GraphicsSurfaces> brush() const noexcept;
\end{itemdecl}
\begin{itemdescr}
\pnum
\returns \tcode{GraphicsSurfaces::surfaces::brush(data())}.

\pnum
\remarks The returned value is the brush.
\end{itemdescr}

\indexlibrarymember{path}{fill}%
\begin{itemdecl}
basic_interpreted_path<GraphicsSurfaces> path() const noexcept;
\end{itemdecl}
\begin{itemdescr}
\pnum
\returns \tcode{GraphicsSurfaces::surfaces::path(data())}.

\pnum
\remarks The returned value is the path.
\end{itemdescr}

\indexlibrarymember{brush_props}{fill}%
\begin{itemdecl}
basic_brush_props<GraphicsSurfaces> brush_props() const noexcept;
\end{itemdecl}
\begin{itemdescr}
\pnum
\returns \tcode{GraphicsSurfaces::surfaces::brush_props(data())}.

\pnum
\remarks The returned value is the brush props.
\end{itemdescr}

\indexlibrarymember{render_props}{fill}%
\begin{itemdecl}
basic_render_props<GraphicsSurfaces> render_props() const noexcept;
\end{itemdecl}
\begin{itemdescr}
\pnum
\returns \tcode{GraphicsSurfaces::surfaces::render_props(data())}.

\pnum
\remarks The returned value is the render props.
\end{itemdescr}

\indexlibrarymember{clip_props}{fill}%
\begin{itemdecl}
basic_clip_props<GraphicsSurfaces> clip_props() const noexcept;
\end{itemdecl}
\begin{itemdescr}
\pnum
\returns \tcode{GraphicsSurfaces::surfaces::clip_props(data())}.

\pnum
\remarks The returned value is the clip props.
\end{itemdescr}

\rSec1 [\iotwod.cmdlists.fill.eq] {Equality operators}%

\indexlibrarymember{operator==}{fill}%
\begin{itemdecl}
template <class GraphicsSurfaces>
bool operator==(
  const typename basic_commands<GraphicsSurfaces::fill& lhs,
  const typename basic_commands<GraphicsSurfaces::fill& rhs) 
  noexcept;
\end{itemdecl}
\begin{itemdescr}
\pnum
\returns \tcode{GraphicsSurfaces::surfaces::equal(lhs.data(), rhs.data())}.
\end{itemdescr}

\indexlibrarymember{operator!=}{fill}%
\begin{itemdecl}
template <class GraphicsSurfaces>
bool operator!=(
  const typename basic_commands<GraphicsSurfaces::fill& lhs,
  const typename basic_commands<GraphicsSurfaces::fill& rhs) 
  noexcept;
\end{itemdecl}
\begin{itemdescr}
\pnum
\returns \tcode{GraphicsSurfaces::surfaces::not_equal(lhs.data(), rhs.data())}.
\end{itemdescr}

%!TEX root = io2d.tex

\rSec0 [\iotwod.cmdlists.commands.mask] {Class template \tcode{basic_commands<GraphicsSurfaces>::mask}}

\rSec1 [\iotwod.cmdlists.mask.intro] {Overview}

\pnum
\indexlibrary{\idxcode{mask}}%
The class template \tcode{basic_commands<GraphicsSurfaces>::mask} describes a command that invokes the \tcode{mask} member function of a surface.

\pnum
It has an \term{optional surface} of type \tcode{optional<reference_wrapper<basic_image_surface<GraphicsSurfaces>>>}. If optional surface has a value, the mask operation is performed on the optional surface instead of the surface that the command is submitted to.

\pnum
If optional surface has a value and the referenced \tcode{basic_image_surface<GraphicsSurfaces>} object has been destroyed or otherwise rendered invalid when a \tcode{basic_command_list<GraphicsSurfaces>} object built using this \tcode{paint} object is used by the program, the effects are undefined.

\pnum
It has a \term{brush} of type \tcode{basic_brush<GraphicsSurfaces>}.

\pnum
It has a \term{mask brush} of type \tcode{basic_brush<GraphicsSurfaces>}.

\pnum
It has a \term{brush props} of type \tcode{basic_brush_props<GraphicsSurfaces>}.

\pnum
It has a \term{mask props} of type \tcode{basic_mask_props<GraphicsSurfaces>}.

\pnum
It has a \term{render props} of type \tcode{basic_render_props<GraphicsSurfaces>}.

\pnum
It has a \term{clip props} of type \tcode{basic_clip_props}.

\pnum
The data are stored in an object of type \tcode{typename GraphicsSurfaces::surfaces::mask_data_type}. It is accessible using the \tcode{data} member functions.

\pnum
The data are used as arguments for the invocation of the \tcode{mask} member function of the appropriate surface when a \tcode{basic_command_list<GraphicsSurfaces>} object built using this \tcode{mask} object is used by the program.

\rSec1 [\iotwod.cmdlists.mask.synopsis] {Synopsis}
\begin{codeblock}
namespace @\fullnamespace{}@ {
  template <class GraphicsSurfaces>
  class basic_commands<GraphicsSurfaces::mask {
  public:
    using graphics_math_type = typename GraphicsSurfaces::graphics_math_type;
    using data_type = typename GraphicsSurfaces::surfaces::mask_data_type;

    // \ref{\iotwod.cmdlists.mask.ctor}, construct:
    mask(const basic_brush<GraphicsSurfaces>& b,
      const basic_brush<GraphicsSurfaces>& mb,
      const basic_brush_props<GraphicsSurfaces>& bp = 
      basic_brush_props<GraphicsSurfaces>{},
      const basic_mask_props<GraphicsSurfaces>& mp = 
      basic_mask_props<GraphicsSurfaces>{},
      const basic_render_props<GraphicsSurfaces>& rp = 
      basic_mask_props<GraphicsSurfaces>{},
      const basic_clip_props<GraphicsSurfaces>& cl = 
      basic_clip_props<GraphicsSurfaces>{}) noexcept;
    mask(reference_wrapper<basic_image_surface<GraphicsSurfaces>> sfc,
      const basic_brush<GraphicsSurfaces>& b,
      const basic_brush<GraphicsSurfaces>& mb,
      const basic_brush_props<GraphicsSurfaces>& bp = 
      basic_brush_props<GraphicsSurfaces>{},
      const basic_mask_props<GraphicsSurfaces>& mp = 
      basic_mask_props<GraphicsSurfaces>{},
      const basic_render_props<GraphicsSurfaces>& rp = 
      basic_mask_props<GraphicsSurfaces>{},
      const basic_clip_props<GraphicsSurfaces>& cl = 
      basic_clip_props<GraphicsSurfaces>{}) noexcept;
    
    // \ref{\iotwod.cmdlists.mask.acc}, accessors:
    const data_type& data() const noexcept;
    data_type& data() noexcept;

    // \ref{\iotwod.cmdlists.mask.mod}, modifiers:
    void surface(
      optional<reference_wrapper<basic_image_surface<GraphicsSurfaces>>> sfc) 
      noexcept;
    void brush(const basic_brush<GraphicsSurfaces>& b) noexcept;
    void mask_brush(const basic_brush<GraphicsSurfaces>& mb) noexcept;
    void brush_props(const basic_brush_props<GraphicsSurfaces>& bp) noexcept;
    void mask_props(const basic_mask_props<GraphicsSurfaces>& mp) noexcept;
    void render_props(const basic_render_props<GraphicsSurfaces>& rp) noexcept;
    void clip_props(const basic_clip_props<GraphicsSurfaces>& cl) noexcept;

    // \ref{\iotwod.cmdlists.mask.obs}, observers:
    optional<reference_wrapper<basic_image_surface<GraphicsSurfaces>>> 
      surface() const noexcept;
    basic_brush<GraphicsSurfaces> brush() const noexcept;
    basic_brush<GraphicsSurfaces> mask_brush() const noexcept;
    basic_brush_props<GraphicsSurfaces> brush_props() const noexcept;
    basic_mask_props<GraphicsSurfaces> mask_props() const noexcept;
    basic_render_props<GraphicsSurfaces> render_props() const noexcept;
    basic_clip_props<GraphicsSurfaces> clip_props() const noexcept;
  };

  // \ref{\iotwod.cmdlists.mask.eq}, equality operators:
  template <class GraphicsSurfaces>
  bool operator==(
    const typename basic_commands<GraphicsSurfaces::mask& lhs,
    const typename basic_commands<GraphicsSurfaces::mask& rhs) 
    noexcept;
  template <class GraphicsSurfaces>
  bool operator!=(
    const typename basic_commands<GraphicsSurfaces::mask& lhs,
    const typename basic_commands<GraphicsSurfaces::mask& rhs) 
    noexcept;
}
\end{codeblock}

\rSec1 [\iotwod.cmdlists.mask.ctor] {Constructors}%

\indexlibrary{\idxcode{mask}!constructor}%
\begin{itemdecl}
mask(const basic_brush<GraphicsSurfaces>& b,
  const basic_brush<GraphicsSurfaces>& mb,
  const basic_brush_props<GraphicsSurfaces>& bp = 
  basic_brush_props<GraphicsSurfaces>{},
  const basic_mask_props<GraphicsSurfaces>& mp = 
  basic_mask_props<GraphicsSurfaces>{},
  const basic_render_props<GraphicsSurfaces>& rp = 
  basic_mask_props<GraphicsSurfaces>{},
  const basic_clip_props<GraphicsSurfaces>& cl = 
  basic_clip_props<GraphicsSurfaces>{}) noexcept;
\end{itemdecl}
\begin{itemdescr}
\pnum
\effects Constructs an object of type \tcode{mask}.

\pnum
\postconditions \tcode{data() == GraphicsSurfaces::surfaces::create_mask(b, mb, bp, mp, rp, cl)}.
\end{itemdescr}

\indexlibrary{\idxcode{mask}!constructor}%
\begin{itemdecl}
mask(reference_wrapper<basic_image_surface<GraphicsSurfaces>> sfc,
  const basic_brush<GraphicsSurfaces>& b,
  const basic_brush<GraphicsSurfaces>& mb,
  const basic_brush_props<GraphicsSurfaces>& bp = 
  basic_brush_props<GraphicsSurfaces>{},
  const basic_mask_props<GraphicsSurfaces>& mp = 
  basic_mask_props<GraphicsSurfaces>{},
  const basic_render_props<GraphicsSurfaces>& rp = 
  basic_mask_props<GraphicsSurfaces>{},
  const basic_clip_props<GraphicsSurfaces>& cl = 
  basic_clip_props<GraphicsSurfaces>{}) noexcept;
\end{itemdecl}
\begin{itemdescr}
\pnum
\effects Constructs an object of type \tcode{mask}.

\pnum
\postconditions \tcode{data() == GraphicsSurfaces::surfaces::create_mask(sfc, b, mb, bp, mp, rp, cl)}.
\end{itemdescr}

\rSec1 [\iotwod.cmdlists.mask.acc] {Accessors}%

\indexlibrarymember{data}{mask}%
\begin{itemdecl}
const data_type& data() const noexcept;
data_type& data() noexcept;
\end{itemdecl}
\begin{itemdescr}
\pnum
\returns A reference to the \tcode{mask} object's data object (See: \ref{\iotwod.cmdlists.mask.intro}).

\pnum
\remarks The behavior of a program is undefined if the user modifies the data contained in the \tcode{data_type} object returned by this function.
\end{itemdescr}

\rSec1 [\iotwod.cmdlists.mask.mod] {Modifiers}%

\indexlibrarymember{surface}{mask}%
\begin{itemdecl}
void surface(
  optional<reference_wrapper<basic_image_surface<GraphicsSurfaces>>> sfc) 
  noexcept;
\end{itemdecl}
\begin{itemdescr}
\pnum
\effects Calls \tcode{GraphicsSurfaces::surfaces::surface(data(), sfc)}.

\pnum
\remarks The optional surface is \tcode{sfc}.
\end{itemdescr}

\indexlibrarymember{brush}{mask}%
\begin{itemdecl}
void brush(const basic_brush<GraphicsSurfaces>& b) noexcept;
\end{itemdecl}
\begin{itemdescr}
\pnum
\effects Calls \tcode{GraphicsSurfaces::surfaces::brush(data(), b)}.

\pnum
\remarks The brush is \tcode{b}.
\end{itemdescr}

\indexlibrarymember{mask_brush}{mask}%
\begin{itemdecl}
void path(const basic_brush<GraphicsSurfaces>& mb) noexcept;
\end{itemdecl}
\begin{itemdescr}
\pnum
\effects Calls \tcode{GraphicsSurfaces::surfaces::mask_brush(data(), mb)}.

\pnum
\remarks The mask brush is \tcode{mb}.
\end{itemdescr}

\indexlibrarymember{brush_props}{mask}%
\begin{itemdecl}
void brush_props(const basic_brush_props<GraphicsSurfaces>& bp) noexcept;
\end{itemdecl}
\begin{itemdescr}
\pnum
\effects Calls \tcode{GraphicsSurfaces::surfaces::brush_props(data(), bp)}.

\pnum
\remarks The brush props is \tcode{bp}.
\end{itemdescr}

\indexlibrarymember{mask_props}{mask}%
\begin{itemdecl}
void mask_props(const basic_mask_props<GraphicsSurfaces>& bp) noexcept;
\end{itemdecl}
\begin{itemdescr}
\pnum
\effects Calls \tcode{GraphicsSurfaces::surfaces::mask_props(data(), bp)}.

\pnum
\remarks The mask props is \tcode{bp}.
\end{itemdescr}

\indexlibrarymember{render_props}{mask}%
\begin{itemdecl}
void render_props(const basic_render_props<GraphicsSurfaces>& rp) noexcept;
\end{itemdecl}
\begin{itemdescr}
\pnum
\effects Calls \tcode{GraphicsSurfaces::surfaces::render_props(data(), rp)}.

\pnum
\remarks The render props is \tcode{rp}.
\end{itemdescr}

\indexlibrarymember{clip_props}{mask}%
\begin{itemdecl}
void clip_props(const basic_clip_props<GraphicsSurfaces>& cl) noexcept;
\end{itemdecl}
\begin{itemdescr}
\pnum
\effects Calls \tcode{GraphicsSurfaces::surfaces::clip_props(data(), cl)}.

\pnum
\remarks The clip props is \tcode{cl}.
\end{itemdescr}

\rSec1 [\iotwod.cmdlists.mask.obs] {Observers}%

\indexlibrarymember{surface}{mask}%
\begin{itemdecl}
optional<reference_wrapper<basic_image_surface<GraphicsSurfaces>>> 
  surface() const noexcept;
\end{itemdecl}
\begin{itemdescr}
\pnum
\returns \tcode{GraphicsSurfaces::surfaces::surface(data())}.

\pnum
\remarks
The returned value is the optional surface.
\end{itemdescr}

\indexlibrarymember{brush}{mask}%
\begin{itemdecl}
basic_brush<GraphicsSurfaces> brush() const noexcept;
\end{itemdecl}
\begin{itemdescr}
\pnum
\returns \tcode{GraphicsSurfaces::surfaces::brush(data())}.

\pnum
\remarks The returned value is the brush.
\end{itemdescr}

\indexlibrarymember{mask_brush}{mask}%
\begin{itemdecl}
basic_brush<GraphicsSurfaces> mask_brush() const noexcept;
\end{itemdecl}
\begin{itemdescr}
\pnum
\returns \tcode{GraphicsSurfaces::surfaces::mask_brush(data())}.

\pnum
\remarks The returned value is the mask brush.
\end{itemdescr}

\indexlibrarymember{brush_props}{mask}%
\begin{itemdecl}
basic_brush_props<GraphicsSurfaces> brush_props() const noexcept;
\end{itemdecl}
\begin{itemdescr}
\pnum
\returns \tcode{GraphicsSurfaces::surfaces::brush_props(data())}.

\pnum
\remarks The returned value is the brush props.
\end{itemdescr}

\indexlibrarymember{mask_props}{mask}%
\begin{itemdecl}
basic_mask_props<GraphicsSurfaces> mask_props() const noexcept;
\end{itemdecl}
\begin{itemdescr}
\pnum
\returns \tcode{GraphicsSurfaces::surfaces::mask_props(data())}.

\pnum
\remarks The returned value is the mask props.
\end{itemdescr}

\indexlibrarymember{render_props}{mask}%
\begin{itemdecl}
basic_render_props<GraphicsSurfaces> render_props() const noexcept;
\end{itemdecl}
\begin{itemdescr}
\pnum
\returns \tcode{GraphicsSurfaces::surfaces::render_props(data())}.

\pnum
\remarks The returned value is the render props.
\end{itemdescr}

\indexlibrarymember{clip_props}{mask}%
\begin{itemdecl}
basic_clip_props<GraphicsSurfaces> clip_props() const noexcept;
\end{itemdecl}
\begin{itemdescr}
\pnum
\returns \tcode{GraphicsSurfaces::surfaces::clip_props(data())}.

\pnum
\remarks The returned value is the clip props.
\end{itemdescr}

\rSec1 [\iotwod.cmdlists.mask.eq] {Equality operators}%

\indexlibrarymember{operator==}{mask}%
\begin{itemdecl}
template <class GraphicsSurfaces>
bool operator==(
  const typename basic_commands<GraphicsSurfaces::mask& lhs,
  const typename basic_commands<GraphicsSurfaces::mask& rhs) 
  noexcept;
\end{itemdecl}
\begin{itemdescr}
\pnum
\returns \tcode{GraphicsSurfaces::surfaces::equal(lhs.data(), rhs.data())}.
\end{itemdescr}

\indexlibrarymember{operator!=}{mask}%
\begin{itemdecl}
template <class GraphicsSurfaces>
bool operator!=(
  const typename basic_commands<GraphicsSurfaces::mask& lhs,
  const typename basic_commands<GraphicsSurfaces::mask& rhs) 
  noexcept;
\end{itemdecl}
\begin{itemdescr}
\pnum
\returns \tcode{GraphicsSurfaces::surfaces::not_equal(lhs.data(), rhs.data())}.
\end{itemdescr}

%!TEX root = io2d.tex

\rSec0 [\iotwod.cmdlists.commands.drawtext] {Class template \tcode{basic_commands<GraphicsSurfaces>::draw_text}}

\rSec1 [\iotwod.cmdlists.drawtext.intro] {Overview}

\pnum
\indexlibrary{\idxcode{draw_text}}%
The class template \tcode{basic_commands<GraphicsSurfaces>::draw_text} describes a command that invokes the \tcode{draw_text} member function of a surface.

\pnum
It has an \term{optional surface} of type \tcode{optional<reference_wrapper<basic_image_surface<GraphicsSurfaces>>>}. If optional surface has a value, the draw_text operation is performed on the optional surface instead of the surface that the command is submitted to.

\pnum
If optional surface has a value and the referenced \tcode{basic_image_surface<GraphicsSurfaces>} object has been destroyed or otherwise rendered invalid when a \tcode{basic_command_list<GraphicsSurfaces>} object built using this \tcode{paint} object is used by the program, the effects are undefined.

\pnum
It has a \term{text location} of type \tcode{variant<basic_point_2d<typename GraphicsSurfaces::graphics_math_type>, basic_bounding_box<typename GraphicsSurfaces::graphics_math_type>>}.

\pnum
It has a \term{brush} of type \tcode{basic_brush<GraphicsSurfaces>}.

\pnum
It has a \term{font} of type \tcode{basic_font<GraphicsSurfaces>}.

\pnum
It has \term{text} of type \tcode{string} comprised of UTF-8 encoded character data.

\pnum
It has a \term{text props} of type \tcode{basic_text_props<GraphicsSurfaces>}.

\pnum
It has a \term{brush props} of type \tcode{basic_brush_props<GraphicsSurfaces>}.

\pnum
It has a \term{stroke props} of type \tcode{basic_stroke_props<GraphicsSurfaces>}.

\pnum
It has a \term{dashes} of type \tcode{basic_dashes<GraphicsSurfaces>}.

\pnum
It has a \term{render props} of type \tcode{basic_render_props<GraphicsSurfaces>}.

\pnum
It has a \term{clip props} of type \tcode{basic_clip_props}.

\pnum
The data are stored in an object of type \tcode{typename GraphicsSurfaces::surfaces::stroke_data_type}. It is accessible using the \tcode{data} member functions.

\pnum
The data are used as arguments for the invocation of the \tcode{draw_text} member function of the appropriate surface when a \tcode{basic_command_list<GraphicsSurfaces>} object built using this \tcode{draw_text} object is used by the program.

\rSec1 [\iotwod.cmdlists.drawtext.synopsis] {Synopsis}
\begin{codeblock}
namespace @\fullnamespace{}@ {
  template <class GraphicsSurfaces>
  class basic_commands<GraphicsSurfaces::draw_text {
  public:
    using graphics_math_type = typename GraphicsSurfaces::graphics_math_type;
    using data_type = typename GraphicsSurfaces::surfaces::draw_text_data_type;

    // \ref{\iotwod.cmdlists.drawtext.ctor}, construct:
    draw_text(const basic_point_2d<graphics_math_type>& pt,
      const basic_brush<GraphicsSurfaces>& b,
      const basic_font<GraphicsSurfaces>& font, string t,
      const basic_text_props<GraphicsSurfaces>& tp = 
      basic_text_props<GraphicsSurfaces>{},
      const basic_brush_props<GraphicsSurfaces>& bp = 
      basic_brush_props<GraphicsSurfaces>{},
      const basic_stroke_props<GraphicsSurfaces>& sp = 
      basic_stroke_props<GraphicsSurfaces>{},
      const basic_dashes<GraphicsSurfaces>& d =
      basic_dashes<GraphicsSurfaces>{},
      const basic_render_props<GraphicsSurfaces>& rp = 
      basic_render_props<GraphicsSurfaces>{},
      const basic_clip_props<GraphicsSurfaces>& cl = 
      basic_clip_props<GraphicsSurfaces>{}) noexcept;
    draw_text(reference_wrapper<basic_image_surface<GraphicsSurfaces>> sfc,
      const basic_point_2d<graphics_math_type>& pt,
      const basic_brush<GraphicsSurfaces>& b,
      const basic_font<GraphicsSurfaces>& font, string t,
      const basic_text_props<GraphicsSurfaces>& tp = 
      basic_text_props<GraphicsSurfaces>{},
      const basic_brush_props<GraphicsSurfaces>& bp = 
      basic_brush_props<GraphicsSurfaces>{},
      const basic_stroke_props<GraphicsSurfaces>& sp = 
      basic_stroke_props<GraphicsSurfaces>{},
      const basic_dashes<GraphicsSurfaces>& d =
      basic_dashes<GraphicsSurfaces>{},
      const basic_render_props<GraphicsSurfaces>& rp = 
      basic_render_props<GraphicsSurfaces>{},
      const basic_clip_props<GraphicsSurfaces>& cl = 
      basic_clip_props<GraphicsSurfaces>{}) noexcept;
    draw_text(const basic_bounding_box<graphics_math_type>& bb,
      const basic_brush<GraphicsSurfaces>& b,
      const basic_font<GraphicsSurfaces>& font, string t,
      const basic_text_props<GraphicsSurfaces>& tp = 
      basic_text_props<GraphicsSurfaces>{},
      const basic_brush_props<GraphicsSurfaces>& bp = 
      basic_brush_props<GraphicsSurfaces>{},
      const basic_stroke_props<GraphicsSurfaces>& sp = 
      basic_stroke_props<GraphicsSurfaces>{},
      const basic_dashes<GraphicsSurfaces>& d = 
      basic_dashes<GraphicsSurfaces>{},
      const basic_render_props<GraphicsSurfaces>& rp = 
      basic_render_props<GraphicsSurfaces>{},
      const basic_clip_props<GraphicsSurfaces>& cl = 
      basic_clip_props<GraphicsSurfaces>{}) noexcept;
    draw_text(reference_wrapper<basic_image_surface<GraphicsSurfaces>> sfc, 
      const basic_bounding_box<graphics_math_type>& bb,
      const basic_brush<GraphicsSurfaces>& b,
      const basic_font<GraphicsSurfaces>& font, string t,
      const basic_text_props<GraphicsSurfaces>& tp = 
      basic_text_props<GraphicsSurfaces>{},
      const basic_brush_props<GraphicsSurfaces>& bp = 
      basic_brush_props<GraphicsSurfaces>{},
      const basic_stroke_props<GraphicsSurfaces>& sp = 
      basic_stroke_props<GraphicsSurfaces>{},
      const basic_dashes<GraphicsSurfaces>& d = 
      basic_dashes<GraphicsSurfaces>{},
      const basic_render_props<GraphicsSurfaces>& rp = 
      basic_render_props<GraphicsSurfaces>{},
      const basic_clip_props<GraphicsSurfaces>& cl = 
      basic_clip_props<GraphicsSurfaces>{}) noexcept;
    
    // \ref{\iotwod.cmdlists.drawtext.acc}, accessors:
    const data_type& data() const noexcept;
    data_type& data() noexcept;

    // \ref{\iotwod.cmdlists.drawtext.mod}, modifiers:
    void surface(
      optional<reference_wrapper<basic_image_surface<GraphicsSurfaces>>> sfc) 
      noexcept;
    void location(const basic_point_2d<graphics_math_type>& pt) noexcept;
    void location(const basic_bounding_box<graphics_math_type>& bb) noexcept;
    void brush(const basic_brush<GraphicsSurfaces>& b) noexcept;
    void font(const basic_font<GraphicsSurfaces>& f) noexcept;
    void text(string t) noexcept;
    void text_props(const basic_text_props<GraphicsSurfaces>& tp) noexcept;
    void brush_props(const basic_brush_props<GraphicsSurfaces>& bp) noexcept;
    void stroke_props(const basic_stroke_props<GraphicsSurfaces>& sp) noexcept;
    void dashes(const basic_dashes<GraphicsSurfaces>& d) noexcept;
    void render_props(const basic_render_props<GraphicsSurfaces>& rp) noexcept;
    void clip_props(const basic_clip_props<GraphicsSurfaces>& cl) noexcept;

    // \ref{\iotwod.cmdlists.drawtext.obs}, observers:
    optional<reference_wrapper<basic_image_surface<GraphicsSurfaces>>> 
      surface() const noexcept;
    variant<basic_point_2d<graphics_math_type>, 
      basic_bounding_box<graphics_math_type>> location() const noexcept;
    basic_brush<GraphicsSurfaces> brush() const noexcept;
    basic_font<GraphicsSurfaces> font() const noexcept;
    string text() const noexcept;
    basic_text_props<GraphicsSurfaces> text_props() const noexcept;
    basic_brush_props<GraphicsSurfaces> brush_props() const noexcept;
    basic_stroke_props<GraphicsSurfaces> stroke_props() const noexcept;
    basic_dashes<GraphicsSurfaces> dashes() const noexcept;
    basic_render_props<GraphicsSurfaces> render_props() const noexcept;
    basic_clip_props<GraphicsSurfaces> clip_props() const noexcept;
  };

  // \ref{\iotwod.cmdlists.drawtext.eq}, equality operators:
  template <class GraphicsSurfaces>
  bool operator==(
    const typename basic_commands<GraphicsSurfaces::draw_text& lhs,
    const typename basic_commands<GraphicsSurfaces::draw_text& rhs) 
    noexcept;
  template <class GraphicsSurfaces>
  bool operator!=(
    const typename basic_commands<GraphicsSurfaces::draw_text& lhs,
    const typename basic_commands<GraphicsSurfaces::draw_text& rhs) 
    noexcept;
}
\end{codeblock}

\rSec1 [\iotwod.cmdlists.drawtext.ctor] {Constructors}%

\indexlibrary{\idxcode{draw_text}!constructor}%
\begin{itemdecl}
draw_text(const basic_point_2d<graphics_math_type>& pt,
  const basic_brush<GraphicsSurfaces>& b,
  const basic_font<GraphicsSurfaces>& font, string t,
  const basic_text_props<GraphicsSurfaces>& tp = 
  basic_text_props<GraphicsSurfaces>{},
  const basic_brush_props<GraphicsSurfaces>& bp = 
  basic_brush_props<GraphicsSurfaces>{},
  const basic_stroke_props<GraphicsSurfaces>& sp = 
  basic_stroke_props<GraphicsSurfaces>{},
  const basic_dashes<GraphicsSurfaces>& d =
  basic_dashes<GraphicsSurfaces>{},
  const basic_render_props<GraphicsSurfaces>& rp = 
  basic_render_props<GraphicsSurfaces>{},
  const basic_clip_props<GraphicsSurfaces>& cl = 
  basic_clip_props<GraphicsSurfaces>{}) noexcept;
\end{itemdecl}
\begin{itemdescr}
\pnum
\effects Constructs an object of type \tcode{draw_text}.

\pnum
\postconditions \tcode{data() == GraphicsSurfaces::surfaces::create_draw_text(pt, b, font, t, tp, bp, sp, d, rp, cl)}.
\end{itemdescr}

\indexlibrary{\idxcode{draw_text}!constructor}%
\begin{itemdecl}
draw_text(reference_wrapper<basic_image_surface<GraphicsSurfaces>> sfc,
  const basic_point_2d<graphics_math_type>& pt,
  const basic_brush<GraphicsSurfaces>& b,
  const basic_font<GraphicsSurfaces>& font, string t,
  const basic_text_props<GraphicsSurfaces>& tp = 
  basic_text_props<GraphicsSurfaces>{},
  const basic_brush_props<GraphicsSurfaces>& bp = 
  basic_brush_props<GraphicsSurfaces>{},
  const basic_stroke_props<GraphicsSurfaces>& sp = 
  basic_stroke_props<GraphicsSurfaces>{},
  const basic_dashes<GraphicsSurfaces>& d =
  basic_dashes<GraphicsSurfaces>{},
  const basic_render_props<GraphicsSurfaces>& rp = 
  basic_render_props<GraphicsSurfaces>{},
  const basic_clip_props<GraphicsSurfaces>& cl = 
  basic_clip_props<GraphicsSurfaces>{}) noexcept;
\end{itemdecl}
\begin{itemdescr}
\pnum
\effects Constructs an object of type \tcode{draw_text}.

\pnum
\postconditions \tcode{data() == GraphicsSurfaces::surfaces::create_draw_text(sfc, pt, b, font, t, tp, bp, sp, d, rp, cl)}.
\end{itemdescr}

\indexlibrary{\idxcode{draw_text}!constructor}%
\begin{itemdecl}
draw_text(const basic_bounding_box<graphics_math_type>& bb,
  const basic_brush<GraphicsSurfaces>& b,
  const basic_font<GraphicsSurfaces>& font, string t,
  const basic_text_props<GraphicsSurfaces>& tp = 
  basic_text_props<GraphicsSurfaces>{},
  const basic_brush_props<GraphicsSurfaces>& bp = 
  basic_brush_props<GraphicsSurfaces>{},
  const basic_stroke_props<GraphicsSurfaces>& sp = 
  basic_stroke_props<GraphicsSurfaces>{},
  const basic_dashes<GraphicsSurfaces>& d = 
  basic_dashes<GraphicsSurfaces>{},
  const basic_render_props<GraphicsSurfaces>& rp = 
  basic_render_props<GraphicsSurfaces>{},
  const basic_clip_props<GraphicsSurfaces>& cl = 
  basic_clip_props<GraphicsSurfaces>{}) noexcept;
\end{itemdecl}
\begin{itemdescr}
\pnum
\effects Constructs an object of type \tcode{draw_text}.

\pnum
\postconditions \tcode{data() == GraphicsSurfaces::surfaces::create_draw_text(bb, b, font, t, tp, bp, sp, d, rp, cl)}.
\end{itemdescr}

\indexlibrary{\idxcode{draw_text}!constructor}%
\begin{itemdecl}
draw_text(reference_wrapper<basic_image_surface<GraphicsSurfaces>> sfc, 
  const basic_bounding_box<graphics_math_type>& bb,
  const basic_brush<GraphicsSurfaces>& b,
  const basic_font<GraphicsSurfaces>& font, string t,
  const basic_text_props<GraphicsSurfaces>& tp = 
  basic_text_props<GraphicsSurfaces>{},
  const basic_brush_props<GraphicsSurfaces>& bp = 
  basic_brush_props<GraphicsSurfaces>{},
  const basic_stroke_props<GraphicsSurfaces>& sp = 
  basic_stroke_props<GraphicsSurfaces>{},
  const basic_dashes<GraphicsSurfaces>& d = 
  basic_dashes<GraphicsSurfaces>{},
  const basic_render_props<GraphicsSurfaces>& rp = 
  basic_render_props<GraphicsSurfaces>{},
  const basic_clip_props<GraphicsSurfaces>& cl = 
  basic_clip_props<GraphicsSurfaces>{}) noexcept;
\end{itemdecl}
\begin{itemdescr}
\pnum
\effects Constructs an object of type \tcode{draw_text}.

\pnum
\postconditions \tcode{data() == GraphicsSurfaces::surfaces::create_draw_text(sfc, bb, b, font, t, tp, bp, sp, d, rp, cl)}.
\end{itemdescr}

\rSec1 [\iotwod.cmdlists.drawtext.acc] {Accessors}%

\indexlibrarymember{data}{draw_text}%
\begin{itemdecl}
const data_type& data() const noexcept;
data_type& data() noexcept;
\end{itemdecl}
\begin{itemdescr}
\pnum
\returns A reference to the \tcode{draw_text} object's data object (See: \ref{\iotwod.cmdlists.drawtext.intro}).

\pnum
\remarks The behavior of a program is undefined if the user modifies the data contained in the \tcode{data_type} object returned by this function.
\end{itemdescr}

\rSec1 [\iotwod.cmdlists.drawtext.mod] {Modifiers}%

\indexlibrarymember{surface}{draw_text}%
\begin{itemdecl}
void surface(
  optional<reference_wrapper<basic_image_surface<GraphicsSurfaces>>> sfc) 
  noexcept;
\end{itemdecl}
\begin{itemdescr}
\pnum
\effects Calls \tcode{GraphicsSurfaces::surfaces::surface(data(), sfc)}.

\pnum
\remarks The optional surface is \tcode{sfc}.
\end{itemdescr}

\indexlibrarymember{location}{draw_text}%
\begin{itemdecl}
void location(const basic_point_2d<graphics_math_type>& pt) noexcept;
\end{itemdecl}
\begin{itemdescr}
\pnum
\effects Calls \tcode{GraphicsSurfaces::surfaces::location(data(), pt)}.

\pnum
\remarks The text location holds \tcode{pt} as its value.
\end{itemdescr}

\indexlibrarymember{location}{draw_text}%
\begin{itemdecl}
void location(const basic_bounding_box<graphics_math_type>& bb) noexcept;
\end{itemdecl}
\begin{itemdescr}
\pnum
\effects Calls \tcode{GraphicsSurfaces::surfaces::location(data(), bb)}.

\pnum
\remarks The text location holds \tcode{bb} at its value.
\end{itemdescr}

\indexlibrarymember{brush}{draw_text}%
\begin{itemdecl}
void brush(const basic_brush<GraphicsSurfaces>& b) noexcept;
\end{itemdecl}
\begin{itemdescr}
\pnum
\effects Calls \tcode{GraphicsSurfaces::surfaces::brush(data(), b)}.

\pnum
\remarks The brush is \tcode{b}.
\end{itemdescr}

\indexlibrarymember{font}{draw_text}%
\begin{itemdecl}
void font(const basic_font<GraphicsSurfaces>& f) noexcept;
\end{itemdecl}
\begin{itemdescr}
\pnum
\effects Calls \tcode{GraphicsSurfaces::surfaces::font(data(), f)}.

\pnum
\remarks The font is \tcode{f}.
\end{itemdescr}

\indexlibrarymember{text}{draw_text}%
\begin{itemdecl}
void text(string t) noexcept;
\end{itemdecl}
\begin{itemdescr}
\pnum
\effects Calls \tcode{GraphicsSurfaces::surfaces::text(data(), b)}.

\pnum
\remarks The text is \tcode{t}.
\end{itemdescr}

\indexlibrarymember{text_props}{draw_text}%
\begin{itemdecl}
void text_props(const basic_text_props<GraphicsSurfaces>& tp) noexcept;
\end{itemdecl}
\begin{itemdescr}
\pnum
\effects Calls \tcode{GraphicsSurfaces::surfaces::text_props(data(), tp)}.

\pnum
\remarks The text props is \tcode{tp}.
\end{itemdescr}

\indexlibrarymember{brush_props}{draw_text}%
\begin{itemdecl}
void brush_props(const basic_brush_props<GraphicsSurfaces>& bp) noexcept;
\end{itemdecl}
\begin{itemdescr}
\pnum
\effects Calls \tcode{GraphicsSurfaces::surfaces::brush_props(data(), bp)}.

\pnum
\remarks The brush props is \tcode{bp}.
\end{itemdescr}

\indexlibrarymember{stroke_props}{draw_text}%
\begin{itemdecl}
void brush_props(const basic_stroke_props<GraphicsSurfaces>& sp) noexcept;
\end{itemdecl}
\begin{itemdescr}
\pnum
\effects Calls \tcode{GraphicsSurfaces::surfaces::stroke_props(data(), sp)}.

\pnum
\remarks The stroke props is \tcode{sp}.
\end{itemdescr}

\indexlibrarymember{dashes}{draw_text}%
\begin{itemdecl}
void dashes(const basic_dashes<GraphicsSurfaces>& d) noexcept;
\end{itemdecl}
\begin{itemdescr}
\pnum
\effects Calls \tcode{GraphicsSurfaces::surfaces::dashes(data(), d)}.

\pnum
\remarks The dashes is \tcode{d}.
\end{itemdescr}

\indexlibrarymember{render_props}{draw_text}%
\begin{itemdecl}
void render_props(const basic_render_props<GraphicsSurfaces>& rp) noexcept;
\end{itemdecl}
\begin{itemdescr}
\pnum
\effects Calls \tcode{GraphicsSurfaces::surfaces::render_props(data(), rp)}.

\pnum
\remarks The render props is \tcode{rp}.
\end{itemdescr}

\indexlibrarymember{clip_props}{draw_text}%
\begin{itemdecl}
void clip_props(const basic_clip_props<GraphicsSurfaces>& cl) noexcept;
\end{itemdecl}
\begin{itemdescr}
\pnum
\effects Calls \tcode{GraphicsSurfaces::surfaces::clip_props(data(), cl)}.

\pnum
\remarks The clip props is \tcode{cl}.
\end{itemdescr}

\rSec1 [\iotwod.cmdlists.drawtext.obs] {Observers}%

\indexlibrarymember{surface}{draw_text}%
\begin{itemdecl}
optional<reference_wrapper<basic_image_surface<GraphicsSurfaces>>> 
  surface() const noexcept;
\end{itemdecl}
\begin{itemdescr}
\pnum
\returns \tcode{GraphicsSurfaces::surfaces::surface(data())}.

\pnum
\remarks
The returned value is the optional surface.
\end{itemdescr}

\indexlibrarymember{location}{draw_text}%
\begin{itemdecl}
variant<basic_point_2d<graphics_math_type>, basic_bounding_box<graphics_math_type>> brush() const noexcept;
\end{itemdecl}
\begin{itemdescr}
\pnum
\returns \tcode{GraphicsSurfaces::surfaces::location(data())}.

\pnum
\remarks The returned value is the text location.
\end{itemdescr}

\indexlibrarymember{brush}{draw_text}%
\begin{itemdecl}
basic_brush<GraphicsSurfaces> brush() const noexcept;
\end{itemdecl}
\begin{itemdescr}
\pnum
\returns \tcode{GraphicsSurfaces::surfaces::brush(data())}.

\pnum
\remarks The returned value is the brush.
\end{itemdescr}

\indexlibrarymember{font}{draw_text}%
\begin{itemdecl}
basic_font<GraphicsSurfaces> font() const noexcept;
\end{itemdecl}
\begin{itemdescr}
\pnum
\returns \tcode{GraphicsSurfaces::surfaces::font(data())}.

\pnum
\remarks The returned value is the font.
\end{itemdescr}

\indexlibrarymember{text_props}{draw_text}%
\begin{itemdecl}
basic_text_props<GraphicsSurfaces> text_props() const noexcept;
\end{itemdecl}
\begin{itemdescr}
\pnum
\returns \tcode{GraphicsSurfaces::surfaces::text_props(data())}.

\pnum
\remarks The returned value is the text props.
\end{itemdescr}

\indexlibrarymember{brush_props}{draw_text}%
\begin{itemdecl}
basic_brush_props<GraphicsSurfaces> brush_props() const noexcept;
\end{itemdecl}
\begin{itemdescr}
\pnum
\returns \tcode{GraphicsSurfaces::surfaces::brush_props(data())}.

\pnum
\remarks The returned value is the brush props.
\end{itemdescr}

\indexlibrarymember{stroke_props}{draw_text}%
\begin{itemdecl}
basic_stroke_props<GraphicsSurfaces> stroke_props() const noexcept;
\end{itemdecl}
\begin{itemdescr}
\pnum
\returns \tcode{GraphicsSurfaces::surfaces::stroke_props(data())}.

\pnum
\remarks The returned value is the stroke props.
\end{itemdescr}

\indexlibrarymember{dashes}{draw_text}%
\begin{itemdecl}
basic_dashes<GraphicsSurfaces> dashes() const noexcept;
\end{itemdecl}
\begin{itemdescr}
\pnum
\returns \tcode{GraphicsSurfaces::surfaces::dashes(data())}.

\pnum
\remarks The returned value is the dashes.
\end{itemdescr}

\indexlibrarymember{render_props}{draw_text}%
\begin{itemdecl}
basic_render_props<GraphicsSurfaces> render_props() const noexcept;
\end{itemdecl}
\begin{itemdescr}
\pnum
\returns \tcode{GraphicsSurfaces::surfaces::render_props(data())}.

\pnum
\remarks The returned value is the render props.
\end{itemdescr}

\indexlibrarymember{clip_props}{draw_text}%
\begin{itemdecl}
basic_clip_props<GraphicsSurfaces> clip_props() const noexcept;
\end{itemdecl}
\begin{itemdescr}
\pnum
\returns \tcode{GraphicsSurfaces::surfaces::clip_props(data())}.

\pnum
\remarks The returned value is the clip props.
\end{itemdescr}

\rSec1 [\iotwod.cmdlists.drawtext.eq] {Equality operators}%

\indexlibrarymember{operator==}{draw_text}%
\begin{itemdecl}
template <class GraphicsSurfaces>
bool operator==(
  const typename basic_commands<GraphicsSurfaces::draw_text& lhs,
  const typename basic_commands<GraphicsSurfaces::draw_text& rhs) 
  noexcept;
\end{itemdecl}
\begin{itemdescr}
\pnum
\returns \tcode{GraphicsSurfaces::surfaces::equal(lhs.data(), rhs.data())}.
\end{itemdescr}

\indexlibrarymember{operator!=}{draw_text}%
\begin{itemdecl}
template <class GraphicsSurfaces>
bool operator!=(
  const typename basic_commands<GraphicsSurfaces::draw_text& lhs,
  const typename basic_commands<GraphicsSurfaces::draw_text& rhs) 
  noexcept;
\end{itemdecl}
\begin{itemdescr}
\pnum
\returns \tcode{GraphicsSurfaces::surfaces::not_equal(lhs.data(), rhs.data())}.
\end{itemdescr}

%!TEX root = io2d.tex

\rSec0 [\iotwod.cmdlists.commands.runfunc] {Class template \tcode{basic_commands<GraphicsSurfaces>::run_function}}

\rSec1 [\iotwod.cmdlists.runfunc.intro] {Overview}

\pnum
\indexlibrary{\idxcode{run_function}}%
The class template \tcode{basic_commands<GraphicsSurfaces>::run_function} describes a command that invokes the user-provided function, passing it a reference to the surface the command list was submitted to, an optional surface, and user data. It allows the user to perform arbitrary operations that are not otherwise possible using the other command types.

\pnum
It has a \term{user-provided function} of type \tcode{variant<function<void(basic_image_surface<GraphicsSurfaces>\&, 
      optional<reference_wrapper<basic_image_surface<GraphicsSurfaces>>>, 
      void*)>, function<void(basic_output_surface<GraphicsSurfaces>\&, 
      optional<reference_wrapper<basic_image_surface<GraphicsSurfaces>>>, 
      void*)>, function<void(basic_unmanaged_output_surface<GraphicsSurfaces>\&, 
      optional<reference_wrapper<basic_image_surface<GraphicsSurfaces>>>, 
      void*)>>}.

\pnum
\begin{note}
The user-defined function is stored in a variant to avoid having three separate classes that essentially provide the same functionality.
\end{note}

\pnum
It has an \term{optional surface} of type \tcode{optional<reference_wrapper<basic_image_surface<GraphicsSurfaces>>>}.

\pnum
It has \tcode{user data} of type \tcode{void*}.

\pnum
The data are stored in an object of type \tcode{typename GraphicsSurfaces::surfaces::run_function_data_type}. It is accessible using the \tcode{data} member functions.

\rSec1 [\iotwod.cmdlists.runfunc.synopsis] {Synopsis}
\begin{codeblock}
namespace @\fullnamespace{}@ {
  template <class GraphicsSurfaces>
  class basic_commands<GraphicsSurfaces::run_function {
  public:
    using graphics_math_type = typename GraphicsSurfaces::graphics_math_type;
    using data_type =
      typename GraphicsSurfaces::surfaces::run_function_data_type;

    // \ref{\iotwod.cmdlists.runfunc.ctor}, construct:
    run_function(const function<void(basic_image_surface<GraphicsSurfaces>&, 
      optional<reference_wrapper<basic_image_surface<GraphicsSurfaces>>>, 
      void*)>& fn, void* ud, 
      optional<reference_wrapper<basic_image_surface<GraphicsSurfaces>>> sfc) 
      noexcept;
    run_function(const function<void(basic_output_surface<GraphicsSurfaces>&, 
      optional<reference_wrapper<basic_image_surface<GraphicsSurfaces>>>, 
      void*)>& fn, void* ud, 
      optional<reference_wrapper<basic_image_surface<GraphicsSurfaces>>> sfc) 
      noexcept;
    run_function(
      const function<void(basic_unmanaged_output_surface<GraphicsSurfaces>&, 
      optional<reference_wrapper<basic_image_surface<GraphicsSurfaces>>>, 
      void*)>& fn, void* ud, 
      optional<reference_wrapper<basic_image_surface<GraphicsSurfaces>>> sfc) 
      noexcept;
    
    // \ref{\iotwod.cmdlists.runfunc.acc}, accessors:
    const data_type& data() const noexcept;
    data_type& data() noexcept;

    // \ref{\iotwod.cmdlists.runfunc.mod}, modifiers:
    void surface(
      optional<reference_wrapper<basic_image_surface<GraphicsSurfaces>>> sfc) 
      noexcept;
    void func(const function<void(basic_image_surface<GraphicsSurfaces>&, 
      optional<reference_wrapper<basic_image_surface<GraphicsSurfaces>>>, 
      void*)>& fn) noexcept;
    void func(const function<void(basic_output_surface<GraphicsSurfaces>&, 
      optional<reference_wrapper<basic_image_surface<GraphicsSurfaces>>>, 
      void*)>& fn) noexcept;
    void func(
      const function<void(basic_unmanaged_output_surface<GraphicsSurfaces>&, 
      optional<reference_wrapper<basic_image_surface<GraphicsSurfaces>>>, 
      void*)>& fn) noexcept;
    void user_data(void* ud) noexcept;

    // \ref{\iotwod.cmdlists.runfunc.obs}, observers:
    optional<reference_wrapper<basic_image_surface<GraphicsSurfaces>>> 
      surface() const noexcept;
    const variant<function<void(basic_image_surface<GraphicsSurfaces>&, 
      optional<reference_wrapper<basic_image_surface<GraphicsSurfaces>>>, 
      void*)>, function<void(basic_output_surface<GraphicsSurfaces>&, 
      optional<reference_wrapper<basic_image_surface<GraphicsSurfaces>>>, 
      void*)>, function<void(basic_unmanaged_output_surface<GraphicsSurfaces>&, 
      optional<reference_wrapper<basic_image_surface<GraphicsSurfaces>>>, 
      void*)>>& func() const noexcept;
    void* user_data() const noexcept;
  };

  // \ref{\iotwod.cmdlists.runfunc.eq}, equality operators:
  template <class GraphicsSurfaces>
  bool operator==(
    const typename basic_commands<GraphicsSurfaces::run_function& lhs,
    const typename basic_commands<GraphicsSurfaces::run_function& rhs) 
    noexcept;
  template <class GraphicsSurfaces>
  bool operator!=(
    const typename basic_commands<GraphicsSurfaces::run_function& lhs,
    const typename basic_commands<GraphicsSurfaces::run_function& rhs) 
    noexcept;
}
\end{codeblock}

\rSec1 [\iotwod.cmdlists.runfunc.ctor] {Constructors}%

\indexlibrary{\idxcode{run_function}!constructor}%
\begin{itemdecl}
run_function(const function<void(basic_image_surface<GraphicsSurfaces>&, 
  optional<reference_wrapper<basic_image_surface<GraphicsSurfaces>>>, 
  void*)>& fn, void* ud, 
  optional<reference_wrapper<basic_image_surface<GraphicsSurfaces>>> sfc) 
  noexcept;
run_function(const function<void(basic_output_surface<GraphicsSurfaces>&, 
  optional<reference_wrapper<basic_image_surface<GraphicsSurfaces>>>, 
  void*)>& fn, void* ud, 
  optional<reference_wrapper<basic_image_surface<GraphicsSurfaces>>> sfc) 
  noexcept;
run_function(
  const function<void(basic_unmanaged_output_surface<GraphicsSurfaces>&, 
  optional<reference_wrapper<basic_image_surface<GraphicsSurfaces>>>, 
  void*)>& fn, void* ud, 
  optional<reference_wrapper<basic_image_surface<GraphicsSurfaces>>> sfc) 
  noexcept;
\end{itemdecl}
\begin{itemdescr}
\pnum
\effects Constructs an object of type \tcode{run_function}.

\pnum
\postconditions \tcode{data() == GraphicsSurfaces::surfaces::create_run_function(fn, ud, sfc)}.
\end{itemdescr}

\rSec1 [\iotwod.cmdlists.runfunc.acc] {Accessors}%

\indexlibrarymember{data}{run_function}%
\begin{itemdecl}
const data_type& data() const noexcept;
data_type& data() noexcept;
\end{itemdecl}
\begin{itemdescr}
\pnum
\returns A reference to the \tcode{run_function} object's data object (See: \ref{\iotwod.cmdlists.runfunc.intro}).
\end{itemdescr}

\rSec1 [\iotwod.cmdlists.runfunc.mod] {Modifiers}%

\indexlibrarymember{surface}{run_function}%
\begin{itemdecl}
void surface(
  optional<reference_wrapper<basic_image_surface<GraphicsSurfaces>>> sfc) 
  noexcept;
\end{itemdecl}
\begin{itemdescr}
\pnum
\effects Calls \tcode{GraphicsSurfaces::surfaces::surface(data(), sfc)}.

\pnum
\remarks The optional surface is \tcode{sfc}.
\end{itemdescr}

\indexlibrarymember{func}{run_function}%
\begin{itemdecl}
void func(const function<void(basic_image_surface<GraphicsSurfaces>&, 
  optional<reference_wrapper<basic_image_surface<GraphicsSurfaces>>>, 
  void*)>& fn) noexcept;
void func(const function<void(basic_output_surface<GraphicsSurfaces>&, 
  optional<reference_wrapper<basic_image_surface<GraphicsSurfaces>>>, 
  void*)>& fn) noexcept;
void func(
  const function<void(basic_unmanaged_output_surface<GraphicsSurfaces>&, 
  optional<reference_wrapper<basic_image_surface<GraphicsSurfaces>>>, 
  void*)>& fn) noexcept;
\end{itemdecl}
\begin{itemdescr}
\pnum
\effects Calls \tcode{GraphicsSurfaces::surfaces::func(data(), fn)}.

\pnum
\remarks The user-defined function holds \tcode{fn} as its value.
\end{itemdescr}

\indexlibrarymember{user_data}{run_function}%
\begin{itemdecl}
void user_data(void* ud) noexcept;
\end{itemdecl}
\begin{itemdescr}
\pnum
\effects Calls \tcode{GraphicsSurfaces::surfaces::user_data(data(), ud)}.

\pnum
\remarks The user data is \tcode{ud}.
\end{itemdescr}

\rSec1 [\iotwod.cmdlists.runfunc.obs] {Observers}%

\indexlibrarymember{surface}{run_function}%
\begin{itemdecl}
optional<reference_wrapper<basic_image_surface<GraphicsSurfaces>>> 
  surface() const noexcept;
\end{itemdecl}
\begin{itemdescr}
\pnum
\returns \tcode{GraphicsSurfaces::surfaces::surface(data())}.

\pnum
\remarks
The returned value is the optional surface.
\end{itemdescr}

\indexlibrarymember{func}{run_function}%
\begin{itemdecl}
const variant<function<void(basic_image_surface<GraphicsSurfaces>&, 
  optional<reference_wrapper<basic_image_surface<GraphicsSurfaces>>>, 
  void*)>, function<void(basic_output_surface<GraphicsSurfaces>&, 
  optional<reference_wrapper<basic_image_surface<GraphicsSurfaces>>>, 
  void*)>, function<void(basic_unmanaged_output_surface<GraphicsSurfaces>&, 
  optional<reference_wrapper<basic_image_surface<GraphicsSurfaces>>>, 
  void*)>>& func() const noexcept;
\end{itemdecl}
\begin{itemdescr}
\pnum
\returns \tcode{GraphicsSurfaces::surfaces::func(data())}.

\pnum
\remarks
The returned value is the user-defined function.
\end{itemdescr}

\indexlibrarymember{user_data}{run_function}%
\begin{itemdecl}
void* user_data() const noexcept;
\end{itemdecl}
\begin{itemdescr}
\pnum
\returns \tcode{GraphicsSurfaces::surfaces::user_data(data())}.

\pnum
\remarks
The returned value is the user data.
\end{itemdescr}

\rSec1 [\iotwod.cmdlists.runfunc.eq] {Equality operators}%

\indexlibrarymember{operator==}{run_function}%
\begin{itemdecl}
template <class GraphicsSurfaces>
bool operator==(
  const typename basic_commands<GraphicsSurfaces::run_function& lhs,
  const typename basic_commands<GraphicsSurfaces::run_function& rhs) 
  noexcept;
\end{itemdecl}
\begin{itemdescr}
\pnum
\returns \tcode{GraphicsSurfaces::surfaces::equal(lhs.data(), rhs.data())}.
\end{itemdescr}

\indexlibrarymember{operator!=}{run_function}%
\begin{itemdecl}
template <class GraphicsSurfaces>
bool operator!=(
  const typename basic_commands<GraphicsSurfaces::run_function& lhs,
  const typename basic_commands<GraphicsSurfaces::run_function& rhs) 
  noexcept;
\end{itemdecl}
\begin{itemdescr}
\pnum
\returns \tcode{GraphicsSurfaces::surfaces::not_equal(lhs.data(), rhs.data())}.
\end{itemdescr}

\addtocounter{SectionDepthBase}{-2}

\addtocounter{SectionDepthBase}{1}
%!TEX root = io2d.tex
\rSec0 [\iotwod.commandlist] {Class template \tcode{basic_command_list}}

\rSec1 [\iotwod.commandlist.intro] {Overview}

\pnum
\indexlibrary{\idxcode{basic_command_list}}%
The class template \tcode{basic_command_list<\graphicssurfacestemplparamnospace{}>} contains the data that results from a back end pre-compiling (interpreting) a sequence of \tcode{basic_commands<\graphicssurfacestemplparamnospace{}>::command_item} objects.

\pnum
The data are stored in an object of type \tcode{typename \graphicssurfacestemplparamnospace{}::surfaces::command_list_data_type}. It is accessible using the \tcode{data} member function.

\rSec1 [\iotwod.commandlist.synopsis] {\tcode{basic_command_list} synopsis}

\begin{codeblock}
namespace @\fullnamespace{}@ {
  template <class @\graphicssurfacestemplparamnospace{}@>
  class basic_command_list {
  public:
    using data_type = typename 
      @\graphicssurfacestemplparamnospace{}@::surfaces::command_list_data_type;
      
    // \ref{\iotwod.commandlist.ctor}, construct:
    basic_command_list() noexcept;
    template <class InputIterator>
    basic_command_list(InputIterator first, InputIterator last);
    explicit basic_command_list(@\stdqualifier{}@initializer_list<typename
      basic_commands<@\graphicssurfacestemplparamnospace{}@>::command_item>> il);    
    
    // \ref{\iotwod.commandlist.acc}, accessors:
    const data_type& data() const noexcept;
  };
}
\end{codeblock}

\rSec1 [\iotwod.commandlist.ctor] {\tcode{basic_command_list} constructors}

\indexlibrary{\idxcode{basic_command_list}!constructor}%
\begin{itemdecl}
basic_command_list() noexcept;
\end{itemdecl}
\begin{itemdescr}
\pnum
\pnum
\effects
Constructs an object of type \tcode{basic_command_list}.

\pnum
\postconditions
\tcode{data() == \graphicssurfacestemplparamnospace{}::surfaces::create_command_list()}.
\end{itemdescr}

\indexlibrary{\idxcode{basic_command_list}!constructor}%
\begin{itemdecl}
explicit basic_command_list(const basic_bounding_box<@\graphicsmathtemplparamnospace{}@>& bb);
\end{itemdecl}
\begin{itemdescr}
\pnum
\effects
Constructs an object of type \tcode{basic_command_list}.

\pnum
\postconditions
\tcode{data() == \graphicssurfacestemplparamnospace{}::surfaces::create_command_list()}.
\end{itemdescr}

\indexlibrary{\idxcode{basic_command_list}!constructor}%
\begin{itemdecl}
template <class InputIterator>
basic_command_list(InputIterator first, InputIterator last);
\end{itemdecl}
\begin{itemdescr}
\pnum
\effects
Constructs an object of type \tcode{basic_command_list}.

\pnum
\postconditions
\tcode{data() == \graphicssurfacestemplparamnospace{}::surfaces::create_command_list(first, last)}.

\pnum
\begin{note}
The contained data is the result of the back end pre-compiling the series of objects of type \tcode{basic_commands<GraphicsSurfaces>::command_item} from \tcode{first} to the last element before \tcode{last}.
\end{note}
\end{itemdescr}

\indexlibrary{\idxcode{basic_command_list}!constructor}%
\begin{itemdecl}
explicit basic_command_list(@\stdqualifier{}@initializer_list<typename
  basic_commands<GraphicsSurfaces>::command_item> il);
\end{itemdecl}
\begin{itemdescr}
\pnum
\effects
Equivalent to: \tcode{basic_command_list\{ il.begin(), il.end() \}}.
\end{itemdescr}

\rSec1 [\iotwod.commandlist.acc] {Accessors}

\indexlibrarymember{data}{basic_command_list}%
\begin{itemdecl}
const data_type& data() const noexcept;
\end{itemdecl}
\begin{itemdescr}
\pnum
\returns A reference to the \tcode{basic_command_list} object's data object (See: \ref{\iotwod.commandlist.intro}).
\end{itemdescr}

\addtocounter{SectionDepthBase}{-1}

\begin{codeblock}
namespace @\fullnamespace{}@ {
  template <class GraphicsSurfaces>
  struct basic_commands {
    using graphics_math_type = typename GraphicsSurfaces::graphics_math_type;

    class clear;
    class paint;
    class stroke;
    class fill;
    class mask;
    class draw_text;
    class run_function;
  }
  using command_item = variant<clear, paint, stroke, fill, mask, draw_text, 
    run_function>;
};
\end{codeblock}
