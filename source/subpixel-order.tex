%!TEX root = io2d.tex
\rSec0 [subpixel.order] {Enum class \tcode{subpixel_order}}

\rSec1 [subpixel.order.summary] {\tcode{subpixel_order} Summary}

\pnum
The \tcode{subpixel_order} enum class is used to request a specific order of 
color channels for each pixel of an output device. When a \tcode{surface} 
object's \tcode{font_options} object has its \tcode{antialias} 
value set to \tcode{antialias::subpixel} and its \tcode{subpixel_order} value 
set to one of these values, an implementation should use the specified 
\tcode{subpixel_order} to render text.
See Table~\ref{tab:subpixel.order.meanings} for the meaning of each
\tcode{subpixel_order} enumerator.

\rSec1 [subpixel.order.synopsis] {\tcode{subpixel_order} Synopsis}

\begin{codeblock}
namespace std { namespace experimental { namespace io2d { inline namespace v1 {
  enum class subpixel_order {
    default_subpixel_order,
    horizontal_rgb,
    horizontal_bgr,
    vertical_rgb,
    vertical_bgr
  };
} } } } // namespaces std::experimental::io2d::v1
\end{codeblock}

\rSec1 [subpixel.order.enumerators] {\tcode{subpixel_order} Enumerators}
\begin{libreqtab2}
 {\tcode{subpixel_order} enumerator meanings}
 {tab:subpixel.order.meanings}
 \\ \topline
 \lhdr{Enumerator}
 & \rhdr{Meaning}
 \\ \capsep
 \endfirsthead
 \continuedcaption\\
 \hline
 \lhdr{Enumerator}
 & \rhdr{Meaning}
 \\ \capsep
 \endhead
 \tcode{default_subpixel_order}
 & The implementation should use the target \tcode{surface} object's default 
 subpixel order.
 \\
 \tcode{horizontal_rgb}
 & The color channels should be arranged horizontally starting with red on the 
 left, followed by green, then blue.
 \\
 \tcode{horizontal_bgr}
 & The color channels should be arranged horizontally starting with blue on the 
 left, followed by green, then red.
 \\
 \tcode{vertical_rgb}
 & The color channels should be arranged vertically starting with red on the 
 top, followed by green, then blue.
 \\
 \tcode{vertical_bgr}
 & The color channels should be arranged vertically starting with blue on the 
 top, followed by green, then red.
 \\
\end{libreqtab2}
