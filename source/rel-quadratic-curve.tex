%!TEX root = io2d.tex
\rSec0 [\iotwod.relquadraticcurve] {Class \tcode{rel_quadratic_curve}}

\pnum
\indexlibrary{\idxcode{rel_quadratic_curve}}%
The class \tcode{rel_quadratic_curve} describes a figure item that is a segment.

\pnum
It has a \term{control point} of type \tcode{basic_point_2d} and an \term{end point} of type \tcode{basic_point_2d}.

\rSec1 [\iotwod.relquadraticcurve.cons] {\tcode{rel_quadratic_curve} constructors}

\indexlibrary{\idxcode{rel_quadratic_curve}!constructor}%
\begin{itemdecl}
rel_quadratic_curve() noexcept;
\end{itemdecl}
\begin{itemdescr}
\pnum
\effects
Equivalent to: \tcode{rel_quadratic_curve\{ basic_point_2d(), basic_point_2d() \};}
\end{itemdescr}

\indexlibrary{\idxcode{rel_quadratic_curve}!constructor}%
\begin{itemdecl}
rel_quadratic_curve(const basic_point_2d<typename GraphicsSurfaces::graphics_math_type>& cpt,
  const basic_point_2d<typename GraphicsSurfaces::graphics_math_type>& ept) noexcept;
\end{itemdecl}
\begin{itemdescr}
\pnum
\effects
Constructs an object of type \tcode{rel_quadratic_curve}.

\pnum
The control point is \tcode{cpt}.

\pnum
The end point is \tcode{ept}.
\end{itemdescr}

\indexlibrary{\idxcode{rel_quadratic_curve}!constructor}%
\begin{itemdecl}
rel_quadratic_curve(const rel_quadratic_curve& other);
rel_quadratic_curve(rel_quadratic_curve&& other) noexcept;
\end{itemdecl}
\begin{itemdescr}
\pnum
\effects
Constructs an object of type \tcode{rel_quadratic_curve}. In the second form, other is left in a valid state with an unspecified value.

\pnum
The control point is \tcode{other.control_pt()}.

\pnum
The end point is \tcode{other.end_pt()}.
\end{itemdescr}

\rSec1 [\iotwod.relquadraticcurve.assign] {\tcode{rel_quadratic_curve} assignment operators}

\indexlibrary{\idxcode{rel_quadratic_curve}!assignment}%
\begin{itemdecl}
rel_quadratic_curve& operator=(const rel_quadratic_curve& other);
\end{itemdecl}
\begin{itemdescr}
\pnum
\effects
If \tcode{*this} and \tcode{other} are not the same object, modifies \tcode{*this} such that \tcode{*this.control_pt()} is \tcode{other.control_pt()} and \tcode{*this.end_pt()} is \tcode{other.end_pt()}

\pnum
If \tcode{*this} and \tcode{other} are the same object, the member has no effect.

\pnum
\returns
\tcode{*this}
\end{itemdescr}

\indexlibrary{\idxcode{rel_quadratic_curve}!assignment}%
\begin{itemdecl}
rel_quadratic_curve& operator=(rel_quadratic_curve&& other) noexcept;
\end{itemdecl}
\begin{itemdescr}
\pnum
\effects
<TODO>

\pnum
\returns
\tcode{*this}
\end{itemdescr}

\rSec1 [\iotwod.relquadraticcurve.modifiers]{\tcode{rel_quadratic_curve} modifiers}

\indexlibrarymember{control_pt}{rel_quadratic_curve}%
\begin{itemdecl}
void control_pt(const basic_point_2d<typename GraphicsSurfaces::graphics_math_type>& cpt) noexcept;
\end{itemdecl}
\begin{itemdescr}
\pnum
\effects
The control point is \tcode{cpt}.
\end{itemdescr}

\indexlibrarymember{end_pt}{rel_quadratic_curve}%
\begin{itemdecl}
void end_pt(const basic_point_2d<typename GraphicsSurfaces::graphics_math_type>& ept) noexcept;
\end{itemdecl}
\begin{itemdescr}
\pnum
\effects
The end point is \tcode{ept}.
\end{itemdescr}

\rSec1 [\iotwod.relquadraticcurve.observers]{\tcode{rel_quadratic_curve} observers}

\indexlibrarymember{control_pt}{rel_quadratic_curve}%
\begin{itemdecl}
basic_point_2d<typename GraphicsSurfaces::graphics_math_type> control_pt() const noexcept;
\end{itemdecl}
\begin{itemdescr}
\pnum
\returns
The control point.
\end{itemdescr}

\indexlibrarymember{end_pt}{rel_quadratic_curve}%
\begin{itemdecl}
basic_point_2d<typename GraphicsSurfaces::graphics_math_type> end_pt() const noexcept;
\end{itemdecl}
\begin{itemdescr}
\pnum
\returns
The end point.
\end{itemdescr}

\rSec1 [\iotwod.relquadraticcurve.ops]{\tcode{rel_quadratic_curve} operators}

\indexlibrarymember{operator==}{rel_quadratic_curve}%
\begin{itemdecl}
template <class GraphicsSurfaces>
bool operator==(const typename basic_figure_items<GraphicsSurfaces>::rel_quadratic_curve& lhs,
  const typename basic_figure_items<GraphicsSurfaces>::rel_quadratic_curve& rhs) noexcept;
\end{itemdecl}
\begin{itemdescr}
\pnum
\returns
\tcode{lhs.control_pt() == rhs.control_pt() \&\& lhs.end_pt() == rhs.end_pt()}.
\end{itemdescr}
