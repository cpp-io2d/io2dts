%!TEX root = io2d.tex
\rSec0 [\iotwod.relquadraticcurve] {Class \tcode{rel_quadratic_curve}}

\pnum
\indexlibrary{\idxcode{rel_quadratic_curve}}%
The class \tcode{rel_quadratic_curve} describes a path item that adds a quadratic \bezierlocal curve path segment to a path.

\pnum
This is a relative path item.

\pnum
It has a \term{control point} of type \tcode{vector_2d} and an \term{end point} of type \tcode{vector_2d}.

\pnum
When interpreting a path group, after the curve is added to the path, the path's current point becomes the end point.


\rSec1 [\iotwod.relquadraticcurve.synopsis] {\tcode{rel_quadratic_curve} synopsis}

\begin{codeblock}
namespace std::experimental::io2d::v1 {
  namespace path_data {
    class rel_quadratic_curve {
    public:
      // \ref{\iotwod.relquadraticcurve.cons}, construct:
      constexpr rel_quadratic_curve() noexcept;
      constexpr rel_quadratic_curve(const vector_2d& cpt, const vector_2d& ept)
        noexcept;

      // \ref{\iotwod.relquadraticcurve.modifiers}, modifiers:
      constexpr void control(const vector_2d& cpt) noexcept;
      constexpr void end(const vector_2d& ept) noexcept;

      // \ref{\iotwod.relquadraticcurve.observers}, observers:
      constexpr vector_2d control() const noexcept;
      constexpr vector_2d end() const noexcept;
    };
    
    \ref{\iotwod.relquadraticcurve.nonmember}, non-members
    constexpr bool operator==(const rel_quadratic_curve& lhs,
      const rel_quadratic_curve& rhs) noexcept;
    constexpr bool operator!=(const rel_quadratic_curve& lhs,
      const rel_quadratic_curve& rhs) noexcept;
  }
}
\end{codeblock}

\rSec1 [\iotwod.relquadraticcurve.cons] {\tcode{rel_quadratic_curve} constructors}

\indexlibrary{\idxcode{rel_quadratic_curve}!constructor}%
\begin{itemdecl}
constexpr rel_quadratic_curve() noexcept;
\end{itemdecl}
\begin{itemdescr}
\pnum
\effects
Equivalent to: \tcode{rel_quadratic_curve\{ vector_2d(), vector_2d() \};}
\end{itemdescr}

\indexlibrary{\idxcode{rel_quadratic_curve}!constructor}%
\begin{itemdecl}
constexpr rel_quadratic_curve(const vector_2d& cpt, const vector_2d& ept)
  noexcept;
\end{itemdecl}
\begin{itemdescr}
\pnum
\effects
Constructs an object of type \tcode{rel_quadratic_curve}.

\pnum
The control point is \tcode{cpt}.

\pnum
The end point is \tcode{ept}.
\end{itemdescr}

\rSec1 [\iotwod.relquadraticcurve.modifiers]{\tcode{rel_quadratic_curve} modifiers}

\indexlibrarymember{rel_quadratic_curve}{control}%
\begin{itemdecl}
constexpr void control(const vector_2d& cpt) noexcept;
\end{itemdecl}
\begin{itemdescr}
\pnum
\effects
The control point is \tcode{cp}.
\end{itemdescr}

\indexlibrarymember{rel_quadratic_curve}{end_point}%
\begin{itemdecl}
constexpr void end(const vector_2d& ept) noexcept;
\end{itemdecl}
\begin{itemdescr}
\pnum
\effects
The end point is \tcode{ept}.
\end{itemdescr}

\rSec1 [\iotwod.relquadraticcurve.observers]{\tcode{rel_quadratic_curve} observers}

\indexlibrarymember{rel_quadratic_curve}{control}%
\begin{itemdecl}
constexpr vector_2d control() const noexcept;
\end{itemdecl}
\begin{itemdescr}
\pnum
\returns
The control point.
\end{itemdescr}

\indexlibrarymember{rel_quadratic_curve}{end}%
\begin{itemdecl}
constexpr vector_2d end() const noexcept;
\end{itemdecl}
\begin{itemdescr}
\pnum
\returns
The end point.
\end{itemdescr}

\rSec1 [\iotwod.relquadraticcurve.nonmember]{Non-member functions}

\indexlibrarymember{operator==}{rel_quadratic_curve}%
\begin{itemdecl}
constexpr bool operator==(const rel_quadratic_curve& lhs,
  const rel_quadratic_curve& rhs) noexcept;
\end{itemdecl}
\begin{itemdescr}
\pnum
\returns
\tcode{lhs.control() == rhs.control() \&\& lhs.end() == rhs.end()}.
\end{itemdescr}

\indexlibrarymember{operator!=}{rel_quadratic_curve}%
\begin{itemdecl}
constexpr bool operator!=(const rel_quadratic_curve& lhs,
  const rel_quadratic_curve& rhs) noexcept;
\end{itemdecl}
\begin{itemdescr}
\pnum
\returns
\tcode{!(lhs == rhs)}.
\end{itemdescr}
