%!TEX root = io2d.tex
\rSec0 [relquadraticcurve] {Class \tcode{rel_quadratic_curve}}

\pnum
\indexlibrary{\idxcode{rel_quadratic_curve}}
The class \tcode{rel_quadratic_curve} describes a path segment that is a quadratic \bezierlocal curve.

\pnum
It has a control point of type \tcode{vector_2d} and an end point of type \tcode{vector_2d}.

\pnum
All of its points are relative to the most recently established current point.

\rSec1 [relquadraticcurve.synopsis] {\tcode{rel_quadratic_curve} synopsis}

\begin{codeblock}
namespace std { namespace experimental { namespace io2d { inline namespace v1 {
  namespace path_data {
    class rel_cubic_curve {
    public:
      // \ref{relquadraticcurve.cons}, construct:
      rel_quadratic_curve(const vector_2d& cp, const vector_2d& ep) noexcept;

      // \ref{relquadraticcurve.modifiers}, modifiers:
      void control_point(const vector_2d& cp) noexcept;
      void end_point(const vector_2d& ep) noexcept;

      // \ref{relquadraticcurve.observers}, observers:
      vector_2d control_point() const noexcept;
      vector_2d end_point() const noexcept;
    };
  };
} } } }
\end{codeblock}

\rSec1 [relquadraticcurve.cons] {\tcode{rel_quadratic_curve} constructors}

\indexlibrary{\idxcode{rel_quadratic_curve}!constructor}
\begin{itemdecl}
    rel_quadratic_curve(const vector_2d& cp, const vector_2d& ep) noexcept;
\end{itemdecl}
\begin{itemdescr}
	\pnum
	\effects
	Constructs an object of type \tcode{rel_cubic_curve}.
	
	\pnum
	The control point shall be set to the value of \tcode{cp}.
	
	\pnum
	The end point shall be set to the value of \tcode{ep}.
\end{itemdescr}

\rSec1 [relquadraticcurve.modifiers]{\tcode{rel_cubic_curve} modifiers}

\indexlibrary{\idxcode{rel_quadratic_curve}!\idxcode{control_point}}
\indexlibrary{\idxcode{control_point}!\idxcode{rel_quadratic_curve}}
\begin{itemdecl}
    void control_point(const vector_2d& cp) noexcept;
\end{itemdecl}
\begin{itemdescr}
	\pnum
	\effects
	The control point shall be set to the value of \tcode{cp}.
\end{itemdescr}

\indexlibrary{\idxcode{rel_quadratic_curve}!\idxcode{end_point}}
\indexlibrary{\idxcode{end_point}!\idxcode{rel_quadratic_curve}}
\begin{itemdecl}
    void end_point(const vector_2d& ep) noexcept;
\end{itemdecl}
\begin{itemdescr}
	\pnum
	\effects
	The end point shall be set to the value of \tcode{ep}.
\end{itemdescr}

\rSec1 [relquadraticcurve.observers]{\tcode{rel_quadratic_curve} observers}

\indexlibrary{\idxcode{rel_quadratic_curve}!\idxcode{control_point}}
\indexlibrary{\idxcode{control_point}!\idxcode{rel_quadratic_curve}}
\begin{itemdecl}
    vector_2d control_point() const noexcept;
\end{itemdecl}
\begin{itemdescr}
	\pnum
	\returns
	The value of the control point.
\end{itemdescr}

\indexlibrary{\idxcode{rel_quadratic_curve}!\idxcode{end_point}}
\indexlibrary{\idxcode{end_point}!\idxcode{rel_quadratic_curve}}
\begin{itemdecl}
    vector_2d end_point() const noexcept;
\end{itemdecl}
\begin{itemdescr}
	\pnum
	\returns
	The value of the end point.
\end{itemdescr}
