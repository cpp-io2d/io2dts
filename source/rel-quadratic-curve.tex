%!TEX root = io2d.tex
\rSec0 [\iotwod.relquadraticcurve] {Class template \tcode{basic_figure_items<GraphicsSurfaces>::rel_quadratic_curve}}

\rSec1 [\iotwod.relquadraticcurve.intro] {Overview}

\pnum
\indexlibrary{\idxcode{rel_quadratic_curve}}%
The class \tcode{basic_figure_items<GraphicsSurfaces>::rel_quadratic_curve} describes a figure item that is a segment.

\pnum
It has a \term{control point} of type \tcode{basic_point_2d<GraphicsSurfaces::graphics_math_type>} and an \term{end point} of type \tcode{basic_point_2d<GraphicsSurfaces::graphics_math_type>}.

\pnum
The data are stored in an object of type \tcode{typename GraphicsSurfaces::paths::rel_quadratic_curve_data_type}. It is accessible using the \tcode{data} member functions.

\rSec1 [\iotwod.relquadraticcurve.synopsis] {Synopsis}
\begin{codeblock}
namespace std::experimemtal::io2d::v1 {
  template <class GraphicsSurfaces>
  class basic_figure_items<GraphicsSurfaces>::rel_quadratic_curve {
  public:
    using graphics_math_type = typename GraphicsSurfaces::graphics_math_type;
    using data_type =
      typename GraphicsSurfaces::paths::rel_quadratic_curve_data_type;

    // \ref{\iotwod.relquadraticcurve.ctor}, construct:
    rel_quadratic_curve();
    rel_quadratic_curve(const basic_point_2d<graphics_math_type>& cpt,
      const basic_point_2d<graphics_math_type>& ept);
    rel_quadratic_curve(const rel_quadratic_curve& other) = default;
    rel_quadratic_curve(rel_quadratic_curve&& other) noexcept = default;

    // assign:
    rel_quadratic_curve& operator=(const rel_quadratic_curve& other) = default;
    rel_quadratic_curve& operator=(rel_quadratic_curve&& other) noexcept = default;

    // \ref{\iotwod.relquadraticcurve.acc}, accessors:
    const data_type& data() const noexcept;
    data_type& data() noexcept;

    // \ref{\iotwod.relquadraticcurve.mod}, modifiers:
    void control_pt(const basic_point_2d<graphics_math_type>& cpt) noexcept;
    void end_pt(const basic_point_2d<graphics_math_type>& ept) noexcept;

    // \ref{\iotwod.relquadraticcurve.obs}, observers:
    basic_point_2d<graphics_math_type> control_pt() const noexcept;
    basic_point_2d<graphics_math_type> end_pt() const noexcept;
  };

  // \ref{\iotwod.relquadraticcurve.eq}, equality operators:
  template <class GraphicsSurfaces>
  bool operator==(
    const typename basic_figure_items<GraphicsSurfaces>::rel_quadratic_curve& lhs,
    const typename basic_figure_items<GraphicsSurfaces>::rel_quadratic_curve& rhs) 
    noexcept;  
  template <class GraphicsSurfaces>
  bool operator!=(
    const typename basic_figure_items<GraphicsSurfaces>::rel_quadratic_curve& lhs,
    const typename basic_figure_items<GraphicsSurfaces>::rel_quadratic_curve& rhs) 
    noexcept;  
}
\end{codeblock}

\rSec1 [\iotwod.relquadraticcurve.ctor] {Constructors}%

\indexlibrary{\idxcode{rel_quadratic_curve}!constructor}%
\begin{itemdecl}
rel_quadratic_curve() noexcept;
\end{itemdecl}
\begin{itemdescr}
\pnum
\effects Equivalent to: \tcode{rel_quadratic_curve\{ basic_point_2d(), basic_point_2d() \};}
\end{itemdescr}

\indexlibrary{\idxcode{rel_quadratic_curve}!constructor}%
\begin{itemdecl}
rel_quadratic_curve(const basic_point_2d<typename GraphicsSurfaces::graphics_math_type>& cpt,
  const basic_point_2d<typename GraphicsSurfaces::graphics_math_type>& ept) noexcept;
\end{itemdecl}
\begin{itemdescr}
\pnum
\effects Constructs an object of type \tcode{rel_quadratic_curve}.

\pnum
\remarks The control point is \tcode{cpt}.

\pnum
\remarks The end point is \tcode{ept}.
\end{itemdescr}

\rSec1 [\iotwod.relquadraticcurve.acc] {Accessors}%

\indexlibrarymember{data}{rel_quadratic_curve}%
\begin{itemdecl}
const data_type& data() const noexcept;
data_type& data() noexcept;
\end{itemdecl}
\begin{itemdescr}
\pnum
\returns A reference to the \tcode{rel_quadratic_curve} object's data object (See: \ref{\iotwod.relquadraticcurve.intro}).
\end{itemdescr}

\rSec1 [\iotwod.relquadraticcurve.mod]{Modifiers}%

\indexlibrarymember{control_pt}{rel_quadratic_curve}%
\begin{itemdecl}
void control_pt(const basic_point_2d<graphics_math_type>& cpt) noexcept;
\end{itemdecl}
\begin{itemdescr}
\pnum
\effects The control point is \tcode{cpt}.
\end{itemdescr}

\indexlibrarymember{end_pt}{rel_quadratic_curve}%
\begin{itemdecl}
void end_pt(const basic_point_2d<graphics_math_type>& ept) noexcept;
\end{itemdecl}
\begin{itemdescr}
\pnum
\effects The end point is \tcode{ept}.
\end{itemdescr}

\rSec1 [\iotwod.relquadraticcurve.obs] {Observers}

\indexlibrarymember{control_pt}{rel_quadratic_curve}%
\begin{itemdecl}
basic_point_2d<graphics_math_type> control_pt() const noexcept;
\end{itemdecl}
\begin{itemdescr}
\pnum
\returns The control point.
\end{itemdescr}

\indexlibrarymember{end_pt}{rel_quadratic_curve}%
\begin{itemdecl}
basic_point_2d<graphics_math_type> end_pt() const noexcept;
\end{itemdecl}
\begin{itemdescr}
\pnum
\returns The end point.
\end{itemdescr}

\rSec1 [\iotwod.relquadraticcurve.eq] {Equality operators}%

\indexlibrarymember{operator==}{rel_quadratic_curve}%
\begin{itemdecl}
template <class GraphicsSurfaces>
bool operator==(
  const typename basic_figure_items<GraphicsSurfaces>::rel_quadratic_curve& lhs,
  const typename basic_figure_items<GraphicsSurfaces>::rel_quadratic_curve& rhs) 
  noexcept;
\end{itemdecl}
\begin{itemdescr}
\pnum
\returns
\tcode{lhs.control_pt() == rhs.control_pt() \&\& lhs.end_pt() == rhs.end_pt()}.
\end{itemdescr}

\indexlibrarymember{operator!=}{rel_quadratic_curve}%
\begin{itemdecl}
template <class GraphicsSurfaces>
bool operator!=(
  const typename basic_figure_items<GraphicsSurfaces>::rel_quadratic_curve& lhs,
  const typename basic_figure_items<GraphicsSurfaces>::rel_quadratic_curve& rhs) 
  noexcept;
\end{itemdecl}
\begin{itemdescr}
\pnum
\returns
\tcode{lhs.control_pt() != rhs.control_pt() || lhs.end_pt() != rhs.end_pt()}.
\end{itemdescr}
