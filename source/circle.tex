%!TEX root = io2d.tex
\rSec0 [\iotwod.circle] {Class \tcode{basic_circle}}

\rSec1 [\iotwod.circle.intro] {\tcode{basic_circle} description}

\pnum
\indexlibrary{\idxcode{circle}}%
The class \tcode{basic_circle} describes a circle.

\pnum
It has a \term{center} of type \tcode{basic_point_2d<GraphicsMath>} and a \term{radius} of type \tcode{float}.

\rSec1 [\iotwod.circle.synopsis] {\tcode{basic_circle} synopsis}

\begin{codeblock}
namespace std::experimental::io2d::v1 {
  template <class GraphicsMath>
  class basic_circle {
  public:
    // \ref{\iotwod.circle.cons}, constructors:
    basic_circle() noexcept;
    basic_circle(const basic_point_2d<GraphicsMath>& ctr, float rad) noexcept;
    basic_circle(const typename GraphicsMath::circle_data_type& val) noexcept;

    // \ref{\iotwod.circle.modifiers}, modifiers:
    void center(const basic_point_2d<GraphicsMath>& ctr) noexcept;
    void radius(float r) noexcept;

    // \ref{\iotwod.circle.observers}, observers:
    basic_point_2d<GraphicsMath> center() const noexcept;
    float radius() const noexcept;
  };

  // \ref{\iotwod.circle.ops}, operators:
  template <class GraphicsMath>
  bool operator==(const basic_circle<GraphicsMath>& lhs,
    const basic_circle<GraphicsMath>& rhs) noexcept;
  template <class GraphicsMath>
  bool operator!=(const basic_circle<GraphicsMath>& lhs,
    const basic_circle<GraphicsMath>& rhs) noexcept;
}
\end{codeblock}

\rSec1 [\iotwod.circle.cons] {\tcode{basic_circle} constructors}

\indexlibrary{\idxcode{basic_circle}!constructor}%
\begin{itemdecl}
basic_circle() noexcept;
\end{itemdecl}
\begin{itemdescr}
\pnum
\effects
Equivalent to: \tcode{basic_circle(\{ 0.0f, 0.0f \}, 0.0f)}.
\end{itemdescr}

\indexlibrary{\idxcode{basic_circle}!constructor}%
\begin{itemdecl}
basic_circle(const basic_point_2d<GraphicsMath>& ctr, float r) noexcept;
\end{itemdecl}
\begin{itemdescr}
\requires
\tcode{r >= 0.0f}.

\pnum
\effects
Constructs an object of type \tcode{basic_circle}.

\pnum
The center is \tcode{ctr}. The radius is \tcode{r}.
\end{itemdescr}

\rSec1 [\iotwod.circle.modifiers]{\tcode{basic_circle} modifiers}

\indexlibrarymember{center}{basic_circle}%
\begin{itemdecl}
void center(const basic_point_2d<GraphicsMath>& ctr) noexcept;
\end{itemdecl}

\begin{itemdescr}
\pnum
\effects
The center is \tcode{ctr}.
\end{itemdescr}

\indexlibrarymember{radius}{basic_circle}%
\begin{itemdecl}
void radius(float r) noexcept;
\end{itemdecl}
\begin{itemdescr}
\requires
\tcode{r >= 0.0f}.

\pnum
\effects
The radius is \tcode{r}.
\end{itemdescr}

\rSec1 [\iotwod.circle.observers]{\tcode{basic_circle} observers}

\indexlibrarymember{center}{basic_circle}%
\begin{itemdecl}
basic_point_2d<GraphicsMath> center() const noexcept;
\end{itemdecl}
\begin{itemdescr}
\pnum
\returns
The center.
\end{itemdescr}

\indexlibrarymember{radius}{basic_circle}%
\begin{itemdecl}
float radius() const noexcept;
\end{itemdecl}
\begin{itemdescr}
\pnum
\returns
The radius.
\end{itemdescr}

\rSec1 [\iotwod.circle.ops] {\tcode{basic_circle} operators}

\indexlibrarymember{operator==}{basic_circle}%
\begin{itemdecl}
bool operator==(const basic_circle<GraphicsMath>& lhs,
  const basic_circle<GraphicsMath>& rhs) noexcept;
\end{itemdecl}
\begin{itemdescr}
\pnum
\returns
\tcode{lhs.center() == rhs.center() \&\& lhs.radius() == rhs.radius();}
\end{itemdescr}
