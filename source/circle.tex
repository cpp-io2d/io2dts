%!TEX root = io2d.tex
\rSec0 [\iotwod.circle] {Class \tcode{circle}}

\rSec1 [\iotwod.circle.intro] {\tcode{circle} description}

\pnum
\indexlibrary{\idxcode{circle}}%
The class \tcode{circle} describes a circle.

\pnum
It has a \term{center} of type \tcode{point_2d} and a \term{radius} of type \tcode{float}.

\rSec1 [\iotwod.circle.synopsis] {\tcode{circle} synopsis}

\begin{codeblock}
namespace std::experimental::io2d::v1 {
  class circle {
  public:
    // \ref{\iotwod.circle.cons}, constructors:
    constexpr circle() noexcept;
    constexpr circle(point_2d ctr, float rad) noexcept;

    // \ref{\iotwod.circle.modifiers}, modifiers:
    constexpr void center(point_2d ctr) noexcept;
    constexpr void radius(float r) noexcept;
    
    // \ref{\iotwod.circle.observers}, observers:
    constexpr point_2d center() const noexcept;
    constexpr float radius() const noexcept;
  };

  // \ref{\iotwod.circle.ops}  
  constexpr bool operator==(const circle& lhs, const circle& rhs) noexcept;
  constexpr bool operator!=(const circle& lhs, const circle& rhs) noexcept;
}
\end{codeblock}

\rSec1 [\iotwod.circle.cons] {\tcode{circle} constructors}

\indexlibrary{\idxcode{circle}!constructor}%
\begin{itemdecl}
constexpr circle() noexcept;
\end{itemdecl}
\begin{itemdescr}
\pnum
\effects
Equivalent to: \tcode{circle(\{ 0.0f, 0.0f \}, 0.0f)}.
\end{itemdescr}

\indexlibrary{\idxcode{circle}!constructor}%
\begin{itemdecl}
constexpr circle(point_2d ctr, float r) noexcept;
\end{itemdecl}
\begin{itemdescr}
\requires
\tcode{r >= 0.0f}.

\pnum
\effects
Constructs an object of type \tcode{circle}.

\pnum
The center is \tcode{ctr}. The radius is \tcode{r}.
\end{itemdescr}

\rSec1 [\iotwod.circle.modifiers]{\tcode{circle} modifiers}

\indexlibrarymember{center}{circle}%
\begin{itemdecl}
constexpr void center(point_2d ctr) noexcept;
\end{itemdecl}

\begin{itemdescr}
\pnum
\effects
The center is \tcode{ctr}.
\end{itemdescr}

\indexlibrarymember{radius}{circle}%
\begin{itemdecl}
constexpr void radius(float r) noexcept;
\end{itemdecl}
\begin{itemdescr}
\requires
\tcode{r >= 0.0f}.

\pnum
\effects
The radius is \tcode{r}.
\end{itemdescr}

\rSec1 [\iotwod.circle.observers]{\tcode{circle} observers}

\indexlibrarymember{center}{circle}%
\begin{itemdecl}
constexpr float center() const noexcept;
\end{itemdecl}
\begin{itemdescr}
\pnum
\returns
The center.
\end{itemdescr}

\indexlibrarymember{radius}{circle}%
\begin{itemdecl}
constexpr float radius() const noexcept;
\end{itemdecl}
\begin{itemdescr}
\pnum
\returns
The radius.
\end{itemdescr}

\rSec1 [\iotwod.circle.ops] {\tcode{circle} operators}

\indexlibrarymember{operator!=}{circle}%
\begin{itemdecl}
constexpr bool operator==(const circle& lhs, const circle& rhs) noexcept;
\end{itemdecl}
\begin{itemdescr}
\pnum
\returns
\tcode{lhs.center() == rhs.center() \&\& lhs.radius() == rhs.radius();}
\end{itemdescr}
