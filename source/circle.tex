%!TEX root = io2d.tex
\rSec0 [circle] {Class \tcode{circle}}

\rSec1 [circle.intro] {\tcode{circle} Description}

\pnum
\indexlibrary{\idxcode{circle}}
The class \tcode{circle} describes a circle.

\pnum
It has a center of type \tcode{vector_2d} and a radius of type \tcode{double}.

\rSec1 [circle.synopsis] {\tcode{circle} synopsis}

\begin{codeblock}
namespace std { namespace experimental { namespace io2d { inline namespace v1 {
  class circle {
  public:
    // \ref{circle.cons}, construct/copy/move/destroy:
    rectangle(const vector_2d& ctr, double rad) noexcept;

    // \ref{circle.modifiers}, modifiers:
    void center(const vector_2d& ctr) noexcept;
    void radius(double rad) noexcept;
    
    // \ref{circle.observers}, observers:
    vector_2d center() const noexcept;
    double radius() const noexcept;
  };
} } } }
\end{codeblock}

\rSec1 [circle.cons] {\tcode{circle} constructors and assignment operators}

\indexlibrary{\idxcode{circle}!constructor}
\begin{itemdecl}
circle(const vector_2d& ctr, double rad) noexcept;
\end{itemdecl}
\begin{itemdescr}
	\pnum
	\effects
	Constructs an object of type \tcode{rectangle}.
	
	\pnum
	The center shall have the value of \tcode{ctr}.
	
	\pnum
	The radius shall have the value of \tcode{rad}.
\end{itemdescr}

\rSec1 [circle.modifiers]{\tcode{circle} modifiers}

\indexlibrary{\idxcode{circle}!\idxcode{center}}
\indexlibrary{\idxcode{center}!\idxcode{circle}}
\begin{itemdecl}
void center(const vector_2d& ctr) noexcept;
\end{itemdecl}

\begin{itemdescr}
	\pnum
	\effects
	The center shall have the value of \tcode{ctr}.
\end{itemdescr}

\indexlibrary{\idxcode{circle}!\idxcode{radius}}
\indexlibrary{\idxcode{radius}!\idxcode{circle}}
\begin{itemdecl}
void radius(double rad) noexcept;
\end{itemdecl}
\begin{itemdescr}
	\pnum
	\effects
	The radius shall have the value of \tcode{rad}.
\end{itemdescr}

\rSec1 [circle.observers]{\tcode{circle} observers}

\indexlibrary{\idxcode{circle}!\idxcode{center}}
\indexlibrary{\idxcode{center}!\idxcode{circle}}
\begin{itemdecl}
double center() const noexcept;
\end{itemdecl}
\begin{itemdescr}
	\pnum
	\returns
	The value of the center.
\end{itemdescr}

\indexlibrary{\idxcode{circle}!\idxcode{radius}}
\indexlibrary{\idxcode{radius}!\idxcode{circle}}
\begin{itemdecl}
double radius() const noexcept;
\end{itemdecl}
\begin{itemdescr}
	\pnum
	\returns
	THe value of the radius.
\end{itemdescr}
