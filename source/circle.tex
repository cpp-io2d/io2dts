%!TEX root = io2d.tex
\rSec0 [\iotwod.circle] {Class \tcode{basic_circle}}

\rSec1 [\iotwod.circle.intro] {\tcode{basic_circle} description}

\pnum
\indexlibrary{\idxcode{basic_circle}}%
The class template \tcode{basic_circle} describes a circle.

\pnum
It has a \term{center} of type \tcode{basic_point_2d<GraphicsMath>} and a \term{radius} of type \tcode{float}.

\pnum
The data are stored in an object of type \tcode{typename GraphicsMath::circle_data_type}. It is accessible using the \tcode{data} member functions.

\rSec1 [\iotwod.circle.synopsis] {\tcode{basic_circle} synopsis}

\begin{codeblock}
namespace @\fullnamespace{}@ {
  template <class GraphicsMath>
  class basic_circle {
  public:
    using data_type = typename GraphicsMath::circle_data_type;
    
    // \ref{\iotwod.circle.cons}, constructors:
    basic_circle() noexcept;
    basic_circle(const basic_point_2d<GraphicsMath>& ctr, float rad) noexcept;
    basic_circle(const typename GraphicsMath::circle_data_type& val) noexcept;

    // \ref{\iotwod.circle.accessors}, accessors:
    const data_type& data() const noexcept;
    data_type& data() noexcept;

    // \ref{\iotwod.circle.modifiers}, modifiers:
    void center(const basic_point_2d<GraphicsMath>& ctr) noexcept;
    void radius(float r) noexcept;

    // \ref{\iotwod.circle.observers}, observers:
    basic_point_2d<GraphicsMath> center() const noexcept;
    float radius() const noexcept;
  };

  // \ref{\iotwod.circle.ops}, operators:
  template <class GraphicsMath>
  bool operator==(const basic_circle<GraphicsMath>& lhs,
    const basic_circle<GraphicsMath>& rhs) noexcept;
  template <class GraphicsMath>
  bool operator!=(const basic_circle<GraphicsMath>& lhs,
    const basic_circle<GraphicsMath>& rhs) noexcept;
}
\end{codeblock}

\rSec1 [\iotwod.circle.cons] {\tcode{basic_circle} constructors}

\indexlibrary{\idxcode{basic_circle}!constructor}%
\begin{itemdecl}
basic_circle() noexcept;
\end{itemdecl}
\begin{itemdescr}
\pnum
\effects
Constructs an object of type \tcode{basic_circle}.

\pnum
\postconditions
\tcode{data() == GraphicsMath::create_circle()}.
\end{itemdescr}

\indexlibrary{\idxcode{basic_circle}!constructor}%
\begin{itemdecl}
basic_circle(const basic_point_2d<GraphicsMath>& ctr, float r) noexcept;
\end{itemdecl}
\begin{itemdescr}
\requires
\tcode{r >= 0.0f}.

\pnum
\effects
Constructs an object of type \tcode{basic_circle}.

\pnum
\postconditions
\tcode{data() == GraphicsMath::create_circle(ctr, r)}.
\end{itemdescr}

\rSec1 [\iotwod.circle.accessors]{\tcode{basic_circle} accessors}

\indexlibrarymember{data}{basic_circle}%
\begin{itemdecl}
const data_type& data() const noexcept;
data_type& data() noexcept;
\end{itemdecl}
\begin{itemdescr}
\pnum
\returns
A reference to the \tcode{basic_circle} object's data object (See: \ref{\iotwod.circle.intro}).
\end{itemdescr}

\rSec1 [\iotwod.circle.modifiers]{\tcode{basic_circle} modifiers}

\indexlibrarymember{center}{basic_circle}%
\begin{itemdecl}
void center(const basic_point_2d<GraphicsMath>& ctr) noexcept;
\end{itemdecl}

\begin{itemdescr}
\pnum
\effects
Equivalent to \tcode{GraphicsMath::center(data(), ctr.data());}
\end{itemdescr}

\indexlibrarymember{radius}{basic_circle}%
\begin{itemdecl}
void radius(float r) noexcept;
\end{itemdecl}
\begin{itemdescr}
\requires
\tcode{r >= 0.0f}.

\pnum
\effects
Equivalent to \tcode{GraphicsMath::radius(data(), r);}
\end{itemdescr}

\rSec1 [\iotwod.circle.observers]{\tcode{basic_circle} observers}

\indexlibrarymember{center}{basic_circle}%
\begin{itemdecl}
basic_point_2d<GraphicsMath> center() const noexcept;
\end{itemdecl}
\begin{itemdescr}
\pnum
\returns
\tcode{(basic_point_2d<GraphicsMath>(GraphicsMath::center(data()))}.
\end{itemdescr}

\indexlibrarymember{radius}{basic_circle}%
\begin{itemdecl}
float radius() const noexcept;
\end{itemdecl}
\begin{itemdescr}
\pnum
\returns
\tcode{GraphicsMath::radius(data())}.
\end{itemdescr}

\rSec1 [\iotwod.circle.ops] {\tcode{basic_circle} operators}

\indexlibrarymember{operator==}{basic_circle}%
\begin{itemdecl}
bool operator==(const basic_circle<GraphicsMath>& lhs,
  const basic_circle<GraphicsMath>& rhs) noexcept;
\end{itemdecl}
\begin{itemdescr}
\pnum
\returns
\tcode{GraphicsMath::equal(lhs.data(), rhs.data())}.
\end{itemdescr}

\indexlibrarymember{operator!=}{basic_circle}%
\begin{itemdecl}
bool operator!=(const basic_circle<GraphicsMath>& lhs,
  const basic_circle<GraphicsMath>& rhs) noexcept;
\end{itemdecl}
\begin{itemdescr}
\pnum
\returns
\tcode{GraphicsMath::not_equal(lhs.data(), rhs.data())}.
\end{itemdescr}
