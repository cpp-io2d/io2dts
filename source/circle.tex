%!TEX root = io2d.tex
\rSec0 [circle] {Class \tcode{circle}}

\rSec1 [circle.intro] {\tcode{circle} Description}

\pnum
\indexlibrary{\idxcode{circle}}
The class \tcode{circle} describes a circle.

\pnum
It has a Center of type \tcode{vector_2d} and a Radius of type \tcode{double}.

\rSec1 [circle.synopsis] {\tcode{circle} synopsis}

\begin{codeblock}
namespace std { namespace experimental { namespace io2d { inline namespace v1 {
  class circle {
  public:
    // \ref{circle.cons}, construct/copy/move/destroy:
    constexpr circle() noexcept;
    contexpr circle(const vector_2d& ctr, double rad) noexcept;
    constexpr circle(const circle&) noexcept = default;
    constexpr circle& operator=(const circle&) noexcept = default;
    circle(circle&&) noexcept = default;
    circle& operator=(circle&&) noexcept = default;

    // \ref{circle.modifiers}, modifiers:
    void center(const vector_2d& ctr) noexcept;
    void radius(double rad) noexcept;
    
    // \ref{circle.observers}, observers:
    constexpr vector_2d center() const noexcept;
    constexpr double radius() const noexcept;
  };
} } } }
\end{codeblock}

\rSec1 [circle.cons] {\tcode{circle} constructors and assignment operators}

\indexlibrary{\idxcode{circle}!constructor}
\begin{itemdecl}
constexpr circle() noexcept;
\end{itemdecl}
\begin{itemdescr}
\pnum
\effects
Constructs an object of type \tcode{circle}.

\pnum
The Center shall have the value of \tcode{vector_2d\{0,0, 0.0\}}.

\pnum
The Radius shall have the value of \tcode{rad}.
\end{itemdescr}

\indexlibrary{\idxcode{circle}!constructor}
\begin{itemdecl}
constexpr circle(const vector_2d& ctr, double rad) noexcept;
\end{itemdecl}
\begin{itemdescr}
\pnum
\effects
Constructs an object of type \tcode{circle}.

\pnum
The Center shall have the value of \tcode{ctr}.

\pnum
The Radius shall have the value of \tcode{rad}.
\end{itemdescr}

\rSec1 [circle.modifiers]{\tcode{circle} modifiers}

\indexlibrary{\idxcode{circle}!\idxcode{center}}
\indexlibrary{\idxcode{center}!\idxcode{circle}}
\begin{itemdecl}
void center(const vector_2d& ctr) noexcept;
\end{itemdecl}

\begin{itemdescr}
\pnum
\effects
The Center shall have the value of \tcode{ctr}.
\end{itemdescr}

\indexlibrary{\idxcode{circle}!\idxcode{radius}}
\indexlibrary{\idxcode{radius}!\idxcode{circle}}
\begin{itemdecl}
void radius(double rad) noexcept;
\end{itemdecl}
\begin{itemdescr}
\pnum
\effects
The Radius shall have the value of \tcode{rad}.
\end{itemdescr}

\rSec1 [circle.observers]{\tcode{circle} observers}

\indexlibrary{\idxcode{circle}!\idxcode{center}}
\indexlibrary{\idxcode{center}!\idxcode{circle}}
\begin{itemdecl}
constexpr double center() const noexcept;
\end{itemdecl}
\begin{itemdescr}
\pnum
\returns
The value of the Center.
\end{itemdescr}

\indexlibrary{\idxcode{circle}!\idxcode{radius}}
\indexlibrary{\idxcode{radius}!\idxcode{circle}}
\begin{itemdecl}
constexpr double radius() const noexcept;
\end{itemdecl}
\begin{itemdescr}
\pnum
\returns
THe value of the Radius.
\end{itemdescr}
