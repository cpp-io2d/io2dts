%!TEX root = io2d.tex
\rSec0 [pathdataitem.curveto] {Class \tcode{path_data_item::curve_to}}

\pnum
\indexlibrary{\idxcode{path_data_item::curve_to}}
The class \tcode{path_data_item::curve_to} describes an operation on a path geometry collection.

\pnum
If the current path geometry has no current point, then this operation behaves exactly as if this object was preceded by a \tcode{path_data_item::move_to} object constructed with the value returned by \tcode{*this.control_point_1()} as its argument.

\pnum
This operation creates a cubic B\'ezier curve from the current point to the point returned by \tcode{*this.end_point()}, with the first control point being the point returned by \tcode{*this.control_point_1()} and the second control point being the point returned by \tcode{*this.control_point_2()}. It then sets the current point to be the point returned by \tcode{*this.end_point()}.

\rSec1 [pathdataitem.curveto.synopsis] {\tcode{path_data_item::curve_to} synopsis}

\begin{codeblock}
namespace std { namespace experimental { namespace io2d { inline namespace v1 {
  class path_data_item::curve_to {
  public:
    // \ref{pathdataitem.curveto.cons}, construct/copy/move/destroy:
    curve_to() noexcept;
    curve_to(const curve_to&) noexcept;
    path_data_item::curve_to& operator=(const curve_to&) noexcept;
    curve_to(curve_to&&) noexcept;
    path_data_item::curve_to& operator=(curve_to&&) noexcept;
    curve_to(const vector_2d& controlPoint1, const vector_2d& controlPoint2,
      const vector_2d& endPoint) noexcept;

    // \ref{pathdataitem.curveto.modifiers}, modifiers:
    void control_point_1(const vector_2d& value) noexcept;
    void control_point_2(const vector_2d& value) noexcept;
    void end_point(const vector_2d& value) noexcept;


    // \ref{pathdataitem.curveto.observers}, observers:
    vector_2d control_point_1() const noexcept;
    vector_2d control_point_2() const noexcept;
    vector_2d end_point() const noexcept;
    virtual path_data_type type() const noexcept override;
    
  private:
    vector_2d _Control_pt1; // \expos
    vector_2d _Control_pt2; // \expos
    vector_2d _End_pt;      // \expos
  };
} } } }
\end{codeblock}

\rSec1 [pathdataitem.curveto.cons] {\tcode{path_data_item::curve_to} constructors and assignment operators}

\indexlibrary{\idxcode{path_data_item::curve_to}!constructor}
\begin{itemdecl}
    curve_to() noexcept;
\end{itemdecl}
\begin{itemdescr}
	\pnum
	\effects
	Constructs an object of type \tcode{path_data_item::curve_to}.
	
	\pnum
	\postconditions
	\tcode{_Control_pt1 == vector_2d(0.0, 0.0)}.

	\tcode{_Control_pt2 == vector_2d(0.0, 0.0)}.

	\tcode{_End_pt == vector_2d(0.0, 0.0)}.
\end{itemdescr}

\indexlibrary{\idxcode{path_data_item::curve_to}!constructor}
\begin{itemdecl}
    curve_to(const vector_2d& controlPoint1, const vector_2d& controlPoint2,
      const vector_2d& endPoint) noexcept;
\end{itemdecl}
\begin{itemdescr}
	\pnum
	\effects
	Constructs an object of type \tcode{path_data_item::curve_to}.
	
	\pnum
	\postconditions
	\tcode{_Control_pt1 == controlPoint1}.

	\tcode{_Control_pt2 == controlPoint2}.

	\tcode{_End_pt == endPoint}.
\end{itemdescr}

\rSec1 [pathdataitem.curveto.modifiers]{\tcode{path_data_item::curve_to} modifiers}

\indexlibrary{\idxcode{path_data_item::curve_to}!\idxcode{control_point_1}}
\indexlibrary{\idxcode{control_point_1}!\idxcode{path_data_item::curve_to}}
\begin{itemdecl}
    void control_point_1(const vector_2d& value) noexcept;
\end{itemdecl}
\begin{itemdescr}
	\pnum
	\postconditions
	\tcode{_Control_pt_1 == value}.
\end{itemdescr}

\indexlibrary{\idxcode{path_data_item::curve_to}!\idxcode{control_point_2}}
\indexlibrary{\idxcode{control_point_2}!\idxcode{path_data_item::curve_to}}
\begin{itemdecl}
    void control_point_2(const vector_2d& value) noexcept;
\end{itemdecl}
\begin{itemdescr}
	\pnum
	\postconditions
	\tcode{_Control_pt_2 == value}.
\end{itemdescr}

\indexlibrary{\idxcode{path_data_item::curve_to}!\idxcode{end_point}}
\indexlibrary{\idxcode{end_point}!\idxcode{path_data_item::curve_to}}
\begin{itemdecl}
    void end_point(const vector_2d& value) noexcept;
\end{itemdecl}
\begin{itemdescr}
	\pnum
	\postconditions
	\tcode{_End_pt == value}.
\end{itemdescr}

\rSec1 [pathdataitem.curveto.observers]{\tcode{path_data_item::curve_to} observers}

\indexlibrary{\idxcode{path_data_item::curve_to}!\idxcode{control_point_1}}
\indexlibrary{\idxcode{control_point_1}!\idxcode{path_data_item::curve_to}}
\begin{itemdecl}
    vector_2d control_point_1() const noexcept;
\end{itemdecl}
\begin{itemdescr}
	\pnum
	\returns
	\tcode{_Control_pt_1}.
\end{itemdescr}

\indexlibrary{\idxcode{path_data_item::curve_to}!\idxcode{control_point_2}}
\indexlibrary{\idxcode{control_point_2}!\idxcode{path_data_item::curve_to}}
\begin{itemdecl}
    vector_2d control_point_2() const noexcept;
\end{itemdecl}
\begin{itemdescr}
	\pnum
	\returns
	\tcode{_Control_pt_2}.
\end{itemdescr}

\indexlibrary{\idxcode{path_data_item::curve_to}!\idxcode{end_point}}
\indexlibrary{\idxcode{end_point}!\idxcode{path_data_item::curve_to}}
\begin{itemdecl}
    vector_2d end_point() const noexcept;
\end{itemdecl}
\begin{itemdescr}
	\pnum
	\returns
	\tcode{_End_pt}.
\end{itemdescr}

\indexlibrary{\idxcode{path_data_item::curve_to}!\idxcode{type}}
\indexlibrary{\idxcode{type}!\idxcode{path_data_item::curve_to}}
\begin{itemdecl}
    virtual path_data_type type() const noexcept override;
\end{itemdecl}
\begin{itemdescr}
	\pnum
	\returns
	\tcode{path_data_type::curve_to}.
\end{itemdescr}
