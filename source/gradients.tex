%!TEX root = io2d.tex
\rSec0 [gradients] {Gradient brushes}

\rSec1 [gradients.colorstops] {Color stops}

\pnum
A gradient has as part of its observable state a collection of color stops.

\pnum
The collection of color stops has an \impldef{color stop!maximum size} maximum size.

\pnum
Associated with the collection of color stops is a type called the \term{size_type}, which is an \impldef{color stop!size_type} unsigned integer type that is large enough that it can serve as an index to address each color stop in any collection of color stops as if the collection of color stops was accessed as an array of color stops.

\pnum
The \term{size} of a collection of color stops is the number of color stops it currently contains; its type is size_type.

\pnum
When a color stop is added to a collection of color stops it is assigned an \term{index value} that is equal to the size of the collection of color stops before the color stop is added.

\pnum
A \term{valid index value} for a collection of color stops is an unsigned integer value in the range $(0,size]$.

\pnum
Color stops in a collection of color stops can be replaced.

\pnum
A color stop which replaces another color stop in a collection of color stops is assigned the index value of the color stop it replaced.
\enterexample
If 3 color stops are added to a collection of color stops and then the color stop at index 1 is replaced, the next color stop which is added will be assigned the index value 3.
\exitexample

\pnum
Color stops in a collection of color stops can be removed.

\pnum
When a color stop with the valid index value $i$ is removed from a collection of color stops, the index value of every color stop in the collection of color stops which had an index value greater than $i$ has its index value decremented by 1.
\enternote
The size of the collection of color stops is reduced by 1 as the result of the removal of one of its color stops.
\exitnote

\rSec1 [gradients.linear] {Linear gradients}

\pnum
A linear gradient is a type of gradient.

\pnum
In addition to the observable state of a gradient, a linear gradient also has a \term{begin point} and an \term{end point} as parts of its observable state, both of which are objects of type \tcode{vector_2d}.

\pnum
A linear gradient for which the distance between its begin point and its end point is not greater than \tcode{numeric_limits<double>::epsilon()} is a \term{degenerate linear gradient}.

\pnum
All attempts to sample from a \tcode{brush} object created using a degenerate linear gradient shall return the color \tcode{rgba_color::transparent_black()}. The remainder of \ref{gradients.linear} is inapplicable to degenerate linear gradients.

\pnum
The begin point and end point of a linear gradient define a line segment, with a color stop offset value of 0.0 corresponding to the begin point and a color stop offset value of 1.0 corresponding to the end point.

\pnum
Color stop offset values in the range $(0,1)$ linearly correspond to points on the line segment.

\pnum
\enterexample
Given a linear gradient with a begin point of \tcode{vector_2d(0.0, 0.0)} and an end point of \tcode{vector_2d(10.0, 5.0)}, a color stop offset value of 0.6 would correspond to the point \tcode{vector_2d(6.0, 3.0)}.
\exitexample

\pnum
To determine the offset value of a point $p$ for a linear gradient, perform the following steps:
\begin{enumeratea}
\item Create a line at the begin point of the linear gradient, the \term{begin line}, and another line at the end point of the linear gradient, the \term{end line}, with each line being perpendicular to the \term{gradient line segment}, which is the line segment delineated by the begin point and the end point.

\item Using the begin line, $p$, and the end line, create a line, the \term{$p$ line}, which is parallel to the gradient line segment.

\item Defining $dp$ as the distance between $p$ and the point where the $p$ line intersects the begin line and $dt$ as the distance between the point where the $p$ line intersects the begin line and the point where the $p$ line intersects the end line, the offset value of $p$ is $dp \div dt$.

\item The offset value shall be negative if
\begin{itemize}
\item $p$ is not on the line segment delineated by the point where the $p$ line intersects the begin line and the point where the $p$ line intersects the end line; and,

\item the distance between $p$ and the point where the $p$ line intersects the begin line is less than the distance between $p$ and the point where the $p$ line intersects the end line.
\end{itemize}
\end{enumeratea}

\rSec1 [gradients.radial] {Radial gradients}

\pnum
A radial gradient is a type of gradient.

\pnum
In addition to the observable state of a gradient, a radial gradient also has a \term{start circle} and an \term{end circle} as part of its observable state, each of which is defined by a \tcode{vector_2d} object that denotes its center and a \tcode{double} value that denotes its radius.

\pnum
A radial gradient is a \term{degenerate radial gradient} if:
\begin{itemize}
\item its start circle has a negative radius; or,
\item its end circle has a negative radius; or,
\item the distance between the center point of its start circle and the center point of its end circle is not greater than \tcode{numeric_limits<double>::epsilon()} and the difference between the radius of its start circle and the radius of its end circle is not greater than \tcode{numeric_limits<double>::epsilon()}; or,
\item its start circle has a radius of 0.0 and its end circle has a radius of 0.0.
\end{itemize}

\pnum
All attempts to sample from a \tcode{brush} object created using a degenerate radial gradient shall return the color \tcode{rgba_color::transparent_black()}. The remainder of \ref{gradients.radial} is inapplicable to degenerate radial gradients.

\pnum
A color stop offset of 0.0 corresponds to all points along the diameter of the start circle or to its center point if it has a radius value of 0.0.

\pnum
A color stop offset of 1.0 corresponds to all points along the diameter of the end circle or to its center point if it has a radius value of 0.0.

\pnum
A radial gradient shall be rendered as a continuous series of interpolated circles defined by the following equations:
\begin{enumeratea}
\item $x(o) = x_{start} + o \times (x_{end} - x_{start})$
\item $y(o) = y_{start} + o \times (y_{end} - y_{start})$
\item $radius(o) = radius_{start} + o \times (radius_{end} - radius_{start})$
\end{enumeratea}
where $o$ is a color stop offset value.

\pnum
The range of potential values for $o$ shall be determined by the \tcode{extend} value of the \tcode{brush} object created using the radial gradient:
\begin{itemize}
\item For \tcode{extend::none}, the range of potential values for $o$ is $[0,1]$.
\item For all other \tcode{extend} values, the range of potential values for $o$ is\\ $[$~\tcode{numeric_limits<double>::lowest(),numeric_limits<double>::max()}~$]$.
\end{itemize}

\pnum
The interpolated circles shall be rendered starting from the smallest potential value of $o$.

\pnum
An interpolated circle shall not be rendered if its value for $o$ results in $radius(o)$ evaluating to a negative value.

\rSec1 [gradients.sampling] {Sampling from gradients}

\pnum
For any offset value $o$, its color value shall be determined according to the following rules:

\begin{enumeratea}
\item If there are less than two color stops or if all color stops have the same offset value, then the color value of every offset value shall be \tcode{rgba_color::transparent_black()} and the remainder of \ref{gradients.sampling} is inapplicable.

\item If exactly one color stop has an offset value equal to $o$, $o$'s color value shall be the color value of that color stop and the remainder of \ref{gradients.sampling} is inapplicable.

\item If two or more color stops have an offset value equal to $o$, $o$'s color value shall be the color value of the color stop which has the lowest index value among the set of color stops that have an offset value equal to $o$ and the remainder of \ref{gradients.sampling} is inapplicable.

\item When no color stop has the offset value of \tcode{0.0}, then, defining $n$ to be the offset value that is nearest to \tcode{0.0} among the offset values in the set of all color stops, if $o$ is in the offset range $[0, n)$, $o$'s color value shall be \tcode{rgba_color::transparent_black()} and the remainder of \ref{gradients.sampling} is inapplicable.
\enternote
Since the range described does not include $n$, it does not matter how many color stops have $n$ as their offset value for purposes of this rule.
\exitnote

\item When no color stop has the offset value of \tcode{1.0}, then, defining $n$ to be the offset value that is nearest to \tcode{1.0} among the offset values in the set of all color stops, if $o$ is in the offset range $(n, 1]$, $o$'s color value shall be \tcode{rgba_color::transparent_black()} and the remainder of \ref{gradients.sampling} is inapplicable.
\enternote
Since the range described does not include $n$, it does not matter how many color stops have $n$ as their offset value for purposes of this rule.
\exitnote

\item Each color stop has, at most, two adjacent color stops: one to its left and one to its right.

\item Adjacency of color stops is initially determined by offset values. If two or more color stops have the same offset value then index values are used to determine adjacency as described below.

\item For each color stop \textit{a}, the \term{set of color stops to its left} are those color stops which have an offset value which is closer to \tcode{0.0} than \textit{a}'s offset value.
\enternote
This includes any color stops with an offset value of \tcode{0.0} provided that \textit{a}'s offset value is not \tcode{0.0}.
\exitnote

\item For each color stop \textit{b}, the \term{set of color stops to its right} are those color stops which have an offset value which is closer to \tcode{1.0} than \textit{b}'s offset value.
\enternote
This includes any color stops with an offset value of \tcode{1.0} provided that \textit{b}'s offset value is not \tcode{1.0}.
\exitnote

\item A color stop which has an offset value of \tcode{0.0} does not have an adjacent color stop to its left.

\item A color stop which has an offset value of \tcode{1.0} does not have an adjacent color stop to its right.

\item If a color stop \textit{a}'s set of color stops to its left consists of exactly one color stop, that color stop is the color stop that is adjacent to \textit{a} on its left.

\item If a color stop \textit{b}'s set of color stops to its right consists of exactly one color stop, that color stop is the color stop that is adjacent to \textit{b} on its right.

\item If two or more color stops have the same offset value then the color stop with the lowest index value is the only color stop from that set of color stops which can have a color stop that is adjacent to it on its left and the color stop with the highest index value is the only color stop from that set of color stops which can have a color stop that is adjacent to it on its right. This rule takes precedence over all of the remaining rules.

\item If a color stop can have an adjacent color stop to its left, then the color stop which is adjacent to it to its left is the color stop from the set of color stops to its left which has an offset value which is closest to its offset value. If two or more color stops meet that criteria, then the color stop which is adjacent to it to its left is the color stop which has the highest index value from the set of color stops to its left which are tied for being closest to its offset value.

\item If a color stop can have an adjacent color stop to its right, then the color stop which is adjacent to it to its right is the color stop from the set of color stops to its right which has an offset value which is closest to its offset value. If two or more color stops meet that criteria, then the color stop which is adjacent to it to its right is the color stop which has the lowest index value from the set of color stops to its right which are tied for being closest to its offset value.

\item Where the value of $o$ is in the range $[0,1]$, its color value shall be determined by interpolating between the color stop, $r$, which is the color stop whose offset value is closest to $o$ without being less than $o$ and which can have an adjacent color stop to its left, and the color stop that is adjacent to $r$ on $r$'s left. The acceptable forms of interpolating between color values is set forth later in this section.

\item Where the value of $o$ is outside the range $[0,1]$, its color value depends on the \tcode{extend} value of the brush which is created using the gradient:
	\begin{itemize}
	\item If the \tcode{extend} value is \tcode{extend::none}, the color value of $o$ shall be \tcode{rgba_color::transparent_black()}.
	
	\item If the \tcode{extend} value is \tcode{extend::pad}, if $o$ is negative then the color value of $o$ shall be the same as if the value of $o$ was \tcode{0.0}, otherwise the color value of $o$ shall be the same as if the value of $o$ was \tcode{1.0}.
	
	\item If the \tcode{extend} value is \tcode{extend::repeat}, then \tcode{1.0} shall be added to or subtracted from $o$ until $o$ is in the range $[0,1]$, at which point its color value is the color value for the modified value of $o$ as determined by these rules.
	\enterexample
	Given $o == 2.1$, after application of this rule $o == 0.1$ and the color value of $o$ shall be the same value as if the initial value of $o$ was $0.1$.
	
	Given $o == -0.3$, after application of this rule $o == 0.7$ and the color value of $o$ shall be the same as if the initial value of $o$ was $0.7$.
	\exitexample
	
	\item If the \tcode{extend} value is \tcode{extend::reflect}, $o$ shall be set to the absolute value of $o$, then 2.0 shall be subtracted from $o$ until $o$ is in the range $[0,2]$, then if $o$ is in the range $(1,2]$ then $o$ shall be set to \tcode{1.0 - (o - 1.0)}, at which point its color value is the color value for the modified value of $o$ as determined by these rules.
	\enterexample
	Given $o == 2.8$, after application of this rule $o == 0.8$ and the color value of $o$ shall be the same value as if the initial value of $o$ was $0.8$.
	
	Given $o == 3.6$, after application of this rule $o == 0.4$ and the color value of $o$ shall be the same value as if the initial value of $o$ was $0.4$.
	
	Given $o == -0.3$, after application of this rule $o == 0.3$ and the color value of $o$ shall be the same as if the initial value of $o$ was $0.3$.
	
	Given $o == -5.8$, after application of this rule $o == 0.2$ and the color value of $o$ shall be the same as if the initial value of $o$ was $0.2$.
	\exitexample
	\end{itemize}
\end{enumeratea}

\pnum
It is unspecified whether the interpolation between the color values of two adjacent color stops is performed linearly on each color channel, is performed within an RGB color space (with or without gamma correction), or is performed by a linear color interpolation algorithm implemented in hardware (typically in a graphics processing unit).

\pnum
Implementations shall interpolate between alpha channel values of adjacent color stops linearly except as provided in the following paragraph.

\pnum
A conforming implementation may use the alpha channel interpolation results from a linear color interpolation algorithm implemented in hardware even if those results differ from the results required by the previous paragraph.
