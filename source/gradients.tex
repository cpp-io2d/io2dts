%!TEX root = io2d.tex
\rSec0 [\iotwod.gradients] {Gradient brushes}

\rSec1 [\iotwod.gradients.common] {Common properties of gradients}
\pnum
Gradients are formed, in part, from a collection of \tcode{gradient_stop} objects.

\pnum
The collection of \tcode{gradient_stop} objects contribute to defining a brush which, when sampled from, returns a value that is interpolated based on those gradient stops.

\rSec1 [\iotwod.gradients.linear] {Linear gradients}

\pnum
A linear gradient is a type of gradient.

\pnum
A linear gradient has a \term{begin point} and an \term{end point}, each of which are objects of type \tcode{point_2d}.

\pnum
A linear gradient for which the distance between its begin point and its end point is not greater than \tcode{numeric_limits<float>::epsilon()} is a \term{degenerate linear gradient}.

\pnum
All attempts to sample from a a degenerate linear gradient return the color \tcode{rgba_color::transparent_black}. The remainder of \ref{\iotwod.gradients} is inapplicable to degenerate linear gradients.

\pnum
The begin point and end point of a linear gradient define a line segment, with a gradient stop offset value of 0.0f corresponding to the begin point and a gradient stop offset value of 1.0f corresponding to the end point.

\pnum
Gradient stop offset values in the range \crange{0.0f}{1.0f} linearly correspond to points on the line segment.

\pnum
\begin{example}
Given a linear gradient with a begin point of \tcode{point_2d(0.0f, 0.0f)} and an end point of \tcode{point_2d(10.0f, 5.0f)}, a gradient stop offset value of 0.6f would correspond to the point \tcode{point_2d(6.0f, 3.0f)}.
\end{example}

\pnum
To determine the offset value of a point $p$ for a linear gradient, perform the following steps:
\begin{enumeratea}
\item Create a line at the begin point of the linear gradient, the \term{begin line}, and another line at the end point of the linear gradient, the \term{end line}, with each line being perpendicular to the \term{gradient line segment}, which is the line segment delineated by the begin point and the end point.

\item Using the begin line, $p$, and the end line, create a line, the \term{$p$ line}, which is parallel to the gradient line segment.

\item Defining $dp$ as the distance between $p$ and the point where the $p$ line intersects the begin line and $dt$ as the distance between the point where the $p$ line intersects the begin line and the point where the $p$ line intersects the end line, the offset value of $p$ is $dp \div dt$.

\item The offset value shall be negative if
\begin{itemize}
\item $p$ is not on the line segment delineated by the point where the $p$ line intersects the begin line and the point where the $p$ line intersects the end line; and,

\item the distance between $p$ and the point where the $p$ line intersects the begin line is less than the distance between $p$ and the point where the $p$ line intersects the end line.
\end{itemize}
\end{enumeratea}

\rSec1 [\iotwod.gradients.radial] {Radial gradients}

\pnum
A radial gradient is a type of gradient.

\pnum
Aa radial gradient has a \term{start circle} and an \term{end circle}, each of which is defined by a \tcode{circle} object.

\pnum
A radial gradient is a \term{degenerate radial gradient} if:
\begin{itemize}
\item its start circle has a negative radius; or,
\item its end circle has a negative radius; or,
\item the distance between the center point of its start circle and the center point of its end circle is not greater than \tcode{numeric_limits<float>::epsilon()} and the difference between the radius of its start circle and the radius of its end circle is not greater than \tcode{numeric_limits<float>::epsilon()}; or,
\item its start circle has a radius of 0.0f and its end circle has a radius of 0.0f.
\end{itemize}

\pnum
All attempts to sample from a \tcode{brush} object created using a degenerate radial gradient return the color \tcode{rgba_color::transparent_black}. The remainder of \ref{\iotwod.gradients} is inapplicable to degenerate radial gradients.

\pnum
A gradient stop offset of 0.0f corresponds to all points along the diameter of the start circle or to its center point if it has a radius value of 0.0f.

\pnum
A gradient stop offset of 1.0f corresponds to all points along the diameter of the end circle or to its center point if it has a radius value of 0.0f.

\pnum
A radial gradient shall be rendered as a continuous series of interpolated circles defined by the following equations:
\begin{enumeratea}
\item $x(o) = x_{start} + o \times (x_{end} - x_{start})$
\item $y(o) = y_{start} + o \times (y_{end} - y_{start})$
\item $radius(o) = radius_{start} + o \times (radius_{end} - radius_{start})$
\end{enumeratea}
where $o$ is a gradient stop offset value.

\pnum
The range of potential values for $o$ shall be determined by the \term{wrap mode} (\ref{\iotwod.brushes.intro}):
\begin{itemize}
\item For \tcode{wrap_mode::none}, the range of potential values for $o$ is \crange{0}{1}.
\item For all other \tcode{wrap_mode} values, the range of potential values for $o$ is\\ $[$~\tcode{numeric_limits<float>::lowest(),numeric_limits<float>::max()}~$]$.
\end{itemize}

\pnum
The interpolated circles shall be rendered starting from the smallest potential value of $o$.

\pnum
An interpolated circle shall not be rendered if its value for $o$ results in $radius(o)$ evaluating to a negative value.

\rSec1 [\iotwod.gradients.sampling] {Sampling from gradients}

\pnum
For any offset value $o$, its color value shall be determined according to the following rules:

\begin{enumeratea}
\item If there are less than two gradient stops or if all gradient stops have the same offset value, then the color value of every offset value shall be \tcode{rgba_color::transparent_black} and the remainder of these rules are inapplicable.

\item If exactly one gradient stop has an offset value equal to $o$, $o$'s color value shall be the color value of that gradient stop and the remainder of these rules are inapplicable.

\item If two or more gradient stops have an offset value equal to $o$, $o$'s color value shall be the color value of the gradient stop which has the lowest index value among the set of gradient stops that have an offset value equal to $o$ and the remainder of \ref{\iotwod.gradients.sampling} is inapplicable.

\item When no gradient stop has the offset value of \tcode{0.0f}, then, defining $n$ to be the offset value that is nearest to \tcode{0.0f} among the offset values in the set of all gradient stops, if $o$ is in the offset range $[0, n)$, $o$'s color value shall be \tcode{rgba_color::transparent_black} and the remainder of these rules are inapplicable.
\begin{note}
Since the range described does not include $n$, it does not matter how many gradient stops have $n$ as their offset value for purposes of this rule.
\end{note}

\item When no gradient stop has the offset value of \tcode{1.0f}, then, defining $n$ to be the offset value that is nearest to \tcode{1.0f} among the offset values in the set of all gradient stops, if $o$ is in the offset range $(n, 1]$, $o$'s color value shall be \tcode{rgba_color::transparent_black} and the remainder of these rules are inapplicable.
\begin{note}
Since the range described does not include $n$, it does not matter how many gradient stops have $n$ as their offset value for purposes of this rule.
\end{note}

\item Each gradient stop has, at most, two adjacent gradient stops: one to its left and one to its right.

\item Adjacency of gradient stops is initially determined by offset values. If two or more gradient stops have the same offset value then index values are used to determine adjacency as described below.

\item For each gradient stop \textit{a}, the \term{set of gradient stops to its left} are those gradient stops which have an offset value which is closer to \tcode{0.0f} than \textit{a}'s offset value.
\begin{note}
This includes any gradient stops with an offset value of \tcode{0.0f} provided that \textit{a}'s offset value is not \tcode{0.0f}.
\end{note}

\item For each gradient stop \textit{b}, the \term{set of gradient stops to its right} are those gradient stops which have an offset value which is closer to \tcode{1.0f} than \textit{b}'s offset value.
\begin{note}
This includes any gradient stops with an offset value of \tcode{1.0f} provided that \textit{b}'s offset value is not \tcode{1.0f}.
\end{note}

\item A gradient stop which has an offset value of \tcode{0.0f} does not have an adjacent gradient stop to its left.

\item A gradient stop which has an offset value of \tcode{1.0f} does not have an adjacent gradient stop to its right.

\item If a gradient stop \textit{a}'s set of gradient stops to its left consists of exactly one gradient stop, that gradient stop is the gradient stop that is adjacent to \textit{a} on its left.

\item If a gradient stop \textit{b}'s set of gradient stops to its right consists of exactly one gradient stop, that gradient stop is the gradient stop that is adjacent to \textit{b} on its right.

\item If two or more gradient stops have the same offset value then the gradient stop with the lowest index value is the only gradient stop from that set of gradient stops which can have a gradient stop that is adjacent to it on its left and the gradient stop with the highest index value is the only gradient stop from that set of gradient stops which can have a gradient stop that is adjacent to it on its right. This rule takes precedence over all of the remaining rules.

\item If a gradient stop can have an adjacent gradient stop to its left, then the gradient stop which is adjacent to it to its left is the gradient stop from the set of gradient stops to its left which has an offset value which is closest to its offset value. If two or more gradient stops meet that criteria, then the gradient stop which is adjacent to it to its left is the gradient stop which has the highest index value from the set of gradient stops to its left which are tied for being closest to its offset value.

\item If a gradient stop can have an adjacent gradient stop to its right, then the gradient stop which is adjacent to it to its right is the gradient stop from the set of gradient stops to its right which has an offset value which is closest to its offset value. If two or more gradient stops meet that criteria, then the gradient stop which is adjacent to it to its right is the gradient stop which has the lowest index value from the set of gradient stops to its right which are tied for being closest to its offset value.

\item Where the value of $o$ is in the range $[0,1]$, its color value shall be determined by interpolating between the gradient stop, $r$, which is the gradient stop whose offset value is closest to $o$ without being less than $o$ and which can have an adjacent gradient stop to its left, and the gradient stop that is adjacent to $r$ on $r$'s left. The acceptable forms of interpolating between color values is set forth later in this section.

\item Where the value of $o$ is outside the range $[0,1]$, its color value depends on the value of wrap mode:
	\begin{itemize}
	\item If wrap mode is \tcode{wrap_mode::none}, the color value of $o$ shall be \tcode{rgba_color::transparent_black}.
	
	\item If wrap mode is \tcode{wrap_mode::pad}, if $o$ is negative then the color value of $o$ shall be the same as-if the value of $o$ was \tcode{0.0f}, otherwise the color value of $o$ shall be the same as-if the value of $o$ was \tcode{1.0f}.
	
	\item If wrap mode is \tcode{wrap_mode::repeat}, then \tcode{1.0f} shall be added to or subtracted from $o$ until $o$ is in the range $[0,1]$, at which point its color value is the color value for the modified value of $o$ as determined by these rules.
	\begin{example}
	Given $o == 2.1$, after application of this rule $o == 0.1$ and the color value of $o$ shall be the same value as-if the initial value of $o$ was $0.1$.
	
	Given $o == -0.3$, after application of this rule $o == 0.7$ and the color value of $o$ shall be the same as-if the initial value of $o$ was $0.7$.
	\end{example}
	
	\item If wrap mode is \tcode{wrap_mode::reflect}, $o$ shall be set to the absolute value of $o$, then 2.0f shall be subtracted from $o$ until $o$ is in the range $[0,2]$, then if $o$ is in the range $(1,2]$ then $o$ shall be set to \tcode{1.0f - (o - 1.0f)}, at which point its color value is the color value for the modified value of $o$ as determined by these rules.
	\begin{example}
	Given $o == 2.8$, after application of this rule $o == 0.8$ and the color value of $o$ shall be the same value as-if the initial value of $o$ was $0.8$.
	
	Given $o == 3.6$, after application of this rule $o == 0.4$ and the color value of $o$ shall be the same value as-if the initial value of $o$ was $0.4$.
	
	Given $o == -0.3$, after application of this rule $o == 0.3$ and the color value of $o$ shall be the same as-if the initial value of $o$ was $0.3$.
	
	Given $o == -5.8$, after application of this rule $o == 0.2$ and the color value of $o$ shall be the same as-if the initial value of $o$ was $0.2$.
	\end{example}
	\end{itemize}
\end{enumeratea}

\pnum
Interpolation between the color values of two adjacent gradient stops is performed linearly on each color channel.
