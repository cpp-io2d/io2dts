%!TEX root = io2d.tex
\rSec0 [\iotwod.fillrule] {Enum class \tcode{fill_rule}}

\rSec1 [\iotwod.fillrule.summary] {\tcode{fill_rule} summary}

\pnum
The \tcode{fill_rule} enum class determines how the filling operation (\ref{\iotwod.surface.filling}) is performed on a path.

\pnum
For each point, draw a ray from that point to infinity which does not pass through the start point or end point of any non-degenerate segment in the path, is not tangent to any non-degenerate segment in the path, and is not coincident with any non-degenerate segment in the path.

\pnum
See Table~\ref{tab:\iotwod.fillrule.meanings} for the meaning of each \tcode{fill_rule} enumerator.

\rSec1 [\iotwod.fillrule.synopsis] {\tcode{fill_rule} synopsis}

\begin{codeblock}
namespace std::experimental::io2d::v1 {
  enum class fill_rule {
    winding,
    even_odd
  };
}
\end{codeblock}

\rSec1 [\iotwod.fillrule.enumerators] {\tcode{fill_rule} enumerators}

\begin{libreqtab2}
 {\tcode{fill_rule} enumerator meanings}
 {tab:\iotwod.fillrule.meanings}
 \\ \topline
 \lhdr{Enumerator}
 & \rhdr{Meaning}
 \\ \capsep
 \endfirsthead
 \continuedcaption\\
 \hline
 \lhdr{Enumerator}
 & \rhdr{Meaning}
 \\ \capsep
 \endhead
 \tcode{winding}
 & If the \term{fill rule} (\ref{\iotwod.brushprops.summary}) is \tcode{fill_rule::winding}, then using the ray described above and beginning with a count of zero, add one to the count each time a non-degenerate segment crosses the ray going left-to-right from its begin point to its end point, and subtract one each time a non-degenerate segment crosses the ray going from right-to-left from its begin point to its end point. If the resulting count is zero after all non-degenerate segments that cross the ray have been evaluated, the point shall not be filled; otherwise the point shall be filled.
 \\
 \tcode{even_odd}
 & If the fill rule is \tcode{fill_rule::even_odd}, then using the ray described above and beginning with a count of zero, add one to the count each time a non-degenerate segment crosses the ray. If the resulting count is an odd number after all non-degenerate segments that cross the ray have been evaluated, the point shall be filled; otherwise the point shall not be filled.
 \begin{note}
 Mathematically, zero is an even number, not an odd number.
 \end{note}
 \\ 
\end{libreqtab2}
