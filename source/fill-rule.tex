%!TEX root = io2d.tex
\rSec0 [\iotwod.fill.rule] {Enum class \tcode{fill_rule}}

\rSec1 [\iotwod.fill.rule.summary] {\tcode{fill_rule} Summary}

\pnum
The \tcode{fill_rule} enum class determines how the individuals paths in
\tcode{path} objects are filled. For information about multiple paths in
a single \tcode{path} object, see the description of the \tcode{path}
class~(\ref{\iotwod.path}).

\pnum
In order to determine whether a point will be included in a fill
operation, create a ray from the point to infinity. The direction of the
ray does not matter provided that it does not pass through the end point
of a path segment and does not intersect the path at a point that is
tangent to the path. The intersections of the ray with the path are used
in conjunction with the current \tcode{fill_rule} value to determine
whether a point is filled. See Table~\ref{tab:\iotwod.fill.rule.meanings}
for the meaning of each \tcode{fill_rule} enumerator.
\enternote
When used below, if the term path is formatted normally, i.e. path, it
refers to an individual path within a \tcode{path} object. If the term is
formatted in code style, i.e. \tcode{path}, it refers to the \tcode{path}
type, which may contain one or more paths.
\exitnote

\pnum
The default \tcode{fill_rule} is \tcode{fill_rule::winding}.

\rSec1 [\iotwod.fill.rule.synopsis] {\tcode{fill_rule} Synopsis}

\begin{codeblock}
namespace std { namespace experimental { namespace io2d { inline namespace v1 {
  enum class fill_rule {
    winding,
    even_odd
  };
} } } } // namespaces std::experimental::io2d::v1
\end{codeblock}

\rSec1 [\iotwod.fill.rule.enumerators] {\tcode{fill_rule} Enumerators}

\begin{libreqtab2}
 {\tcode{fill_rule} enumerator meanings}
 {tab:\iotwod.fill.rule.meanings}
 \\ \topline
 \lhdr{Enumerator}
 & \rhdr{Meaning}
 \\ \capsep
 \endfirsthead
 \continuedcaption\\
 \hline
 \lhdr{Enumerator}
 & \rhdr{Meaning}
 \\ \capsep
 \endhead
 \tcode{winding}
 & Starting with a count of zero, whenever the path crosses the ray from
 left to right, add one to the count, and whenever the path crosses the
 ray from right to left, subtract one from the count. If the count is
 zero the point is not filled, otherwise it is filled.
 \\
 \tcode{even_odd}
 & Count the number of times the ray intersects the path. If the count
 is even, the point is not filled, otherwise it is filled.
 \\ 
\end{libreqtab2}
