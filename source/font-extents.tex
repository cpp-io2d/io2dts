%!TEX root = io2d.tex
\rSec0 [fontextents] {Class \tcode{font_extents}}

\rSec1 [fontextents.synopsis] {\tcode{font_extents} synopsis}

\begin{codeblock}
namespace std { namespace experimental { namespace io2d { inline namespace v1 {
  class font_extents {
  public:
    // \ref{fontextents.cons}, construct/copy/move/destroy:
    font_extents() noexcept;
    font_extents(const font_extents& other) noexcept;
    font_extents& operator=(const font_extents& other) noexcept;
    font_extents(font_extents&& other) noexcept;
    font_extents& operator=(font_extents&& other) noexcept;
    font_extents(double ascent, double descent, double height) noexcept;

    // \ref{fontextents.modifiers}, modifiers:
    void ascent(double value) noexcept;
    void descent(double value) noexcept;
    void height(double value) noexcept;

    // \ref{fontextents.observers}, observers:
    double ascent() const noexcept;
    double descent() const noexcept;
    double height() const noexcept;

  private:
    double _Asc;    // \expos
    double _Desc;   // \expos
    double _Height; // \expos
  };
} } } }
\end{codeblock}

\rSec1 [fontextents.intro] {\tcode{font_extents} Description}

\pnum
\indexlibrary{\idxcode{font_extents}}
The class \tcode{font_extents} describes metric information for a font.

\pnum
It is used by a \tcode{surface} object to report certain metrics of its currently selected font in the \tcode{surface} object's untransformed coordinate space units.

\pnum
These metrics cover all glyphs in a font and thus may be noticeably larger than the values obtained by getting the \tcode{text_extents} for a particular string.

\pnum
\enternote
This object's observable values can be manipulated by library users for their convenience. But since the \tcode{font_extents} object returned by \tcode{surface::font_extents()} is not a reference or a pointer, the changes do not reflect back to the surface or its current font.
\exitnote

\rSec1 [fontextents.cons] {\tcode{font_extents} constructors and assignment operators}

\indexlibrary{\idxcode{font_extents}!constructor}
\begin{itemdecl}
    font_extents() noexcept;
\end{itemdecl}
\begin{itemdescr}
	\pnum
	\effects
	Constructs an object of type \tcode{font_extents}.
	
	\pnum
	\postconditions
	\tcode{_Asc == 0.0}.
	
	\tcode{_Desc == 0.0}.
	
	\tcode{_Height == 0.0}.
\end{itemdescr}

\indexlibrary{\idxcode{font_extents}!constructor}
\begin{itemdecl}
    font_extents(double ascent, double descent, double height) noexcept;
\end{itemdecl}
\begin{itemdescr}
	\pnum
	\effects
	Constructs an object of type \tcode{font_extents}.
	
	\pnum
	\postconditions
	\tcode{_Asc == ascent}.
	
	\tcode{_Desc == descent}.
	
	\tcode{_Height == height}.
\end{itemdescr}

\rSec1 [fontextents.modifiers]{\tcode{font_extents} modifiers}

\indexlibrary{\idxcode{font_extents}!\idxcode{ascent}}
\indexlibrary{\idxcode{ascent}!\idxcode{font_extents}}
\begin{itemdecl}
    void ascent(double value) noexcept;
\end{itemdecl}

\begin{itemdescr}
	\pnum
	\postconditions
	\tcode{_Asc == value}.
\end{itemdescr}

\indexlibrary{\idxcode{font_extents}!\idxcode{descent}}
\indexlibrary{\idxcode{descent}!\idxcode{font_extents}}
\begin{itemdecl}
    void descent(double value) noexcept;
\end{itemdecl}

\begin{itemdescr}
	\pnum
	\postconditions
	\tcode{_Desc == value}.
	
\end{itemdescr}

\indexlibrary{\idxcode{font_extents}!\idxcode{height}}
\indexlibrary{\idxcode{height}!\idxcode{font_extents}}
\begin{itemdecl}
    void height(double value) noexcept;
\end{itemdecl}

\begin{itemdescr}
	\pnum
	\postconditions
	\tcode{_Height == value}.
	
\end{itemdescr}

\rSec1 [fontextents.observers]{\tcode{font_extents} observers}

\indexlibrary{\idxcode{font_extents}!\idxcode{ascent}}
\indexlibrary{\idxcode{ascent}!\idxcode{font_extents}}
\begin{itemdecl}
    double ascent() const noexcept;
\end{itemdecl}
\begin{itemdescr}
	\pnum
	\returns
	\tcode{_Asc}.
	
	\pnum
	\remarks
	This value is the distance in untransformed coordinate space units from the top of the font's bounding box to the font's baseline.
	
	\pnum
	Some glyphs may extend slightly above the top of the font's bounding box due to hinting or for aesthetic reasons.

\end{itemdescr}

\indexlibrary{\idxcode{font_extents}!\idxcode{descent}}
\indexlibrary{\idxcode{descent}!\idxcode{font_extents}}
\begin{itemdecl}
    double descent() const noexcept;
\end{itemdecl}
\begin{itemdescr}
	\pnum
	\returns
	\tcode{_Desc}.
	
	\pnum
	\remarks
	This value is the distance in untransformed coordinate space units from the bottom of the font's bounding box to the font's baseline.
	
	\pnum
	Some glyphs may extend slightly below the bottom of the font's bounding box due to hinting or for aesthetic reasons.
	
	\pnum
	\enternote
	Some font rendering technologies express this value as a negative value. Because it is defined here as a distance from the baseline, the value should typically be positive or zero. It would only be negative if the font's baseline was set below the bottom of its bounding box, which, while highly unlikely, is not impossible.
	\exitnote

\end{itemdescr}

\indexlibrary{\idxcode{font_extents}!\idxcode{height}}
\indexlibrary{\idxcode{height}!\idxcode{font_extents}}
\begin{itemdecl}
    double height() const noexcept;
\end{itemdecl}
\begin{itemdescr}
	\pnum
	\returns
	\tcode{_Height}.
	
	\pnum
	\remarks
	This value is the font designer's suggested distance, in untransformed coordinate space units, from the baseline of one line of text to the baseline of a consecutive line of text.
	
	\pnum
	This value is may be greater than the sum of \tcode{ascent()} and \tcode{descent()}. This occurs when a font includes a value known as a line gap or as external leading, which is additional whitespace added for aesthetic reasons.
	
	\pnum	
	Fonts whose \tcode{height()} is equal to their \tcode{ascent() + descent()} likely include line gap in their ascent or descent rather than specifying it separately.

\end{itemdescr}
