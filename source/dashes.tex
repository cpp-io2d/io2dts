%!TEX root = io2d.tex

\rSec0 [\iotwod.dashes] {Class template \tcode{basic_dashes}}

\rSec1 [\iotwod.dashes.intro] {\tcode{basic_dashes} class template}

\pnum
\indexlibrary{\idxcode{basic_dashes}}%
The class template \tcode{basic_dashes} describes a pattern for determining, in conjunction with other properties, what points on a path are included when a stroking operation is performed.

\pnum
It has an \term{offset} of type \tcode{float} and a \term{pattern} of an \unspec type capable of sequentially storing floating-point values.

\pnum
The data are stored in an object of type \tcode{typename GraphicsSurfaces::surface_props_data::dashes_props_data_type}. It is accessible using the \tcode{data} member function.

\rSec1 [\iotwod.dashes.synopsis] {Synopsis}

\begin{codeblock}
namespace @\fullnamespace{}@ {
  template <class GraphicsSurfaces>
  class basic_dashes {
  public:
  using data_type = 
    typename GraphicsSurfaces::surface_state_props::dashes_data_type;
  public:
    // \ref{\iotwod.dashes.cons}, constructors:
    basic_dashes() noexcept;
    template <class InputIterator>
    basic_dashes(float o, InputIterator first, InputIterator last);
    basic_dashes(float o, @\stdqualifier{}@initializer_list<float> il);

    // \ref{\iotwod.dashes.observers}, observers:
    const data_type& data() const noexcept;
  };

  // \ref{\iotwod.dashes.ops}, operators:
  template <class GraphicsSurfaces>
  bool operator==(const basic_dashes<GraphicsSurfaces>& lhs,
    const basic_dashes<GraphicsSurfaces>& rhs) noexcept;
  template <class GraphicsSurfaces>
  bool operator!=(const basic_dashes<GraphicsSurfaces>& lhs,
    const basic_dashes<GraphicsSurfaces>& rhs) noexcept;
}
\end{codeblock}

\rSec1 [\iotwod.dashes.cons] {Constructors}

\indexlibrary{\idxcode{basic_dashes}!constructor}%
\begin{itemdecl}
basic_dashes() noexcept;
\end{itemdecl}
\begin{itemdescr}
\pnum
\effects
Constructs an object of type \tcode{basic_dashes}.

\pnum
\postconditions
\tcode{data() == GraphicsSurfaces::surface_state_props::create_dashes()}.

\pnum
\remarks
The offset is \tcode{0.0f} and the pattern contains no values.
\end{itemdescr}

\indexlibrary{\idxcode{basic_dashes}!constructor}%
\begin{itemdecl}
template <class InputIterator>
basic_dashes(float o, InputIterator first, InputIterator last);
\end{itemdecl}
\begin{itemdescr}
\pnum
\requires
The value type of \tcode{InputIterator} is \tcode{float}.

\pnum
Each value from \tcode{first} through \tcode{last - 1} is greater than or equal to \tcode{0.0f}.

\pnum
\effects
Constructs an object of type \tcode{basic_dashes}.

\pnum
\postconditions
\tcode{data() == GraphicsSurfaces::surface_state_props::create_dashes(o, first, last)}.

\pnum
\remarks
The offset is \tcode{o} and the pattern is the sequential list of value beginning at \tcode{first} and ending at \tcode{last - 1}.
\end{itemdescr}

\indexlibrary{\idxcode{basic_dashes}!constructor}%
\begin{itemdecl}
basic_dashes(float o, @\stdqualifier{}@initializer_list<float> il);
\end{itemdecl}
\begin{itemdescr}
\pnum
\requires
Each value in \tcode{il} is greater than or equal to \tcode{0.0f}.

\pnum
\effects
Constructs an object of type \tcode{basic_dashes}.

\pnum
\postconditions
\tcode{data() == GraphicsSurfaces::surface_state_props::create_dashes(o, il)}.
\end{itemdescr}

\rSec1 [\iotwod.dashes.observers] {Observers}

\indexlibrarymember{data}{basic_brush}
\begin{itemdecl}
const data_type& data() const noexcept;
\end{itemdecl}
\begin{itemdescr}
\pnum
\returns
A reference to the \tcode{basic_dashes} object's data object (See \ref{\iotwod.dashes.intro}).
\end{itemdescr}

\rSec1 [\iotwod.dashes.ops] {Operators}

\pnum
\begin{itemdecl}
template <class GraphicsSurfaces>
bool operator==(const basic_dashes<GraphicsSurfaces>& lhs,
  const basic_dashes<GraphicsSurfaces>& rhs) noexcept;
\end{itemdecl}
\begin{itemdescr}
\pnum
\returns
\tcode{GraphicsSurfaces::surface_state_props::equal(lhs.data(), rhs.data())}.
\end{itemdescr}

\pnum
\begin{itemdecl}
template <class GraphicsSurfaces>
bool operator!=(const basic_dashes<GraphicsSurfaces>& lhs,
  const basic_dashes<GraphicsSurfaces>& rhs) noexcept;
\end{itemdecl}
\begin{itemdescr}
\pnum
\returns
\tcode{GraphicsSurfaces::surface_state_props::not_equal(lhs.data(), rhs.data())}.
\end{itemdescr}
