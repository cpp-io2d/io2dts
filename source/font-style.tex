%!TEX root = io2d.tex

\rSec0 [\iotwod.text.style] {Enum class \tcode{font_style}}

\rSec1 [\iotwod.text.style.summary] {\tcode{font_style} summary}

\pnum
The \tcode{font_style} enum class specifies that a specific font pattern shall be used. If this font pattern is not available in the requested font family, a similar font family that contains a font face with this font pattern shall be used when creating the \tcode{basic_font} object.
%The names of the enumerators correspond to the names of the \term{usWidthClass} values in the \term{OS/2} table described in the OFF Font Format and represent the same meaning as their counterparts in the OFF Font Format.

\rSec1 [\iotwod.text.style.synopsis] {\tcode{font_style} synopsis}

\indexlibrary{\idxcode{font_style}}
\begin{codeblock}
namespace @\fullnamespace{}@ {
  enum class font_style {
    normal,
    italic,
    oblique
  };
}
\end{codeblock}

\rSec1 [\iotwod.text.style.enumerators] {\tcode{font_style} enumerators}

\pnum
All enumerators are defined in terms of bit flags set in the \term{fsSelection} value of the \tcode{OS/2} table of the font.

\begin{libreqtab2}
 {\tcode{font_style} enumerator meanings}
 {tab:\iotwod.text.style.meanings}
 \\ \topline
 \lhdr{Enumerator}
 & \rhdr{Meaning}
 \\ \capsep
 \endfirsthead
 \continuedcaption\\
 \hline
 \lhdr{Enumerator}
 & \rhdr{Meaning}
 \\ \capsep
 \endhead
 \tcode{normal}
 & A font face with bit 6 (REGULAR) set to 1.
 \\ \rowsep
 \tcode{italic}
 & A font face with bit 0 (ITALIC) set to 1.
 \\ \rowsep
 \tcode{oblique}
 & A font face with bit 9 (OBLIQUE) set to 1 or bit 1 (ITALIC) set to 1, with font rendering preference given to the font face that has bit 9 set to 1 if both font faces are present.
 \\
\end{libreqtab2}
