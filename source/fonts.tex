%!TEX root = io2d.tex

\rSec0 [typesetting] {Typesetting}

%\pnum
%Fonts describe how text should be rendered and composed. This \documenttypename leaves much of how the rendering and composing happens up to the implementors.
%
%\pnum
%Fonts exist to describe how text is rendered. The only requirements that exist in this \documenttypename are:
%\begin{itemize}
%\item The text to be rendered is in the UTF-8 character encoding.
%\item The default font face of the implementation shall correctly render all of the characters of the Unicode Basic Multilingual Plane C0 Controls and Basic Latin.
%\end{itemize}
%
\pnum
Typesetting is the process of rendering and composing text.

%\pnum
% ***FIXME***
%
%\pnum
%***FIXME*** bitmap fonts versus outline(?) fonts and how to deal with these while leaving the door open to other possible font types that won't be described (because they don't yet exist or are rarely used), etc.

\pnum
This section is forthcoming in a future revision.

\addtocounter{SectionDepthBase}{1}
%!TEX root = io2d.tex
\rSec0 [\iotwod.fontslant] {Enum class \tcode{font_slant}}

\rSec1 [\iotwod.fontslant.summary] {\tcode{font_slant} Summary}

\pnum
The \tcode{font_slant} enum class specifies the slant requested for rendering 
text.

\pnum
These values have different meanings for different scripts. For some scripts 
they may have no meaning at all. Further, not all typefaces will support every 
value.

\pnum
As such, these values are requests which implementations should honor if possible.

\pnum
See Table~\ref{tab:\iotwod.fontslant.meanings} for the meaning of each
\tcode{font_slant} enumerator.

\rSec1 [\iotwod.fontslant.synopsis] {\tcode{font_slant} Synopsis}

\begin{codeblock}
namespace std { namespace experimental { namespace io2d { inline namespace v1 {
  enum class font_slant {
    normal,
    italic,
    oblique
  };
} } } } // namespaces std::experimental::io2d::v1
\end{codeblock}

\rSec1 [\iotwod.fontslant.enumerators] {\tcode{font_slant} Enumerators}
\begin{libreqtab2}
 {\tcode{font_slant} enumerator meanings}
 {tab:\iotwod.fontslant.meanings}
 \\ \topline
 \lhdr{Enumerator}
 & \rhdr{Meaning}
 \\ \capsep
 \endfirsthead
 \continuedcaption\\
 \hline
 \lhdr{Enumerator}
 & \rhdr{Meaning}
 \\ \capsep
 \endhead
 \tcode{normal}
 & The text shall be rendered in whatever is a normal type for the font. If a font has both an italic type and an oblique type but does not have another type, then the italic type shall be the normal type for the font unless the font includes data that specifies otherwise.
 \enternote
 If a font only has an italic type or only an oblique type then that is the normal type for the font.
 \exitnote  
 \\
 \tcode{italic}
 & The text should be rendered in whatever is an italic type for the font.
 If a font does not have an italic type but does have an oblique type, the 
 oblique type shall be used if \tcode{italic} is requested. If a font has neither an italic type nor and oblique type, the normal type shall be used.
 \\
 \tcode{oblique}
 & The text should be rendered in whatever is an oblique type for the font.
 If a font does not have an oblique type but does have an italic type, the 
 italic type shall be used if \tcode{oblique} is requested. If a font has neither an italic type nor and oblique type, the normal type shall be used.
 \\
\end{libreqtab2}

%!TEX root = io2d.tex

\rSec0 [\iotwod.text.weight] {Enum class \tcode{font_weight}}

\rSec1 [\iotwod.text.weight.summary] {\tcode{font_weight} summary}

\pnum
The \tcode{font_weight} enum class indicates the visual weight (degree of blackness or thickness of strokes) of the characters in a font. The names of the enumerators correspond to the names of the \term{usWeightClass} values in the \term{OS/2} table described in the OFF Font Format and represent the same meaning as their counterparts in the OFF Font Format.

\pnum
The names of the enumerators substitute _ for - in order to conform to \Cpp{} syntax.

\rSec1 [\iotwod.text.weight.synopsis] {\tcode{font_weight} synopsis}

\indexlibrary{\idxcode{font_weight}}
\begin{codeblock}
namespace @\fullnamespace{}@ {
  enum class font_weight {
    thin,
    extra_light,
    light,
    normal,
    medium,
    semi_bold,
    bold,
    extra_bold,
    black
  };
}
\end{codeblock}

%!TEX root = io2d.tex
\rSec0 [\iotwod.subpixel.order] {Enum class \tcode{subpixel_order}}

\rSec1 [\iotwod.subpixel.order.summary] {\tcode{subpixel_order} Summary}

\pnum
The \tcode{subpixel_order} enum class is used to request a specific order of 
color channels for each pixel of an output device. When a \tcode{surface} 
object's \tcode{font_options} object has its \tcode{antialias} 
value set to \tcode{antialias::subpixel} and its \tcode{subpixel_order} value 
set to one of these values, an implementation should use the specified 
\tcode{subpixel_order} to render text.
See Table~\ref{tab:\iotwod.subpixel.order.meanings} for the meaning of each
\tcode{subpixel_order} enumerator.

\rSec1 [\iotwod.subpixel.order.synopsis] {\tcode{subpixel_order} Synopsis}

\begin{codeblock}
namespace std { namespace experimental { namespace io2d { inline namespace v1 {
  enum class subpixel_order {
    default_subpixel_order,
    horizontal_rgb,
    horizontal_bgr,
    vertical_rgb,
    vertical_bgr
  };
} } } } // namespaces std::experimental::io2d::v1
\end{codeblock}

\rSec1 [\iotwod.subpixel.order.enumerators] {\tcode{subpixel_order} Enumerators}
\begin{libreqtab2}
 {\tcode{subpixel_order} enumerator meanings}
 {tab:\iotwod.subpixel.order.meanings}
 \\ \topline
 \lhdr{Enumerator}
 & \rhdr{Meaning}
 \\ \capsep
 \endfirsthead
 \continuedcaption\\
 \hline
 \lhdr{Enumerator}
 & \rhdr{Meaning}
 \\ \capsep
 \endhead
 \tcode{default_subpixel_order}
 & The implementation should use the target \tcode{surface} object's default 
 subpixel order.
 \\
 \tcode{horizontal_rgb}
 & The color channels should be arranged horizontally starting with red on the 
 left, followed by green, then blue.
 \\
 \tcode{horizontal_bgr}
 & The color channels should be arranged horizontally starting with blue on the 
 left, followed by green, then red.
 \\
 \tcode{vertical_rgb}
 & The color channels should be arranged vertically starting with red on the 
 top, followed by green, then blue.
 \\
 \tcode{vertical_bgr}
 & The color channels should be arranged vertically starting with blue on the 
 top, followed by green, then red.
 \\
\end{libreqtab2}

%!TEX root = io2d.tex
\rSec0 [fontextents] {Class \tcode{font_extents}}

\rSec1 [fontextents.synopsis] {\tcode{font_extents} synopsis}

\begin{codeblock}
namespace std { namespace experimental { namespace io2d { inline namespace v1 {
  class font_extents {
  public:
    // \ref{fontextents.cons}, construct/copy/move/destroy:
    font_extents() noexcept;
    font_extents(const font_extents& other) noexcept;
    font_extents& operator=(const font_extents& other) noexcept;
    font_extents(font_extents&& other) noexcept;
    font_extents& operator=(font_extents&& other) noexcept;
    font_extents(double ascent, double descent, double height) noexcept;

    // \ref{fontextents.modifiers}, modifiers:
    void ascent(double value) noexcept;
    void descent(double value) noexcept;
    void height(double value) noexcept;

    // \ref{fontextents.observers}, observers:
    double ascent() const noexcept;
    double descent() const noexcept;
    double height() const noexcept;

  private:
    double _Asc;    // \expos
    double _Desc;   // \expos
    double _Height; // \expos
  };
} } } }
\end{codeblock}

\rSec1 [fontextents.intro] {\tcode{font_extents} Description}

\pnum
\indexlibrary{\idxcode{font_extents}}
The class \tcode{font_extents} describes metric information for a font.

\pnum
It is used by a \tcode{surface} object to report certain metrics of its currently selected font in the \tcode{surface} object's untransformed coordinate space units.

\pnum
These metrics cover all glyphs in a font and thus may be noticeably larger than the values obtained by getting the \tcode{text_extents} for a particular string.

\pnum
\enternote
This object's observable values can be manipulated by library users for their convenience. But since the \tcode{font_extents} object returned by \tcode{surface::font_extents()} is not a reference or a pointer, the changes do not reflect back to the surface or its current font.
\exitnote

\rSec1 [fontextents.cons] {\tcode{font_extents} constructors and assignment operators}

\indexlibrary{\idxcode{font_extents}!constructor}
\begin{itemdecl}
    font_extents() noexcept;
\end{itemdecl}
\begin{itemdescr}
	\pnum
	\effects
	Constructs an object of type \tcode{font_extents}.
	
	\pnum
	\postconditions
	\tcode{_Asc == 0.0}.
	
	\tcode{_Desc == 0.0}.
	
	\tcode{_Height == 0.0}.
\end{itemdescr}

\indexlibrary{\idxcode{font_extents}!constructor}
\begin{itemdecl}
    font_extents(double ascent, double descent, double height) noexcept;
\end{itemdecl}
\begin{itemdescr}
	\pnum
	\effects
	Constructs an object of type \tcode{font_extents}.
	
	\pnum
	\postconditions
	\tcode{_Asc == ascent}.
	
	\tcode{_Desc == descent}.
	
	\tcode{_Height == height}.
\end{itemdescr}

\rSec1 [fontextents.modifiers]{\tcode{font_extents} modifiers}

\indexlibrary{\idxcode{font_extents}!\idxcode{ascent}}
\indexlibrary{\idxcode{ascent}!\idxcode{font_extents}}
\begin{itemdecl}
    void ascent(double value) noexcept;
\end{itemdecl}

\begin{itemdescr}
	\pnum
	\postconditions
	\tcode{_Asc == value}.
\end{itemdescr}

\indexlibrary{\idxcode{font_extents}!\idxcode{descent}}
\indexlibrary{\idxcode{descent}!\idxcode{font_extents}}
\begin{itemdecl}
    void descent(double value) noexcept;
\end{itemdecl}

\begin{itemdescr}
	\pnum
	\postconditions
	\tcode{_Desc == value}.
	
\end{itemdescr}

\indexlibrary{\idxcode{font_extents}!\idxcode{height}}
\indexlibrary{\idxcode{height}!\idxcode{font_extents}}
\begin{itemdecl}
    void height(double value) noexcept;
\end{itemdecl}

\begin{itemdescr}
	\pnum
	\postconditions
	\tcode{_Height == value}.
	
\end{itemdescr}

\rSec1 [fontextents.observers]{\tcode{font_extents} observers}

\indexlibrary{\idxcode{font_extents}!\idxcode{ascent}}
\indexlibrary{\idxcode{ascent}!\idxcode{font_extents}}
\begin{itemdecl}
    double ascent() const noexcept;
\end{itemdecl}
\begin{itemdescr}
	\pnum
	\returns
	\tcode{_Asc}.
	
	\pnum
	\remarks
	This value is the distance in untransformed coordinate space units from the top of the font's bounding box to the font's baseline.
	
	\pnum
	Some glyphs may tiling slightly above the top of the font's bounding box due to hinting or for aesthetic reasons.

\end{itemdescr}

\indexlibrary{\idxcode{font_extents}!\idxcode{descent}}
\indexlibrary{\idxcode{descent}!\idxcode{font_extents}}
\begin{itemdecl}
    double descent() const noexcept;
\end{itemdecl}
\begin{itemdescr}
	\pnum
	\returns
	\tcode{_Desc}.
	
	\pnum
	\remarks
	This value is the distance in untransformed coordinate space units from the bottom of the font's bounding box to the font's baseline.
	
	\pnum
	Some glyphs may tiling slightly below the bottom of the font's bounding box due to hinting or for aesthetic reasons.
	
	\pnum
	\enternote
	Some font rendering technologies express this value as a negative value. Because it is defined here as a distance from the baseline, the value should typically be positive or zero. It would only be negative if the font's baseline was set below the bottom of its bounding box, which, while highly unlikely, is not impossible.
	\exitnote

\end{itemdescr}

\indexlibrary{\idxcode{font_extents}!\idxcode{height}}
\indexlibrary{\idxcode{height}!\idxcode{font_extents}}
\begin{itemdecl}
    double height() const noexcept;
\end{itemdecl}
\begin{itemdescr}
	\pnum
	\returns
	\tcode{_Height}.
	
	\pnum
	\remarks
	This value is the font designer's suggested distance, in untransformed coordinate space units, from the baseline of one line of text to the baseline of a consecutive line of text.
	
	\pnum
	This value is may be greater than the sum of \tcode{ascent()} and \tcode{descent()}. This occurs when a font includes a value known as a line gap or as external leading, which is additional whitespace added for aesthetic reasons.
	
	\pnum	
	Fonts whose \tcode{height()} is equal to their \tcode{ascent() + descent()} likely include line gap in their ascent or descent rather than specifying it separately.

\end{itemdescr}

%!TEX root = io2d.tex
\rSec0 [textextents] {Class \tcode{text_extents}}

\rSec1 [textextents.synopsis] {\tcode{text_extents} synopsis}

\begin{codeblock}
namespace std { namespace experimental { namespace io2d { inline namespace v1 {
  class text_extents {
  public:
    // \ref{textextents.cons}, construct/copy/move/destroy:
    text_extents() noexcept;
    text_extents(const text_extents& other) noexcept;
    text_extents& operator=(const text_extents& other) noexcept;
    text_extents(font_extents&& other) noexcept;
    text_extents& operator=(font_extents&& other) noexcept;
    text_extents(double xBearing, double yBearing, double width,
      double height, double xAdvance, double yAdvance) noexcept;

    // \ref{textextents.modifiers}, modifiers:
    void x_bearing(double value) noexcept;
    void y_bearing(double value) noexcept;
    void width(double value) noexcept;
    void height(double value) noexcept;
    void x_advance(double value) noexcept;
    void y_advance(double value) noexcept;

    // \ref{textextents.observers}, observers:
    double x_bearing() const noexcept;
    double y_bearing() const noexcept;
    double width() const noexcept;
    double height() const noexcept;
    double x_advance() const noexcept;
    double y_advance() const noexcept;

  private:
    double _X_bear; // \expos
    double _Y_bear; // \expos
    double _Width;  // \expos
    double _Height; // \expos
    double _X_adv;  // \expos
    double _Y_adv;  // \expos
  };
} } } }
\end{codeblock}

\rSec1 [textextents.intro] {\tcode{text_extents} Description}

\pnum
\indexlibrary{\idxcode{text_extents}}
The class \tcode{text_extents} describes extents for a string.

\pnum
It is used by a \tcode{surface} object to report the extents of a string in the \tcode{surface} object's untransformed coordinate space units if the string were rendered with the currently selected font.

\pnum
\enternote
This object's observable values can be manipulated by library users for their convenience. But since the \tcode{text_extents} object returned by \tcode{surface::text_extents()} is not a reference or a pointer, the changes do not reflect back to the surface or its current font.
\exitnote

\rSec1 [textextents.cons] {\tcode{text_extents} constructors and assignment operators}

\indexlibrary{\idxcode{text_extents}!constructor}
\begin{itemdecl}
    text_extents() noexcept;
\end{itemdecl}
\begin{itemdescr}
	\pnum
	\effects
	Constructs an object of type \tcode{text_extents}.
	
	\pnum
	\postconditions
    \tcode{_X_bear == 0.0}.
    
    \tcode{_Y_bear == 0.0}.
    
    \tcode{_Width == 0.0}.
    
    \tcode{_Height == 0.0}.
    
    \tcode{_X_adv == 0.0}.
    
    \tcode{_Y_adv == 0.0}.

\end{itemdescr}

\indexlibrary{\idxcode{text_extents}!constructor}
\begin{itemdecl}
    text_extents(double xBearing, double yBearing, double width,
      double height, double xAdvance, double yAdvance) noexcept;
\end{itemdecl}
\begin{itemdescr}
	\pnum
	\effects
	Constructs an object of type \tcode{text_extents}.
	
	\pnum
	\postconditions
    \tcode{_X_bear == xBearing}.
    
    \tcode{_Y_bear == yBearing}.
    
    \tcode{_Width == width}.
    
    \tcode{_Height == height}.
    
    \tcode{_X_adv == xAdvance}.
    
    \tcode{_Y_adv == yAdvance}.

\end{itemdescr}

\rSec1 [textextents.modifiers]{\tcode{text_extents} modifiers}

\indexlibrary{\idxcode{text_extents}!\idxcode{x_bearing}}
\indexlibrary{\idxcode{x_bearing}!\idxcode{text_extents}}
\begin{itemdecl}
    void x_bearing(double value) noexcept;
\end{itemdecl}
\begin{itemdescr}
	\pnum
	\postconditions
	\tcode{_X_bear == value}.
\end{itemdescr}

\indexlibrary{\idxcode{text_extents}!\idxcode{y_bearing}}
\indexlibrary{\idxcode{y_bearing}!\idxcode{text_extents}}
\begin{itemdecl}
    void y_bearing(double value) noexcept;
\end{itemdecl}
\begin{itemdescr}
	\pnum
	\postconditions
	\tcode{_Y_bear == value}.
	
\end{itemdescr}

\indexlibrary{\idxcode{text_extents}!\idxcode{width}}
\indexlibrary{\idxcode{width}!\idxcode{text_extents}}
\begin{itemdecl}
    void width(double value) noexcept;
\end{itemdecl}
\begin{itemdescr}
	\pnum
	\postconditions
	\tcode{_Width == value}.
	
\end{itemdescr}
	
\indexlibrary{\idxcode{text_extents}!\idxcode{height}}
\indexlibrary{\idxcode{height}!\idxcode{text_extents}}
\begin{itemdecl}
    void height(double value) noexcept;
\end{itemdecl}
\begin{itemdescr}
	\pnum
	\postconditions
	\tcode{_Height == value}.
	
\end{itemdescr}
	
\indexlibrary{\idxcode{text_extents}!\idxcode{x_advance}}
\indexlibrary{\idxcode{x_advance}!\idxcode{text_extents}}
\begin{itemdecl}
    void x_advance(double value) noexcept;
\end{itemdecl}
\begin{itemdescr}
	\pnum
	\postconditions
	\tcode{_X_adv == value}.
	
\end{itemdescr}
	
\indexlibrary{\idxcode{text_extents}!\idxcode{y_advance}}
\indexlibrary{\idxcode{y_advance}!\idxcode{text_extents}}
\begin{itemdecl}
    void y_advance(double value) noexcept;
\end{itemdecl}
\begin{itemdescr}
	\pnum
	\postconditions
	\tcode{_Y_adv == value}.
	
\end{itemdescr}

\rSec1 [textextents.observers]{\tcode{text_extents} observers}

\indexlibrary{\idxcode{text_extents}!\idxcode{x_bearing}}
\indexlibrary{\idxcode{x_bearing}!\idxcode{text_extents}}
\begin{itemdecl}
    double x_bearing() const noexcept;
\end{itemdecl}
\begin{itemdescr}
	\pnum
	\returns
	\tcode{_X_bear}.
	
	\pnum
	\remarks
	This value is the x axis offset of the leftmost visible part of the text as rendered from the x coordinate of the specified position at which to draw the text.
	
	\pnum
	Leading and trailing spaces can affect this value due to the fact that spaces change the position of other text and thus can change the position of the first visible text that is rendered. 

	\pnum
	\enternote
	Because this value is an offset from the specified position and is given in untransformed units, it remains the same regardless of the position at which the text will be drawn.
	
	\pnum
	This value will typically be negative, zero, or slightly positive depending on the font used and the text being rendered (e.g. scripts that are written right-to-left will normally have a negative \tcode{x_bearing()} value).
	\exitnote
\end{itemdescr}

\indexlibrary{\idxcode{text_extents}!\idxcode{y_bearing}}
\indexlibrary{\idxcode{y_bearing}!\idxcode{text_extents}}
\begin{itemdecl}
    double y_bearing() const noexcept;
\end{itemdecl}
\begin{itemdescr}
	\pnum
	\returns
	\tcode{_Y_bear}.
	
	\pnum
	\remarks
	This value is the y axis offset of the topmost visible part of the text as rendered from the y coordinate of the specified position at which to draw the text. 

	\pnum
	\enternote
	This value may range from negative to positive depending on the font origin chosen by the font designer. Usually this value is negative.
	\exitnote
\end{itemdescr}

\indexlibrary{\idxcode{text_extents}!\idxcode{width}}
\indexlibrary{\idxcode{width}!\idxcode{text_extents}}
\begin{itemdecl}
    double width() const noexcept;
\end{itemdecl}
\begin{itemdescr}
	\pnum
	\returns
	\tcode{_Width}.
	
	\pnum
	\remarks
	This value is the width of the text as rendered from its leftmost visible part to its rightmost visible part.
	
	\pnum
	This value may include a de minimus amount of whitespace, e.g. 1 to 2 pixels when pixels are the coordinate space unit. This allowance is meant to cover discrepancies between expected rendering results and actual results which can arise due to techniques such as font hinting, antialiasing, and subpixel rendering.

\end{itemdescr}

\indexlibrary{\idxcode{text_extents}!\idxcode{height}}
\indexlibrary{\idxcode{height}!\idxcode{text_extents}}
\begin{itemdecl}
    double height() const noexcept;
\end{itemdecl}
\begin{itemdescr}
	\pnum
	\returns
	\tcode{_Height}.
	
	\pnum
	\remarks
	This value is the height of the text as rendered from its topmost visible part to its bottommost visible part.
	
	\pnum
	This value may include a de minimus amount of whitespace, e.g. 1 to 2 pixels when pixels are the coordinate space unit. This allowance is meant to cover discrepancies between expected rendering results and actual results which can arise due to techniques such as font hinting, antialiasing, and subpixel rendering.

\end{itemdescr}

\indexlibrary{\idxcode{text_extents}!\idxcode{x_advance}}
\indexlibrary{\idxcode{x_advance}!\idxcode{text_extents}}
\begin{itemdecl}
    double x_advance() const noexcept;
\end{itemdecl}
\begin{itemdescr}
	\pnum
	\returns
	\tcode{_X_adv}.
	
	\pnum
	\remarks
	This value is amount to add to the x coordinate of the original specified position in order to properly draw text that will immediately follow this text on the same line.
	
	\pnum
	In vertically oriented text, this value will typically be zero.

\end{itemdescr}

\indexlibrary{\idxcode{text_extents}!\idxcode{y_advance}}
\indexlibrary{\idxcode{y_advance}!\idxcode{text_extents}}
\begin{itemdecl}
    double y_advance() const noexcept;
\end{itemdecl}
\begin{itemdescr}
	\pnum
	\returns
	\tcode{_Y_adv}.
	
	\pnum
	\remarks
	This value is amount to add to the y coordinate of the original specified position in order to properly draw text that will immediately follow this text on the same line.
	
	\pnum
	In horizontally oriented text, this value will typically be zero.

\end{itemdescr}

%!TEX root = io2d.tex
\rSec0 [\iotwod.fontoptions] {Class \tcode{font_options}}

\rSec1 [\iotwod.fontoptions.synopsis] {\tcode{font_options} synopsis}

\begin{codeblock}
namespace std { namespace experimental { namespace io2d { inline namespace v1 {
  class font_options {
    public:
    // \ref{\iotwod.fontoptions.cons}, construct/copy/destroy:
    font_options() noexcept;
    font_options(const font_options& other) noexcept;
    font_options& operator=(const font_options& other) noexcept;
    font_options(font_options&& other) noexcept;
    font_options& operator=(font_options&& other) noexcept;
    font_options(std::experimental::io2d::antialias a,
      std::experimental::io2d::subpixel_order so) noexcept;

    // \ref{\iotwod.fontoptions.modifiers}, modifiers:
    void antialias(std::experimental::io2d::antialias value) noexcept;
    void subpixel_order(std::experimental::io2d::subpixel_order value) noexcept;

    // \ref{\iotwod.fontoptions.observers}, observers:
    std::experimental::io2d::antialias antialias() const noexcept;
    std::experimental::io2d::subpixel_order subpixel_order() const noexcept;

  private:
    std::experimental::io2d::antialias _Antialias;           // \expos
    std::experimental::io2d::subpixel_order _Subpixel_order; // \expos
  };
} } } }
\end{codeblock}

\rSec1 [\iotwod.fontoptions.intro] {\tcode{font_options} Description}

\pnum
\indexlibrary{\idxcode{font_options}}
The \tcode{font_options} class describes an object that holds values which specify certain aspects of how text should be rendered.

\rSec1 [\iotwod.fontoptions.cons] {\tcode{font_options} constructors and assignment operators}

\indexlibrary{\idxcode{font_options}!constructor}
\begin{itemdecl}
    font_options() noexcept;
\end{itemdecl}
\begin{itemdescr}
	\pnum
	\effects
	Constructs an object of type \tcode{font_options}.
	
	\pnum
	\postconditions
	\tcode{_Antialias == std::experimental::io2d::antialias::default_antialias}.
	
	\tcode{_Subpixel_order == std::experimental::io2d::subpixel_order::default_subpixel_order}.
	
\end{itemdescr}

\indexlibrary{\idxcode{font_options}!constructor}
\begin{itemdecl}
    font_options(std::experimental::io2d::antialias a,
      std::experimental::io2d::subpixel_order so) noexcept;
\end{itemdecl}
\begin{itemdescr}
	\pnum
	\effects
	Constructs an object of type \tcode{font_options}.
	
	\pnum
	\postconditions
	\tcode{_Antialias == a}.
	
	\tcode{_Subpixel_order == so}.
	
\end{itemdescr}

\rSec1 [\iotwod.fontoptions.modifiers] {\tcode{font_options} modifiers}

\indexlibrary{\idxcode{font_options}!\idxcode{antialias}}
\indexlibrary{\idxcode{antialias}!\idxcode{font_options}}
\begin{itemdecl}
    void antialias(std::experimental::io2d::antialias value) noexcept;
\end{itemdecl}
\begin{itemdescr}
	\pnum
	\postconditions
	\tcode{_Antialias == value}.
	
\end{itemdescr}

\indexlibrary{\idxcode{font_options}!\idxcode{subpixel_order}}
\indexlibrary{\idxcode{subpixel_order}!\idxcode{font_options}}
\begin{itemdecl}
    void subpixel_order(std::experimental::io2d::subpixel_order value) noexcept;
\end{itemdecl}
\begin{itemdescr}
	\pnum
	\postconditions
	\tcode{_Subpixel_order == value}.
	
\end{itemdescr}

\rSec1 [\iotwod.fontoptions.observers] {\tcode{font_options} observers}

\indexlibrary{\idxcode{font_options}!\idxcode{antialias}}
\indexlibrary{\idxcode{antialias}!\idxcode{font_options}}
\begin{itemdecl}
std::experimental::io2d::antialias antialias() const noexcept;
\end{itemdecl}
\begin{itemdescr}
	\pnum
	\returns
	\tcode{_Antialias}.

\end{itemdescr}

\indexlibrary{\idxcode{font_options}!\idxcode{subpixel_order}}
\indexlibrary{\idxcode{subpixel_order}!\idxcode{font_options}}
\begin{itemdecl}
std::experimental::io2d::subpixel_order subpixel_order() const noexcept;
\end{itemdecl}
\begin{itemdescr}
	\pnum
	\returns
	\tcode{_Subpixel_order}.
	
\end{itemdescr}

%!TEX root = io2d.tex

\rSec0 [fontresourcefactory] {Class \tcode{font_resource_factory}}

\rSec1 [fontresourcefactory.synopsis] {\tcode{font_resource_factory} synopsis}

\begin{codeblock}
namespace std { namespace experimental { namespace io2d { inline namespace v1 {
  class font_resource_factory {
  public:
    // \ref{fontresourcefactory.cons}, construct/copy/move
    font_resource_factory();
    font_resource_factory(const font_resource_factory&);
    font_resource_factory& operator=(const font_resource_factory&)\;
    font_resource_factory(font_resource_factory&&) noexcept\;
    font_resource_factory& operator=(font_resource_factory&&) noexcept;
    explicit font_resource_factory(const string& family, 
      experimental::io2d::font_slant fs = 
      experimental::io2d::font_slant::normal,
      experimental::io2d::font_weight fw = 
      experimental::io2d::font_weight::normal,
      const matrix_2d& fm = matrix_2d::init_scale({ 16.0, 16.0 }),
      const experimental::io2d::font_options& fo = 
      experimental::io2d::font_options(),
      matrix_2d& sm = matrix_2d::init_identity());
    font_resource_factory(const string& family, 
      error_code& ec,
      experimental::io2d::font_slant fs = 
      experimental::io2d::font_slant::normal,
      experimental::io2d::font_weight fw = 
      experimental::io2d::font_weight::normal,
      const matrix_2d& fm = matrix_2d::init_scale({ 16.0, 16.0 }),
      const experimental::io2d::font_options& fo = 
      experimental::io2d::font_options(),
      matrix_2d& sm = matrix_2d::init_identity()) noexcept;

    // \ref{fontresourcefactory.modifiers}, modifiers
    void font_family(const string& f);
    void font_family(const string& f, error_code& ec) noexcept;
    void font_slant(const experimental::io2d::font_slant fs);
    void font_slant(const experimental::io2d::font_slant fs,
      error_code& ec) noexcept;
    void font_weight(experimental::io2d::font_weight fw);
    void font_weight(experimental::io2d::font_weight fw,
      error_code& ec) noexcept;
    void font_options(const experimental::io2d::font_options& fo) noexcept;
    void font_matrix(const matrix_2d& m) noexcept;
    void surface_matrix(const matrix_2d& m) noexcept;

    // \ref{fontresourcefactory.observers}, observers
    string font_family() const;
    string font_family(::std::error_code& ec) const noexcept;
    experimental::io2d::font_slant font_slant() const noexcept;
    experimental::io2d::font_weight font_weight() const noexcept;
    experimental::io2d::font_options font_options() const noexcept;
    matrix_2d font_matrix() const noexcept;
    matrix_2d surface_matrix() const noexcept;

  private:
    string _Family;                                 // \expos
    experimental::io2d::font_slant _Font_slant;     // \expos
    experimental::io2d::font_weight _Font_weight;   // \expos
    experimental::io2d::font_options _Font_options; // \expos
    matrix_2d _Font_matrix;                         // \expos
    matrix_2d _Surface_matrix;                      // \expos
  };
} } } }
\end{codeblock}

\rSec1 [fontresourcefactory.intro] {\tcode{font_resource_factory} Description}

\pnum
\indexlibrary{\idxcode{font_resource_factory}}
The \tcode{font_resource_factory} class is a factory class used to create \tcode{font_resource} objects.

\rSec1 [fontresourcefactory.fontmatching] {Font matching}

\pnum
The fonts that are available for use by implementations can vary depending on the execution environment, including during runtime.

\pnum
\enterexample
While a program is executing, the execution environment could execute one or more other programs that cause a font to become available or become unavailable.
\exitexample

\pnum
Once a font has been chosen by a \tcode{font_resource_factory} object or selected as the font resource of a \tcode{font_resource} object, then if the execution environment provides functionality to prevent the font from becoming unavailable, implementations shall use that functionality to prevent the font from becoming unavailable.

\pnum
If an implementation used the functionality described in the previous paragraph to prevent a font from becoming unavailable, then when there are no longer any \tcode{font_resource_factory} objects and \tcode{font_resource} objects that require the font, the implementation shall no longer prevent the font from becoming unavailable and shall make use of whatever functionality the execution environment provides, if any, to undo the effects of the functionality described in the previous paragraph that was used to prevent the font from becoming unavailable.

\pnum
If a font chosen by a \tcode{font_resource_factory} object becomes unavailable, then if a \tcode{font_resource} object is constructed using that \tcode{font_resource_factory}, the implementation shall choose an available font in the manner set forth below.

\pnum
When an implementation is required to choose a font, the implementation shall evaluate the value of the \tcode{string} object containing the name of the desired font family of the font, the value of the \tcode{font_slant} enumerator describing the desired posture of the font, and the value of the \tcode{font_weight} enumerator describing the desired weight of the font.

\pnum
If a font matching the desired font family, desired posture, and desired weight is available for use by the implementation, then that font shall be used and those desired values shall be the values of the font family, posture, and weight of the \tcode{font_resource_factory} object or \tcode{font_resource} object for which a font is being chosen.

\pnum
Otherwise, the values of the font family, posture, and weight of the \tcode{font_resource_factory} object or \tcode{font_resource} object for which a font is being chosen shall be the values of the font that the implementation selected as being the nearest match to the font specified by the desired font family, desired posture, and desired weight.

\pnum
The methodology used by an implementation to choose a nearest match to a font is \unspecnorm.

\rSec1 [fontresourcefactory.cons] {\tcode{font_resource_factory} constructors and assignment operators}

\indexlibrary{\idxcode{font_resource_factory}!constructor}
\begin{itemdecl}
explicit font_resource_factory(const string& family, 
  experimental::io2d::font_slant fs = experimental::io2d::font_slant::normal,
  experimental::io2d::font_weight fw = experimental::io2d::font_weight::normal,
  const matrix_2d& fm = matrix_2d::init_scale({ 16.0, 16.0 }),
  const experimental::io2d::font_options& fo = 
  experimental::io2d::font_options(),
  matrix_2d& sm = matrix_2d::init_identity());

font_resource_factory(const string& family, 
  error_code& ec,
  experimental::io2d::font_slant fs = experimental::io2d::font_slant::normal,
  experimental::io2d::font_weight fw = experimental::io2d::font_weight::normal,
  const matrix_2d& fm = matrix_2d::init_scale({ 16.0, 16.0 }),
  const experimental::io2d::font_options& fo = 
  experimental::io2d::font_options(),
  matrix_2d& sm = matrix_2d::init_identity()) noexcept;
\end{itemdecl}
\begin{itemdescr}
\pnum
\effects
The implementation shall use the values of \tcode{family}, \tcode{fs}, and \tcode{fw} as the desired font family, desired posture, and desired weight, respectively, for the font matching process described in \ref{fontresourcefactory.fontmatching} and shall set the values of \tcode{_Family}, \tcode{_Font_slant}, and \tcode{_Font_weight} to the values of the font family, posture, and weight, respectively, produced by that font matching process.

\pnum
\tcode{_Font_options == fo}.

\pnum
\tcode{_Font_matrix == fm}.

\pnum
\tcode{_Surface_matrix == sm}.

\pnum
\throws
As specified in Error reporting (\ref{\iotwod.err.report}).

\pnum
\errors
\tcode{errc::not_enough_memory} if there was a failure to allocate memory.

\pnum
Other errors, if any, produced by this function are \impldef{font_resource_factory!constructor}.
\end{itemdescr}

\rSec1 [fontresourcefactory.modifiers] {\tcode{font_resource_factory} modifiers}

\indexlibrary{\idxcode{font_resource_factory}!\idxcode{font_family}}
\indexlibrary{\idxcode{font_family}!\idxcode{font_resource_factory}}
\begin{itemdecl}
void font_family(const string& f);
void font_family(const string& f, error_code& ec) noexcept;
\end{itemdecl}
\begin{itemdescr}
\pnum
\effects
The implementation shall use the values of \tcode{f}, \tcode{_Font_slant}, and \tcode{_Font_weight} as the desired font family, desired posture, and desired weight, respectively, for the font matching process described in \ref{fontresourcefactory.fontmatching} and shall set the values of \tcode{_Family}, \tcode{_Font_slant}, and \tcode{_Font_weight} to the values of the font family, posture, and weight, respectively, produced by that font matching process.

\pnum
\throws
As specified in Error reporting (\ref{\iotwod.err.report}).

\pnum
\errors
\tcode{errc::not_enough_memory} if there was a failure to allocate memory.

\pnum
Other errors, if any, produced by this function are \impldef{font_resource_factory!font_family}.
\end{itemdescr}

\indexlibrary{\idxcode{font_resource_factory}!\idxcode{font_slant}}
\indexlibrary{\idxcode{font_slant}!\idxcode{font_resource_factory}}
\begin{itemdecl}
void font_slant(const experimental::io2d::font_slant fs);
void font_slant(const experimental::io2d::font_slant fs, 
  error_code& ec) noexcept;
\end{itemdecl}
\begin{itemdescr}
\pnum
\effects
The implementation shall use the values of \tcode{_Family}, \tcode{fs}, and \tcode{_Font_weight} as the desired font family, desired posture, and desired weight, respectively, for the font matching process described in \ref{fontresourcefactory.fontmatching} and shall set the values of \tcode{_Family}, \tcode{_Font_slant}, and \tcode{_Font_weight} to the values of the font family, posture, and weight, respectively, produced by that font matching process.

\pnum
\throws
As specified in Error reporting (\ref{\iotwod.err.report}).

\pnum
\errors
\tcode{errc::not_enough_memory} if there was a failure to allocate memory.

\pnum
Other errors, if any, produced by this function are \impldef{font_resource_factory!font_slant}.
\end{itemdescr}

\indexlibrary{\idxcode{font_resource_factory}!\idxcode{font_weight}}
\indexlibrary{\idxcode{font_weight}!\idxcode{font_resource_factory}}
\begin{itemdecl}
void font_weight(experimental::io2d::font_weight fw);
void font_weight(experimental::io2d::font_weight fw,
  error_code& ec) noexcept;
\end{itemdecl}
\begin{itemdescr}
\pnum
\effects
The implementation shall use the values of \tcode{_Family}, \tcode{_Font_slant}, and \tcode{fw} as the desired font family, desired posture, and desired weight, respectively, for the font matching process described in \ref{fontresourcefactory.fontmatching} and shall set the values of \tcode{_Family}, \tcode{_Font_slant}, and \tcode{_Font_weight} to the values of the font family, posture, and weight, respectively, produced by that font matching process.

\pnum
\throws
As specified in Error reporting (\ref{\iotwod.err.report}).

\pnum
\errors
\tcode{errc::not_enough_memory} if there was a failure to allocate memory.

\pnum
Other errors, if any, produced by this function are \impldef{font_resource_factory!font_weight}.
\end{itemdescr}

\indexlibrary{\idxcode{font_resource_factory}!\idxcode{font_options}}
\indexlibrary{\idxcode{font_options}!\idxcode{font_resource_factory}}
\begin{itemdecl}
void font_options(const experimental::io2d::font_options& fo) noexcept;
\end{itemdecl}
\begin{itemdescr}
\pnum
\effects
\tcode{_Font_options == fo}.
\end{itemdescr}

\indexlibrary{\idxcode{font_resource_factory}!\idxcode{font_matrix}}
\indexlibrary{\idxcode{font_matrix}!\idxcode{font_resource_factory}}
\begin{itemdecl}
void font_matrix(const matrix_2d& m) noexcept;
\end{itemdecl}
\begin{itemdescr}
\pnum
\effects
\tcode{_Font_matrix == m}.
\end{itemdescr}

\indexlibrary{\idxcode{font_resource_factory}!\idxcode{surface_matrix}}
\indexlibrary{\idxcode{surface_matrix}!\idxcode{font_resource_factory}}
\begin{itemdecl}
void surface_matrix(const matrix_2d& m) noexcept;
\end{itemdecl}
\begin{itemdescr}
\pnum
\effects
\tcode{_Surface_matrix == m}.
\end{itemdescr}

\rSec1 [fontresourcefactory.observers] {\tcode{font_resource_factory} observers}

\indexlibrary{\idxcode{font_resource_factory}!\idxcode{font_family}}
\indexlibrary{\idxcode{font_family}!\idxcode{font_resource_factory}}
\begin{itemdecl}
string font_family() const;
string font_family(::std::error_code& ec) const noexcept;
\end{itemdecl}
\begin{itemdescr}
\pnum
\returns
\tcode{_Family}.

\throws
As specified in Error reporting (\ref{\iotwod.err.report}).

\pnum
\errors
\tcode{errc::not_enough_memory} if there was a failure to allocate memory.
\end{itemdescr}

\indexlibrary{\idxcode{font_resource_factory}!\idxcode{font_slant}}
\indexlibrary{\idxcode{font_slant}!\idxcode{font_resource_factory}}
\begin{itemdecl}
experimental::io2d::font_slant font_slant() const noexcept;
\end{itemdecl}
\begin{itemdescr}
\pnum
\returns
\tcode{_Font_slant}.
\end{itemdescr}

\indexlibrary{\idxcode{font_resource_factory}!\idxcode{font_weight}}
\indexlibrary{\idxcode{font_weight}!\idxcode{font_resource_factory}}
\begin{itemdecl}
experimental::io2d::font_weight font_weight() const noexcept;
\end{itemdecl}
\begin{itemdescr}
\pnum
\returns
\tcode{_Font_weight}.
\end{itemdescr}

\indexlibrary{\idxcode{font_resource_factory}!\idxcode{font_options}}
\indexlibrary{\idxcode{font_options}!\idxcode{font_resource_factory}}
\begin{itemdecl}
experimental::io2d::font_options font_options() const noexcept;
\end{itemdecl}
\begin{itemdescr}
\pnum
\returns
\tcode{_Font_options}.
\end{itemdescr}

\indexlibrary{\idxcode{font_resource_factory}!\idxcode{font_matrix}}
\indexlibrary{\idxcode{font_matrix}!\idxcode{font_resource_factory}}
\begin{itemdecl}
matrix_2d font_matrix() const noexcept;
\end{itemdecl}
\begin{itemdescr}
\pnum
\returns
\tcode{_Font_matrix}.
\end{itemdescr}

\indexlibrary{\idxcode{font_resource_factory}!\idxcode{surface_matrix}}
\indexlibrary{\idxcode{surface_matrix}!\idxcode{font_resource_factory}}
\begin{itemdecl}
matrix_2d surface_matrix() const noexcept;
\end{itemdecl}
\begin{itemdescr}
\pnum
\returns
\tcode{_Surface_matrix}.
\end{itemdescr}

%!TEX root = io2d.tex

\rSec0 [fontresource] {Class \tcode{font_resource}}

\rSec1 [fontresource.synopsis] {\tcode{font_resource} synopsis}

\begin{codeblock}
namespace std { namespace experimental { namespace io2d { inline namespace v1 {
  class font_resource {
  public:
    // \ref{fontresource.cons}, construct/copy/move
    font_resource() = delete;
    font_resource(const font_resource&) noexcept;
    font_resource& operator=(const font_resource&) noexcept;
    font_resource(font_resource&&) noexcept;
    font_resource& operator=(font_resource&&) noexcept;
    explicit font_resource(const font_resource_factory& f);
    font_resource(const font_resource_factory& f, error_code& ec) noexcept;

    // \ref{fontresource.observers}, observers
    string font_family() const;
    string font_family(error_code& ec) const noexcept;
    experimental::io2d::font_slant font_slant() const noexcept;
    experimental::io2d::font_weight font_weight() const noexcept;
    experimental::io2d::font_options font_options() const noexcept;
    matrix_2d font_matrix() const noexcept;
    matrix_2d surface_matrix() const noexcept;
    experimental::io2d::font_extents font_extents() const noexcept;
    experimental::io2d::text_extents text_extents(const string& utf8) const 
      noexcept;
    glyph_run make_glyph_run(const string& utf8, const vector_2d& pos) const;
    glyph_run make_glyph_run(const string& utf8, const vector_2d& pos, 
      error_code& ec) const noexcept;

  private:
    shared_ptr<cairo_scaled_font_t> _Scaled_font;	// \expos
    shared_ptr<string> _Font_family;			// \expos
    experimental::io2d::font_slant _Font_slant;		// \expos
    experimental::io2d::font_weight _Font_weight;	// \expos
    shared_ptr<cairo_font_options_t> _Font_options;	// \expos
  };
} } } }
\end{codeblock}

\rSec1 [fontresource.intro] {\tcode{font_resource} Description}

\pnum
\indexlibrary{\idxcode{font_resource}}
\pnum
The \tcode{font_resource} class defines an immutable font resource that is used in creating \tcode{glyph_run} objects and in the Typesetting rendering and composing operation (\ref{surface.rendering} and \ref{surface.typesetting}).

\rSec1 [fontresource.cons] {\tcode{font_resource} constructors and assignment operators}

\indexlibrary{\idxcode{font_resource}!constructor}
\begin{itemdecl}
explicit font_resource(const font_resource_factory& f);
font_resource(const font_resource_factory& f, error_code& ec) noexcept;
\end{itemdecl}
\begin{itemdescr}
\pnum
\effects
Creates a \tcode{font_resource} object representing a font resource for which the font family, posture, and weight, respectively, shall be the results obtained from the font matching process described in \ref{fontresourcefactory.fontmatching} when the desired font family, desired posture, and desired weight are \tcode{f.font_family()}, \tcode{f.font_slant()}, and \tcode{f.font_weight()}, respectively.

\pnum
The \tcode{font_resource} object shall have \tcode{f.font_options()} as its \tcode{font_options} value.

\pnum
The \tcode{font_resource} object shall have \tcode{f.font_matrix()} as its font matrix.

\pnum
The \tcode{font_resource} object shall have \tcode{f.surface_matrix()} as its surface matrix.

\pnum
\throws
As specified in Error reporting (\ref{\iotwod.err.report}).

\pnum
\errors
\tcode{errc::not_enough_memory} if there was a failure to allocate memory.

\pnum
Other errors, if any, produced by this function are \impldef{font_resource!constructor}.
\end{itemdescr}

\rSec1 [fontresource.observers] {\tcode{font_resource} observers}

\indexlibrary{\idxcode{font_resource}!\idxcode{font_family}}
\indexlibrary{\idxcode{font_family}!\idxcode{font_resource}}
\begin{itemdecl}
string font_family() const;
string font_family(error_code& ec) const noexcept;
\end{itemdecl}
\begin{itemdescr}
\pnum
\returns

\throws
As specified in Error reporting (\ref{\iotwod.err.report}).

\pnum
\errors
\tcode{errc::not_enough_memory} if there was a failure to allocate memory.
\end{itemdescr}

\indexlibrary{\idxcode{font_resource}!\idxcode{font_slant}}
\indexlibrary{\idxcode{font_slant}!\idxcode{font_resource}}
\begin{itemdecl}
experimental::io2d::font_slant font_slant() const noexcept;
\end{itemdecl}
\begin{itemdescr}
\pnum
\returns

\end{itemdescr}

\indexlibrary{\idxcode{font_resource}!\idxcode{font_weight}}
\indexlibrary{\idxcode{font_weight}!\idxcode{font_resource}}
\begin{itemdecl}
experimental::io2d::font_weight font_weight() const noexcept;
\end{itemdecl}
\begin{itemdescr}
\pnum
\returns

\end{itemdescr}

\indexlibrary{\idxcode{font_resource}!\idxcode{font_options}}
\indexlibrary{\idxcode{font_options}!\idxcode{font_resource}}
\begin{itemdecl}
experimental::io2d::font_options font_options() const noexcept;
\end{itemdecl}
\begin{itemdescr}
\pnum
\returns

\end{itemdescr}

\indexlibrary{\idxcode{font_resource}!\idxcode{font_matrix}}
\indexlibrary{\idxcode{font_matrix}!\idxcode{font_resource}}
\begin{itemdecl}
matrix_2d font_matrix() const noexcept;
\end{itemdecl}
\begin{itemdescr}
\pnum
\returns

\end{itemdescr}

\indexlibrary{\idxcode{font_resource}!\idxcode{surface_matrix}}
\indexlibrary{\idxcode{surface_matrix}!\idxcode{font_resource}}
\begin{itemdecl}
matrix_2d surface_matrix() const noexcept;
\end{itemdecl}
\begin{itemdescr}
\pnum
\returns

\end{itemdescr}

\indexlibrary{\idxcode{font_resource}!\idxcode{font_extents}}
\indexlibrary{\idxcode{font_extents}!\idxcode{font_resource}}
\begin{itemdecl}
experimental::io2d::font_extents font_extents() const noexcept;
\end{itemdecl}
\begin{itemdescr}
\pnum
\returns

\end{itemdescr}

\indexlibrary{\idxcode{font_resource}!\idxcode{text_extents}}
\indexlibrary{\idxcode{text_extents}!\idxcode{font_resource}}
\begin{itemdecl}
experimental::io2d::text_extents text_extents(const string& utf8) const 
  noexcept;
\end{itemdecl}
\begin{itemdescr}
\pnum
\returns

\end{itemdescr}

\indexlibrary{\idxcode{font_resource}!\idxcode{make_glyph_run}}
\indexlibrary{\idxcode{make_glyph_run}!\idxcode{font_resource}}
\begin{itemdecl}
glyph_run make_glyph_run(const string& utf8, const vector_2d& pos) const;
glyph_run make_glyph_run(const string& utf8, const vector_2d& pos, 
  error_code& ec) const noexcept;
\end{itemdecl}
\begin{itemdescr}
\pnum
\returns

\pnum
\throws
As specified in Error reporting (\ref{\iotwod.err.report}).

\pnum
\errors
\tcode{errc::not_enough_memory} if there was a failure to allocate memory.

\pnum
Other errors, if any, produced by this function are \impldef{font_resource!make_glyph_run}.

\end{itemdescr}

%!TEX root = io2d.tex

\rSec0 [glyphrun] {Class \tcode{glyph_run}}

\rSec1 [glyphrun.synopsis] {\tcode{glyph_run} synopsis}

\begin{codeblock}
namespace std { namespace experimental { namespace io2d { inline namespace v1 {
  class glyph_run {
    friend ::std::experimental::io2d::font_resource;
    friend surface;
    friend path_factory;

  public:
    class glyph {
      friend glyph_run;
    public:
      typedef unsigned long index_type; // impldef
    private:
      index_type _Index = 0;
      double _X = 0.0;
      double _Y = 0.0;
      cairo_glyph_t* _Native_glyph = nullptr;
    public:
      glyph() noexcept = default;
      glyph(const glyph&) noexcept = default;
      glyph& operator=(const glyph&) noexcept = default;
      glyph(glyph&&) noexcept = default;
      glyph& operator=(glyph&&) noexcept = default;
      glyph(index_type i, double x, double y) noexcept;

      // Modifiers
      void index(index_type i) noexcept;
      void x(double val) noexcept;
      void y(double val) noexcept;

      // Observers
      index_type index() const noexcept;
      double x() const noexcept;
      double y() const noexcept;
    };

    class cluster {
      friend glyph_run;
      // Clusters are useful when dealing with ligatures and ordering in scripts that require complex text layout.
      int _Glyph_count = 0; // Note: It's possible that processing could result in a cluster where one or more characters map to zero glyphs due to the previous or next cluster.
      int _Byte_count = 0; // Note: UTF-8 is variable byte length. A single cluster can map to multiple characters. Lastly, it's possible that processing could result in a cluster with one or more glyphs map to zero characters due to the previous or next cluster.
      cairo_text_cluster_t* _Native_cluster = nullptr;
    public:
      cluster() noexcept = default;
      cluster(const cluster&) noexcept = default;
      cluster& operator=(const cluster&) noexcept = default;
      cluster(cluster&&) noexcept = default;
      cluster& operator=(cluster&&) noexcept = default;
      cluster(int glyphs, int bytes) noexcept;

      // Modifiers
      void glyph_count(int count) noexcept;
      void byte_count(int count) noexcept;

      // Observers
      int glyph_count() const noexcept;
      int byte_count() const noexcept;
    };

  private:
    ::std::string _Text_string;
    ::std::experimental::io2d::font_resource _Font_resource;
    ::std::vector<glyph> _Glyphs;
    ::std::vector<cluster> _Clusters;
    ::std::shared_ptr<cairo_glyph_t> _Cairo_glyphs;
    ::std::shared_ptr<cairo_text_cluster_t> _Cairo_text_clusters;
    vector_2d _Position;
    cairo_text_cluster_flags_t _Text_cluster_flags;

    glyph_run(const ::std::experimental::io2d::font_resource& fr, const ::std::string& utf8, const vector_2d& pos);
    glyph_run(const ::std::experimental::io2d::font_resource& fr, const ::std::string& utf8, const vector_2d& pos, ::std::error_code& ec) noexcept;

  public:
    glyph_run(const glyph_run&) = default;
    glyph_run& operator=(const glyph_run&) = default;
    glyph_run(glyph_run&&) noexcept = default;
    glyph_run& operator=(glyph_run&&) noexcept;

    // Modifiers
    ::std::vector<glyph>& glyphs() noexcept;
    ::std::vector<cluster>& clusters() noexcept;

    // Observers
    const ::std::string& original_text() const noexcept;
    const ::std::vector<glyph>& glyphs() const noexcept;
    const ::std::vector<cluster>& clusters() const noexcept;
    const ::std::experimental::io2d::font_resource& font_resource() const noexcept;
    bool reversed_clusters() const noexcept;
    vector_2d position() const noexcept;
    text_extents extents() const noexcept;
  };
} } } }
\end{codeblock}

\rSec1 [glyphrun.intro] {\tcode{glyph_run} Description}

\pnum
\indexlibrary{\idxcode{glyph_run}}
The \tcode{glyph_run} class describes ***FIXME***

\rSec1 [glyphrun.cons] {\tcode{glyph_run} constructors and assignment operators}

\indexlibrary{\idxcode{glyph_run}!constructor}
\begin{itemdecl}
\end{itemdecl}
\begin{itemdescr}
\pnum
\effects

\pnum
\throws
As specified in Error reporting (\ref{\iotwod.err.report}).

\pnum
\remarks

\pnum
\errors

\pnum
\realnotes

\end{itemdescr}

\rSec1 [glyphrun.modifiers] {\tcode{glyph_run} modifiers}

\indexlibrary{\idxcode{glyph_run}!\idxcode{}}
\indexlibrary{\idxcode{}!\idxcode{glyph_run}}
\begin{itemdecl}
\end{itemdecl}
\begin{itemdescr}
\pnum
\effects

\pnum
\throws
As specified in Error reporting (\ref{\iotwod.err.report}).

\pnum
\remarks

\pnum
\errors

\pnum
\realnotes

\end{itemdescr}

\rSec1 [glyphrun.observers] {\tcode{glyph_run} observers}

\indexlibrary{\idxcode{glyph_run}!\idxcode{}}
\indexlibrary{\idxcode{}!\idxcode{glyph_run}}
\begin{itemdecl}
\end{itemdecl}
\begin{itemdescr}
\pnum
\returns

\end{itemdescr}

\addtocounter{SectionDepthBase}{1}
%!TEX root = io2d.tex

\rSec0 [glyphrun.glyph] {Class \tcode{glyph_run::glyph}}

\rSec1 [glyphrun.glyph.synopsis] {\tcode{glyph_run::glyph} synopsis}

\begin{codeblock}
namespace std { namespace experimental { namespace io2d { inline namespace v1 {
} } } }
\end{codeblock}

\rSec1 [glyphrun.glyph.intro] {\tcode{glyph_run::glyph} Description}

\pnum
\indexlibrary{\idxcode{glyph_run::glyph}}
The \tcode{glyph_run::glyph} class describes ***FIXME***

\rSec1 [glyphrun.glyph.cons] {\tcode{glyph_run::glyph} constructors and assignment operators}

\indexlibrary{\idxcode{glyph_run::glyph}!constructor}
\begin{itemdecl}
\end{itemdecl}
\begin{itemdescr}
\pnum
\effects

\pnum
\throws
As specified in Error reporting (\ref{\iotwod.err.report}).

\pnum
\remarks

\pnum
\errors

\pnum
\realnotes

\end{itemdescr}

\rSec1 [glyphrun.glyph.modifiers] {\tcode{glyph_run::glyph} modifiers}

\indexlibrary{\idxcode{glyph_run::glyph}!\idxcode{}}
\indexlibrary{\idxcode{}!\idxcode{glyph_run::glyph}}
\begin{itemdecl}
\end{itemdecl}
\begin{itemdescr}
\pnum
\effects

\pnum
\throws
As specified in Error reporting (\ref{\iotwod.err.report}).

\pnum
\remarks

\pnum
\errors

\pnum
\realnotes

\end{itemdescr}

\rSec1 [glyphrun.glyph.observers] {\tcode{glyph_run::glyph} observers}

\indexlibrary{\idxcode{glyph_run::glyph}!\idxcode{}}
\indexlibrary{\idxcode{}!\idxcode{glyph_run::glyph}}
\begin{itemdecl}
\end{itemdecl}
\begin{itemdescr}
\pnum
\returns

\end{itemdescr}

%!TEX root = io2d.tex

\rSec0 [glyphrun.cluster] {Class \tcode{glyph_run::cluster}}

\rSec1 [glyphrun.cluster.synopsis] {\tcode{glyph_run::cluster} synopsis}

\begin{codeblock}
namespace std { namespace experimental { namespace io2d { inline namespace v1 {
} } } }
\end{codeblock}

\rSec1 [glyphrun.cluster.intro] {\tcode{glyph_run::cluster} Description}

\pnum
\indexlibrary{\idxcode{glyph_run::cluster}}
The \tcode{glyph_run::cluster} class describes ***FIXME***

\rSec1 [glyphrun.cluster.cons] {\tcode{glyph_run::cluster} constructors and assignment operators}

\indexlibrary{\idxcode{glyph_run::cluster}!constructor}
\begin{itemdecl}
\end{itemdecl}
\begin{itemdescr}
\pnum
\effects

\pnum
\throws
As specified in Error reporting (\ref{\iotwod.err.report}).

\pnum
\remarks

\pnum
\errors

\pnum
\realnotes

\end{itemdescr}

\rSec1 [glyphrun.cluster.modifiers] {\tcode{glyph_run::cluster} modifiers}

\indexlibrary{\idxcode{glyph_run::cluster}!\idxcode{}}
\indexlibrary{\idxcode{}!\idxcode{glyph_run::cluster}}
\begin{itemdecl}
\end{itemdecl}
\begin{itemdescr}
\pnum
\effects

\pnum
\throws
As specified in Error reporting (\ref{\iotwod.err.report}).

\pnum
\remarks

\pnum
\errors

\pnum
\realnotes

\end{itemdescr}

\rSec1 [glyphrun.cluster.observers] {\tcode{glyph_run::cluster} observers}

\indexlibrary{\idxcode{glyph_run::cluster}!\idxcode{}}
\indexlibrary{\idxcode{}!\idxcode{glyph_run::cluster}}
\begin{itemdecl}
\end{itemdecl}
\begin{itemdescr}
\pnum
\returns

\end{itemdescr}

\addtocounter{SectionDepthBase}{-1}
\addtocounter{SectionDepthBase}{-1}
