%!TEX root = io2d.tex
\rSec0 [\iotwod.meshpatch] {Class \tcode{mesh_patch}}

\rSec1 [\iotwod.meshpatch.intro] {Overview}

\pnum
\indexlibrary{\idxcode{mesh_patch}}%
The \tcode{mesh_patch} class describes one of several types of patches used in creating \tcode{brush} object's with a brush type of \tcode{brush_type::mesh}. 

\rSec1 [\iotwod.meshpatch.bicubic] {Bicubic tensor-product patch}

\pnum
The most general of the four types of patches is the bicubic tensor-product patch. It is used in creating \tcode{brush} objects of type \tcode{brush_type::mesh}.

\pnum
The patch is described by 16 points and four colors.

\pnum
For purposes of interpreting the formula that describes the patch, the points are arranged as follows: \\
$
~[~[~P03~P13~P23~P33~]~] \\
~[~[~P02~P12~P22~P32~]~] \\
~[~[~P01~P11~P12~P31~]~] \\
~[~[~P00~P10~P20~P30~]~]$

\pnum
The colors are arranged as follows: \\
$
~[~[~C1~C2~]~] \\
~[~[~C0~C3~]~]$

\pnum
The formula that describes the patch is: \\
$
\displaystyle q(u,v) = \sum_{i=0}^{3}\sum_{j=0}^{3}k_{ij}\times{}B_{i}(u)\times{}B_{j}(v)
$ \\
\\
with $k_{ij}$ being the point P\textit{i}\textit{j} above and with $B_{i}(u)$ and $B_{j}(v)$ being Bernstein basis polynomials.

\pnum
\begin{note}
For example, where \textit{i} is 0 and \textit{j} is 2, the point is P02.
\end{note}

\pnum
The inputs $u$ and $v$ are each a value in the range \orange{0}{1}, with $u$ and $v$ representing horizontal and vertical variation, respectively, within a unit square. Unlike in the standard coordinate space, vertical variation here is oriented from bottom to top.

\pnum
When sampling from the patch, the point that results from evaluating the patch at point $(u,v)$ has the color $Cr$, which is determined as follows: \\
$
Cr=(C0\times{}((1 - u)\times{}(1-v)))+(C1\times{}((1-u)\times{}v))+(C2\times{}(u\times{}v))+(C3\times{}(u\times{}(1-v)))$

\rSec1 [\iotwod.meshpatch.forms] {Other forms supported by \tcode{mesh_patch}}

\pnum
The \tcode{mesh_patch} class supports four types of patches. In addition to the 

\rSec1 [\iotwod.meshpatch.synopsis] {\tcode{mesh_patch} synopsis}

\begin{codeblock}
namespace @\fullnamespace{}@ {
  class bicubic_patch {
  public:
  	// \ref{\iotwod.bicubicpatch.cons}, construct:
    constexpr color_stop(float o, const rgba_color& c);
    
    // \ref{\iotwod.bicubicpatch.modifiers}, modifiers:
    constexpr void offset(float val) noexcept;
	constexpr void color(const rgba_color& val) noexcept;
	
    // \ref{\iotwod.bicubicpatch.observers}, observers:
	constexpr float offset() const noexcept;
	constexpr rgba_color color() const noexcept;
  };
}
\end{codeblock}

\rSec1 [\iotwod.bicubicpatch.cons]{\tcode{bicubic_patch} constructors}

\indexlibrary{\idxcode{bicubic_patch}!constructor}%
\begin{itemdecl}
constexpr bicubic_patch(float o, const rgba_color& c) noexcept;
\end{itemdecl}
\begin{itemdescr}
\pnum
\effects
Constructs a \tcode{color_stop} object.

\pnum
The offset shall be set to the value of \tcode{o}.

\pnum
The color shall be set to the value of \tcode{c}.
\end{itemdescr}

\rSec1 [\iotwod.bicubicpatch.modifiers]{\tcode{bicubic_patch} modifiers}

\indexlibrarymember{offset}{bicubic_patch}%
\begin{itemdecl}
constexpr void offset(float val) noexcept;
\end{itemdecl}
\begin{itemdescr}
\pnum
\effects
The offset shall be set to the value of \tcode{val}.
\end{itemdescr}

\rSec1 [\iotwod.bicubicpatch.observers]{\tcode{bicubic_patch} observers}

\indexlibrarymember{offset}{bicubic_patch}%
\begin{itemdecl}
constexpr float offset() const noexcept;
\end{itemdecl}
\begin{itemdescr}
\pnum
\returns
The value of the offset.
\end{itemdescr}
