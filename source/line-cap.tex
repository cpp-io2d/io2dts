%!TEX root = io2d.tex
\rSec0 [\iotwod.linecap] {Enum class \tcode{line_cap}}

\rSec1 [\iotwod.linecap.summary] {\tcode{line_cap} Summary}

\pnum
The \tcode{line_cap} enum class specifies how the ends of lines should be 
rendered when a \tcode{path_group} object is stroked. See 
Table~\ref{tab:\iotwod.linecap.meanings} for the meaning of each 
\tcode{line_cap} enumerator.

\rSec1 [\iotwod.linecap.synopsis] {\tcode{line_cap} Synopsis}

\begin{codeblock}
namespace std { namespace experimental { namespace io2d { inline namespace v1 {
  enum class line_cap {
    none,
    round,
    square
  };
} } } }
\end{codeblock}

\rSec1 [\iotwod.linecap.enumerators] {\tcode{line_cap} Enumerators}
\begin{libreqtab2}
 {\tcode{line_cap} enumerator meanings}
 {tab:\iotwod.linecap.meanings}
 \\ \topline
 \lhdr{Enumerator}
 & \rhdr{Meaning}
 \\ \capsep
 \endfirsthead
 \continuedcaption\\
 \hline
 \lhdr{Enumerator}
 & \rhdr{Meaning}
 \\ \capsep
 \endhead
 \tcode{none}
 & The line has no cap. It terminates exactly at the end point.
 \\
 \tcode{round}
 & The line has a circular cap, with the end point serving as the 
 center of the circle and the line width serving as its diameter.
 \\
 \tcode{square}
 & The line has a square cap, with the end point serving as the center 
 of the square and the line width serving as the length of each side.
 \\
\end{libreqtab2}
