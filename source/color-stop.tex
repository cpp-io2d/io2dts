%!TEX root = io2d.tex
\rSec0 [\iotwod.colorstop] {Class \tcode{color_stop}}

\rSec1 [\iotwod.colorstop.intro] {Overview}
\pnum
\indexlibrary{\idxcode{color_stop}}%
The class \tcode{color_stop} describes a color stop that is used by gradient brushes.

\pnum
It has an offset of type \tcode{float} and a color of type \tcode{rgba_color}.

\rSec1 [\iotwod.colorstop.synopsis] {\tcode{color_stop} synopsis}

\begin{codeblock}
namespace std::experimental::io2d::v1 {
  class color_stop {
  public:
    // \ref{\iotwod.colorstop.cons}, construct:
    constexpr color_stop() noexcept;
    constexpr color_stop(float o, const rgba_color& c) noexcept;
    
    // \ref{\iotwod.colorstop.modifiers}, modifiers:
    constexpr void offset(float o) noexcept;
    constexpr void color(const rgba_color& c) noexcept;
	
    // \ref{\iotwod.colorstop.observers}, observers:
    constexpr float offset() const noexcept;
    constexpr rgba_color color() const noexcept;
  };
  // \ref{\iotwod.colorstop.ops}, operators:
  constexpr bool operator==(const color_stop& lhs, const color_stop& rhs)
    noexcept;
  constexpr bool operator!=(const color_stop& lhs, const color_stop& rhs)
    noexcept;
}
\end{codeblock}

\rSec1 [\iotwod.colorstop.cons] {\tcode{color_stop} constructors}

\indexlibrary{\idxcode{color_stop}!constructor}%
\begin{itemdecl}
constexpr color_stop() noexcept;
\end{itemdecl}
\begin{itemdescr}
\pnum
\effects
Equivalent to: \tcode{color_stop(0.0f, rgba_color::transparent_black)}.
\end{itemdescr}

\indexlibrary{\idxcode{color_stop}!constructor}%
\begin{itemdecl}
constexpr color_stop(float o, const rgba_color& c) noexcept;
\end{itemdecl}
\begin{itemdescr}
\pnum
\requires
\tcode{o >= 0.0f} and \tcode{o <= 1.0f}.

\pnum
\effects
Constructs a \tcode{color_stop} object.

\pnum
The offset is \tcode{o}. The color is \tcode{c}.
\end{itemdescr}

\rSec1 [\iotwod.colorstop.modifiers] {\tcode{color_stop} modifiers}

\indexlibrarymember{offset}{color_stop}%
\begin{itemdecl}
constexpr void offset(float o) noexcept;
\end{itemdecl}
\begin{itemdescr}
\pnum
\requires
\tcode{o >= 0.0f} and \tcode{o <= 1.0f}.

\pnum
\effects
The offset is \tcode{o}.
\end{itemdescr}

\indexlibrarymember{color}{color_stop}%
\begin{itemdecl}
constexpr void color(const rgba_color& c) noexcept;
\end{itemdecl}
\begin{itemdescr}
\pnum
\effects
The color is \tcode{c}.
\end{itemdescr}

\rSec1 [\iotwod.colorstop.observers] {\tcode{color_stop} observers}

\indexlibrarymember{offset}{color_stop}%
\begin{itemdecl}
constexpr float offset() const noexcept;
\end{itemdecl}
\begin{itemdescr}
\pnum
\returns
The offset.
\end{itemdescr}

\indexlibrarymember{color}{color_stop}%
\begin{itemdecl}
constexpr rgba_color color() const noexcept;
\end{itemdecl}
\begin{itemdescr}
\pnum
\returns
The color.
\end{itemdescr}

\rSec1 [\iotwod.colorstop.ops] {\tcode{color_stop} operators}

\indexlibrarymember{operator==}{color_stop}%
\begin{itemdecl}
constexpr bool operator==(const color_stop& lhs, const color_stop& rhs)
  noexcept;
\end{itemdecl}
\begin{itemdescr}
\pnum
\returns
\tcode{lhs.offset() == rhs.offset() \&\& lhs.color() == rhs.color();}
\end{itemdescr}
