%!TEX root = io2d.tex
\rSec0 [\iotwod.gradientstop] {Class \tcode{gradient_stop}}

\rSec1 [\iotwod.gradientstop.intro] {Overview}
\pnum
\indexlibrary{\idxcode{gradient_stop}}%
The class \tcode{gradient_stop} describes a gradient stop that is used by gradient brushes.

\pnum
It has an \term{offset} of type \tcode{float} and an \term{offset color} of type \tcode{rgba_color}.

\rSec1 [\iotwod.gradientstop.synopsis] {\tcode{gradient_stop} synopsis}

\begin{codeblock}
namespace std::experimental::io2d::v1 {
  class gradient_stop {
  public:
    // \ref{\iotwod.gradientstop.cons}, construct:
    constexpr gradient_stop() noexcept;
    constexpr gradient_stop(float o, rgba_color c) noexcept;
    
    // \ref{\iotwod.gradientstop.modifiers}, modifiers:
    constexpr void offset(float o) noexcept;
    constexpr void color(rgba_color c) noexcept;
	
    // \ref{\iotwod.gradientstop.observers}, observers:
    constexpr float offset() const noexcept;
    constexpr rgba_color color() const noexcept;
  };
  // \ref{\iotwod.gradientstop.ops}, operators:
  constexpr bool operator==(const gradient_stop& lhs, const gradient_stop& rhs)
    noexcept;
  constexpr bool operator!=(const gradient_stop& lhs, const gradient_stop& rhs)
    noexcept;
}
\end{codeblock}

\rSec1 [\iotwod.gradientstop.cons] {\tcode{gradient_stop} constructors}

\indexlibrary{\idxcode{gradient_stop}!constructor}%
\begin{itemdecl}
constexpr gradient_stop() noexcept;
\end{itemdecl}
\begin{itemdescr}
\pnum
\effects
Equivalent to: \tcode{gradient_stop(0.0f, rgba_color::transparent_black)}.
\end{itemdescr}

\indexlibrary{\idxcode{gradient_stop}!constructor}%
\begin{itemdecl}
constexpr gradient_stop(float o, rgba_color c) noexcept;
\end{itemdecl}
\begin{itemdescr}
\pnum
\requires
\tcode{o >= 0.0f} and \tcode{o <= 1.0f}.

\pnum
\effects
Constructs a \tcode{gradient_stop} object.

\pnum
The offset is \tcode{o} rounded to the nearest multiple of \tcode{0.00001f}. The offset color is \tcode{c}.
\end{itemdescr}

\rSec1 [\iotwod.gradientstop.modifiers] {\tcode{gradient_stop} modifiers}

\indexlibrarymember{offset}{gradient_stop}%
\begin{itemdecl}
constexpr void offset(float o) noexcept;
\end{itemdecl}
\begin{itemdescr}
\pnum
\requires
\tcode{o >= 0.0f} and \tcode{o <= 1.0f}.

\pnum
\effects
The offset is \tcode{o} rounded to the nearest multiple of \tcode{0.00001f}.
\end{itemdescr}

\indexlibrarymember{color}{gradient_stop}%
\begin{itemdecl}
constexpr void color(rgba_color c) noexcept;
\end{itemdecl}
\begin{itemdescr}
\pnum
\effects
The offset color is \tcode{c}.
\end{itemdescr}

\rSec1 [\iotwod.gradientstop.observers] {\tcode{gradient_stop} observers}

\indexlibrarymember{offset}{gradient_stop}%
\begin{itemdecl}
constexpr float offset() const noexcept;
\end{itemdecl}
\begin{itemdescr}
\pnum
\returns
The offset.
\end{itemdescr}

\indexlibrarymember{color}{gradient_stop}%
\begin{itemdecl}
constexpr rgba_color color() const noexcept;
\end{itemdecl}
\begin{itemdescr}
\pnum
\returns
The offset color.
\end{itemdescr}

\rSec1 [\iotwod.gradientstop.ops] {\tcode{gradient_stop} operators}

\indexlibrarymember{operator==}{gradient_stop}%
\begin{itemdecl}
constexpr bool operator==(const gradient_stop& lhs, const gradient_stop& rhs)
  noexcept;
\end{itemdecl}
\begin{itemdescr}
\pnum
\returns
\tcode{lhs.offset() == rhs.offset() \&\& lhs.color() == rhs.color();}
\end{itemdescr}
