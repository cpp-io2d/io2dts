%!TEX root = io2d.tex
\rSec0 [antialias] {Enum class \tcode{antialias}}

\rSec1 [antialias.summary] {\tcode{antialias} Summary}

\pnum
The antialias enum class specifies the type of anti-aliasing that the rendering
system uses for rendering and composing paths. See 
Table~\ref{tab:antialias.meanings} for the meaning of each
\tcode{antialias} enumerator.

\rSec1 [antialias.synopsis] {\tcode{antialias} Synopsis}

\indexlibrary{\idxcode{antialias}}
\begin{codeblock}
namespace std { namespace experimental { namespace io2d { inline namespace v1 {
  enum class antialias {
    none,
    fast,
    good,
    best
  };
} } } }
\end{codeblock}

\rSec1 [antialias.enumerators] {\tcode{antialias} Enumerators}

\begin{libreqtab2}
 {\tcode{antialias} enumerator meanings}
 {tab:antialias.meanings}
 \\ \topline
 \lhdr{Enumerator}
 & \rhdr{Meaning}
 \\ \capsep
 \endfirsthead
 \continuedcaption\\
 \hline
 \lhdr{Enumerator}
 & \rhdr{Meaning}
 \\ \capsep
 \endhead
 & No anti-aliasing is performed.
 \\
 \tcode{fast}
 & Some form of anti-aliasing shall be used when this option is selected, but the form used is \impldefplain{antialiasing!fast}.
 \begin{note}
 By specifying this value, the user is hinting that faster anti-aliasing is 
 preferable to better anti-aliasing.
 \end{note}
 \\
 \tcode{good}
 & Some form of anti-aliasing shall be used when this option is selected, but the form used is \impldefplain{antialiasing!good}.
 \begin{note}
 By specifying this value, the user is hinting that sacrificing some performance 
 to obtain better anti-aliasing is acceptable but that performance is still a 
 concern.
 \end{note}
 \\
 \tcode{best}
 & Some form of anti-aliasing shall be used when this option is selected, but the form used is \impldefplain{antialiasing!best}.
 \begin{note}
 By specifying this value, the user is hinting that anti-aliasing is more 
 important than performance.
 \end{note}
 \\
\end{libreqtab2}
