%!TEX root = io2d.tex
\rSec0 [\iotwod.antialias] {Enum class \tcode{antialias}}

\rSec1 [\iotwod.antialias.summary] {\tcode{antialias} summary}

\pnum
The antialias enum class specifies the type of anti-aliasing that the rendering
system uses for rendering paths. See 
Table~\ref{tab:\iotwod.antialias.meanings} for the meaning of each
\tcode{antialias} enumerator.

\rSec1 [\iotwod.antialias.synopsis] {\tcode{antialias} synopsis}

\indexlibrary{\idxcode{antialias}}%
\begin{codeblock}
namespace @\fullnamespace{}@ {
  enum class antialias {
    none,
    fast,
    good,
    best
  };
}
\end{codeblock}

\rSec1 [\iotwod.antialias.enumerators] {\tcode{antialias} enumerators}

\begin{libreqtab2}
 {\tcode{antialias} enumerator meanings}
 {tab:\iotwod.antialias.meanings}
 \\ \topline
 \lhdr{Enumerator}
 & \rhdr{Meaning}
 \\ \capsep
 \endfirsthead
 \continuedcaption\\
 \hline
 \lhdr{Enumerator}
 & \rhdr{Meaning}
 \\ \capsep
 \endhead
 \tcode{none}
 & No anti-aliasing is performed when performing a rendering operation.
 \\ \rowsep
 \tcode{fast}
 & Some form of anti-aliasing should be used when performing a rendering operation but performance is more important than the quality of the results. The technique used is \impldef{antialiasing!fast}.
 \\ \rowsep
 \tcode{good}
 & Some form of anti-aliasing should be used when performing a rendering operation and the sacrificing some performance to obtain better anti-aliasing results than would likely be obtained from \tcode{antialias::fast} is acceptable. The technique used is \impldef{antialiasing!good}.
 \\ \rowsep
 \tcode{best}
 & Some form of anti-aliasing should be used when performing a rendering operation and better anti-aliasing results than would likely be obtained from \tcode{antialias::fast} and \tcode{antialias::good} are desired even if performance degrades significantly. The technique used is \impldef{antialiasing!best}.
 \begin{note}
 This might commonly be chosen when a user is going to render something once and cache the results for repeated use or when a user is rendering something that does not necessarily need performance suitable for real-time computer graphics applications.
 \end{note}
 \\
\end{libreqtab2}
