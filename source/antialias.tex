%!TEX root = io2d.tex
\rSec0 [\iotwod.antialias] {Enum class \tcode{antialias}}

\rSec1 [\iotwod.antialias.summary] {\tcode{antialias} Summary}

\pnum
The antialias enum class specifies the type of anti-aliasing that the rendering
system shall use for rendering text. See 
Table~\ref{tab:\iotwod.antialias.meanings} for the meaning of each
\tcode{antialias} enumerator.

\rSec1 [\iotwod.antialias.synopsis] {\tcode{antialias} Synopsis}

\indexlibrary{\idxcode{antialias}}
\begin{codeblock}
namespace std { namespace experimental { namespace io2d { inline namespace v1 {
  enum class antialias {
    default_antialias,
    none,
    gray,
    subpixel,
    fast,
    good,
    best
  };
} } } }
\end{codeblock}

\rSec1 [\iotwod.antialias.enumerators] {\tcode{antialias} Enumerators}

\begin{libreqtab2}
 {\tcode{antialias} enumerator meanings}
 {tab:\iotwod.antialias.meanings}
 \\ \topline
 \lhdr{Enumerator}
 & \rhdr{Meaning}
 \\ \capsep
 \endfirsthead
 \continuedcaption\\
 \hline
 \lhdr{Enumerator}
 & \rhdr{Meaning}
 \\ \capsep
 \endhead
 \tcode{default_antialias}
 & The meaning of this value is \impldef{antialiasing!default}.
 \\
 \tcode{none}
 & No anti-aliasing.
 \\
 \tcode{gray}
 & Monochromatic anti-aliasing.
 \enternote
 When rendering black text on a white background, this would produce gray-scale 
 \\
 \tcode{subpixel}
 & Anti-aliasing that breaks pixels into their constituent color channels and 
 manipulates those color channels individually. The meaning of this value for 
 any rendering operation other than \tcode{surface::show_text}, 
 \tcode{surface::show_glyphs}, and \tcode{surface::show_text_glyphs} is 
 \impldef{antialias!subpixel}.
 \\
 \tcode{fast}
 & The meaning of this value is \impldef{antialiasing!fast}. Implementations 
 shall enable some form of anti-aliasing when this option is selected.
 \enternote
 By choosing this value, the user is hinting that faster anti-aliasing is 
 preferable to better anti-aliasing.
 \exitnote
 \\
 \tcode{good}
 & The meaning of this value is \impldef{antialiasing!good}. Implementations 
 shall enable some form of anti-aliasing when this option is selected.
 \enternote
 By choosing this value, the user is hinting that sacrificing some performance 
 to obtain better anti-aliasing is acceptable but that performance is still a 
 concern.
 \\
 \tcode{best}
 & The meaning of this value is \impldef{antialiasing!best}. Implementations 
 shall enable some form of text anti-aliasing when this option is selected.
 \enternote
 By choosing this value, the user is hinting that better anti-aliasing is more 
 important than performance.
 \\
\end{libreqtab2}
