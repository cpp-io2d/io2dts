%!TEX root = io2d.tex
\rSec0 [\iotwod.antialias] {Enum class \tcode{antialias}}

\rSec1 [\iotwod.antialias.summary] {\tcode{antialias} Summary}

\pnum
The antialias enum class specifies the type of antialiasing that the rendering
system is requested to use for rendering \tcode{surface} and 
\tcode{pattern} objects, and for rendering text. See 
Table~\ref{tab:\iotwod.antialias.meanings} for the meaning of each
\tcode{antialias} enumerator.

\pnum
\enternote
The value \tcode{antialias::subpixel} is only expected to be meaningful in the 
context of rendering text.
\exitnote

\rSec1 [\iotwod.antialias.synopsis] {\tcode{antialias} Synopsis}

\indexlibrary{\idxcode{antialias}}
\begin{codeblock}
namespace std { namespace experimental { namespace io2d { inline namespace v1 {
  enum class antialias {
    default_antialias,
    none,
    gray,
    subpixel,
    fast,
    good,
    best
  };
} } } } // namespaces std::experimental::io2d::v1
\end{codeblock}

\rSec1 [\iotwod.antialias.enumerators] {\tcode{antialias} Enumerators}

\begin{libreqtab2}
 {\tcode{antialias} enumerator meanings}
 {tab:\iotwod.antialias.meanings}
 \\ \topline
 \lhdr{Enumerator}
 & \rhdr{Meaning}
 \\ \capsep
 \endfirsthead
 \continuedcaption\\
 \hline
 \lhdr{Enumerator}
 & \rhdr{Meaning}
 \\ \capsep
 \endhead
 \tcode{default_antialias}
 & \impldef{antialiasing!default}.
 \\
 \tcode{none}
 & No antialiasing.
 \\
 \tcode{gray}
 & Monochromatic antialiasing.
 \\
 \tcode{subpixel}
 & Antialiasing that breaks pixels into their constituent color channels and 
 manipulates those color channels individually. The meaning of this value for 
 any rendering operation other than \tcode{surface::show_text}, 
 \tcode{surface::show_glyphs}, and \tcode{surface::show_text_glyphs} is 
 \impldef{antialias!subpixel}.
 \\
 \tcode{fast}
 & \impldef{antialiasing!fast}.
 \enternote
 By choosing this value, the user is hinting that some antialiasing is
 desired but that performance is more important that appearance.
 \exitnote
 \\
 \tcode{good}
 & \impldef{antialiasing!good}.
 \enternote
 By choosing this value, the user is hinting that antialiasing is
 desired and that sacrificing some performance to obtain antialiasing
 is acceptable. Implementations that provide antialiasing shall
 enable some form of antialiasing when this option is selected unless
 there is a compelling performance reason not to do so.
 \\
 \tcode{best}
 & \impldef{antialiasing!best}.
 \enternote
 By choosing this value, the user is hinting that antialiasing is
 more important than performance. Implementations that provide
 antialiasing shall enable some form of antialiasing when this
 option is selected, preferably the form that the implementor
 believes to be the form that generally provides the best visual
 results.
 \\
\end{libreqtab2}
