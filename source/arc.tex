%!TEX root = io2d.tex
\rSec0 [\iotwod.arc] {Class \tcode{arc}}

\rSec1 [\iotwod.arc.general] {In general}

\pnum
\indexlibrary{\idxcode{arc}}%
The class \tcode{arc} describes a figure item that is a segment.

\pnum
It has a \term{radius} of type \tcode{basic_point_2d}, a \term{rotation} of type \tcode{float}, and a \term{start angle} of type \tcode{float}.

\pnum
It forms a portion of the circumference of a circle. The centre of the circle is implied by the start point, the radius and the start angle of the arc.

\rSec1 [\iotwod.arc.cons] {\tcode{arc} constructors}

\indexlibrary{\idxcode{arc}!constructor}%
\begin{itemdecl}
arc() noexcept;
\end{itemdecl}
\begin{itemdescr}
\pnum
\effects
Equivalent to: \tcode{arc\{ basic_point_2d(10.0f, 10.0f), pi<float>, pi<float> \};}.
\end{itemdescr}

\indexlibrary{\idxcode{arc}!constructor}%
\begin{itemdecl}
arc(const basic_point_2d<typename GraphicsSurfaces::graphics_math_type>& rad,
  float rot, float sang) noexcept;
\end{itemdecl}
\begin{itemdescr}
\pnum
\effects
Constructs an object of type \tcode{arc}.

\pnum
The radius is \tcode{rad}.

\pnum
The rotation is \tcode{rot}.

\pnum
The start angle is \tcode{sang}.
\end{itemdescr}

\indexlibrary{\idxcode{arc}!constructor}%
\begin{itemdecl}
arc(const arc& other);
arc(arc&& other) noexcept;
\end{itemdecl}
\begin{itemdescr}
\pnum
\effects
Constructs an object of type \tcode{arc}. In the second form, other is left in a valid state with an unspecified value.

\pnum
The radius is \tcode{other.radius()}.

\pnum
The rotation is \tcode{other.rotation()}.

\pnum
The start angle is \tcode{other.start_angle()}.
\end{itemdescr}

\rSec1 [\iotwod.arc.assign] {\tcode{arc} assignment operators}

\indexlibrary{\idxcode{arc}!assignment}%
\begin{itemdecl}
arc& operator=(const arc& other);
\end{itemdecl}
\begin{itemdescr}
\pnum
\effects
If \tcode{*this} and \tcode{other} are not the same object, modifies \tcode{*this} such that \tcode{*this.radius()} is \tcode{other.radius()}, \tcode{*this.rotation()} is \tcode{other.rotation()} and \tcode{*this.start_angle()} is \tcode{other.start_angle()}

\pnum
If \tcode{*this} and \tcode{other} are the same object, the member has no effect.

\pnum
\returns
\tcode{*this}
\end{itemdescr}

\indexlibrary{\idxcode{arc}!assignment}%
\begin{itemdecl}
arc& operator=(arc&& other) noexcept;
\end{itemdecl}
\begin{itemdescr}
\pnum
\effects
<TODO>

\pnum
\returns
\tcode{*this}
\end{itemdescr}

\rSec1 [\iotwod.arc.modifiers]{\tcode{arc} modifiers}

\indexlibrarymember{radius}{arc}%
\begin{itemdecl}
void radius(const basic_point_2d<typename GraphicsSurfaces::graphics_math_type>& rad) noexcept;
\end{itemdecl}
\begin{itemdescr}
\pnum
\effects
The radius is \tcode{rad}.
\end{itemdescr}

\indexlibrarymember{rotation}{arc}%
\begin{itemdecl}
constexpr void rotation(float rot) noexcept;
\end{itemdecl}
\begin{itemdescr}
\pnum
\effects
The rotation is \tcode{rot}.
\end{itemdescr}

\indexlibrarymember{start_angle}{arc}%
\begin{itemdecl}
void start_angle(float sang) noexcept;
\end{itemdecl}
\begin{itemdescr}
\pnum
\effects
The start angle is \tcode{sang}.
\end{itemdescr}

\rSec1 [\iotwod.arc.observers]{\tcode{arc} observers}

\indexlibrarymember{radius}{arc}%
\begin{itemdecl}
basic_point_2d<typename GraphicsSurfaces::graphics_math_type> radius() const noexcept;
\end{itemdecl}
\begin{itemdescr}
\pnum
\returns
The radius.
\end{itemdescr}

\indexlibrarymember{rotation}{arc}%
\begin{itemdecl}
float rotation() const noexcept;
\end{itemdecl}
\begin{itemdescr}
\pnum
\returns
The rotation.
\end{itemdescr}

\indexlibrarymember{start_angle}{arc}%
\begin{itemdecl}
float start_angle() const noexcept;
\end{itemdecl}
\begin{itemdescr}
\pnum
\returns
The start angle.
\end{itemdescr}

\indexlibrarymember{center}{arc}%
\begin{itemdecl}
basic_point_2d<typename GraphicsSurfaces::graphics_math_type> center(const basic_point_2d<typename
  GraphicsSurfaces::graphics_math_type>& cpt, const basic_matrix_2d<typename
  GraphicsSurfaces::graphics_math_type>& m = basic_matrix_2d<typename
  GraphicsSurfaces::graphics_math_type>{}) const noexcept;
\end{itemdecl}
\begin{itemdescr}
\pnum
\returns
As-if:
\begin{codeblock}
auto lmtx = m;
lmtx.m20 = 0.0f;
lmtx.m21 = 0.0f;
auto centerOffset = point_for_angle(two_pi<float> - start_angle(), radius());
centerOffset.y = -centerOffset.y;
return cpt - centerOffset * lmtx;
\end{codeblock}
\end{itemdescr}

\indexlibrarymember{start_angle}{arc}%
\begin{itemdecl}
basic_point_2d<typename GraphicsSurfaces::graphics_math_type> end_pt(const basic_point_2d<typename
  GraphicsSurfaces::graphics_math_type>& cpt, const basic_matrix_2d<typename
  GraphicsSurfaces::graphics_math_type>& m = basic_matrix_2d<typename
  GraphicsSurfaces::graphics_math_type>{}) const noexcept;
\end{itemdecl}
\begin{itemdescr}
\pnum
\returns
As-if:
\begin{codeblock}
auto lmtx = m;
auto tfrm = matrix_2d::init_rotate(start_angle() + rotation());
lmtx.m20 = 0.0f;
lmtx.m21 = 0.0f;
auto pt = (radius() * tfrm);
pt.y = -pt.y;
return cpt + pt * lmtx;
\end{codeblock}
\end{itemdescr}

\rSec1 [\iotwod.arc.ops]{\tcode{arc} operators}

\indexlibrarymember{operator==}{arc}%
\begin{itemdecl}
template <class GraphicsSurfaces>
bool operator==(const typename basic_figure_items<GraphicsSurfaces>::arc& lhs,
  const typename basic_figure_items<GraphicsSurfaces>::arc& rhs) noexcept;
\end{itemdecl}
\begin{itemdescr}
\pnum
\returns
\begin{codeblock}
lhs.radius() == rhs.radius() && lhs.rotation() == rhs.rotation() &&
lhs.start_angle() && rhs.start_angle()
\end{codeblock}
\end{itemdescr}
