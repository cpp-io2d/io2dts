%!TEX root = io2d.tex
\rSec0 [\iotwod.arc] {Class \tcode{arc}}

\rSec1 [\iotwod.arc.general] {In general}

\pnum
\indexlibrary{\idxcode{arc}}%
The class \tcode{arc} describes a path item that is a path segment.

\pnum
It has a \term{radius} of type \tcode{vector_2d}, a \term{rotation} of type \tcode{float}, and a \term{start angle} of type \tcode{float}.

\rSec1 [\iotwod.arc.synopsis] {\tcode{arc} synopsis}

\begin{codeblock}
namespace std::experimental::io2d::v1 {
  namespace path_data {
    class arc {
    public:
      // \ref{\iotwod.arc.cons}, construct/copy/move/destroy:
      constexpr arc() noexcept;
      constexpr arc(vector_2d rad,
        float rot, float sang) noexcept;

      // \ref{\iotwod.arc.modifiers}, modifiers:
      constexpr void radius(vector_2d rad) noexcept;
      constexpr void rotation(float rot) noexcept;
      constexpr void start_angle(float radians) noexcept;

      // \ref{\iotwod.arc.observers}, observers:
      constexpr vector_2d radius() const noexcept;
      constexpr float rotation() const noexcept;
      constexpr float start_angle() const noexcept;
      vector_2d center(vector_2d cpt, const matrix_2d& m = matrix_2d{}) 
        const noexcept;
      vector_2d end_pt(vector_2d cpt, const matrix_2d& m = matrix_2d{}) 
        const noexcept;
    };
    
    // \ref{\iotwod.arc.ops}, operators:
    constexpr bool operator==(const arc& lhs, const arc& rhs) noexcept;
    constexpr bool operator!=(const arc& lhs, const arc& rhs) noexcept;
  }
}
\end{codeblock}

\rSec1 [\iotwod.arc.cons] {\tcode{arc} constructors}

\indexlibrary{\idxcode{arc}!constructor}%
\begin{itemdecl}
constexpr arc() noexcept;
\end{itemdecl}
\begin{itemdescr}
\pnum
\effects
Equivalent to: \tcode{arc\{ vector_2d(10.0f, 10.0f), pi<float>, pi<float> \};}.
\end{itemdescr}

\indexlibrary{\idxcode{arc}!constructor}%
\begin{itemdecl}
constexpr arc(vector_2d rad, float rot,
  float start_angle = pi<float>) noexcept;
\end{itemdecl}
\begin{itemdescr}
\pnum
\effects
Constructs an object of type \tcode{arc}.

\pnum
The radius is \tcode{rad}.

\pnum
The rotation is \tcode{rot}.

\pnum
The start angle is \tcode{sang}.
\end{itemdescr}

\rSec1 [\iotwod.arc.modifiers]{\tcode{arc} modifiers}

\indexlibrarymember{radius}{arc}%
\begin{itemdecl}
constexpr void radius(vector_2d rad) noexcept;
\end{itemdecl}
\begin{itemdescr}
\pnum
\effects
The radius is \tcode{rad}.
\end{itemdescr}

\indexlibrarymember{rotation}{arc}%
\begin{itemdecl}
constexpr void rotation(float rot) noexcept;
\end{itemdecl}
\begin{itemdescr}
\pnum
\effects
The rotation is \tcode{rot}.
\end{itemdescr}

\indexlibrarymember{start_angle}{arc}%
\begin{itemdecl}
constexpr void start_angle(float sang) noexcept;
\end{itemdecl}
\begin{itemdescr}
\pnum
\effects
The start angle is \tcode{sang}.
\end{itemdescr}

\rSec1 [\iotwod.arc.observers]{\tcode{arc} observers}

\indexlibrarymember{radius}{arc}%
\begin{itemdecl}
constexpr vector_2d radius() const noexcept;
\end{itemdecl}
\begin{itemdescr}
\pnum
\returns
The radius.
\end{itemdescr}

\indexlibrarymember{rotation}{arc}%
\begin{itemdecl}
constexpr float rotation() const noexcept;
\end{itemdecl}
\begin{itemdescr}
\pnum
\returns
The rotation.
\end{itemdescr}

\indexlibrarymember{start_angle}{arc}%
\begin{itemdecl}
constexpr float start_angle() const noexcept;
\end{itemdecl}
\begin{itemdescr}
\pnum
\returns
The start angle.
\end{itemdescr}

\indexlibrarymember{center}{arc}%
\begin{itemdecl}
vector_2d center(vector_2d cpt, const matrix_2d& m = matrix_2d{})
  const noexcept;
\end{itemdecl}
\begin{itemdescr}
\pnum
\returns
As-if:
\begin{codeblock}
auto lmtx = m;
lmtx.m20 = 0.0f;
lmtx.m21 = 0.0f;
auto centerOffset = point_for_angle(two_pi<float> - _Start_angle, _Radius);
centerOffset.y(-centerOffset.y());
return cpt - centerOffset * lmtx;
\end{codeblock}
\end{itemdescr}

\indexlibrarymember{start_angle}{arc}%
\begin{itemdecl}
vector_2d end_pt(vector_2d cpt, const matrix_2d& m = matrix_2d{})
  const noexcept;
\end{itemdecl}
\begin{itemdescr}
\pnum
\returns
As-if:
\begin{codeblock}
auto lmtx = m;
auto tfrm = matrix_2d::init_rotate(_Start_angle + _Rotation);
lmtx.m20 = 0.0f;
lmtx.m21 = 0.0f;
auto pt = (_Radius * tfrm);
pt.y(-pt.y());
return cpt + pt * lmtx;
\end{codeblock}
\end{itemdescr}

\rSec1 [\iotwod.arc.ops]{\tcode{arc} operators}

\indexlibrarymember{operator==}{arc}%
\begin{itemdecl}
constexpr bool operator==(const arc& lhs, const arc& rhs) noexcept;
\end{itemdecl}
\begin{itemdescr}
\pnum
\returns
\begin{codeblock}
lhs.radius() == rhs.radius() && lhs.rotation() == rhs.rotation() &&
lhs.start_angle() && rhs.start_angle()
\end{codeblock}
\end{itemdescr}
