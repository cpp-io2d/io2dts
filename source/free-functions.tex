%!TEX root = io2d.tex
\rSec0 [\iotwod.standalone] {Standalone functions}

\rSec1 [\iotwod.standalone.synopsis] {Standalone functions synopsis}

\begin{codeblock}
namespace std::experimental::io2d::v1 {
  template <class GraphicsSurfaces>
  basic_image_surface<GraphicsSurfaces> copy_surface(
    basic_image_surface<GraphicsSurfaces>& sfc) noexcept;
  template <class GraphicsSurfaces>
  basic_image_surface<GraphicsSurfaces> copy_surface(
    basic_output_surface<GraphicsSurfaces>& sfc) noexcept;
  template <class T>
  constexpr T degrees_to_radians(T d) noexcept;
  template <class T>
  constexpr T radians_to_degrees(T r) noexcept;
  float angle_for_point(point_2d ctr, point_2d pt) noexcept;
  point_2d point_for_angle(float ang, float rad = 1.0f) noexcept;
  point_2d point_for_angle(float ang, point_2d rad) noexcept;
  point_2d arc_start(point_2d ctr, float sang, point_2d rad, 
    const matrix_2d& m = matrix_2d{}) noexcept;
  point_2d arc_center(point_2d cpt, float sang, point_2d rad, 
    const matrix_2d& m = matrix_2d{}) noexcept;
  point_2d arc_end(point_2d cpt, float eang, point_2d rad, 
    const matrix_2d& m = matrix_2d{}) noexcept;
}
\end{codeblock}

\rSec1 [\iotwod.standalone.copysurface] {\tcode{copy_surface}}

\indexlibrary{\idxcode{copy_surface}}%
\begin{itemdecl}
template <class GraphicsSurfaces>
basic_image_surface<GraphicsSurfaces> copy_surface(
  basic_image_surface<GraphicsSurfaces>& sfc) noexcept;
template <class GraphicsSurfaces>
basic_image_surface<GraphicsSurfaces> copy_surface(
  basic_output_surface<GraphicsSurfaces>& sfc) noexcept;
\end{itemdecl}
\begin{itemdescr}
\pnum
\returns
\tcode{GraphicsSurfaces::surfaces::copy_surface(sfc)}.
\end{itemdescr}

\rSec1 [\iotwod.standalone.degtorad] {\tcode{degrees_to_radians}}
\begin{itemdecl}
template <class T>
constexpr T degrees_to_radians(T d) noexcept;
\end{itemdecl}
\begin{itemdescr}
\returns
If \tcode{d} is positive and is less than one thousandth of a degree, then \tcode{static_cast<T>(0)}. If \tcode{d} is negative and is less than one thousandth of a degree, then \tcode{-static_cast<T>(0)}. Otherwise, the value obtained from converting the degrees value \tcode{d} to radians.

\remarks
This function shall not participate in overload resolution unless \tcode{T} is a floating-point type.
\end{itemdescr}

\rSec1 [\iotwod.standalone.radtodeg] {\tcode{radians_to_degrees}}
\begin{itemdecl}
template <class T>
constexpr T radians_to_degrees(T r) noexcept;
\end{itemdecl}
\begin{itemdescr}
\returns
If \tcode{r} is positive and is less than one thousandth of a degree in radians, then \tcode{static_cast<T>(0)}. If \tcode{r} is negative and is less than one thousandth of a degree in radians, then \tcode{-static_cast<T>(0)}. Otherwise, the value obtained from converting the radians value \tcode{r} to degrees.

\remarks
This function shall not participate in overload resolution unless \tcode{T} is a floating-point type.
\end{itemdescr}

\rSec1 [\iotwod.standalone.angleforpoint] {\tcode{angle_for_point}}

\indexlibrary{\idxcode{angle_for_point}}%
\begin{itemdecl}
float angle_for_point(point_2d ctr, point_2d pt) noexcept;
\end{itemdecl}
\begin{itemdescr}
\pnum
\returns
The angle, in radians, of \tcode{pt} as a point on a circle with a center at \tcode{ctr}. If the angle is less that \tcode{pi<float> / 180000.0f}, returns \tcode{0.0f}.
\end{itemdescr}

\rSec1 [\iotwod.standalone.pointforangle] {\tcode{point_for_angle}}

\indexlibrary{\idxcode{point_for_angle}}%
\begin{itemdecl}
point_2d point_for_angle(float ang, float rad = 1.0f) noexcept;
point_2d point_for_angle(float ang, point_2d rad) noexcept;
\end{itemdecl}
\begin{itemdescr}
\pnum
\requires
If it is a \tcode{float}, \tcode{rad} is greater than \tcode{0.0f}. If it is a \tcode{point_2d}, \tcode{rad.x} or \tcode{rad.y} is greater than \tcode{0.0f} and neither is less than \tcode{0.0f}.

\pnum
\returns
The result of rotating the point \tcode{point_2d\{ 1.0f, 0.0f \}}, around an origin of \tcode{point_2d\{ 0.0f, 0.0f \}} by \tcode{ang} radians, with a positive value of \tcode{ang} meaning counterclockwise rotation and a negative value meaning clockwise rotation, with the result being multiplied by \tcode{rad}.
\end{itemdescr}

\rSec1 [\iotwod.standalone.arcstart] {\tcode{arc_start}}

\indexlibrary{\idxcode{arc_start}}%
\begin{itemdecl}
point_2d arc_start(point_2d ctr, float sang, point_2d rad, 
  const matrix_2d& m = matrix_2d{}) noexcept;
\end{itemdecl}
\begin{itemdescr}
\pnum
\requires
\tcode{rad.x} and \tcode{rad.y} are both greater than \tcode{0.0f}.

\pnum
\returns
As-if:
\begin{codeblock}
auto lmtx = m;
lmtx.m20 = 0.0f; lmtx.m21 = 0.0f;
auto pt = point_for_angle(sang, rad);
return ctr + pt * lmtx;
\end{codeblock}

\pnum
\begin{note}
Among other things, this function is useful for determining the point at which a new figure should begin if the first item in the figure is an arc and the user wishes to clearly define its center.
\end{note}
\end{itemdescr}

\rSec1 [\iotwod.standalone.arccenter] {\tcode{arc_center}}

\indexlibrary{\idxcode{arc_center}}%
\begin{itemdecl}
point_2d arc_center(point_2d cpt, float sang, point_2d rad, 
  const matrix_2d& m = matrix_2d{}) noexcept;
\end{itemdecl}
\begin{itemdescr}
\pnum
\requires
\tcode{rad.x} and \tcode{rad.y} are both greater than \tcode{0.0f}.

\pnum
\returns
As-if:
\begin{codeblock}
auto lmtx = m;
lmtx.m20 = 0.0f; lmtx.m21 = 0.0f;
auto centerOffset = point_for_angle(two_pi<float> - sang, rad);
centerOffset.y = -centerOffset.y;
return cpt - centerOffset * lmtx;
\end{codeblock}
\end{itemdescr}

\rSec1 [\iotwod.standalone.arcend] {\tcode{arc_end}}

\indexlibrary{\idxcode{arc_end}}%
\begin{itemdecl}
point_2d arc_end(point_2d cpt, float eang, point_2d rad, 
  const matrix_2d& m = matrix_2d{}) noexcept;
\end{itemdecl}
\begin{itemdescr}
\pnum
\requires
\tcode{rad.x} and \tcode{rad.y} are both greater than \tcode{0.0f}.

\pnum
\returns
As-if:
\begin{codeblock}
auto lmtx = m;
auto tfrm = matrix_2d::init_rotate(eang);
lmtx.m20 = 0.0f; lmtx.m21 = 0.0f;
auto pt = (rad * tfrm);
pt.y = -pt.y;
return cpt + pt * lmtx;
\end{codeblock}
\end{itemdescr}
