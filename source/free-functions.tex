%!TEX root = io2d.tex
\rSec0 [\iotwod.standalone] {Standalone functions}

\rSec1 [\iotwod.standalone.synopsis] {Standalone functions synopsis}

\begin{codeblock}
namespace std { namespace experimental { namespace io2d { inline namespace v1 {
  int format_stride_for_width(format format, int width) noexcept;
  display_surface make_display_surface(int preferredWidth,
    int preferredHeight, format preferredFormat,
    scaling scl = scaling::letterbox,
    refresh_rate rr = refresh_rate::as_fast_as_possible, double fps = 30.0);
  display_surface make_display_surface(int preferredWidth,
    int preferredHeight, format preferredFormat, error_code& ec,
    scaling scl = scaling::letterbox,
    refresh_rate rr = refresh_rate::as_fast_as_possible, double fps = 30.0) noexcept;
  display_surface make_display_surface(int preferredWidth,
    int preferredHeight, format preferredFormat, int preferredDisplayWidth, 
    int preferredDisplayHeight, scaling scl = scaling::letterbox,
    refresh_rate rr = refresh_rate::as_fast_as_possible, double fps = 30.0);
  display_surface make_display_surface(int preferredWidth,
    int preferredHeight, format preferredFormat, int preferredDisplayWidth, 
    int preferredDisplayHeight, ::std::error_code& ec,
    scaling scl = scaling::letterbox,
    refresh_rate rr = refresh_rate::as_fast_as_possible, double fps = 30.0) noexcept;
  image_surface make_image_surface(format format, int width, int height);
  image_surface make_image_surface(format format, int width, int height, 
    error_code& ec) noexcept;
} } } } // namespaces std::experimental::io2d::v1
\end{codeblock}

\rSec1 [\iotwod.standalone.formatstrideforwidth] {\tcode{format_stride_for_width}}

\indexlibrary{\idxcode{format_stride_for_width}}
\begin{itemdecl}
int format_stride_for_width(format fmt, int width) noexcept;
\end{itemdecl}
\begin{itemdescr}
\pnum
\returns
The size in bytes of a row of pixels with a visual data format of \tcode{fmt} that is \tcode{width} pixels wide. This value may be larger than the value obtained by multiplying the number of bytes specified by the \tcode{format} enumerator specified by \tcode{fmt} by the number of pixels specified by \tcode{width}.

\pnum
If \tcode{fmt == format::invalid}, this function shall return \tcode{0}.
\end{itemdescr}

\rSec1 [\iotwod.standalone.makedisplaysurface] {\tcode{make_display_surface}}

\indexlibrary{\idxcode{make_display_surface}}
\begin{itemdecl}
display_surface make_display_surface(int preferredWidth,
  int preferredHeight, format preferredFormat,
  scaling scl = scaling::letterbox,
  refresh_rate rr = refresh_rate::as_fast_as_possible, double fps = 30.0);
display_surface make_display_surface(int preferredWidth,
  int preferredHeight, format preferredFormat, error_code& ec,
  scaling scl = scaling::letterbox,
  refresh_rate rr = refresh_rate::as_fast_as_possible, double fps = 30.0) 
  noexcept;
display_surface make_display_surface(int preferredWidth,
  int preferredHeight, format preferredFormat, int preferredDisplayWidth, 
  int preferredDisplayHeight, scaling scl = scaling::letterbox,
  refresh_rate rr = refresh_rate::as_fast_as_possible, double fps = 30.0);
display_surface make_display_surface(int preferredWidth,
  int preferredHeight, format preferredFormat, int preferredDisplayWidth, 
  int preferredDisplayHeight, ::std::error_code& ec,
  scaling scl = scaling::letterbox,
  refresh_rate rr = refresh_rate::as_fast_as_possible, double fps = 30.0) 
  noexcept;
\end{itemdecl}
\begin{itemdescr}
\pnum
\returns
Returns a \tcode{display_surface} object that is exactly the same as-if the equivalent \tcode{display_surface} constructor was called with the same arguments.

\pnum
\throws
As specified in Error reporting (\ref{\iotwod.err.report}).

\pnum
\errors
The errors, if any, produced by this function are the same as the errors for the equivalent \tcode{display_surface} constructor (\ref{displaysurface.cons}).
\end{itemdescr}

\rSec1 [\iotwod.standalone.makeimagesurface] {\tcode{make_image_surface}}

\indexlibrary{\idxcode{make_image_surface}}
\begin{itemdecl}
image_surface make_image_surface(int width, int height,
  format fmt = format::argb32);
image_surface make_image_surface(int width, int height,
  error_code& ec, format fmt = format::argb32) noexcept;
\end{itemdecl}
\begin{itemdescr}
\pnum
\returns
Returns an \tcode{image_surface} object that is exactly the same as-if the \tcode{image_surface} constructor was called with the same arguments.

\pnum
\throws
As specified in Error reporting (\ref{\iotwod.err.report}).

\pnum
\errors
The errors, if any, produced by this function are the same as the errors for the equivalent \tcode{display_surface} constructor (\ref{imagesurface.cons}).
\end{itemdescr}
