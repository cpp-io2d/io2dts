%!TEX root = io2d.tex
\rSec0 [\iotwod.absquadraticcurve] {Class \tcode{abs_quadratic_curve}}

\pnum
\indexlibrary{\idxcode{abs_quadratic_curve}}%
The class \tcode{abs_quadratic_curve} describes a path item that is a path segment.

\pnum
It has a \term{control point} of type \tcode{vector_2d} and an \term{end point} of type \tcode{vector_2d}.

\rSec1 [\iotwod.absquadraticcurve.synopsis] {\tcode{abs_quadratic_curve} synopsis}

\begin{codeblock}
namespace std::experimental::io2d::v1 {
  namespace path_data {
    class abs_quadratic_curve {
    public:
      // \ref{\iotwod.absquadraticcurve.cons}, construct:
      constexpr abs_quadratic_curve() noexcept;
      constexpr abs_quadratic_curve(const vector_2d& cpt, const vector_2d& ept)
        noexcept;

      // \ref{\iotwod.absquadraticcurve.modifiers}, modifiers:
      constexpr void control_pt(const vector_2d& cpt) noexcept;
      constexpr void end_pt(const vector_2d& ept) noexcept;

      // \ref{\iotwod.absquadraticcurve.observers}, observers:
      constexpr vector_2d control_pt() const noexcept;
      constexpr vector_2d end_pt() const noexcept;
    };
    
    // \ref{\iotwod.absquadraticcurve.ops}, operators:
    constexpr bool operator==(const abs_quadratic_curve& lhs,
      const abs_quadratic_curve& rhs) noexcept;
    constexpr bool operator!=(const abs_quadratic_curve& lhs,
      const abs_quadratic_curve& rhs) noexcept;
  }
}
\end{codeblock}

\rSec1 [\iotwod.absquadraticcurve.cons] {\tcode{abs_quadratic_curve} constructors}

\indexlibrary{\idxcode{abs_quadratic_curve}!constructor}%
\begin{itemdecl}
constexpr abs_quadratic_curve() noexcept;
\end{itemdecl}
\begin{itemdescr}
\pnum
\effects
Equivalent to: \tcode{abs_quadratic_curve\{ vector_2d(), vector_2d() \};}
\end{itemdescr}

\indexlibrary{\idxcode{abs_quadratic_curve}!constructor}%
\begin{itemdecl}
constexpr abs_quadratic_curve(const vector_2d& cpt, const vector_2d& ept)
  noexcept;
\end{itemdecl}
\begin{itemdescr}
\pnum
\effects
Constructs an object of type \tcode{abs_quadratic_curve}.

\pnum
The control point is \tcode{cpt}.

\pnum
The end point is \tcode{ept}.
\end{itemdescr}

\rSec1 [\iotwod.absquadraticcurve.modifiers]{\tcode{abs_quadratic_curve} modifiers}

\indexlibrarymember{control_pt}{abs_quadratic_curve}%
\begin{itemdecl}
constexpr void control_pt(const vector_2d& cpt) noexcept;
\end{itemdecl}
\begin{itemdescr}
\pnum
\effects
The control point is \tcode{cpt}.
\end{itemdescr}

\indexlibrarymember{end_pt}{abs_quadratic_curve}%
\begin{itemdecl}
constexpr void end_pt(const vector_2d& ept) noexcept;
\end{itemdecl}
\begin{itemdescr}
\pnum
\effects
The end point is \tcode{ept}.
\end{itemdescr}

\rSec1 [\iotwod.absquadraticcurve.observers]{\tcode{abs_quadratic_curve} observers}

\indexlibrarymember{control_pt}{abs_quadratic_curve}%
\begin{itemdecl}
constexpr vector_2d control_pt() const noexcept;
\end{itemdecl}
\begin{itemdescr}
\pnum
\returns
The control point.
\end{itemdescr}

\indexlibrarymember{end_pt}{abs_quadratic_curve}%
\begin{itemdecl}
constexpr vector_2d end_pt() const noexcept;
\end{itemdecl}
\begin{itemdescr}
\pnum
\returns
The end point.
\end{itemdescr}

\rSec1 [\iotwod.absquadraticcurve.ops]{\tcode{abs_quadratic_curve} operators}

\indexlibrarymember{operator==}{abs_quadratic_curve}%
\begin{itemdecl}
constexpr bool operator==(const abs_quadratic_curve& lhs,
  const abs_quadratic_curve& rhs) noexcept;
\end{itemdecl}
\begin{itemdescr}
\pnum
\returns
\tcode{lhs.control_pt() == rhs.control_pt() \&\& lhs.end_pt() == rhs.end_pt()}.
\end{itemdescr}
