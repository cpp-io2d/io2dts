%!TEX root = io2d.tex
\rSec0 [absquadraticcurve] {Class \tcode{abs_quadratic_curve}}

\pnum
\indexlibrary{\idxcode{abs_quadratic_curve}}
The class \tcode{abs_quadratic_curve} describes a path segment that is a quadratic \bezierlocal curve.

\pnum
It has a control point of type \tcode{vector_2d} and an end point of type \tcode{vector_2d}.

\rSec1 [absquadraticcurve.synopsis] {\tcode{abs_quadratic_curve} synopsis}

\begin{codeblock}
namespace std { namespace experimental { namespace io2d { inline namespace v1 {
  namespace path_data {
    class abs_cubic_curve {
    public:
      // \ref{absquadraticcurve.cons}, construct:
      constexpr abs_quadratic_curve() noexcept;
      constexpr abs_quadratic_curve(const vector_2d& cpt, const vector_2d& ept)
        noexcept;

      // \ref{absquadraticcurve.modifiers}, modifiers:
      constexpr void control_point(const vector_2d& cpt) noexcept;
      constexpr void end_point(const vector_2d& ept) noexcept;

      // \ref{absquadraticcurve.observers}, observers:
      constexpr vector_2d control_point() const noexcept;
      constexpr vector_2d end_point() const noexcept;
    };
  };
} } } }
\end{codeblock}

\rSec1 [absquadraticcurve.cons] {\tcode{abs_quadratic_curve} constructors}

\indexlibrary{\idxcode{abs_quadratic_curve}!constructor}
\begin{itemdecl}
constexpr abs_quadratic_curve() noexcept;
\end{itemdecl}
\begin{itemdescr}
\pnum
\effects
Constructs an object of type \tcode{abs_quadratic_curve}.

\pnum
The control point shall be set to the value of \tcode{vector_2d\{0.0, 0.0\}}.

\pnum
The end point shall be set to the value of \tcode{vector_2d\{0.0, 0.0\}}.
\end{itemdescr}

\indexlibrary{\idxcode{abs_quadratic_curve}!constructor}
\begin{itemdecl}
constexpr abs_quadratic_curve(const vector_2d& cpt, const vector_2d& ept)
  noexcept;
\end{itemdecl}
\begin{itemdescr}
\pnum
\effects
Constructs an object of type \tcode{abs_quadratic_curve}.

\pnum
The control point shall be set to the value of \tcode{cpt}.

\pnum
The end point shall be set to the value of \tcode{ept}.
\end{itemdescr}

\rSec1 [absquadraticcurve.modifiers]{\tcode{abs_quadratic_curve} modifiers}

\indexlibrary{\idxcode{abs_quadratic_curve}!\idxcode{control_point}}
\begin{itemdecl}
constexpr void control_point(const vector_2d& cpt) noexcept;
\end{itemdecl}
\begin{itemdescr}
\pnum
\effects
The control point shall be set to the value of \tcode{cpt}.
\end{itemdescr}

\indexlibrary{\idxcode{abs_quadratic_curve}!\idxcode{end_point}}
\begin{itemdecl}
constexpr void end_point(const vector_2d& ept) noexcept;
\end{itemdecl}
\begin{itemdescr}
\pnum
\effects
The end point shall be set to the value of \tcode{ept}.
\end{itemdescr}

\rSec1 [absquadraticcurve.observers]{\tcode{abs_quadratic_curve} observers}

\indexlibrary{\idxcode{abs_quadratic_curve}!\idxcode{control_point}}
\begin{itemdecl}
constexpr vector_2d control_point() const noexcept;
\end{itemdecl}
\begin{itemdescr}
\pnum
\returns
The value of the control point.
\end{itemdescr}

\indexlibrary{\idxcode{abs_quadratic_curve}!\idxcode{end_point}}
\begin{itemdecl}
constexpr vector_2d end_point() const noexcept;
\end{itemdecl}
\begin{itemdescr}
\pnum
\returns
The value of the end point.
\end{itemdescr}
