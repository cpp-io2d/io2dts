%!TEX root = io2d.tex

\rSec0 [\iotwod.imagefileformat] {Enum class \tcode{image_file_format}}

\rSec1 [\iotwod.imagefileformat.summary] {\tcode{image_file_format} summary}

\pnum
The \tcode{image_file_format} enum class specifies the data format that an \tcode{image_surface} object is constructed from or saved to. This allows data in a format that is required to be supported to be read or written regardless of its extension.

\pnum
It also has a value that allows implementations to support additional file formats if it recognizes them.

\rSec1 [\iotwod.imagefileformat.synopsis] {\tcode{image_file_format} synopsis}

\indexlibrary{\idxcode{image_file_format}}
\begin{codeblock}
namespace std::experimental::io2d::v1 {
  enum class image_file_format {
    unknown,
    png,
    jpg,
    tiff
  };
}
\end{codeblock}

\rSec1 [\iotwod.imagefileformat.enumerators] {\tcode{image_file_format} enumerators}

\begin{libreqtab2}
 {\tcode{imagefileformat} enumerator meanings}
 {tab:\iotwod.imagefileformat.meanings}
 \\ \topline
 \lhdr{Enumerator}
 & \rhdr{Meaning}
 \\ \capsep
 \endfirsthead
 \continuedcaption\\
 \hline
 \lhdr{Enumerator}
 & \rhdr{Meaning}
 \\ \capsep
 \endhead
 \tcode{unknown}
 & The format is unknown because it is not an image file format that is required to be supported. It may be known and supported by the implementation.
 \\
 \tcode{png}
 & The PNG format.
 \\
 \tcode{jpg}
 & The JPEG format.
 \\
 \tcode{tiff}
 & The TIFF format.
 \\
\end{libreqtab2}
