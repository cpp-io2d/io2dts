%!TEX root = io2d.tex

\rSec0 [\iotwod.paths.interpretation]{Path interpretation}

\pnum
This subclause describes how to interpret a path for use in a rendering and composing operation.

\pnum
Interpreting a path consists of sequentially evaluating the figure items contained in the figures in the path and transforming them into zero or more figures as-if in the manner specified in this subclause.

\pnum
The interpretation of a path requires the state data specified in Table~\ref{tab:\iotwod.paths.interpretation.state}.

\begin{floattable}
{Path interpretation state data}{tab:\iotwod.paths.interpretation.state}{llll}
\hline
\hdstyle{Name} &
\hdstyle{Description} &
\hdstyle{Type} &
\hdstyle{Initial value} \\ \hline
\tcode{mtx} &
Path transformation matrix &
\tcode{matrix_2d} &
\tcode{matrix_2d\{ \}} \\
\tcode{currPt} &
Current point &
\tcode{point_2d} &
\unspec \\
\tcode{lnfPt} &
Last new figure point &
\tcode{point_2d} &
\unspec \\
\tcode{mtxStk} &
Matrix stack &
\tcode{stack<matrix_2d>} &
\tcode{stack<matrix_2d>\{ \}} \\\hline
\end{floattable}

\FloatBarrier

\pnum
When interpreting a path, until a \tcode{figure_items::abs_new_figure} figure item is reached, a path shall only contain path command figure items; no diagnostic is required. If a figure is a degenerate figure, none of its figure items have any effects, with two exceptions:
\begin{itemize}
\item the path's \tcode{figure_items::abs_new_figure} or \tcode{figure_items::rel_new_figure} figure item sets the value of \tcode{currPt} as-if the figure item was interpreted; and,
\item any path command figure items are evaluated with full effect.
\end{itemize}.

\pnum
The effects of a figure item contained in a \tcode{figure_items::figure_item} object when that object is being evaluated during path interpretation are described in Table~\ref{tab:\iotwod.paths.interpretation.effects}.

\begin{libreqtab2a} {Figure item interpretation effects} {tab:\iotwod.paths.interpretation.effects}
\\ \topline
\lhdr{Figure item} & \rhdr{Effects} \\ \capsep
\endfirsthead
\continuedcaption\\
\topline
\lhdr{Figure item} & \rhdr{Effects} \\ \capsep
\endhead

\tcode{figure_items::abs_new_figure p} &
Creates a new figure. Sets \tcode{currPt} to \tcode{mtx.transform_pt(\{ 0.0f, 0.0f \}) + p.at()}. Sets \tcode{lnfPt} to \tcode{currPt}. \\ \rowsep

\tcode{figure_items::rel_new_figure p} &
Let \tcode{mm} equal \tcode{mtx}.  Let \tcode{mm.m20} equal {0.0f}. Let \tcode{mm.m21} equal \tcode{0.0f}. Creates a new figure. Sets \tcode{currPt} to \tcode{currPt + p.at() * mm}. Sets \tcode{lnfPt} to \tcode{currPt}. \\ \rowsep

\tcode{figure_items::close_figure p} &
Creates a line from \tcode{currPt} to \tcode{lnfPt}. Makes the current figure a closed figure. Creates a new figure. Sets \tcode{currPt} to \tcode{lnfPt}. \\ \rowsep

\tcode{figure_items::abs_matrix p} &
Calls \tcode{mtxStk.push(mtx)}. Sets \tcode{mtx} to \tcode{p.matrix()}. \\ \rowsep

\tcode{figure_items::rel_matrix p} &
Calls \tcode{mtxStk.push(mtx)}. Sets \tcode{mtx} to \tcode{mtx * p.matrix()}. \\ \rowsep

\tcode{figure_items::revert_matrix p} &
If \tcode{mtxStk.empty()} is \tcode{false}, sets \tcode{mtx} to \tcode{mtxStk.top()} then calls \tcode{mtxStk.pop()}. Otherwise sets \tcode{mtx} to its initial value as specified in Table~\ref{tab:\iotwod.paths.interpretation.state}. \\ \rowsep

\tcode{figure_items::abs_line p} &
Let \tcode{pt} equal \tcode{mtx.transform_pt(p.to() - currPt) + currPt}. Creates a line from \tcode{currPt} to \tcode{pt}. Sets \tcode{currPt} to \tcode{pt}. \\ \rowsep

\tcode{figure_items::rel_line p} &
Let \tcode{mm} equal \tcode{mtx}. Let \tcode{mm.m20} equal {0.0f}. Let \tcode{mm.m21} equal \tcode{0.0f}. Let \tcode{pt} equal \tcode{currPt + p.to() * mm}. Creates a line from \tcode{currPt} to \tcode{pt}. Sets \tcode{currPt} to \tcode{pt}. \\ \rowsep

\tcode{figure_items::abs_quadratic_curve p} &
Let \tcode{cpt} equal \tcode{mtx.transform_pt(p.control_pt() - currPt) + currPt}. Let \tcode{ept} equal \tcode{mtx.transform_pt(p.end_pt() - currPt) + currPt}. Creates a quadratic \bezierlocal curve from \tcode{currPt} to \tcode{ept} using \tcode{cpt} as the curve's control point. Sets \tcode{currPt} to \tcode{ept}. \\ \rowsep

\tcode{figure_items::rel_quadratic_curve p} &
Let \tcode{mm} equal \tcode{mtx}. Let \tcode{mm.m20} equal {0.0f}. Let \tcode{mm.m21} equal \tcode{0.0f}. Let \tcode{cpt} equal \tcode{currPt + p.control_pt() * mm}. Let \tcode{ept} equal \tcode{currPt + p.control_pt() * mm + p.end_pt() * mm}. Creates a quadratic \bezierlocal curve from \tcode{currPt} to \tcode{ept} using \tcode{cpt} as the curve's control point. Sets \tcode{currPt} to \tcode{ept}. \\ \rowsep

\tcode{figure_items::abs_cubic_curve p} &
Let \tcode{cpt1} equal \tcode{mtx.transform_pt(p.control_pt1() - currPt) + currPt}. Let \tcode{cpt2} equal \tcode{mtx.transform_pt(p.control_pt2() - currPt) + currPt}. Let \tcode{ept} equal \tcode{mtx.transform_pt(p.end_pt() - currPt) + currPt}. Creates a cubic \bezierlocal curve from \tcode{currPt} to \tcode{ept} using \tcode{cpt1} as the curve's first control point and \tcode{cpt2} as the curve's second control point. Sets \tcode{currPt} to \tcode{ept}. \\ \rowsep

\tcode{figure_items::rel_cubic_curve p} &
Let \tcode{mm} equal \tcode{mtx}. Let \tcode{mm.m20} equal {0.0f}. Let \tcode{mm.m21} equal \tcode{0.0f}. Let \tcode{cpt1} equal \tcode{currPt + p.control_pt1() * mm}. Let \tcode{cpt2} equal \tcode{currPt + p.control_pt1() * mm + p.control_pt2() * mm}. Let \tcode{ept} equal \tcode{currPt + p.control_pt1() * mm + p.control_pt2() * mm + p.end_pt() * mm}. Creates a cubic \bezierlocal curve from \tcode{currPt} to \tcode{ept} using \tcode{cpt1} as the curve's first control point and \tcode{cpt2} as the curve's second control point. Sets \tcode{currPt} to \tcode{ept}. \\ \rowsep

\tcode{figure_items::arc p} &
Let \tcode{mm} equal \tcode{mtx}. Let \tcode{mm.m20} equal {0.0f}. Let \tcode{mm.m21} equal \tcode{0.0f}. Creates an arc. It begins at \tcode{currPt}, which is at \tcode{p.start_angle()} radians on the arc and rotates \tcode{p.rotation()} radians. If \tcode{p.rotation()} is positive, rotation is counterclockwise, otherwise it is clockwise. The center of the arc is located at \tcode{p.center(currPt, mm)}. The arc ends at \tcode{p.end_pt(currPt, mm)}. Sets \tcode{currPt} to \tcode{p.end_pt(currPt, mm)}.

\begin{note} \tcode{p.radius()}, which specifies the radius of the arc, is implicitly included in the above statement of effects by the specifications of the center of the arc and the end of the arcs. The use of the current point as the origin for the application of the path transformation matrix is also implicitly included by the same specifications. \end{note} \\
\end{libreqtab2a}
