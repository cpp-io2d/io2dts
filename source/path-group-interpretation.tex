%!TEX root = io2d.tex

\rSec0 [\iotwod.paths.interpretation]{Path group interpretation}

\pnum
This subclause describes how to interpret a path group for use in a rendering and composing operation.

\pnum
Interpreting a path group consists of sequentially evaluating the \tcode{path_data::path_item} objects in a path group and transforming them into zero or more paths as-if in the manner specified in this subclause.

\pnum
The interpretation of a path group requires the state data specified in Table~\ref{tab:\iotwod.paths.interpretation.state}.

\begin{floattable}
{Path group interpretation state data}{tab:\iotwod.paths.interpretation.state}{llll}
\hline
\hdstyle{Name} &
\hdstyle{Description} &
\hdstyle{Type} &
\hdstyle{Initial value} \\ \hline
\tcode{mtx} &
Path group transformation matrix &
\tcode{matrix_2d} &
\tcode{matrix_2d\{ \}} \\
\tcode{currPt} &
Current point &
\tcode{vector_2d} &
\unspec \\
\tcode{lnPt} &
Last new point &
\tcode{vector_2d} &
\unspec \\
\tcode{mtxStk} &
Matrix stack &
\tcode{stack<matrix_2d>} &
\tcode{stack<matrix_2d>\{ \}} \\\hline
\end{floattable}

\FloatBarrier

\pnum
When interpreting a path group, until a \tcode{path_data::abs_new_path} path item is reached, a path shall only contain path group instruction path items; no diagnostic is required. If a path is a degenerate path, none of its path items have any effects, with two exceptions:
\begin{itemize}
\item the path's \tcode{path_data::abs_new_path} or \tcode{path_data::rel_new_path} path item sets the value of \tcode{currPt} as-if the path item was interpreted; and,
\item any path group instruction path items are evaluated with full effect.
\end{itemize}.

%\pnum
%\begin{note}
%The requirement above stating "until a \tcode{path_data::abs_new_path} path item is reached..." uses the word "reached" to make it clear that whether the first path is a degenerate path is irrelevant to that requirement.
%\end{note}
\pnum
The effects of a path item contained in a \tcode{path_data::path_item} object when that object is being evaluated during path group interpretation are described in Table~\ref{tab:\iotwod.paths.interpretation.effects}.

\begin{libreqtab2a} {Path item interpretation effects} {tab:\iotwod.paths.interpretation.effects}
\\ \topline
\lhdr{Path item} & \rhdr{Effects} \\ \capsep
\endfirsthead
\continuedcaption\\
\topline
\lhdr{Path item} & \rhdr{Effects} \\ \capsep
\endhead

\tcode{path_data::abs_new_path p} &
Creates a new path. Sets \tcode{currPt} to \tcode{mtx.transform_pt(\{ 0.0f, 0.0f \}) + p.at()}. Sets \tcode{lnPt} to \tcode{currPt}. \\ \rowsep

\tcode{path_data::rel_new_path p} &
Let \tcode{mm} equal \tcode{mtx}. Evaluate \tcode{mm.m20(0.0f)} and \tcode{mm.m21(0.0f)}. Creates a new path. Sets \tcode{currPt} to \tcode{currPt + p.at() * mm}. Sets \tcode{lnPt} to \tcode{currPt}. \\ \rowsep

\tcode{path_data::close_path p} &
Creates a line from \tcode{currPt} to \tcode{lnPt}. Makes the current path a closed path. Creates a new path. Sets \tcode{currPt} to \tcode{lnPt}. \\ \rowsep

\tcode{path_data::abs_matrix p} &
Calls \tcode{mtxStk.push(mtx)}. Sets \tcode{mtx} to \tcode{p.matrix()}. \\ \rowsep

\tcode{path_data::rel_matrix p} &
Calls \tcode{mtxStk.push(mtx)}. Sets \tcode{mtx} to \tcode{mtx * p.matrix()}. \\ \rowsep

\tcode{path_data::revert_matrix p} &
If \tcode{mtxStk.empty()} is \tcode{false}, sets \tcode{mtx} to \tcode{ognStk.top()} then calls \tcode{ognStk.pop()}. Otherwise sets \tcode{mtx} to its initial value as specified in Table~\ref{tab:\iotwod.paths.interpretation.state}. \\ \rowsep

\tcode{path_data::abs_line p} &
Let \tcode{pt} equal \tcode{mtx.transform_pt(p.to() - currPt) + currPt}. Creates a line from \tcode{currPt} to \tcode{pt}. Sets \tcode{currPt} to \tcode{pt}. \\ \rowsep

\tcode{path_data::rel_line p} &
Let \tcode{mm} equal \tcode{mtx}. Evaluate \tcode{mm.m20(0.0f)} and \tcode{mm.m21(0.0f)}. Let \tcode{pt} equal \tcode{currPt + p.to() * mm}. Creates a line from \tcode{currPt} to \tcode{pt}. Sets \tcode{currPt} to \tcode{pt}. \\ \rowsep

\tcode{path_data::abs_quadratic_curve p} &
Let \tcode{cpt} equal \tcode{mtx.transform_pt(p.control_pt() - currPt) + currPt}. Let \tcode{ept} equal \tcode{mtx.transform_pt(p.end_pt() - currPt) + currPt}. Creates a quadratic \bezierlocal curve from \tcode{currPt} to \tcode{ept} using \tcode{cpt} as the curve's control point. Sets \tcode{currPt} to \tcode{ept}. \\ \rowsep

\tcode{path_data::rel_quadratic_curve p} &
Let \tcode{mm} equal \tcode{mtx}. Evaluate \tcode{mm.m20(0.0f)} and \tcode{mm.m21(0.0f)}. Let \tcode{cpt} equal \tcode{currPt + p.control_pt() * mm}. Let \tcode{ept} equal \tcode{currPt + p.control_pt() * mm + p.end_pt() * mm}. Creates a quadratic \bezierlocal curve from \tcode{currPt} to \tcode{ept} using \tcode{cpt} as the curve's control point. Sets \tcode{currPt} to \tcode{ept}. \\ \rowsep

\tcode{path_data::abs_cubic_curve p} &
Let \tcode{cpt1} equal \tcode{mtx.transform_pt(p.control_pt1() - currPt) + currPt}. Let \tcode{cpt2} equal \tcode{mtx.transform_pt(p.control_pt2() - currPt) + currPt}. Let \tcode{ept} equal \tcode{mtx.transform_pt(p.end_pt() - currPt) + currPt}. Creates a cubic \bezierlocal curve from \tcode{currPt} to \tcode{ept} using \tcode{cpt1} as the curve's first control point and \tcode{cpt2} as the curve's second control point. Sets \tcode{currPt} to \tcode{ept}. \\ \rowsep

\tcode{path_data::rel_cubic_curve p} &
Let \tcode{mm} equal \tcode{mtx}. Evaluate \tcode{mm.m20(0.0f)} and \tcode{mm.m21(0.0f)}. Let \tcode{cpt1} equal \tcode{currPt + p.control_pt1() * mm}. Let \tcode{cpt2} equal \tcode{currPt + p.control_pt1() * mm + p.control_pt2() * mm}. Let \tcode{ept} equal \tcode{currPt + p.control_pt1() * mm + p.control_pt2() * mm + p.end_pt() * mm}. Creates a cubic \bezierlocal curve from \tcode{currPt} to \tcode{ept} using \tcode{cpt1} as the curve's first control point and \tcode{cpt2} as the curve's second control point. Sets \tcode{currPt} to \tcode{ept}. \\ \rowsep

\tcode{path_data::arc p} &
Let \tcode{mm} equal \tcode{mtx}. Evaluate \tcode{mm.m20(0.0f)} and \tcode{mm.m21(0.0f)}. Creates an arc. It begins at \tcode{currPt}, which is at \tcode{p.start_angle()} radians on the arc and rotates \tcode{p.rotation()} radians. If \tcode{p.rotation()} is positive, rotation is counterclockwise, otherwise it is clockwise. The center of the arc is located at \tcode{p.center(currPt, mm)}. The arc ends at \tcode{p.end_pt(currPt, mm)}. Sets \tcode{currPt} to \tcode{p.end_pt(currPt, mm)}.

\begin{note} \tcode{p.radius()}, which specifies the radius of the arc, is implicitly included in the above statement of effects by the specifications of the center of the arc and the end of the arcs. The use of the current point as the origin for the application of the path group transformation matrix is also implicitly included by the same specifications. \end{note} \\
\end{libreqtab2a}
