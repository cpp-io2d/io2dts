%!TEX root = io2d.tex
\rSec0 [pathdataitem.changematrix] {Class \tcode{change_matrix}}

\rSec1 [pathdataitem.changematrix.synopsis] {\tcode{change_matrix} synopsis}

\begin{codeblock}
namespace std { namespace experimental { namespace io2d { inline namespace v1 {
  class change_matrix : public path_data {
  public:
    // \ref{pathdataitem.changematrix.cons}, construct/copy/move/destroy:
    change_matrix() noexcept;
    change_matrix(const change_matrix& other) noexcept;
    change_matrix& operator=(const change_matrix& other) noexcept;
    change_matrix(change_matrix&& other) noexcept;
    change_matrix& operator=(change_matrix&& other) noexcept;
    change_matrix(const matrix_2d& m) noexcept;

    // \ref{pathdataitem.changematrix.modifiers}, modifiers:
    void matrix(const matrix_2d& value) noexcept;

    // \ref{pathdataitem.changematrix.observers}, observers:
    matrix_2d matrix() const noexcept;
    virtual path_data_type type() const noexcept override;
    
  private:
    matrix_2d _Matrix; // \expos
  };
} } } }
\end{codeblock}

\rSec1 [pathdataitem.changematrix.intro] {\tcode{change_matrix} Description}

\pnum
\indexlibrary{\idxcode{change_matrix}}
The class \tcode{change_matrix} describes an operation on a path geometry collection.

\pnum
This operation changes the transformation matrix for a path geometry collection to be the value returned by \tcode{*this.matrix()}. As shown in \ref{pathgeometries.processing}, the new transformation matrix does not affect any operations that came before this operation. It is only used in processing operations that come after it. It continues to be used until another \tcode{change_matrix} object is encountered or the end of the path geometry collection is reached.

\rSec1 [pathdataitem.changematrix.cons] {\tcode{change_matrix} constructors and assignment operators}

\indexlibrary{\idxcode{change_matrix}!constructor}
\begin{itemdecl}
    change_matrix() noexcept;
\end{itemdecl}
\begin{itemdescr}
	\pnum
	\effects
	Constructs an object of type \tcode{change_matrix}.
	
	\pnum
	\postconditions
	\tcode{_Matrix == matrix_2d\{\}}.
\end{itemdescr}

\indexlibrary{\idxcode{change_matrix}!constructor}
\begin{itemdecl}
    change_matrix(const matrix_2d& m) noexcept;
\end{itemdecl}
\begin{itemdescr}
	\pnum
	\effects
	Constructs an object of type \tcode{change_matrix}.
	
	\pnum
	\postconditions
	\tcode{_Matrix == m}.
\end{itemdescr}

\rSec1 [pathdataitem.changematrix.modifiers]{\tcode{change_matrix} modifiers}

\indexlibrary{\idxcode{change_matrix}!\idxcode{matrix}}
\indexlibrary{\idxcode{matrix}!\idxcode{change_matrix}}
\begin{itemdecl}
    void matrix(const matrix_2d& value) noexcept;
\end{itemdecl}
\begin{itemdescr}
	\pnum
	\postconditions
	\tcode{_Matrix == value}.
	
\end{itemdescr}

\rSec1 [pathdataitem.changematrix.observers]{\tcode{change_matrix} observers}

\indexlibrary{\idxcode{change_matrix}!\idxcode{matrix}}
\indexlibrary{\idxcode{matrix}!\idxcode{change_matrix}}
\begin{itemdecl}
    matrix_2d matrix() const noexcept;
\end{itemdecl}
\begin{itemdescr}
	\pnum
	\returns
	\tcode{_Matrix}.

\end{itemdescr}

\indexlibrary{\idxcode{change_matrix}!\idxcode{type}}
\indexlibrary{\idxcode{type}!\idxcode{change_matrix}}
\begin{itemdecl}
    virtual path_data_type type() const noexcept override;
\end{itemdecl}
\begin{itemdescr}
	\pnum
	\returns
	\tcode{path_data_type::change_matrix}.

\end{itemdescr}
