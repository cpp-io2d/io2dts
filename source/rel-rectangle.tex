%!TEX root = io2d.tex
\rSec0 [relrectangle] {Class \tcode{rel_rectangle}}

\pnum
\indexlibrary{\idxcode{rel_rectangle}}
The class \tcode{rel_rectangle} describes a path instruction that adds a rectangle to the current path.

\pnum
It has an X coordinate of type \tcode{double}, a Y coordinate of type \tcode{double}, a Width of type \tcode{double}, and a Height of type \tcode{double}.

\rSec1 [relrectangle.synopsis] {\tcode{rel_rectangle} synopsis}

\begin{codeblock}
namespace std { namespace experimental { namespace io2d { inline namespace v1 {
  namespace path_data {
    class rel_rectangle {
    public:
      // \ref{relrectangle.cons}, constructors:
      constexpr rel_rectangle() noexcept;
      constexpr rel_rectangle(double x, double y, double w, double h) noexcept;
      constexpr rel_rectangle(const vector_2d& tl, const vector_2d& br) 
        noexcept;
      constexpr rel_rectangle(const rectangle& r);

      // \ref{relrectangle.modifiers}, modifiers:
      constexpr void x(double value) noexcept;
      constexpr void y(double value) noexcept;
      constexpr void width(double value) noexcept;
      constexpr void height(double value) noexcept;
      constexpr void top_left(const vector_2d& value) noexcept;
      constexpr void bottom_right(const vector_2d& value) noexcept;
      constexpr void top_left_bottom_right(const vector_2d& tl,
        const vector_2d& br) noexcept;

      // \ref{relrectangle.observers}, observers:
      constexpr double x() const noexcept;
      constexpr double y() const noexcept;
      constexpr double width() const noexcept;
      constexpr double height() const noexcept;
      constexpr double left() const noexcept;
      constexpr double right() const noexcept;
      constexpr double top() const noexcept;
      constexpr double bottom() const noexcept;
      constexpr vector_2d top_left() const noexcept;
      constexpr vector_2d bottom_right() const noexcept;
    };
  }
} } } }
\end{codeblock}

\rSec1 [relrectangle.cons] {\tcode{rel_rectangle} constructors}

\indexlibrary{\idxcode{rel_rectangle}!constructor}
\begin{itemdecl}
constexpr rel_rectangle() noexcept;
\end{itemdecl}
\begin{itemdescr}
\pnum
\effects
Constructs an object of type \tcode{rel_rectangle}.

\pnum
The X coordinate, Y coordinate, Width, and Height shall each be set to the value \tcode{0.0}.
\end{itemdescr}

\indexlibrary{\idxcode{rel_rectangle}!constructor}
\begin{itemdecl}
constexpr rel_rectangle(double x, double y, double w, double h) noexcept;
\end{itemdecl}
\begin{itemdescr}
\pnum
\effects
Constructs an object of type \tcode{rel_rectangle}.

\pnum
The X coordinate shall be set to the value of \tcode{x}.

\pnum
The Y coordinate shall be set to the value of \tcode{y}.

\pnum
The Width shall be set to the value of \tcode{w}.

\pnum
The Height shall be set to the value of \tcode{h}.
\end{itemdescr}

\indexlibrary{\idxcode{rel_rectangle}!constructor}
\begin{itemdecl}
constexpr rel_rectangle(const vector_2d& tl, const vector_2d& br) noexcept;
\end{itemdecl}
\begin{itemdescr}
\pnum
\effects
Constructs an object of type \tcode{rel_rectangle}.

\pnum
The X coordinate shall be set to the value of \tcode{tl.x()}.

\pnum
The Y coordinate shall be set to the value of \tcode{tl.y()}.

\pnum
The Width shall be set to the value of \tcode{max(0.0, br.x() - tl.x())}.

\pnum
The Height shall be set to the value of \tcode{max(0.0, br.y() - tl.y())}.
\end{itemdescr}

\rSec1 [relrectangle.modifiers]{\tcode{rel_rectangle} modifiers}

\indexlibrary{\idxcode{rel_rectangle}!\idxcode{x}}
\begin{itemdecl}
constexpr void x(double val) noexcept;
\end{itemdecl}

\begin{itemdescr}
\pnum
\effects
The X coordinate shall be set to the value of \tcode{val}.
\end{itemdescr}

\indexlibrary{\idxcode{rel_rectangle}!\idxcode{y}}
\begin{itemdecl}
constexpr void y(double value) noexcept;
\end{itemdecl}
\begin{itemdescr}
\pnum
\effects
The Y coordinate shall be set to the value of \tcode{val}.
\end{itemdescr}

\indexlibrary{\idxcode{rel_rectangle}!\idxcode{width}}
\begin{itemdecl}
constexpr void width(double value) noexcept;
\end{itemdecl}
\begin{itemdescr}
\pnum
\effects
The Width shall be set to the value of \tcode{val}.
\end{itemdescr}

\indexlibrary{\idxcode{rel_rectangle}!\idxcode{height}}
\begin{itemdecl}
constexpr void height(double value) noexcept;
\end{itemdecl}
\begin{itemdescr}
\pnum
\effects
The Height shall be set to the value of \tcode{val}.
\end{itemdescr}

\indexlibrary{\idxcode{rel_rectangle}!\idxcode{top_left}}
\begin{itemdecl}
constexpr void top_left(const vector_2d& val) noexcept;
\end{itemdecl}
\begin{itemdescr}
\pnum
\effects
The X coordinate shall be set to the value of \tcode{val.x()}.

\effects
The Y coordinate shall be set to the value of \tcode{val.y()}.
\end{itemdescr}

\indexlibrary{\idxcode{rel_rectangle}!\idxcode{bottom_right}}
\begin{itemdecl}
constexpr void bottom_right(const vector_2d& val) noexcept;
\end{itemdecl}
\begin{itemdescr}
\pnum
\effects
The Width shall be set to the value of \tcode{max(0.0, val.x() - *this.x())}.

\pnum
The Height shall be set to the value of \tcode{max(0.0, value.y() - *this.y())}.
\end{itemdescr}

\rSec1 [relrectangle.observers]{\tcode{rel_rectangle} observers}

\indexlibrary{\idxcode{rel_rectangle}!\idxcode{x}}
\begin{itemdecl}
constexpr double x() const noexcept;
\end{itemdecl}
\begin{itemdescr}
\pnum
\returns
The value of the X coordinate.
\end{itemdescr}

\indexlibrary{\idxcode{rel_rectangle}!\idxcode{y}}
\begin{itemdecl}
constexpr double y() const noexcept;
\end{itemdecl}
\begin{itemdescr}
\pnum
\returns
The value of the Y coordinate.
\end{itemdescr}

\indexlibrary{\idxcode{rel_rectangle}!\idxcode{width}}
\begin{itemdecl}
constexpr double width() const noexcept;
\end{itemdecl}
\begin{itemdescr}
\pnum
\returns
The value of the Width.
\end{itemdescr}

\indexlibrary{\idxcode{rel_rectangle}!\idxcode{height}}
\begin{itemdecl}
constexpr double height() const noexcept;
\end{itemdecl}
\begin{itemdescr}
\pnum
\returns
The value of the Height.
\end{itemdescr}

\indexlibrary{\idxcode{rel_rectangle}!\idxcode{top_left}}
\begin{itemdecl}
constexpr vector_2d top_left() const noexcept;
\end{itemdecl}
\begin{itemdescr}
\pnum
\returns
A \tcode{vector_2d} object constructed from the value of the X coordinate as its first argument and the value of the Y coordinate as its second argument.
\end{itemdescr}

\indexlibrary{\idxcode{rel_rectangle}!\idxcode{bottom_right}}
\begin{itemdecl}
constexpr vector_2d bottom_right() const noexcept;
\end{itemdecl}
\begin{itemdescr}
\pnum
\returns
A \tcode{vector_2d} object constructed from the value of the Width added to the value of the X coordinate as its first argument and the value of the Height added to the value of the Y coordinate as its second argument.
\end{itemdescr}
