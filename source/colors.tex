%!TEX root = io2d.tex

\rSec0 [\iotwod.colors] {Colors}

\rSec1 [\iotwod.colors.intro] {Introduction to color}

\pnum
Color involves many disciplines and has been the subject of many papers, treatises, experiments, studies, and research work in general.

\pnum
While color is an important part of computer graphics, it is only necessary to understand a few concepts from the study of color for computer graphics.

\pnum
A color model defines color mathematically without regard to how humans actually perceive color. These color models are composed of some combination of channels which each channel representing alpha or an ideal color. Color models are useful for working with color computationally, such as in composing operations, because their channel values are homogeneously spaced. 

\pnum
A color space, for purposes of computer graphics, is the result of mapping the ideal color channels from a color model, after making any necessary adjustment for alpha, to color channels that are calibrated to align with human perception of colors. Since the perception of color varies from person to person, color spaces use the science of colorimetry to define those perceived colors in order to obtain uniformity to the extent possible. As such, the uniform display of the colors in a color space on different output devices is possible. The values of color channels in a color space are not necessarily homogeneously spaced because of human perception of color.

\pnum
Color models are often termed \term{linear} while color spaces are often termed \term{gamma corrected}. The mapping of a color model, such as the RGB color model, to a color space, such as the sRGB color space, is often the application of gamma correction.

\pnum
Gamma correction is the process of transforming homogeneously spaced visual data to visual data that, when displayed, matches the intent of the untransformed visual data.

\pnum
For example a color that is 50\% of the maximum intensity of red when encoded as homogeneously spaced visual data, will likely have a different intensity value when it has been gamma corrected so that a human looking at on a computer display will see it as being 50\% of the maximum intensity of red that the computer display is capable of producing. Without gamma correction, it would likely have appeared as though it was closer to the maximum intensity than the untransformed data intended it to be.

\pnum
In addition to color channels, colors in computer graphics often have an alpha channel. The value of the alpha channel represents transparency of the color channels when they are combined with other visual data using certain composing algorithms. When using alpha, it should be used in a premultiplied format in order to obtain the desired results when applying multiple composing algorithms that utilize alpha.

\rSec1 [\iotwod.colors.reqs] {Color usage requirements}

\pnum
During rendering and composing operations, color data is linear and, when it has an alpha channel associated with it, in premultiplied format. Implementations shall make any necessary conversions to ensure this. 

\addtocounter{SectionDepthBase}{1}
%!TEX root = io2d.tex
\rSec0 [\iotwod.rgbacolor] {Struct \tcode{rgba_color}}

\rSec1 [\iotwod.rgbacolor.intro] {\tcode{rgba_color} Description}

\pnum
\indexlibrary{\idxcode{rgba_color}}%
The \tcode{rgba_color} struct represents a premultiplied color with four 
channels.

\pnum
There are three color channels, each of which is an unsigned normalized double: 
red, green, and blue.

\pnum
There is also an alpha channel, which is an unsigned normalized double.

\pnum
The use of an unsigned normalized double allows for storing unsigned normalized 
integer values with easy, albeit potentially lossy, conversion between 
different bits per channel (e.g. storing a 32-bit X8R8G8B8 value and retrieving 
its nearest 16-bit equivalent R5G6B5 value). More importantly, use of a double 
ensures sufficient precision for lossless storage and retrieval, without 
conversion, of 32-bit color formats and even 64-bit (16-bits per channel) color 
formats.

\rSec1 [\iotwod.rgbacolor.synopsis] {\tcode{rgba_color} synopsis}

\begin{codeblock}
namespace std { namespace experimental { namespace io2d { inline namespace v1 {
  struct rgba_color {
    double r;
    double g;
    double b;
    double a;
    
    // \ref{\iotwod.rgbacolor.constants}, static constants:
    const static rgba_color alice_blue;
    const static rgba_color antique_white;
    const static rgba_color aqua;
    const static rgba_color aquamarine;
    const static rgba_color azure;
    const static rgba_color beige;
    const static rgba_color bisque;
    const static rgba_color black;
    const static rgba_color blanched_almond;
    const static rgba_color blue;
    const static rgba_color blue_violet;
    const static rgba_color brown;
    const static rgba_color burly_wood;
    const static rgba_color cadet_blue;
    const static rgba_color chartreuse;
    const static rgba_color chocolate;
    const static rgba_color coral;
    const static rgba_color cornflower_blue;
    const static rgba_color cornsilk;
    const static rgba_color crimson;
    const static rgba_color cyan;
    const static rgba_color dark_blue;
    const static rgba_color dark_cyan;
    const static rgba_color dark_goldenrod;
    const static rgba_color dark_gray;
    const static rgba_color dark_green;
    const static rgba_color dark_grey;
    const static rgba_color dark_khaki;
    const static rgba_color dark_magenta;
    const static rgba_color dark_olive_green;
    const static rgba_color dark_orange;
    const static rgba_color dark_orchid;
    const static rgba_color dark_red;
    const static rgba_color dark_salmon;
    const static rgba_color dark_sea_green;
    const static rgba_color dark_slate_blue;
    const static rgba_color dark_slate_gray;
    const static rgba_color dark_slate_grey;
    const static rgba_color dark_turquoise;
    const static rgba_color dark_violet;
    const static rgba_color deep_pink;
    const static rgba_color deep_sky_blue;
    const static rgba_color dim_gray;
    const static rgba_color dim_grey;
    const static rgba_color dodger_blue;
    const static rgba_color firebrick;
    const static rgba_color floral_white;
    const static rgba_color forest_green;
    const static rgba_color fuchsia;
    const static rgba_color gainsboro;
    const static rgba_color ghost_white;
    const static rgba_color gold;
    const static rgba_color goldenrod;
    const static rgba_color gray;
    const static rgba_color green;
    const static rgba_color green_yellow;
    const static rgba_color grey;
    const static rgba_color honeydew;
    const static rgba_color hot_pink;
    const static rgba_color indian_red;
    const static rgba_color indigo;
    const static rgba_color ivory;
    const static rgba_color khaki;
    const static rgba_color lavender;
    const static rgba_color lavender_blush;
    const static rgba_color lawn_green;
    const static rgba_color lemon_chiffon;
    const static rgba_color light_blue;
    const static rgba_color light_coral;
    const static rgba_color light_cyan;
    const static rgba_color light_goldenrod_yellow;
    const static rgba_color light_gray;
    const static rgba_color light_green;
    const static rgba_color light_grey;
    const static rgba_color light_pink;
    const static rgba_color light_salmon;
    const static rgba_color light_sea_green;
    const static rgba_color light_sky_blue;
    const static rgba_color light_slate_gray;
    const static rgba_color light_slate_grey;
    const static rgba_color light_steel_blue;
    const static rgba_color light_yellow;
    const static rgba_color lime;
    const static rgba_color lime_green;
    const static rgba_color linen;
    const static rgba_color magenta;
    const static rgba_color maroon;
    const static rgba_color medium_aquamarine;
    const static rgba_color medium_blue;
    const static rgba_color medium_orchid;
    const static rgba_color medium_purple;
    const static rgba_color medium_sea_green;
    const static rgba_color medium_slate_blue;
    const static rgba_color medium_spring_green;
    const static rgba_color medium_turquoise;
    const static rgba_color medium_violet_red;
    const static rgba_color midnight_blue;
    const static rgba_color mint_cream;
    const static rgba_color misty_rose;
    const static rgba_color moccasin;
    const static rgba_color navajo_white;
    const static rgba_color navy;
    const static rgba_color old_lace;
    const static rgba_color olive;
    const static rgba_color olive_drab;
    const static rgba_color orange;
    const static rgba_color orange_red;
    const static rgba_color orchid;
    const static rgba_color pale_goldenrod;
    const static rgba_color pale_green;
    const static rgba_color pale_turquoise;
    const static rgba_color pale_violet_red;
    const static rgba_color papaya_whip;
    const static rgba_color peach_puff;
    const static rgba_color peru;
    const static rgba_color pink;
    const static rgba_color plum;
    const static rgba_color powder_blue;
    const static rgba_color purple;
    const static rgba_color red;
    const static rgba_color rosy_brown;
    const static rgba_color royal_blue;
    const static rgba_color saddle_brown;
    const static rgba_color salmon;
    const static rgba_color sandy_brown;
    const static rgba_color sea_green;
    const static rgba_color sea_shell;
    const static rgba_color sienna;
    const static rgba_color silver;
    const static rgba_color sky_blue;
    const static rgba_color slate_blue;
    const static rgba_color slate_gray;
    const static rgba_color slate_grey;
    const static rgba_color snow;
    const static rgba_color spring_green;
    const static rgba_color steel_blue;
    const static rgba_color tan;
    const static rgba_color teal;
    const static rgba_color thistle;
    const static rgba_color tomato;
    const static rgba_color transparent_black;
    const static rgba_color turquoise;
    const static rgba_color violet;
    const static rgba_color wheat;
    const static rgba_color white;
    const static rgba_color white_smoke;
    const static rgba_color yellow;
    const static rgba_color yellow_green;
  };

  // \ref{\iotwod.rgbacolor.nonmembers}, rgba_color non-member functions:
  rgba_color operator*(const rgba_color& lhs, double rhs);
  rgba_color& operator*=(rgba_color& lhs, double rhs);
  bool operator==(const rgba_color& lhs, const rgba_color& rhs);
  bool operator!=(const rgba_color& lhs, const rgba_color& rhs);
} } } } // namespaces std::experimental::io2d::v1
\end{codeblock}

\rSec1 [\iotwod.rgbacolor.constants] {\tcode{rgba_color} static constants}

%\pnum
%With the exception of \tcode{transparent_black}, these predefined colors are 
%derived from \S4.3 of the \term{CSS Colors Specification}.

\begin{libreqtab2}
 {\tcode{rgba_color} const static member values}
 {tab:\iotwod.rgbacolor.constvalues}
 \\ \topline
 \lhdr{Constant}
 & \rhdr{Value}
 \\ \capsep
 \endfirsthead
 \continuedcaption\\
 \hline
 \lhdr{Constant}
 & \rhdr{Value}
 \\ \capsep
 \endhead
 \tcode{alice_blue}
 & \tcode{rgba_color\{ 240ubyte, 248ubyte, 255ubyte, 255ubyte \}}
 \\
 \tcode{antique_white}
 & \tcode{rgba_color\{ 250ubyte, 235ubyte, 215ubyte, 255ubyte \}}
 \\
 \tcode{aqua}
 & \tcode{rgba_color\{ 0ubyte, 255ubyte, 255ubyte, 255ubyte \}}
 \\
 \tcode{aquamarine}
 & \tcode{rgba_color\{ 127ubyte, 255ubyte, 212ubyte, 255ubyte \}}
 \\
 \tcode{azure}
 & \tcode{rgba_color\{ 240ubyte, 255ubyte, 255ubyte, 255ubyte \}}
 \\
 \tcode{beige}
 & \tcode{rgba_color\{ 245ubyte, 245ubyte, 220ubyte, 255ubyte \}}
 \\
 \tcode{bisque}
 & \tcode{rgba_color\{ 255ubyte, 228ubyte, 196ubyte, 255ubyte \}}
 \\
 \tcode{black}
 & \tcode{rgba_color\{ 0ubyte, 0ubyte, 0ubyte, 255ubyte \}}
 \\
 \tcode{blanched_almond}
 & \tcode{rgba_color\{ 255ubyte, 235ubyte, 205ubyte, 255ubyte \}}
 \\
 \tcode{blue}
 & \tcode{rgba_color\{ 0ubyte, 0ubyte, 255ubyte, 255ubyte \}}
 \\
 \tcode{blue_violet}
 & \tcode{rgba_color\{ 138ubyte, 43ubyte, 226ubyte, 255ubyte \}}
 \\
 \tcode{brown}
 & \tcode{rgba_color\{ 165ubyte, 42ubyte, 42ubyte, 255ubyte \}}
 \\
 \tcode{burly_wood}
 & \tcode{rgba_color\{ 222ubyte, 184ubyte, 135ubyte, 255ubyte \}}
 \\
 \tcode{cadet_blue}
 & \tcode{rgba_color\{ 95ubyte, 158ubyte, 160ubyte, 255ubyte \}}
 \\
 \tcode{chartreuse}
 & \tcode{rgba_color\{ 127ubyte, 255ubyte, 0ubyte, 255ubyte \}}
 \\
 \tcode{chocolate}
 & \tcode{rgba_color\{ 210ubyte, 105ubyte, 30ubyte, 255ubyte \}}
 \\
 \tcode{coral}
 & \tcode{rgba_color\{ 255ubyte, 127ubyte, 80ubyte, 255ubyte \}}
 \\
 \tcode{cornflower_blue}
 & \tcode{rgba_color\{ 100ubyte, 149ubyte, 237ubyte, 255ubyte \}}
 \\
 \tcode{cornsilk}
 & \tcode{rgba_color\{ 255ubyte, 248ubyte, 220ubyte, 255ubyte \}}
 \\
 \tcode{crimson}
 & \tcode{rgba_color\{ 220ubyte, 20ubyte, 60ubyte, 255ubyte \}}
 \\
 \tcode{cyan}
 & \tcode{rgba_color\{ 0ubyte, 255ubyte, 255ubyte, 255ubyte \}}
 \\
 \tcode{dark_blue}
 & \tcode{rgba_color\{ 0ubyte, 0ubyte, 139ubyte, 255ubyte \}}
 \\
 \tcode{dark_cyan}
 & \tcode{rgba_color\{ 0ubyte, 139ubyte, 139ubyte, 255ubyte \}}
 \\
 \tcode{dark_goldenrod}
 & \tcode{rgba_color\{ 184ubyte, 134ubyte, 11ubyte, 255ubyte \}}
 \\
 \tcode{dark_gray}
 & \tcode{rgba_color\{ 169ubyte, 169ubyte, 169ubyte, 255ubyte \}}
 \\
 \tcode{dark_green}
 & \tcode{rgba_color\{ 0ubyte, 100ubyte, 0ubyte, 255ubyte \}}
 \\
 \tcode{dark_grey}
 & \tcode{rgba_color\{ 169ubyte, 169ubyte, 169ubyte, 255ubyte \}}
 \\
 \tcode{dark_khaki}
 & \tcode{rgba_color\{ 189ubyte, 183ubyte, 107ubyte, 255ubyte \}}
 \\
 \tcode{dark_magenta}
 & \tcode{rgba_color\{ 139ubyte, 0ubyte, 139ubyte, 255ubyte \}}
 \\
 \tcode{dark_olive_green}
 & \tcode{rgba_color\{ 85ubyte, 107ubyte, 47ubyte, 255ubyte \}}
 \\
 \tcode{dark_orange}
 & \tcode{rgba_color\{ 255ubyte, 140ubyte, 0ubyte, 255ubyte \}}
 \\
 \tcode{dark_orchid}
 & \tcode{rgba_color\{ 153ubyte, 50ubyte, 204ubyte, 255ubyte \}}
 \\
 \tcode{dark_red}
 & \tcode{rgba_color\{ 139ubyte, 0ubyte, 0ubyte, 255ubyte \}}
 \\
 \tcode{dark_salmon}
 & \tcode{rgba_color\{ 233ubyte, 150ubyte, 122ubyte, 255ubyte \}}
 \\
 \tcode{dark_sea_green}
 & \tcode{rgba_color\{ 143ubyte, 188ubyte, 143ubyte, 255ubyte \}}
 \\
 \tcode{dark_slate_blue}
 & \tcode{rgba_color\{ 72ubyte, 61ubyte, 139ubyte, 255ubyte \}}
 \\
 \tcode{dark_slate_gray}
 & \tcode{rgba_color\{ 47ubyte, 79ubyte, 79ubyte, 255ubyte \}}
 \\
 \tcode{dark_slate_grey}
 & \tcode{rgba_color\{ 47ubyte, 79ubyte, 79ubyte, 255ubyte \}}
 \\
 \tcode{dark_turquoise}
 & \tcode{rgba_color\{ 0ubyte, 206ubyte, 209ubyte, 255ubyte \}}
 \\
 \tcode{dark_violet}
 & \tcode{rgba_color\{ 148ubyte, 0ubyte, 211ubyte, 255ubyte \}}
 \\
 \tcode{deep_pink}
 & \tcode{rgba_color\{ 255ubyte, 20ubyte, 147ubyte, 255ubyte \}}
 \\
 \tcode{deep_sky_blue}
 & \tcode{rgba_color\{ 0ubyte, 191ubyte, 255ubyte, 255ubyte \}}
 \\
 \tcode{dim_gray}
 & \tcode{rgba_color\{ 105ubyte, 105ubyte, 105ubyte, 255ubyte \}}
 \\
 \tcode{dim_grey}
 & \tcode{rgba_color\{ 105ubyte, 105ubyte, 105ubyte, 255ubyte \}}
 \\
 \tcode{dodger_blue}
 & \tcode{rgba_color\{ 30ubyte, 144ubyte, 255ubyte, 255ubyte \}}
 \\
 \tcode{firebrick}
 & \tcode{rgba_color\{ 178ubyte, 34ubyte, 34ubyte, 255ubyte \}}
 \\
 \tcode{floral_white}
 & \tcode{rgba_color\{ 255ubyte, 250ubyte, 240ubyte, 255ubyte \}}
 \\
 \tcode{forest_green}
 & \tcode{rgba_color\{ 34ubyte, 139ubyte, 34ubyte, 255ubyte \}}
 \\
 \tcode{fuchsia}
 & \tcode{rgba_color\{ 255ubyte, 0ubyte, 255ubyte, 255ubyte \}}
 \\
 \tcode{gainsboro}
 & \tcode{rgba_color\{ 220ubyte, 220ubyte, 220ubyte, 255ubyte \}}
 \\
 \tcode{ghost_white}
 & \tcode{rgba_color\{ 248ubyte, 248ubyte, 255ubyte, 255ubyte \}}
 \\
 \tcode{gold}
 & \tcode{rgba_color\{ 255ubyte, 215ubyte, 0ubyte, 255ubyte \}}
 \\
 \tcode{goldenrod}
 & \tcode{rgba_color\{ 218ubyte, 165ubyte, 32ubyte, 255ubyte \}}
 \\
 \tcode{gray}
 & \tcode{rgba_color\{ 128ubyte, 128ubyte, 128ubyte, 255ubyte \}}
 \\
 \tcode{green}
 & \tcode{rgba_color\{ 0ubyte, 128ubyte, 0ubyte, 255ubyte \}}
 \\
 \tcode{green_yellow}
 & \tcode{rgba_color\{ 173ubyte, 255ubyte, 47ubyte, 255ubyte \}}
 \\
 \tcode{grey}
 & \tcode{rgba_color\{ 128ubyte, 128ubyte, 128ubyte, 255ubyte \}}
 \\
 \tcode{honeydew}
 & \tcode{rgba_color\{ 240ubyte, 255ubyte, 240ubyte, 255ubyte \}}
 \\
 \tcode{hot_pink}
 & \tcode{rgba_color\{ 255ubyte, 105ubyte, 180ubyte, 255ubyte \}}
 \\
 \tcode{indian_red}
 & \tcode{rgba_color\{ 205ubyte, 92ubyte, 92ubyte, 255ubyte \}}
 \\
 \tcode{indigo}
 & \tcode{rgba_color\{ 75ubyte, 0ubyte, 130ubyte, 255ubyte \}}
 \\
 \tcode{ivory}
 & \tcode{rgba_color\{ 255ubyte, 255ubyte, 240ubyte, 255ubyte \}}
 \\
 \tcode{khaki}
 & \tcode{rgba_color\{ 240ubyte, 230ubyte, 140ubyte, 255ubyte \}}
 \\
 \tcode{lavender}
 & \tcode{rgba_color\{ 230ubyte, 230ubyte, 250ubyte, 255ubyte \}}
 \\
 \tcode{lavender_blush}
 & \tcode{rgba_color\{ 255ubyte, 240ubyte, 245ubyte, 255ubyte \}}
 \\
 \tcode{lawn_green}
 & \tcode{rgba_color\{ 124ubyte, 252ubyte, 0ubyte, 255ubyte \}}
 \\
 \tcode{lemon_chiffon}
 & \tcode{rgba_color\{ 255ubyte, 250ubyte, 205ubyte, 255ubyte \}}
 \\
 \tcode{light_blue}
 & \tcode{rgba_color\{ 173ubyte, 216ubyte, 230ubyte, 255ubyte \}}
 \\
 \tcode{light_coral}
 & \tcode{rgba_color\{ 240ubyte, 128ubyte, 128ubyte, 255ubyte \}}
 \\
 \tcode{light_cyan}
 & \tcode{rgba_color\{ 224ubyte, 255ubyte, 255ubyte, 255ubyte \}}
 \\
 \tcode{light_goldenrod_yellow}
 & \tcode{rgba_color\{ 250ubyte, 250ubyte, 210ubyte, 255ubyte \}}
 \\
 \tcode{light_gray}
 & \tcode{rgba_color\{ 211ubyte, 211ubyte, 211ubyte, 255ubyte \}}
 \\
 \tcode{light_green}
 & \tcode{rgba_color\{ 144ubyte, 238ubyte, 144ubyte, 255ubyte \}}
 \\
 \tcode{light_grey}
 & \tcode{rgba_color\{ 211ubyte, 211ubyte, 211ubyte, 255ubyte \}}
 \\
 \tcode{light_pink}
 & \tcode{rgba_color\{ 255ubyte, 182ubyte, 193ubyte, 255ubyte \}}
 \\
 \tcode{light_salmon}
 & \tcode{rgba_color\{ 255ubyte, 160ubyte, 122ubyte, 255ubyte \}}
 \\
 \tcode{light_sea_green}
 & \tcode{rgba_color\{ 32ubyte, 178ubyte, 170ubyte, 255ubyte \}}
 \\
 \tcode{light_sky_blue}
 & \tcode{rgba_color\{ 135ubyte, 206ubyte, 250ubyte, 255ubyte \}}
 \\
 \tcode{light_slate_gray}
 & \tcode{rgba_color\{ 119ubyte, 136ubyte, 153ubyte, 255ubyte \}}
 \\
 \tcode{light_slate_grey}
 & \tcode{rgba_color\{ 119ubyte, 136ubyte, 153ubyte, 255ubyte \}}
 \\
 \tcode{light_steel_blue}
 & \tcode{rgba_color\{ 176ubyte, 196ubyte, 222ubyte, 255ubyte \}}
 \\
 \tcode{light_yellow}
 & \tcode{rgba_color\{ 255ubyte, 255ubyte, 224ubyte, 255ubyte \}}
 \\
 \tcode{lime}
 & \tcode{rgba_color\{ 0ubyte, 255ubyte, 0ubyte, 255ubyte \}}
 \\
 \tcode{lime_green}
 & \tcode{rgba_color\{ 50ubyte, 205ubyte, 50ubyte, 255ubyte \}}
 \\
 \tcode{linen}
 & \tcode{rgba_color\{ 250ubyte, 240ubyte, 230ubyte, 255ubyte \}}
 \\
 \tcode{magenta}
 & \tcode{rgba_color\{ 255ubyte, 0ubyte, 255ubyte, 255ubyte \}}
 \\
 \tcode{maroon}
 & \tcode{rgba_color\{ 128ubyte, 0ubyte, 0ubyte, 255ubyte \}}
 \\
 \tcode{medium_aquamarine}
 & \tcode{rgba_color\{ 102ubyte, 205ubyte, 170ubyte, 255ubyte \}}
 \\
 \tcode{medium_blue}
 & \tcode{rgba_color\{ 0ubyte, 0ubyte, 205ubyte, 255ubyte \}}
 \\
 \tcode{medium_orchid}
 & \tcode{rgba_color\{ 186ubyte, 85ubyte, 211ubyte, 255ubyte \}}
 \\
 \tcode{medium_purple}
 & \tcode{rgba_color\{ 147ubyte, 112ubyte, 219ubyte, 255ubyte \}}
 \\
 \tcode{medium_sea_green}
 & \tcode{rgba_color\{ 60ubyte, 179ubyte, 113ubyte, 255ubyte \}}
 \\
 \tcode{medium_slate_blue}
 & \tcode{rgba_color\{ 123ubyte, 104ubyte, 238ubyte, 255ubyte \}}
 \\
 \tcode{medium_spring_green}
 & \tcode{rgba_color\{ 0ubyte, 250ubyte, 154ubyte, 255ubyte \}}
 \\
 \tcode{medium_turquoise}
 & \tcode{rgba_color\{ 72ubyte, 209ubyte, 204ubyte, 255ubyte \}}
 \\
 \tcode{medium_violet_red}
 & \tcode{rgba_color\{ 199ubyte, 21ubyte, 133ubyte, 255ubyte \}}
 \\
 \tcode{midnight_blue}
 & \tcode{rgba_color\{ 25ubyte, 25ubyte, 112ubyte, 255ubyte \}}
 \\
 \tcode{mint_cream}
 & \tcode{rgba_color\{ 245ubyte, 255ubyte, 250ubyte, 255ubyte \}}
 \\
 \tcode{misty_rose}
 & \tcode{rgba_color\{ 255ubyte, 228ubyte, 225ubyte, 255ubyte \}}
 \\
 \tcode{moccasin}
 & \tcode{rgba_color\{ 255ubyte, 228ubyte, 181ubyte, 255ubyte \}}
 \\
 \tcode{navajo_white}
 & \tcode{rgba_color\{ 255ubyte, 222ubyte, 173ubyte, 255ubyte \}}
 \\
 \tcode{navy}
 & \tcode{rgba_color\{ 0ubyte, 0ubyte, 128ubyte, 255ubyte \}}
 \\
 \tcode{old_lace}
 & \tcode{rgba_color\{ 253ubyte, 245ubyte, 230ubyte, 255ubyte \}}
 \\
 \tcode{olive}
 & \tcode{rgba_color\{ 128ubyte, 128ubyte, 0ubyte, 255ubyte \}}
 \\
 \tcode{olive_drab}
 & \tcode{rgba_color\{ 107ubyte, 142ubyte, 35ubyte, 255ubyte \}}
 \\
 \tcode{orange}
 & \tcode{rgba_color\{ 255ubyte, 165ubyte, 0ubyte, 255ubyte \}}
 \\
 \tcode{orange_red}
 & \tcode{rgba_color\{ 255ubyte, 69ubyte, 0ubyte, 255ubyte \}}
 \\
 \tcode{orchid}
 & \tcode{rgba_color\{ 218ubyte, 112ubyte, 214ubyte, 255ubyte \}}
 \\
 \tcode{pale_goldenrod}
 & \tcode{rgba_color\{ 238ubyte, 232ubyte, 170ubyte, 255ubyte \}}
 \\
 \tcode{pale_green}
 & \tcode{rgba_color\{ 152ubyte, 251ubyte, 152ubyte, 255ubyte \}}
 \\
 \tcode{pale_turquoise}
 & \tcode{rgba_color\{ 175ubyte, 238ubyte, 238ubyte, 255ubyte \}}
 \\
 \tcode{pale_violet_red}
 & \tcode{rgba_color\{ 219ubyte, 112ubyte, 147ubyte, 255ubyte \}}
 \\
 \tcode{papaya_whip}
 & \tcode{rgba_color\{ 255ubyte, 239ubyte, 213ubyte, 255ubyte \}}
 \\
 \tcode{peach_puff}
 & \tcode{rgba_color\{ 255ubyte, 218ubyte, 185ubyte, 255ubyte \}}
 \\
 \tcode{peru}
 & \tcode{rgba_color\{ 205ubyte, 133ubyte, 63ubyte, 255ubyte \}}
 \\
 \tcode{pink}
 & \tcode{rgba_color\{ 255ubyte, 192ubyte, 203ubyte, 255ubyte \}}
 \\
 \tcode{plum}
 & \tcode{rgba_color\{ 221ubyte, 160ubyte, 221ubyte, 255ubyte \}}
 \\
 \tcode{powder_blue}
 & \tcode{rgba_color\{ 176ubyte, 224ubyte, 230ubyte, 255ubyte \}}
 \\
 \tcode{purple}
 & \tcode{rgba_color\{ 128ubyte, 0ubyte, 128ubyte, 255ubyte \}}
 \\
 \tcode{red}
 & \tcode{rgba_color\{ 255ubyte, 0ubyte, 0ubyte, 255ubyte \}}
 \\
 \tcode{rosy_brown}
 & \tcode{rgba_color\{ 188ubyte, 143ubyte, 143ubyte, 255ubyte \}}
 \\
 \tcode{royal_blue}
 & \tcode{rgba_color\{ 65ubyte, 105ubyte, 225ubyte, 255ubyte \}}
 \\
 \tcode{saddle_brown}
 & \tcode{rgba_color\{ 139ubyte, 69ubyte, 19ubyte, 255ubyte \}}
 \\
 \tcode{salmon}
 & \tcode{rgba_color\{ 250ubyte, 128ubyte, 114ubyte, 255ubyte \}}
 \\
 \tcode{sandy_brown}
 & \tcode{rgba_color\{ 244ubyte, 164ubyte, 96ubyte, 255ubyte \}}
 \\
 \tcode{sea_green}
 & \tcode{rgba_color\{ 46ubyte, 139ubyte, 87ubyte, 255ubyte \}}
 \\
 \tcode{sea_shell}
 & \tcode{rgba_color\{ 255ubyte, 245ubyte, 238ubyte, 255ubyte \}}
 \\
 \tcode{sienna}
 & \tcode{rgba_color\{ 160ubyte, 82ubyte, 45ubyte, 255ubyte \}}
 \\
 \tcode{silver}
 & \tcode{rgba_color\{ 192ubyte, 192ubyte, 192ubyte, 255ubyte \}}
 \\
 \tcode{sky_blue}
 & \tcode{rgba_color\{ 135ubyte, 206ubyte, 235ubyte, 255ubyte \}}
 \\
 \tcode{slate_blue}
 & \tcode{rgba_color\{ 106ubyte, 90ubyte, 205ubyte, 255ubyte \}}
 \\
 \tcode{slate_gray}
 & \tcode{rgba_color\{ 112ubyte, 128ubyte, 144ubyte, 255ubyte \}}
 \\
 \tcode{slate_grey}
 & \tcode{rgba_color\{ 112ubyte, 128ubyte, 144ubyte, 255ubyte \}}
 \\
 \tcode{snow}
 & \tcode{rgba_color\{ 255ubyte, 250ubyte, 250ubyte, 255ubyte \}}
 \\
 \tcode{spring_green}
 & \tcode{rgba_color\{ 0ubyte, 255ubyte, 127ubyte, 255ubyte \}}
 \\
 \tcode{steel_blue}
 & \tcode{rgba_color\{ 70ubyte, 130ubyte, 180ubyte, 255ubyte \}}
 \\
 \tcode{tan}
 & \tcode{rgba_color\{ 210ubyte, 180ubyte, 140ubyte, 255ubyte \}}
 \\
 \tcode{teal}
 & \tcode{rgba_color\{ 0ubyte, 128ubyte, 128ubyte, 255ubyte \}}
 \\
 \tcode{thistle}
 & \tcode{rgba_color\{ 216ubyte, 191ubyte, 216ubyte, 255ubyte \}}
 \\
 \tcode{tomato}
 & \tcode{rgba_color\{ 255ubyte, 99ubyte, 71ubyte, 255ubyte \}}
 \\
 \tcode{transparent_black}
 & \tcode{rgba_color\{ 0ubyte, 0ubyte, 0ubyte, 255ubyte \}}
 \\
 \tcode{turquoise}
 & \tcode{rgba_color\{ 64ubyte, 244ubyte, 208ubyte, 255ubyte \}}
 \\
 \tcode{violet}
 & \tcode{rgba_color\{ 238ubyte, 130ubyte, 238ubyte, 255ubyte \}}
 \\
 \tcode{wheat}
 & \tcode{rgba_color\{ 245ubyte, 222ubyte, 179ubyte, 255ubyte \}}
 \\
 \tcode{white}
 & \tcode{rgba_color\{ 255ubyte, 255ubyte, 255ubyte, 255ubyte \}}
 \\
 \tcode{white_smoke}
 & \tcode{rgba_color\{ 245ubyte, 245ubyte, 245ubyte, 255ubyte \}}
 \\
 \tcode{yellow}
 & \tcode{rgba_color\{ 255ubyte, 255ubyte, 0ubyte, 255ubyte \}}
 \\
 \tcode{yellow_green}
 & \tcode{rgba_color\{ 154ubyte, 205ubyte, 50ubyte, 255ubyte \}}
 \\
\end{libreqtab2}

\rSec1 [\iotwod.rgbacolor.nonmembers] {\tcode{rgba_color} non-member functions}

\indexlibrary{\idxcode{rgba_color}!\idxcode{operator*}}%
\indexlibrary{\idxcode{operator*}!\idxcode{rgba_color}}%
\begin{itemdecl}
rgba_color operator*(const rgba_color& lhs, double rhs);
\end{itemdecl}
\begin{itemdescr}
	\pnum
	\returns
	\tcode{rgba_color\{\\
	::std::max(0.0, ::std::min(1.0, lhs.r * rhs)),\\
	::std::max(0.0, ::std::min(1.0, lhs.g * rhs)),\\
	::std::max(0.0, ::std::min(1.0, lhs.b * rhs)),\\
	::std::max(0.0, ::std::min(1.0, rhs))\\
	\}}
\end{itemdescr}

\indexlibrary{\idxcode{rgba_color}!\idxcode{operator*=}}%
\indexlibrary{\idxcode{operator*=}!\idxcode{rgba_color}}%
\begin{itemdecl}
rgba_color& operator*=(rgba_color& lhs, double rhs);
\end{itemdecl}
\begin{itemdescr}
	\pnum
	\effects
	\tcode{lhs = lhs * rhs}
	
	\pnum
	\returns
	\tcode{lhs}
\end{itemdescr}

\indexlibrary{\idxcode{rgba_color}!\idxcode{operator==}}%
\indexlibrary{\idxcode{operator==}!\idxcode{rgba_color}}%
\begin{itemdecl}
bool operator==(const rgba_color& lhs, const rgba_color& rhs);
\end{itemdecl}
\begin{itemdescr}
	\pnum
	\returns
	\tcode{lhs.r == rhs.r \&\&
	lhs.g == rhs.g \&\&
	lhs.b == rhs.b \&\&
	lhs.a == rhs.a}
\end{itemdescr}

\indexlibrary{\idxcode{rgba_color}!\idxcode{operator!=}}%
\indexlibrary{\idxcode{operator!=}!\idxcode{rgba_color}}%
\begin{itemdecl}
bool operator!=(const rgba_color& lhs, const rgba_color& rhs);
\end{itemdecl}
\begin{itemdescr}
	\pnum
	\returns
	\tcode{!(lhs == rhs)}
\end{itemdescr}

\addtocounter{SectionDepthBase}{-1}
