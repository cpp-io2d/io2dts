%!TEX root = io2d.tex
\rSec0 [pathdataitem.newsubpath] {Class \tcode{path_data_item::new_sub_path}}

\pnum
\indexlibrary{\idxcode{path_data_item::new_sub_path}}
The class \tcode{path_data_item::new_sub_path} describes an operation on a path geometry collection.

\pnum
This operation starts a new path geometry. The new path geometry has no current point.

\rSec1 [pathdataitem.newsubpath.synopsis] {\tcode{path_data_item::new_sub_path} synopsis}

\begin{codeblock}
namespace std { namespace experimental { namespace io2d { inline namespace v1 {
  class path_data_item::new_sub_path {
  public:
    // construct/copy/move/destroy:
    new_sub_path() noexcept;
    new_sub_path(const new_sub_path&) noexcept;
    path_data_item::new_sub_path& operator=(const new_sub_path&) noexcept;
    new_sub_path(new_sub_path&&) noexcept;
    path_data_item::new_sub_path& operator=(new_sub_path&&) noexcept;

    // \ref{pathdataitem.newsubpath.observers}, observers:
    virtual path_data_type type() const noexcept override;
  };
} } } }
\end{codeblock}

\rSec1 [pathdataitem.newsubpath.observers]{\tcode{path_data_item::new_sub_path} observers}

\indexlibrary{\idxcode{path_data_item::new_sub_path}!\idxcode{type}}
\indexlibrary{\idxcode{type}!\idxcode{path_data_item::new_sub_path}}
\begin{itemdecl}
    virtual path_data_type type() const noexcept override;
\end{itemdecl}
\begin{itemdescr}
	\pnum
	\returns
	\tcode{path_data_type::new_sub_path}.

\end{itemdescr}
