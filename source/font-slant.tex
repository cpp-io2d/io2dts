%!TEX root = io2d.tex
\rSec0 [\iotwod.fontslant] {Enum class \tcode{font_slant}}

\rSec1 [\iotwod.fontslant.summary] {\tcode{font_slant} Summary}

\pnum
The \tcode{font_slant} enum class specifies the slant requested for rendering 
text.
\enternote
These values have different meanings for different scripts. For some scripts 
they may have no meaning at all. Further, not all typefaces will support every 
value. As such, the values are requests which should be honored if possible and 
meaningful, not requirements.
\exitnote
See Table~\ref{tab:\iotwod.fontslant.meanings} for the meaning of each
\tcode{font_slant} enumerator.

\rSec1 [\iotwod.fontslant.synopsis] {\tcode{font_slant} Synopsis}

\begin{codeblock}
namespace std { namespace experimental { namespace io2d { inline namespace v1 {
  enum class font_slant {
    normal,
    italic,
    oblique
  };
} } } } // namespaces std::experimental::io2d::v1
\end{codeblock}

\rSec1 [\iotwod.fontslant.enumerators] {\tcode{font_slant} Enumerators}
\begin{libreqtab2}
 {\tcode{font_slant} enumerator meanings}
 {tab:\iotwod.fontslant.meanings}
 \\ \topline
 \lhdr{Enumerator}
 & \rhdr{Meaning}
 \\ \capsep
 \endfirsthead
 \continuedcaption\\
 \hline
 \lhdr{Enumerator}
 & \rhdr{Meaning}
 \\ \capsep
 \endhead
 \tcode{normal}
 & The text should be rendered in whatever is a normal type for the script.
 \\
 \tcode{italic}
 & The text should be rendered in whatever is an italic type for the script.
 \enternote
 If a font does not have an italic type but does have an oblique type, the 
 oblique type shall be used if \tcode{italic} is requested.
 \exitnote
 \\
 \tcode{oblique}
 & The text should be rendered in whatever is an oblique type for the script.
 \enternote
 If a font does not have an oblique type but does have an italic type, the 
 italic type shall be used if \tcode{oblique} is requested.
 \exitnote
 \\
\end{libreqtab2}
