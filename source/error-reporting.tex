%!TEX root = io2d.tex
\rSec0[\iotwod.err.report]{Error reporting}

\pnum
2D graphics library functions that can produce errors occasionally provide two overloads: one that throws an exception to report errors and another that reports errors using an \tcode{error_code} object. This provides for situations where errors are not truly exceptional.

\pnum
report errors as follows, unless otherwise specified:

\pnum
When an error prevents the function from meeting its specifications:
% and the cause of the error is described in the function's \textit{Error conditions} description:

\begin{itemize}
\item Functions that do not take argument of type \tcode{error_code\&} throw an exception of type \tcode{system_error} or of an \impldefplain{implementation system_error-derived type} type that derives from \tcode{system_error}. The exception object shall include the enumerator specified by the function as part of its observable state.
%If the cause of the error is described in the function's \textit{Error conditions} description, the exception object includes the enumerator specified by the function. It may also include \impldefplain{error condition, additional information} additional information. Other data included in the exception object, if any, is \impldefplain{error condition, undocumented}.

\item Functions that take an argument of type \tcode{error_code\&} assigns the specified enumerator to the provided \tcode{error_code} object and then returns.
\end{itemize}

\pnum
Failure to allocate storage is reported by throwing an exception as described in [res.on.exception.handling] in \CppXVII.

\pnum
Destructor operations defined in this \documenttypename{} shall not throw exceptions. Every destructor in this \documenttypename{} shall behave as-if it had a non-throwing exception specification.

\pnum
If no error occurs in a function that takes an argument of type \tcode{error_code\&}, \tcode{error_code::clear} shall be called on the \tcode{error_code} object immediately before the function returns.
