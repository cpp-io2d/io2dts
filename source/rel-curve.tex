%!TEX root = io2d.tex
\rSec0 [pathfactory.pathrelcurve] {Class \tcode{path_factory::path_rel_curve}}

\pnum
\indexlibrary{\idxcode{path_factory::path_rel_curve}}
The class \tcode{path_factory::path_rel_curve} describes a path segment that is a cubic \bezierlocal curve.

\pnum
It has a first control point of type \tcode{vector_2d}, a second control point of type \tcode{vector_2d}, and an end point of type \tcode{vector_2d}.

\pnum
All of its points are relative to the most recently established current point.

\rSec1 [pathfactory.pathrelcurve.synopsis] {\tcode{path_factory::path_rel_curve} synopsis}

\begin{codeblock}
namespace std { namespace experimental { namespace io2d { inline namespace v1 {
  class path_factory::path_rel_curve {
  public:
    // \ref{pathfactory.pathrelcurve.cons}, construct
    path_rel_curve(const vector_2d& cp1, const vector_2d& cp2,
      const vector_2d& ep) noexcept;

    // \ref{pathfactory.pathrelcurve.modifiers}, modifiers:
    void control_point_1(const vector_2d& cp) noexcept;
    void control_point_2(const vector_2d& cp) noexcept;
    void end_point(const vector_2d& ep) noexcept;

    // \ref{pathfactory.pathrelcurve.observers}, observers:
    vector_2d control_point_1() const noexcept;
    vector_2d control_point_2() const noexcept;
    vector_2d end_point() const noexcept;
  };
} } } }
\end{codeblock}

\rSec1 [pathfactory.pathrelcurve.cons] {\tcode{path_factory::path_rel_curve} constructors}
\indexlibrary{\idxcode{path_factory::path_rel_curve_to}!constructor}
\begin{itemdecl}
    path_rel_curve(const vector_2d& cp1, const vector_2d& cp2,
      const vector_2d& ep) noexcept;
\end{itemdecl}
\begin{itemdescr}
	\pnum
	\effects
	Constructs an object of type \tcode{path_factory::path_rel_curve}.
	
	\pnum
	The first control point shall be set to the value of \tcode{cp1}.
	
	\pnum
	The second control point shall be set to the value of \tcode{cp2}.
	
	\pnum
	The end point shall be set to the value of \tcode{ep}.
\end{itemdescr}

\rSec1 [pathfactory.pathrelcurve.modifiers]{\tcode{path_factory::path_rel_curve} modifiers}

\indexlibrary{\idxcode{path_factory::path_rel_curve}!\idxcode{control_point_1}}
\indexlibrary{\idxcode{control_point_1}!\idxcode{path_factory::path_rel_curve}}
\begin{itemdecl}
    void control_point_1(const vector_2d& cp) noexcept;
\end{itemdecl}
\begin{itemdescr}
	\pnum
	\effects
	The first control point shall be set to the value of \tcode{cp}.
\end{itemdescr}

\indexlibrary{\idxcode{path_factory::path_rel_curve}!\idxcode{control_point_2}}
\indexlibrary{\idxcode{control_point_2}!\idxcode{path_factory::path_rel_curve}}
\begin{itemdecl}
    void control_point_2(const vector_2d& value) noexcept;
\end{itemdecl}
\begin{itemdescr}
	\pnum
	\effects
	The second control point shall be set to the value of \tcode{cp}.
\end{itemdescr}

\indexlibrary{\idxcode{path_factory::path_rel_curve}!\idxcode{end_point}}
\indexlibrary{\idxcode{end_point}!\idxcode{path_factory::path_rel_curve}}
\begin{itemdecl}
    void end_point(const vector_2d& value) noexcept;
\end{itemdecl}
\begin{itemdescr}
	\pnum
	\effects
	The end point shall be set to the value of \tcode{ep}.
\end{itemdescr}

\rSec1 [pathfactory.pathrelcurve.observers]{\tcode{path_factory::path_rel_curve} observers}

\indexlibrary{\idxcode{path_factory::path_rel_curve}!\idxcode{control_point_1}}
\indexlibrary{\idxcode{control_point_1}!\idxcode{path_factory::path_rel_curve}}
\begin{itemdecl}
    vector_2d control_point_1() const noexcept;
\end{itemdecl}
\begin{itemdescr}
	\pnum
	\returns
	The value of the first control point.
\end{itemdescr}

\indexlibrary{\idxcode{path_factory::path_rel_curve}!\idxcode{control_point_2}}
\indexlibrary{\idxcode{control_point_2}!\idxcode{path_factory::path_rel_curve}}
\begin{itemdecl}
    vector_2d control_point_2() const noexcept;
\end{itemdecl}
\begin{itemdescr}
	\pnum
	\returns
	The value of the second control point.
\end{itemdescr}

\indexlibrary{\idxcode{path_factory::path_rel_curve}!\idxcode{end_point}}
\indexlibrary{\idxcode{end_point}!\idxcode{path_factory::path_rel_curve}}
\begin{itemdecl}
    vector_2d end_point() const noexcept;
\end{itemdecl}
\begin{itemdescr}
	\pnum
	\returns
	The value of the end point.
\end{itemdescr}
