%!TEX root = io2d.tex
\rSec0 [\iotwod.outputsurface] {Class \tcode{basic_output_surface}}

\rSec1 [\iotwod.outputsurface.summary] {\tcode{basic_output_surface} summary}

\pnum
\indexlibrary{\idxcode{basic_output_surface}}%
<TODO>

\rSec1 [\iotwod.outputsurface.synopsis] {\tcode{basic_output_surface} synopsis}

\begin{codeblock}
namespace std::experimental::io2d::v1 {
  template <class GraphicsSurfaces>
  class basic_output_surface {
  public:
    using graphics_math_type = typename GraphicsSurfaces::graphics_math_type;

    // \ref{\iotwod.outputsurface.cons}, constructors:
    basic_output_surface(int preferredWidth, int preferredHeight, io2d::format preferredFormat,
      io2d::scaling scl = io2d::scaling::letterbox,
      io2d::refresh_style rr = io2d::refresh_style::as_fast_as_possible, float fps = 30.0f);
    basic_output_surface(int preferredWidth, int preferredHeight, io2d::format preferredFormat,
      error_code& ec, io2d::scaling scl = io2d::scaling::letterbox,
      io2d::refresh_style rr = io2d::refresh_style::as_fast_as_possible, float fps = 30.0f)
      noexcept;
    basic_output_surface(int preferredWidth, int preferredHeight, io2d::format preferredFormat,
      int preferredDisplayWidth, int preferredDisplayHeight,
      io2d::scaling scl = io2d::scaling::letterbox,
      io2d::refresh_style rr = io2d::refresh_style::as_fast_as_possible, float fps = 30.0f);
    basic_output_surface(int preferredWidth, int preferredHeight, io2d::format preferredFormat,
      int preferredDisplayWidth, int preferredDisplayHeight, error_code& ec,
      io2d::scaling scl = io2d::scaling::letterbox,
      io2d::refresh_style rr = io2d::refresh_style::as_fast_as_possible, float fps = 30.0f)
      noexcept;
	
    // \ref{\iotwod.outputsurface.observers}, observers:
    io2d::format format() const noexcept;
    basic_display_point<graphics_math_type> dimensions() const noexcept;
    basic_display_point<graphics_math_type> max_dimensions() const noexcept;
    basic_display_point<graphics_math_type> display_dimensions() const noexcept;
    basic_display_point<graphics_math_type> max_display_dimensions() const noexcept;
    io2d::scaling scaling() const noexcept;
    optional<basic_brush<GraphicsSurfaces>> letterbox_brush() const noexcept;
    optional<basic_brush_props<GraphicsSurfaces>> letterbox_brush_props() const noexcept;
    bool auto_clear() const noexcept;

    // \ref{\iotwod.outputsurface.mofifiers}, modifiers:
    int begin_show();
    void end_show();
    void clear();
    void flush();
    void flush(error_code& ec) noexcept;
    void mark_dirty();
    void mark_dirty(error_code& ec) noexcept;
    void mark_dirty(const basic_bounding_box<graphics_math_type>& extents);
    void mark_dirty(const basic_bounding_box<graphics_math_type>& extents, error_code& ec)
      noexcept;
    void paint(const basic_brush<GraphicsSurfaces>& b,
      const optional<basic_brush_props<GraphicsSurfaces>>& bp = nullopt,
      const optional<basic_render_props<GraphicsSurfaces>>& rp = nullopt,
      const optional<basic_clip_props<GraphicsSurfaces>>& cl = nullopt);
    template <class Allocator>
    void stroke(const basic_brush<GraphicsSurfaces>& b,
      const basic_path_builder<GraphicsSurfaces, Allocator>& pb,
      const optional<basic_brush_props<GraphicsSurfaces>>& bp = nullopt,
      const optional<basic_stroke_props<GraphicsSurfaces>>& sp = nullopt,
      const optional<basic_dashes<GraphicsSurfaces>>& d = nullopt,
      const optional<basic_render_props<GraphicsSurfaces>>& rp = nullopt,
      const optional<basic_clip_props<GraphicsSurfaces>>& cl = nullopt);
    void stroke(const basic_brush<GraphicsSurfaces>& b,
      const basic_interpreted_path<GraphicsSurfaces>& ip,
      const optional<basic_brush_props<GraphicsSurfaces>>& bp = nullopt,
      const optional<basic_stroke_props<GraphicsSurfaces>>& sp = nullopt,
      const optional<basic_dashes<GraphicsSurfaces>>& d = nullopt,
      const optional<basic_render_props<GraphicsSurfaces>>& rp = nullopt,
      const optional<basic_clip_props<GraphicsSurfaces>>& cl = nullopt);
    template <class Allocator>
    void fill(const basic_brush<GraphicsSurfaces>& b,
      const basic_path_builder<GraphicsSurfaces, Allocator>& pb,
      const optional<basic_brush_props<GraphicsSurfaces>>& bp = nullopt,
      const optional<basic_render_props<GraphicsSurfaces>>& rp = nullopt,
      const optional<basic_clip_props<GraphicsSurfaces>>& cl = nullopt);
    void fill(const basic_brush<GraphicsSurfaces>& b,
      const basic_interpreted_path<GraphicsSurfaces>& ip,
      const optional<basic_brush_props<GraphicsSurfaces>>& bp = nullopt,
      const optional<basic_render_props<GraphicsSurfaces>>& rp = nullopt,
      const optional<basic_clip_props<GraphicsSurfaces>>& cl = nullopt);
    void mask(const basic_brush<GraphicsSurfaces>& b,
      const basic_brush<GraphicsSurfaces>& mb,
      const optional<basic_brush_props<GraphicsSurfaces>>& bp = nullopt,
      const optional<basic_mask_props<GraphicsSurfaces>>& mp = nullopt,
      const optional<basic_render_props<GraphicsSurfaces>>& rp = nullopt,
      const optional<basic_clip_props<GraphicsSurfaces>>& cl = nullopt);
    void draw_callback(const function<void(basic_output_surface& sfc)>& fn);
    void size_change_callback(const function<void(basic_output_surface& sfc)>& fn);
    void dimensions(basic_display_point<graphics_math_type> dp);
    void dimensions(basic_display_point<graphics_math_type> dp, error_code& ec) noexcept;
    void display_dimensions(basic_display_point<graphics_math_type> dp);
    void display_dimensions(basic_display_point<graphics_math_type> dp, error_code& ec) noexcept;
    void scaling(io2d::scaling scl) noexcept;
    void user_scaling_callback(const
      function<basic_bounding_box<graphics_math_type>(const basic_output_surface&, bool&)>& fn);
    void letterbox_brush(const optional<basic_brush<GraphicsSurfaces>>& b,
      const optional<basic_brush_props<GraphicsSurfaces>>& bp = nullopt) noexcept;
    void letterbox_brush_props(const optional<basic_brush_props<GraphicsSurfaces>>& bp) noexcept;
    void auto_clear(bool val) noexcept;
    void redraw_required(bool val = true) noexcept;
  };
}
\end{codeblock}

\rSec1 [\iotwod.outputsurface.cons] {\tcode{basic_output_surface} constructors}

\indexlibrary{\idxcode{basic_output_surface}!constructor}%
\begin{itemdecl}
basic_output_surface(int preferredWidth, int preferredHeight, io2d::format preferredFormat,
  io2d::scaling scl = io2d::scaling::letterbox,
  io2d::refresh_style rr = io2d::refresh_style::as_fast_as_possible, float fps = 30.0f);
basic_output_surface(int preferredWidth, int preferredHeight, io2d::format preferredFormat,
  error_code& ec, io2d::scaling scl = io2d::scaling::letterbox,
  io2d::refresh_style rr = io2d::refresh_style::as_fast_as_possible, float fps = 30.0f)
  noexcept;
\end{itemdecl}
\begin{itemdescr}
\pnum
<TODO>
\end{itemdescr}

\indexlibrary{\idxcode{basic_output_surface}!constructor}%
\begin{itemdecl}
basic_output_surface(int preferredWidth, int preferredHeight, io2d::format preferredFormat,
  int preferredDisplayWidth, int preferredDisplayHeight,
  io2d::scaling scl = io2d::scaling::letterbox,
  io2d::refresh_style rr = io2d::refresh_style::as_fast_as_possible, float fps = 30.0f);
basic_output_surface(int preferredWidth, int preferredHeight,
  io2d::format preferredFormat, int preferredDisplayWidth, int preferredDisplayHeight,
  error_code& ec, io2d::scaling scl = io2d::scaling::letterbox,
  io2d::refresh_style rr = io2d::refresh_style::as_fast_as_possible, float fps = 30.0f)
  noexcept;
\end{itemdecl}
\begin{itemdescr}
\pnum
<TODO>
\end{itemdescr}

\rSec1 [\iotwod.outputsurface.observers] {\tcode{basic_output_surface} observers}

\indexlibrarymember{format}{basic_output_surface}%
\begin{itemdecl}
io2d::format format() const noexcept;
\end{itemdecl}
\begin{itemdescr}
\pnum
\returns
The pixel format.
\end{itemdescr}

\indexlibrarymember{dimensions}{basic_output_surface}%
\begin{itemdecl}
basic_display_point<graphics_math_type> dimensions() const noexcept;
\end{itemdecl}
\begin{itemdescr}
\pnum
\returns
<TODO>The height and width.
\end{itemdescr}

\indexlibrarymember{max_dimensions}{basic_output_surface}%
\begin{itemdecl}
basic_display_point<graphics_math_type> max_dimensions() const noexcept;
\end{itemdecl}
\begin{itemdescr}
\pnum
\returns
<TODO>The maximum available height and width of a \tcode{basic_output_surcace} for the device.
\end{itemdescr}

\indexlibrarymember{display_dimensions}{basic_output_surface}%
\begin{itemdecl}
basic_display_point<graphics_math_type> display_dimensions() const noexcept;
\end{itemdecl}
\begin{itemdescr}
\pnum
\returns
<TODO>
\end{itemdescr}

\indexlibrarymember{max_display_dimensions}{basic_output_surface}%
\begin{itemdecl}
basic_display_point<graphics_math_type> max_display_dimensions() const noexcept;
\end{itemdecl}
\begin{itemdescr}
\pnum
\returns
<TODO>
\end{itemdescr}

\indexlibrarymember{scaling}{basic_output_surface}%
\begin{itemdecl}
io2d::scaling scaling() const noexcept;
\end{itemdecl}
\begin{itemdescr}
\pnum
\returns
The scaling type.
\end{itemdescr}

\indexlibrarymember{letterbox_brush}{basic_output_surface}%
\begin{itemdecl}
optional<basic_brush<GraphicsSurfaces>> letterbox_brush() const noexcept;
\end{itemdecl}
\begin{itemdescr}
\pnum
\returns
An \tcode{optional<basic_brush<GraphicsSurfaces>>} object constructed using the user-provided letterbox brush or, if the letterbox brush is set to its default value, an empty \tcode{optional<basic_brush<GraphicsSurfaces>>} object.
\end{itemdescr}

\indexlibrarymember{letterbox_brush_props}{basic_output_surface}%
\begin{itemdecl}
optional<basic_brush_props<GraphicsSurfaces>> letterbox_brush_props() const noexcept;
\end{itemdecl}
\begin{itemdescr}
\pnum
\returns
An \tcode{optional<basic_brush_props<GraphicsSurfaces>>} object constructed using the user-provided letterbox brush props or, if the letterbox brush props is set to its default value, an empty \tcode{optional<basic_brush_props<GraphicsSurfaces>>} object.
\end{itemdescr}

\indexlibrarymember{auto_clear}{basic_output_surface}%
\begin{itemdecl}
bool auto_clear() const noexcept;
\end{itemdecl}
\begin{itemdescr}
\pnum
\returns
The value of auto clear.
\end{itemdescr}

\rSec1 [\iotwod.outputsurface.mofifiers] {\tcode{basic_output_surface} modifiers}

\indexlibrarymember{begin_show}{basic_output_surface}%
\begin{itemdecl}
int begin_show();
\end{itemdecl}
\begin{itemdescr}
\pnum
\effects
Performs the following actions in a continuous loop:
\begin{enumerate}
\item Handle any implementation and host environment matters. If there are no pending implementation or host environment matters to handle, proceed immediately to the next action.
\item Run the size change callback if doing so is required by its specification and it does not have a value equivalent to its default value.
\item If the refresh rate requires that the draw callback be called then:
\begin{enumeratea}
\item Evaluate auto clear and perform the actions required by its specification, if any.
\item Run the draw callback.
\item Ensure that all operations from the draw callback that can effect the back buffer have completed.
\item Transfer the contents of the back buffer to the display buffer using sampling with an \unspecnorm filter. If the user scaling callback does not have a value equivalent to its default value, use it to determine the position where the contents of the back buffer shall be transferred to and whether or not the letterbox brush should be used. Otherwise use the value of scaling type to determine the position and whether the letterbox brush should be used.
\end{enumeratea}
\end{enumerate}

\pnum
If \tcode{basic_output_surface::end_show} is called from the draw callback, the implementation shall finish executing the draw callback and shall immediately cease to perform any actions in the continuous loop other than handling any implementation and host environment matters needed to exit the loop properly.

\pnum
No later than when this function returns, the output device shall cease to display the contents of the display buffer.

\pnum
What the output device shall display when it is not displaying the contents of the display buffer is \unspecnorm.

\pnum
\returns
The possible values and meanings of the possible values returned are \impldefplain{basic_output_surface!begin_show return value}.

\pnum
\throws
As specified in Error reporting (\ref{\iotwod.err.report}).

\pnum
\remarks
Since this function calls the draw callback and can call the size change callback and the user scaling callback, in addition to the errors documented below, any errors that the callback functions produce can also occur.

\pnum
\errors
\tcode{errc::operation_would_block} if the value of draw callback is equivalent to its default value or if it becomes equivalent to its default value before this function returns.

\pnum
Other errors, if any, produced by this function are \impldefplain{basic_output_surface!begin_show}.
\end{itemdescr}

\indexlibrarymember{end_show}{basic_output_surface}%
\begin{itemdecl}
void end_show();
\end{itemdecl}
\begin{itemdescr}
\pnum
\effects
If this function is called outside of the draw callback while it is being executed in the \tcode{basic_output_surface::begin_show} function's continuous loop, it does nothing.

\pnum
Otherwise, the implementation initiates the process of exiting the \tcode{basic_output_surface::begin_show} function's continuous loop.

\pnum
If possible, any procedures that the host environment requires in order to cause the \tcode{basic_output_surface::show} function's continuous loop to stop executing without error should be followed.

\pnum
The \tcode{basic_output_surface::begin_show} function's loop continues execution until it returns.
\end{itemdescr}

\indexlibrarymember{clear}{basic_output_surface}%
\begin{itemdecl}
void clear();
\end{itemdecl}
\begin{itemdescr}
\pnum
\effects
<TODO>
\end{itemdescr}

\indexlibrarymember{flush}{basic_output_surface}%
\begin{itemdecl}
void flush();
void flush(error_code& ec) noexcept;
\end{itemdecl}
\begin{itemdescr}
\pnum
\effects
If the implementation does not provide a native handle to the surface's visual data, this function does nothing.

\pnum
If the implementation does provide a native handle to the surface's visual data, then the implementation performs every action necessary to ensure that all operations on the surface that produce observable effects occur.

\pnum
The implementation performs any other actions necessary to ensure that the surface will be usable again after a call to \tcode{basic_output_surface::mark_dirty}.

\pnum
Once a call to \tcode{basic_output_surface::flush} is made, \tcode{basic_output_surface::mark_dirty} shall be called before any other member function of the surface is called or the surface is used as an argument to any other function.

\pnum
\throws
As specified in Error reporting (\ref{\iotwod.err.report}).

\pnum
\remarks
This function exists to allow the user to take control of the underlying surface using an implementation-provided native handle without introducing a race condition. The implementation's responsibility is to ensure that the user can safely use the underlying surface.

\pnum
\errors
The potential errors are \impldefplain{basic_output_surface::flush errors}.

\pnum
Implementations should avoid producing errors here.

\pnum
If the implementation does not provide a native handle to the \tcode{basic_output_surface} object's visual data, this function shall not produce any errors.

\pnum
\begin{note}
There are several purposes for \tcode{basic_output_surface::flush} and \tcode{basic_output_surface::mark_dirty}.

\pnum
One is to allow implementation wide latitude in how they implement the rendering and composing operations (\ref{\iotwod.surface.rendering}), such as batching calls and then sending them to graphics acceleration hardware at appropriate times.

\pnum
Another is to give implementations the chance during the call to \tcode{basic_output_surface::flush} to save any internal state that might be modified by the user and then restore it during the call to \tcode{basic_output_surface::mark_dirty}.

\pnum
Other uses of this pair of calls are also possible.
\end{note}
\end{itemdescr}

\indexlibrarymember{mark_dirty}{basic_output_surface}%
\begin{itemdecl}
void mark_dirty();
void mark_dirty(error_code& ec) noexcept;
void mark_dirty(const basic_bounding_box<graphics_math_type>& extents);
void mark_dirty(const basic_bounding_box<graphics_math_type>& extents, error_code& ec) noexcept;
\end{itemdecl}
\begin{itemdescr}
\pnum
\effects
If the implementation does not provide a native handle to the \tcode{basic_output_surface} object's visual data, this function shall do nothing.

\pnum
If the implementation does provide a native handle to the \tcode{basic_output_surface} object's visual data, then:
\begin{itemize}
\item If called without a \tcode{basic_bounding_box} argument, informs the implementation that external changes using a native handle potentially modified all of the surface's visual data.
\item If called with a \tcode{basic_bounding_box} argument, informs the implementation that external changes using a native handle potentially modified the surface's visual data within the bounds specified by the \term{bounding rectangle} \tcode{basic_bounding_box\{ round(extents.x()), round (extents.y()), round(extents.width()), round(extents.height())\}}. No part of the bounding rectangle shall be outside of the bounds of the surface's visual data; no diagnostic is required.
\end{itemize}

\pnum
\throws
As specified in Error reporting (\ref{\iotwod.err.report}).

\pnum
\remarks
After external changes are made to this \tcode{basic_output_surface} object's visual data using a native pointer, this function shall be called before using this \tcode{basic_output_surface} object; no diagnostic is required.

\pnum
\errors
The errors, if any, produced by this function are \impldefplain{basic_output_surface!mark_dirty}.

\pnum
If the implementation does not provide a native handle to the \tcode{basic_output_surface} object's visual data, this function shall not produce any errors.
\end{itemdescr}

\indexlibrarymember{paint}{basic_output_surface}%
\begin{itemdecl}
void paint(const basic_brush<GraphicsSurfaces>& b,
  const optional<basic_brush_props<GraphicsSurfaces>>& bp = nullopt,
  const optional<basic_render_props<GraphicsSurfaces>>& rp = nullopt,
  const optional<basic_clip_props<GraphicsSurfaces>>& cl = nullopt);
\end{itemdecl}
\begin{itemdescr}
\pnum
\effects
Performs the painting rendering and composing operation as specified by \ref{\iotwod.surface.painting}.

\pnum
The meanings of the parameters are specified by \ref{\iotwod.surface.rendering}.

\pnum
\throws
As specified in Error reporting (\ref{\iotwod.err.report}).

\pnum
\errors
The errors, if any, produced by this function are \impldefplain{basic_output_surface!paint}.
\end{itemdescr}

\indexlibrarymember{stroke}{basic_output_surface}%
\begin{itemdecl}
template <class Allocator>
void stroke(const basic_brush<GraphicsSurfaces>& b,
  const basic_path_builder<GraphicsSurfaces, Allocator>& pb,
  const optional<basic_brush_props<GraphicsSurfaces>>& bp = nullopt,
  const optional<basic_stroke_props<GraphicsSurfaces>>& sp = nullopt,
  const optional<basic_dashes<GraphicsSurfaces>>& d = nullopt,
  const optional<basic_render_props<GraphicsSurfaces>>& rp = nullopt,
  const optional<basic_clip_props<GraphicsSurfaces>>& cl = nullopt);
void stroke(const basic_brush<GraphicsSurfaces>& b,
  const basic_interpreted_path<GraphicsSurfaces>& ip,
  const optional<basic_brush_props<GraphicsSurfaces>>& bp = nullopt,
  const optional<basic_stroke_props<GraphicsSurfaces>>& sp = nullopt,
  const optional<basic_dashes<GraphicsSurfaces>>& d = nullopt,
  const optional<basic_render_props<GraphicsSurfaces>>& rp = nullopt,
  const optional<basic_clip_props<GraphicsSurfaces>>& cl = nullopt);
\end{itemdecl}
\begin{itemdescr}
\pnum
\effects
Performs the stroking rendering and composing operation as specified by \ref{\iotwod.surface.stroking}.

\pnum
The meanings of the parameters are specified by \ref{\iotwod.surface.rendering}.

\pnum
\throws
As specified in Error reporting (\ref{\iotwod.err.report}).

\pnum
\errors
The errors, if any, produced by this function are \impldefplain{basic_output_surface!stroke}.
\end{itemdescr}

\indexlibrarymember{fill}{basic_output_surface}%
\begin{itemdecl}
template <class Allocator>
void fill(const basic_brush<GraphicsSurfaces>& b,
  const basic_path_builder<GraphicsSurfaces, Allocator>& pb,
  const optional<basic_brush_props<GraphicsSurfaces>>& bp = nullopt,
  const optional<basic_render_props<GraphicsSurfaces>>& rp = nullopt,
  const optional<basic_clip_props<GraphicsSurfaces>>& cl = nullopt);
void fill(const basic_brush<GraphicsSurfaces>& b,
  const basic_interpreted_path<GraphicsSurfaces>& ip,
  const optional<basic_brush_props<GraphicsSurfaces>>& bp = nullopt,
  const optional<basic_render_props<GraphicsSurfaces>>& rp = nullopt,
  const optional<basic_clip_props<GraphicsSurfaces>>& cl = nullopt);
\end{itemdecl}
\begin{itemdescr}
\pnum
\effects
Performs the filling rendering and composing operation as specified by \ref{\iotwod.surface.filling}.

\pnum
The meanings of the parameters are specified by \ref{\iotwod.surface.rendering}.

\pnum
\throws
As specified in Error reporting (\ref{\iotwod.err.report}).

\pnum
\errors
The errors, if any, produced by this function are \impldefplain{basic_output_surface!fill}.
\end{itemdescr}

\indexlibrarymember{mask}{basic_output_surface}%
\begin{itemdecl}
void mask(const basic_brush<GraphicsSurfaces>& b,
  const basic_brush<GraphicsSurfaces>& mb,
  const optional<basic_brush_props<GraphicsSurfaces>>& bp = nullopt,
  const optional<basic_mask_props<GraphicsSurfaces>>& mp = nullopt,
  const optional<basic_render_props<GraphicsSurfaces>>& rp = nullopt,
  const optional<basic_clip_props<GraphicsSurfaces>>& cl = nullopt);
\end{itemdecl}
\begin{itemdescr}
\pnum
\effects
Performs the masking rendering and composing operation as specified by \ref{\iotwod.surface.masking}.

\pnum
The meanings of the parameters are specified by \ref{\iotwod.surface.rendering}.

\pnum
\throws
As specified in Error reporting (\ref{\iotwod.err.report}).

\pnum
\errors

The errors, if any, produced by this function are \impldefplain{surface!mask}.
\end{itemdescr}

\indexlibrarymember{draw_callback}{basic_output_surface}%
\begin{itemdecl}
void draw_callback(const function<void(basic_output_surface& sfc)>& fn);
\end{itemdecl}
\begin{itemdescr}
\pnum
\effects
<TODO>
\end{itemdescr}

\indexlibrarymember{size_change_callback}{basic_output_surface}%
\begin{itemdecl}
void size_change_callback(const function<void(basic_output_surface& sfc)>& fn);
\end{itemdecl}
\begin{itemdescr}
\pnum
\effects
<TODO>
\end{itemdescr}

\indexlibrarymember{dimensions}{basic_output_surface}%
\begin{itemdecl}
void dimensions(basic_display_point<graphics_math_type> dp);
void dimensions(basic_display_point<graphics_math_type> dp, error_code& ec) noexcept;
\end{itemdecl}
\begin{itemdescr}
\pnum
\effects
<TODO>
\end{itemdescr}

\indexlibrarymember{display_dimensions}{basic_output_surface}%
\begin{itemdecl}
void display_dimensions(basic_display_point<graphics_math_type> dp);
void display_dimensions(basic_display_point<graphics_math_type> dp, error_code& ec) noexcept;
\end{itemdecl}
\begin{itemdescr}
\pnum
\effects
<TODO>
\end{itemdescr}

\indexlibrarymember{scaling}{basic_output_surface}%
\begin{itemdecl}
void scaling(io2d::scaling scl) noexcept;
\end{itemdecl}
\begin{itemdescr}
\pnum
\effects
<TODO>
\end{itemdescr}

\indexlibrarymember{user_scaling_callback}{basic_output_surface}%
\begin{itemdecl}
void user_scaling_callback(const
  function<basic_bounding_box<graphics_math_type>(const basic_output_surface&, bool&)>& fn);
\end{itemdecl}
\begin{itemdescr}
\pnum
\effects
<TODO>
\end{itemdescr}

\indexlibrarymember{letterbox_brush}{basic_output_surface}%
\begin{itemdecl}
void letterbox_brush(const optional<basic_brush<GraphicsSurfaces>>& b,
  const optional<basic_brush_props<GraphicsSurfaces>>& bp = nullopt) noexcept;
void letterbox_brush_props(const optional<basic_brush_props<GraphicsSurfaces>>& bp) noexcept;
\end{itemdecl}
\begin{itemdescr}
\pnum
\effects
<TODO>
\end{itemdescr}

\indexlibrarymember{auto_clear}{basic_output_surface}%
\begin{itemdecl}
void auto_clear(bool val) noexcept;
\end{itemdecl}
\begin{itemdescr}
\pnum
\effects
<TODO>
\end{itemdescr}

\indexlibrarymember{redraw_required}{basic_output_surface}%
\begin{itemdecl}
void redraw_required(bool val = true) noexcept;
\end{itemdecl}
\begin{itemdescr}
\pnum
\effects
<TODO>
\end{itemdescr}