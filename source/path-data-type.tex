%!TEX root = io2d.tex
\rSec0 [\iotwod.pathdatatype] {Enum class \tcode{path_data_type}}

\rSec1 [\iotwod.pathdatatype.summary] {\tcode{path_data_type} Summary}

\pnum
The \tcode{path_data_type} enum class specifies the polymorphic type of a 
\tcode{path_data} object.
See Table~\ref{tab:\iotwod.pathdatatype.meanings} for the meaning of each
\tcode{path_data_type} enumerator.

\rSec1 [\iotwod.pathdatatype.synopsis] {\tcode{path_data_type} Synopsis}

\begin{codeblock}
namespace std { namespace experimental { namespace io2d { inline namespace v1 {
  enum class path_data_type {
    move_to,
    line_to,
    curve_to,
    new_sub_path,
    close_path,
    rel_move_to,
    rel_line_to,
    rel_curve_to,
    arc,
    arc_negative,
    change_matrix,
    change_origin
  };
} } } }
\end{codeblock}

\rSec1 [\iotwod.pathdatatype.enumerators] {\tcode{path_data_type} Enumerators}

\begin{libreqtab2}
 {\tcode{path_data_type} enumerator meanings}
 {tab:\iotwod.pathdatatype.meanings}
 \\ \topline
 \lhdr{Enumerator}
 & \rhdr{Meaning}
 \\ \capsep
 \endfirsthead
 \continuedcaption\\
 \hline
 \lhdr{Enumerator}
 & \rhdr{Meaning}
 \\ \capsep
 \endhead
 \tcode{move_to}
 & Set the \term{current point} and the \term{last move to point}. This starts 
 a new sub-path.
 \\
 \tcode{line_to}
 & Create a line from the \tcode{current point} to \tcode{p.data.line}. Make 
 the new point the \term{current point}.
 \\
 \tcode{curve_to}
 & Create a cubic B\'ezier curve. The start point is the \term{current 
 point}. The first control point is \tcode{p.data.curve.pt1}. The second 
 control point is \tcode{p.data.curve.pt2}. The end point is 
 \tcode{p.data.curve.pt3}. Set the \term{current point} to be equal to 
 \tcode{p.data.curve.pt3}.
 \\
 \tcode{new_sub_path}
 & Create a new sub-path. There is no \tcode{current point} or \tcode{last move 
 to point} after this operation.
 \\
 \tcode{close_path}
 & If there is a \term{current point}, close the \term{current path} by 
 creating a line from the \tcode{current point} to the \tcode{last move to 
 point}, then set the \term{current point} to the value of the \term{last move 
 to point}. If there is no \term{current point} then do nothing.
 The line created here, if any, differs from \tcode{line_to} in that the line 
 here is joined to the initial path segment instead of each end having line 
 caps.
 \\
 \tcode{rel_move_to}
 & Adds the point in \tcode{p.data.move} to the \term{current point} and then 
 sets the \term{current point} and the \term{last move to point} to be equal to 
 the resulting point.
 \\
 \tcode{rel_line_to}
 & Adds the point in \tcode{p.data.line} to the \term{current point} and 
 creates a line from the \term{current point} to the resulting point. Make the 
 resulting point the new \term{current point}.
 \\
 \tcode{rel_curve_to}
 & Create a B\'ezier curve. The start point is the \term{current point}. 
 The first control point is the point created by adding \term{current point} 
 and \tcode{p.data.curve.pt1}. The second control point is the point created by 
 adding \term{current point} and \tcode{p.data.curve.pt2}. The end point is the 
 point created by adding \term{current point} and \tcode{p.data.curve.pt3}. 
 Make the point created by adding \term{current point} and 
 \tcode{p.data.curve.pt3} the new \tcode{current point}.
 \\
 \tcode{arc}
 & Create a circular arc. \enternote The transformation matrix is used to modify a circular arc to create an elliptical arc. \exitnote The center of the arc is \tcode{p.data.arc.center}. Its radius is \tcode{p.data.arc.radius}. It starts at the point at \tcode{p.data.arc.angle1} radians which is \tcode{p.data.arc.radius} units from \tcode{p.data.arc.center} and proceeds clockwise to the point at \tcode{p.data.arc.angle2} radians which is \tcode{p.data.arc.radius} units from \tcode{p.data.arc.center}. If \tcode{p.data.arc.angle2} is less that \tcode{p.data.arc.angle1}, continually increase \tcode{p.data.arc.angle2} by $2 \times \pi$ until it is equal to or greater than \tcode{p.data.arc.angle1}.
 \\
 \tcode{arc_negative}
 & Create a circular arc. \enternote The transformation matrix is used to modify a circular arc to create an elliptical arc. \exitnote The center of the arc is \tcode{p.data.arc.center}. Its radius is \tcode{p.data.arc.radius}. It starts at the point at \tcode{p.data.arc.angle1} radians which is \tcode{p.data.arc.radius} units from \tcode{p.data.arc.center} and proceeds counterclockwise to the point at \tcode{p.data.arc.angle2} radians which is \tcode{p.data.arc.radius} units from \tcode{p.data.arc.center}. If \tcode{p.data.arc.angle2} is greater that \tcode{p.data.arc.angle1}, continually decrease \tcode{p.data.arc.angle2} by $2 \times \pi$ until it is equal to or less than \tcode{p.data.arc.angle1}.
 \\
 \tcode{change_matrix}
 & Establish a new transformation matrix.
 \\
 \tcode{change_origin}
 & Establish a new origin point.
 \\
\end{libreqtab2}
