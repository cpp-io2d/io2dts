%!TEX root = io2d.tex
\rSec0 [pathbuilder] {Class \tcode{path_builder}}

\pnum
\indexlibrary{\idxcode{path_builder}}
The class \tcode{path_builder} is a container that stores and manipulates objects of type \tcode{path_data::data} from which \tcode{path_group} objects are created.

\pnum
A \tcode{path_builder} is a contiguous container. (See [container.requirements.general] in \cppseventeen.)

\pnum
The collection of \tcode{path_data::data} objects in a path builder is referred to as its path group.

\pnum
In addition to its path group, a path builder has an origin of type \tcode{vector_2d}, a transformation matrix of type \tcode{matrix_2d}, a current point of type \tcode{std::optional<vector_2d>}, and a last-move-to point of type \tcode{vector_2d}.

\pnum
When a path builder is default constructed:

\begin{itemize}
\item The path group shall be empty.
\item The current point shall not contain a value.
\item The transformation matrix shall have a value of \tcode{matrix_2d::init_identity\{ \}}.
\item The origin shall have a value of \tcode{vector_2d\{ \}}.
\end{itemize}

\rSec1 [pathbuilder.synopsis] {\tcode{path_builder} synopsis}

\begin{codeblock}
namespace std { namespace experimental { namespace io2d { inline namespace v1 {
  template <class Allocator = allocator<path_data::path_data_types>>
  class path_builder {
  public:
    using value_type = path_data::path_data_types;
    using allocator_type = Allocator;
    using reference = value_type&;
    using const_reference = const value_type&;
    using size_type       = @\impdefx{type of \tcode{path_builder::size_type}}@. // See [container.requirements] in \cppseventeen.
    using difference_type = @\impdefx{type of \tcode{path_builder::size_type}}@. // See [container.requirements] in \cppseventeen.
    using iterator       = @\impdefx{type of \tcode{path_builder::iterator}}@. // See [container.requirements] in \cppseventeen.
    using const_iterator = @\impdefx{type of \tcode{path_builder::const_iterator}}@. // See [container.requirements] in \cppseventeen.
    using reverse_iterator       = std::reverse_iterator<iterator>;
    using const_reverse_iterator = std::reverse_iterator<const_iterator>;
    
    // \ref{pathbuilder.cons}, construct, copy, move, destroy:
    path_builder() noexcept(noexcept(Allocator())) :
      path_builder(Allocator()) { }
    explicit path_builder(const Allocator&) noexcept;
    explicit path_builder(size_type n, const Allocator& = Allocator());
    path_builder(size_type n, const value_type& value,
      const Allocator& = Allocator());
    template <class InputIterator>
    path_builder(InputIterator first, InputIterator last,
      const Allocator& = Allocator());
    path_builder(const path_builder& x);
    path_builder(path_builder&&) noexcept;
    path_builder(const path_builder&, const Allocator&);
    path_builder(path_builder&&, const Allocator&);
    path_builder(initializer_list<value_type>, const Allocator& = Allocator());
    ~path_builder();
    path_builder& operator=(const path_builder& x);
    path_builder& operator=(path_builder&& x)
      noexcept(
      allocator_traits<Allocator>::propagate_on_container_move_assignment::value
      ||
      allocator_traits<Allocator>::is_always_equal::value);
    path_builder& operator=(initializer_list<value_type>);
    template <class InputIterator>
    void assign(InputIterator first, InputIterator last);
    void assign(size_type n, const value_type& u);
    void assign(initializer_list<value_type>);
    allocator_type get_allocator() const noexcept;
    
    // \ref{pathbuilder.iterators}, iterators:
    iterator begin() noexcept;
    const_iterator begin() const noexcept;
    const_iterator cbegin() const noexcept;

    iterator end() noexcept;
    const_iterator end() const noexcept;
    const_iterator cend() const noexcept;
    
    reverse_iterator rbegin() noexcept;
    const_reverse_iterator rbegin() const noexcept;
    const_reverse_iterator crbegin() const noexcept;

    reverse_iterator rend() noexcept;
    const_reverse_iterator rend() const noexcept;
    const_reverse_iterator crend() const noexcept;
    
    // \ref{pathbuilder.capacity}, capacity
    bool empty() const noexcept;
    size_type size() const noexcept;
    size_type max_size() const noexcept;
    size_type capacity() const noexcept;
    void resize(size_type sz);
    void resize(size_type sz, const value_type& c);
    void reserve(size_type n);
    void shrink_to_fit();

    // element access:
    reference operator[](size_type n);
    const_reference operator[](size_type n) const;
    const_reference at(size_type n) const;
    reference at(size_type n);
    reference front();
    const_reference front() const;
    reference back();
    const_reference back() const;

    // \ref{pathbuilder.modifiers}, modifiers:
    void new_path() noexcept;
    void close_path() noexcept;
    void arc_clockwise(const vector_2d& center, double radius, double angle1,
      double angle2) noexcept;
    void arc_counterclockwise(const vector_2d& center, double radius,
      double angle1, double angle2) noexcept;
    void ellipse(const experimental::io2d::ellipse& val) noexcept;
    void ellipse(const circle& val) noexcept;
    void cubic_curve_to(const vector_2d& pt0, const vector_2d& pt1,
      const vector_2d& pt2) noexcept;
    void line_to(const vector_2d& pt) noexcept;
    void move_to(const vector_2d& pt) noexcept;
    void quadratic_curve_to(const vector_2d& pt0, const vector_2d& pt2)
      noexcept;
    void rectangle(const experimental::io2d::rectangle& r) noexcept;
    void rel_rectangle(const experimental::io2d::rectangle& r) noexcept;
    void rel_ellipse(const experimental::io2d::ellipse& val) noexcept;
    void rel_ellipse(const circle& val) noexcept;
    void rel_cubic_curve_to(const vector_2d& dpt0, const vector_2d& dpt1,
    const vector_2d& dpt2) noexcept;
    void rel_line_to(const vector_2d& dpt) noexcept;
    void rel_move_to(const vector_2d& dpt) noexcept;
    void rel_quadratic_curve_to(const vector_2d& pt0, const vector_2d& pt2)
      noexcept;
    void transform_matrix(const matrix_2d& m) noexcept;
    void origin(const vector_2d& pt) noexcept;
    
    template <class... Args>
    reference emplace_back(Args&&... args);
    void push_back(const value_type& x);
    void push_back(value_type&& x);
    void pop_back();
    template <class... Args>
    iterator emplace(const_iterator position, Args&&... args);
    iterator insert(const_iterator position, const value_type& x);
    iterator insert(const_iterator position, value_type&& x);
    iterator insert(const_iterator position, size_type n, const value_type& x);
    template <class InputIterator>
    iterator insert(const_iterator position, InputIterator first,
      InputIterator last);
    iterator insert(const_iterator position,
      initializer_list<value_type> il);
    iterator erase(const_iterator position);
    iterator erase(const_iterator first, const_iterator last);
    void swap(path_builder&)
      noexcept(allocator_traits<Allocator>::propagate_on_container_swap::value 
        || allocator_traits<Allocator>::is_always_equal::value);
    void clear() noexcept;

    // \ref{pathbuilder.observers}, observers:
    experimental::io2d::rectangle path_extents() const noexcept;
  };
  
  template <class Allocator>
  bool operator==(const path_builder<Allocator>& lhs, 
    const path_builder<Allocator>& rhs);
  template <class Allocator>
  bool operator!=(const path_builder<Allocator>& lhs, 
    const path_builder<Allocator>& rhs);
  
  // \ref{pathbuilder.special}, specialized algorithms:
  template <class Allocator>
  void swap(path_builder<Allocator>& lhs, path_builder<Allocator>& rhs)
    noexcept(noexcept(lhs.swap(rhs)));
} } } }
\end{codeblock}

\rSec1 [pathbuilder.containerrequirements] {\tcode{path_builder} container requirements}

\pnum
This class shall be considered a sequence container, as defined in [containers] in \cppseventeen, and all sequence container requirements that apply specifically to \tcode{vector} shall also apply to this class.

\rSec1 [pathbuilder.cons] {\tcode{path_builder} constructors, copy, and assignment}

\indexlibrary{\idxcode{path_builder}!constructor}
\begin{itemdecl}
explicit path_builder(const Allocator&);
\end{itemdecl}
\begin{itemdescr}
\pnum
\effects
Constructs an empty \tcode{path_builder}, using the specified allocator.

\pnum
\complexity
Constant.
\end{itemdescr}

\indexlibrary{\idxcode{path_builder}!constructor}
\begin{itemdecl}
explicit path_builder(size_type n, const Allocator& = Allocator());
\end{itemdecl}
\begin{itemdescr}
\pnum
\effects
Constructs a \tcode{path_builder} with \tcode{n} default-inserted elements using the specified allocator.

\pnum
\complexity
Linear in \tcode{n}.
\end{itemdescr}

\indexlibrary{\idxcode{path_builder}!constructor}
\begin{itemdecl}
path_builder(size_type n, const value_type& value,
  const Allocator& = Allocator());
\end{itemdecl}
\begin{itemdescr}
\pnum
\requires
\tcode{value_type} shall be \tcode{CopyInsertable} into \tcode{*this}.

\pnum
\effects
Constructs a \tcode{path_builder} with n copies of \tcode{value}, using the specified allocator.

\pnum
\complexity
Linear in \tcode{n}.
\end{itemdescr}

\indexlibrary{\idxcode{path_builder}!constructor}
\begin{itemdecl}
template <class InputIterator>
path_builder(InputIterator first, InputIterator last,
  const Allocator& = Allocator());
\end{itemdecl}
\begin{itemdescr}
\pnum
\effects
Constructs a \tcode{path_builder} equal to the range \range{first}{last}, using the specified allocator.

\pnum
\complexity
Makes only $N$ calls to the copy constructor of \tcode{value_type} (where $N$
is the distance between
\tcode{first}
and
\tcode{last})
and no reallocations if iterators \tcode{first} and \tcode{last} are of forward, bidirectional, or random access categories.
It makes order
\tcode{N}
calls to the copy constructor of
\tcode{value_type}
and order
$\log(N)$
reallocations if they are just input iterators.

\end{itemdescr}

\rSec1 [pathbuilder.capacity] {\tcode{path_builder} capacity}

\indexlibrary{\idxcode{path_builder}!\idxcode{capacity}}
\indexlibrary{\idxcode{capacity}!\idxcode{path_builder}}
\begin{itemdecl}
size_type capacity() const noexcept;
\end{itemdecl}
\begin{itemdescr}
\pnum
\returns
The total number of elements that the path builder can hold without requiring reallocation.
\end{itemdescr}

\indexlibrary{\idxcode{path_builder}!\idxcode{reserve}}
\begin{itemdecl}
void reserve(size_type n);
\end{itemdecl}
\begin{itemdescr}
\pnum
\requires
\tcode{value_type} shall be \tcode{MoveInsertable} into \tcode{*this}.

\pnum
\effects
A directive that informs a path builder of a planned change in size, so that it can manage the storage
allocation accordingly. After \tcode{reserve()}, \tcode{capacity()} is greater or equal to the argument of \tcode{reserve} if
reallocation happens; and equal to the previous value of \tcode{capacity()} otherwise. Reallocation happens
at this point if and only if the current capacity is less than the argument of \tcode{reserve()}. If an exception
is thrown other than by the move constructor of a non-\tcode{CopyInsertable} type, there are no effects.

\pnum
\complexity
It does not change the size of the sequence and takes at most linear time in the size of the
sequence.

\pnum
\throws
\tcode{length_error} if \tcode{n >
max_size()}.\footnote{\tcode{reserve()} uses \tcode{Allocator::allocate()} which
may throw an appropriate exception.}

\pnum
\remarks
Reallocation invalidates all the references, pointers, and iterators
referring to the elements in the sequence.
No reallocation shall take place during insertions that happen
after a call to
\tcode{reserve()}
until the time when an insertion would make the size of the vector
greater than the value of
\tcode{capacity()}.
\end{itemdescr}

\indexlibrary{\idxcode{path_builder}!\idxcode{shrink_to_fit}}
\begin{itemdecl}
void shrink_to_fit();
\end{itemdecl}
\begin{itemdescr}
\pnum
\requires
\tcode{value_type} shall be \tcode{MoveInsertable} into \tcode{*this}.

\pnum
\effects
\tcode{shrink_to_fit} is a non-binding request to reduce
\tcode{capacity()} to \tcode{size()}.
\enternote
The request is non-binding to allow latitude for
implementation-specific optimizations.
\exitnote
It does not increase \tcode{capacity()}, but may reduce \tcode{capacity()}
by causing reallocation. 
If an exception is thrown other than by the move constructor
of a non-\tcode{CopyInsertable} \tcode{value_type} there are no effects.

\pnum
\complexity Linear in the size of the sequence.

\pnum
\remarks Reallocation invalidates all the references, pointers, and 
iterators referring to the elements in the sequence. If no reallocation 
happens, they remain valid.
\end{itemdescr}

\indexlibrary{\idxcode{path_builder}!\idxcode{swap}}
\begin{itemdecl}
void swap(path_builder&)
  noexcept(allocator_traits<Allocator>::propagate_on_container_swap::value ||
  allocator_traits<Allocator>::is_always_equal::value);
\end{itemdecl}
\begin{itemdescr}
\pnum
\effects
Exchanges the contents and
\tcode{capacity()}
of
\tcode{*this}
with that of \tcode{x}.

\pnum
\complexity
Constant time.
\end{itemdescr}

\indexlibrary{\idxcode{path_builder}!\idxcode{resize}}
\begin{itemdecl}
void resize(size_type sz);
\end{itemdecl}
\begin{itemdescr}
\pnum
\effects
If \tcode{sz < size()}, erases the last \tcode{size() - sz} elements
from the sequence. Otherwise, appends \tcode{sz - size()} default-inserted 
elements to the sequence.

\pnum
\requires
\tcode{value_type} shall be
\tcode{MoveInsertable} and \tcode{DefaultInsertable} into \tcode{*this}.

\pnum
\remarks
If an exception is thrown other than by the move constructor of a 
non-\tcode{CopyInsertable}
\tcode{value_type} there are no effects.
\end{itemdescr}

\indexlibrary{\idxcode{path_builder}!\idxcode{resize}}
\begin{itemdecl}
void resize(size_type sz, const value_type& c);
\end{itemdecl}
\begin{itemdescr}
\pnum
\effects
If \tcode{sz < size()}, erases the last \tcode{size() - sz} elements
from the sequence. Otherwise,
appends \tcode{sz - size()} copies of \tcode{c} to the sequence.

\pnum
\requires
\tcode{value_type} shall be \tcode{CopyInsertable} into \tcode{*this}.

\pnum
\remarks
If an exception is thrown there are no effects.
\end{itemdescr}

\rSec1 [pathbuilder.modifiers] {\tcode{path_builder} modifiers}

\indexlibrary{\idxcode{path_builder}!\idxcode{new_path}}
\begin{itemdecl}
void new_path() noexcept;
\end{itemdecl}
\begin{itemdescr}
\pnum
\effects
Adds a \tcode{path_data::path_data_types} object constructed from \tcode{path_data::new_path()} to the end of the path group.
\end{itemdescr}

\indexlibrary{\idxcode{path_builder}!\idxcode{close_path}}
\begin{itemdecl}
void close_path() noexcept;
\end{itemdecl}
\begin{itemdescr}
\pnum
\requires
The current point contains a value.

\pnum
\effects
Adds a \tcode{path_data::path_data_types} object constructed from \tcode{path_data::close_path()} to the end of the path group.
\end{itemdescr}

\indexlibrary{\idxcode{path_builder}!\idxcode{arc_clockwise}}
\indexlibrary{\idxcode{arc_clockwise}!\idxcode{path_builder}}
\begin{itemdecl}
void arc_clockwise(const vector_2d& center, double radius, double angle1,
  double angle2) noexcept;
\end{itemdecl}
\begin{itemdescr}
\pnum
\effects
Adds a \tcode{path_data::path_data_types} object constructed from \\ \tcode{path_data::arc_clockwise(center, radius, angle1, angle2)} to the end of the path group.
\end{itemdescr}

\indexlibrary{\idxcode{path_builder}!\idxcode{arc_counterclockwise}}
\begin{itemdecl}
void arc_counterclockwise(const vector_2d& center, double radius,
  double angle1, double angle2) noexcept;
\end{itemdecl}
\begin{itemdescr}
\pnum
\effects
Adds a \tcode{path_data::path_data_types} object constructed from \\
\tcode{path_data::arc_counterclockwise(center, \\
radius, angle1, angle2)} 
to the end of the path group.
\end{itemdescr}

\indexlibrary{\idxcode{path_builder}!\idxcode{ellipse}}
\begin{itemdecl}
void ellipse(const experimental::io2d::ellipse& val) noexcept;
void ellipse(const circle& val) noexcept;
\end{itemdecl}
\begin{itemdescr}
\pnum
\effects
Adds a \tcode{path_data::path_data_types} object constructed from \tcode{path_data::abs_ellipse(val)} to the end of the path group.
\end{itemdescr}

\indexlibrary{\idxcode{path_builder}!\idxcode{cubic_curve_to}}
\begin{itemdecl}
void cubic_curve_to(const vector_2d& pt0, const vector_2d& pt1,
  const vector_2d& pt2) noexcept;
\end{itemdecl}
\begin{itemdescr}
\pnum
\effects
\pnum
Adds a \tcode{path_data::path_data_types} object constructed from \tcode{path_data::abs_cubic_curve(pt0, pt1, pt2)} to the end of the path group.
\end{itemdescr}

\indexlibrary{\idxcode{path_builder}!\idxcode{line_to}}
\begin{itemdecl}
void line_to(const vector_2d& pt) noexcept;
\end{itemdecl}
\begin{itemdescr}
\pnum
Adds a \tcode{path_data::path_data_types} object constructed from \tcode{path_data::abs_line(pt)} to the end of the path group.
\end{itemdescr}

\indexlibrary{\idxcode{path_builder}!\idxcode{move_to}}
\begin{itemdecl}
void move_to(const vector_2d& pt) noexcept;
\end{itemdecl}
\begin{itemdescr}
\pnum
\effects
Adds a \tcode{path_data::path_data_types} object constructed from \tcode{path_data::abs_move(pt)} to the end of the path group.
\end{itemdescr}

\indexlibrary{\idxcode{path_builder}!\idxcode{quadratic_curve_to}}
\begin{itemdecl}
void quadratic_curve_to(const vector_2d& pt0, const vector_2d& pt1) noexcept;
\end{itemdecl}
\begin{itemdescr}
\pnum
\effects
\pnum
Adds a \tcode{path_data::path_data_types} object constructed from \tcode{path_data::abs_quadratic_curve(pt0, pt1)} to the end of the path group.
\end{itemdescr}

\indexlibrary{\idxcode{path_builder}!\idxcode{rectangle}}
\begin{itemdecl}
void rectangle(const experimental::io2d::rectangle& r) noexcept;
\end{itemdecl}
\begin{itemdescr}
\pnum
\effects
Adds a \tcode{path_data::path_data_types} object constructed from \tcode{path_data::abs_rectangle(\{ r \})} to the end of the path group.
\end{itemdescr}

\indexlibrary{\idxcode{path_builder}!\idxcode{rel_rectangle}}
\begin{itemdecl}
void rel_rectangle(const experimental::io2d::rectangle& r) noexcept;
\end{itemdecl}
\begin{itemdescr}
\pnum
\effects
Adds a \tcode{path_data::path_data_types} object constructed from \tcode{path_data::rel_rectangle(\{ r \})} to the end of the path group.
\end{itemdescr}

\indexlibrary{\idxcode{path_builder}!\idxcode{rel_ellipse}}
\begin{itemdecl}
void rel_ellipse(const experimental::io2d::ellipse& val) noexcept;
void rel_ellipse(const circle& val) noexcept;
\end{itemdecl}
\begin{itemdescr}
\pnum
\effects
Adds a \tcode{path_data::path_data_types} object constructed from \tcode{path_data::rel_ellipse(\{ val \})} to the end of the path group.
\end{itemdescr}


\indexlibrary{\idxcode{path_builder}!\idxcode{rel_cubic_curve_to}}
\begin{itemdecl}
void rel_cubic_curve_to(const vector_2d& dpt0, const vector_2d& dpt1,
  const vector_2d& dpt2) noexcept;
\end{itemdecl}
\begin{itemdescr}
\pnum
\effects
Adds a \tcode{path_data::path_data_types} object constructed from \tcode{path_data::rel_cubic_curve(dpt0, dpt1, dpt2)} to the end of the path group.
\end{itemdescr}

\indexlibrary{\idxcode{path_builder}!\idxcode{rel_line_to}}
\begin{itemdecl}
void rel_line_to(const vector_2d& dpt) noexcept;
\end{itemdecl}
\begin{itemdescr}
\pnum
\effects
Adds a \tcode{path_data::path_data_types} object constructed from \tcode{path_data::rel_line(pt)} to the end of the path group.
\end{itemdescr}

\indexlibrary{\idxcode{path_builder}!\idxcode{rel_move_to}}
\begin{itemdecl}
void rel_move_to(const vector_2d& dpt) noexcept;
\end{itemdecl}
\begin{itemdescr}
\pnum
\effects
Adds a \tcode{path_data::path_data_types} object constructed from \tcode{path_data::rel_move(dpt)} to the end of the path group.
\end{itemdescr}

\indexlibrary{\idxcode{path_builder}!\idxcode{rel_quadratic_curve_to}}
\begin{itemdecl}
void rel_quadratic_curve_to(const vector_2d& dpt0, const vector_2d& dpt1)
  noexcept;
\end{itemdecl}
\begin{itemdescr}
\pnum
\effects
Adds a \tcode{path_data::path_data_types} object constructed from \tcode{path_data::rel_quadratic_curve(dpt0, dpt1)} to the end of the path group.
\end{itemdescr}

\indexlibrary{\idxcode{path_builder}!\idxcode{transform_matrix}}
\begin{itemdecl}
void transform_matrix(const matrix_2d& m) noexcept;
\end{itemdecl}
\begin{itemdescr}
\pnum
\requires
The matrix \tcode{m} shall be invertible.

\pnum
\effects
Adds a \tcode{path_data::path_data_types} object constructed from \tcode{(path_data::change_matrix(m)} to the end of the path group.
\end{itemdescr}

\indexlibrary{\idxcode{path_builder}!\idxcode{origin}}
\begin{itemdecl}
void origin(const vector_2d& pt) noexcept;
\end{itemdecl}
\begin{itemdescr}
\pnum
\effects
Adds a \tcode{path_data::path_data_types} object constructed from \tcode{path_data::change_origin(pt)} to the end of the path group.
\end{itemdescr}

\indexlibrary{\idxcode{path_builder}!\idxcode{insert}}
\indexlibrary{\idxcode{path_builder}!\idxcode{emplace_back}}
\indexlibrary{\idxcode{path_builder}!\idxcode{push_back}}
\begin{itemdecl}
iterator insert(const_iterator position, const value_type& x);
iterator insert(const_iterator position, value_type&& x);
iterator insert(const_iterator position, size_type n, const value_type& x);
template <class InputIterator>
iterator insert(const_iterator position, InputIterator first,
  InputIterator last);
iterator insert(const_iterator position, initializer_list<value_type>);
template <class... Args>
reference emplace_back(Args&&... args);
template <class... Args>
iterator emplace(const_iterator position, Args&&... args);
void push_back(const value_type& x);
void push_back(value_type&& x);
\end{itemdecl}

\begin{itemdescr}
\pnum
\remarks
Causes reallocation if the new size is greater than the old capacity.
Reallocation invalidates all the references, pointers, and iterators
referring to the elements in the sequence.
If no reallocation happens, all the iterators and references before the insertion point remain valid.
If an exception is thrown other than by
the copy constructor, move constructor,
assignment operator, or move assignment operator of
\tcode{value_type} or by any \tcode{InputIterator} operation
there are no effects.
If an exception is thrown while inserting a single element at the end and
\tcode{value_type} is \tcode{CopyInsertable} or \tcode{is_nothrow_move_constructible_v<value_type>}
is \tcode{true}, there are no effects.
Otherwise, if an exception is thrown by the move constructor of a non-\tcode{CopyInsertable}
\tcode{value_type}, the effects are unspecified.

\pnum
\complexity
The complexity is linear in the number of elements inserted plus the 
distance to the end of the path builder.
\end{itemdescr}

\indexlibrary{\idxcode{path_builder}!\idxcode{erase}}
\indexlibrary{\idxcode{erase}!\idxcode{path_builder}}
\indexlibrary{\idxcode{path_builder}!\idxcode{pop_back}}
\indexlibrary{\idxcode{pop_back}!\idxcode{path_builder}}
\begin{itemdecl}
iterator erase(const_iterator position);
iterator erase(const_iterator first, const_iterator last);
void pop_back();
\end{itemdecl}

\begin{itemdescr}
\pnum
\effects
Invalidates iterators and references at or after the point of the erase.

\pnum
\complexity
The destructor of \tcode{value_type} is called the number of times equal to 
the number of the elements erased, but the assignment operator
of \tcode{value_type} is called the number of times equal to the number of
elements in the path builder after the erased elements.

\pnum
\throws
Nothing unless an exception is thrown by the copy constructor, move 
constructor, assignment operator, or move assignment operator of
\tcode{value_type}.
\end{itemdescr}

\indexlibrary{\idxcode{path_builder}!\idxcode{clear}}
\indexlibrary{\idxcode{clear}!\idxcode{path_builder}}
\begin{itemdecl}
void clear() noexcept;
\end{itemdecl}
\begin{itemdescr}
\pnum
\postconditions
\pnum
The current point shall not contain a value.

\pnum
The transformation matrix shall have a value of \tcode{matrix_2d::init_identity\{ \}}.

\pnum
The origin shall have a value of \tcode{vector_2d\{ \}}.

\pnum
\remarks
The postconditions listed above are in addition to sequence container requirements for this function.
\end{itemdescr}

\rSec1 [pathbuilder.iterators] {\tcode{path_builder} iterators}

\indexlibrary{\idxcode{path_builder}!\idxcode{begin}}
\indexlibrary{\idxcode{begin}!\idxcode{path_builder}}
\indexlibrary{\idxcode{path_builder}!\idxcode{cbegin}}
\indexlibrary{\idxcode{cbegin}!\idxcode{path_builder}}
\begin{itemdecl}
iterator begin() noexcept;
const_iterator begin() const noexcept;
const_iterator cbegin() const noexcept;
\end{itemdecl}
\begin{itemdescr}
\pnum
\returns
An iterator referring to the first \tcode{path_data::path_data_types} item in the path group.

\pnum
\remarks
Changing a \tcode{path_data::path_data_types} object or otherwise modifying the path group in a way that violates the preconditions of that \tcode{path_data::path_data_types} object or of any subsequent \tcode{path_data::path_data_types} object in the path group shall result in undefined behavior when the path group is processed as described in \ref{paths.processing} unless all of the violations are fixed prior to such processing.
\end{itemdescr}

\indexlibrary{\idxcode{path_builder}!\idxcode{end}}
\indexlibrary{\idxcode{end}!\idxcode{path_builder}}
\indexlibrary{\idxcode{path_builder}!\idxcode{cend}}
\indexlibrary{\idxcode{cend}!\idxcode{path_builder}}
\begin{itemdecl}
iterator end() noexcept;
const_iterator end() const noexcept;
const_iterator cend() const noexcept;
\end{itemdecl}
\begin{itemdescr}
\pnum
\returns
An iterator which is the past-the-end value.

\pnum
\remarks
Changing a \tcode{path_data::path_data_types} object or otherwise modifying the path group in a way that violates the preconditions of that \tcode{path_data::path_data_types} object or of any subsequent \tcode{path_data::path_data_types} object in the path group shall result in undefined behavior when the path group is processed as described in \ref{paths.processing} unless all of the violations are fixed prior to such processing.
\end{itemdescr}

\indexlibrary{\idxcode{path_builder}!\idxcode{rbegin}}
\indexlibrary{\idxcode{rbegin}!\idxcode{path_builder}}
\indexlibrary{\idxcode{path_builder}!\idxcode{crbegin}}
\indexlibrary{\idxcode{crbegin}!\idxcode{path_builder}}
\begin{itemdecl}
reverse_iterator rbegin() noexcept;
const_reverse_iterator rbegin() const noexcept;
const_reverse_iterator crbegin() const noexcept;
\end{itemdecl}
\begin{itemdescr}
\pnum
\returns
An iterator which is semantically equivalent to \tcode{reverse_iterator(end)}.

\pnum
\remarks
Changing a \tcode{path_data::path_data_types} object or otherwise modifying the path group in a way that violates the preconditions of that \tcode{path_data::path_data_types} object or of any subsequent \tcode{path_data::path_data_types} object in the path group shall result in undefined behavior when the path group is processed as described in \ref{paths.processing} all of the violations are fixed prior to such processing.
\end{itemdescr}

\indexlibrary{\idxcode{path_builder}!\idxcode{rend}}
\indexlibrary{\idxcode{rend}!\idxcode{path_builder}}
\indexlibrary{\idxcode{path_builder}!\idxcode{crend}}
\indexlibrary{\idxcode{crend}!\idxcode{path_builder}}
\begin{itemdecl}
reverse_iterator rend() noexcept;
const_reverse_iterator rend() const noexcept;
const_reverse_iterator crend() const noexcept;
\end{itemdecl}
\begin{itemdescr}
\pnum
\returns
An iterator which is semantically equivalent to \tcode{reverse_iterator(begin)}.

\pnum
\remarks
Changing a \tcode{path_data::path_data_types} object or otherwise modifying the path group in a way that violates the preconditions of that \tcode{path_data::path_data_types} object or of any subsequent \tcode{path_data::path_data_types} object in the path group shall result in undefined behavior when the path group is processed as described in \ref{paths.processing} unless all of the violations are fixed prior to such processing.
\end{itemdescr}

\rSec1 [pathbuilder.observers] {\tcode{path_builder} observers}

\indexlibrary{\idxcode{path_builder}!\idxcode{path_extents}}
\indexlibrary{\idxcode{path_extents}!\idxcode{path_builder}}
\begin{itemdecl}
experimental::io2d::rectangle path_extents() const noexcept;
\end{itemdecl}
\begin{itemdescr}
\pnum
\returns
A \tcode{rectangle} object which contains the extents of the path segments, including degenerate path segments, in the path group when it is processed as described in \ref{paths.processing}.
\enternote
By using path segments, this description intentionally excludes points established by \tcode{move_to} and \tcode{rel_move_to} operations from the extents value except where those points are subsequently used in defining a path segment.
\exitnote
\end{itemdescr}

\indexlibrary{\idxcode{path_builder}!\idxcode{has_current_point}}
\indexlibrary{\idxcode{has_current_point}!\idxcode{path_builder}}
\begin{itemdecl}
bool has_current_point() const noexcept;
\end{itemdecl}
\begin{itemdescr}
\pnum
\returns
If the current point contains a value, \tcode{true}, otherwise \tcode{false}.
\end{itemdescr}

\indexlibrary{\idxcode{path_builder}!\idxcode{current_point}}
\indexlibrary{\idxcode{current_point}!\idxcode{path_builder}}
\begin{itemdecl}
vector_2d current_point() const noexcept;
\end{itemdecl}
\begin{itemdescr}
\pnum
\requires
The current point contains a value.

\pnum
\returns
The value of the current point.
\end{itemdescr}

\indexlibrary{\idxcode{path_builder}!\idxcode{transform_matrix}}
\indexlibrary{\idxcode{transform_matrix}!\idxcode{path_builder}}
\begin{itemdecl}
matrix_2d transform_matrix() const noexcept;
\end{itemdecl}
\begin{itemdescr}
\pnum
\returns
The value of the transformation matrix.
\end{itemdescr}

\indexlibrary{\idxcode{path_builder}!\idxcode{origin}}
\indexlibrary{\idxcode{origin}!\idxcode{path_builder}}
\begin{itemdecl}
vector_2d origin() const noexcept;
\end{itemdecl}
\begin{itemdescr}
\pnum
\returns
The value of the origin.
\end{itemdescr}

\rSec1[pathbuilder.special] {\tcode{path_builder} specialized algorithms}

\indexlibrary{path_builder}{swap}
\indexlibrary{swap}{path_builder}
\begin{itemdecl}
template <class Allocator>
void swap(path_builder<Allocator>& lhs, path_builder<Allocator>& rhs)
  noexcept(noexcept(lhs.swap(rhs)));
\end{itemdecl}
\begin{itemdescr}
\pnum
\effects
As if by \tcode{lhs.swap(rhs)}.
\end{itemdescr}

