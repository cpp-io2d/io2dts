%!TEX root = io2d.tex
\rSec0 [\iotwod.pathbuilder] {Class \tcode{basic_path_builder}}

\pnum
\indexlibrary{\idxcode{basic_path_builder}}%
The class \tcode{basic_path_builder} is a container that stores and manipulates objects of type \tcode{figure_items::figure_item} from which \tcode{interpreted_path} objects are created.

\pnum
A \tcode{basic_path_builder} is a contiguous container. (See [container.requirements.general] in \CppXVII.)

\pnum
The collection of \tcode{figure_items::figure_item} objects in a path builder is referred to as its path.

\rSec1 [\iotwod.pathbuilder.synopsis] {\tcode{basic_path_builder} synopsis}%

\begin{codeblock}
namespace std::experimental::io2d::v1 {
  template <class GraphicsSurfaces,
            class Allocator = ::std::allocator<typename
              basic_figure_items<GraphicsSurfaces>::figure_item>>
  class basic_path_builder {
  public:
    using value_type             = typename basic_figure_items<GraphicsSurfaces>::figure_item;
    using allocator_type         = Allocator;
    using reference              = value_type&;
    using const_reference        = const value_type&;
    using size_type              = @\impdefx{type of \tcode{basic_path_builder::size_type}}@. // See [container.requirements] in \CppXVII.
    using difference_type        = @\impdefx{type of \tcode{basic_path_builder::size_type}}@. // See [container.requirements] in \CppXVII.
    using iterator               = @\impdefx{type of \tcode{basic_path_builder::iterator}}@. // See [container.requirements] in \CppXVII.
    using const_iterator         = @\impdefx{type of \tcode{basic_path_builder::const_iterator}}@. // See [container.requirements] in \CppXVII.
    using reverse_iterator       = std::reverse_iterator<iterator>;
    using const_reverse_iterator = std::reverse_iterator<const_iterator>;

    // \ref{\iotwod.pathbuilder.cons}, construct, copy, move, destroy:
    basic_path_builder() noexcept(noexcept(Allocator()));
    explicit basic_path_builder(const Allocator&) noexcept;
    explicit basic_path_builder(size_type n, const Allocator& = Allocator());
    basic_path_builder(size_type n, const value_type& value, const Allocator& = Allocator());
    template <class InputIterator>
    basic_path_builder(InputIterator first, InputIterator last, const Allocator& = Allocator());
    basic_path_builder(const basic_path_builder& x);
    basic_path_builder(basic_path_builder&&) noexcept;
    basic_path_builder(const basic_path_builder&, const Allocator&);
    basic_path_builder(basic_path_builder&&, const Allocator&);
    basic_path_builder(initializer_list<value_type>, const Allocator& = Allocator());
    ~basic_path_builder();
    basic_path_builder& operator=(const basic_path_builder& x);
    basic_path_builder& operator=(basic_path_builder&& x) noexcept(
      allocator_traits<Allocator>::propagate_on_container_move_assignment::value ||
      allocator_traits<Allocator>::is_always_equal::value);
    basic_path_builder& operator=(initializer_list<value_type>);
    template <class InputIterator>
    void assign(InputIterator first, InputIterator last);
    void assign(size_type n, const value_type& u);
    void assign(initializer_list<value_type>);
    allocator_type get_allocator() const noexcept;

    // \ref{\iotwod.pathbuilder.capacity}, capacity
    bool empty() const noexcept;
    size_type size() const noexcept;
    size_type max_size() const noexcept;
    size_type capacity() const noexcept;
    void resize(size_type sz);
    void resize(size_type sz, const value_type& c);
    void reserve(size_type n);
    void shrink_to_fit();

    // element access:
    reference operator[](size_type n);
    const_reference operator[](size_type n) const;
    const_reference at(size_type n) const;
    reference at(size_type n);
    reference front();
    const_reference front() const;
    reference back();
    const_reference back() const;

    // \ref{\iotwod.pathbuilder.modifiers}, modifiers:
    void new_figure(const basic_point_2d<typename
      GraphicsSurfaces::graphics_math_type>& pt) noexcept;
    void rel_new_figure(const basic_point_2d<typename
      GraphicsSurfaces::graphics_math_type>& pt) noexcept;
    void close_figure() noexcept;
    void matrix(const basic_matrix_2d<typename
      GraphicsSurfaces::graphics_math_type>& m) noexcept;
    void rel_matrix(const basic_matrix_2d<typename
      GraphicsSurfaces::graphics_math_type>& m) noexcept;
    void revert_matrix() noexcept;
    void line(const basic_point_2d<typename GraphicsSurfaces::graphics_math_type>& pt) noexcept;
    void rel_line(const basic_point_2d<typename
      GraphicsSurfaces::graphics_math_type>& dpt) noexcept;
    void quadratic_curve(const basic_point_2d<typename
      GraphicsSurfaces::graphics_math_type>& pt0, const basic_point_2d<typename
      GraphicsSurfaces::graphics_math_type>& pt2) noexcept;
    void rel_quadratic_curve(const basic_point_2d<typename
      GraphicsSurfaces::graphics_math_type>& pt0, const basic_point_2d<typename
      GraphicsSurfaces::graphics_math_type>& pt2) noexcept;
    void cubic_curve(const basic_point_2d<typename
      GraphicsSurfaces::graphics_math_type>& pt0, const basic_point_2d<typename
      GraphicsSurfaces::graphics_math_type>& pt1, const basic_point_2d<typename
      GraphicsSurfaces::graphics_math_type>& pt2) noexcept;
    void rel_cubic_curve(const basic_point_2d<typename
      GraphicsSurfaces::graphics_math_type>& dpt0, const basic_point_2d<typename
      GraphicsSurfaces::graphics_math_type>& dpt1, const basic_point_2d<typename
      GraphicsSurfaces::graphics_math_type>& dpt2) noexcept;
    void arc(const basic_point_2d<typename
      GraphicsSurfaces::graphics_math_type>& rad, float rot, float sang = pi<float>) noexcept;
    template <class... Args>
    reference emplace_back(Args&&... args);
    void push_back(const value_type& x);
    void push_back(value_type&& x);
    void pop_back();
    template <class... Args>
    iterator emplace(const_iterator position, Args&&... args);
    iterator insert(const_iterator position, const value_type& x);
    iterator insert(const_iterator position, value_type&& x);
    iterator insert(const_iterator position, size_type n, const value_type& x);
    template <class InputIterator>
    iterator insert(const_iterator position, InputIterator first, InputIterator last);
    iterator insert(const_iterator position,
    initializer_list<value_type> il);
    iterator erase(const_iterator position);
    iterator erase(const_iterator first, const_iterator last);
    void swap(basic_path_builder&) noexcept(
      allocator_traits<Allocator>::propagate_on_container_swap::value ||
      allocator_traits<Allocator>::is_always_equal::value);
    void clear() noexcept;

    // \ref{\iotwod.pathbuilder.iterators}, iterators:
    iterator begin() noexcept;
    const_iterator begin() const noexcept;
    const_iterator cbegin() const noexcept;
    iterator end() noexcept;
    const_iterator end() const noexcept;
    const_iterator cend() const noexcept;
    reverse_iterator rbegin() noexcept;
    const_reverse_iterator rbegin() const noexcept;
    const_reverse_iterator crbegin() const noexcept;
    reverse_iterator rend() noexcept;
    const_reverse_iterator rend() const noexcept;
    const_reverse_iterator crend() const noexcept;
  };

  template <class GraphicsSurfaces, class Allocator>
  bool operator==(const basic_path_builder<GraphicsSurfaces, Allocator>& lhs,
    const basic_path_builder<GraphicsSurfaces, Allocator>& rhs) noexcept;
  template <class GraphicsSurfaces, class Allocator>
  bool operator!=(const basic_path_builder<GraphicsSurfaces, Allocator>& lhs,
    const basic_path_builder<GraphicsSurfaces, Allocator>& rhs) noexcept;
  template <class GraphicsSurfaces, class Allocator>
  void swap(basic_path_builder<GraphicsSurfaces, Allocator>& lhs,
    basic_path_builder<GraphicsSurfaces, Allocator>& rhs) noexcept(noexcept(lhs.swap(rhs)));
}
\end{codeblock}

\rSec1 [\iotwod.pathbuilder.containerrequirements] {\tcode{basic_path_builder} container requirements}

\pnum
This class is a sequence container, as defined in [containers] in \CppXVII, and all sequence container requirements that apply specifically to \tcode{vector} shall also apply to this class.

\rSec1 [\iotwod.pathbuilder.cons] {\tcode{basic_path_builder} constructors, copy, and assignment}

\indexlibrary{\idxcode{basic_path_builder}!constructor}%
\begin{itemdecl}
basic_path_builder() noexcept(noexcept(Allocator()));
\end{itemdecl}
\begin{itemdescr}
\pnum
\effects
Constructs an empty \tcode{basic_path_builder}.
\end{itemdescr}
	
\indexlibrary{\idxcode{basic_path_builder}!constructor}%
\begin{itemdecl}
explicit basic_path_builder(const Allocator&);
\end{itemdecl}
\begin{itemdescr}
\pnum
\effects
Constructs an empty \tcode{basic_path_builder}, using the specified allocator.

\pnum
\complexity
Constant.
\end{itemdescr}

\indexlibrary{\idxcode{basic_path_builder}!constructor}%
\begin{itemdecl}
explicit basic_path_builder(size_type n, const Allocator& = Allocator());
\end{itemdecl}
\begin{itemdescr}
\pnum
\effects
Constructs a \tcode{basic_path_builder} with \tcode{n} default-inserted elements using the specified allocator.

\pnum
\complexity
Linear in \tcode{n}.
\end{itemdescr}

\indexlibrary{\idxcode{basic_path_builder}!constructor}%
\begin{itemdecl}
basic_path_builder(size_type n, const value_type& value,
  const Allocator& = Allocator());
\end{itemdecl}
\begin{itemdescr}
\pnum
\requires
\tcode{value_type} shall be \tcode{CopyInsertable} into \tcode{*this}.

\pnum
\effects
Constructs a \tcode{basic_path_builder} with n copies of \tcode{value}, using the specified allocator.

\pnum
\complexity
Linear in \tcode{n}.
\end{itemdescr}

\indexlibrary{\idxcode{basic_path_builder}!constructor}%
\begin{itemdecl}
template <class InputIterator>
basic_path_builder(InputIterator first, InputIterator last,
  const Allocator& = Allocator());
\end{itemdecl}
\begin{itemdescr}
\pnum
\effects
Constructs a \tcode{basic_path_builder} equal to the range \range{first}{last}, using the specified allocator.

\pnum
\complexity
Makes only $N$ calls to the copy constructor of \tcode{value_type} (where $N$
is the distance between
\tcode{first}
and
\tcode{last})
and no reallocations if iterators \tcode{first} and \tcode{last} are of forward, bidirectional, or random access categories.
It makes order
\tcode{N}
calls to the copy constructor of
\tcode{value_type}
and order
$\log(N)$
reallocations if they are just input iterators.

\end{itemdescr}

\rSec1 [\iotwod.pathbuilder.capacity] {\tcode{basic_path_builder} capacity}%

\indexlibrarymember{capacity}{basic_path_builder}%
\begin{itemdecl}
size_type capacity() const noexcept;
\end{itemdecl}
\begin{itemdescr}
\pnum
\returns
The total number of elements that the path builder can hold without requiring reallocation.
\end{itemdescr}

\indexlibrarymember{basic_path_builder}{reserve}%
\begin{itemdecl}
void reserve(size_type n);
\end{itemdecl}
\begin{itemdescr}
\pnum
\requires
\tcode{value_type} shall be \tcode{MoveInsertable} into \tcode{*this}.

\pnum
\effects
A directive that informs a path builder of a planned change in size, so that it can manage the storage
allocation accordingly. After \tcode{reserve()}, \tcode{capacity()} is greater or equal to the argument of \tcode{reserve} if
reallocation happens; and equal to the previous value of \tcode{capacity()} otherwise. Reallocation happens
at this point if and only if the current capacity is less than the argument of \tcode{reserve()}. If an exception
is thrown other than by the move constructor of a non-\tcode{CopyInsertable} type, there are no effects.

\pnum
\complexity
It does not change the size of the sequence and takes at most linear time in the size of the
sequence.

\pnum
\throws
\tcode{length_error} if \tcode{n >
max_size()}.\footnote{\tcode{reserve()} uses \tcode{Allocator::allocate()} which
may throw an appropriate exception.}

\pnum
\remarks
Reallocation invalidates all the references, pointers, and iterators
referring to the elements in the sequence.
No reallocation shall take place during insertions that happen
after a call to
\tcode{reserve()}
until the time when an insertion would make the size of the vector
greater than the value of
\tcode{capacity()}.
\end{itemdescr}

\indexlibrarymember{basic_path_builder}{shrink_to_fit}%
\begin{itemdecl}
void shrink_to_fit();
\end{itemdecl}
\begin{itemdescr}
\pnum
\requires
\tcode{value_type} shall be \tcode{MoveInsertable} into \tcode{*this}.

\pnum
\effects
\tcode{shrink_to_fit} is a non-binding request to reduce
\tcode{capacity()} to \tcode{size()}.
\begin{note}
The request is non-binding to allow latitude for
implementation-specific optimizations.
\end{note}
It does not increase \tcode{capacity()}, but may reduce \tcode{capacity()}
by causing reallocation. 
If an exception is thrown other than by the move constructor
of a non-\tcode{CopyInsertable} \tcode{value_type} there are no effects.

\pnum
\complexity Linear in the size of the sequence.

\pnum
\remarks Reallocation invalidates all the references, pointers, and 
iterators referring to the elements in the sequence. If no reallocation 
happens, they remain valid.
\end{itemdescr}

\indexlibrarymember{basic_path_builder}{swap}%
\begin{itemdecl}
void swap(basic_path_builder&)
  noexcept(allocator_traits<Allocator>::propagate_on_container_swap::value ||
  allocator_traits<Allocator>::is_always_equal::value);
\end{itemdecl}
\begin{itemdescr}
\pnum
\effects
Exchanges the contents and
\tcode{capacity()}
of
\tcode{*this}
with that of \tcode{x}.

\pnum
\complexity
Constant time.
\end{itemdescr}

\indexlibrary{basic_path_builder}{resize}%
\begin{itemdecl}
void resize(size_type sz);
\end{itemdecl}
\begin{itemdescr}
\pnum
\effects
If \tcode{sz < size()}, erases the last \tcode{size() - sz} elements
from the sequence. Otherwise, appends \tcode{sz - size()} default-inserted 
elements to the sequence.

\pnum
\requires
\tcode{value_type} shall be
\tcode{MoveInsertable} and \tcode{DefaultInsertable} into \tcode{*this}.

\pnum
\remarks
If an exception is thrown other than by the move constructor of a 
non-\tcode{CopyInsertable}
\tcode{value_type} there are no effects.
\end{itemdescr}

\indexlibrary{basic_path_builder}{resize}%
\begin{itemdecl}
void resize(size_type sz, const value_type& c);
\end{itemdecl}
\begin{itemdescr}
\pnum
\effects
If \tcode{sz < size()}, erases the last \tcode{size() - sz} elements
from the sequence. Otherwise,
appends \tcode{sz - size()} copies of \tcode{c} to the sequence.

\pnum
\requires
\tcode{value_type} shall be \tcode{CopyInsertable} into \tcode{*this}.

\pnum
\remarks
If an exception is thrown there are no effects.
\end{itemdescr}

\rSec1 [\iotwod.pathbuilder.modifiers] {\tcode{basic_path_builder} modifiers}

\indexlibrarymember{basic_path_builder}{new_figure}%
\begin{itemdecl}
void new_figure(point_2d pt) noexcept;
\end{itemdecl}
\begin{itemdescr}
\pnum
\effects
Adds a \tcode{figure_items::figure_item} object constructed from \tcode{figure_items::abs_new_figure(pt)} to the end of the path.
\end{itemdescr}

\indexlibrarymember{basic_path_builder}{rel_new_figure}%
\begin{itemdecl}
void rel_new_figure(point_2d pt) noexcept;
\end{itemdecl}
\begin{itemdescr}
\pnum
\effects
Adds a \tcode{figure_items::figure_item} object constructed from \tcode{figure_items::rel_new_figure(pt)} to the end of the path.
\end{itemdescr}

\indexlibrarymember{basic_path_builder}{close_figure}%
\begin{itemdecl}
void close_figure() noexcept;
\end{itemdecl}
\begin{itemdescr}
\pnum
\requires
The current point contains a value.

\pnum
\effects
Adds a \tcode{figure_items::figure_item} object constructed from \tcode{figure_items::close_figure()} to the end of the path.
\end{itemdescr}

\indexlibrarymember{basic_path_builder}{set_matrix}%
\begin{itemdecl}
void matrix(const matrix_2d& m) noexcept;
\end{itemdecl}
\begin{itemdescr}
\pnum
\requires
The matrix \tcode{m} shall be invertible.

\pnum
\effects
Adds a \tcode{figure_items::figure_item} object constructed from \tcode{(figure_items::abs_matrix(m)} to the end of the path.
\end{itemdescr}

\indexlibrarymember{basic_path_builder}{modify_matrix}%
\begin{itemdecl}
void rel_matrix(const matrix_2d& m) noexcept;
\end{itemdecl}
\begin{itemdescr}
\pnum
\requires
The matrix \tcode{m} shall be invertible.

\pnum
\effects
Adds a \tcode{figure_items::figure_item} object constructed from \tcode{(figure_items::rel_matrix(m)} to the end of the path.
\end{itemdescr}

\indexlibrarymember{basic_path_builder}{revert_matrix}%
\begin{itemdecl}
void revert_matrix() noexcept;
\end{itemdecl}
\begin{itemdescr}
\pnum
\effects
Adds a \tcode{figure_items::figure_item} object constructed from \tcode{(figure_items::revert_matrix()} to the end of the path.
\end{itemdescr}

\indexlibrarymember{basic_path_builder}{line}%
\begin{itemdecl}
void line(point_2d pt) noexcept;
\end{itemdecl}
\begin{itemdescr}
\pnum
Adds a \tcode{figure_items::figure_item} object constructed from \tcode{figure_items::abs_line(pt)} to the end of the path.
\end{itemdescr}

\indexlibrarymember{basic_path_builder}{rel_line}%
\begin{itemdecl}
void rel_line(point_2d dpt) noexcept;
\end{itemdecl}
\begin{itemdescr}
\pnum
\effects
Adds a \tcode{figure_items::figure_item} object constructed from \tcode{figure_items::rel_line(pt)} to the end of the path.
\end{itemdescr}

\indexlibrarymember{basic_path_builder}{quadratic_curve}%
\begin{itemdecl}
void quadratic_curve(point_2d pt0, point_2d pt1) noexcept;
\end{itemdecl}
\begin{itemdescr}
\pnum
\effects
Adds a \tcode{figure_items::figure_item} object constructed from\\ \tcode{figure_items::abs_quadratic_curve(pt0, pt1)} to the end of the path.
\end{itemdescr}

\indexlibrarymember{basic_path_builder}{rel_quadratic_curve}%
\begin{itemdecl}
void rel_quadratic_curve(point_2d dpt0, point_2d dpt1)
  noexcept;
\end{itemdecl}
\begin{itemdescr}
\pnum
\effects
Adds a \tcode{figure_items::figure_item} object constructed from\\ \tcode{figure_items::rel_quadratic_curve(dpt0, dpt1)} to the end of the path.
\end{itemdescr}

\indexlibrarymember{basic_path_builder}{cubic_curve}%
\begin{itemdecl}
void cubic_curve(point_2d pt0, point_2d pt1,
  point_2d pt2) noexcept;
\end{itemdecl}
\begin{itemdescr}
\pnum
\effects
\pnum
Adds a \tcode{figure_items::figure_item} object constructed from \tcode{figure_items::abs_cubic_curve(pt0, pt1, pt2)} to the end of the path.
\end{itemdescr}

\indexlibrarymember{basic_path_builder}{rel_cubic_curve}%
\begin{itemdecl}
void rel_cubic_curve(point_2d dpt0, point_2d dpt1,
  point_2d dpt2) noexcept;
\end{itemdecl}
\begin{itemdescr}
\pnum
\effects
Adds a \tcode{figure_items::figure_item} object constructed from \tcode{figure_items::rel_cubic_curve(dpt0, dpt1, dpt2)} to the end of the path.
\end{itemdescr}

\indexlibrarymember{basic_path_builder}{arc}%
\begin{itemdecl}
void arc(point_2d rad, float rot, float sang) noexcept;
\end{itemdecl}
\begin{itemdescr}
\pnum
\effects
Adds a \tcode{figure_items::figure_item} object constructed from \\ \tcode{figure_items::arc(rad, rot, sang)} to the end of the path.
\end{itemdescr}

\indexlibrarymember{basic_path_builder}{insert}%
\indexlibrarymember{basic_path_builder}{emplace_back}%
\indexlibrarymember{basic_path_builder}{push_back}%
\begin{itemdecl}
iterator insert(const_iterator position, const value_type& x);
iterator insert(const_iterator position, value_type&& x);
iterator insert(const_iterator position, size_type n, const value_type& x);
template <class InputIterator>
iterator insert(const_iterator position, InputIterator first,
  InputIterator last);
iterator insert(const_iterator position, initializer_list<value_type>);
template <class... Args>
reference emplace_back(Args&&... args);
template <class... Args>
iterator emplace(const_iterator position, Args&&... args);
void push_back(const value_type& x);
void push_back(value_type&& x);
\end{itemdecl}

\begin{itemdescr}
\pnum
\remarks
Causes reallocation if the new size is greater than the old capacity.
Reallocation invalidates all the references, pointers, and iterators
referring to the elements in the sequence.
If no reallocation happens, all the iterators and references before the insertion point remain valid.
If an exception is thrown other than by
the copy constructor, move constructor,
assignment operator, or move assignment operator of
\tcode{value_type} or by any \tcode{InputIterator} operation
there are no effects.
If an exception is thrown while inserting a single element at the end and
\tcode{value_type} is \tcode{CopyInsertable} or \tcode{is_nothrow_move_constructible_v<value_type>}
is \tcode{true}, there are no effects.
Otherwise, if an exception is thrown by the move constructor of a non-\tcode{CopyInsertable}
\tcode{value_type}, the effects are unspecified.

\pnum
\complexity
The complexity is linear in the number of elements inserted plus the 
distance to the end of the path builder.
\end{itemdescr}

\indexlibrarymember{erase}{basic_path_builder}%
\indexlibrarymember{pop_back}{basic_path_builder}%
\begin{itemdecl}
iterator erase(const_iterator position);
iterator erase(const_iterator first, const_iterator last);
void pop_back();
\end{itemdecl}

\begin{itemdescr}
\pnum
\effects
Invalidates iterators and references at or after the point of the erase.

\pnum
\complexity
The destructor of \tcode{value_type} is called the number of times equal to 
the number of the elements erased, but the assignment operator
of \tcode{value_type} is called the number of times equal to the number of
elements in the path builder after the erased elements.

\pnum
\throws
Nothing unless an exception is thrown by the copy constructor, move 
constructor, assignment operator, or move assignment operator of
\tcode{value_type}.
\end{itemdescr}

\rSec1 [\iotwod.pathbuilder.iterators] {\tcode{basic_path_builder} iterators}

\indexlibrarymember{begin}{basic_path_builder}%
\indexlibrarymember{cbegin}{basic_path_builder}%
\begin{itemdecl}
iterator begin() noexcept;
const_iterator begin() const noexcept;
const_iterator cbegin() const noexcept;
\end{itemdecl}
\begin{itemdescr}
\pnum
\returns
An iterator referring to the first \tcode{figure_items::figure_item} item in the path.

\pnum
\remarks
Changing a \tcode{figure_items::figure_item} object or otherwise modifying the path in a way that violates the preconditions of that \tcode{figure_items::figure_item} object or of any subsequent \tcode{figure_items::figure_item} object in the path produces undefined behavior when the path is interpreted as described in \ref{\iotwod.paths.interpretation} unless all of the violations are fixed prior to such interpretation.
\end{itemdescr}

\indexlibrarymember{end}{basic_path_builder}%
\indexlibrarymember{cend}{basic_path_builder}%
\begin{itemdecl}
iterator end() noexcept;
const_iterator end() const noexcept;
const_iterator cend() const noexcept;
\end{itemdecl}
\begin{itemdescr}
\pnum
\returns
An iterator which is the past-the-end value.

\pnum
\remarks
Changing a \tcode{figure_items::figure_item} object or otherwise modifying the path in a way that violates the preconditions of that \tcode{figure_items::figure_item} object or of any subsequent \tcode{figure_items::figure_item} object in the path produces undefined behavior when the path is interpreted as described in \ref{\iotwod.paths.interpretation} unless all of the violations are fixed prior to such interpretation.
\end{itemdescr}

\indexlibrarymember{rbegin}{basic_path_builder}%
\indexlibrarymember{crbegin}{basic_path_builder}%
\begin{itemdecl}
reverse_iterator rbegin() noexcept;
const_reverse_iterator rbegin() const noexcept;
const_reverse_iterator crbegin() const noexcept;
\end{itemdecl}
\begin{itemdescr}
\pnum
\returns
An iterator which is semantically equivalent to \tcode{reverse_iterator(end)}.

\pnum
\remarks
Changing a \tcode{figure_items::figure_item} object or otherwise modifying the path in a way that violates the preconditions of that \tcode{figure_items::figure_item} object or of any subsequent \tcode{figure_items::figure_item} object in the path produces undefined behavior when the path is interpreted as described in \ref{\iotwod.paths.interpretation} all of the violations are fixed prior to such interpretation.
\end{itemdescr}

\indexlibrarymember{rend}{basic_path_builder}%
\indexlibrarymember{crend}{basic_path_builder}%
\begin{itemdecl}
reverse_iterator rend() noexcept;
const_reverse_iterator rend() const noexcept;
const_reverse_iterator crend() const noexcept;
\end{itemdecl}
\begin{itemdescr}
\pnum
\returns
An iterator which is semantically equivalent to \tcode{reverse_iterator(begin)}.

\pnum
\remarks
Changing a \tcode{figure_items::figure_item} object or otherwise modifying the path in a way that violates the preconditions of that \tcode{figure_items::figure_item} object or of any subsequent \tcode{figure_items::figure_item} object in the path produces undefined behavior when the path is interpreted as described in \ref{\iotwod.paths.interpretation} unless all of the violations are fixed prior to such interpretation.
\end{itemdescr}

\rSec1[\iotwod.pathbuilder.special] {\tcode{basic_path_builder} specialized algorithms}

\indexlibrary{\idxcode{swap}!\idxcode{basic_path_builder}}%
\begin{itemdecl}
template <class Allocator>
void swap(basic_path_builder<Allocator>& lhs, basic_path_builder<Allocator>& rhs)
  noexcept(noexcept(lhs.swap(rhs)));
\end{itemdecl}
\begin{itemdescr}
\pnum
\effects
As if by \tcode{lhs.swap(rhs)}.
\end{itemdescr}
