%!TEX root = io2d.tex
\rSec0 [\iotwod.io2derror] {Enum class \tcode{io2d_error}}

\rSec1 [\iotwod.io2derror.summary] {\tcode{io2d_error} Summary}
\pnum
The \tcode{io2d_error} enum class is used as an \tcode{error_condition} 
associated with \tcode{io2d_error_category}. See 
Table~\ref{tab:\iotwod.io2derror.meanings} for the meaning of each
status code.

\rSec1 [\iotwod.io2derror.synopsis] {\tcode{io2d_error} Synopsis}

\indexlibrary{\idxcode{io2d_error}}%
\begin{codeblock}
namespace std { namespace experimental { namespace io2d { inline namespace v1 {
  enum class io2d_error {
    success,
    invalid_restore,
    no_current_point,
    invalid_matrix,
    invalid_status,
    null_pointer,
    invalid_string,
    invalid_path_data,
    read_error,
    write_error,
    surface_finished,
    invalid_dash,
    clip_not_representable,
    invalid_stride,
    user_font_immutable,
    user_font_error,
    invalid_clusters,
    device_error,
    invalid_mesh_construction,
  };
} } } }
\end{codeblock}

\rSec1 [\iotwod.io2derror.enumerators] {\tcode{io2d_error} Enumerators}

\begin{libreqtab2}
  {\tcode{io2d_error} enumerator meanings}
  {tab:\iotwod.io2derror.meanings}
  \\ \topline
  \lhdr{Enumerator}
  & \rhdr{Meaning}
  \\ \capsep
  \endfirsthead
  \continuedcaption\\
  \hline
  \lhdr{Enumerator}
  & \rhdr{Meaning}
  \\ \capsep
  \endhead
 \tcode{success}
 & The operation completed successfully.
 \\
 \tcode{invalid_restore}
 & A call was made to \tcode{surface::restore} for which no prior call to
 \tcode{surface::save} was made.
 \\
 \tcode{no_current_point}
 & The operation requires a current point but no current point was set.
 This is usually the result of a call to \tcode{path_builder::rel_curve_to},
 \tcode{path_builder::rel_line_to}, or \tcode{path_builder::rel_move_to} and
 can be corrected by first establishing a current point with a non-"rel"
 member function call such as \tcode{path_builder::move_to}.
 \\
 \tcode{invalid_matrix}
 & A \tcode{matrix_2d} that the operation depends on is invalid. To be valid
 a \tcode{matrix_2d} must be invertible.
 \\
 \tcode{invalid_status}
 & An internal error has occurred. The conditions and circumstances under which 
 this \tcode{io2d_error} value occurs are 
 \impldef{\idxcode{status}!\idxcode{null_pointer}}.
 \enternote
 This value should only be used when no other \tcode{io2d_error} value is
 appropriate. It signifies that an implementation-specific error
 occurred such as passing a bad native handle as an argument.
 \exitnote
 \\
 \tcode{null_pointer}
 & A null pointer value was unexpectedly encountered. The conditions and 
 circumstances under which this \tcode{io2d_error} value occurs are 
 \impldef{\idxcode{status}!\idxcode{null_pointer}}.
 \\
 \tcode{invalid_string}
 & A UTF-8 string value was expected but the string is not a valid UTF-8
 string.
 \\
 \tcode{invalid_path_data}
 & Invalid data was encountered in a \tcode{path} or a \tcode{path_builder}
 object.
 \enternote
 This status value should only occur when a user creates invalid path data and 
 appends it to a path.
 \exitnote
 \\
 \tcode{read_error}
 & An error occurred while attempting to read data from an input stream.
 \\
 \tcode{write_error}
 & An error occurred while attempting to write data to an output stream.
 \\
 \tcode{surface_finished}
 & An attempt was made to use or manipulate a \tcode{surface} object or
 \tcode{surface}-derived object which is no longer valid.
 \enternote
 This can occur due to a previous call to \tcode{surface::finish} or as a
 result of erroneous usage of a native handle.
 \exitnote
 \\
 \tcode{invalid_dash}
 & An invalid dash value was specified in a call to \tcode{surface::set_dashes}.
 \\
 \tcode{clip_not_representable}
 & A call was made to \tcode{surface::get_clip_rectangles} when the
 \tcode{surface} object's current clipping region could not be represented
 with rectangles.
 \\
 \tcode{invalid_stride}
 & An invalid stride value was used. Surface formats may require padding at
 the end of each row of pixel data depending on the implementation and the
 current graphics chipset, if any. Use \tcode{format_stride_for_width} to
 obtain the correct stride value.
 \\
 \tcode{user_font_immutable}
 & User font immutable.
 \enternote
 Reserved.
 \exitnote
 \\
 \tcode{user_font_error}
 & User font error.
 \enternote
 Reserved.
 \exitnote
 \\
 \tcode{invalid_clusters}
 & A call was made to \tcode{surface::show_text_glyphs} with a
 \tcode{std::vector<text_clusters>} argument that does not properly map
 the UTF-8 \tcode{std::string} code points to the \tcode{std::vector<glyph>}
 glyphs.
 \\
 \tcode{device_error}
 & The operation failed. The \tcode{device} encountered an error.
 \enternote
 The conditions and circumstances in which this \tcode{io2d_error} value occurs 
 are 
 \impldef{\idxcode{status}!\idxcode{device_error}}.
 \exitnote
 \\
 \tcode{invalid_mesh_construction}
 & A mesh construction operation on a \tcode{mesh_pattern_builder} object
 failed. Mesh construction operations are only permitted in between a call to
 either \tcode{mesh_pattern_builder::begin_patch} or
 \tcode{mesh_pattern_builder::begin_edit_patch} and
 \tcode{mesh_pattern_builder::end_patch}.
 \\
\end{libreqtab2}
