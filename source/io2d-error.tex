%!TEX root = io2d.tex
\rSec0 [\iotwoderror] {Enum class \tcode{io2d_error}}

\rSec1 [\iotwoderror.summary] {\tcode{io2d_error} Summary}
\pnum
The \tcode{io2d_error} enum class is an enumeration holding error condition values which are used with the \tcode{io2d_error_category} class. See 
Table~\ref{tab:\iotwoderror.meanings} for the meaning of each
error condition value.

\rSec1 [\iotwoderror.synopsis] {\tcode{io2d_error} Synopsis}

\indexlibrary{\idxcode{io2d_error}}
\begin{codeblock}
namespace std { namespace experimental { namespace io2d { inline namespace v1 {
  enum class io2d_error {
    success,
    invalid_pop_state,
    no_current_point,
    invalid_matrix,
    invalid_status,
    null_pointer,
    invalid_string,
    invalid_path_data,
    read_error,
    write_error,
    surface_finished,
    invalid_dash,
    invalid_index,
    clip_not_representable,
    invalid_stride,
    device_error,
    invalid_mesh_construction,
  };
} } }

  template<>
  struct is_error_condition_enum<experimental::io2d::io2d_error>
  : public std::true_type{ };
}
\end{codeblock}

\rSec1 [\iotwoderror.enumerators] {\tcode{io2d_error} Enumerators}

\begin{libreqtab2}
  {\tcode{io2d_error} enumerator meanings}
  {tab:\iotwoderror.meanings}
  \\ \topline
  \lhdr{Enumerator}
  & \rhdr{Meaning}
  \\ \capsep
  \endfirsthead
  \continuedcaption\\
  \hline
  \lhdr{Enumerator}
  & \rhdr{Meaning}
  \\ \capsep
  \endhead
 \tcode{success}
 & The operation completed successfully.
 \\
 \tcode{invalid_pop_state}
 & A call was made to \tcode{surface::pop_state} for which no prior call to
 \tcode{surface::push_state} was made.
 \\
 \tcode{no_current_point}
 & A path segment or path instruction encountered during path processing requires a value for current point but current point has no value.
 \\
 \tcode{invalid_matrix}
 & A \tcode{matrix_2d} value that the operation depends on is invalid. Except as otherwise specified, this means that the \tcode{matrix_2d} value is not invertible.
 \\
 \tcode{invalid_status}
 & An internal error has occurred. The conditions and circumstances under which 
 this \tcode{io2d_error} value occurs are 
 \impldefplain{\idxcode{io2d_error}!\idxcode{invalid_status}}.
 \begin{note}
 This value should only be used when no other \tcode{io2d_error} value is
 appropriate. It signifies that an implementation-specific error
 occurred such as passing a bad native handle as an argument.
 \end{note}
 \\
 \tcode{null_pointer}
 & A null pointer value was unexpectedly encountered. The conditions and 
 circumstances under which this \tcode{io2d_error} value occurs are 
 \impldefplain{\idxcode{io2d_error}!\idxcode{null_pointer}}.
 \\
 \tcode{invalid_string}
 & A UTF-8 string value was expected but the string is not a valid UTF-8
 string.
 \\
 \tcode{invalid_path_data}
 & Invalid data was encountered in a \tcode{path_group} or a \tcode{path_builder}
 object.
 \begin{note}
 This status value should only occur when a user creates invalid path data and 
 appends it to a path.
 \end{note}
 \\
 \tcode{read_error}
 & An error occurred while attempting to read data from an input stream.
 \\
 \tcode{write_error}
 & An error occurred while attempting to write data to an output stream.
 \\
 \tcode{surface_finished}
 & An attempt was made to use or manipulate a \tcode{surface} object or
 \tcode{surface}-derived object which is no longer valid.
 \begin{note}
 This can occur due to a previous call to \tcode{surface::finish} or as a
 result of erroneous usage of a native handle.
 \end{note}
 \\
 \tcode{invalid_dash}
 & An invalid dash value was specified in a call to \tcode{surface::set_dashes}.
 \\
 \tcode{invalid_index}
 & An index value was specified in a call to a function which is outside the range of index values that are currently valid.
 \\
 \tcode{clip_not_representable}
 & A call was made to \tcode{surface::get_clip_rectangles} when the
 \tcode{surface} object's current clipping region could not be represented
 with rectangles.
 \\
 \tcode{invalid_stride}
 & An invalid stride value was used. Surface formats may require padding at
 the end of each row of pixel data depending on the implementation and the
 current graphics chipset, if any. Use \tcode{format_stride_for_width} to
 obtain the correct stride value.
 \\
 \tcode{device_error}
 & The operation failed. The \tcode{device} encountered an error.
 \begin{note}
 The conditions and circumstances in which this \tcode{io2d_error} value occurs 
 are 
 \impldefplain{\idxcode{io2d_error}!\idxcode{device_error}}.
 \end{note}
 \\
\end{libreqtab2}
