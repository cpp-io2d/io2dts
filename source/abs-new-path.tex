%!TEX root = io2d.tex
\rSec0 [\iotwod.absnewfigure] {Class \tcode{abs_new_figure}}

\pnum
\indexlibrary{\idxcode{abs_new_figure}}%
The class \tcode{abs_new_figure} describes a figure item that is a new figure command.

\pnum
It has an \term{at point} of type \tcode{point_2d}.

\rSec1 [\iotwod.absnewfigure.cons] {\tcode{abs_new_figure} constructors}%

\indexlibrary{\idxcode{abs_new_figure}!constructor}%
\begin{itemdecl}
abs_new_figure();
\end{itemdecl}
\begin{itemdescr}
\pnum
\effects
Equivalent to: \tcode{abs_new_figure\{basic_point_2d<typename GraphicsSurfaces::graphics_math_type>()\};}
\end{itemdescr}

\indexlibrary{\idxcode{abs_new_figure}!constructor}%
\begin{itemdecl}
explicit abs_new_figure(const basic_point_2d<typename
  GraphicsSurfaces::graphics_math_type>& pt);
\end{itemdecl}
\begin{itemdescr}
\pnum
\effects
Constructs an object of type \tcode{abs_new_figure}.

\pnum
The at point is \tcode{pt}.
\end{itemdescr}

\indexlibrary{\idxcode{abs_new_figure}!constructor}%
\begin{itemdecl}
abs_new_figure(const abs_new_figure& other);
abs_new_figure(abs_new_figure&& other) noexcept;
\end{itemdecl}
\begin{itemdescr}
\pnum
\effects
Constructs an object of type \tcode{abs_new_figure}. In the second form, other is left in a valid state with an unspecified value.
	
\pnum
The at point is \tcode{other.at()}.
\end{itemdescr}

\rSec1 [\iotwod.absnewfigure.assign] {\tcode{abs_new_figure} assignment operators}%

\indexlibrary{\idxcode{abs_new_figure}!assignment}%
\begin{itemdecl}
abs_new_figure& operator=(const abs_new_figure& other);
\end{itemdecl}
\begin{itemdescr}
\pnum
\effects
If \tcode{*this} and \tcode{other} are not the same object, modifies \tcode{*this} such that \tcode{*this.at()} is \tcode{other.at()}

\pnum
If \tcode{*this} and \tcode{other} are the same object, the member has no effect.
	
\pnum
\returns
\tcode{*this}
\end{itemdescr}

\indexlibrary{\idxcode{abs_new_figure}!assignment}%
\begin{itemdecl}
abs_new_figure& operator=(abs_new_figure&& other) noexcept;
\end{itemdecl}
\begin{itemdescr}
\pnum
\effects
<TODO>

\pnum
\returns
\tcode{*this}
\end{itemdescr}

\rSec1 [\iotwod.absnewfigure.modifiers]{\tcode{abs_new_figure} modifiers}%

\indexlibrarymember{at}{abs_new_figure}%
\begin{itemdecl}
void at(const basic_point_2d<typename GraphicsSurfaces::graphics_math_type>& pt) noexcept;
\end{itemdecl}
\begin{itemdescr}
\pnum
\effects
The at point is \tcode{pt}.
\end{itemdescr}

\rSec1 [\iotwod.absnewfigure.observers]{\tcode{abs_new_figure} observers}%

\indexlibrarymember{at}{abs_new_figure}%
\begin{itemdecl}
basic_point_2d<typename GraphicsSurfaces::graphics_math_type> at() const noexcept;
\end{itemdecl}
\begin{itemdescr}
\pnum
\returns
The at point.
\end{itemdescr}

\rSec1 [\iotwod.absnewfigure.ops]{\tcode{abs_new_figure} operators}%

\indexlibrarymember{operator==}{abs_new_figure}%
\begin{itemdecl}
template <class GraphicsSurfaces>
bool operator==(const typename basic_figure_items<GraphicsSurfaces>::abs_new_figure& lhs,
  const typename basic_figure_items<GraphicsSurfaces>::abs_new_figure& rhs) noexcept;
\end{itemdecl}
\begin{itemdescr}
\pnum
\returns
\tcode{lhs.at() == rhs.at()}.
\end{itemdescr}
