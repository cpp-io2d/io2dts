%!TEX root = io2d.tex
\rSec0 [absnewpath] {Class \tcode{abs_new_path}}%

\pnum
\indexlibrary{\idxcode{abs_new_path}}%
The class \tcode{abs_new_path} describes a path item that creates a new path and makes the previous path, if any, an open path unless it was made a closed path by a \tcode{close_path} object.

\pnum
It has an \term{at point} of type \tcode{vector_2d}.

\pnum
When interpreting a path group, the path's last-move-to-point and current point are set to the value of the at point.

\rSec1 [absnewpath.synopsis] {\tcode{abs_new_path} synopsis}%

\begin{codeblock}
namespace std { namespace experimental { namespace io2d { inline namespace v1 {
  namespace path_data {
    class abs_new_path {
    public:
      // \ref{absnewpath.cons}, construct:
      constexpr abs_new_path() noexcept;
      constexpr explicit abs_new_path(const vector_2d& pt) noexcept;

      // \ref{absnewpath.modifiers}, modifiers:
      constexpr void at(const vector_2d& pt) noexcept;

      // \ref{absnewpath.observers}, observers:
      constexpr vector_2d at() const noexcept;
    };
    
    // \ref{absnewpath.nonmember}, non-members:
    constexpr bool operator==(const abs_new_path& lhs, const abs_new_path& rhs) 
      noexcept;
    constexpr bool operator!=(const abs_new_path& lhs, const abs_new_path& rhs) 
      noexcept;
  }
} } } }
\end{codeblock}

\rSec1 [absnewpath.cons] {\tcode{abs_new_path} constructors}%

\indexlibrary{\idxcode{abs_new_path}!constructor}%
\begin{itemdecl}
constexpr abs_new_path() noexcept;
\end{itemdecl}
\begin{itemdescr}
\pnum
\effects
Equivalent to: \tcode{abs_new_path\{ vector_2d() \};}
\end{itemdescr}

\indexlibrary{\idxcode{abs_new_path}!constructor}%
\begin{itemdecl}
constexpr explicit abs_new_path(const vector_2d& pt) noexcept;
\end{itemdecl}
\begin{itemdescr}
\pnum
\effects
Constructs an object of type \tcode{abs_new_path}.

\pnum
The at point is \tcode{pt}.
\end{itemdescr}

\rSec1 [absnewpath.modifiers]{\tcode{abs_new_path} modifiers}%

\indexlibrarymember{at}{abs_new_path}%
\begin{itemdecl}
constexpr void at(const vector_2d& pt) noexcept;
\end{itemdecl}
\begin{itemdescr}
\pnum
\effects
The at point is \tcode{pt}.
\end{itemdescr}

\rSec1 [absnewpath.observers]{\tcode{abs_new_path} observers}%

\indexlibrarymember{at}{abs_new_path}%
\begin{itemdecl}
constexpr vector_2d at() const noexcept;
\end{itemdecl}
\begin{itemdescr}
\pnum
\returns
The at point.
\end{itemdescr}

\rSec1 [absnewpath.nonmember]{Non-member functions}%

\indexlibrary{operator==}{abs_new_path}%
\begin{itemdecl}
constexpr bool operator==(const abs_new_path& lhs, const abs_new_path& rhs) 
  noexcept;
\end{itemdecl}
\begin{itemdescr}
\pnum
\returns
\tcode{lhs.at() == rhs.at()}.
\end{itemdescr}

\indexlibrary{operator!=}{abs_new_path}%
\begin{itemdecl}
constexpr bool operator!=(const abs_new_path& lhs, const abs_new_path& rhs) 
  noexcept;
\end{itemdecl}
\begin{itemdescr}
\pnum
\returns
\tcode{!(lhs == rhs)}.
\end{itemdescr}
