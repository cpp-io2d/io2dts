%!TEX root = io2d.tex
\rSec0[\iotwod.err.report]{Error reporting}

\pnum
2D graphics library functions often provide two overloads, one that throws an exception to report graphics subsystem errors, and another that sets an \tcode{error_code}.

\pnum
\enternote
This supports two common use cases:

\begin{itemize}
\item Uses where graphics subsystem errors are truly exceptional and indicate a serious failure. Throwing an exception is the most appropriate response. This is the preferred default for most everyday programming.

\item Uses where graphics subsystem errors are routine and do not necessarily represent failure. Returning an error code is the most appropriate response. This allows application specific error handling, including simply ignoring the error.
\end{itemize}

\exitnote

\pnum
Functions \textbf{not} having an argument of type \tcode{error_code\&} report errors as follows, unless otherwise specified:

\begin{itemize}
\item When a call by the implementation to an operating system or other underlying API results in an error that prevents the function from meeting its specifications and the cause of the error is described in the function's \textit{Error conditions} description:
	\begin{itemize}
		\item If the description calls for \tcode{errc::argument_out_of_domain} or \tcode{io2d_error::invalid_index}, the exception type shall be \tcode{out_of_range} constructed with an \impldef{\tcode{errc::argument_out_of_domain}!\tcode{what_arg} value} \tcode{what_arg} argument value.
		\item If the description calls for \tcode{errc::invalid_argument}, the exception type shall be \tcode{invalid_argument} constructed with an \impldef{\tcode{errc::invalid_argument}!\tcode{what_arg} value} \tcode{what_arg} argument value.
		\item If the description calls for \tcode{errc::not_enough_memory}, the error shall be reported by throwing an exception as described in \CppXIV \S17.6.5.12 [res.on.exception.handling].
		\item In all other cases the exception type shall be \tcode{system_error} constructed with an \tcode{ec} argument value formed by passing the specified enumerator value to \tcode{make_error_code} and an \impldef{other error codes!\tcode{what_arg} value} \tcode{what_arg} argument value, unless otherwise specified.
	\end{itemize}

\item When a call by the implementation to an operating system or other underlying API results in an error that prevents the function from meeting its specifications and the cause of the error is \textbf{not} described in the function's \textit{Error conditions} description and is not a failure to allocate storage, an exception of type \tcode{system_error} shall be thrown constructed with its \tcode{error_code} argument set as appropriate for the specific operating system dependent error. Implementations should document these errors where possible.

\item Failure to allocate storage is reported by throwing an exception as described in \CppXIV \S17.6.5.12 [res.on.exception.handling].

\item Destructors throw nothing.
\end{itemize}

\pnum
Functions having an argument of type \tcode{error_code\&} report errors as follows, unless otherwise specified:

\begin{itemize}
\item When a call by the implementation to an operating system or other underlying API results in an error that prevents the function from meeting its specifications and the cause of the error is described in the function's \textit{Error conditions} description, the \tcode{error_code\&} argument is set as appropriate for the specified enumerator.

\item When a call by the implementation to an operating system or other underlying API results in an error that prevents the function from meeting its specifications and the cause of the error is \textbf{not} described in the function's \textit{Error conditions} description and is not a failure to allocate storage, the \tcode{error_code\&} argument is set as appropriate for the specific operating system dependent error. Implementations should document these errors where possible.

\item If a failure to allocate storage occurs, the \tcode{error_code\&} argument shall be set to \tcode{make_error_code(errc::not_enough_memory)}.

\item Otherwise, \tcode{clear()} is called on the \tcode{error_code\&} argument.
\end{itemize}
