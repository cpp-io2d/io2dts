%!TEX root = io2d.tex
\rSec0 [absline] {Class \tcode{abs_line}}

\pnum
\indexlibrary{\idxcode{abs_line}}
The class \tcode{abs_line} describes a path segment that is a line.

\pnum
It has an end point of type \tcode{vector_2d}.

\rSec1 [absline.synopsis] {\tcode{abs_line} synopsis}

\begin{codeblock}
namespace std { namespace experimental { namespace io2d { inline namespace v1 {
  namespace path_data {
    class abs_line {
    public:
      // \ref{absline.cons}, construct:
      explicit abs_line(const vector_2d& pt) noexcept;

      // \ref{absline.modifiers}, modifiers:
      void to(const vector_2d& pt) noexcept;

      // \ref{absline.observers}, observers:
      vector_2d to() const noexcept;
    };
  };
} } } }
\end{codeblock}

\rSec1 [absline.cons] {\tcode{abs_line} constructors and assignment operators}

\indexlibrary{\idxcode{abs_line}!constructor}
\begin{itemdecl}
    explicit abs_line(const vector_2d& pt) noexcept;
\end{itemdecl}
\begin{itemdescr}
	\pnum
	\effects
	Constructs an object of type \tcode{abs_line}.
	
	\pnum
	The end point shall be set to the value of \tcode{pt}.
\end{itemdescr}

\rSec1 [absline.modifiers]{\tcode{abs_line} modifiers}

\indexlibrary{\idxcode{abs_line}!\idxcode{to}}
\indexlibrary{\idxcode{to}!\idxcode{abs_line}}
\begin{itemdecl}
    void to(const vector_2d& pt) noexcept;
\end{itemdecl}
\begin{itemdescr}
	\pnum
	\effects
	The end point shall be set to the value of \tcode{pt}.
\end{itemdescr}

\rSec1 [absline.observers]{\tcode{abs_line} observers}

\indexlibrary{\idxcode{abs_line}!\idxcode{to}}
\indexlibrary{\idxcode{to}!\idxcode{abs_line}}
\begin{itemdecl}
    vector_2d to() const noexcept;
\end{itemdecl}
\begin{itemdescr}
	\pnum
	\returns
	The value of the end point.
\end{itemdescr}
