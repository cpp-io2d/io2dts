%!TEX root = io2d.tex
\rSec0 [pathdataitem.relmoveto] {Class \tcode{rel_move_to}}

\rSec1 [pathdataitem.relmoveto.synopsis] {\tcode{rel_move_to} synopsis}

\begin{codeblock}
namespace std { namespace experimental { namespace io2d { inline namespace v1 {
  class rel_move_to : public path_data {
  public:
    // \ref{pathdataitem.relmoveto.cons}, construct/copy/move/destroy:
    rel_move_to() noexcept;
    rel_move_to(const rel_move_to& other) noexcept;
    rel_move_to& operator=(const rel_move_to& other) noexcept;
    rel_move_to(rel_move_to&& other) noexcept;
    rel_move_to& operator=(rel_move_to&& other) noexcept;
    rel_move_to(const vector_2d& pt) noexcept;

    // \ref{pathdataitem.relmoveto.modifiers}, modifiers:
    void to(const vector_2d& pt) noexcept;

    // \ref{pathdataitem.relmoveto.observers}, observers:
    vector_2d to() const noexcept;
    virtual path_data_type type() const noexcept override;
    
  private:
    vector_2d _Data; // \expos
  };
} } } }
\end{codeblock}

\rSec1 [pathdataitem.relmoveto.intro] {\tcode{rel_move_to} Description}

\pnum
\indexlibrary{\idxcode{rel_move_to}}
The class \tcode{rel_move_to} describes an operation on a path geometry collection.

\pnum
This operation starts a new path geometry and sets its current point and last-move-to point to the point that is the sum of the previous path geometry's current point and the point returned by \tcode{*this.to()}.

\pnum
If the existing path geometry does not have a current point when this operation is requested the path geometry collection is malformed.

\rSec1 [pathdataitem.relmoveto.cons] {\tcode{rel_move_to} constructors and assignment operators}

\indexlibrary{\idxcode{rel_move_to}!constructor}
\begin{itemdecl}
    rel_move_to() noexcept;
\end{itemdecl}
\begin{itemdescr}
	\pnum
	\effects
	Constructs an object of type \tcode{rel_move_to}.
	
	\pnum
	\postconditions
	\tcode{_Data == vector_2d(0.0, 0.0)}.
\end{itemdescr}

\indexlibrary{\idxcode{rel_move_to}!constructor}
\begin{itemdecl}
    rel_move_to(const vector_2d& pt) noexcept;
\end{itemdecl}
\begin{itemdescr}
	\pnum
	\effects
	Constructs an object of type \tcode{rel_move_to}.
	
	\pnum
	\postconditions
	\tcode{_Data == pt}.
\end{itemdescr}

\rSec1 [pathdataitem.relmoveto.modifiers]{\tcode{rel_move_to} modifiers}

\indexlibrary{\idxcode{rel_move_to}!\idxcode{to}}
\indexlibrary{\idxcode{to}!\idxcode{rel_move_to}}
\begin{itemdecl}
    void to(const vector_2d& pt) noexcept;
\end{itemdecl}
\begin{itemdescr}
	\pnum
	\postconditions
	\tcode{_Data == pt}.
	
\end{itemdescr}

\rSec1 [pathdataitem.relmoveto.observers]{\tcode{rel_move_to} observers}

\indexlibrary{\idxcode{rel_move_to}!\idxcode{to}}
\indexlibrary{\idxcode{to}!\idxcode{rel_move_to}}
\begin{itemdecl}
    vector_2d to() const noexcept;
\end{itemdecl}
\begin{itemdescr}
	\pnum
	\returns
	\tcode{_Data}.

\end{itemdescr}

\indexlibrary{\idxcode{rel_move_to}!\idxcode{type}}
\indexlibrary{\idxcode{type}!\idxcode{rel_move_to}}
\begin{itemdecl}
    virtual path_data_type type() const noexcept override;
\end{itemdecl}
\begin{itemdescr}
	\pnum
	\returns
	\tcode{path_data_type::rel_move_to}.

\end{itemdescr}
