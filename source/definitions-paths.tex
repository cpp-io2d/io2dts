%!TEX root = io2d.tex

\indexdefn{graphics subsystem}%
\definition{graphics subsystem}{\iotwod.defns.graphicssubsystem}
collection of unspecified operating system and library functionality used to render and display 2D computer graphics

\indexdefn{graphics resource}%
\definition{graphics resource}{\iotwod.defns.graphicsresource}
\defncontext{graphics resource} object of unspecified type used by an implementation
\begin{note}
By its definition a graphics resource is an implementation detail. Often it will be a graphics subsystem object (e.g. a graphics device or a render target) or an aggregate composed of multiple graphics subsystem objects. However the only requirement placed upon a graphics resource is that the implementation is able to use it to provide the functionality required of the graphics resource.
\end{note}

\indexdefn{graphics resource}%
\indexdefn{graphics resource!graphics data graphics resource}%
\definition{graphics resource}{\iotwod.defns.graphicsresource.graphicsdata}
\defncontext{graphics data graphics resource} object of unspecified type used by an implementation to provide access to and allow manipulation of visual data

\indexdefn{\pixmap}%
\definition{\pixmap}{\iotwod.defns.pixmap}
raster graphics data graphics resource

\indexdefn{filter}%
\definition{filter}{\iotwod.defns.filter}
mathematical function that determines the visual data value of a point for a graphics data graphics resource

\indexdefn{composition algorithm}%
\definition{composition algorithm}{\iotwod.defns.compositionalgorithm}
an algorithm that combines a source visual data element and a destination visual data element producing a visual data element that has the same visual data format as the destination visual data element

\indexdefn{compose}%
\definition{compose}{\iotwod.defns.compose}
to combine part or all of a source graphics data graphics resource with a destination graphics data graphics resource in the manner specified by a composition algorithm

\indexdefn{composing operation}%
\definition{composing operation}{\iotwod.defns.composingoperation}
an operation that performs composing
%an operation that uses a composition algorithm to combine part or all of a source of visual data capable of being treated as though it were a \pixmap with a \pixmap

\indexdefn{artifact}%
\definition{artifact}{\iotwod.defns.artifact}
an error in the results of the application of a composing operation 

\indexdefn{sample}%
\definition{sample}{\iotwod.defns.sample}
to use a filter to obtain the visual data for a given point from a graphics data graphics resource

\indexdefn{aliasing}%
\definition{aliasing}{\iotwod.defns.alias}
the presence of visual artifacts in the results of rendering due to 
sampling imperfections

\indexdefn{anti-aliasing}%
\definition{anti-aliasing}{\iotwod.defns.antialias}
the application of a function or algorithm while composing to 
reduce aliasing
\begin{note}
Certain algorithms can produce ``better'' results, i.e. results with less artifacts or with less pronounced artifacts, when rendering text with anti-aliasing due to the nature of text rendering. As such, it often makes sense to provide the ability to choose one type of anti-aliasing for text rendering and another for all other rendering and to provide different sets of anti-aliasing types to choose from for each of the two operations.
\end{note}

\indexdefn{graphics state data}%
\definition{graphics state data}{\iotwod.defns.graphicsstatedata}
data which specify how some part of the process of rendering or of a composing operation shall be performed in part or in whole

\indexdefn{render}%
\definition{render}{\iotwod.defns.render}
to transform a path group into graphics data in the manner specified by a set of graphics state data

\indexdefn{rendering operation}%
\definition{rendering operation}{\iotwod.defns.renderingoperation}
an operation that performs rendering

\indexdefn{rendering and composing operation}%
\definition{rendering and composing operation}{\iotwod.defns.renderingandcomposingop}
an operation that is either a composing operation or a rendering operation followed by a composing operation

