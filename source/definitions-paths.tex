%!TEX root = io2d.tex

\indexdefn{path segment}%
\definition{path segment}{\iotwod.defns.pathseg}
line, \bezierlocal curve, or arc, each of which has a start point and an end point

\indexdefn{control point}%
\definition{control point}{\iotwod.defns.controlpt}
point other than the start point and end point that is used in defining a \bezierlocal curve

\indexdefn{degenerate path segment}%
\definition{degenerate path segment}{\iotwod.defns.degenepathseg}
path segment that has the same values for its start point, end point, and, if any, control points

\indexdefn{initial path segment}%
\definition{initial path segment}{\iotwod.defns.initialpathseg}
path segment whose start point is not defined as being the end point of another path segment
\begin{note}
It is possible for the initial path segment and final path segment to be the same path segment.
\end{note}

\indexdefn{final path segment}%
\definition{final path segment}{\iotwod.defns.finalpathseg}
path segment whose end point does not define the start point of any other path segment
\begin{note}
It is possible for the initial path segment and final path segment to be the same path segment.
\end{note}

\indexdefn{path instruction}%
\definition{path instruction}{\iotwod.defns.newpathinstruction}
\defncontext{new path instruction} instruction that creates a new path

\indexdefn{path instruction}%
\definition{path instruction}{\iotwod.defns.closepathinstruction}
\defncontext{close path instruction} instruction that creates a line path segment from the current point to the , a path and establishes a new path

\indexdefn{path item}%
\definition{path item}{\iotwod.defns.pathitem}
path segment, new path instruction, close path instruction, or path group instruction

\indexdefn{path}%
\definition{path}{\iotwod.defns.path}
collection of path items where the end point of each path segment, except the final path segment, defines the start point of exactly one other path segment in the collection

\indexdefn{degenerate path}%
\definition{degenerate path}{\iotwod.defns.degenpath}
path composed entirely of a new path instruction, zero or more degenerate path segments, zero of more path group items, and, optionally, a close path instruction

\indexdefn{current point}%
\definition{current point}{\iotwod.defns.currentpoint}
point used as the start point of a path segment

\indexdefn{new path point}%
\definition{new path point}{\iotwod.defns.newpathpt}
point in a path that is the start point of the initial path segment

\indexdefn{path group}%
\definition{path group}{\iotwod.defns.pathgroup}
collection of paths

\indexdefn{path group transformation matrix}%
\definition{path group transformation matrix}{\iotwod.defns.pathgrptransform}
affine transformation matrix used to apply affine transformations to the points in a path group

\indexdefn{path group instruction}%
\definition{path group instruction}
instruction that modifies the path group transformation matrix

\indexdefn{closed path}%
\definition{closed path}{\iotwod.defns.closedpath}
path with one or more path segments where the new path point is used to define the end point of the path's final path segment

\indexdefn{open path}%
\definition{open path}{\iotwod.defns.openpath}
path with one or more path segments where the new path point is not used to define the end point of the path's final path segment
\begin{note}
Even if the start point of the initial path segment and the end point of the final path segment are assigned the same coordinates, the path is still an open path since the final path segment's end point is not defined as being the start point of the initial segment but instead merely happens to have the same value as that point.
\end{note}
